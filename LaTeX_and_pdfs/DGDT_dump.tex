% file: DGDT_dump.tex
% Differential Geometry, Differential Topology, in unconventional ``grande'' format; fitting a widescreen format
% 
% github        : ernestyalumni
% linkedin      : ernestyalumni 
% wordpress.com : ernestyalumni
%
% This code is open-source, governed by the Creative Common license.  Use of this code is governed by the Caltech Honor Code: ``No member of the Caltech community shall take unfair advantage of any other member of the Caltech community.'' 
% 

\documentclass[10pt]{amsart}
\pdfoutput=1
\usepackage{mathtools,amssymb,lipsum,caption}

\usepackage{graphicx}
\usepackage{hyperref}
\usepackage[utf8]{inputenc}
\usepackage{listings}
\usepackage[table]{xcolor}
\usepackage{pdfpages}
%\usepackage[version=3]{mhchem}
\usepackage{mhchem}

\usepackage{tikz}
\usetikzlibrary{matrix,arrows}

\usepackage{multicol}

\hypersetup{colorlinks=true,citecolor=[rgb]{0,0.4,0}}

\oddsidemargin=15pt
\evensidemargin=5pt
\hoffset-45pt
\voffset-55pt
\topmargin=-4pt
\headsep=5pt
\textwidth=1120pt
\textheight=595pt
\paperwidth=1200pt
\paperheight=700pt
\footskip=40pt








\newtheorem{theorem}{Theorem}
\newtheorem{corollary}{Corollary}
%\newtheorem*{main}{Main Theorem}
\newtheorem{lemma}{Lemma}
\newtheorem{proposition}{Proposition}

\newtheorem{definition}{Definition}
\newtheorem{remark}{Remark}

\newenvironment{claim}[1]{\par\noindent\underline{Claim:}\space#1}{}
\newenvironment{claimproof}[1]{\par\noindent\underline{Proof:}\space#1}{\hfill $\blacksquare$}

%This defines a new command \questionhead which takes one argument and
%prints out Question #. with some space.
\newcommand{\questionhead}[1]
  {\bigskip\bigskip
   \noindent{\small\bf Question #1.}
   \bigskip}

\newcommand{\problemhead}[1]
  {
   \noindent{\small\bf Problem #1.}
   }

\newcommand{\exercisehead}[1]
  { \smallskip
   \noindent{\small\bf Exercise #1.}
  }

\newcommand{\solutionhead}[1]
  {
   \noindent{\small\bf Solution #1.}
   }


\title{The Differential Geometry Differential Topology Dump}
\author{Ernest Yeung \href{mailto:ernestyalumni@gmail.com}{ernestyalumni@gmail.com}}
\date{28 juillet 2016}
\keywords{Differential Geometry, Differential Topology}
\begin{document}

\definecolor{darkgreen}{rgb}{0,0.4,0}
\lstset{language=Python,
 frame=bottomline,
 basicstyle=\scriptsize,
 identifierstyle=\color{blue},
 keywordstyle=\bfseries,
 commentstyle=\color{darkgreen},
 stringstyle=\color{red},
 }
%\lstlistoflistings

\maketitle



\begin{multicols*}{2}

  
\setcounter{tocdepth}{1}
\tableofcontents



\begin{abstract}
Everything about Differential Geometry, Differential Topology

\end{abstract}

\part{}


\begin{theorem}[(Implicit Function Thm.)]
  Let open subset $U\subseteq \mathbb{R}^n \times \mathbb{R}^d$, $(x,y) = (x^1 \dots x^n, y^1 \dots y^k) $ on $U$.  \\
  Suppose smooth $\Phi:U\to \mathbb{R}^k$, $(a,b) \in U$, $c=\Phi(a,b)$

  If $k\times k$ matrix $\frac{ \partial \Phi^i}{ \partial y^j}(a,b)$ nonsingular, then $\exists $ neighborhoods $\begin{aligned} & \quad \\
    & V_0 \subseteq \mathbb{R}^n \text{ of $a$ } \\
    & W_0 \subseteq \mathbb{R}^k \text{ of $b$ } \end{aligned}$ and smooth $F:V_0 \to W_0$ s.t.

  $\Phi^{-1}(c) \bigcap (V_0\times W_0)$ is graph of $F$, i.e. \\
  $\Phi(x,y) =c$ for $(x,y) \in V_0\times W_0$ iff $y=F(x)$.  
  \end{theorem}




\end{multicols*}

\begin{thebibliography}{9}

  \bibitem{}

\end{thebibliography}


\end{document}
