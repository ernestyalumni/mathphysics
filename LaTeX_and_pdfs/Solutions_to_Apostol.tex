%Solutions_to_Apostol.tex 
%
\documentclass[twoside]{amsart}
\usepackage{amssymb,latexsym}
\usepackage{times}

%\usepackage{graphics}

\oddsidemargin-0.15cm
\evensidemargin-0.15cm
\topmargin-1.8cm     %I recommend adding these three lines to increase the 
\textwidth17.5cm   %amount of usable space on the page (and save trees)
\textheight24.5cm  

%This next line (when uncommented) allow you to use encapsulated
%postscript files for figures in your document
%\usepackage{epsfig}

%plain makes sure that we have page numbers
\pagestyle{plain}

\theoremstyle{plain}
\newtheorem{theorem}{Theorem}
\newtheorem{axiom}{Axiom}
\newtheorem{lemma}{Lemma}
\newtheorem{proposition}{Proposition}

\theoremstyle{definition}
\newtheorem{definition}{Definition}

\title{	Solutions to \emph{ Calculus } Volume 1 by Tom Apostol.	}

\author{
  Ernest Yeung,  - Praha 10, \v Cesk\`a Republika 
       }
%\date{Winter 2006}

%This defines a new command \questionhead which takes one argument and
%prints out Question #. with some space.
\newcommand{\questionhead}[1]
  {\bigskip\bigskip
   \noindent{\small\bf Question #1.}
   \bigskip}

\newcommand{\problemhead}[1]
  {\bigskip\bigskip
   \noindent{\small\bf Problem #1.}
   \medskip}

\newcommand{\exercisehead}[1]
  {\smallskip
   \noindent{\small\bf Exercise #1.}}
%   \medskip}

\newcommand{\solutionhead}[1]
  {\medskip\bigskip
   \noindent{\small\bf Solution #1.}
   \medskip}


%-----------------------------------
\begin{document}
%-----------------------------------

\maketitle

%-----------------------------------
\section*{Solutions to \emph{ \textsc{ Volume 1} One-Variable Calculus, with an Introduction to Linear Algebra } }

\subsection*{ I 2.5 Exercises - Introduction to set theory, Notations for designating sets, Subsets, Unions, intersections, complements }

\exercisehead{10} \emph{Distributive laws }
\[
\begin{aligned}
  \text{ Let } & X = A \cap ( B \cup C), Y  = (A \cap B) \cup (A \cap C) \\
  & \text{ Suppose } x \in X \\
  & \quad \begin{aligned}
    & x \in A \text{ and } x \in (B \cup C) \Longrightarrow x \in A \text{ and $x$ is in at least $B$ or in $C$ } \\
    & \text{ then $x$ is in at least either $(A \cap B)$ or $(A \cap C)$ } \\
    & x \in Y, X \subseteq Y 
  \end{aligned} \\
  & \text{ Suppose } y \in Y \\
  & \quad \begin{aligned}
    & \text{ $y$ is at least in either $(A \cap B)$ or $A \cap C$ } \\
    & \text{ then $y \in A$ and either in $B$ or $C$ } \\
    & y \in X, Y \subseteq X
  \end{aligned} \\
  & X = Y
\end{aligned}
\]
\[
\begin{aligned}
  \text{ Let } & X = A \cup (B \cap C), Y = (A \cup B) \cap (A \cup C) \\
  & \text{ Suppose } x \in X \\
  & \quad \begin{aligned}
    & \text{ then $x$ is at least either in $A$ or in $(B \cap C)$ }
    \\
    & \text{ if $x \in A$, $x \in Y$ } \\
    & \text{ if $x \in (B \cap C)$, $x \in Y$ }
    & x \in Y, X \subseteq Y 
  \end{aligned} \\
  & \text{ Suppose } y \in Y \\
  & \quad \begin{aligned}
    & \text{ then $y$ is at least in $A$ or in $B$ and $y$ is at least in $A$ or in $C$ } \\
    & \text{ if $y \in A$, then $y \in X$ } \\
    & \text{ if $y \in A \cap B$ or $ y \in A \cup C, y \in X $ (various carvings out of $A$, simply ) } \\
    & \text{ if $y \in (B \cap C)$, $y \in X $ }
    & y \in X, Y \subseteq X 
  \end{aligned} \\
  & X = Y
\end{aligned}
\]

\exercisehead{11} 
If $x \in A \cup A$, then $x $ is at least in $A$ or in $A$.  Then $x \in A$.  So $A \cup A \subseteq A$.  Of course $A \subseteq A \cup A$.  

If $x \in A \cap A$, then $x $ is in $A$ and in $A$.  Then $x \in A$.  So $A \cap A \subseteq A$.  Of course $A \subseteq A \cap A$.  

\exercisehead{12}
Let $x \in A$.  $y \in A \cup B$ if $y$ is at least in $A$ or in $B$.  $x$ is in $A$ so $x \in A \cup B$.  $\Longrightarrow A \subseteq A \cup B$.  

Suppose $\exists b \in B$ and $b \notin A$.  $b \in A \cup B$ but $b \notin A$.  so $A \subseteq A \cup B$.  

\exercisehead{13} 
Let $x \in A \cup \emptyset$, then $x$ is at least in $A$ or in $\emptyset$.  If $x \in \emptyset$, then $x$ is a null element (not an element at all).  Then actual elements must be in $A$.  $\Longrightarrow A \cup \emptyset \subseteq A$.  

Let $x \in A$.  Then $x \in A \cup \emptyset$.  $A \subseteq A \cup \emptyset$.  $\Longrightarrow A = A \cup \emptyset$.  

\exercisehead{14} 
From distributivity, $A \cup (A \cap B) = (A \cup A) \cap (A \cup B) = A \cap (A \cup B) $.  \\
If $x \in A \cap (A \cup B)$, $x \in A$ and $x \in A \cup B$, i.e. $x \in A$ and $x$ is at least in $A$ or in $B$.  \\
$\Longrightarrow x$ is in $A$ and is in $B$ or is not in $B$.  Then $x \in A$.  $\Longrightarrow A \cap (A \cup B) \subseteq A$.  Of course, $A \subseteq A \cap (A \cup B)$.  $\Longrightarrow A \cap (A \cup B) = A \cup (A \cap B) = A$.  

\exercisehead{15}
$\forall a \in A, a\in C$ and $\forall b \in B, b \in C$.  Consider $x \in A \cup B$.  $x$ is at least in $A$ or in $B$.  In either case, $x \in C$.  $\Longrightarrow A \cup B \subseteq C$.  

\exercisehead{16} 
\[
    \begin{aligned}
      \text{ if } & C \subseteq A \text{ and } C \subseteq B, \text{ then } C \subseteq A \cap B \\
      & \forall c \in C, c \in A \text{ and } c \in B \\
      & x \in A \cap B, x \in A \text{ and } x \in B.  \text{ Then } \forall c \in C, c \in A \cap B.  \quad C \subseteq A \cap B 
    \end{aligned}
    \]
    
\exercisehead{17}
\begin{enumerate}
  \item \[
\begin{aligned}
  \text{ if } & A \subset B \text{ and } B \subset C \text{ then } \\
  & \forall a \in A, a \in B.  \forall b \in B, b \in C.  \\ 
  & \text{ then since } a \in B, a \in C, \exists c \in C \text{ such that } c \notin B.  \\
  & \forall a \in A, a \in B \text{ so } a \neq c \forall a.  \Longrightarrow A \subset C 
\end{aligned}
\]
\item If $A \subseteq B, B \subseteq C, A \subseteq C$ since, $\forall a \in A, a \in B, \forall b \in B, b \in C.$  Then since $a \in B, a \in C$.  $A \subseteq C$
\item $A \subset B$ and $B \subseteq C$.  $B \subset C$ or $B=C$.  $A \subset B$ only.  Then $A \subset C$.  
\item Yes, since $\forall a \in A, a \in B$.
\item No, since $x \neq A$ (sets as elements are different from elements)
\end{enumerate}

\exercisehead{18} $ A - (B \cap C) = (A - B) \cup (A-C)$
\[
\begin{aligned}
  \text{ Suppose } & x \in A - (B \cap C) \\
  & \text{ then $ x \in A $ and $x \notin B \cap C \Longrightarrow x \notin B \cap C $ } \\
  & \text{ then $x$ is not in even at least one $B$ or $C$ } \\
  & \Longrightarrow x \in (A-B) \cup (A-C) \\
  \text{ Suppose } & x \in (A-B) \cup (A - C ) \\
  & \text{ then $x$ is at least in $(A-B)$ or in $(A-C) \Longrightarrow x $ is at least in $A$ and not in $B$ or in $A$ and not in $C$ }\\
  & \text{ then consider when one of the cases is true and when both cases are true } \Longrightarrow x \in A - (B \cap C)
\end{aligned}
\]

\exercisehead{19} 
\[
\begin{aligned}
  \text{ Suppose } & x \in B - \bigcup_{A \in \mathcal{F} } A \\
  & \begin{aligned}
    & \text{ then } x \in B, x \notin \bigcup_{A \in \mathcal{F} } A \\
    & \quad x \notin \bigcup_{A \in \mathcal{F} } A \Longrightarrow x \notin A, \forall A \in \mathcal{F} 
  \end{aligned} \\
  & \text{ since } \forall A \in \mathcal{F}, x \in B, x \notin A, \text{ then } x \in \bigcap_{A\in \mathcal{F}} (B-A)
\end{aligned} \quad \quad \quad 
\begin{aligned}
  \text{ Suppose } & x \in \bigcap_{A \in \mathcal{F}} (B-A) \\
  & \begin{aligned}
    & \text{ then } x \in B -A_1 \text{ and } x \in B - A_2 \text{ and } \dots \\ 
    & \text{ then } \forall A \in \mathcal{F}, x \in B, x \notin A \\
    & \text{ then } x \notin \text{ even at least one } A \in \mathcal{F} \\
  \end{aligned} \\
  & \Longrightarrow x \in B - \bigcup_{A \in \mathcal{F}} A 
\end{aligned}
\]
\quad \medskip \\

\[
\begin{aligned}
  \text{ Suppose } & x \in B - \bigcap_{A \in \mathcal{F} } A \\
  & \begin{aligned}
    & \text{ then } x \notin \bigcap_{A \in \mathcal{F}} A \\
    & \text{ then at most $ x \in A$ for $\forall A \in \mathcal{F}$ but one } \\
    & \text{ then $x$ is at least in one $B-A$ }\\
    & \Longrightarrow x \in \bigcup_{A \in \mathcal{F} } (B-A) 
  \end{aligned} \quad \quad \quad 
  \begin{aligned}
    \text{ Suppose } & x \in \bigcup_{A \in \mathcal{F} } (B-A) \\
    & \begin{aligned}
      & \text{ then $ x$ is at least in one $B-A$ } \\
      & \text{ then for $A \in \mathcal{F}, x \in B $ and $x \notin A$ }\\
      & \text{ Consider } \forall A \in \mathcal{F} \\
      & \quad \Longrightarrow \text{ then } x \in B - \bigcap_{A \in \mathcal{F}} A 
    \end{aligned} 
\end{aligned}
\end{aligned}
\]

\exercisehead{20} 
\begin{enumerate}
\item (ii) is correct.  \\
  \[
  \begin{aligned}
    \text{ Suppose } & x \in (A - B) - C \\
    & \begin{aligned}
      & \text{ then } x \in A-B, x\notin C \\
      & \text{ then } x \in A \text{ and } x \notin B \text{ and } x \notin C 
      \end{aligned} \\
    & x \notin B \text{ and } x \notin C \Longrightarrow x \notin \text{ even at least $B$ or $C$ } \\
    & x \in A - (B \cup C )
  \end{aligned}
  \quad \quad \quad 
  \begin{aligned}
    \text{ Suppose } & x \in A - (B \cup C) \\
    & \begin{aligned}
      & \text{ then } x \in A, x\notin (B \cup C) \\
      & \text{ then } x \in A \text{ and } x \notin B \text{ and } x \notin C 
      \end{aligned} \\
      & \Longrightarrow x \in (A - B) - C 
  \end{aligned}
  \]
  To show that (i) is sometimes wrong, \\
  \[
\begin{aligned}
  \text{ Suppose } & y \in A - (B - C) \\
&  \begin{aligned}
    & y \in A  \text{ and } y \notin B - C \\
    & \quad y \notin B - C \\ 
    & \quad \quad \text{ then } y \notin B \text{ or } y \in C \text{ or } y \notin C 
  \end{aligned} \\
& \text{ (where does this lead to?) } \\
\end{aligned}
  \]
Consider directly. 
\[
\begin{aligned}
  \text{ Suppose } & x \in ( A- B) \cup C \\
&  \begin{aligned}
    & \text{ then $x$ is at least in $A-B$ or in $C$ } \\
    & \text{ then $x$ is at least in $A$ and $\notin B$ or in C } 
  \end{aligned} \\
  \text{ Suppose } & x = c \in C \text{ and } c \notin A
\end{aligned}
\]
\item \[
\begin{gathered}
  \text{ If } C \subseteq A, \\
  A- ( B- C) = (A-B) \cup C
\end{gathered}
\] 
\end{enumerate}

%%%%%%%%%%%%%%%%%%%%%%%%%%%%%%%%%%%%%%%%%%%%%%%%%%%%%%%%%%%%%%%%%%%%%%%%%%%%%%%%%%%%%%%%%%%%%%
\subsection*{I 3.3 Exercises - The field axioms }
%%%%%%%%%%%%%%%%%%%%%%%%%%%%%%%%%%%%%%%%%%%%%%%%%%%%%%%%%%%%%%%%%%%%%%%%%%%%%%%%%%%%%%%%%%%%%%

The goal seems to be to abstract these so-called real numbers into just $x$'s and $y$'s that are purely built upon these axioms.  

\exercisehead{1}
Thm. I.5. $a(b-c)=ab-ac$.  
\[
\begin{aligned}
  & \text{ Let } y = ab-ac; x = a(b-c) \\
  & \text{ Want: } x = y \\
  & ac + y = ab \text{ (by Thm. I.2, possibility of subtraction) } \\
  & \text{ Note that by Thm. I.3, } a(b-c) = a(b+ (-c)) = ab + a(-c) \text{ (by distributivity axiom) } \\
  & ac + x = ac + ab + a(-c) = a(c + (-c)) + ab = a( 0+b) = ab 
\end{aligned}
\]
But there exists exactly one $y$ or $x$ by Thm. I.2.  $x=y$.  

Thm. I.6.  $0 \cdot a = a \cdot 0 = 0 $.  
\[
\begin{gathered}
0 (a) = a(0) \text{ (by commutativity axiom) } \\
\text{ Given } b \in \mathcal{R} \text{ and } 0 \in \mathcal{R}, \exists \text{ exactly one } -b \text{ s.t. } b -a =0 \\
0 (a) = (b + (-b)) a = ab -ab = 0 \text{ (by Thm. I.5. and Thm. I.2) } 
\end{gathered}
\]

Thm. I.7. 
\[
\begin{gathered}
ab = ac \\
\text{ By Axiom 4}, \exists y \in \mathcal{R} \text{ s.t. } ay = 1 \\
\text{ since products are uniquely determined, } yab= y ac \Longrightarrow (ya)b = (ya) c \Longrightarrow 1 (b) = 1 (c) \\
\Longrightarrow b = c
\end{gathered}
\]

Thm. I.8. Possibility of Division.  \\
Given $a,b, a\neq 0$, choose $y$ such that $ay = 1$.  \\
Let $x=yb$.  
\[
ax = ayb = 1(b) = b
\]
Therefore, there exists at least one $x$ such that $ax =b$.  But by Thm. I.7, there exists only one $x$ (since if $az-b$, and so $x=z$).  

Thm. I.9. If $a \neq 0$, then $b/a = b (a^{-1})$.   \\
\[
\begin{gathered}
  \begin{aligned}
    \text{ Let } & x = \frac{b}{a} \text{ for } ax = b \\
    & y = a^{-1} \text{ for } ay = 1 \\
  \end{aligned} \\
\text{ Want: } x = by \\
\text{ Now } b(1) = b, \text{ so } ax = b = b (ay) = a(by) \\
\Longrightarrow x = by \text{ (by Thm. I.7) }
\end{gathered}
\]

Thm. I.10.  If $a\neq 0$, then $(a^{-1})^{-1} = a $.  \\
Now $ab=1$ for $b=a^{-1}$.  But since $b \in \mathcal{R}$ and $b\neq 0$ (otherwise $1=0$, contradiction), then using Thm. I.8 on $b$, $ab= b(a) = 1 ; \quad a=b^{-1}$.  

Thm.I.11.  If $ab=0, a=0$ or $b=0$.  \\
$ab = 0 = a(0) \Longrightarrow b = 0$ or $ab = ba = b(0) \Longrightarrow a =0$.  (we used Thm. I.7, cancellation law for multiplication)

Thm. I.12.  Want: $x=y$ if $x = (-a)b$ and $y=-(ab)$.  \\
\[
\begin{gathered}
ab + y =0 \\
ab + x = ab + (-a)b = b(a+ (-a)) = b (a -a ) = b(0) = 0 \\
0 \text{ is unique, so } ab + y = ab + x \text{ implies } x = y (\text{ by Thm. I.1 } )
\end{gathered}
\]

Thm. I.13.  Want: $x + y =z$, if $a=bx, c=dy, (ad+bc) = (bd)z$.  
\[
(bd)(x+y) = bdx+bdy = ad +bc = (bd)z
\]
So using $b,d \neq 0$, which is given, and Thm. I.7, then $x+y=z$.  

Thm. I.14.  Want: $xy=z$ for $bx = a, dy=c, ac= (bd)z$.  
\[
(bd)(xy) = (bx)(dy) = ac = (bd)z
\]
$b, d \neq 0$, so by Thm. I.7, $xy=z$.  

Thm.I.15.  Want: $x = yz$, if $bx=a, dy=c, (bc)z = ad$
\[
\begin{aligned}
& (bc)z = b(dy)z = d(byz) = da \\
& d\neq 0 \text{ so by Thm. I.7, by } z = a, byz = a bx \\
& b \neq 0 \text{ so by Thm. I.7}, yz = x
\end{aligned}
\]

\exercisehead{2} Consider $0+z = 0$.  By Thm. I.2, there exists exactly one $z$, $z = -0$.  By Axiom 4, $z=0$.  $0=-0$.  

\exercisehead{3} Consider $1(z) z(1) = 1$.  Then $z=1^{-1}$.  But by Axiom 4, there exists distinct $1$ such that $z(1) = 1$, so $z=1$.  

\exercisehead{4} Suppose there exists $x$ such that $0x=1$, but $0x = 0$ and $0$ and $1$ are distinct, so \emph{ zero has no reciprocal }.  

\exercisehead{5} $a+(-a) =0 , 0+0=0$.  Then
\[
\begin{gathered}
  a + (-a) +b + (-b) = (a+b) + (-a) +(-b) = 0 \\
  -(a+b) = -a +(-b) = -a-b   
\end{gathered}
\]

\exercisehead{6} $a+(-a) = 0, b+(-b) = 0$, so
\[
a+(-a) + b + (-b) = a+(-b) + (-a)+b = (a-b) + (-a) + b = 0 + 0 = 0
\]
$-(a-b) = -a + b$.  

\exercisehead{7} \[
(a-b)+(b-c) = a+(-b) + b + (-c) = a + (b + (-b)) + (-c) = a-c
\]

\exercisehead{8}
\[
\begin{aligned}
& (ab) x = 1 \quad \quad & (ab)^{-1} = x \\
& a(bx) = 1  & a^{-1} = bx \\
& b(ax) = 1 & b^{-1} - ax 
\end{aligned}
\]
\[
a^{-1}b^{-1} = (abx)x = 1(x) = (ab)^{-1}
\]

\exercisehead{9} Want: $x=y=z$, if 
\begin{align*}
& z = \frac{a}{-b}  \quad & a = zt \quad \quad & b+t = 0 \\
& y = \frac{(-a)}{b} \quad & by =u \quad & a+ u = 0 \\
& x = -\left( \frac{a}{b} \right) \quad & \left( \frac{a}{b} \right) + x = v+x = 0 \quad &  vb =a 
\end{align*}

\[
\begin{gathered}
a+(-a) = vb + by = b (v+y) = 0 \\
\text{ if } b\neq 0, v+y=0, \text{ but } v+x = 0 \\
\text{ by Thm. I.1 }, x = y 
\end{gathered}
\]
\[
\begin{gathered}
  b+t = 0, \text{ then } z (b+t) = zb + zt  = zb + a = z (0) = 0 \\
  a+ zb = 0 \Longrightarrow -a = zb = by \\
\text{ since } b\neq  0, z = y \text{ so } x = y =z
\end{gathered}
\]

\exercisehead{10} Since $b,d \neq 0$, 
Let 
\[
\begin{aligned} 
  & z = \frac{ ad -bc}{bd}  & (bd)z = ad - bc \, \text{ by previous exercise or Thm. I.8, the possibility of division } \\
  & x = \frac{a}{b}  & bx = a \\
  & t = \frac{-c}{d} & dt = -c \text{ (By Thm. I.3, we know that $b-a = b + (-a)$ ) }
\end{aligned}
\]
\[
\begin{gathered}
  dbx + bdt = (bd) (x+y) = ad-bc =(bd)z   \\
  b,d \neq 0, \text{ so } x+ y = z
\end{gathered}
\]

%-----------------------------------%-----------------------------------%-----------------------------------
\subsection*{ I 3.5 Exercises - The order axioms }
%-----------------------------------%-----------------------------------%-----------------------------------

\begin{theorem}[I.18] If $a<b$ and $c>0$ then $ac<bc$ \end{theorem}
\begin{theorem}[I.19] If $a<b$ and $c>0$, then $ac<bc$ \end{theorem}
\begin{theorem}[I.20] If $a\neq 0$, then $a^2>0$ \end{theorem}
\begin{theorem}[I.21] $1>0$ \end{theorem}
\begin{theorem}[I.22] If $a<b$ and $c<0$, then $ac > bc$.  \end{theorem}
\begin{theorem}[I.23] If $a<b$ and $-a > -b$.  In particular, if $a<0$, then $-a >0$.   \end{theorem}
\begin{theorem}[I.24] If $ab>0$, then both $a$ and $b$ are positive or both are negative.  \end{theorem}
\begin{theorem}[I.25] If $a<c$ and $b<d$, then $a+b<c+d$.  \end{theorem}

\exercisehead{1} \begin{enumerate}
\item By Thm. I.19, $-c>0$
\[
\begin{gathered}
  a(-c) < b(-c) \to -ac < -bc  \\
  -bc - (-ac) = ac -bc > 0.  \text{ Then } ac >bc \text{ (by definition of $>$ ) }  
\end{gathered}
\]
\item \[
  \begin{gathered}
    a<b \to a+0 < b+0 \to a+b +(-b) < b +a + (-a) \to (a+b) -b < (a+b) + (-a)  \\
    \text{ By Thm.I.18 } (a+b) + -(a+b) + (-b) < (a+b) -(a+b) + (-a)  \\
    -b < -a
  \end{gathered}
\]
\item 
\[
\begin{aligned}
& \text{ If  $a=0$ or $b=0$, $ab=0$, but $0 \ngtr 0$ } \\
& \text{ If  $a>0$, then if $b>0$, $ab> 0 (b) = 0$.  If $b<0, ab < 0 (b) =0$.  So if $a>0$, then $b>0$. } \\
& \text{ If  $a<0$, then if $b>0, ab < 0(b) = 0 $.  If $b<0, ab> 0(b)= 0 $.  So if $a<0$, then $b<0$.  }
\end{aligned}
\]
\item \[
\begin{gathered}
  a<c \text{ so } a+b <c+b = b+c  \\
  b<d \text{ so } b+c < d+c  \\
  \text{ By Transitive Law }, a+b<d+c
\end{gathered}
\]
\end{enumerate}

\exercisehead{2} If $x=0, x^2 =0$.  $0+1 = 1 \neq 0$.  So $x \neq 0$.  
\[
\begin{gathered}
  \text{ If } x\neq 0, x^2 > 0, \text{ and by Thm. I.21 }, 1 > 0  \\
  x^2 +1 > 0 + 0 = 0 \to x^2 + 1 \neq 0  \\
  \Longrightarrow \nexists x \in \mathbb{R} \text{ such that } x^2 +1 =0 
\end{gathered}
\]

\exercisehead{3} 
\[
a<0, b < 0 , a+b < 0 + 0 = 0 \text{ ( By Thm. I.25) }
\]

\exercisehead{4} Consider $ax=1$.  
\[
ax = 1 > 0.  \text{ By Thm. I.24 }, a,x \text{ are both positive or $a,x$ are both negative }
\]
\exercisehead{5} Define $x,y$ such that $ax=1, by=1$.  We want $x>y$ when $b>a$.  
\[
\begin{gathered}
xb-ax =xb - 1 > 0 \Longrightarrow bx > 1 = by \\
b > 0 \text{ so } x>y
\end{gathered}
\]

\exercisehead{6} \[
\begin{aligned}
  & \text{ If } a=b \text{ and } b=c, \text{ then } a=c \\
  & \text{ If } a=b \text{ and } b<c, \text{ then } a<c \\
  & \text{ If } a<b \text{ and } b=c, \text{ then } a< c \\
  & \text{ If } a<b \text{ and } b<c, \text{ then } a<c \text{ (by transitivity of the inequality) } \\
  & \Longrightarrow a \leq c
\end{aligned}
\]

\exercisehead{7} If $a\leq b$ and $b\leq c$, then $a\leq c$.  If $a=c$, then by previous proof, $a=b$.  

\exercisehead{8} 
If $a \leq b$ and $b\leq c$, then $a\leq c$.  If $a=c$, then by previous proof, $a=b$.  

\exercisehead{8} If $a$ or $b$ is zero, $a^2$ or $b^2 = 0 $.  By Thm. I.20, $b^2 \geq 0$ or $a^2 \geq 0$, respectively.  

Otherwise, if neither are zero, by transitivity, $a^2+b^2 > 0$.  

\exercisehead{9} Suppose $a\geq x$.  Then $a-x \geq 0$.  

If $a \in \mathbb{R}$ so $\exists y \in \mathbb{R}$, such that $a-y = 0 $.  \\
\phantom{If} Consider $y+1 \in \mathbb{R}$ (by closure under addition).  
\[
a-(y+1) = a-y-1 = 0-1 < 0 \text{ Contradiction that } a \geq y+1
\]

\exercisehead{10} \[
\begin{aligned}
  & \text{ If } x = 0, \text{ done. }  \\
  & \text{ If } x > 0, x \text{ is a positive real number.  Let $ h=\frac{x}{2}$ }.  \\
  & \Longrightarrow \frac{x}{2} > x \text{ Contradiction.} 
\end{aligned}
\]

\subsection*{ I 3.12 Exercises - Integers and rational numbers, Geometric interpretation of real numbers as points on a line, Upper bound of a set, maximum element, least upper bound (supremum), The least-upper-bound axiom (completeness axiom), The Archimedean property of the real-number system, Fundamental properties of the suprenum and infimum }

We use Thm I.30, the Archimedean property of real numbers, alot.
\begin{theorem}[I.30] If $x > 0 $ and if $y$ is an arbitrary real number, there exists a positive integer $n$ such that $nx >y$.  
\end{theorem}

We will use the least upper-bound axiom (completeness axiom) alot for continuity and differentiation theorems later.  Apostol states it as an axiom; in real analysis, the existence of a sup for nonempty, bounded sets can be shown with an algorithm to zoom into a sup with monotonically increasing and monotonically decreasing sequence of ``guesses'' and showing its difference is a Cauchy sequence.  

\begin{axiom}[Least upper-bound axiom]
Every nonempty set $S$ of real numbers which is bounded above has a suprenum; that is, there's a real number $B$ s.t. $B=sup S$.  
\end{axiom}

\exercisehead{1} $0 < y-x$.  
\[
\begin{gathered}
  \Longrightarrow n(y-x) > h > 0 , n \in \mathbb{Z}^+, h \text{ arbitrary } \\
  y - x > h/n \Longrightarrow y > x + h/n > x \\
\text{ so let } z = x + h/n \text{ Done. }
\end{gathered}
\]

\exercisehead{2} $x\in \mathbb{R}$ so $\exists n \in \mathbb{Z}^+$ such that $n>x$ (Thm. I.29).  

Set of negative integers is unbounded below because \\
\quad If $\forall m \in \mathbb{Z}^-, -x > -m$, then $-x$ is an upper bound on $\mathbb{Z}^+$.  Contradiction of Thm. I.29.  
$\Longrightarrow \exists m \in \mathbb{Z}$ such that $ m < x < n$

\exercisehead{3} Use Archimedian property.  \\
$x > 0$ so for $1, \exists n \in \mathbb{Z}^+$ such that $nx >1, x > \frac{1}{n}$.  

\exercisehead{4} $x$ is an arbitrary real number.  By Thm. I.29 and well-ordering principle, there exists a smallest $n+1$ positive integer such that $x< n+1$ (consider the set of all $m+1 > x$ and so by well-ordering principle, there must be a smallest element of this specific set of positive integers).  

If $x = n$ for some positive integer $n$, done.  

Otherwise, note that if $x <n $, then $n+1$ couldn't have been the smallest element such that $m>x$.  $x >n$.  

\exercisehead{5} If $x=n$, done.  Otherwise, consider all $m >x$.  By well-ordering principle, there exists a smallest element $n$ such that $n>x$.  

If $x+1<n$, then $x<n-1$, contradicting the fact that $n$ is the smallest element such that $x<n$.  Thus $x+1>n$.  

\exercisehead{6} $y-x >0$.  
\[
\begin{gathered}
  n(y-x) > h, \, h \text{ arbitrary }, n \in \mathbb{Z}^+  \\
  y > x + h/n = z > x
\end{gathered}
\]
Since $h$ was arbitrary, there are infinitely many numbers in between $x,y$.  

\exercisehead{7} $x = \frac{a}{b} \in \mathbb{Q}, y \notin \mathbb{Q}$.  
\[
\begin{aligned}
  x \pm y = \frac{ a\pm by}{b} & \\
  & \text{ If $a\pm by$ was an integer, say $m$, then $y = \pm \left( \frac{ a - mb}{b } \right) $ which is rational.  Contradiction.}   \\
  xy = \frac{a}{b} \frac{y}{1} & = \frac{ay}{b}  \\
  & \text{ If $ay$ was an integer, $ay=n$, $y=\frac{n}{a}$, but $y$ is irrational. $\Longrightarrow xy$ is irrational. } \\
  \frac{x}{y} & \\
  & \text{ $y$ is not an integer }
\end{aligned}
\]

\exercisehead{8} Proof by counterexamples.  We want that the sum or product of 2 irrational numbers is not always irrational.  If $y$ is irrational, $y+1$ is irrational, otherwise, if $y+1 \in Q, \, y \in Q$ by closure under addition.  \\
$\quad \Longrightarrow y+1 -y = 1$  

Likewise, $y \frac{1}{y}  = 1$.  

\exercisehead{9} 
\[
\begin{gathered}
  y-x > 0 \Longrightarrow n(y-x) > k, n \in \mathbb{Z}^+, k \text{ arbitrary.  Choose $k$ to be irrational.  Then $k/n$ irrational. } \\
  y > \frac{k}{n} + x > x.  \text{ Let } z = x + \frac{k}{n}, z \text{ irrational }.  
\end{gathered}
\]

\exercisehead{10} 
\begin{enumerate}
  \item Suppose $n=2m_1$ and $n+1 = 2 m_2$.  \\
    \[
2m_1 +1 = 2 m_2 \quad 2(m_1 - m_2) = 1 \quad m_1 - m_2 = \frac{1}{2}.  \text{ But $m_1-m_2$ can only be an integer.  }
\]
  \item By the well-ordering principle, if  $x \in \mathbb{Z}^+$ is neither even and odd, consider the set of all $x$.  There must exist a smallest element $x_0$ of this set.  But since $x_0 \in \mathbb{Z}^+$, then there must exist a $n < x$ such that $n +1 = x_0$.  $n$ is even or odd since it doesn't belong in the above set.  So $x_0$ must be odd or even.  Contradiction.  
  \item \[
    \begin{aligned}
      & (2m_1)(2m_2) = 2(2m_1 m_2) \text{ even } \\
      & 2m_1 + 2m_2 = 2(m_1 +m_2) \text{ even }  \\
      & (2m_1+1)+ (2m_2 +1) = 2(m_1+m_2+1) \Longrightarrow \text{ sum of two odd numbers is even } \\
      & 
      \begin{gathered} 
	(n_1+1)(n_2+1) = n_1 n_2 + n_1 + n+2 + 1 = 2(2m_1 m_2)  \\
	2(2m_1 m_2)- (n_1+n_2) -1 \text{ odd, the product of two odd numbers $n_1,n_2$ is odd }
      \end{gathered}
    \end{aligned}
    \]
\item If $n^2$ even, $n$ is even, since for $n=2m$, $(2m)^2 = 4m^2 = 2(2m^2)$ is even.  \\ \, \\
  $a^2 = 2b^2$.  $2(b^2)$ even.  $a^2$ even, so $a$ even.  
  \[
  \begin{aligned}
    &  \text{ If $a$ even } a = 2n.  a^2 = 4 n^2 \\
    &  \text{ If $b$ odd , $b^2$ odd.  $b$ has no factors of $2$ } b^2 \neq 4 n^2  
  \end{aligned}
  \]
Thus $b$ is even.  
\item For $\frac{p}{q}$, If $p$ or $q$ or both are odd, then we're done.  \\
  Else, when $p,q$ are both even, $p=2^l m, q = 2^n p, m, p $ odd.  
  \[
  \frac{p}{q} = \frac{ 2^l m}{ 2^n p} = \frac{ 2^{l-n} m}{p} \text{ and at least $m$ or $p$ odd }
  \]
\end{enumerate}

\exercisehead{11} $\frac{a}{b}$ can be put into a form such that $a$ or $b$ at least is odd by the previous exercise.  

However, $a^2=2b^2$, so $a$ even, $b$ even, by the previous exercise, part (d) or 4th part.  Thus $\frac{a}{b}$ cannot be rational.  

\exercisehead{12} \emph{ The set of rational numbers satisfies the Archimedean property but not the least-upper-bound property.}  \\
Since $\frac{p}{q} \in \mathbb{Q} \subseteq \mathbb{R}$, $n\frac{p_1}{q_1} > \frac{p_2}{q_2}$ since if $q_1,q_2 >0$, 
\[
\frac{n p_1 q_2}{q_1 q_2} > \frac{ q_1 p_2}{q_1 q_2}  \quad np_1 q_2 > q_1 p_2
\]
$n$ exists since $(p_1 q_2), (q_1 p_2) \in \mathbb{R}$.  

\emph{The set of rational numbers does not satisfy the least-upper-bound property}.  \\
Consider a nonempty set of rational numbers $S$ bounded above so that $\forall x = \frac{r}{s} \in S, x < b$.  \medskip \\
Suppose $x < b_1, x <b_2 \forall x \in S$.  
\[
\frac{r}{s} < b_2 < n b_1 \text{ but likewise } \frac{r}{s} < b_1 < m b_2 , \, n, m \in \mathbb{Z}^+
\]
So it's possible that $b_1 > b_2$, but also $b_2 > b_1$.  


%-----------------------------------%-----------------------------------%-----------------------------------
\subsection*{ I 4.4 Exercises - An example of a proof by mathematical induction, The principle of mathematical induction, The well-ordering principle }
%-----------------------------------%-----------------------------------%-----------------------------------

Consider these 2 proofs. 

\[
\begin{gathered}
  N + N + \dots + N  = N^2 \\
  (N-1) + (N-2) + \dots + (N-(N-1)) + (N-N) = N^2 -\sum_{j=1}^N j = \sum_{j=1}^{N-1} j \\
  N^2 + N = 2 \sum_{j=1}^N j \quad \Longrightarrow \sum_{j=1}^N j = \frac{ N(N+1)}{2}
\end{gathered}
\]

An interesting property is that 
\[
S = \sum_{j=m}^n j = \sum_{j=m}^n (n+m-j) 
\]
So that
\[
\begin{gathered}
  \sum_{j=1}^N j = \sum_{j=m}^N j + \sum_{j=1}^m j = \sum_{j=m}^N j + \frac{ m(m+1)}{2} = \frac{ N(N+1)}{2}  \\
  \sum_{j=m}^N j = \frac{ N(N+1) - m(m+1)}{2} = \frac{ (N-m)(N+m +1)}{2}
\end{gathered}
\]

Another way to show this is the following.  
\[
\begin{gathered}
  \begin{matrix}
    S = & 1 + & 2 + & \dots + & (N-2) + & (N-1) + & N  \\
    \text{ but } S = & N + & N - 1 + & \dots + & 3 + & 2 + & 1  
  \end{matrix} \\
  2S = (N+1)N  \quad S = \frac{N(N+1)}{2}  
\end{gathered}
\]

Telescoping series will let you get $\sum_{j=1}^N j^2$ and other powers of $j$.  
\[
\begin{gathered}
  \sum_{j=1}^N (2j -1) = 2 \frac{ N(N+1)}{2} - N = N^2 \\
  \begin{aligned}
    & \sum_{j=1}^N (j^2 - (j-1)^2 ) = \sum_{j=1}^N (j^2 - (j^2 - 2j +1)) = \sum_{j=1}^N (2j-1) = 2 \left( \frac{N(N+1)}{2} \right) - N = N^2  \\
    & \sum_{j=1}^N (j^3 - (j-1)^3 ) = N^3 = \sum_{j=1}^N (j^3 - (j^3 - 3j^2 + 3j -1)) =\sum_{j=1}^N (3j^2 - 3j +1)  
  \end{aligned}  \\
  \Longrightarrow 3 \sum_{j=1}^N j^2 = - 3 \frac{ N(N+1)}{2} + N = N^3 \Longrightarrow \frac{ 2 N^3 +2N -3N^2 -3N}{2} = \frac{ N(N+1)(2N+1) }{6} = \sum_{j=1}^N j^2 \\
  \begin{aligned}
    & \sum_{j=1}^N j^4 - (j-1)^4 = N^4 = \sum_{j=1}^N j^4 -(j^4 -4 j^3 + 6j^2 -4j +1) = \sum_{j=1}^N 4j^3 -6j^2 +4j -1 =  \\
    & = 4 \sum_{j=1}^N j^3 - 6 \frac{ N(N+1)(2N+1)}{6} + 4 \frac{ N(N+1)}{2} - N = N^4  \\
    \Longrightarrow \sum_{j=1}^N j^3 & =\frac{1}{4} (N^4 + N(N+1)(2N+1)-2N(N+1)+N) = \frac{1}{4} (N^4 + (2N)N(N+1) - N(N+1) + N) \\
    & = \frac{1}{4} (N^4 +2N^3 +2N^2 -N^2 -N +N) = \frac{1}{4} N^2 (N^2 +2N +1) = \frac{1}{4} \frac{ (N(N+1))^2 }{2}
  \end{aligned}
\end{gathered}
\]

\exercisehead{1} Induction proof.  
\[
\frac{ 1 (1+1)}{2} \quad \sum_{j=1}^{N+1} j = \sum_{j=1}^n j + n+1 = \frac{ n(n+1)}{2} + n+1 = \frac{ n(n+1) + 2 (n+1)}{2} = \frac{ (n+2)(n+1)}{2}
\]

\exercisehead{6}
\begin{enumerate}
\item 
\[
A(k+1) = A(k) + k+1 = \frac{1}{8} (2k+1)^2 + k+1 = \frac{1}{8} (4 k^2 + 4k + 1 ) + \frac{ 8k+8}{8} = \frac{ (2k+3)^2 }{8}  
\]
\item The $n=1$ case isn't true.  
\item 
\[
\begin{gathered}
  1+2 + \dots + n = \frac{ (n+1)n}{2} = \frac{ n^2+ n}{2} < \frac{ n^2+ n+\frac{1}{4} }{2}  \\
  \text{ and } \left( \frac{2n+1}{2} \right)^2 \frac{1}{2} = \frac{ (n+1/2)^2}{2} = \frac{ n^2 + n+1/4}{2}
\end{gathered}
\]
\end{enumerate}

\exercisehead{7}
\[
\begin{gathered}
  \begin{aligned} 
    & (1+x)^2 > 1 + 2 x + 2x^2  \\
    & 1+ 2x + x^2 > 1 + 2x + 2x^2  \\
    & 0 > x^2 \Longrightarrow \text{ Impossible }
  \end{aligned} \\
\begin{gathered}
  (1+x)^3 = 1 + 3 x + 3x^2 + x^3 > 1 + 3x + 3 x^2 \\
  \Longrightarrow x^3 > 0 
\end{gathered}
\end{gathered}
\]
By well-ordering principle, we could argue that $n=3$ must be the smallest number such that $(1+x)^n > 1+2x + 2x^2$.  Or we could find, explicitly
\[
(1+x)^n = \sum_{j=0}^n \binom{n}{j} x^j = 1 + nx + \frac{ n(n-1)}{2} x^2 + \sum_{j=3}^n \binom{n}{j} x^j  
\]
and 
\[
\begin{aligned}
  \frac{ n(n-1)}{2} & > n \\
  n^2 - n & > 2 n \\
  n^2 & > 3 n \\
  n & > 3
\end{aligned}
\]

\exercisehead{8} 
\[
\begin{aligned}
  & a_2 \leq c a_1, a_3 \leq c a_2 \leq c^2 a_1  \\ 
  & a_{n+1} \leq c a_n \leq c a_1 c^{n-1} = a_1 c^n 
\end{aligned}
\]

\exercisehead{9} 
\[
\begin{aligned}
  & n=1, \, \sqrt{1} = 1 \\
  & \sqrt{ 1^2 +1^2 } = \sqrt{2} \, \sqrt{ (\sqrt{2})^2 + 1^2 } = \sqrt{3} \\
  & \sqrt{ (\sqrt{n})^2 + 1^2 } = \sqrt{ n+1}
\end{aligned}
\]

\exercisehead{10}\[ 
\begin{aligned}
  & 1 = qb + r  \\
  & \quad q = 0, b = 1, r = 1 \\
  & 2 = qb + r, q = 0, r =2, b = 1,2 \text{ or } r = 0, q =2; q=1, r = 0   \\
  & \quad \text{ Assume } n = qb + r; 0 \leq r < b; b \in \mathbb{Z}^+, b \text{ fixed }  \\
  & \quad  n+1 = qb + r + 1 = qb + 1 + r = qb + 1 + b -1 = (q+1)b + 0  
\end{aligned}
\]

\exercisehead{11} For $n>1$, $n=2,3$ are prime.  $n=4 = 2(2)$, a product of primes. 

Assume the $k-1$th case.  Consider $\frac{k}{j}$, $1 \leq j \leq k$. \\
If $\frac{k}{j} \in \mathbb{Z}^+$, only for $j=1, j=k$, then $k$ prime.  \\
If $\frac{k}{j} \in \mathbb{Z}^+$, for some $1<j<k$, $\frac{k}{j} = c \in \mathbb{Z}^+$.  $c,j <k$.  

Thus $k=cj$.  $c,j$ are products of primes or are primes, by induction hypothesis.  Thus $k$ is a product of primes.  

\exercisehead{12} $n=2$.  $G_1,G_2$ are blonde.  $G_1$ has blue eyes.  Consider $G_2$.  $G_2$ may not have blue eyes.  Then $G_1,G_2$ are not all blue-eyed.  

%-----------------------------------%-----------------------------------%-----------------------------------
\subsection*{ I 4.7 Exercises - Proof of the well-ordering principle, The summation notation }
%-----------------------------------%-----------------------------------%-----------------------------------
\quad \\
\exercisehead{1} \begin{enumerate}
\item $\frac{n(n+1)}{2} = \sum_{k=1}^4 k = 10 $
\item $\sum_{n=2}^5 2^{n-2} = \sum_{n=0}^3 2^n = 1+14=15$ 
\item $2 \sum_{r=0}^3 2^{2r} = 2 \sum_{r=0}^3 4^r = 170$
  \item $\sum_{j=1}^4 j^j = 1+4+27+4^4 = 288$
  \item $\sum_{j=0}^5 (2j+1) = 2 \frac{5(6)}{2} + 6(1) = 36 $
  \item $\sum \frac{1}{k(k+1)} = \sum_{k=1}^n \left( \frac{1}{k} -\frac{1}{k+1} \right) = 1 - \frac{1}{n+1} = \frac{n}{n+1}  $
\end{enumerate}

\exercisehead{2} \begin{enumerate}
  \item Want: $\sum_{k=1}^n (a_k + b_k) = \sum_{k=1}^n a_k + \sum_{k=1}^n b_k \quad \, \text{(additive property)}$ 
\[
\begin{aligned}
  a_1 + b_1 & = a_1 + b_1 \\
  (a_1 + b_1) + (a_2 + b_2) & = (a_1+ a_2) + (b_1+b_2) \\
  (a_1+ b_1) + (a_2+b_2) + (a_3 + b_3) & = (a_1 + a_2 + a_3) + (b_1 + b_2 + b_3) \\
  \sum_{k=1}^{n+1} (a_k+b_k) = \sum_{k=1}^n (a_k + b_k) + a_{n+1} + b_{n+1} & = \sum_{k=1}^n a_k + a_{n+1} + \sum_{k=1}^n b_k + b_{n+1} = \sum_{k=1}^{n+1} a_k + \sum_{k=1}^{n+1} b_k 
\end{aligned}
\]
\item Want: $\sum_{k=1}^n (ca_k) = c\sum_{k=1}^n a_k$ \quad \, (homogeneous property).  
\[
\begin{aligned}
  ca_1 & = (c)a_1 \\
  ca_1 + ca_2 & = c(a_1 + a_2) \\
  ca_1 + ca_2 + ca_3 & = c(a_1 + a_2 + a_3) \\
  \sum_{k=1}^{n+1} (ca_k) & = c\sum_{k=1}^n a_k + ca_{n+1} = c \sum_{k=1}^{n+1} a_k 
\end{aligned}
\]
\item Want: $\sum_{k=1}^n (a_k - a_{k-1}) = a_n - a_0$ \quad \, (telescoping property) 
\[
\begin{aligned}
  \sum_{k=1}^n (a_k - a_{k-1}) & = \sum_{k=1}^n (a_k + (-a_{k-1})) = \sum_{k=1}^n a_k + \sum_{k=1}^n (-1)a_{k-1} = \\
  & = a_n + \sum_{k=1}^{n-1} a_k + (-1)\sum_{k=1}^{n-1} a_k -a_0 = a_n - a_0 
\end{aligned}
\]
\end{enumerate}

\exercisehead{3} $\sum_{k=1}^n 1 = \sum_{k=1} (k - (k-1)) = n $ 

\exercisehead{4} \[
\begin{gathered}
  k^2 - (k-1)^2 = k^2 - (k^2 - 2k + 1) = 2k - 1 \\
\sum_{k=1}^n (2k-1) = \sum_{k=1}^n k^2 - (k-1)^2 = n^2 - 0 = n^2 
\end{gathered}
\]

\exercisehead{5} $\sum_{k=1}^n k = \frac{n^2 + n}{2} = \frac{ n (n+1)}{2}$

\exercisehead{6} 
\[
\begin{gathered}
  \sum_{k=1}^n k^3 - (k-1)^3 = n^3 = \sum_{k=1}^n 3k^2 - 3k + 1 = 3 \left( \sum_{k=1}^n (k^2) - \frac{ n (n+1)}{2} + \frac{n}{3} \right) \\
  \Longrightarrow \sum_{k=1}^n k^2 = \frac{n^3}{3} + \frac{n^2}{2} + \frac{n}{6}
\end{gathered}
\]

\exercisehead{7} 
\[
\begin{gathered}
  k^4 - (k-1)^4 = - (-4k^3 + 6k^2 + - 4k + 1) = 4k^3 -6k^2 + 4k - 1 \\
  \sum_{k=1}^n k^4 - (k-1)^4 = n^4 = 4 \sum_{k=1}^n k^3 - 6 \left( \frac{n^3}{3} + \frac{n^2}{2} + \frac{n}{6} \right) + 4 \left( \frac{n^2 }{2} + \frac{n}{2} \right) - n \\
  \Longrightarrow \frac{n^4+2n^3 + n^2 }{4} = \sum_{k=1}^n k^3 
\end{gathered}
\]

\exercisehead{8} \begin{enumerate}
\item \[
\begin{gathered}
  -\sum_{k=1}^n (x^{k+1} - x^k) = -(x^{n+1} - x) = \\
  =  (1-x) \sum_{k=1}^n x^k = x(1-x^n) \\
  \Longrightarrow \sum_{k=0}^n x^k = \frac{ x(1-x^n)}{1-x} + \frac{1-x}{1-x} = \frac{1-x^{n+1}}{1-x}
\end{gathered}
\]
\item If $x=1$, the sum equals $(n+1)$.  
\end{enumerate}

\exercisehead{9} \[
\begin{aligned}
  n=1 & (-1)(3) + 5 = 2 = 2n  \\
  n=2 & (-1)(3) + 5 + (-1)7 + 9 = 4 = 2n \\
  n & \sum_{k=1}^{2n} (-1)^k (2k+1) = 2n \\
  n+1 & \sum_{k=1}^{2(n+1)} (-1)^k (2k+1) = \sum_{k=1}^{2n}(-1)^k (2k+1) + (-1)^{2n+1} (4n+3) + (-1)^{2n+2}(4n +5) = \\
  & = 2n+2 = 2(n+1)
\end{aligned}
\]

\exercisehead{10} \begin{enumerate}
  \item $a_m + a_{m+1} + \dots + a_{m+n} $ 
  \item \[ 
    \begin{aligned}
      n=1 & \frac{1}{2} = \frac{1}{1} - \frac{1}{2} = \frac{1}{2}  \\
      n+1 & \sum_{k={n+2}}^{2(n+1)} \frac{1}{k} = \sum_{m=1}^{2n} \frac{(-1)^{m+1}}{m} - \frac{1}{n+1} + \frac{1}{2n+1} + \frac{1}{2n+2} = \sum_{m=1}^{2n} \frac{ (-1)^{m+1} }{m} + - \frac{1}{2(n+1)} + \frac{ (-1)^{2n+1+1} }{ (2n+1)}  \\
      & = \sum_{m=1}^{2(n+1)} \frac{ (-1)^{m+1}}{m}
    \end{aligned}
    \]
\end{enumerate}

\exercisehead{13} 
\[
\begin{aligned}
  n=1 & 2 (\sqrt{2} -1) < 1 < 2 \text{ since } \frac{1}{2} > \sqrt{2} - 1 \\
  n \text{ case } & (\sqrt{ n+1} - \sqrt{n} )(\sqrt{n+1} + \sqrt{n} ) = n+1 - n = 1 < \frac{ \sqrt{n+1} + \sqrt{n} }{ 2 \sqrt{n}} = \frac{1}{2} (\sqrt{ 1 + \frac{1}{n} } + 1) \\
  n+1 \text{ case } & (\sqrt{ n+2 } - \sqrt{ n+1 })( \sqrt{ n+2 } + \sqrt{ n+1} ) = n+2- (n+1) = 1 \\
  & \frac{ \sqrt{n+2} + \sqrt{n+1} }{ 2 \sqrt{n +1} } = \frac{ 1 + \sqrt{ 1 + \frac{1}{n+1} } }{2} > 1 
\end{aligned}
\]
So then, using the telescoping property,
\[
\sum_{n=1}^{n-1} 2 ( \sqrt{n+1} - \sqrt{n} ) = 2 (\sqrt{m} -1 ) < \sum_{n=1}^m \frac{1}{ \sqrt{n}} < \sum_{n=1}^m 2 (\sqrt{n} - \sqrt{n -1} ) = 2 ( \sqrt{m} -1 ) < 2 \sqrt{m} -1 
\]


%-----------------------------------%-----------------------------------%-----------------------------------
\subsection*{ I 4.9 Exercises - Absolute values and the triangle inequality }
%-----------------------------------%-----------------------------------%-----------------------------------

\exercisehead{1} 
\begin{enumerate}
\item $|x| = 0 \,  \text{iff} \,  x = 0 $  \\
  If $x=0$, $x = 0 , -x = -0 = 0$.  If $|x|=0, x=0 , -x = 0$.  
\item \[
  |-x| = \begin{cases} -x & \text{ if } -x \geq 0  \\  x & \text{ if } -x \leq 0 \end{cases} = \begin{cases} x & \text{ if } x \geq 0 \\  -x & \text{ if } x \leq 0 \end{cases}
\]
\item $|x-y|=|y-x|$ by previous exercise and $(-1)(x-y) = y-x$ (by distributivity)
\item $|x|^2 = \begin{cases} (x)^2 & \text{ if } x \geq 0 \\ (-x)^2 & \text{ if } x \leq 0 \end{cases} = x^2 $ 
\item $\sqrt{x^2} = \begin{cases} x & \text{ if } x \geq 0 \\  -x & \text{ if } x \leq 0 \end{cases} = |x|  $
\item We want to show that $|xy| = |x||y|$  
\[
\begin{aligned}
& |xy| = \begin{cases} xy & \text{ if } xy \geq 0  \\ -xy & \text{ if } xy \leq 0 \end{cases}  = \begin{cases} xy & \text{ if } x,y \geq 0 \text{ or } x,y \leq 0 \\  -xy & \text{ if } x,-y \geq 0 \text{ or } -x,y \leq 0 \end{cases} \\
& |x||y| = \begin{cases}  x|y| & \text{ if } x\geq 0 \\  -x|y| & \text{ if } x\leq 0 \end{cases} = \begin{cases} xy & \text{ if } x,y \geq 0 \\ -xy & \text{ if } x,-y \geq 0 \\ -xy & \text{ if } -x,y \geq 0 \\ xy & \text{ if } -x,-y \geq 0 \end{cases} 
\end{aligned}
\]  
\item By previous exercise, since 
\[
\begin{gathered}
  \left| \frac{x}{y} \right| = |xy^{-1}| = |x||y^{-1}|  \\
  \left| \frac{1}{y} \right| = \begin{cases} \frac{1}{y} & \text{ if } \frac{1}{y} \geq 0 \\  \frac{-1}{y} & \text{ if } \frac{1}{y} \leq 0 \end{cases} \quad \frac{1}{ |y|} = \begin{cases}  \frac{1}{y} & \text{ if } \frac{1}{y} \geq 0 \\  \frac{-1}{y} & \text{ if } \frac{1}{y} \leq 0 \end{cases}   
\end{gathered}
\]
\item We know that $|a-b| \leq |a-c| + |b-c|$.  
\[
\text{ Let } c=0 \Longrightarrow |x-y| \leq |x| + |y| 
\]
\item $x = a-b, b-c = -y$.  
\[
|x| \leq |x-y| + |-y| \quad |x| - |y| \leq |x-y|
\]
\item \[
\begin{aligned}
  & ||x|-|y|| = \begin{cases} |x|-|y| & \text{ if } |x| - |y| \geq 0 \\  |y|-|x| & \text{ if } |x| - |y| \leq 0 \end{cases} \\ 
  & |x| \leq |x-y| + |-y| \Longrightarrow |x| - |y| \leq |x-y| \\
  & |y| \leq |y-x| + |-x| \Longrightarrow |y| - |x| \leq |y-x| = |x-y|  
\end{aligned}
\]
\end{enumerate}

\exercisehead{4}
\[
\begin{gathered}
  \Rightarrow \\
  \begin{aligned} & \quad \text{ If $\forall k = 1 \dots n ; \, a_k x + b_k = 0 $ }   \\
    & \left( \sum_{k=1}^n a_k (-x a_k) \right)^2 = \left( x \sum_{k=1}^n a_k^2 \right)^2 = \left( \sum_{k=1}^n a_k^2 \right)\left( \sum_{k=1}^n (-x a_k)^2 \right) = \left( \sum_{k=1}^n a_k^2 \right)\left( \sum_{k=1}^n b_k^2 \right) \\
    \end{aligned} \\
  \Leftarrow \text{ Proving $a_k x + b_k =0 $ means $ x = -\frac{b_k}{a_k}, \, a_k \neq 0 $ } \\
  (a_1 b_1 + a_2 b_2 + \dots + a_n b_n)^2 = \sum_{j=1}^n a_j^2 b_j^2 + \sum_{j \neq q}^n a_j a_k b_j b_k = = \sum_{j=1}^n a_j^2 b_j^2 + \sum_{j \neq k }^n a_j^2 b_k^2 \\
  \Longrightarrow a_j^2 b_k^2 - a_j a_k b_j b_k = a_j b_k (a_j b_k -a_k b_j) = 0 \\
  \text{ if } a_j, b_k \neq 0, \, a_j b_k -  a_k b_j = 0 \Longrightarrow a_k \left( \frac{b_j}{-a_j}\right)  + b_k  = 0   
\end{gathered}
\]

\exercisehead{8} The trick of this exercise is the following algebraic trick (``multiplication by conjugate'') and using telescoping property of products: \[
\begin{gathered}
  (1-x^{2^j} )( 1+ x^{2^j} ) = 1 - x^{2^j + 2^j} = 1 - x^{2^{j+1}} \\
 \prod_{j=1}^1 1+ x^{2^{j-1} } = \prod_{j=1}^1 \frac{1 - x^{2^j} }{ 1-x^{2^{j-1} } } = \frac{ 1-x^{2^n}}{ 1-x} \\
\text{ if } x = 1, 2^n
\end{gathered}
\]

\exercisehead{10}
\[
\begin{gathered}
  \begin{aligned}
    & x > 1  \\
    & x^2 > x \\
    & x^3 > x^2 > x 
  \end{aligned}
 \quad x^{n+1} = x^n x > x^2 > x \\
 \begin{aligned}
   &   0 <x <1 \\
   &  x^2 < x \\
   & X^3 < x^2 < x 
 \end{aligned}
 \quad x^{n+1} = x^n x < x^2 < x \Longrightarrow x^{n+1} < x 
\end{gathered}
\]

\exercisehead{11} Let $S = \{ n \in \mathbb{Z}^+ | 2^n < n! \} $.  \\
By well-ordering principle, $\exists $ smallest $n_0 \in S$.  Now \\
$\quad 2^4 = 16, \, 4! = 24$.  So $S$ starts at $n=4$.  

\exercisehead{12} \begin{enumerate} \item 
\[
\begin{aligned}
  & \left(1+\frac{1}{n}\right)^n = \sum_{j=0}^n \binom{n}{k} \left( \frac{1}{n} \right)^j = \sum_{k=0}^n \frac{ n!}{(n-k)!k!} \left( \frac{1}{n} \right)^k  \\
  & \prod_{r=0}^{k-1} \left( 1- \frac{r}{n} \right) = \prod_{r=0}^{k-1} \left( \frac{n-r}{n} \right) = \left( \frac{1}{n^k} \right) \frac{ n!}{(n-k)!} \\ 
  & \sum_{k=1}^n \frac{1}{k!} \prod_{r=0}^{k-1} \left( 1- \frac{r}{n} \right) = \left( \frac{1}{n^k} \right) \frac{ n!}{(n-k)!}
\end{aligned}
\]
\item 
\[
\begin{gathered} (1+\frac{1}{n})^n = 1 + \sum_{k=1}^n \left( \frac{1}{k!} \prod_{r=0}^{k-1} (1- \frac{r}{n} ) \right) < 1 + \sum_{k=1}^n \frac{1}{k!} < 1+ \sum_{k=1}^n \frac{1}{2^k} = 1+ \frac{ \frac{1}{2} - \left( \frac{1}{2} \right)^{n+1} }{\frac{1}{2} } = 1+ (1-\left( \frac{1}{2} \right)^n ) \\
< 3 
\end{gathered}
\]
The first inequality obtained from the fact that if $0<x<1, x^n < x < 1 $.  The second inequality came from the previous exercise, that $\frac{1}{k!} < \frac{1}{2^k}$.  
\[
(1+\frac{1}{n})^n = \sum_{k=0}^n \binom{n}{k} \left( \frac{1}{n} \right)^k = 1 + \frac{1}{n} + \sum_{k=1}^{n-1} \binom{n}{k} \left( \frac{1}{n} \right)^k = 1 + \frac{1}{n} + \sum_{k=2}^{n-1} \binom{n}{k} \left( \frac{1}{n} \right)^k + \frac{n}{1} \left( \frac{1}{n} \right) > \\
> 2 
\]
\end{enumerate}

\exercisehead{13}
\begin{enumerate}
\item 
\[
\begin{gathered}
  S = \sum_{k=0}^{p-1} \left( \frac{b}{a} \right)^k = \frac{ 1 - \left(\frac{b}{a} \right)^p }{ 1- \frac{b}{a} }  \\
  \sum_{k=0}^{p-1} b^k a^{p-1-k} = a^{p-1} \frac{ 1 - \left(\frac{b}{a} \right)^p }{ 1- \frac{b}{a} } = \frac{ b^p - a^p}{ b-a}
\end{gathered}
\]
\item 
\item Given 
\[
n^p < \frac{ (n+1)^{p+1} - n^{p+1} }{ p+1} < (n+1)^p 
\]
We want 
\[
\sum_{k=1}^{n-1} k^p < \frac{ n^{p+1}}{p+1} < \sum_{k=1}^n k^p  
\]
\[
\begin{aligned}
 n=2 & 1^p < \frac{ 2^{p+1}}{p+1} < 1^p + 2^p  \\
 & p=1 \\
 & \phantom{ p=1} 1 < 2^2/2 = 2.  2 < 1+2 =3 \\
 & p-2 \\
 & \phantom{ p=2} 1 < 8/3 < 1+4 =5 
\end{aligned}
\]
\end{enumerate}


%-----------------------------------%-----------------------------------%-----------------------------------
\subsection*{ I 4.10 Miscellaneous exercises involving induction }
%-----------------------------------%-----------------------------------%-----------------------------------

\exercisehead{13}\begin{enumerate}
\item
\item 
\item Let $n=2$.  
\[
\sum_{k=1}^{2-1} k^p = 1^p = 1, \, \frac{ n^{p+1}}{ p+1} = \frac{ 2^{p+1} }{ p+1 }\, \sum_{k=1}^2 k^p = 1+2^p
\]

What makes this exercise hard is that \textbf{ we have to use induction on $p$ itself}.  Let $p=1$.  
\[
1 < \frac{ 2^{1+1}}{ 1+2} = 2 < 1+2^1 = 3 
\]
Now assume $p$th case.  Test the $p+1$ case.  
\[
\begin{gathered}
  \frac{ 2^{p+2}}{ p+2} = \frac{ 2(p+1)}{ p+2} \left( \frac{ 2^{p+1}}{ p+1} \right) > 1 \\
  \text{ since } p+2 < 2p+2 = 2(p+1) \text{ for } p \in \mathbb{Z}^+
\end{gathered}
\]
For the right-hand inequality, we will use the fact just proven, that $2^{p} - (p) > 0 $ and $p$th case rewritten in this manner
\[
(1+2^p) > \frac{ 2^{p+1}}{ p+1} \Longrightarrow (1+2^p)(p+1) > 2^{p+1}
\]
So  
\[
\begin{aligned}
  (p+2)(1+2^{p+1}) & = (p+2) + ((p+1) + 1)2^p (2) = (p+2)+ 2(p+1) 2^p + 2^p (2) > \\
  & > (p+2) + 2 (2^{p+1} - (p+1)) + 2^p (2) = -p + 2^{p+2} + 2^{p+1} > 2^{p+2}
\end{aligned}
\]
So the $n=2$ case is true for all $p \in \mathbb{Z}^+$.  

Assume $n$th case is true.  We now prove the $n+1$ case.
\[
\begin{aligned}
  \sum_{k=1}^n k^p & = \sum_{k=1}^{n-1} k^p + n^p < \frac{n^{p+1}}{ p+1} + n^p < \frac{ n^{p+1}}{p+1} + \frac{ (n+1)^{p+1} -n^{p+1}}{ p+1} = \frac{ (n+1)^{p+1}}{ p+1} \\
  \sum_{k=1}^{n+1} k^p & = \sum_{k=1}^n k^p + (n+1)^p > \frac{ n^{p+1}}{p+1} + \frac{ (n+1)^{p+1} -n^{p+1}}{ p+1} = \frac{ (n+1)^{p+1}}{ p+1}
\end{aligned}
\]
We had used the inequality proven in part b, $ n^p < \frac{ (n+1)^{p+1} - n^{p+1} }{ p+1} < (n+1)^p$.  
\end{enumerate}

\exercisehead{14} Use induction to prove a general form of Bernoulli's inequality.  

\[
\begin{gathered}
  1+ a_1 = 1+a_1 \\
  (1+a_1)(1+a_2) = 1 + a_2 + a_1 + a_1 a_2 \geq 1 + a_1 + a+2 \\
  \quad \\ 
  \text{ Test the $n+1$ case } \\
  \begin{aligned}
    (1+ a_1) (1+a_2) \dots (1+ a_{n+1}) & \geq (1+ a_1 + a_2 + \dots + a_n ) (1+ a_{n+1} ) = \\
    & = 1 + a_1 + a_2 + \dots + a_n + a_{n+1} + a_{n+1} ( a_1 + a_2 + \dots a_n ) \geq \\
    & \geq 1 + a_1 + a_2 + \dots + a_n + a_{n+1}
    \end{aligned}
\end{gathered}
\]
Note that the last step depended upon the given fact that all the numbers were of the same sign.  

For $a_1 = a_2 = \dots = a_n = x$, then we have $(1+x)^n \geq 1+ nx$.  
\[
(1+x)^n = \sum_{j=0}^n \binom{n}{j} x^j = 1+nx 
\]
Since $x$ and $n$ are arbitrary, we can compare terms of $x^j$'s.  Then $x=0$.  

\exercisehead{15}
$\frac{2!}{2^2} = \frac{1}{2} \, \frac{ 3!}{3^3 } = \frac{2}{9} < 1$.  

So we've shown the $n=2, n=3$ cases.  Assume the $n$th case, that $\frac{n!}{n^n } \leq \left( \frac{1}{2} \right)^k $, where $k$ is the greatest integer $\leq \frac{n}{2}$.  

\[
\frac{(n+1)!}{ (n+1)^{n+1} } \geq \frac{ (n+1)n^n \left( \frac{1}{2} \right)^k }{ (n+1)^{n+1} } = \left( \frac{n}{ n+1} \right)^n \left( \frac{1}{2} \right)^k = \left( 1 - \frac{1}{n+1} \right)^n \left( \frac{1}{2} \right)^k < \frac{1}{2} \left( \frac{1}{2} \right)^k = \left( \frac{1}{2} \right)^{k+1}
\]
where in the second to the last step, we had made this important observation:
\[
k \leq \frac{ n }{2} \Longrightarrow k + \frac{1}{2} \leq \frac{ n+1}{2} \Longrightarrow \frac{1}{n+1} \leq \frac{1}{2k+1} < \frac{1}{2} 
\]

\exercisehead{16}
\[
\begin{gathered}
  a_1 = 1 < \frac{ 1+ \sqrt{5}}{2} \\
  a_2 = 2 < \left( \frac{ 1+ \sqrt{5}}{ 2} \right)^2  = \frac{ 1+ 2 \sqrt{5} + 5 }{ 4} = \frac{ 6 + 2 \sqrt{5}}{ 4}  \\
\begin{aligned}
a_{n+1} & = a_n + a_{n-1} < \left( \frac{1+ \sqrt{5}}{ 2}\right)^n + \left( \frac{ 1+ \sqrt{5} }{ 2} \right)^{n-1} = \left( \frac{1+\sqrt{5}}{2} \right)^n \left( 1 + \frac{2}{1+\sqrt{5}} \right) = \\ 
& = \left( \frac{ 1+ \sqrt{5}}{ 2} \right)^n \left( \frac{ 2 (1- \sqrt{5})}{ 1-5 } + \frac{4}{4} \right) = \left( \frac{ 1+ \sqrt{5}}{ 2 } \right)^{n+1} 
\end{aligned}
\end{gathered}
\]

\exercisehead{17} Use Cauchy-Schwarz, which says
\[
\left( \sum a_k b_k \right)^2 \leq \left( \sum a_k^2 \right) \left( \sum b_k^2 \right) 
\]
Let $a_k = x_k^p$ and $b_k=1$.  Then Cauchy-Schwarz says

\[
\left( \sum x_k^p \right)^2 \leq \left( \sum x_k^{2p} \right) n \Longrightarrow \sum (x_k^{2p}) \geq \frac{ \left( \sum x_k^p \right)^2 }{ n } 
\]

We define $M_p$ as follows:
\[
M_p = \left( \frac{ \sum_{k=1}^n x_k^p }{ n }\right)^{1/p } 
\]
So then 
\[
\begin{gathered} 
  n M_p^p = \sum_{k=1}^n x_k^p  \\
  \begin{aligned}
    M_{2p} &= \left( \frac{ \sum_{k=1}^n x_k^{2p} }{ n} \right)^{1/2p } \\
    n M_{2p}^{2p} &= \sum_{k=1}^n x_k^{2p} 
  \end{aligned}
  \sum x_k^{2p} = nM_{2p}^{2p} \geq \frac{ (n M_p^p)^2 }{ n } = n M_p^{2p} \\
  M_{2p}^{2p} \geq M_p^{2p} \Longrightarrow M_{2p } \geq M_p
\end{gathered}
\]

\exercisehead{18} 
\[
\begin{gathered}
  \left( \frac{ a^4 + b^4 + c^4}{ 3 } \right)^{1/4} \geq \left( \frac{ a^2 + b^2 + c^2 }{ 3 } \right)^{1/2} = \frac{ 2^{3/2}}{ 3^{1/2} } \text{ since } \\
  a^4 + b^4 +c^4 \geq \frac{64}{3} 
\end{gathered}
\]

\exercisehead{19} $a_k = 1, \, \sum_{k=1}^n 1 =n$

Now consider the case of when not all $a_k = 1$.  
\[
\begin{gathered}
  a_1 = 1 \\
\begin{aligned}
  &  a_1 a_2 = 1 \text{ and suppose, without loss of generality $a_1 > 1$.  Then $ 1 > a_2$.  } \\
  & (a_1 - 1 )(a_2 -1) < 0 \\
  & a_1 a_2 - a_1 -a_2 + 1 < 0 \Longrightarrow a_1 + a_2 > 2 
\end{aligned}  \\
\text{ (consider $n+1$ case ) If $a_1 a_2 \dots a_{n+1} = 1 $, then suppose $a_1>1, \, a_{n+1} <1$ without loss of generality } \\
\begin{aligned}
  b_1 & = a_1 a_{n+1}  \\
  & b_1 a_2 \dots a_n = 1 \Longrightarrow b_1 + a_2 + \dots + a_n \geq n \text{ (by the induction hypothesis) } \\
  & (a_1 -1) (a_{n+1} -1 ) = a_1 a_{n+1} - a_1 - a_{n+1} + 1 < 0, \, b_1 < a_1 + a_{n+1} - 1 \\
  & \Longrightarrow a_1 + a_{n+1} - 1 +a_2 + \dots + a_n > b_1 + a_2 + \dots + a_n \geq n \\
  & \Longrightarrow a_1 + a_2 + \dots + a_{n+1} \geq n+1 
\end{aligned}
\end{gathered}
\]


%-----------------------------------%-----------------------------------%-----------------------------------
\subsection*{ 1.7 Exercises - The concept of area as a set function }
%-----------------------------------%-----------------------------------%-----------------------------------
We will use the following axioms:

Assume a class $\mathcal{M}$ of measurable sets (i.e. sets that can be assigned an area), set function $a$, $a:\mathcal{M} \to \mathbb{R}$.  
\begin{itemize}
\item \begin{axiom}[Nonnegative property] \begin{equation}
    \forall S \in \mathcal{M}, a(S) \geq 0 \end{equation}   \end{axiom}
\item \begin{axiom}[Additive property]If $S,T \in \mathcal{M}$, then $S \cup T, S \cap T \in \mathcal{M}$ and 
  \begin{equation}
    a(S \cup T) = a(S) + a(T) - a(S \cap T)  
  \end{equation}
\end{axiom} 
\item \begin{axiom}[Difference property] If $S,T \in \mathcal{M}, S \subseteq T$ then $T-S \in \mathcal{M}$ and 
  \begin{equation}
    a(T-S) = a(T) -a(S)
  \end{equation}
\end{axiom}
\item \begin{axiom}[Invariance under congruence] 
  If $S \in \mathcal{M}, T=S$, then $T \in \mathcal{M}$, $a(T) = a(S)$
\end{axiom}
\item \begin{axiom}[Choice of scale]
$\forall$ rectangle $R \in \mathcal{M}$, if $R$ has edge lengths $h,k$ then $a(R) = hk$
\end{axiom}
\item \begin{axiom}[Exhaustion property]
  Let $Q$ such that 
  \begin{equation}\label{E:S_subset_Q_subset_T}
    S \subseteq Q \subseteq T 
  \end{equation}
If $\exists$ only one $c$ such that $a(S) \leq c \leq a(T), \forall S,T$ such that they satisfy Eqn. (\ref{E:S_subset_Q_subset_T})  \\
\phantom{If} then $Q$ measurable and $a(Q) = c$ 
\end{axiom}
\end{itemize}

\exercisehead{1} 
\begin{enumerate} 
\item We need to say that we consider a line segment or a point to be a special case of a rectangle allowing $h$ or $k$ (or both) to be zero.  

  Let $T_l = \{ \text{ line segment containing $x_0$ } \}, Q = \{ x_0 \}$.  \\
  \phantom{Let} For $Q$, only $\emptyset \subset Q$  

  By Axiom 3, let $T=S$.  
  \[
  \begin{gathered}
    a(T-S) = a(\emptyset) = a(T) - a(T) = 0 \\
    \emptyset \subset Q \subseteq T_l \Longrightarrow a(\emptyset) \leq a(Q) \leq a(T_l) \Longrightarrow 0 \leq a(Q) \leq 0 \\
    \Longrightarrow a(Q) = 0 
  \end{gathered}
  \]
\item \[
  a\left( \bigcup_{j=1}^N Q_j \right) = \sum_{j=1}^N a(Q_j)
  \]
  if $Q_j$'s disjoint.  Let $Q_j = \{ x_j \}$.  \bigskip \\
  Since $a(Q_j)=0$.  By previous part, $a\left( \bigcup_{j=1}^N Q_j \right) = 0$ 
\end{enumerate}

\exercisehead{2} Let $A,B$ be rectangles.  By Axiom 5, $A,B$ are measurable.  By Axiom 2, $A \cap B$ measurable.  
\[
a(A\cap B) = \sqrt{ a^2 + b^2 }d + ab - ( \frac{1}{2} ab + \sqrt{ a^2 + b^2 } d ) = \frac{1}{2}ab
\]

\exercisehead{3} Prove that every trapezoid and every parallelogram is measurable and derive the usual formulas for their areas.  

A trapezoid is simply a rectangle with a right triangle attached to each end of it.  $T_r = R + T_1 + T_2$.  $T_1, T_2$ are right triangles and so by the previous problem, $T_1, T_2$ are measurable.  Then $T_r$ is measurable by the Additive property axiom (note that the triangles and the rectangle don't overlap).  

We can compute the area of a trapezoid:
\[
\begin{gathered}
T_r = R + T_1 + T_2 \Longrightarrow a(T_r) = a(R) + a(T_1) + a(T_2) \\
a(T_r) = hb_1 + \frac{1}{2} h (b_2-b_1)/2 + \frac{1}{2} h (b_2 - b_1)/2 = \frac{1}{2}h (b_1 + h_2)
\end{gathered}
\]

$P=R$ (a parallelogram consists of a right triangle rotated by $\pi$ and attached to the other side of the same right triangle; the two triangles do not overlap).  Since two right triangles are measurable, the parallelogram, $P$ is measurable.  \\ 
Using the Additive Axiom, $a(P) = 2 a(T) = 2 \frac{1}{2} bh = bh$

\exercisehead{4} A point $(x,y)$ in the plane is called a lattice point if both coordinates $x$ and $y$ are integers.  Let $P$ be a polygon whose vertices are lattice points.  The area of $P$ is $I + \frac{1}{2}B - 1$, where $I$ denotes the number of lattice points inside the polygon and $B$ denotes the number on the boundary.  
\begin{enumerate}
\item Consider one side of the rectangle lying on a coordinate axis with one end on the origin.  If the rectangle side has length $l$, then $l+1$ lattice points lie on this side (you have to count one more point at the $0$ point.  Then consider the same number of lattice points on the opposite side.  We have $2(l+1)$ lattice points so far, for the \emph{boundary}.  

The other pair of sides will contribute $2(h-1)$ lattice points, the $-1$ to avoid double counting.  Thus $2(l + h)=B$.  

$I= (h-1)(l-1)$ by simply considering multiplication of $(h-1)$ rows and $(l-1)$ columns of lattice points inside the rectangle.  \bigskip \\

$I + \frac{1}{2}B - 1 = hl - h - l + 1 + (l + h ) - 1 = hl = a(R)$
\item
\item 
\end{enumerate}

\exercisehead{5} Prove that a triangle whose vertices are lattice points cannot be equilateral. \bigskip \\

My way: I will take, for granted, that we know an equilateral triangle has angles of $\pi/3$ for all its angles.  \medskip \\
Even if we place two of the vertices on lattice points, so that its length is $2L$, and put the midpoint and an intersecting perpendicular bisector on a coordinate axis (a picture would help), but the ratio of the perpendicular bisector to the third vertex to half the length of the triangle is $\cot{\pi/3} = \frac{1}{\sqrt{3}}$.  Even if we go down by an integer number $L$, $L$ steps down, we go ``out'' to the third vertex by an irrational number $\sqrt{3}L$.  Thus, the third vertex cannot lie on a lattice point.  


\exercisehead{6} Let $A = \{ 1, 2, 3, 4,5 \}$ and let $\mathcal{M}$ denote the class of all subsets of $A$.  (There are 32 altogether counting $A$ itself and the empty set $\emptyset$).  (My Note: the set of all subsets, in this case, $\mathcal{M}$, is called a \emph{power set} and is denoted $2^A$.  This is because the way to get the total number of elements of this power set, $|2^{A}|$, or the size, think of assigning to each element a ``yes,'' if it's in some subset, or ``no'', if it's not.  This is a great way of accounting for all possible subsets and we correctly get all possible subsets.) For each set $S$ in $\mathcal{M}$, let $n(S)$ denote the number of distinct elements in $S$.  If $S = \{ 1,2,3,4 \}$ and $T = \{ 3,4,5 \}$, 
\[
\begin{aligned}
  & n(S \bigcup T) = 5 \\
  & n(S \bigcap T) = 2 \\
  & n(S-T) = n(\{ 1,2 \} ) = 2 \\
  & n(T-S) = n(\{ 5 \}) = 1
\end{aligned}
\]
$n$ satisfies nonnegative property because by definition, there's no such thing as a negative number of elements.  If $S,T$ are subsets of $A$, so are $S \bigcup T$, $S \bigcap T$ since every element in $S \bigcup T$, $S \bigcap T$ is in $S$.  Thus $n$ could be assigned to it, so that it's \emph{measurable}.  Since $n$ counts only distinct elements, then $n(S \bigcup T) = n(S) + a(T) - a(S \bigcap T)$, where $-a(S \bigcap T)$ ensures there is no double counting of distinct elements.  Thus, the Additive Property Axiom is satisfied.  \bigskip \\

For $S \subseteq T$, then $\forall \, x \in T-S$, $x \in T, \, x \notin S$  Now $S \subseteq T$, so $\forall \, x \in S$, $x \in T$.  Thus $T-S$ is complementary to $S$ ``with respect to'' $T$.  $n(S) + n(T-S) = n(T)$, since $n$ counts up distinct elements.  


%-----------------------------------%-----------------------------------%-----------------------------------
\subsection*{ 1.11 Exercises - Intervals and ordinate sets, Partitions and step functions, Sum and product of step function }
%-----------------------------------%-----------------------------------%-----------------------------------

\exercisehead{4} \begin{enumerate}
\item 
\[
\begin{gathered}
\begin{aligned} & [x+n] = y \leq x + n, y\in \mathbb{Z}; y - n \leq x  \\
  & [x] + n =z + n \leq x+ n 
\end{aligned} \\
\text{ If } y-n < z, \text{ then } y < z + n \leq x + n.  \text{ then $y$ wouldn't be the greatest integer less than $x+n$ } \\
\Longrightarrow y = z + n 
\end{gathered}
\]
\item \[
\begin{gathered}
  [x] = y_2 \leq x \, -[x] = - y_2 \geq -x \, -y_2 -1 \leq x \\
  -x \geq y_1 = [-x] = -y_2 -1 = -[x] -1 ; \text{ ( and $y_1 = -y_2 - 1$ since $-y_2 >-x$ ) }  \\
  \text{ If $x$ is an integer $-[x] = [-x]$ }
\end{gathered}
\]
\item Let $x=q_1 + r_1, y=q_2 + r_2; 0 \leq r_1, r_2 < 1$.  
\[
\begin{gathered}
  [x+y] = [q_1 + q_2 + r_1 + r_2] = \begin{cases} q_1 +q_2 \\ q_1 + q_2 +1 \text{ if } r_1+r_2 \geq 1 \end{cases} \\
  [x] + [y] = q_1+q_2 \quad [x]+[y]+1 = q_1 +q_2 +1 
\end{gathered}
  \]
\item \[
\begin{aligned}
  & \text{ If $x$ is an integer }, [2x] = 2x = [x]+[x+\frac{1}{2} ] = [x]+[x] = 2x \\
  & [x] + [x+\frac{1}{2}] = q + \begin{cases} q & \text{ if } r < \frac{1}{2} \\ 2q+1 \text{ if } r > \frac{1}{2} \end{cases} \\
  & [2x] = [2(q+r)] = [2q+2r] = \begin{cases} 2q & \text{ if } r < \frac{1}{2} \\ 2q+1 & \text{ if } r > \frac{1}{2} \end{cases} 
\end{aligned}
\]
\item 
\[
\begin{aligned}
  & [x] + [x+\frac{1}{3}] + [x+\frac{2}{3} ] = q + \begin{cases} q & \text{ if } r < \frac{2}{3} \\ q+1 \text{ if } r > \frac{2}{3} \end{cases} + \begin{cases} q & \text{ if } r < \frac{1}{3} \\ q+1 & \text{ if } r > \frac{1}{3} \end{cases} = \begin{cases} 3q & \text{ if } r < \frac{1}{3} \\ 3q+1 & \text{ if } \frac{1}{3} < r < \frac{2}{3} \\ 3q+2 & \text{ if } r > \frac{2}{3} \end{cases}  \\
  & [3x] = [3(q+r)] = [3q+3r] = \begin{cases} 3q & \text{ if } r < \frac{1}{3} \\ 3q+1 & \text{ if } \frac{1}{3} < r < \frac{2}{3} \\ 3q+2 & \text{ if } r > \frac{2}{3} \end{cases} 
\end{aligned}
\] 
\end{enumerate}

\exercisehead{5} Direct proof.  \\
\[
  [nx] = [n(q+r)] = \begin{cases} nq & \text{ if } r < \frac{1}{n}  \\
    nq +1 &  \text{ if } \frac{1}{n} < r < \frac{2}{n}  \\
    nq +n-1 & \text{ if } r > \frac{n-1}{n}
  \end{cases}  
\]

\exercisehead{6}\[ 
\begin{gathered}
  a(R) = hk = I_R + \frac{1}{2} B_R - 1 \\
  \sum_{n=a}^b [f(n)] = [f(a)] + [f(a+1)] + \dots + [f(b)] \\
\end{gathered}
\]
$[f(n)] = g \leq f(n), g \in \mathbb{Z}$, so that if $f(n)$ is an integer,$ g= f(n)$, and if $f(n)$ is not an integer, $g$ is the largest integer such that $g<f(n)$, so that all lattice points included and less than $g$ are included.  

\exercisehead{7}
\begin{enumerate}
\item Consider a right triangle with lattice points as vertices.  Consider $b+1$ lattice points as the base with $b$ length.  

Start from the vertex and move across the base by increments of $1$.  

The main insight is that the slope of the hypotenuse of the right triangle is $\frac{a}{b}$ so as we move $1$ along the base, the hypotenuse (or the $y$-value, if you will) goes up by $\frac{a}{b}$.  Now
\begin{multline}
\left[ \frac{na}{b} \right] = \text{ number of interior points at $x=n$ and below the hypotenuse line of the right triangle of sides $a,b$,} \\
\text{ including points on the hypotenuse }
\end{multline}

\[
\begin{gathered}
  \sum_{n=1}^{b-1} \left[ \frac{na}{b} \right] + \frac{1}{2} ((a+1) +b ) -1 = \frac{ab}{2} \\
  \text{ Now } \frac{ (a-1)(b-1)}{2} = \frac{ab}{2} - \frac{a}{2} - \frac{b}{2} + \frac{1}{2} \\
  \Longrightarrow \sum_{n=1}^{b-1} \left[ \frac{na}{b} \right] = \frac{(a-1)(b-1)}{2}
\end{gathered}
\]
\item $a,b \in \mathbb{Z}^+$ 
\[
\begin{gathered}
  \sum_{n=1}^{b-1} \left[ \frac{na}{b} \right] = \sum_{n=1}^{b-1} \left[ \frac{ a(b-n)}{b} \right] \, \text{ (reverses order of summation) } \\ 
  \begin{aligned}
  \sum_{n=1}^{b-1} \left[ a - \frac{an}{b} \right] & = \begin{cases} - \sum_{n=1}^{b-1} \left[ \frac{an}{b} - a \right] & \text{ if $\frac{an}{b} -a 4$ is an integer (but $a \left( \frac{n}{b} - 1 \right) $ can't be!) } \\
    -  \sum_{n=1}^{b-1} \left( \left[ \frac{an}{b} - a \right] -1 \right) & \text{ otherwise } 
  \end{cases} \\
  & = -  \sum_{n=1}^{b-1} \left( \left[ \frac{an}{b} - a \right] -1 \right) = -   \sum_{n=1}^{b-1} \left( \left[ \frac{an}{b} \right] -a \right)- (b-1) = \\
  & = -\sum_{n=1}^{b-1} \left[ \frac{an}{b} \right] + a(b-1) - (b-1)
\end{aligned} \\
  \sum_{n=1}^{b-1} \left[ \frac{na}{b} \right] = \frac{(a-1)(b-1)}{2}
\end{gathered}
\]
\end{enumerate}

\exercisehead{8} Recall that for the step function $f=f(x)$, there's a partition $P = \{ x_0,x_1, \dots, x_n \}$ of $[a,b]$ such that $f(x) = c_k$ if $x \in I_k$.

Given that $\chi_s(x) = \begin{cases} 1 & \forall x \in S \\ 0 & \forall x \notin S \end{cases}$.  

If $x \in [a,b]$, then $x$ must only lie in one open subinterval $I_j$, since real numbers obey transitivity.  
\[
\begin{gathered}
  \sum_{k=1}^n c_k \chi_{I_k}(x) = c_j \text{ for } x \in I_j
  \Longrightarrow \sum_{k=1}^n c_k \chi_{I_k} (x) =f(x) \forall x \in [a,b]
\end{gathered}
\]

\subsection*{ 1.15 Exercises - The definition of the integral for step functions, Properties of the integral of a step function, Other notations for integrals }

\exercisehead{1}\begin{enumerate}
  \item $\int_1^3 [x] dx = (-1)+1 +(2) = 2 $
  \item $ \int_{-1}^3 [x+ \frac{1}{2} ] dx = \int_{-1/2}^{7/2} [x] dx = (-1)\frac{1}{2} + (1)(1) + (2)(1) + \frac{1}{2} 3 = 4 $ 
  \item $ \int_{-1}^3 ([x] + [x+\frac{1}{2} ] ) dx = 6$  
  \item $ \int_{-1}^3 2 [x] dx = 4   $
  \item $ \int_{-1}^3 [2x] dx = \frac{1}{2} \int_{-2}^6 [x] dx = \frac{1}{2} \left( (-2)1 + (-1) + (1) + 2 +3 +4 +5 \right)= 6$
  \item $\int_{-1}^3 [-x] dx = -\int_{1}^{-3} [x]dx = \int_{-3}^1 [x] dx = -3 + -2 + -1 =-6$
\end{enumerate}

\exercisehead{2} \[
s = \begin{cases} 
  5/2 & \text{ if } 0 < x < 2 \\
  -1 & \text{ if } 2 < x < 5
\end{cases} 
\]

\exercisehead{3} $[x] = y \leq x$ so $-y \geq -x$.  \medskip \\
$-y-1 \leq -x $, otherwise if $-y-1 \geq -x$, $y+1 \leq x$ and so $y$ wouldn't be the largest integer $\leq x$.  \medskip \\
$\quad \Longrightarrow [x] + [-x] = y -y -1 = -1$

Or use Exercise 4(c), pp. 64.  
\[
\int_a^b ([x] + [-x]) dx = \int_a^b [x-x] dx = \int_a^b (-1) dx = a-b
\]

\exercisehead{4} 
\begin{enumerate}
\item $ n \in \mathbb{Z}^+ , \, \int_0^n [t] dt = \sum_{t=0}^{n-1} t = \frac{ (n-1)(n-1+1)}{ 2} = \frac{ (n-1)n}{2} $
\item 
\end{enumerate}

\exercisehead{5}
\begin{enumerate}
\item $\int_0^2 [t^2] dt = \int_1^2 [t^2] dt = 1(\sqrt{2} -1) + 2 ( \sqrt{3}-\sqrt{2} ) + 3(2- \sqrt{3} ) = 5 - \sqrt{2} -\sqrt{3}$ 
\item $\int_{-3}^3 [t^2] dt = \int_0^3 [t^2] dt + \int_{-3}^0 [t^2] dt = \int_0^3 [t^2] dt + - \int_3^0 [t^2] dt = 2 \int_0^3 [t^2] dt $ 
\[
\begin{gathered}
  \begin{aligned}
    \int_2^3 [t^2] dt &= 4 (\sqrt{5} -2 ) + 5 (\sqrt{6} - \sqrt{5}) + 6 (\sqrt{7} - \sqrt{6}) + 7 (\sqrt{8} - \sqrt{7}) + 8 (3 - \sqrt{8} ) \\
    & 16 -\sqrt{5} - \sqrt{6} - \sqrt{7} - \sqrt{8} 
  \end{aligned} \\
  \int_0^2 [t^2] dt + \int_2^3 [t^2] dt = 21 - 3\sqrt{2} - \sqrt{3} - \sqrt{5} -\sqrt{6} - \sqrt{7} - \sqrt{8}  \\
  \Longrightarrow \int_{-3}^3 [t^2] dt = 42 - 2 ( 3\sqrt{2} + \sqrt{3} + \sqrt{5} + \sqrt{6} + \sqrt{7} )
\end{gathered}
\]
\end{enumerate}

\exercisehead{6} 
\begin{enumerate}
\item $\int_0^n [t]^2 dt = \int_1^n [t]^2 dt = \sum_{j=1}^{n-1} j^2 = \frac{ (n-1)n (2n-1)}{6}$
\item $\int_0^x [t]^2 dt = \sum_{j=1}^{[x-1]} j^2 + q^2 r $ where $x = q+r, \, q \in \mathbb{Z}^+, 0 \leq r <1 $. 
\[
\begin{gathered}
  \int_0^x [t]^2 dt  = \frac{ q(q-1)(2q-1) }{ 6} + q^2 r = 2 (x-1) = 2(q+r-1)  \\
  \Longrightarrow q(q-1)(2q-1) + 6 q^2 r = 12 q + 12 r -12 \\
  \Longrightarrow x =1 , x=5/2
\end{gathered}
\]
\end{enumerate}

\exercisehead{7} \begin{enumerate}
\item \[
\begin{aligned}
  & \int_0^9 [\sqrt{t} ] dt = \int_1^9 [\sqrt{t}] dt = 3(1) + 5(2) = 13 \\
  & \int_0^16 [\sqrt{t} ] dt = 3(1) + 5(2) + 7(3) = 34 = \frac{ (4)(3)(17)}{ 6} \\
  \text{ Assume } & \int_0^{n^2} [\sqrt{t} ] dt = \frac{ n(n-1)(4n+1)}{6} \\
  & \int_0^{(n+1)^2 } [\sqrt{t}] dt = \int_0^{n^2} [\sqrt{t} ] dt + \int_{n^2}^{(n+1)^2} [\sqrt{t}] dt = \frac{ n(n-1)(4n+1)}{6} + n ((n+1)^2 -n^2) = \\ 
  & \quad \quad = \frac{ (n^2 - n)(4n+1) + 6n (2n+1)}{ 6} = \frac{ 4n^3 + n^2 -4n^2 -n + 12n^2+6n}{6} = \frac{ 4n^3 +9n^2 +5n}{6} \\
  \text{ indeed }, & \\
  & \frac{ (n+1)(n)(4(n+1)+1)}{6} = \frac{ (n^2+n)(4n+5)}{ 6} = \frac{ 4n^3 +5n^2 +4n^2 +5n}{6} 
\end{aligned}
\]
\end{enumerate}

\exercisehead{8} $\int_{a+c}^{b+c} f(x) dx = \int_{a+c -c}^{b+c-c} f(x -(-c)) dx = \int_a^b f(x+c)dx $

\exercisehead{9} $ \int_{ka}^{kb} f(x) dx = \frac{1}{ \frac{1}{k} } \int_{(ka)/k}^{(kb)/k} f\left( \frac{x}{1/k} \right) dx = k \int_a^b f(kx) dx $

\exercisehead{10} Given $s(x) = (-1)^n n $ if $n \leq x < n+1; n = 0,1,2, \dots p-1; s(p)=0, \, p\in \mathbb{Z}^+$.  $f(p) = \int_0^p s(x) dx$.  \medskip \\
So for $f(3) = \int_0^3 s(x) dx$, we need to consider $n=0,1,2$.  
\[
\begin{aligned}
  & s(0 \leq x < 1 ) = 0 \\
  & s(1 \leq x < 2 ) = (-1)(1) \\
  & s(2 \leq x <3 ) = 2; \\
  & s(3 \leq x < 4) = -3 
\end{aligned}
\]
So then
\[
\begin{aligned}
  & f(3) = (-1)(1) + 2(1) =1 \\
  & f(4) = 1 + (-3)(1) = -2 \\
  & f(f(3)) = f(1) = 0
\end{aligned}
\]

We obtain this formula
\[
f(p) = \begin{cases} \frac{p}{2} (-1)^{p+1} & \text{ $p$ even } \\
  \frac{ p-1}{2} (-1)^{p+1} & \text{ $p$ even } \end{cases} \text{ since }
\]
\[
\begin{aligned}
  f(p+1) & = f(p) + \int_p^{p+1} s(x) dx =  \begin{cases} \frac{ p-1}{2} (-1)^{p+1} & \text{ $p$ even } \end{cases} + (-1)^p p \\
  & = \begin{cases} \frac{ -p}{2} & \text{ $p$ even } \\ \frac{ p-1}{2} & \text{ $p$ odd } \end{cases} + \begin{cases} p \\ -p \end{cases} = \begin{cases} \frac{p}{2} \\ \frac{ -p-1}{2} \end{cases} = \\
  & = \begin{cases} - \frac{ (p+1)}{2} & \text{ if $p+1$ even }  \\ \frac{p}{2} & \text{ if $p+1$ odd } \end{cases}
\end{aligned}
\]
Thus, $p=14, p=15$.  

\exercisehead{11} 
\begin{enumerate} 
\item 
\[
\begin{aligned}
  & \int_a^b s(x) dx = \sum_{k=1}^n s_k^3 (x_k - x_{k-1}) \\
  & \int_a^b s + \int_b^c s = \sum_{k=1}^{n_1} s_k^2 (x_k-x_{k-1}) + \sum_{k=n_1}^{n_2} s_k^3 (x_k -x_{k-1}) = \sum_{k=1}^{n_2} s_k^3 (x_k - x_{k-1}) = \int_a^c s(x) dx 
\end{aligned}
\]
\item $\int_a^b (s+t) = \sum_{k=1}^{n_3} (s+t)_k^3 (x_k - x_{k-1}) \neq \int_a^b s + \int_a^b t $
\item $ \int_a^b cs = \sum_{k=1}^n (cs)^3 (x_k -x_{k-1}) \neq c \int_a^b s$
\item Consider these facts that are true, that $x_{k-1} < x < x_k, \, s(x) = s_k; \, x_0 = a+c, \, x_n = b+c$, \\ 
  $x_{k-1} - c < x - c < x_l -c \Longrightarrow y_{k-1} < y < y_k$ so then $s(y+c) = s_k$.  
\[
\begin{aligned}
  \sum_{k=1}^n s_k^3 (x_k - x_{k-1}) & = \sum_{k=1}^k s_k^3 (x_k - c -(x_{k-1} -c)) = \\
  & = \sum_{k=1}^n s_k^3 (y_k - y_{k-1}) = \int_a^b s(y+c) dy
\end{aligned}
\]
\item $s<t$, $\int_a^b s = \sum_{k=1}^n s_k^3 (x_k -x_{k-1})$.  
\[
\begin{aligned}
\text{ if } & 0 <s , s^3 < s^2 t < st^2 < t^3   \\
\text{ if } & s< 0 t , s^3 < \text{ and } t^3 > 0 \\ 
\text{ if } & s<t< 0, \, s^3 < s^2 t , \, s(st) < t(ts) = t^2 s \, \begin{gathered} ts > t^2 \\ t^2 s < t^3 \end{gathered} \\
s^3 < s^2 t < t^2 s < t^3 
\end{aligned}
\]
Then $\int_a^b s < \int_a^b t$.  
\end{enumerate}

\exercisehead{12} \begin{enumerate}
\item $\int_a^b s + \int_b^c s = \sum_{k=1}^{n_1} s_k (x_k^2 - x_{k-1}^2 ) + \sum_{k=n_1}^{n_2} s_k (x_k^2 - x_{k-1}^2 ) = \sum_{k=1}^{n_3} s_k (x_k^2 - x_{k-1}^2 ) = \int_a^c s $
\item $\int_a^b (s+t) = \sum_{k=1}^{n_3} (s+t)_k (x_k^2 - x_{k-1}^2 ) = \sum_{k=1}^{n_3} (s_k +t_k)(x_k^2 - x_{k-1}^2 ) = \sum_{k=1}^{n_3} s_k (x_k^2 - x_{k-1}^2 ) + \sum_{k=1}^{n_3} t_k (x_k^2 - x_{k-1}^2 ) $ \medskip \\
since $P_3 = \{ x_k \} $ is a finer partition than the partition for $s, P_1, t, P_2$, then consider \medskip \\
$s_k (y_j^2 - y_{j-1}^2 ) = s_k ((x_{k+1}^2 - x_k^2 ) + (x_k^2 - x_{k-1}^2 ) ) $, so 
\[
\begin{aligned}
  \sum_{k=1}^{n_3} s_k (x_k^2 - x-{k-1}^2 ) + \sum_{k=1}^{n_3} t_k (x_k^2 - x-{k-1}^2 ) & = \sum_{j=1}^{n_1} s_j (x_j^2 - x-{j-1}^2 ) + \sum_{j=1}^{n_2} t_j (x_j^2 - x-{j-1}^2 ) = \\
  & = \int_a^b s + \int_a^b t 
\end{aligned}
\]
\item $\int_a^b cs = \sum_{k=1}^n cs_k (x_k^2 - x_{k-1}^2 ) = c \sum_{k=1}^n s_k (x_k^2 - x_{k-1}^2 )  = c \int_a^b s $
\item $\int_{a+c}^{b+c} s(x) dx = \sum_{k=1}^n s_k(x_k^2 - x_{k-1}^2) $ where
\[
\begin{aligned}
  & s(x) = s_k \text{ if } x_{k-1} < x < x_k  \\
  & x(y +c) = s_k \text{ if } x_{k-1} < y + c < x_k \Longrightarrow  x_{k-1} -c < y  < x_k -c \Longrightarrow  y_{k-1} < y < y_k
\end{aligned}
\]
where $P' = \{ y_k \}$ is a partition on $[a,b]$

\[
\begin{aligned}
  \int_a^b s(y+c) dy & = \sum_{k=1}^n s_k (y_k^2 - y_{k-1}^2 ) = \\
  & = \sum_{k=1}^n s_k ((x_k - c)^2 - (x_{k-1} -c)^2 )  =  \sum_{k=1}^n s_k ( x_k^2 - 2 x_k c + c^2 - (x_{k-1}^2 -2 x_{k-1} c +c^2 )   ) = \\
  & = \sum_{k=1}^n s_k (x_k^2 - x_{k-1}^2 - 2 c (x_k - x_{k-1})) \neq \sum_{k=1}^n s_k (x_k^2 - x_{k-1}^2 )
\end{aligned}
\]
\item Since $x_k^2 - x_{k-1}^2 > 0, \int_a^b s dx = \sum_{k=1}^n s_k (x_k^2 - x_{k-1}^2 ) < \sum_{k=1}^n t_k (x_k^2 - x_{k-1}^2 ) = \int_a^b t dx$ \\
Note that we had shown previously that the integral doesn't change under finer partition.
\end{enumerate}

\exercisehead{13} 

\[
\begin{gathered}
  \int_a^b s(x) dx \sum_{k=1}^n s_k (x_k - x_{k-1}); \int_a^b t(x) dx = \sum_{k=1}^{n_2} t_k (y_k - y_{k-1}) \\
  P = \{ x_0, x_1, \dots, x_n \}, Q = \{ y_0, y_1, \dots, y_n \} 
\end{gathered}
\]
Note that $x_0 = y_0 = a; x_n = y_{n_2} = b$.  

Consider $P \bigcup Q = R$.  $R$ consists of $n_3$ elements, (since $n_3 \leq n + n_2$ some elements of $P$ and $Q$ may be the same.  $R$ is another partition on $[a,b]$ (by partition definition) since $x_k, y_k \in \mathbb{R}$ and since real numbers obey transitivity, $\{ x_k, y_k \}$ can be arranged such that $a<z_1 < z_2 < \dots <z_{n_3 -2} < b$ where $z_k = x_k \text{ or } y_k$.  
\[
\begin{gathered}
  (s+t)(x) = s(x) + t(x) = s_j + t_k \text{ if } x_{j-1} < x < x_j; y_{j-1} < x < y_j \\
  \begin{aligned}
    & \text{ If } x_{j-1} \lessgtr y_{j-1}, \text{ let } z_{l-1} = y_{j-1}, x_{j-1} \text{ and } \\
    & \text{ If } x_j \lessgtr y_j, \text{ let } z_l = x_j, y_j 
  \end{aligned}
\end{gathered}
\]
\[
\begin{gathered}
\text{ Let } s_j = s_l; t_k = t_l \\
(s+t)(x) = s(x) +t(x) = s_l + t_l, \text{ if } z_{l-1} < x < z_l \\
\int_a^b (s(x) +t(x)) dx = \int_a^b ((s+t)(x)) dx = \sum_{l=1}^{n_3} (s_l+t)l) (z_l - z_{l-1}) = \sum_{l=1}^{n_3} s_l(z_l - z_{l-1}) + \sum_{l=1}^{n_3} t_l (z_l -z_{l-1})
\end{gathered}
\]

In general, it was shown (Apostol I, pp. 66) that any finer partition doesn't change the integral $R$ is a finer partition.  So
\[
\sum_{l=1}^{n_3} s_l (z_l-z_{l-1}) + \sum_{l=1}^n t_l (z_l-z_{l-1}) = \sum_{k=1}^n s_k (x_k -x_{k-1}) + \sum_{k=1}^{n_2} t_k (y_k - y_{k-1}) = \int_a^b s(x) dx + \int_a^b t(x) dx
\]

%
%Use $\sum_{k=1}^n (a_k +b_k) = \sum_{k=1}^n a_k + \sum_{k=1}^n b_k$.  
%\[
%\int_a^b s(x) dx = \sum_{k=1}^n s_k (x_k-x_{k-1}); \, \int_a^b t(x) dx = \sum_{k=1}^{n_2} t_k (y_k-y_{k-1})
%\]
%$P=\{ x_0, x_1, x_2, \dots, x_n \}, \, Q = \{ y_0, y_1, \dots, y_{n_2} \}$ \\
%Note that $x_0 = y_0 =a, x_n = y_{n_2} = b$.
%
%Since $x_k, y_k \in \mathbb{R}$ and since real numbers obey transitivity, consider $P \bigcup Q =R$.  \\
%\quad $R$ is another partition on $[a.b]$ (by partition definition).  
%
%\[
%\begin{aligned}
%\int_a^b [s(x) +t(x)] dx & = \int_a^b ((s+t)(x) ) dx = \sum_{k=1}^{n_3} (s_k+t_k) (z_k -z_{k-1}) =  \\
% & = \sum_{k=1}^{n_3} s_k (z_k - z_{k-1}) + \sum_{k=1}^{n_3} t_k (z_k - z_{k-1})
%\end{aligned}
%\]
%In general, it was shown (Apostol I, pp.66) that any finer partition doesn't change the integral.  $R$ is a finer partition.  So
%\[
%\int_a^b (s(x) + t(x)) dx = \int_a^b s(x) dx + \int_a^b t(x) dx
%\]

\exercisehead{14} Prove Theorem 1.4 (the linearity property).  

\[
\begin{aligned}
  c_1 \int_a^b s(x) dx + c_2 \int_a^b t(x) dx & = c_1 \sum_{k=1}^n s_k (x_k - x_{k-1}) + c_2 \sum_{k=1}^{n_2} t_k (x_k -x_{k-1}) = \\
  & = \sum_{l=1}^{n_3} c_1 s_l (z_l - z_{l-1}) + \sum_{l=1}^{n_3} c_2 t_l (z_l - z_{l-1}) = \sum_{l=1}^{n_3} (c_1 s_l + c_2 t_l) (z_l - z_{l-1}) = \\
  & = \int_a^b (c_1 s + c_2t) (x) dx 
\end{aligned}
\]

We relied on the fact that we could define a finer partition from two partitions of the same interval.  

\exercisehead{15} Prove Theorem 1.5 (the comparison theorem).  

\[
\begin{gathered}
  s(x) < t(x) \, \forall x \in [a,b]; \, s(x) (z_l -z_{l-1}) < t(x) (z_l - z_{l-1}) \, (z_l - z_{l-1} > 0 ) \\
  \begin{aligned}
    \int_a^b s(x) dx & = \sum_{k=1}^n s_k (x_k -x_{k-1}) = \sum_{l=1}^{n_3} s_l (z_l-z_{l-1} ) < \sum_{l=1}^{n_3} t_l (z_l -z_{l-1}) = \sum_{k=1}^{n2} t_k (y_l-y_{k-1}) = \\
    & = \int_a^b t(x) dx 
  \end{aligned} \\
  \Longrightarrow \int_a^b s(x) dx <  \int_a^b t(x) dx 
\end{gathered}
\] 

\exercisehead{16} Prove Theorem 1.6 (additivity with respect to the interval).  

Use the hint: $P_1$ is a partition of $[a,c]$, $P_2$ is a partition of $[c,b]$, then the points of $P_1$ along with those of $P_2$ form a partition of $[a,b]$.
\[
\int_a^c s(x) dx + \int_a^b s(x) dx = \sum_{k=1}^{n_1} s_l(x_k - x_{k-1}) + \sum_{k=1}^{n_2} s_k (x_k -x_{k-1}) = \sum_{k=1}^{n_3} s_k (x_k - x_{k-1}) = \int_a^b s(x) dx
\]

\exercisehead{17} Prove Theorem 1.7 (invariance under translation).  

\[
\begin{gathered}
  P' = \{ y_0, y_1, \dots, y_n \}; \, y_k = x_k +c; \\
  \begin{aligned}
    \Longrightarrow & x_{k-1} + c < y < x_k +c \\
    & x_{k-1} < y -c < x_k 
  \end{aligned} \\
y_k - y_{k-1} = x_k +c - (x_{k-1} +c ) = x_k -x_{k-1}
\end{gathered}
\]

\[
s(y-c) = s_k \text{ if $ x_{k-1} < y-c < x_k$ }, \, k =1, 2, \dots n
\]
\[
\int_a^b s(x) dx = \sum_{k=1}^n s_k (x_k -x_{k-1}) = \sum_{k=1}^n s_k (y_k =y_{k-1}) = \int_{y_0}^{y_n} s(y-c) dy = \int_{a+c}^{b+c} s(x-c) dx 
\]


\subsection*{ 1.26 Exercises - The integral of more general functions, Upper and lower integrals, The area of an ordinate set expressed as an integral, Informal remarks on the theory and technique of integration, Monotonic and piecewise monotonic functions.  Definitions and examples, Integrability of bounded monotonic functions, Calculation of the integral of a bounded monotonic function, Calculation of the integral $\int_0^b x^p dx$ when $p$ is a positive integer, The basic properties of the integral, Integration of polynomials }

\quad \\

\exercisehead{16} $\int_0^2 | (x-1)(3x-1)| dx = $ \\
\[
\begin{aligned}
  & \int_1^2 (x-1)(3x-1) dx = \int_1^2 (3x^2 - 4x +1) dx = \left. (x^3 -2x^2 + x ) \right|_1^2 = 2 \\
  & \int_{1/3}^1 (1-x) (3x-1) dx = - \left. (x^3 - 2x^2 +x) \right|_{1/3}^1 = \frac{4}{27} \\
  & \int_0^{1/3} (x-1)(3x-1) dx = \frac{4}{27}
\end{aligned}
\]
So the final answer for the integral is $62/27$.  

\exercisehead{17} $\int_0^3 (2x-5)^3 dx = 8 \int_0^3 (x-\frac{5}{2} )^3 dx = 8 \int_{-5/2}^{3-5/2} x^3 dx = 8 \left. \frac{1}{4} x^4 \right|_{-5/2}^{1/2} = \frac{39}{2}$

\exercisehead{18} $\int_{-3}^3 (x^2 -3)^3 dx = \int_0^3 (x^2-3)^3 + \int_{-3}^x (x^2-3)^3 = \int_0^3 (x^2 -3)^2 + -\int_0^3 (x^2 -3)^3 = 0$

\subsection*{ 2.4 Exercises - Introduction, The area of a region between two graphs expressed as an integral, Worked examples }

\exercisehead{15} $f =x^2, g=cx^3, c>0$ \medskip \\
For $0 < x < \frac{1}{c}$, $cx < 1$ (since $c>0$).  So $cx^3 < x^2$ (since $x^2>0$).  
\[
\begin{aligned}
  \int f -g &= \int x^2 -cx^3 = \left. \left( \frac{1}{3}x^3 - \frac{c}{4} x^4 \right) \right|_0^{1/c} = \frac{1}{12c^3} \\
  \int f -g & = \frac{2}{3} = \frac{1}{12c^3} ; \, \boxed{ c = \frac{1}{2\sqrt{2}}}
\end{aligned}
\]

\exercisehead{16} $f = x(1-x), g = ax$.  
\[
\int f - g = \int_0^{1-a} x -x^2 - ax = \left. \left( (1-a)\frac{1}{2} x^2 - \frac{1}{3} x^3 \right) \right|_0^{1-a} = (1-a)^3 \frac{1}{6} = 9/2 \Longrightarrow \boxed{ a = -2 }
\]

\exercisehead{17} $\pi = 2 \int_{-1}^1 \sqrt{1-x^2 } dx $
\begin{enumerate}
\item 
\[
\int_{-3}^3 \sqrt{ 9-x^2 } dx = 3 \int_{-3}^3 \sqrt{ 1 - \left( \frac{x}{3} \right)^2 } = 3(3) \int_{-1}^1 \sqrt{ 1 -x^2} = \boxed{ \frac{ 9 \pi }{ 2 } }
\]

Now 
\[
\int_{ka}^{kb} f\left( \frac{x}{k} \right) dx = k \int_a^b f dx
\]
\item 
\[
\int_0^2 \sqrt{ 1 - \frac{1}{4} x^2 } dx = 2 \int_0^1 \sqrt{ 1 - x^2} dx = \frac{ 2 \pi}{4} = \boxed{ \frac{\pi}{2} }
\]
\item $\int_{-2}^2 (x-3)\sqrt{ 4 -x^2 } dx $
\[
\begin{aligned}
  & \int_{-2}^2 x \sqrt{4-x^2 } dx = (-1)\int_2^{-2} -x \sqrt{ 4 -x^2 } \Longrightarrow 2 \int_{-2}^2 x \sqrt{4-x^2} = 0 \\
  & -3 \int_{-2}^2 2 \sqrt{1- \left( \frac{x}{2} \right)^2 } dx = (-6)(2) \int_{-1}^1 \sqrt{ 1-x^2} = \boxed{ -6\pi }
\end{aligned}
\]
\end{enumerate}

\exercisehead{18} Consider a circle of radius 1 and a twelve-sided dodecagon inscribed in it.  Divide the dodecagon by isosceles triangle pie slices.  The interior angle that is the vertex angle of these triangles is $360/12=30$ degrees.  \medskip \\
Then the length of the bottom side of each triangle is given by the law of cosines:
\[
c^2 = 1 + 1 - 2(1)(1) \cos{30^{\circ}} = 2 \left( 1- \frac{ \sqrt{3}}{2} \right) \Longrightarrow c = \sqrt{2} \sqrt{ 1 -\frac{ \sqrt{3}}{2} }
\]
The height is given also by the law of cosines
\[
h = 1 \cos{15^{\circ}} = \sqrt{ \frac{ 1+ \cos{30^{\circ}}}{ 2 } } = \sqrt{ \frac{ 1 + \frac{\sqrt{3}}{2} }{2} }
\]

The area of the dodecagon is given by adding up twelve of those isosceles triangles
\[
(12) \frac{1}{2} \left( \sqrt{ 1 + \frac{ \sqrt{3}}{2} } \left( \frac{1}{ \sqrt{2}} \right) \right) \left(  \sqrt{2} \sqrt{ 1 -\frac{ \sqrt{3}}{2} } \right) = 3
\]
So $3 < \pi$.  \medskip \\
Now consider a dodecagon that's circumscribing the circle of radius 1.  
\[
(12)\frac{1}{2} \left( 2 \sqrt{ \frac{ 1 - \frac{ \sqrt{3}}{2} }{ 1 + \frac{ \sqrt{3}}{2} } } \right) (1) = \boxed{ 12\left( 2- \frac{ \sqrt{3}}{ 2} \right) } > \pi
\]

\exercisehead{19}
\begin{enumerate}
\item $(x,y) \in E \text{ if } x =ax_1, y=by_1$ such that $x_1^2 + y_1^2 \leq 1$ \medskip \\
  $\Longrightarrow \left( \frac{x}{a} \right)^2 + \left( \frac{y}{b} \right)^2 =1$
\item \[
\begin{gathered}
  y = b\sqrt{ 1 - \left( \frac{x}{a} \right)^2 } \\
  2 \int_{-a}^a b \sqrt{ 1 - \left( \frac{x}{a} \right)^2 } = 2ba \int_{-1}^1 \sqrt{1-x^2 } = ba \frac{\pi}{2} (2) = \pi ba
\end{gathered}
\] 
\end{enumerate}

\exercisehead{20} Let $f$ be nonnegative and integrable on $[a,b]$ and let $S$ be its ordinate set.  \medskip \\
Suppose $x$ and $y$ coordinates of $S$ were expanded in different ways $x = k_1 x_1, y = k_2 y_1$.  

If $f(x_1) = y_1, g(x) = k_2 f\left( \frac{x}{k_1} \right) = k_2 y_1 = y $.  \medskip \\
\quad \quad integrating $g$ on $[k_1 a, k_1 b]$, 
\[
\int_{k_1 a}^{k_1 b} g(x) dx = \int_{k_1 a}^{k_1 b} k_2 f\left( \frac{x}{k_1} \right) dx = k_2 k_1 \int_a^b f(x) dx = k_2 k_1 A
\]

\subsection*{ 2.8 Exercises - The trigonometric functions, Integration formulas for the sine and cosine, A geometric description of the sine and cosine functions } 

\exercisehead{1} 
\begin{enumerate}
\item $\sin{\pi} = \sin{0} =0$.  sine is periodic by $2\pi$, so by induction, $\sin{n\pi} =0$.  
\[
\begin{aligned}
  & \sin{ 2(n+1)\pi} = \sin{ 2 \pi n + 2\pi } = \sin{ 2\pi n } = 0  \\
  & \sin{ (2(n+1) + 1) \pi} = \sin{ (2n+3)\pi } = \sin{ ((2n+1)\pi + 2\pi ) } = \sin{ (2n+1)\pi } = 0 
\end{aligned}
\]
\item $\cos{\pi/2} = \cos{-\pi/2} =0$
\[
\begin{gathered}
  \text{ by induction, } \cos{ \pi/2 + 2\pi j} = \cos{ \pi/2 (1+4j ) } \\
  \cos{ -\pi/2 + 2\pi j } = \cos{ (4j-1)\pi/2}, j \in \mathbb{Z}^+
\end{gathered}
\]
\end{enumerate}

\exercisehead{2} \begin{enumerate}
\item $\sin{\pi/2}=1, \sin{\pi/2(1+4j) } = 1, j \in \mathbb{Z}^+$.  
\item $\cos{x} = 1, \cos{0} = 1, \cos{2\pi j} = 1$
\end{enumerate}

\exercisehead{3} \[
\begin{aligned}
  & \sin{ x+\pi} = -\sin{x + \pi/2 + \pi/2} = \cos{ x+ \pi/2} = -\sin{x} \\
  & \cos{x+\pi} = \cos{ x + \pi/2 + \pi/2} = -\sin{ x + \pi/2} = -\cos{x}
\end{aligned}
\]

\exercisehead{4} 
\begin{gather*}
\sin{3x} = \sin{2x}\cos{x} + \sin{x} \cos{2x} = 2 \sin{x} \cos^2{x} + \sin{x} (\cos^2{x} - \sin^2{x} ) = 3 \cos^2{x} \sin{x} - \sin^3{x} = \\
= 3(1- \sin^2{x})\sin{x} -\sin^3{x} = 3 \sin{x} -4 \sin^3{x}
\end{gather*}

\[
\begin{aligned}
  \cos{3x} & = \cos{2x}\cos{x} - \sin{2x}\sin{x} = (\cos^2{x} - \sin^2{x}) \cos{x} - (2\sin{x} \cos{x})\sin{x} = \cos{x} -  4 \sin^2{x} \cos{x}  \\
  \cos{3x} & = -3 \cos{x} + 4\cos^3{x}
\end{aligned}
\]

\exercisehead{5} 
\begin{enumerate} 
\item This is the most direct solution.  Using results from Exercise 4 (and it really helps to choose the cosine relationship, not the sine relationship), 
\[
\begin{gathered}
  \cos{3x} = 4 \cos^3{x} - 3 \cos{x} \\
  \begin{aligned} 
    x = \pi/6 & \\
    & \cos{ 3 \pi/6} = 0 = 4 \cos^3{ \pi/6} - 3 \cos{ \pi/6} = \cos{\pi/6} (4 \cos^2{ \pi/6} - 3 ) = 0 \\
    & \Longrightarrow \cos{\pi/6} = \sqrt{3}/2, \sin{\pi/6} = 1/2 (\text{ by Pythagorean theorem })
\end{aligned}
\end{gathered}
\]
\item $\sin{2\pi/6} = 2 \cos{\pi/6}\sin{\pi/6} = \sqrt{3}{2}$, $\cos{\pi/3} = 1/2$ (by Pythagorean theorem)
\item $cos{2\pi/4} = 0 = 2 \cos{\pi/4} - 1, \cos{\pi/4} = 1/\sqrt{2} = \sin{\pi/4}$ 
\end{enumerate}

Note that the most general way to solve a cubic is to use this formula.  For $x^3 + bx^2 +cx +d =0$,
\[
\begin{gathered}
  \begin{aligned}
    R & = \frac{ 9bc - 27d -2b^3}{ 54 } \\
    Q & = \frac{ 3c -b^2 }{9} 
  \end{aligned}
  \quad 
  \begin{aligned}
    & S = (R + \sqrt{ Q^3 + R^2 })^{1/3} \\
    & T = (R - \sqrt{ Q^3 + R^2 })^{1/3}
  \end{aligned} \\
  \begin{aligned}
    & x_1 = S+T -b/3 \\
    & x_2 = -1/2 (S+T) - b/3 + 1/2 \sqrt{-3} (S-T)  \\
    & x_3 = -1/2 (S+T) -b/3 - 1/2 \sqrt{-3} (S-T)
  \end{aligned}
\end{gathered}
\]

\exercisehead{6} 
\[
\tan{x-y} = \frac{ \sin{x-y}}{\cos{x-y}} = \frac{ \sin{x} \cos{y} - \sin{y} \cos{x} }{ \cos{x} \cos{y} + \sin{x} \sin{y} } \left( \frac{ \frac{ 1}{ \cos{x} \cos{y} } }{  \frac{ 1}{ \cos{x} \cos{y} } } \right) = \frac{ \tan{x} -\tan{y}}{ 1 + \tan{x} \tan{y} } 
\]
if $\tan{x}\tan{y} \neq -1 $

Similarly,
\[
\begin{aligned}
& \tan{x+y} = \frac{ \sin{x+y}}{\cos{x+y}} = \frac{ \sin{x}\cos{y} + \sin{y} \cos{x} }{ \cos{x} \cos{y} - \sin{x} \sin{y} } = \frac{ \tan{x} + \tan{y}}{ 1 - \tan{x} \tan{y} }, \, \tan{x} \tan{y} \neq 1 \\
& \cot{x+y} = \frac{ \cos{x+y}}{ \sin{x+y}} = \frac{ \cos{x} \cos{y} -\sin{x} \sin{y}}{ \sin{x} \cos{y}+ \sin{y}\cos{x}} = \frac{ \cot{x}\cot{y} - 1 }{ \cot{y} + \cot{x} }
\end{aligned}
\]

\exercisehead{7} 
$3 \sin{x + \pi/3} = A \sin{x} + B\cos{x} = 3 (\sin{x} \frac{1}{2} + \frac{ \sqrt{3}}{2} \cos{x} ) = \frac{3}{2} \sin{x} + \frac{ 3 \sqrt{3}}{2} \cos{x} $

\exercisehead{8} 
\[
\begin{gathered}
  C\sin{x+\alpha} = C(\sin{x}\cos{\alpha} + \cos{x} \sin{\alpha} ) = C\cos{\alpha} \sin{x} + C \sin{\alpha} \cos{x} \\
  A = C\cos{\alpha}, B = C \sin{\alpha} 
\end{gathered}
\]

\exercisehead{9} If $A=0$, $B \cos{x} = B \sin{ \pi/2 + x } = C\sin{ x + \alpha}$ so $C = B, \alpha = \pi/2$ if $A=0$.  

If $A\neq 0$, 
\[
\begin{aligned}
  A \sin{x} + B \cos{x} & = A (\sin{x} + \frac{B}{A} \cos{x} ) = = A (\sin{x} + \tan{\alpha}\cos{x} ) \\
  & = \frac{A}{\cos{\alpha}} ( \cos{\alpha} \sin{x} + \sin{\alpha} \cos{x} ) = \frac{A}{\cos{\alpha}} (\sin{ x+\alpha} )
\end{aligned}
\]
where $-\pi/2 < \alpha <\pi/4$, $B/A = \tan{\alpha}$, $C = \frac{A}{ \cos{\alpha}}$

\exercisehead{10} $C\sin{x+\alpha} = C\sin{x} \cos{\alpha} + C\cos{x} \sin{\alpha}$.
\[
C\cos{\alpha} = -2, C\sin{\alpha} = -2, C = -2\sqrt{2}, \alpha = \pi/4
\]

\exercisehead{11} If $A=0$, $C=B, \alpha =0$.  If $B=0$, $A = -C, \alpha = \pi/2$.  Otherwise,
\[
A\sin{x} + B\cos{x} = B ( \cos{x} + \frac{A}{B} \sin{x} ) = \frac{B}{ \cos{\beta}} ( \cos{x}\cos{\beta} + \sin{\beta}\sin{x} ) = C \cos{x + \alpha}
\]
where $\frac{A}{B} = \tan{\beta}$, $\alpha= -\beta$, $C = \frac{B}{ \cos{\beta}}$.

\exercisehead{12} 
\[
\sin{x} = \cos{x} = \sqrt{ 1 -\cos^2{x} } \Longrightarrow \cos{x} = 1/\sqrt{2} \Longrightarrow x = \frac{\pi}{4}
\]
Try $5\pi/4$.  $\sin{ 5\pi/4} = \cos{3\pi/4} = -\sin{\pi/4} = -1/\sqrt{2}$.  \\
$\cos{ 5\pi/4} = -\sin{ 3\pi/4} = -\cos{\pi/4} = -1/\sqrt{2}$.  So $\sin{5\pi/4} = \cos{5\pi/4}$.  $x = 5\pi/4$ must be the other root.  

So $\theta = \pi/4 + \pi n$ (by periodicity of sine and cosine).  

\exercisehead{13}
\[
\begin{gathered}
  \sin{x} - \cos{x} = 1 = \sqrt{1- \cos^2{x} } = 1+ \cos{x}  \\
  \Longrightarrow 1 - \cos^2{x} = 1 + 2 \cos{x} + \cos^2{x} \Longrightarrow 0 = 2 \cos{x} (1+\cos{x})  \\
  \cos{x} = -1, x = \pi/2 + 2 \pi n
\end{gathered}
\]

\exercisehead{14} 
\[
\begin{aligned}
  & \cos{x-y} + \cos{x+y} = \cos{x} \cos{y} + \sin{x}\sin{y} + \cos{x} \cos{y} - \sin{x} \sin{y} = 2 \cos{x} \cos{y}  \\
  & \cos{x-y} - \cos{x+y} = \sin{x} \cos{y} - \sin{y} \cos{x} + \sin{x} \cos{y} + \sin{y} \cos{x} = 2 \sin{x} \cos{y} \\
  & \sin{x-y} + \sin{x+y} = \sin{x} \cos{y} -\sin{y}\cos{x} + \sin{x} \cos{y} + \sin{y} \cos{x} = 2 \sin{x} \cos{y} 
\end{aligned}
\]

\exercisehead{15} 
\[
\begin{aligned}
  \frac{ \sin{x+h} - \sin{x}}{ h } & = \frac{ \sin{ (x + h/2)} \cos{ h/2 } + \cos{ (x + h/2)} \sin{ h/2 } - \sin{(x+h)} \cos{ h/2 } - \cos{x + h/2 } \sin{ h/2} }{ h }  \\
  & = \frac{ \sin{ h/2}}{ h/2} \cos{ (x + h/2) }
\end{aligned}
\] 

\[
\begin{aligned}
  \frac{ \cos{ x + h } - \cos{ x } }{ h } & = \frac{ \cos{ (x+h/2)} \cos{ h/2} - \sin{ (x + h/2)} \sin{h/2} - (\cos{ (x+h/2)}\cos{h/2} + \sin{(x+h/2)}\sin{h/2}) }{ h } \\
  & = -\frac{ \sin{h/2}}{ h/2} \sin{ (x+h/2)}
\end{aligned}
\]

\exercisehead{16} 
\begin{enumerate}
\item \[
\begin{gathered}
  \sin{2x} = 2 \sin{x} \cos{x} \\
  \text{ if } \sin{2x} = 2 \sin{x} \text{ and } x \neq 0, x \neq \pi n, \cos{x} = 1 \text{ but } x \neq \pi n \Longrightarrow \boxed{ x = 2 \pi n }
\end{gathered}
\]
\item $\cos{x+y} = \cos{x} \cos{y} - \sin{x} \sin{y} = \cos{x} + \cos{y}$.  
\[
\begin{gathered}
  \cos{x} \cos{y} - \cos{x} - \cos{y} = \sin{y} \sqrt{1-\cos^2{x} }  \\
  \text{ Letting } A = \cos{x}, B = \cos{y}, \\
  A^2 B^2 +A^2 +B^2 -2A^2 B - 2AB^2 + 2AB = 1 - A^2 - B^2 + A^2 B^2 \\
  A^2 + B^2 -A^2 B - AB^2 + AB = 1/2 \\
  B^2 (1-A) + B(A-A^2) +A^2 -1/2 = 0 \\
  \begin{aligned}
  B & = \frac{ A(1-A) \pm \sqrt{A^2(1-A)^2 - 4 (1-A)(A^2 - 1/2) } }{ 1-A} = A \pm \frac{ 1}{ \sqrt{1-A}} (A^2 (1-A) -4 (A^2 -1/2) )^{1/2} = \\ 
  & = A \pm \frac{1}{\sqrt{1-A}} ( -3A^2 -A^3 +2 )^{1/2}
\end{aligned}
\end{gathered}
\]
Note that $-1 \leq B \leq 1$, but for $|A| \leq 1$.

Solve for the roots of $-3A^2-A^3 +2$, $A_0 = -1, -1 +\sqrt{3}, -1 -\sqrt{3}$.  So suppose $\cos{x} = 9/10$.  Then there is no real number for $y$ such that $\cos{y}$ would be real and satisfy the above equation.

\item $ \sin{x+y} = \sin{x} \cos{y} + \sin{y} \cos{x} = \sin{x} + \sin{y}$ 
\[
\Longrightarrow \sin{y}(1- \cos{x}) + \sin{y} + -\cos{x}\sin{y} =0, \Longrightarrow y=2\pi n
\]
Checking our result, we find that $\sin{(2\pi n + y)} = \sin{2\pi n} + \sin{y} (1)$
\item \[
\begin{gathered}
 \int_0^y \sin{x} dx = \left. -\cos{x} \right|_0^y  = - (\cos{y} -1) = 1 - \cos{y} = \sin{y}  \\
 \Longrightarrow 1 -\cos{y} = \sqrt{ 1 - \cos^2{y} }  \\
 1 - 2 \cos{y} + \cos^2{y} = 1 - \cos^2{y} \Longrightarrow  \cos{y} ( \cos{y}-1) = 0 ; y = \frac{ 2(j+1)\pi}{2}, 2\pi n
\end{gathered}
\]
\end{enumerate}

\exercisehead{17} $\int_a^b \sin{x} dx = \left.  -\cos{x} \right|_a^b = -\cos{b} + \cos{a}$ 
\begin{enumerate}
\item $ - \frac{ \sqrt{3}}{2} + 1 $ 
\item $ - \frac{ \sqrt{2}}{2} + 1 $
\item $ \frac{1}{2} $
\item $1$
\item $2$
\item $0$ We were integrating over one period, over one positive semicircle and over one negative semicircle.
\item $0$ We had integrated over two equal parts, though it only shaded in up to $x=1$.  
\item $ -\frac{\sqrt{2}}{2} + \frac{ \sqrt{3}}{2} $
\end{enumerate}

\exercisehead{18} $\int_0^{\pi} (x + \sin{x}) dx = \left. (\frac{1}{2} x^2 - \cos{x} ) \right|_0^{\pi} = \frac{\pi^2}{2} - (-1-1) = \frac{ \pi^2}{2} + 2   $

\exercisehead{19} $\int_0^{\pi/2} (x^2 + \cos{x} )dx = \left. (\frac{1}{3} x^3 + \sin{x} ) \right|_0^{\pi/2} = \frac{1}{3} (\pi/2)^3 + 1 $

\exercisehead{20} $ \int_0^{\pi/2} (\sin{x} - \cos{x} )dx = \left. (-\cos{x} - \sin{x} ) \right|_0^{\pi/2} = -1 - (-1) = 0 $

\exercisehead{21} $\int_0^{\pi/2} |\sin{x} - \cos{x} | dx = (\text{ by symmetry } ) 2 \int_0^{\pi/4} (\cos{x} - \sin{x} )dx = \left. 2 (\sin{x} + \cos{x}) \right|_0^{\pi/4} = 2 (\sqrt{2}-1)$

\exercisehead{22} $\int_0^{\pi} (\frac{1}{2} + \cos{t})dt = \left. (\frac{1}{2}t + \sin{t} ) \right|_0^{\pi} = \frac{\pi}{2}$

\exercisehead{23} 
\[
\begin{aligned}
  \int_0^{2\pi/3} (\frac{1}{2} + \cos{t})dt +   \int_{2\pi/3}^{\pi} -(\frac{1}{2} + \cos{t})dt & = \left. (\frac{t}{2} + \sin{t}) \right|_0^{2 \pi/3} +  \left. (\frac{t}{2} + \sin{t}) \right|_{\pi}^{2\pi/3} \\
  & = 2 ( \frac{ \pi}{3} + \frac{ \sqrt{3}}{2} ) - \frac{\pi}{2} = \frac{\pi}{6} + \sqrt{3}
\end{aligned}
\]

\exercisehead{24}
If $-\pi < x \leq -\frac{2 \pi}{3} $, 
\[
\int_{-\pi}^x -(\frac{1}{2} + \cos{t} )dt = \int_x^{-\pi} (\frac{1}{2} + \cos{t} )dt = \left. \left( \frac{t}{2}  + \sin{t} \right) \right|_x^{-\pi} = - \frac{\pi}{2} - \frac{ x}{2} - \sin{x}
\]
If $-2\pi/3 \leq x \leq 2\pi/3$, 
\[
\begin{aligned}
  \int_{-\pi}^{-2\pi/3 } -(\frac{1}{2} + \cos{t}) dt + \int_{-2\pi/3}^x (\frac{1}{2} + \cos{t} )dt & = \frac{ -\pi/6} + \sqrt{3}/2 + \left. (t/2 + \sin{t} ) \right|_{-2\pi/3}^x      \\
  & = x/2 + \sin{x} -\pi/3 -\sqrt{3}/2 +\sqrt{3}/2 - \pi/6 = \frac{x}{2} + \sin{x} - \pi/3
\end{aligned}
\]
If $2\pi/3 \leq x \leq \pi$, 
\[
\sqrt{3}/2 + \int_{2\pi/3}^x -(1/2 + \cos{t})dt = \sqrt{3}/2 + \left. (t/2 + \sin{t}) \right|_x^{2\pi/3} = \pi/3 +\sqrt{3} -x/2 -\sin{x}
\]


\exercisehead{25} $\int_x^{x^2} (t^2 + \sin{t})dt = (\frac{1}{3} t^3 + -\cos{t} ) = \frac{x^6 -x^3}{3} + \cos{x} - \cos{x^2}$

\exercisehead{26} $ \int_0^{\pi/2} \sin{2x} dx = \left. \left( \frac{ - \cos{(2x)}}{2} \right) \right|_0^{\pi/2} = (-1/2)(-1-1)=1$

\exercisehead{27} $ \int_0^{\pi/3} \cos{x/2}dx = \left. 2 \sin{x/2} \right|_0^{\pi/3} = 2 \frac{1}{2} =1$

\exercisehead{28} \[
\begin{gathered}
\begin{aligned}
 \int_0^x \cos{ (a+bt)} dt   & = \int_0^x (\cos{a}\cos{bt} - \sin{a} \sin{bt} )dt = \left. \left( \frac{\cos{a}}{b} \sin{bt} -\sin{a} (-\cos{bt}/b ) \right) \right|_0^x = \\
  & = \frac{ \cos{a}}{b} \sin{bx} + \frac{ \sin{a}}{ b } (\cos{bx} -1 ) = \frac{1}{b} \sin{a +bx} - \sin{a}/b  
\end{aligned} \\
\begin{aligned}
  \int_0^x \sin{(a+bt)}dt & = \int_0^x (\sin{a}\cos{bt} + \sin{bt} \cos{a} ) dt = \left. \left( \frac{\sin{a}}{b} \sin{bt} - \frac{ \cos{a}}{b} \cos{bt} \right) \right|_0^x = \\
  & = \frac{1}{b} (\cos{bx+a} + \cos{a} )
\end{aligned}
\end{gathered}
\]

\exercisehead{29} \begin{enumerate}
\item \[
\begin{aligned}
\int_0^x \sin^3{t}dt & = \int_0^x \frac{ 3 \sin{t} -\sin{3t}}{4} dt  = \left. \left( -\frac{3}{4} \cos{t} + \cos{3t}/12 \right) \right|_0^x = -3/4 (\cos{x} -1) + \frac{ \cos{3x} -1}{12} = \\
& = \frac{1}{3} -\frac{3}{4} \cos{x} + \frac{1}{12} ( \cos{2x} \cos{x} - \sin{2x} \sin{x} ) = 2/3 - 1/3 \cos{x} (2 + \sin^2{x} ) 
\end{aligned}
\]
\item 
\[
\begin{aligned}
  \int_0^x \cos^3{t}dt & = \int_0^x \frac{1}{4} (\cos{3t} + 3 \cos{t} )dt = \left. \left( \frac{1}{4} \frac{\sin{3t}}{ 3 } + \frac{3}{4} \sin{t} \right) \right|_0^x = \\
  & = \frac{1}{12} (\sin{2x} \cos{x} + \sin{x} \cos{2x} ) + \frac{3}{4} \sin{x} = \frac{1}{12} ( 2 \sin{x} \cos{x} + \sin{x} (2 \cos^2{x} -1)) = \\
  & = \frac{ \sin{x} \cos^2{x} + 2 \sin{x} }{ 3} 
\end{aligned}
\] 
\end{enumerate}

\exercisehead{30} Now using the definition of a periodic function,
\[
f(x) = f(x+p); f(x + (n+1)p) = f(x + np + p ) = f(x+ np) = f(x) 
\]
and knowing that we could write any real number in the following form,
\[
a = np + r; 0 \leq < p, r \in \mathbb{R}; n \in \mathbb{Z}
\]
then
\[
\begin{aligned}
  \int_a^{a+p} f(x) dx & = \int_r^{r+p} f(x+np) dx = \int_r^{r+p} f(x) dx = \int_r^p f + \int_p^{r+p} f(x)dx = \\
  & = \int_r^p f + \int_0^r f(x-p) dx = \int_r^p f + \int_0^r f = \int_0^p f
\end{aligned}
\]

\exercisehead{31} \begin{enumerate}
\item \[
\begin{aligned}
 \int_0^{2\pi} \sin{nx} dx  & = \frac{1}{n} \int_0^{2\pi n} \sin{x} dx = \left. \frac{1}{n} (-\cos{x}) \right|_0^{2\pi n} = -\frac{1}{n} (1-1) =0 \\
 \int_0^{2\pi} \cos{nx} dx & = \frac{1}{n} \int_0^{2\pi n} \cos{x} dx = \left. \frac{1}{n} \sin{x} \right|_0^{2\pi n} = 0 
\end{aligned}
\]
\item \[
\begin{aligned}
& \int_0^{2\pi} \sin{nx} \cos{mx} dx = \int_0^{2\pi}\frac{1}{2} (\sin{ (n+m)x } + \sin{ (n-m)x } ) dx = 0 + 0 = 0  \\
& \int_0^{2\pi} \sin{nx} \sin{mx} dx = \int_0^{2\pi}\frac{1}{2} (\cos{ (n-m)x } + \cos{ (n+m)x } ) dx = 0 + 0 = 0  \\
& \int_0^{2\pi} \cos{nx} \cos{mx} dx = \int_0^{2\pi}\frac{1}{2} (\cos{ (n-m)x } + \cos{ (n+m)x } ) dx = 0 + 0 = 0  
\end{aligned}
\]
While
\[
\begin{aligned}
  & \int_0^{2\pi } \sin^2{nx} dx = \int_0^{2\pi} \frac{ 1 - \cos{2nx}}{2} dx = \pi \\
  & \int_0^{2\pi } \cos^2{nx} dx = \int_0^{2\pi } \frac{1 + \cos{2nx}}{2} dx = \pi
\end{aligned}
\]
\end{enumerate}

\exercisehead{32} Given that $x \neq 2 \pi n; \sin{x/2} \neq 0$, 
\[
\begin{aligned}
\sum_{k=1}^n 2 \sin{x/2} \cos{kx} & = 2 \sin{x/2} \sum_{k=1}^n \cos{kx} = \sum_{k=1}^n \sin{(2k+1)\frac{x}{2} } - \sin{(2k-1)\frac{x}{2} }  = \sin{(2n+1)\frac{x}{2} } - \sin{x/2}  \\
& = \sin{nx}\cos{x/2} + \sin{x/2}\cos{nx} - \sin{x/2} = \\
& = 2 \sin{nx/2} \cos{nx/2} \cos{x/2} + \sin{x/2} (1- 2\sin^2{nx/2}) - \sin{x/2} = \\
& = 2(\sin{nx/2})(\cos{(n+1)x/2 } )
\end{aligned}
\]

\exercisehead{33} Recall that
\[
\begin{aligned}
  \cos{(2k+1) x/2} - \cos{(2k-1) x/2 } & = \cos{kx + x/2} - \cos{kx -x/2} = \\ 
  & = \cos{kx}\cos{x/2} - \sin{kx}\sin{x/2} - (\cos{kx} \cos{x/2} + \sin{kx} \sin{x/2} ) = \\
  & = -2 \sin{kx} \sin{x/2} 
\end{aligned}
\]
\[
\begin{aligned}
  -2\sin{x/2} \sum_{k=1}^n \sin{kx}  & = \sum_{k=1}^n (\cos{(2k+1) x/2} - \cos{(2k-1)x/2} ) = \cos{(2n+1)x/2} - \cos{x/2} = \\ 
  & = \cos{nx+x/2} - \cos{x/2} 
\end{aligned}
\]
Now
\[
\begin{aligned}
  \sin{ nx/2} \sin{ nx/2 + x/2} & = \sin{ nx/2} ( \sin{nx/2}\cos{x/2} + \sin{x/2} \cos{nx/2} ) = \\
  & = \sin^2{nx/2} \cos{x/2} + \sin{x/2} \cos{nx/2} \sin{nx/2} = \\
  & = \left( \frac{1 -\cos{nx}}{2} \right) \cos{x/2} + \frac{ \sin{nx}}{2} \sin{x/2} = \\
  & = \frac{1}{2} \left( \cos{x/2} - \cos{x/2} \cos{nx} + \sin{nx} \sin{x/2} \right) = \frac{1}{2} ( \cos{x/2} - \cos{(nx_x/2 ) } 
\end{aligned}
\]
Then
\[
\begin{gathered}
  -2\sin{x/2} \sum_{k=1}^n \sin{kx} = -2 \sin{nx/2} \sin{\frac{1}{2} (n+1)x }  \\
  \sum_{k=1}^n \sin{kx} = \frac{ \sin{nx/2} \sin{\frac{1}{2} (n+1)x } }{ \sin{x/2} }
\end{gathered}
\]

\exercisehead{34} Using triangle OAP, not the right triangle, if $0<x < \pi/2$
\[
\begin{gathered}
  \frac{1}{2} \cos{x} \sin{x} < \frac{1}{2} \sin{x} < \frac{x}{2}  \\
  \Longrightarrow \sin{x} < x 
\end{gathered}
\]
Now if $0 > x > -\pi/2$, $\sin{x} <0$, 
\[
|\sin{x}| = - \sin{x} = \sin{-x} = \sin{|x|} < |x|
\]

\subsection*{ 2.11 Exercises - Polar coordinates, The integral for area in polar coordinates }
\quad \\
\exercisehead{1} 
\[
\begin{gathered}
  (x-1)^2 +y^2 = 1 \\
  \Longrightarrow x^2 - 2x + 1 + y^2 = 1 
\end{gathered} \quad \quad \, \Longrightarrow r^2 = 2r \cos{\theta} \text{ or } \boxed{ r = 2\cos{\theta}} 
\]

\exercisehead{2} 
\[
\begin{gathered}
  x^2 + y^2 -x = \sqrt{ x^2 + y^2 } \\
\Longrightarrow r^2 = r(\cos{\theta} + 1 ) \text{ or } r = 1 + \cos{\theta} 
\end{gathered}
\]

\exercisehead{3} 
\[
\begin{gathered}
  (x^2 + y^2)^2 = x^2 - y^2; \quad \, y^2 \leq x^2 \\
  r^4 = r^2 (\cos^2{\theta} - \sin^2{\theta} ) 
\end{gathered} \quad \quad \, \Longrightarrow \begin{aligned}
  r^2 & = \cos{2 \theta} \\
  r = \sqrt{ \cos{2 \theta}} \text{ for } \cos{2 \theta} > 0
\end{aligned}
\]

\exercisehead{4} 
\[
(x^2 +y^2)^2 = r^4 = |x^2 - y^2 | = |r^2 \cos{2\theta} | \quad \quad \Longrightarrow \begin{aligned}
  r^2 & = |\cos{2\theta} | \\
  r & = \sqrt{ |\cos{2\theta} | }
\end{aligned}
\]
\quad \\

Recall that undefined the measure of an angle to be $2$ times the area of the sector divided by the radius squared (we chose this because it doesn't change with a change in circle size).  Then if $r^2(\theta)$ is integrable, then $A = \int_{\theta_a}^{\theta_b} \frac{1}{2} \theta r^2 d\theta$.  \quad \\

See sketches for the following exercises.  Otherwise, we compute the area of the sector.  

\exercisehead{5} Spiral of Archimedes: $f(\theta) = \theta$; \quad $0 \leq \theta \leq 2\pi$ 
\[
A = \int_0^{2\pi} \frac{1}{2} \theta^2 d\theta = \frac{1}{6} (2\pi)^3 = \boxed{ \frac{ 4 \pi^3}{3} }
\]

\exercisehead{6} Circle tangent to $y$-axis: $f(\theta) = 2\cos{\theta}$; \quad $-\pi/2 \leq \theta \leq \pi/2$ 
\[
A = \frac{1}{2} \int_{-\pi/2}^{\pi/2} 4 \cos^2{\theta}d\theta = 2 \int_{-\pi/2}^{\pi/2} \frac{ (1+\cos{2\theta})}{2} d\theta = \boxed{ \pi }
\]

\exercisehead{7} Two circles tangent to $y$-axis: $f(\theta) = 2 |\cos{\theta} |$; \quad $0\leq \theta \leq 2\pi$
\[
\frac{1}{2} \int_0^{2\pi} 4 \cos^2{\theta} d\theta = 2 \int_0^{2\pi} \frac{1 + \cos{2\theta}}{2} = \boxed{ 2 \pi } 
\]

\exercisehead{8} Circle tangent to $x$-axis: $f(\theta) = 4\sin{\theta}$ ; \quad $0\leq \theta \leq \pi$ 
\[
\frac{1}{2} \int_0^{\pi} 16 \sin^2{\theta} d\theta = 8 \int_0^{\pi} \left( \frac{ 1 - \cos{2\theta} }{ 2 } \right) d\theta = \boxed{ 4 \pi }
\]

\exercisehead{9} Two circles tangent to $x$-axis: $f(\theta) = 4 |\sin{\theta} |$; \quad \, $0 \leq \theta \leq 2 \pi$ 
\[
\frac{1}{2} \int_0^{2\pi } 16 \sin^2{\theta} d\theta = 8 \int_0^{2\pi} \frac{1- \cos{2\theta}}{2} d\theta = \boxed{ 8 \pi}
\]

\exercisehead{10} Rosepetal: $f(\theta) = \sin{2\theta}$; \quad $0\leq \theta \leq \pi/2$ \\
\[
\frac{1}{2} \int_0^{\pi/2} \sin^2{2\theta} d\theta = \frac{1}{2} \int_0^{\pi/2} d\theta \left( \frac{1- \cos{4 \theta} }{2} \right) = \boxed{ \frac{ \pi }{8} }
\]

\exercisehead{11} Four-leaved rose $f(\theta) = |\sin{2\theta}|$; \quad $0\leq \theta \leq 2\pi$ \\
\[
\frac{1}{2} \int_0^{2\pi} \sin^2{2\theta} d\theta = \frac{1}{2} \int_0^{2\pi} d\theta \left( \frac{1 - \cos{4\theta} }{2} \right) = \boxed{ \frac{ \pi}{2} }
\]

\exercisehead{13} Four-leaf clover: $f(\theta) = \sqrt{ |\cos{2\theta} | }$; \quad $0\leq \theta \leq 2\pi$ 
\[
\begin{gathered}
  \frac{1}{2} \int_0^{2\pi} |\cos{2\theta}| d\theta  = \\
  \begin{aligned} 
    & = \frac{1}{2} \left( \int_0^{\pi/4} c(2\theta) - \int_{\pi/4}^{3\pi/4} c(2\theta) + \int_{3\pi/4}^{5\pi/4} c(2\theta) - \int_{5\pi/4}^{7\pi/4} c(2\theta) + \int_{7\pi/4}^{2\pi} c(2\theta) \right) = \\
    & = \frac{1}{2} \left(  \left. \frac{ s(2\theta)}{2} \right|_0^{\pi/4} - \left. \frac{ s(2\theta)}{2 } \right|_{\pi/4}^{3\pi/4} + \left. \frac{ s(2\theta)}{2 } \right|_{3\pi/4}^{5\pi/4} - \left. \frac{ s(2\theta)}{2 } \right|_{5\pi/4}^{7\pi/4} + \left. \frac{ s(2\theta)}{2 } \right|_{7\pi/4}^{2\pi} \right) = \\
    & = \frac{1}{2} \left( \frac{1}{2} - \left( \frac{ -1 -1 }{2} \right) + \left( \frac{ 1 - (-1)  }{2} \right) - \left( \frac{-1 -1 }{2} \right) + \left( \frac{ - (-1) }{2} \right) \right) = \boxed{2}
\end{aligned}
\end{gathered}
\]

\exercisehead{14} Cardiod: $f(\theta) = 1 + \cos{\theta}$; \quad $0\leq \theta \leq 2\pi$ \\
\[
\frac{1}{2} \int_0^{2\pi} 1 + 2 \cos{\theta} + \cos^2{\theta} = \frac{1}{2} (2 \pi + \frac{1}{2} (2\pi) ) = \boxed{ \frac{3\pi}{2} }
\]

\exercisehead{15} Limacon: $f(\theta) = 2 +\cos{\theta}$; \quad $0\leq \theta \leq 2\pi$ 
\[
\frac{1}{2} \int_0^{2\pi} 4 + 4 \cos{\theta} + \cos^2{\theta} = \frac{1}{2} ( 4 (2\pi) + \frac{1}{2} (2\pi ) ) = \boxed{ \frac{9\pi}{2} } 
\]

\subsection*{ 2.17 Exercises - Average value of a function  }

\exercisehead{1} $\frac{1}{b-a} \int x^2 dx = \frac{1}{3} (b^2 + ab + a^2)$

\exercisehead{2} $ \frac{1}{1-0} \int x^2 + x^3 = \frac{7}{12} $

\exercisehead{3} $ \frac{1}{4-0} \int x^{1/2} = \frac{4}{3}$  

\exercisehead{4} $ \frac{1}{8-1} \int x^{1/3} = \frac{45}{28}$

\exercisehead{5} $\frac{1}{ \pi/2 -0} \int_0^{\pi/2} \sin{x} = \frac{2}{\pi}$

\exercisehead{6} $ \frac{1}{ \pi/2 - -\pi/2} \int \cos{x} = 2/\pi$

\exercisehead{7} $\frac{1}{ \pi/2 -0} \int \sin{2x} = -1/\pi ( -1 -1)= 2/\pi$

\exercisehead{8} $\frac{1}{ \pi/4 -0} \int \sin{x} \cos{x} = \frac{1}{\pi}$

\exercisehead{9} $\frac{1}{\pi/2 -0} \int \sin^2{x} = \left. \frac{1}{\pi} (x- \sin{2x}/2) \right|_0^{\pi} = \frac{1}{2}$

\exercisehead{10} $\frac{1}{\pi-0} \int \cos^2{x} = \frac{1}{2}$

\exercisehead{11} 
\begin{enumerate}
\item $\frac{1}{a-0} \int x^2 = a^2/3 = c^2 \Longrightarrow c = a/\sqrt{3}$
\item $\frac{1}{a-0} \int x^n = \left. \frac{1}{a} \frac{1}{ n+1} x^{n+1} \right|_0^a = \frac{a^n}{n+1} = c^n \Longrightarrow c = \frac{a}{(n+1)^{1/n}} $
\end{enumerate}

\exercisehead{12} 
\[
\begin{aligned}
  & A = \int wf/ \int w  \quad \int w x^2 = k \int x \\
  & \int x^3 = \frac{1}{4} x^4 = k \frac{1}{2} x^2; k =\frac{1}{2},  w = x \\
  & \int x^4 = \frac{1}{5} x^5 = k \frac{1}{3} x^3; k = \frac{3}{5}, w= x^2 \\
  & \int x^5 = \frac{1}{6} x^6 = k \frac{1}{4} x^4; k =\frac{2}{3}, w = x^3 
\end{aligned}
\]

\exercisehead{13} 
\[
\begin{aligned}
  & A(f+g) = \frac{1}{b-a} \int f + g = \frac{1}{b-a} \int f + \frac{1}{b-a} \int g = A(f) + A(g) \\
  & A(cf) = \frac{1}{b-a} \int cf = c \left( \frac{1}{b-a} \right) \int f \\
  & A(f) = \frac{1}{b-a} \int f \leq \frac{1}{b-a} \int g = A(g)
\end{aligned}
\]

\exercisehead{14}
\[
\begin{gathered}
\begin{aligned}
  A (c_1 f + c_2 g ) & = \frac{ \int w (c_1 f + c_2 g ) }{ \int w }  = \frac{ c_1 \int wf }{ \int w } + \frac{ c_2 \int wg}{ \int w } \\ 
  & = c_1 A(f) + c_2 A(g) 
\end{aligned} \\
f \leq g \, w > 0 ( \text{ nonnegative } ), \Longrightarrow wf \leq wg
\end{gathered}
\]

\exercisehead{15} 
\[
\begin{gathered}
  A_a^b(f) = \frac{1}{b-a} \int_a^b f = \frac{1}{b-a} \left( \int_a^c f + \int_c^b f \right) = \left( \frac{c-a}{b-a} \right)\left( \frac{ \int_a^c f }{ c-a} \right) + \frac{ b-a-(c-a)}{b-a} \frac{ \int_a^b f }{ b-c} \\
  \begin{gathered}
    a< c < b \\
    0 < \frac{c-a}{b-a} < 1 
\end{gathered}
  \text{ Let } t = \frac{c-a}{b-a}  \\
  \Longrightarrow A_a^b(f) = t A_a^c(f) + (1-t)A_c^b(f) \\
  A_a^b(f) = \frac{ \int_a^b wf }{ \int_a^b w } = \frac{ \int_a^c w }{ \int_a^b w } \frac{ \int_a^c wf }{ \int_a^c w } + \left( \frac{ \int_a^b w - \int_a^c w }{ \int_a^b w } \right) \frac{ \int_c^b wf}{ \int_c^b w } \\
  0 < \frac{ \int_a^c w }{ \int_a^b w } < 1 \text{ since $w$ is a nonnegative function.  Let $t = \frac{ \int_a^c w }{ \int_a^b w } $} \\
    \Longrightarrow A_a^b(f) = t A_a^c(f) + (1-t)A_c^b(f) 
\end{gathered}
\]

\exercisehead{16} Recall that $x_{cm} = \frac{ \int x \rho}{ \int \rho}$ or $r_{cm} = \frac{ \int r dm}{ M}$.  
\[
\begin{aligned}
  & x_{cm} = \frac{ \int_0^L x }{ \int_0^L 1 } = \frac{L}{2}  \\
  & I_{cm} = \int r^2 dm = \int x^2 (1) = L^3/3  \\
  & r^2 = \frac{ I_{cm}}{ \int_0^L 1 } = L^2/3 \Longrightarrow r = \frac{L}{\sqrt{3}}
\end{aligned}
\]

\exercisehead{17} 
\[
\begin{aligned}
  & x_{cm} = \frac{ \int_0^{l/2} x + \int_{L/2}^L 2x dx }{ \frac{L}{2} + 2 (L-L/2) } = \frac{ yL^2}{12}  \\
  & I_{cm} = \int_0^{L/2} x^2 + \int_{L/2}^L 2 x^2 = 5L^3/8 
  & r^2 = \frac{ 5L^3/8}{ 3L/2} = \frac{ 5 L^2}{12} \Longrightarrow r = \frac{ \sqrt{5} L}{ 2 \sqrt{3}}
\end{aligned}
\]

\exercisehead{18} $\rho(x) = x$ for $0 \leq x \leq L$
\[
\begin{aligned}
& x_{cm} = \frac{ \int x x dx }{ \int x dx } = \frac{ \left. \frac{1}{3} x^3 \right|_0^L }{ \left. \frac{1}{2} x^2 \right|_0^L } = \frac{2}{3} L \\
& I_{cm} = \int x^2 x dx = L^4/4 \\
& r^2 = \frac{ L^4/4}{ L^2/2} = L^2/2 \quad r = \frac{L}{\sqrt{2}}
\end{aligned}
\]

\exercisehead{19}
\[
\begin{aligned}
  & x_{cm} = \frac{ \int x x dx + \int x \frac{L}{2} dx }{ \int x dx + \int L/2 } = \frac{ \left. \frac{1}{3} x^3 \right|_0^{L/2} + \left. \frac{L}{2} (x^2/2) \right|_{L/2}^L}{ \left. \frac{1}{2}x^2 \right|_0^{L/2} + \frac{L}{2} (L-L/2) } = 11L/18 \\
  & I_{cm} = \int x^2 x dx + \int x^2 L/2 dx = L^4 31/192 \\
  &  r^2 = I_{cm}/ (L^2 3/8 ) = L^2 31/72 \quad r = \frac{ \sqrt{31} L}{ 6\sqrt{2}}
\end{aligned}
\]

\exercisehead{20} $ \rho(x) = x^2 \text{ for } 0 \leq x \leq L$
\[
\begin{aligned}
  & x_{cm} = \frac{ \int x x^2 dx }{ \int x^2 } = 3L/4 \\
  & I_{cm} = \int x^2 x^2 dx = L^5/5 \\
  & r^2 = \frac{ I_{cm}}{ \frac{1}{3}L^3 } = \frac{3}{5} L^2 \quad  r = \sqrt{ \frac{3}{5} } L 
\end{aligned}
\]

\exercisehead{21} 
\[
\begin{aligned}
  & x_{cm} = \frac{ \int_0^{L/2} x x^2 dx + \int_{L/2}^L x \frac{L^2}{4} dx }{ \int_0^{L/2} x^2 dx + \int_{L/2}^L \frac{L^2}{4} dx } = 21 L/32  \\
  & I_{cm} = int_0^{L/2} x^2 x^2 dx + \int_{L/2}^L x^2 \frac{L^2}{4} dx = 19 L^5/ 240 \\
  & r^2 = \frac{I_{cm}}{ L^3/6} = 19 L^2/40 \Longrightarrow r = \sqrt{19}L/ 2 \sqrt{10}
\end{aligned}
\]

\exercisehead{22} \textbf{ Be flexible about how you can choose a convenient origin to evaluate the center-of-mass from } \\
Let $\rho = cx^n$
\[
\begin{gathered}
c\int_0^L x^{n} dx = \frac{1}{n+1} L^{n+1} c = M \\
\Longrightarrow c = \frac{ (n+1) M}{ L^{n+1} } \\
c\int_0^L x x^n dx = c \frac{1}{n+2} L^{n+2} = \frac{ n+1}{n+2} ML = \frac{3 ML}{4}
\end{gathered}
\]

\[
\begin{gathered}
x_{cm} = \frac{ \int x \rho}{ M} = \frac{3L}{4} \Longrightarrow \int x \rho = \frac{n+1}{n+2} = \frac{3}{4} \Longrightarrow n =2 \\
\boxed{ \rho = \frac{ 3M}{L^3}x^2 }
\end{gathered}
\]

\exercisehead{23} 
\begin{enumerate}
  \item 
\[
\frac{1}{ \pi/2 -0 } \int 3 \sin{2t} = \frac{6}{\pi} 
\]
\item 
\[
\frac{1}{\pi/2 - 0 } \int 9 \sin^2{2t} = 9/2 \Longrightarrow v_{rms} = 3 \sqrt{2}/2
\]
\end{enumerate}

\exercisehead{24}
$T=2\pi$ (just look at the functions themselves) 

\[
\frac{1}{2\pi} \int_0^{2\pi} 160 \sin{t} 2 \sin{(t - \pi/6) } = 80\sqrt{3}
\]

\subsection*{ 2.19 Exercises - The integral as a function of the upper limit.  Indefinite integrals.  }

\exercisehead{1} $\int_0^x (1+t+t^2)dt = x + \frac{1}{2} x^2 + \frac{1}{3} x^3 $

\exercisehead{2} $2y + 2y^2 + 8y^3/3 $

\exercisehead{3} $2x + 2x^2 + 8x^3/3 - (-1+1/2 + -1/3 ) = 2 (x + x^2 + 4x^3/3) + 5/6$

\exercisehead{4} $\int_1^{1-x} (1-2t +3t^2) dt = \left. (t-t^2 +t^3) \right|_1^{1-x} = -2x +2x^2 -x^3$

\exercisehead{5} $ \int_{-2}^x t^4 + t^2 = \left. \frac{1}{5}t^5 + \frac{1}{3}t^3 \right|_{-2}^x = \frac{x^5}{5} + \frac{x^3}{3} + \frac{40}{3}$

\exercisehead{6} $\int_x^{x^2} t^4 +2t^2 + 1 = \left. \left( \frac{t^5}{5} + \frac{2}{3} t^3 + t \right) \right|_x^{x^2} = \frac{1}{5} (x^{10}-x^5 ) + \frac{2}{3} (x^6 - x^3 ) + x^2 -x $

\exercisehead{7} $\left. \left( \frac{2}{3}t^{3/2} +t \right) \right|_1^x = \frac{2}{3} (x^{3/2}-1)+(x-1)$

\exercisehead{8} $\left. \left( \frac{2}{3} t^{3/2} + \frac{4}{5}t^{5/4} \right) \right|_x^{x^2} = \frac{2}{3} (x^3 - x^{3/2}) + \frac{4}{5} (x^{5/2} - x^{5/4} )$

\exercisehead{9} $\left. \sin{t} \right|_{i\pi}^x = \sin{x}$

\exercisehead{10}$ \left. \left( \frac{t}{2} + \sin{t} \right) \right|_0^{x^2} = \frac{x^2}{2} +\sin{x^2}$

\exercisehead{11} $\left. \left( \frac{1}{2} t + \cos{t} \right) \right|_x^{x^2} = \frac{x^2-x}{2} + \cos{x^2} - \cos{x} $

\exercisehead{12} $\left. \left( \frac{1}{3} u^3 + -\frac{1}{3} \cos{3u} \right) \right|_0^x = \frac{x^3}{3} + - \frac{1}{3} (\cos{3x} -1)$

\exercisehead{13} $\left. \left( \frac{1}{3} v^3 + \frac{ \cos{3v}}{ -3} \right) \right|_x^{x^2} = \frac{x^6 - x^3}{3} + \frac{-1}{3} (\cos{3x^2} - \cos{3x} )$

\exercisehead{14} $\int \frac{ 1 - \cos{2x} }{2} + x = \left. \left( \frac{1}{2} x - \frac{ \sin{2x}}{4} + \frac{1}{2} x^2 \right) \right|_0^y = \boxed{ \frac{y}{2} - \frac{ \sin{2y}}{4} + \frac{y^2}{2} }$

\exercisehead{15} $\left. \left( \frac{-\cos{2w}}{2} + 2 \sin{\frac{w}{2} } \right) \right|_0^x = \boxed{ - \frac{ (\cos{2x} - 1)}{ 2 } + 2 \sin{ \frac{x}{2}} }$

\exercisehead{16} $\int_{-\pi}^x (\frac{1}{2} + \cos{t} )^2 dt = \int_{-\pi}^x \frac{1}{4} + \cos{t} + \cos^2{t} = \frac{1}{4} (x + \pi) + \sin{x} + \left. \frac{1}{2} \left( t + \frac{ \sin{2t}}{2} \right) \right|_{-\pi}^x = \frac{3}{4} (x+\pi) + \sin{x} + \frac{1}{4} \sin{2x} $

\exercisehead{17} $ \int_0^x (t^3 - t) dt = \frac{1}{3} \int_{\sqrt{2}}^x (t-t^3 )dt$

Note that $t^3-t < 0 $ for $0 < t \leq 1$ and $t^3-t >0$ for $t>1$.  $t-t^3 <0$ for $t > \sqrt{2}$.  

\[
\begin{gathered}
  \frac{1}{4}x^4 - \frac{1}{2} x^2 = \frac{1}{3} \left. \left( \frac{1}{2} t^2 - \frac{1}{4} t^4 \right) \right|_{\sqrt{2}}^x = \frac{1}{6}x^2 - \frac{1}{12} x^4 \\
  \Longrightarrow \frac{1}{3} x^4 - \frac{2}{3} x^2 = 0 \Longrightarrow x = 0, \, \boxed{ x = \sqrt{2}}
\end{gathered}
\]

$\int_0^1 (t^3 - t)dt + \int_1^{\sqrt{2}} (t^3 - t)dt$ ``cancel'' each other out.  

\exercisehead{18} $f(x)=x -[x] - \frac{1}{2}$ if $x$ is not an integer; $f(x) = 0 $ if $x\in \mathbb{Z}$.  

For any real number, $x = q+r, 0 \leq r <1, q \in \mathbb{Z}$.  So then 
\[
\begin{gathered}
  x - [x] = r \\
  f(x) = r - \frac{1}{2} 
\end{gathered}
\]
\begin{enumerate}
\item To show the periodicity, consider
\[
\begin{gathered}
  f(x+1) = x+1 - [x+1] - \frac{1}{2} = r - \frac{1}{2} = f(x) \text{ since}
  x+1 = q+1 +r , \, [x+1] = q+1 \\
  x+1 - [x+1] = r - \frac{1}{2}
\end{gathered}
\]
\item $P(x) = \int_0^x f(t) dt = \int_0^x (t - \frac{1}{2}) = \frac{1}{2} x^2 - \frac{1}{2} x$ because given $0< x \leq 1$, then $q=0$ for $x$, so we can use $r=t$.  

To show periodicity,
\[
\begin{gathered}
  P(x+1) = \int_0^{x+1} f(t) dt = \int_0^1 f(t)dt + \int_1^{x+1} f(t)dt = 0 +\int_0^x f(t+1)dt = \int_0^x f(t)dt = P(x) \\
  \text{ since } \int_0^1 f(t)dt = \left. \frac{1}{2} (x^2 - x) \right|_0^1 = 0
\end{gathered}
\]
\item Since $P$ itself is periodic by $1$, then we can consider $0\leq x <1$ only.  Now $x-[x]=r$ and $P(x) = \frac{1}{2} (r^2 - r)$.  So $P(x) = \frac{1}{2} ( (x-[x])^2 - (x-[x]))$.  
\item \[
\begin{gathered}
  \int_0^1 (P(t) +c ) dt = 0 \, \Longrightarrow \int_0^1 P(t) dt = -c \\
  0 \leq t \leq 1 \text{ so } P(t) = \frac{1}{2} (t^2 - t)  \\
  \Longrightarrow \int_0^1 P(t) dt = \frac{1}{2} \left. \left( \frac{1}{3} t^3 - \frac{1}{2} t^2 \right) \right|_0^1 = \frac{1}{2} \frac{-1}{6} \Longrightarrow \boxed{ c = \frac{1}{12} }
\end{gathered}
\]
\item $Q(x) = \int_0^x (P(t) + c) dt $
\[
\begin{gathered}
  \begin{aligned}
    Q(x+1) & = \int_0^{x+1} (P(t) +c)dt = \int_0^1 (P(t)+c)dt + \int_1^{x+1} (P(t) +c)dt = \\ 
    & = 0 + \int_0^x (P(t+1)+c) dt = \int_0^x (P(t) +c ) dt = Q(x) 
  \end{aligned} \\
\text{ so without loss of generality, consider } 0 \leq x < 1 \\
\Longrightarrow Q(x) \int_0^x \frac{1}{2} (t^2 - t) + \frac{1}{12} = \boxed{ \frac{1}{6}x^3 - \frac{1}{4} x^2 + \frac{x}{12} }
\end{gathered}
\]
\end{enumerate} 

\exercisehead{19} $g(2n) = \int_0^{2\pi} f(t)dt $ \\
Consider 
\[
\begin{aligned}
  \int_{-1}^1 f(t)dt & = \int_0^1 f(t) dt + \int_{-1}^0 f(t) dt = \int_0^1 f(t) dt + \frac{1}{-1} \int_1^0 f(-1 t) dt = \\
  & = \int_0^1 f + \int_1^0 f(t) dt = 0 
\end{aligned}
\]
Consider that $\int_1^3 f(t) dt = \int_{-1}^1 f(t+2)dt = \int_{-1}^1 f(t) dt =0$.  Then, by induction,
\[
\int_1^{2n+1} f = \int_1^{2n-1} f + \int_{2n-1}^{2n+1} f(t)dt = 0 + \int_{-1}^1 f(t +2n) dt = \int_{-1}^1 f(t)dt = 0
\]
\begin{enumerate}
\item \[
\begin{aligned}
  g(2n) & = \int_0^1 f + \int_1^{2n -1} f + \int_{2n-1}^{2n} f = \int_0^1 f + \int_{-1}^0 f(t) dt = \int_0^1 f + -\int_1^0 f(-t) dt \\
  & = \int_0^1 f + \int_1^0 f = 0 
\end{aligned}
\]
\item \[
\begin{gathered}
  g(-x) = \int_0^{-x} f = -\int_0^x f(-t)dt = \int_0^x f(t)dt = g(x) \\
  g(x+2) = \int_0^{x+2} f(t) dt = \int_0^2 f + \int_2^{x+2} f = \int_0^x f(t+2)dt = \int_0^x f(t) dt = g(x)
\end{gathered}
\]
\end{enumerate}

\exercisehead{20} \begin{enumerate}
\item $g$ is odd since \[
  g(-x) = \int_0^{-x} f(t) dt = -\int_0^x f(-t)dt = -\int_0^x f(t) dt = -g(x) \\
\]
Now
\[
\begin{gathered}
  g(x+2) = \int_0^{x+2} f = \int_0^2 f + \int_2^{x+2} f = g(2) + \int_0^x f(t+2) dt = g(2) + \int_0^x f(t)dt = g(2) + g(x) \\
  \Longrightarrow g(x+2) -g(x) = g(2)
\end{gathered}
\]
\item \[
\begin{gathered}
\begin{aligned}
  g(2) & = \int_0^2 f = \int_1^2 f + \int_0^1 f = \int_1^2 f + A = \int_{-1}^0 f(t+2)dt + A = \int_{-1}^0 f(t)dt + A = \\
  & - \int_1^0 f(-t)dt + A = 2A
\end{aligned} \\
g(5) - g(3) = g(2) \\
g(3) = g(2) + \int_2^3 f(t) dt = 2A + \int_0^1 f(t+2)dt = 2A + A = 3A  \\
\Longrightarrow g(5) = 3A + 2A = 5A 
\end{gathered}
\]
\item The key observation is to see that $g$ must repeat itself by a change of $2$ in the argument.  To make $g(1)=g(3)=g(5)$, they're different, \textbf{ unless} $A=0$!
\end{enumerate}

\exercisehead{21} From the given, we can derive
\[
\begin{gathered}
  g(x) = f(x+5), f(x) = \int_0^x g(t)dt \\
  \Longrightarrow f(5) = \int_0^5 g(t) dt = g(0) = 7 
\end{gathered}
\]
\begin{enumerate}
\item The \textbf{ key insight } I uncovered was, when stuck, one of the things you can do, is to \textbf{ think geometrically } and \textbf{ draw a picture}.  
\[
\begin{gathered}
g(-x) = f(-x+5) = g(x) = -f(x-5)  \\
\Longrightarrow -g(x) = f(x-5)
\end{gathered}
\]
\item \[
\int_0^5 f(t)dt = \int_{-5}^0 f(t+5)dt = \int_{-5}^0 g(t) dt = - \int_0^{-5} g(t) dt = \int_0^5 g(-t)dt = \int_0^5 g(t) dt = f(5) = 7 
\]
\item 
\[
\begin{aligned}
  \int_0^x f(t) dt & = \int_{-5}^{x-5} f(t+5)dt = \int_{-5}^{x-5} g(t) dt = \int_0^{x-5} g + \int_{-5}^0 g = f(x-5) + -\int_0^{-5} g(t)dt = \\
  & f(x-5) + \int_0^5 g(-t)dt = f(x-5) + f(5) =-g(x) + g(0)  
\end{aligned}
\]
where we've used $f(x-5) = -g(x)$ in the second and third to the last step.  
\end{enumerate}


\subsection*{ 3.6 Exercises - Informal description of continuity, The definition of the limit of a function, The definition of continuity of a function, The basic limit theorems.  More examples of continuous functions, Proofs of the basic limit theorems }

Polynomials are continuous.  

\exercisehead{1} $\lim_{x \to 2 } \frac{1}{x^2} = \frac{1}{ \lim_{x\to 2} x^2 } =\frac{1}{4}$ 

\exercisehead{2} $ \frac{ \lim_{x \to 0} (25 x^3 + 2) }{ \lim_{x\to 0} (75 x^7 -2 ) } = -1$ 

\exercisehead{3} $\lim_{x\to 2} \frac{ (x-2)(x+2)}{(x-2)} = 4$ 

\exercisehead{4} $\lim_{x\to 1}\frac{ (2x-1)(x-1)}{x-1} = 1  $

\exercisehead{5} $\lim_{h\to 0} \frac{ t^2 +2th +h^2 -t^2}{h} = 2t $

\exercisehead{6} $\lim_{x\to 0} \frac{ (x-a)(x+a)}{ (x+a)^2} = -1$

\exercisehead{7} $\lim_{a\to 0} \frac{ (x-a)(x+a)}{ (x+a)^2 } = 1$

\exercisehead{8} $\lim_{x\to a} \frac{ (x-a)(x+a)}{ (x+a)^2} = 0$

\exercisehead{9} $\lim_{t \to 0} \tan{t} = \frac{ \lim_{t\to 0} \sin{t} }{ \lim_{x\to 0} \cos{t} } = \frac{0}{1} = 0 $

\exercisehead{10} $\lim_{t \to 0} (\sin{2t} + t^2 \cos{5t} ) = \lim_{t\to 0} \sin{2t} + \lim_{t\to 0} t^2 \lim_{t\to 0} \cos{5t} = 0 + 0 =0$

\exercisehead{11} $\lim_{x\to 0^+} \frac{ |x|}{x} = 1$

\exercisehead{12} $\lim_{x \to 0^-} \frac{ |x|}{x} =-1$

\exercisehead{13} $\lim_{x \to 0^+} \frac{ \sqrt{x^2}}{x} = +1$

\exercisehead{14} $\lim_{x \to 0^-} \frac{ \sqrt{x^2}}{ x} =-1$

\exercisehead{15} $\lim_{x\to 0} \frac{ 2\sin{x} \cos{x}}{x} = 2$

\exercisehead{16} $\lim_{x\to 0} \frac{ 2 \sin{x} \cos{x}}{ \cos{2x} \sin{x}} =2$

\exercisehead{17} $\lim_{x \to 0} \frac{ \sin{x} \cos{4x} + \sin{4x} \cos{x}}{ \sin{x} } = 1 + \lim_{x\to 0} \frac{ 2\sin{2x} \cos{2x}}{ \sin{x}} = 1 + 2 \left( \lim_{x\to 0} \frac{ 2 \sin{x} \cos{x} \cos{2x}}{ \sin{x}} \right) =5$

\exercisehead{18} $\lim_{x\to 0} \frac{ 5 \sin{5x}}{5x} - \lim_{x\to 0} \frac{ 3 \sin{3x}}{3x} = 5 -3 =2$

\exercisehead{19} 
\[
\begin{gathered}
\lim_{x\to 0} \frac{ \sin{ \left( \frac{x+a}{2} + \frac{x-a}{2} \right)} -  \sin{ \left( \frac{x+a}{2} - \left( \frac{x-a}{2} \right) \right)} }{ x-a} = \\
= \lim_{x \to 0} \left( \frac{ \sin{ \frac{ x+a}{2} } \cos{\frac{x-a}{2} } + \sin{ \frac{ x-a}{2} } \cos{\frac{x+a}{2} }  - \left( \sin{ \frac{ x+a}{2} } \cos{\frac{x-a}{2} } - \sin{ \frac{ x-a}{2} } \cos{\frac{x+a}{2} } \right) }{ x-a} \right) = \\
= \lim_{x\to a} \frac{ 2\sin{\frac{x-a}{2} } \cos{ \frac{x+a}{2}} }{ x-a} = \cos{a}
\end{gathered}
\] 

\exercisehead{20} $\lim_{x\to 0} \frac{ 2\sin^2{x/2}}{ 4 (x/2)^2 } = \frac{1}{2} \left( \lim_{x \to 0} \frac{ \sin{x/2}}{ x/2} \right) = \frac{1}{2}$

\exercisehead{21} $\lim_{x\to 0} \frac{ 1- \sqrt{ 1-x^2} }{ x^2} \left( \frac{ 1+\sqrt{1 -x^2}}{ 1+\sqrt{1-x^2}} \right) = \lim_{x\to 0} \frac{ 1-(1-x^2)}{ x^2 (1+ \sqrt{1-x^2} ) } = \boxed{ \frac{1}{2} }$

\exercisehead{22} $b,c$ are given.  \\
$\sin{c} = ac +b, a = \frac{ \sin{c} -b}{c}, \, c \neq 0$.  \\
if $c=0$, then $b=0, a \in \mathbb{R}$.  

\exercisehead{23} $b,c$ are given.  \\
$2\cos{c} = ac^2 +b, a = \frac{ 2\cos{c} -b}{c^2}, \, c \neq 0$.  \\
If $c=0$, then $b=2, a \in \mathbb{R}$.  

\exercisehead{24} \[
\begin{aligned}
  & \text{ tangent is continuous for } x \notin (2n+1) \pi/2 \\
  &  \text{ cotangent is continuous for } x \notin 2n \pi  
\end{aligned}
\]

\exercisehead{25} $\lim_{x \to 0} f(x) = \infty$.  No $f(0)$ cannot be defined.  

\exercisehead{26} \begin{enumerate}
\item $|\sin{x} -0| = |\sin{x} | < |x|$.  Choose $\delta = \epsilon$ for a given $\epsilon$.  \\
  \quad Then $\forall \epsilon > 0, \exists \delta >0$ such that $|\sin{x} -0 | < \epsilon$ when $|x| < \delta$.  
\item 
\[
|\cos{x} -1| = | - 2 \sin^2{x/2} | = 2 |\sin{x/2}|^2 < 2 |\frac{x}{2} |^2 = \frac{ |x|^2}{ 2 } < 2 \epsilon/2 = \epsilon
\]
If we had chosen $\delta_0 =\sqrt{2 \epsilon}$ for a given $\epsilon$.  $|x-0| < \delta = \sqrt{2 \epsilon }$.  
\item \[
\begin{gathered}
  |\sin{x} (\cos{h} -1) + \cos{x} \sin{h} | \leq |\sin{x} | | \cos{h} -1 | + |\cos{h} | |\sin{h} | < \frac{ \epsilon}{2} +  \frac{ \epsilon}{2} = \epsilon \\
  \begin{aligned}
    |\cos{x+h} - \cos{x} | & = | \cos{x} \cos{h} -\sin{x}\sin{h} -\cos{x} | = | \cos{x} ( \cos{h} -1 ) - \sin{x} \sin{h} | \leq \\
    & \leq |\cos{x} | |\cos{h} -1 | + |\sin{x} | |\sin{h} | < \frac{ \epsilon}{2} +  \frac{ \epsilon}{2} = \epsilon 
  \end{aligned} \\
  \text{ since } \forall \epsilon > 0 \exists \delta_1, \delta_2 > 0 \text{ such that } |\cos{h} -1| < \epsilon_0; |\sin{h}| < \epsilon \text{ whenever } |h| < \min{(\delta_1, \delta_2 )} \\
  \text{ Choose } \delta_3 \text{ such that if } |h| <\delta_3; |\cos{h} -1| < \frac{\epsilon}{2}; |\sin{h}| < \frac{ \epsilon}{2} 
\end{gathered}
\]
\end{enumerate}

\exercisehead{27} $f(x)-A = \sin{\frac{1}{x}} -A$.  \\
Let $x = \frac{1}{n \pi }$.  
\[
|f(x) -A| = |\sin{n\pi} -A | > ||\sin{n\pi} | -|A| | > |1-|A|| 
\]
Consider $|x-0| = |x| = \frac{1}{n\pi} \leq \delta(n)$.  Consider $\epsilon_0 = \frac{ |1 - |A||}{2}$.  Then suppose a $\delta(n) \geq |x-0|$ but $|f(x)-A| > \epsilon_0$.  Thus, contradiction.  

\exercisehead{28} Consider $x \leq \frac{1}{n}, n \in \mathbb{Z}^+ ,\, n > M(n)$ ($n$ is a given constant)
\[
\begin{gathered}
  f(x) = \left[ \frac{1}{x} \right] = [n] = n, \,  \text{ for } m > M(n), x = \frac{1}{m} 
  f(x) > M(n)
\end{gathered}
\]
so $\forall \epsilon > 0$, we cannot find $\delta = \frac{1}{n}$ such that $|f(x) -A| < \epsilon $ for $x< \delta$.  \\
\quad So $f(x) \to \infty$ as $x \to 0^+$.  

Consider $\frac{1}{n} \geq x > 0, \, n \in \mathbb{Z}^-$; $-n > M(n)$.  
\[
f(x) = \left[ \frac{1}{x} \right] = [n] = n < -M(n)
\]
Since integers are unbounded, we can consider $n<A$, so that 
\[
|f(x) -A| > ||f| - |A|| = -n -|A| > M(n) - |A|.  \text{ Choose $n$ such that $M(n)-|A| > 0$ }
\]

\exercisehead{29} 
\[
|f-A| = |(-1)^{[1/x] } - A | \geq ||(-1)^{[1/x]} | -|A| | = |1-|A||
\]
Choose $\epsilon < |1-|A||$.  Then $\forall \delta >0 $ ( such that $|x| < \delta$ ), $|f-A| > \epsilon$.  Thus there's no value for $f(0)$ we could choose to make this function continuous at $0$.  

\exercisehead{30} Since
\[
|f(x)| = |x| |(-1)^{[1/x]}| = |x|
\]
So $\forall \epsilon$, let $\delta = \epsilon$.  

\exercisehead{31} $f$ continuous at $x_0$.  \\
Choose some $\epsilon_0, 0 < \epsilon_0 < \min{ (b-x_0, x_0 -a) }$.  Then $\exists \delta_0 = \delta(x_0, \epsilon_0)$.  \medskip \\
Consider $\epsilon_1 = \frac{ \epsilon_0}{2}$ and $\delta_1 = \delta(x_0, \epsilon_1)$  \\
\quad Consider $x_1 \in (x_0 - \delta_1, x_0 + \delta_1)$, so that $|f(x_1) - f(x_0)| < \epsilon_1$.  

Proceed to construct a $\delta$ for $x_1$, some $\delta(x_1;\epsilon_0)$
\[
|x-x_1| = |x-x_0 + x_0 -x_1| < |x-x_0 | + |x_0 -x_1|
\]
Without loss of generality, we can specify $x_1$ such that $|x_0 -x_1| < \frac{ \delta_1}{2}$.  Also, ``pick'' only the $x$'s such that 
\[
\begin{gathered}
  |x-x_0| < \frac{ \delta_1}{ 2} < \delta_1 \\
  \Longrightarrow |x - x_1| < \frac{ \delta_1}{2} + \frac{ \delta_1}{2} = \delta_1
\end{gathered}
\]

Thus, ``for these $x$'s''
\[
| f(x) - f(x_1) | = |f(x) -f(x_0) + f(x_0) -f(x_1) | < | f(x) -f(x_0)| + |f(x_1) -f(x_0)| < \epsilon_1 + \epsilon_1 = \epsilon_0
\]
So $\forall \epsilon_0, \exists \delta_1$ for $x_1$.  $f$ is continuous at $x_1 \in (a,b)$.  Thus, there must be infinitely many points that are continuous in $(a,b)$, and at the very least, some or all are ``clustered'' around some neighborhood about the one point given to make $f$ continuous.  

\exercisehead{32} Given $\epsilon = \frac{1}{n}$, $|f(x)| = |x \sin{ \frac{1}{x} } | = |x| | \sin{ 1/x} | < |x| (1) $.  \\
\quad Let $\delta = \delta(n) = \frac{1}{n}$, so that $|x| < \frac{1}{n}$.  \\
$\Longrightarrow |f(x) | <\frac{1}{n}$

\exercisehead{33} \begin{enumerate}
\item Consider $x_0 \in [a,b]$.  \medskip \\
Choose some $\epsilon_0$, $0 < \epsilon_0 < \min{ (b-x_0, x_0 -a ) } \neq 0$ , \, \text{ ( $x_0$ could be $a$ or $b$ ) } 

Consider, without loss of generality, only ``$x$'s'' such that $x \in [a,b]$.  
\[
|f(x) -f(x_0)| \leq |x -x_0|
\]
Let $\delta_0 = \delta(\epsilon_0,x_0) = \epsilon_0$ $\Longrightarrow |f(x)-f(x_0)| < \epsilon_0$.  \medskip \\
Since we didn't specify $x_0, \forall x_0 \in [a,b], f$ is continuous at $x_0$.  
\item \[
\begin{gathered}
\left| \int_a^b f(x) dx - (b-a) f(a) \right| = \left| \int_a^b (f(x) -f(a)) dx \right| \leq \int_a^b |f(x) -f(a) | dx \leq \\
\leq \int_a^b |x-a| dx = \left. (\frac{1}{2} x^2 -ax ) \right|_a^b = \frac{1}{2} (b-a)(b+a) - a(b-a) = \frac{ (b-a)^2}{2} 
\end{gathered}
\]
\item \[
\begin{gathered}
  \left| \int_a^b f(x) dx - (b-a) f(c) \right| = \left| \int_a^b (f(x) -f(c) ) dx \right| \leq \int_a^b |f(x) -f(c) | dx \leq \int_a^b |x-c| dx = \\
  = \int_a^c (c-x)dx + \int_c^b (x-c) dx = c(c-a) - \frac{1}{2} (c-a)(c+a) + \frac{1}{2} (b-c)(b+c) -c(b-c) = \\
  = \frac{1}{2} ( (c-a)^2 + (b-c)^2) 
\end{gathered}
\]

Draw a figure for clear, geometric reasoning.  

Consider a square of length $(b-a)$ and a $45-45$ right triangle inside.  From the figure, it's obvious that right triangles of $c-a$ length and $(b-c)$ length lie within the $(b-a)$ right triangle.  \medskip \\
Compare the trapezoid of $c-a,b-a$ bases with the $b-a$ right triangle.  
\[
\frac{1}{2} (b-c)(b-a+c-a) = \frac{1}{2} (b-c) (b-c+2(c-a)) > \frac{1}{2} (b-c)^2
\]

Indeed, the trapezoid and $c-a$ right triangle equals the $b-a$ trapezoid since
\[
\begin{gathered}
  \frac{1}{2} (b-c) (b-a+c-a) + \frac{1}{2} (c-a)^2 = \frac{1}{2} (b^2 -c^2 -2ab + 2ac + c^2 -2ca + a^2) = \frac{1}{2} (b-a)^2 \\
  \Longrightarrow \frac{1}{2} (b-a)^2 > \frac{1}{2} (b-c)^2 + \frac{1}{2} (c-a)^2 \\
  \text{ so then } \left| \int_a^b f(x) dx - (b-a)f(c) \right| \leq \frac{ (b-a)^2}{2} 
\end{gathered}
\]
\end{enumerate}

%-----------------------------------%-----------------------------------%-----------------------------------
\subsection*{ 3.8 Exercises - Composite functions and continuity }
%-----------------------------------%-----------------------------------%-----------------------------------
\quad \\
\exercisehead{1} $f(x) = x^2 - 2x$; $g(x) = x+1$ 
\[
f(g(x)) = (x+1)^2 - 2(x+1) = x^2 - 1 
\]
$h: \mathbb{R} \to \mathbb{R}$

\exercisehead{2} $f(x) = x+1$; \quad $g(x) = x^2 - 2x$
\[
 h = x^2 - 2x + 1 
\]
$h:\mathbb{R} \to \mathbb{R}$

\exercisehead{3} $f(x) = \sqrt{x}$ if $x\geq 0$, $g(x) = x^2$ 
\[
h(x) = |x|
\]
$h: \mathbb{R} \to \mathbb{R}^+$

\exercisehead{4} $f(x) = \sqrt{x}$ if $x\geq 0$ $g(x) = -x^2$ \\
$h$ undefined.  $D(h) = \emptyset$

\exercisehead{5} $f(x) = x^2$; $g(x) = \sqrt{x}$ if $x\geq 0$
\[
h(x) = x \text{ if } x \geq 0 
\]
$h:\mathbb{R}^+ \to \mathbb{R}^+$

\exercisehead{6} $f(x) = -x^2$; $g(x) = \sqrt{x}$ if $x\geq 0$ 
\[
 h = -x \text{ if } x \geq 0
\]
$h: \mathbb{R}^+ \to \mathbb{R}^-$

\exercisehead{7} $f(x) = \sin{x}$; \quad $g(x) = \sqrt{x}$ if $x\geq 0$
\[
h(x) = \sin{\sqrt{x}} \text{ if } x \geq 0
\]
$h: \mathbb{R}^+ \to [-1,1]$

\exercisehead{8} $f(x) = \sqrt{x}$ if $x\geq 0$, $g(x) = \sin{x}$ 
\[
h(x) = \sqrt{ \sin{x}} \text{ if } 2 \pi j \leq x \leq 2\pi j + \pi; \, j \in \mathbb{Z}^+
\]
$h: \{ x | 2\pi j \leq x \leq 2 \pi j + \pi; \, j \in \mathbb{Z}^+ \} \to [0,1]$

\exercisehead{9} $f(x) = \sqrt{x}$ if $x \geq 0$, $g(x) = x + \sqrt{x}$ if $x>0$ 
\[
h(x) = \sqrt{ x + \sqrt{x}}
\]
$h: \{ x > 0 \} \to \{ x > 0 \}$

\exercisehead{10} $f(x)  =\sqrt{ x + \sqrt{x}}$ if $x>0$; $g(x) = x + \sqrt{x}$ if $x>0$ 
\[
h(x) = \sqrt{ x + \sqrt{x} + \sqrt{ x + \sqrt{x}} }
\]
$h: \{ x >0 \} \to \{ x >0 \}$

\exercisehead{11} $\lim_{x \to -2} \frac{x^3 + 8 }{ x^2  - 4} = \lim_{x\to -2} \frac{ (x+2) (x^2 - 2x +4) }{ (x+2)(x-1) } = \boxed{ -3 }$ \\
(used $lim f g = AB$ and $\lim \frac{f}{g} = \frac{A}{B}$)

\exercisehead{12} $\lim_{x\to 4} \sqrt{ 1 + \sqrt{x}} = \sqrt{3}$.  Used continuity of composite function theorem.  

\exercisehead{13} $\lim_{t\to 0} \frac{\sin{\tan{t}}}{\sin{t}} = \lim_{t\to 0} \frac{\sin{(\tan{t}) }}{ \tan{t} } \left( \frac{1}{\cos{t}  }\right) = \lim_{t\to 0} \frac{ \sin{\tan{t}}}{\tan{t}} \lim_{t\to 0} \frac{1}{\cos{t}} = 1$.  \\
( we used $lim fg = AB$ in the last step; $\lim_{t \to 0} \frac{\sin{(\tan{t})}}{\tan{t}}$ is from continuity of composite functionsl this had been a composite function $f \circ g$ of $f = \sin{x}$ and $g=\tan{x}$).  

\exercisehead{14} $\lim_{x\to \pi/2} \frac{\sin{\cos{x}}}{\cos{x}} = 1$ (by continuity of composite functions; this was a composite function $f \circ g$ of $f=\sin{x}$ and $g=\cos{x}$).  

\exercisehead{15} $\lim_{t\to \pi} \frac{ \sin{(t-\pi)}}{ t- \pi} = 1$ (by continuity of composite functions; this was a composite function $f \circ g$ of $f=\sin{x}$ and $g = t - \pi$).  

\exercisehead{16} $\lim_{x\to 1} \frac{ \sin{(x^2 - 1)}}{ x-1} = \lim_{x\to 1} \frac{ \sin{(x^2 - 1 )}}{x^2 - 1 } (x+1) = 2 $ (we used continuity of composite functions and product of limits.  

\exercisehead{17} Since $0 < x \sin{\frac{1}{x}} < x$; by squeeze principle, $\lim_{x\to 0} x \sin{\frac{1}{x}} =0$

\exercisehead{18} $\lim_{x\to 0} \frac{1- \cos{2x}}{x^2} = \lim_{x\to 0} \frac{2 \sin^2{x}}{x^2} = 2$

\exercisehead{19} $\lim_{x\to 0} \frac{ \sqrt{1+ x} - \sqrt{1-x}}{x} \left( \frac{ \sqrt{ 1+x} + \sqrt{ 1- x}}{ \sqrt{ 1+x} + \sqrt{ 1- x} } \right) = \boxed{ 1} $

\exercisehead{20} $\lim_{ x\to 0} \frac{1 - \sqrt{ 1 - 4x^2}}{ x^2 } \left( \frac{ 1 + \sqrt{ 1-  4x^2}}{ 1+ \sqrt{ 1-4x^2}} \right) =\boxed{ 2}$ 

\exercisehead{21} $f(x) = \frac{ x+|x|}{2}$ \quad $g(x) = \begin{cases} x & \text{ for } x < 0 \\ x^2 & \text{ for } x \geq 0 \end{cases}$ 
\[
h(x) = \begin{cases} x^2 & x \geq 0 \\ 0 & x <0 \end{cases} 
\]
$h(x)$ continuous everywhere.  

\exercisehead{22} $f(x) = \begin{cases} 1 & \text{ if } |x| \leq 1 \\
0 & \text{ if } |x| >1 \end{cases} $ \quad \quad $g(x) = \begin{cases} 2 - x^2 & \text{ if } |x| \leq 2 \\
2 & \text{ if } |x| >2 \end{cases}$ 
\[
h(x) = \begin{cases} 0 & \text{ if } |x| \leq 1, \sqrt{3} \leq |x| \\
  1 & \text{ if } 1 \leq |x| \leq \sqrt{3} 
\end{cases}
\]
$h$ cont. for $|x| <1$, $\sqrt{3} < |x|$, $1 < |x| <\sqrt{3}$

\exercisehead{23} $h(x) = g(f(x))$
\[
h(x) = \begin{cases} 0 & \text{ for } x < 0 \\ x^2 \text{ for } x \geq 0 \end{cases} 
\]
$h(x)$ continuous everywhere.  

%-----------------------------------%-----------------------------------%-----------------------------------
\subsection*{ 3.11 Exercises - Bolzano's theorem for continuous functions, The intermediate-value theorem for continuous functions }
%-----------------------------------%-----------------------------------%-----------------------------------

These theorems form the foundation for continuity and will be valuable for differentiation later.  

\begin{theorem}[Bolzano's Theorem]  \quad \\
Let $f$ be cont. at $\forall x \in [a,b]$.  \\
Assume $f(a), \, f(b)$ have opposite signs.   \\
Then $\exists$ at least one $c \in (a,b)$ s.t. $f(c) =0$.  
\end{theorem} 

\begin{proof}
  Let $f(a) < 0 , \, f(b) > 0 $.  \\
  Want: Fine one value $c \in (a,b)$ s.t. $f(c)=0$ \\
  Strategy: find the largest $c$.  \\
  Let $S = \{ \text{ all $x \in [a,b]$ s.t. $f(x) \leq 0$ } \}$. \\
$S$ is nonempty since $f(a) < 0$.  $S$ is bounded since all $S \subseteq [a,b]$.  \\
  $\Longrightarrow S$ has a suprenum.  \\
Let $c = sup S$.  

If $f(c) > 0, \, \exists (c-\delta, c+\delta)$ s.t. $f>0$ \\
\phantom{ If } $c-\delta$ is an upper bound on $S$ \\
\phantom{ If c-d } but $c$ is a least upper bound on $S$.  Contradiction.  

If $f(c) < 0 , \, \exists (c-\delta, c+ \delta)$ s.t. $f<0$ \\
\phantom{ if } $c + \delta$ is an upper bound on $S$ \\
\phantom{ if c-d } but $c$ is an upper bound on $S$.  Contradiction.  
\end{proof}

\begin{theorem}[Sign-preserving Property of Continuous functions] \quad \\
Let $f$ be cont. at $c$ and suppose that $f(c) \neq 0$.  \\\
\quad \quad then $\exists (c-\delta, c+\delta)$ s.t. $f$ be on $(c-\delta, c+\delta)$ has the same sign as $f(c)$.  
\end{theorem}

\begin{proof}
  Suppose $f(c) >0$.   \\
  $\forall \epsilon > 0, \, \exists \delta > 0 $ s.t. $f(c) - \epsilon < f(x) < f(c) + \epsilon$ if $c-\delta < x < c+\delta$ (by continuity).  \\
  Choose $\delta$ for $\epsilon = \frac{f(c)}{2}$.  Then 
\[
\frac{f(c)}{2} < f(x) < \frac{3f(c)}{2} \quad \, \forall x \in (c-\delta, c+\delta)
\]
Then $f$ has the same sign as $f(c)$.  
\end{proof}

\begin{theorem}[Intermediate value theorem] \quad \\
Let $f$ be cont. at each pt. on $[a,b]$.  \\
Choose any $x_1, x_2 \in [a,b]$ s.t. $x_1 < x_2$.  s.t. $f(x_1) \neq f(x_2)$.  \\
Then $f$ takes on every value between $f(x_1)$ and $f(x_2)$ somewhere in $(x_1,x_2)$.  
\end{theorem}

\begin{proof}
  Suppose $f(x_1) < f(x_2)$  \\
  \phantom{ Suppose } Let $k$ be any value between $f(x_1)$ and $f(x_2)$ 

Let $g = f-k$ 
\[
\begin{aligned}
  g(x_1) & = f(x_1) - k < 0 \\
  g(x_2) & = f(x_2) - k > 0 
\end{aligned}
\]
By Bolzano, $\exists c \in (x_1,x_2)$ s.t. $g(c) = 0 \quad \, \Longrightarrow f(c) = k$
\end{proof}


\exercisehead{1} $f(0) = c_0$.  $f(0) \gtrless 0$.  \medskip \\
Since $\lim_{x\to \infty} \frac{ c_k x^k }{ c_{k-1} x^{k-1}} = \lim_{x \to \infty} \frac{ c_k}{c_{k-1}} x = \infty$  $\exists M > 0 $ such that $|c_n M^n | > | \sum_{k=0}^{n-1} c_k M^k $.  So then
\[
f(M) = c_n M^n + \sum_{k=0}^{n-1} c_k M^k \lessgtr c_n 
\] 
By Bolzano's theorem $\exists b \in (0, M)$ such that $f(b)=0$.  

\exercisehead{2} Try alot of values systematically.  I also cheated by taking the derivatives and feeling out where the function changed direction.  
\begin{enumerate}
\item If $P(x) = 3x^4 -2x^3 - 36 x^2 +36x -8$, $P(-4) = 168, P(-3) = -143$, $P(0) = -8, P(\frac{1}{2}) = \frac{15}{16}$, $P(1) = -7$, $P(-3) = -35, P(4) = 200$ 
\item If $P(x) = 2x^4 -14x^2 +14x -1$, $P(-4)=231, P(-3)=-7$, $P(0)=-1, P(\frac{1}{2}) = \frac{1}{8}$, $P(\frac{3}{2}) = -\frac{11}{8}, P(2) =2$
\item If $P(x) = x^4 + 4x^3 + x^2 - 6x +2$, $P(-3) = 2, P(-\frac{5}{2}) = -\frac{3}{16}, P(-2) =2$, $P(\frac{1}{3}) = \frac{22}{81}, P(\frac{1}{2}) = -\frac{3}{16}, P(\frac{2}{3}) = -\frac{14}{81}, P(1) = 2$.  
\end{enumerate}

\exercisehead{3}.  Consider $f(x) = x^{2j+1} - a$.  $f(0) = -a >0$.  \medskip \\
Since $a$ is a constant, choose $M<0$ such that $M^{2j+1} - a < 0$.  $f(M) < 0$.  

By Bolzano's theorem, there is at least one $b \in (M,0)$ such that $f(b) = b^{2j+1} - a = 0$.  \medskip \\
Since $x^{2j+1} - a$ is monotonically increasing, there is exactly one $b$.  

\exercisehead{4} $\tan{x} $ is not continuous at $x=\pi/2$. 

\exercisehead{5} Consider $g(x) = f(x) - x$.  Then $g(x)$ is continuous on $[0,1]$ since $f$ is.  

Since $0 \leq f(x) \leq 1$ for each $x \in [0,1]$, consider $g(1) = f(1) -1$, so that $-1 \leq g(1) \leq 0$.  Likewise $0 \leq g(0) \leq 1$.  

If $g(1) = 0$ or $g(0) =0$, we're done ($g(0) = f(0) - 0 = 0$.  $f(0)=0$. Or $g(1) = f(1) -1 = 0$, $f(1) =1$ ).  \medskip \\
\phantom{ If } Otherwise, if $-1 \leq g(1) < 0 $ and $0<g(0) \leq 1$, then by Bolzano's theorem, $\exists$ at least one $c$ such that \\
\phantom{ If } $g(c) = 0$ ($g(c) = f(c) -c =0$.  $f(c) = c$).  \\

\exercisehead{6} Given $f(a) \leq a$, $f(b) \geq b$, \medskip \\
Consider $g(x) = f(x) -x \leq 0$.  Then $g(a) = f(a) -a \leq 0 $, $g(b) = f(b) -b \geq 0$.  

Since $f$ is continuous on $[a,b]$ (so is $g$) and since $g(a), g(b)$ are of opposite signs, by Bolzano's theorem, $\exists$ at least one $c$ such that $g(c) =0$, so that $f(c) =c$.  

%-----------------------------------%-----------------------------------%-----------------------------------
\subsection*{ 3.15 Exercises - The process of inversion, Properties of functions preserved by inversion, Inverses of piecewise monotonic functions }
%-----------------------------------%-----------------------------------%-----------------------------------

\exercisehead{1} $D = \mathbb{R}$, $g(y) = y-1$

\exercisehead{2} $ D= \mathbb{R}$, $g(y) = \frac{1}{2} (y-5)$

\exercisehead{3} $ D = \mathbb{R}$, $g(y) = 1-y$

\exercisehead{4} $D = \mathbb{R}$, $g(y) = y^{1/3}$

\exercisehead{5} $D= \mathbb{R}$, 
\[
g(y) = \begin{cases} 
  y & \text{ if } y < 1 \\
  \sqrt{y} & \text{ if } 1 \leq y \leq 16 \\
  \left( \frac{y}{8} \right)^2 & \text{ if } y > 16 
  \end{cases}
\]

\exercisehead{6} $f(M_f) = f(f^{-1} \left( \frac{1}{n} \sum_{i=1}^n f(a_i) \right) ) = \frac{1}{n} \sum_{i=1}^n f(a_i)$

\exercisehead{7} $f(a_1) \lessgtr \frac{1}{n} \sum_{i=1}^n f(a_i) \lessgtr f(a_n)$.  Since $f$ is strictly monotonic.  

$g$ preserves monotonicity.  
\[
\Longrightarrow a_1 \lessgtr M_f \lessgtr a_n
\]

\exercisehead{8} $h(x) = a f(x) + b, \, a \neq 0$
\[
M_h = H\left( \frac{1}{n} \sum_{i=1}^n h(a_i) \right) = H \left( \frac{1}{n} \sum_{i=1}^n \left( a f(a_i) + b\right) \right) = H \left( a \frac{1}{n} \sum_{i=1}^n f(a_i) + b \right)
\]

The inverse for $h$ is $g\left( \frac{h-b}{a} \right) = H(h) = h^{-1}$.  So then 
\[
M_h = g\left( \frac{1}{n} \sum_{i=1}^n f(a_i) \right) = M_f
\]
The average is invariant under translation and expansion in ordinate values.  



%-----------------------------------%-----------------------------------%-----------------------------------
\subsection*{ 3.20 Exercises - The extreme-value theorem for continuous functions, The small-span theorem for continuous functions (uniform continuity), The integrability theorem for continuous functions }
%-----------------------------------%-----------------------------------%-----------------------------------
\quad \\

Since for $c \in [a,b]$, $m = \min_{x\in [a,b]} f \leq f(c) \leq \max_{x \in [a,b]} f = M$ \medskip \\
\quad \quad and $ \frac{ \int_a^b f(x) g(x) dx }{ \int_a^b g(x) dx } = f(c)$

\exercisehead{1}
\[
\begin{gathered}
  g = x^9 > 0 \text{ for } x \in [0,1]; f = \frac{1}{\sqrt{1+x}} \, m = \frac{1}{\sqrt{2}}, M = 1 \\
  \int_0^1 x^9 = \left. \frac{1}{10} x^{10} \right|_0^1 = \frac{1}{10} \\
  \frac{1}{10 \sqrt{2}} \leq \int_0^1 \frac{ x^9}{ \sqrt{1+x}} dx \leq \frac{1}{10}
\end{gathered}
\]

\exercisehead{2} 
\[
\begin{gathered}
  \sqrt{1-x^2} = \frac{ 1-x^2}{ \sqrt{1-x^2}}. \, f = \frac{1}{\sqrt{1-x^2}} \, g = (1-x^2) \, M = \frac{2}{\sqrt{3}}, m =1 \\
  \int_0^{1/2} (1-x^2) dx = \left. (x- \frac{1}{3} x^3 )\right|_0^{1/2} = \frac{11}{24} \\
  \frac{11}{24} \leq \int_0^{1/2} \sqrt{1-x^2} dx \leq \frac{11}{24} \sqrt{ \frac{4}{3} }
\end{gathered}
\]

\exercisehead{3} 
\[
\begin{gathered}
  f = \frac{1}{1+x^6} \, g = 1-x^2 + x^4 \, \int_0^a 1-x^2 +x^4 = \left. \left( x - \frac{1}{3}x^3 + \frac{1}{5} x^5 \right) \right|_0^a = a - \frac{a^3}{3} + \frac{a^5}{5} \\
  m = \frac{1}{1+a^6} \, M = 1 \\
  \frac{1}{1+a^6} \left( a- \frac{a^3}{3} + \frac{a^5}{5} \right) \leq \int_0^a \frac{1}{1+x^2} dx \leq \left( a - \frac{a^3}{3} + \frac{a^5}{5} \right) 
\end{gathered}
\]

So if $a=\frac{1}{10}$, $(a-a^3/3 + a^5/5) = a - 0.333 \dots a^3 + 0.2 a^5 = 0.099669$

\exercisehead{4} (b) is wrong, since it had chosen $g =\sin{t}$, but $g$ needed to be nonnegative.  

\exercisehead{5} At worst, we could have utilized the fundamental theorem of calculus.  
\[
\begin{aligned}
  \int \sin{t^2} dt & = \int \left( \frac{1}{2t} \right) (2t \sin{t^2}) dt = \frac{1}{2c} \left. (-\cos{t^2}) \right|_{\sqrt{n\pi}}^{\sqrt{(n+1)\pi }} = \\
  & = \frac{-1}{2c} ( (-1)^{n+1} - (-1)^n ) = \boxed{ \frac{1}{c} (-1)^n }
\end{aligned}
\]

\exercisehead{6} 
$\int_a^b (f)(1) = f(c) \int_a^b 1 = f(c) (b-a)$.  Then $f(c) = \frac{ \int_a^b f}{ b-a} = 0$ for some $c \in [a,b]$ by Mean-value theorem for integrals.  

\exercisehead{7} $f$ nonnegative.  Consider $f$ at a point of continuity $c$, and suppose $f(c) > 0$.  Then $\frac{1}{2} f(c) > 0$.  
\[
\begin{gathered}
  |f(x) -f(c) | < \epsilon \Longrightarrow f(c) - \epsilon < f(x) < f(c) + \epsilon  \\
  \text{ Let } \epsilon = \frac{1}{2} f(c) \, \exists \delta > 0 \text{ for } \epsilon = \frac{1}{2} f(c) \\
  \int_{c-\delta}^{c+\delta} f(x) dx > \frac{1}{2} f(c) (2\delta ) = f(c) \delta > 0 \\
\end{gathered}
\]
But $\int_a^b f(x) dx = 0 $ and $f$ is nonnegative.  $f(c) = 0$.  

\exercisehead{8} 
\[
\begin{gathered}
  m \int g \leq \int fg \leq M \int g \Longrightarrow m \int g \leq 0 \leq M \int g \, \forall g \\
  m \leq 0 \leq M \text{ for } \int g = 1 \text{ but also } \\
  -m \leq 0 \leq -M \Longrightarrow m \geq 0 \, M \leq 0 \text{ for } \int g = -1
\end{gathered}
\]
So because of this contradiction, $m=M=0$.  By intermediate value theorem, $f=0, \, \forall x \in [a,b]$.  


%-----------------------------------%-----------------------------------%-----------------------------------
\subsection*{ 4.6 Exercises - Historical introduction, A problem involving velocity, The derivative of a function, Examples of derivatives, The algebra of derivatives}
%-----------------------------------%-----------------------------------%-----------------------------------
\quad \\
\exercisehead{1} $f' =1-2x$, $f'(0) =1, f'(1/2) = 0, f'(1) =-1, f'(10) = -19$

\exercisehead{2} $f'=x^2 + x-2$
\begin{enumerate}
\item $f'=0, x=1,-2$
\item $f'(x) =-2, x=0,-1$
\item $f'=10, \, x=-4, 3$
\end{enumerate}

\exercisehead{3} $f'= 2x+3$

\exercisehead{4} $f'= 4 x^3+ \cos{x}$

\exercisehead{5} $ f' = 4 x^3 \sin{x} + x^4 \cos{x}$

\exercisehead{6} $ f' = \frac{ -1}{ (x+1)^2 }$

\exercisehead{7} $f' = \frac{-1}{(x^2 +1)^2} (2x) + 5x^4 \cos{x} + x^5 (-\sin{x})$

\exercisehead{8} $f' = \frac{x-1 - (x) }{ (x-1)^2 } = \frac{-1}{(x-1)^2 }$

\exercisehead{9} $f' = \frac{-1}{ (2+\cos{x})^2 } (-\sin{x}) = \frac{\sin{x}}{ (2+\cos{x})^2 }$

\exercisehead{10} 
\[
\frac{ (2x+3) (x^4 +x^2 +1 ) - (4x^3 + 2x) (x^2 + 3x + 2)}{ (x^4 + x^2 +1)^2 } = \frac{ -2x^5 -9x^4 + 12 x^3 - 3x^2 - 2x + 3 }{ (x^4 + x^2 +1)^2 }
\]

\exercisehead{11} 
\[
f' = \frac{ (-\cos{x} )(2 - \cos{x}) - (\sin{x})(2-\sin{x}) }{ (2 - \cos{x} )^2 } = \frac{ -2 \cos{x} - 2 \sin{x} +1 }{ (2-\cos{x})^2 }
\]

\exercisehead{12} \[
f' = \frac{ (\sin{x} + x \cos{x} )(1+x^2) -2x (x\sin{x})}{ (1+x^2 )^2 } = \frac{ \sin{x} + x \cos{x} + x^3 \cos{x} - x^2 \sin{x} }{ (1+x^2)^2 }
\]

\exercisehead{13} \begin{enumerate}
\item \[
\begin{aligned}
  & \frac{ f(t+h) -f(t)}{ h} = \frac{ v_0h - 32 th - 16h^2}{ h } = v_0 +32 t - 16 h  \\
  & f'(t) = v_0 -32 t
\end{aligned}
\]
\item $t = \frac{v_0}{32}$
\item $-v_0$
\item $T = \frac{v_0}{16}, \, v_0 = 16 \text{ for } 1 sec. \, v_0 = 160 \text{ for } 10 sec. \, \frac{ v_0}{16} \text{ for } T sec. $
\item $f'' = -32$
\item $h = -20 t^2 $
\end{enumerate}

\exercisehead{14} $V = s^3, \, \frac{dV}{dS} = 3s^2 $

\exercisehead{15} \begin{enumerate}
\item  $\frac{dA}{dr} = 2\pi r = C$ 
\item $\frac{dV}{dr} = 4\pi r^2 = A$
\end{enumerate}

\exercisehead{16} $f'=\frac{1}{ 2\sqrt{x} }$  

\exercisehead{17} $f' = \frac{ -1}{(1+\sqrt{x})^2 } \left( \frac{ 1}{ 2 \sqrt{x} } \right)$

\exercisehead{18} $f' = \frac{3}{2} x^{1/2}$

\exercisehead{19} $\frac{-3}{2} x^{-5/2}$

\exercisehead{20} $f' = \frac{1}{2} x^{-1/2} + \frac{1}{3} x^{-2/3} + \frac{1}{4} x^{-3/4} \, x > 0$

\exercisehead{21} $f' = - \frac{1}{2} x^{-3/2} + -\frac{1}{3} x^{-4/3} -\frac{1}{4} x^{-5/4}$

\exercisehead{22} $f' = \frac{ \frac{1}{2} x^{-1/2} (1+x) - \sqrt{x} }{ (1+x)^2} = \frac{1}{ 2 \sqrt{x} (1+x)^2 }$

\exercisehead{23} $f' = \frac{ (1+ \sqrt{x}) - x \frac{1}{2} \frac{1}{ \sqrt{x}} }{ (1+\sqrt{x})^2} = \frac{ 1+ \frac{1}{2} \sqrt{x} }{ (1+\sqrt{x})^2 }$

\exercisehead{24} 
\[
\begin{gathered}
\begin{aligned}
  & g = f_1 f_2 \\
  & g' = f_1' f_2 + f_1 f_2' \, \frac{g'}{g} = \frac{f_1'}{f_1} + \frac{f_2'}{f_2}
\end{aligned} \\
\begin{aligned}
  g = f_1 f_2 \dots f_n f_{n+1} & \\
  & g' = (f_1 f_2 \dots f_n )' f_{n+1} + (f_1 f_2 \dots f_n) f_{n+1}'; \\
  & \frac{ g'}{g} = \frac{ (f_1 f_2 \dots f_n)' }{ f_1 f_2 \dots f_n } + \frac{ f_{n+1}'}{ f_{n+1}} \\
  & \quad = \frac{ f_1'}{f_1} + \frac{ f_2'}{ f_2 } + \dots + \frac{ f_n'}{f_n} + \frac{ f_{n+1}' }{ f_{n+1} }
\end{aligned}
\end{gathered}
\]

\exercisehead{25}
\[
\begin{aligned}
  & (\tan{x})' = \left( \frac{\cos{x}}{\sin{x}} \right)' = \frac{ \cos^2{x} - (-\sin{x}) \sin{x} }{ \cos^2{x} } = \sec^2{x} \\
  & (\cot{x})' = \left( \frac{ \cos{x}}{ \sin{x} } \right)' = \frac{ -\sin{x} \sin{x} - \cos{x} \cos{x} }{ \sin^2{x} } = -\csc^2{x} \\
  & (\sec{x})' = \frac{ -1}{\cos^2{x}} (-\sin{x}) = \tan{x} \sec{x} \\
  & (\csc{x})' = \frac{ -1}{ \sin^2{x} } \cos{x} = -\cot{x} \csc{x}
\end{aligned}
\]

\exercisehead{35} 
\[
\begin{aligned}
f' & = \frac{ (2ax+b) (\sin{x}+\cos{x}) - (\cos{x} - \sin{x}) (ax^2 + bx +c) }{ (\sin{x} + \cos{x})^2 } = \\ 
& = \frac{ (2ax + b)( \sin{x} +\cos{x}) - (\cos{x} -\sin{x})(ax^2 + bx +c ) }{ (\sin{x} + \cos{x})^2 }
\end{aligned}
\]

\exercisehead{36} 
\[
f' = a \sin{x} + (ax + b) \cos{x} + c \cos{x} + (cx+d) (-\sin{x}) = ax \cos{x} + (b+c) \cos{x} + (a-d) \sin{x} -cx \sin{x} 
\]
So then $a=1, d=1, b=d, c=0$.  

\exercisehead{37} 
\[
\begin{aligned}
  g' & = (2ax+b)\sin{x} + (ax^2 +bx +c)\cos{x} + (2dx +e )\cos{x} + (dx^2 + ex +f) (-\sin{x}) = \\
  & = ax^2 \cos{x} - dx^2 \sin{x} + (2a -e) x \sin{x} + (b+2d) x \cos{x} + (b-f)\sin{x} + (c+e)\cos{x} 
\end{aligned}
\]

$g = x^2 \sin{x}$.  So $d=-1, b=2, f=2, a=0, e=0, c=0$.  

\exercisehead{38} $1 + x + x^2 + \dots + x^n = \frac{ x^{n+1} -1 }{ x-1}$
\begin{enumerate}
\item
 \[
\begin{gathered}
  \begin{aligned}
    (1+x +x^2 + \dots + x^n)' & = 1 + 2x + \dots + nx^{n-1} = \frac{ (n+1) x^n (x-1) - (1)(x^{n+1} -1 ) }{ (x-1)^2 } \\
    & = \frac{ (n+1)(x^{n+1} - x^n ) - x^{n+1} +1 }{ (x-1)^2 } = \frac{ nx^{n+1} - (n_1)x^n +1 }{ (x-1)^2 } 
  \end{aligned} \\
  x(1+2x + \dots + nx^{n-1}) = x+2x^2 + \dots + nx^n = \frac{ nx^{n+2} - (n+1)x^{n+1} +x }{ (x-1)^2 } \\
\end{gathered}
\]
\item 
  \[
  \begin{gathered}
    \begin{aligned}
      (x+2x^2 + \dots + nx^n)' & = (1+ 2^2 x^1 + \dots + n^2 x^{n-1} ) = \\
      & = \frac{ (n(n+2)x^{n+1} - (n+1)^2 x^n +1 ) (x-1)^2 -2 (x-1)(nx^{n+2} - (n+1)x^{n+1} + x ) }{ (x-1)^4 } 
    \end{aligned}
     \\
    \begin{aligned}
      x+2^2x^2 + \dots + n^2 x^n & = \frac{ (n(n+2)x^{n+2} - (n+1)^2 x^{n+1} + x)(x-1) -2 (nx^{n+3} - (n+1)x^{n+2} +x^2)}{ (x-1)^3} = \\ 
      & = \frac{ n^2 x^{n+3} + (-2n^2 - 2n+1)x^{n+2} + (n+1)^2 x^{n+1} - x^2 -x }{ (x-1)^3}
    \end{aligned}
  \end{gathered}
  \]
\end{enumerate}

\exercisehead{39} 
\[
\begin{aligned}
  & \frac{ f(x+h) - f(x)}{h } = \frac{ (x+h)^n -x^n}{ h} \\
  & (x+h)^n = \sum_{j=0}^n \binom{n}{j} x^{n-j} h^j \\
  & \frac{ (x+h)^n -x^n}{h} = \frac{ \sum_{j=1}^n \binom{n}{j} x^{n-j} h^j }{ h } = \sum_{j=1}^n x^{n-j} h^{j-1} \binom{n}{j} \\
  & \lim_{h\to 0 } \frac{ (x+h)^n -x^n}{ h} = \binom{n}{1} x^{n-1} = \boxed{ nx^{n-1}}
\end{aligned}
\]

%-----------------------------------%-----------------------------------%-----------------------------------
\subsection*{ 4.9 Exercises - Geometric interpretation of the derivative as a slope, Other notations for derivatives }
%-----------------------------------%-----------------------------------%-----------------------------------
\quad \\
\exercisehead{6} 
\begin{enumerate}
\item 
  \[
  \begin{aligned}
    & f = x^2 + ax +b \\ 
    & f(x_1) = x_1^2 + ax_1 +b \\
    & f(x_2) = x_2^2 + ax_2 +b
  \end{aligned}
  \quad 
  \begin{aligned}
    \frac{ f(x_2) - f(x_1)}{ x_2-x_1} & = \frac{ x_2^2 -x_1^2 + a(x_2 -x_1) }{ x_2 - x_1} \\
    & = \boxed{ x_2 + x_1 +a }
  \end{aligned}
  \]
\item \[
\begin{gathered}
  f' = 2x + a \\
  m = x_2 + x_1 +a = 2x + a \, \boxed{ x = \frac{ x_2 +x_1}{2} }
\end{gathered}
\]
\end{enumerate}

\exercisehead{7} The line $y=-x$ as slope $-1$.  \[
\begin{gathered}
  y=x^3 - 6x^2 + 8x \, y' = 3 x^2 - 12 x + 9 \\
  3x^2 - 12 x + 8 = -1 \Longrightarrow x = 3,1
\end{gathered}
\]
The line and the curve meet under the condition
\[
-x = x^3 -6x^2 + 8x \Longrightarrow x= 3; f(3) =-3
\]
At $x=0$, the line and the curve also meet. 

\exercisehead{8} $f = x(1-x^2)$.  $f'=1-3x^2$.  
\[
  f'(-1) = -2 \Longrightarrow \boxed{ y = -2x -2 } \\
\]

For the other line, 
\[
\begin{gathered}
  f'(a) = 1-3a^2 \\
  \Longrightarrow y(-1) = 0 = (1-3a^2)(-1)+b \,  \Longrightarrow b = 1-3a^2
\end{gathered}
\]

Now $f(a) = a(1-a^2) = a-a^3$ at this point.  The line and the curve must meet at this point.  
\[
\begin{gathered}
\begin{aligned}
  y(a) & = (1-3a^2)a + (1-3a^2) = \\
  & = a- 3a^3 + 1 -3a^2 = a-a^3 
\end{aligned} \\
\Longrightarrow -2a^3 + 1 -3a^2 = 0 = a^3 - \frac{1}{2} + \frac{3}{2} a^2 
\end{gathered}
\]

The answer could probably be guessed at, but let's review some tricks for solving cubics.  

First, do a translation in the $x$ direction to center the origin on the point of inflection.  Find the point of inflection by taking the second derivative.  
\[
\begin{gathered}
  f'' = 6a +3 \, \Longrightarrow a= -\frac{1}{2} \\
  \text{ So } \\
  a = x - \frac{1}{2}  \\
  \Longrightarrow (x- \frac{1}{2})^3 + \frac{3}{2} (x-\frac{1}{2} )^2 - \frac{1}{2} = x^3 = \frac{3}{4}x - \frac{1}{4} = 0 
\end{gathered}
\]

Then recall this neat trigonometric fact:
\[
\begin{gathered}
  \cos{3x} = \cos{2x} \cos{x} - \sin{2x} \sin{x} = 4 \cos^3{x} -3 \cos{x} \\
  \Longrightarrow \cos^3{x} = \frac{3}{4} \cos{x} - \frac{ \cos{3x}}{ 4} = 0 
\end{gathered}
\]

Particularly for this problem, we have $\cos{3x} =1$.  So $x = 0, 2\pi/3, 4\pi/3$.  $\cos{x} = 1, -\frac{1}{2}$.  Plugging $\cos{x} \to x$ back into what we have for $a$, $a=-1$, which we already have in the previous part, and $a=\frac{1}{2}$.  So
\[
\begin{gathered}
  \boxed{ f\left( \frac{1}{2} \right) = \frac{3}{8} } \\
\boxed{   y(x) = \left( \frac{1}{4} x \right) + \frac{1}{4} }
\end{gathered}
\]

\exercisehead{9} 
\[
\begin{aligned}
  & f(x) = \begin{cases} x^2 & \text{ if } x \leq c \\
    ax+b & \text{ if } x > c \end{cases} \\
  & f'(x) = \begin{cases} 2x & \text{ if } x \leq c \\
    a & \text{ if } x > c \end{cases} \\
& \boxed{ a =2c ; \, b=-c^2 }
\end{aligned}
\]

\exercisehead{10} 
\[
f(x) = \begin{cases} 
  \frac{1}{ |x| } & \text{ if } |x| > c \\
  a+bx^2 & \text{ if } |x| \leq c 
\end{cases}
\]
Note that $c \nless 0$ since $|x| \leq c$, for the second condition.  

\[
f'(x) = \begin{cases} -\frac{1}{x^2} & \text{ if } x > c \\
  \frac{1}{x^2} & \text{ if } x < c \\
  2bx & \text{ if } |x| \leq c 
\end{cases} 
\]
So $\boxed{ b = -\frac{1}{2c^3} , \, a= \frac{3}{2c} }$.

\exercisehead{11} \[
f'=\begin{cases} \cos{x} & \text{ if } x \leq c \\
a & \text{ if } a > c
\end{cases}
\]

\exercisehead{12} $f(x) = \left( \frac{1-\sqrt{2}}{1+\sqrt{2}} \right) = \frac{1-A}{1+A}$
\[
\begin{gathered}
  \mathbf{ A = \sqrt{x} }\, A' = a = \frac{1}{2} x^{-1/2} = \frac{1}{2A} ; A''= -\frac{1}{4} x^{-3/2} = - \frac{1}{4A^3}  \\
  \begin{aligned} 
    f' & = \frac{ -A'(1+A) - A' (1-A) }{ (1+A)^2 } = \frac{ -2A'}{(1+A)^2 } = \frac{ -1}{ \sqrt{x} (1+\sqrt{x})^2 } \\
    f'' & = \frac{1}{ (A(1+A^2))^2} (A'(1+A)^2 + A(2)(1+A)A' ) = \frac{ 3\sqrt{x} +1}{ 2 x^{3/2} (1+\sqrt{x})^3 } \\
    f''' & = \frac{1}{2} \left( \frac{ \frac{-1}{A^2}A'(A^2 (1+A)^3) - (2AA'(1+A)^3 + 3A^2(1+A)^2 A' )(3+\frac{1}{A}) }{ (A^2(1+A)^3 )^2 } \right) \\ 
    & = \frac{-3}{4} \left( \frac{ \frac{1}{A} + 4 +5A}{ A^4 (1+A^4) } \right) = -\frac{3}{4} \frac{ (1+4 \sqrt{x} + 5x )}{ \sqrt{x} (x + \sqrt{x})^4 }
  \end{aligned}
\end{gathered}
\]

\exercisehead{13} \[
\begin{gathered}
\begin{aligned}
  & P = ax^3 + bx^2 + cx + d \\
  & P' = 3ax^2 + 2bx + c \\
  & P'' = 6ax + 2b 
\end{aligned} \quad 
\begin{aligned}
  & P''(0) = 2b =10 \Longrightarrow b =5 \\
   & P'(0) =c = -1 
  & P(0) = d =-2
\end{aligned} \\
P(1) = a+5 +-1 +-2 = a +2 = -2 \Longrightarrow a =-4
\end{gathered}
\]

\exercisehead{14} 
\[
fg = 2, \, \frac{f'}{g'} = 2 \, \frac{f'}{g} =4 \, \frac{g'}{g} =2 \, f=\frac{1}{2}, \, g =4 
\]
\begin{enumerate}
\item 
\[
h' = \frac{f'g-g'f}{ g^2} = \frac{f'}{g} - \frac{g'}{g} \frac{f}{g} = 4 - 2 (\frac{1}{8} ) = \frac{15}{4} 
\]
\item \[
k' =f' g + fg' = 4 g^2 + f2g = 64 +4 = 68 
\]
\item \[
  \lim_{x\to 0} \frac{ g'(x)}{f'(x)} = \frac{ \lim_{x\to 0} g'(x)}{ \lim_{x\to 0} f'(x)} = \boxed{ \frac{1}{2} }
\]
\end{enumerate}

\exercisehead{15} 
\begin{enumerate}
\item True, by definition of $f'(a)$.  
\item \[
\lim_{h\to 0} \frac{ f(a) - f(a-h)}{ h} = - \lim_{h\to 0} \frac{f(a-h) -f(a) }{ h} = \lim_{-h\to 0} \frac{ f(a-h) -f(a) }{ -h }  = f'(a) 
\]
True, by definition of $f'(a)$. 
\item \[
\lim_{t\to 0} \frac{ f(a+2t) - f(a) }{ t } = 2 \lim_{2t\to 0} \frac{ f(a+2t) -f(a) }{ 2t} = 2f'(a) 
\]
False. 
\item \[
\begin{aligned}
  \lim_{t\to 0} \frac{ f(a+2t) - f(a) + f(a) - f(a+t) }{ 2t } & = \\
  & \lim_{2t\to 0} \frac{ f(a+2t) - f(a) }{ 2t } + - \frac{1}{2}\lim_{t\to 0} \frac{ f(a+t) -f(a)}{ t} = \\  
  & f'(a) - \frac{1}{2} f'(a) = \frac{1}{2} f'(a) 
\end{aligned}
\]
False.  
\end{enumerate}

\exercisehead{16} \begin{enumerate}
\item
\[
\begin{gathered}
\begin{aligned}
  D^* (f+g) & = \lim_{h\to 0} \frac{ (f(x+h) + g(x+h))^2 - (f(x) +g(x))^2 }{ h } = \lim_{h\to 0} \frac{ (F+G)^2 - (f+g)^2 }{ h } = \\
  & = D^* f + D^* g + \lim_{h \to 0} \frac{ 2FG - 2fg}{ h} 
\end{aligned} \\
\begin{aligned}
  \lim_{h \to 0} \frac{ 2FG - 2fg}{h} & = \lim_{h \to 0} \frac{ (2(FG) - 2fG )(F+f) }{ (F+f) h } + \frac{ (2fG - 2fg)(G+g)}{ (g+G)h } = \\
  & =   \lim_{h \to 0}  \frac{2g}{F+h} \lim_{h \to 0}  \frac{ F^2 - f^2}{h} + \lim_{h \to 0}  \frac{2f }{ G+g} \lim_{h \to 0}  \frac{ G^2 - g^2}{h} = \\
  & = \frac{g}{f} D^* f + \frac{f}{g} D^* g 
\end{aligned}
\end{gathered}
\]

\[
\begin{aligned}
  D^* (f-g) &= \lim_{h\to 0} \frac{ (f(x+h) - g(x+h))^2 - (f(x) -g(x))^2 }{ h } = \\ 
  & = \lim_{h\to 0} \frac{ (F-G)^2 - (f-g)^2 }{ h } = \\
  & = D^* f + D^* g + -\lim_{h \to 0} \frac{ 2FG - 2fg}{ h} \\
  & = D^*f + D^*g - \frac{g}{f}D^*f + \frac{f}{g}D^* g
\end{aligned} 
\]

\[
\begin{aligned}
  D^* (fg) & = \lim_{h \to 0} \frac{ ((fg)(x+h))^2 - ((fg)(x))^2 }{ h } = \\
  & = \lim_{h\to 0 } \frac{ (f^2 (x+h))(g^2(x+h)) -f^2(x) g^2(x+h) + (g^2(x+h) - g^2(x))f^2(x) }{ h} = \\
  & = g^2 D^*f + f^2 D^* g   
\end{aligned}
\]
\[
\begin{aligned}
  D^*(f/g) & = \lim_{h\to 0} \frac{ \frac{f^2(x+h)}{g^2(x+h)} - \frac{f^2(x) }{ g^2(x) } }{ h} = 
  \lim_{h\to 0} \frac{ \frac{ f^2(x+h) - f^2(x) }{ g^2(x+h)} + \frac{f^2(x) }{ g^2(x+h) } - \frac{ f^2(x) }{ g^2(x)} }{ h } = \\ 
  & = \frac{ D^* f}{g^2} + \frac{f^2}{g^4} (-D^* g ) \text{ when } g(x) \neq 0
\end{aligned}
\]
\item 
\item
\end{enumerate}

%-----------------------------------%-----------------------------------%-----------------------------------
\subsection*{ 4.12 Exercises - The chain rule for differentiating composite functions, Applications of the chain rule.  Related rates and implicit differentiation }
%-----------------------------------%-----------------------------------%-----------------------------------

\exercisehead{1} $-2 \sin{2x} - 2\cos{x} $ 

\exercisehead{2} $\frac{x}{ \sqrt{1+x^2 } } $

\exercisehead{3} $ -2 x \cos{x^2} + 2x (x^2-2) \sin{x^2} + 2 \sin{x^3} + 6 x^3 \cos{x^3 } $

\exercisehead{4} \[
\begin{aligned}
  f' & = \cos{ (\cos^2{x} )} ( -2 \cos{x} \sin{x} ) \cos{(\sin^2{x}) } + \sin{ (\cos^2{x} ) } \sin{ (\sin^2{x} ) } (2\sin{x} \cos{x} ) = \\
  & = -\sin{2x} (\cos{(\cos{2x} ) } )
\end{aligned}
\]

\exercisehead{5} \[ f' = n \sin^{n-1}{x} \cos{x} \cos{nx} + -n \sin{nx} \sin^n{x}  \]

\exercisehead{6} 
\[
f' = \cos{ (\sin{ (\sin{x} ) } ) } (\cos{(\sin{x} ) } ) (\cos{x} )
\]

\exercisehead{7}
\[
f' = \frac{ 2 \sin{x} \cos{x} \sin{x^2} - 2x \cos{x^2} \sin^2{x} }{ \sin^2{x^2} } = \frac{ \sin{2x} \sin{x^2} - 2x \sin^2{x} \cos{x^2} }{ \sin^2{x^2} } 
\]

\exercisehead{8} $f' = \frac{1}{2} \sec^2{ \frac{x}{2} } + \frac{1}{2} \csc^2{ \frac{x}{2} }$

\exercisehead{9} $f' = 2\sec^2{x} \tan{x} + - 2\csc^2{x} \cot{x}$

\exercisehead{10} $f' = \sqrt{1+x^2} + \frac{ x^2}{\sqrt{1+x^2}} = \frac{1+2x^2 }{ \sqrt{1+x^2}}$

\exercisehead{11} $f' = \frac{4}{(4-x^2)^{3/2}}$

\exercisehead{12} 
\[
f' = \frac{1}{3} \left( \frac{ 1+x^3}{ 1-x^3} \right)^{-2/3} \left( \frac{ 3x^2 (2)}{ (1-x^3)^2 } \right) = \frac{2x^2}{(1-x^3)^2 } \left( \frac{ 1+x^3}{1-x^3 } \right)^{-2/3} 
\]

\exercisehead{13} \textbf{ This exercise is important}.  It shows a neat integration trick. 
\[
\begin{gathered}
\begin{aligned}
  f(x) & = \frac{1}{ \sqrt{ 1+x^2 } (x+ \sqrt{1+x^2 } ) } =  \frac{1}{ \sqrt{ 1+x^2 } (x+ \sqrt{1+x^2 } ) } \left( \frac{ x - \sqrt{ 1 + x^2 } }{ x - \sqrt{ 1 +x^2 } } \right) = \\
  & = \frac{ x - \sqrt{1+x^2 } }{ - \sqrt{ 1 +x^2 } } = 1 - \frac{ x}{ \sqrt{ 1 +x^2 } }  \\
\end{aligned} \\
f' = \frac{ \sqrt{1+x^2 } -\frac{ x^2 }{ \sqrt{1+x^2}} }{ 1+x^2} = \frac{1}{ (1+x^2)^{3/2} }
\end{gathered}
\]

\exercisehead{14} 
\[
\frac{1}{2} ( x+ \sqrt{ x + \sqrt{x} } )^{-1/2} ( 1+ \frac{1}{2} (x + \sqrt{x} )^{-1/2} \left( 1 + \frac{1}{2 \sqrt{x} } \right) )
\]

\exercisehead{15} 
\[
f' = (2+x^2 )^{1/2}(3+x^3)^{1/3} + (1+x)x(2+x^2)^{-1/2} (3+x^2)^{1/3} + (1+x)(2+x^2)^{1/2}(3+x^3)^{-2/3} x^2
\]

\exercisehead{16} 
\[
\begin{gathered}
  f' = \frac{ -1}{ \left( 1+ \frac{1}{x} \right)^2 } \left( \frac{-1}{x^2 } \right) = \frac{1}{ (x+1)^2 } \, g' = \frac{ -1}{ \left( 1+ \frac{1}{f} \right)^2 } \left( \frac{-1}{f^2 } \right) f' = \frac{f'}{(f+1)^2 } \\
  g' = \frac{ (x+1)^{-2}}{ \left( \frac{x}{x+1} +1 \right)^2 } = \frac{1}{ (2x+1)^2 } 
\end{gathered}
\]

\exercisehead{17} $h'=f'g'$
\[
\begin{matrix}
  x & h & h' & k & k' \\
  0 & f(2) = 0 & 2(-5) = -10 & g(1) = 0 & 1(5) =5 \\
  1 & f(0) = 1 & 5(1) =5 & g(3) =1 & -6(-2) = 12 \\
  2 & f(3) = 2 & 4(1) =4 & g(0) = 2 & -5(2) = -10 \\
  3 & f(1) = 3 & -2(-6) = 12 & g(2) = 3 & 1(4) = 4 
\end{matrix} 
\]

\exercisehead{18} 
\[
\begin{gathered}
  g(x) = x f(x^2) \\
  g'(x) = f(x^2) + x(2x) f'(x^2) = f(x^2) + 2x^2 f'(x^2) \\
  g''(x) = 2x f'(x^2) + 4xf'(x^2) + 2x^2 (2x) f''(x^2) = 6x f'(x^2) + 4 x^3 f''(x^2) 
\end{gathered}
\]
\[
\begin{matrix}
  x & g(x) & g'(x) & g''(x) \\
  0 & 0 & 0 & 0 \\
  1 & 1 & 3 & 10 \\
  2 & 12 & 6 + 8(3) = 30 & 12(3) + 32(0) =36 
\end{matrix}
\]

\exercisehead{19} 
\begin{enumerate}
\item \[
g' = \frac{df(x^2)}{dx^2} 2x = 2x f' 
\]
\item \[
g' = 2\sin{x} \cos{x} f' -2\cos{x} \sin{x} f' = (\sin{2x})(f'(\sin^2{x}) - f'(\cos^2{x}) ) 
\]
\item 
\[
g' = \frac{ df(f(x))}{ d(f(x))} f' 
\]
\item 
\[
g' = \frac{df(f(f(x)))}{ d(f(f(x)))} \frac{d(f(f(x)))}{d(f(x))}\frac{df}{dx} 
\]
\end{enumerate}

\exercisehead{20} $V=s^3, \, s=s(t) \, \frac{dV}{dt} = 3s^2 \frac{ds}{dt}$.  
\[
\begin{aligned}
& s=5 cm \, 75 cm^3/sec \\
& s=10 cm \, 300 cm^3/sec \\
& s=x cm \, 3x^2 cm^3/sec
\end{aligned}
\]

\exercisehead{21} 
\[
\begin{gathered}
  l = \sqrt{x^2 + h^2 } \, \frac{dl}{dt} = \frac{1}{l} x \frac{dx}{dt} \\
  \frac{dx}{dt} = \frac{l}{x} \frac{dl}{dt} = \frac{10 mi }{ - \sqrt{10^2 - 8^2 } } \left( -4 mi/sec \right) = \frac{20}{3} \frac{mi}{sec} \left( \frac{3600sec}{1hr } \right) 
\end{gathered}
\]

\exercisehead{22} 
\[
\begin{gathered}
  l^2 = x^2 + s^2 \\
  2l \frac{dl}{dt} = 2x \frac{dx}{dt} \, \frac{dl}{dt} = \frac{x}{l} \frac{dx}{dt} \\
  \begin{aligned}
    & \frac{dl}{dt} \left( x = \frac{s}{2} \right) = 20\sqrt{5} \\
    & \frac{dl}{dt} \left( x = s \right) = 50 \sqrt{2} 
  \end{aligned}
\end{gathered}
\]

\exercisehead{23}
\[
\frac{dl}{dt} = \frac{x}{l} \frac{dx}{dt} = \frac{3}{5} 12 = \boxed{ \frac{36}{5} mi/hr }
\]

\exercisehead{24} Given the preliminary information
\[
\frac{r}{h} = \frac{2}{5} = \alpha, \, V = \frac{1}{3} \pi r^2 h = \frac{1}{3} \pi \alpha^2 h^3 
\]
\begin{enumerate}
\item \[
\begin{gathered}
  V = \frac{ \pi r^2}{ h^2} ( h^2 y -hy^2 + \frac{1}{3} y^3 ) \\
  \frac{dV}{dt} = \frac{ \pi r^2}{h^2 } ( h^2 - 2h y + y^2 ) \frac{dy}{dt} \\
  \frac{dy}{dt} = \frac{ h^2}{ \pi r^2 } \left( \frac{1}{ h^2 - 2hy + y^2 } \right) \frac{dV}{dt} =  \frac{ 10^2}{ \pi 4^2 } \left( \frac{ 1}{ 10^2 - 2(10)5 + 25 } \right) 5  = \frac{5}{4\pi}
\end{gathered}
\]
\item 
\[
\frac{dV}{dt} = \pi \alpha^2 h^2 \frac{dh}{dt}, \, \frac{dh}{dt} = \frac{1}{ \pi \alpha^2 h^2 } \frac{dV}{dt} = \frac{ 5}{4\pi}
\]
\end{enumerate}

\exercisehead{25} \[
\begin{gathered}
  \alpha = \frac{r}{h} = \frac{3}{2} \\
  \frac{dV}{dt} = \pi \alpha^2 h^2 \frac{dh}{dt} \\
  c -1 \pi \frac{9}{4} (2^2 )4 = 36 \pi \Longrightarrow \boxed{ c = 36 \pi +1 }
\end{gathered}
\]

\exercisehead{26} The constraint equation, using Pythagorean theorem on the geometry of a bottom hemisphere, is
\[
r^2 = R^2 - (R-h)^2 =  2Rh -h^2  
\]
So then 
\[
r\frac{dr}{dt} = (R-h)\frac{dh}{dt} 
\]
\[
\begin{gathered}
  V = \int \pi r^2 dh \Longrightarrow \frac{dV}{dh} = \pi r^2 = \pi (2Rh - h^2 ) \\
  \Longrightarrow \frac{dV}{dh} = \pi (2 (10(5) -25)) = 50 \pi 
\end{gathered}
\]
\[
\begin{gathered}
\frac{dV}{dt} = \frac{dV}{dh} \frac{dh}{dt} , \, \Longrightarrow \frac{dh}{dt} = \frac{dV}{dt} \left( \frac{1}{ \pi (2Rh - h^2 ) } \right) \\
\left( \frac{r}{R-h} \right) \frac{dr}{dt} = \frac{dV}{dt} \left( \frac{1}{ \pi (2 Rh - h^2 ) } \right) \\
\Longrightarrow 
\begin{aligned} \frac{dr}{dt} &= \frac{dV}{dt} \left( \frac{R-h}{ r\pi (2Rh -h^2 )} \right) = \\
  & = (5\sqrt{3}) \left( \frac{ 10-5}{ \pi (2(10)5 - 25)^{3/2} } \right) = \frac{1}{15\pi}
\end{aligned}
\end{gathered}
\]

\exercisehead{27} \textbf{ I suppose the area of the triangle is $0$ at $t = 0$. } 

Now the point on vertex $B$ moves up along the $y$ axis according to $y=1+2t$.  $y\left( \frac{7}{2} \right) = 8$.  
\[
\begin{aligned}
A & = \frac{1}{2} \sqrt{ (y-1) \frac{36}{7} } y  \\
\frac{dA}{dt} & = \frac{1}{2} \left( \sqrt{ \frac{36}{y} } \frac{1}{2} \frac{ 1}{ \sqrt{y-1} } y + \sqrt{ (y-1) \frac{36}{7} } \right) \frac{dy}{dt} = \\
& = \frac{1}{2} \left( \frac{ 6 }{ 2(7) } 8 + 6\right) (2) = \frac{66}{7}
\end{aligned}
\]

\exercisehead{28} From the given information, $h = 3r +3$.  The volume formula is $V= \frac{ \pi R^2}{3} H$.  So then
\[
\begin{aligned}
  V & = \pi/3 r^2 (3r+3) = \pi r^3 + \pi r^2 \\
  \frac{dV}{dr} & = \pi r (3r +2) \frac{dr}{dt}
\end{aligned}
\]
With the given information, we get
\[
\frac{dr}{dt} = \frac{1}{\pi (6) (20) } 
\]
Using this, we can plug this back in for the different case:
\[
\frac{dV}{dt} = n = \pi(36)(110)/(120 \pi ) = \boxed{ 33 }
\]

\exercisehead{29} \begin{enumerate}
\item $\frac{dy}{dt} = 2x \frac{dx}{dt }; \, \text{ when } x = \frac{1}{2}, y = \frac{1}{4}, \frac{dy}{dt} = \frac{dx}{dt}$ 
\item $\boxed{ t = \frac{\pi}{6} }$
\end{enumerate}

\exercisehead{30} \begin{enumerate}
\item $3x^2 +3 y^2 y' =0 \Longrightarrow x^2 + y^2 y' = 0$ 
\item \[
\begin{gathered}
  2x + 2y y'^2 + y^2 y'' = 0 \Longrightarrow y^2 y'' = -2(x + yy'^2 ) \\
  \Longrightarrow y'' = -2 \left( \frac{ xy^4 + yx^4 }{ y^6 } \right) = -2xy^{-5} 
\end{gathered}
\]
\end{enumerate}

\exercisehead{31} \[
\frac{1}{2} \frac{1}{ \sqrt{x}} + \frac{1}{2 \sqrt{y}} y' = 0 \, y' = \frac{ -\sqrt{y}}{ \sqrt{x}} < 0 
\]

\exercisehead{32} 
\[
\pm \sqrt{ \frac{ 12 - 3x^2 }{4} }
\]

\[
\begin{gathered}
\begin{aligned}
  & 6x + 8 y y' = 0 \Longrightarrow y' = \frac{ -3x}{4y } \\
  & 3 + 4 (y'^2 + y y'' ) = 0 
\end{aligned} \\
y'' = \left( -\frac{3}{4} -y'^2 \right)\frac{1}{y} = \frac{ -9}{ 4y^3 }
\end{gathered}
\]

\exercisehead{33} \[
\begin{gathered}
\sin{xy} + x \cos^2{xy} (y + xy' ) + 4x = 0 \\
y' x^2 \cos{xy} + xy \cos{xy} + \sin{xy} + 4x = 0 
\end{gathered}
\]

\exercisehead{34} $y=x^4$.  $y^n = x^m$.
\[
\begin{gathered}
  y^n = x^m, \, y' n y^{n-1} = mx^{m-1}; \, y' = \frac{ mx^{m-1}}{ ny^{n-1}} = \frac{m}{n} \frac{ x^{m-1}}{ x^{ m (1 - 1/n) }} = \\
  y' = \frac{m}{n} x^{m/n -1 }
\end{gathered}
\]

%-----------------------------------%-----------------------------------%-----------------------------------
\subsection*{ 4.15 Exercises - Applications of differentiation to extreme values of functions, The mean-value theorem for derivatives }
%-----------------------------------%-----------------------------------%-----------------------------------

Let's recap what was shown in the past two sections: 
\begin{theorem}[Theorem 4.3] \quad \\
Let $f$ be defined on $I$.  \\
Assume $f$ has a rel. extrema at an int. pt. $c \in I$.  \\
If $\exists \, f'(c), \, \, f'(c) = 0$; the converse is not true.  
\end{theorem}
\begin{proof}
$  Q(x) = \frac{ f(x) - f(c) }{ x - c} $ if $ x \neq c, \quad Q(c) = f'(c)$ \\
$\exists f'(c)$, so $Q(x) \to Q(c)$ as $x \to c$ so $Q$ is continuous at $c$. \\ 
If $Q(c) > 0$, \quad $\frac{f(x) - f(c)}{ x-c } > 0 $.  For $x - c \gtrless 0$, \quad $f(x) \gtrless f(c)$, thus contradicting the rel. max or rel. min. (no neighborhood about $c$ exists for one!) \\
If $Q(c) < 0$, \quad $\frac{f(x) - f(c)}{ x-c } < 0 $.  For $x - c \gtrless 0$, \quad $f(x) \lessgtr f(c)$, thus contradicting the rel. max or rel. min. (no neighborhood about $c$ exists for one!) 

Converse is not true: e.g. saddle points.
\end{proof}

\begin{theorem}[Rolle's Theorem] \quad \\
Let $f$ be cont. on $[a,b]$, \quad $\exists f'(x) \quad \forall x \in (a,b)$ and let 
\[
f(a)=f(b)
\]
then $\exists$ at least one $c \in (a,b)$, such that $f'(c) = 0$.  
\end{theorem}

\begin{proof}
  Suppose $f'(x) \neq 0 \quad \, \forall x \in (a,b)$.  \\
  By extreme value theorem, $\exists$ abs. max (min) $M, \, m$ somewhere on $[a,b]$.   \\
  $M,m$ on endpoints $a,b$ (Thm 4.3).  \\
  $F(a) = f(b)$, so $m = M$.  $f$ constant on $[a,b]$.  
Contradict $f'(x) \neq 0$
\end{proof}

\begin{theorem}[Mean-value theorem for Derivatives]
  Assume $f$ is cont. everywhere on $[a,b]$, $\exists f'(x) \quad \forall x \in (a,b)$.  \\
  $\exists$ at least one $c \in (a,b)$ such that 
\begin{equation}
  f(b) - f(a) = f'(c) (b-a)
\end{equation}
\end{theorem}

\begin{proof}
  \[
  \begin{gathered}
\begin{aligned}
  &  h(x) = f(x)(b-a) - x(f(b) - f(a)) \\
  &  h(a) = f(a)b - f(a)a - af(b) + af(a) \\
  &  h(b) = f(b)(b-a) - b(f(b) - f(a)) = bf(a) - af(b) = h(a) 
\end{aligned} \\
    \Longrightarrow \exists c \in (a,b), \text{ such that } h'(c) = 0 = f'(c)(b-a) - (f(b) - f(a)) 
  \end{gathered}
  \]
\end{proof}

\begin{theorem}[Cauchy's Mean-Value Formula]
  Let $f,g$ cont. on $[a,b]$, $\exists f', \, g' $\quad \, $\forall x \in (a,b)$  \\
Then $\exists \, c \in (a,b)$.  x
\begin{equation}
  f'(c) (g(b) - g(a)) = g'(c) (f(b) - f(a)) \quad \text{ (note how it's symmetrical) }
\end{equation}
\end{theorem}

\begin{proof}
\[
\begin{gathered}
\begin{aligned}
  h(x) & = f(x)(g(b) - g(a)) - g(x)(f(b)-f(a)) \\
  h(a) & = f(a)(g(b) -g(a)) - g(a)(f(b) - f(a)) = f(a)g(b) - g(a)f(b) \\
  h(b) & = f(b)(g(b)-g(a)) - g(b)(f(b)-f(a)) \\
\end{aligned} \\
\Longrightarrow h'(c) = f'(c) (g(b) - g(a)) - g'(c) (f(b) - f(a)) = 0 \quad \text{ (by Rolle's Thm.) }
\end{gathered}
\]
\end{proof}

\exercisehead{1} For any quadratic polynomial $y = y(x) = Ax^2 + Bx + C$, 
\[
\begin{gathered}
  \begin{aligned} 
    y(a) & = A a^2 + Ba + C \\
    y(b) & = Ab^2 + Bb + C 
  \end{aligned} \\
  \frac{ y(b) - y(a) }{ b-a } = \frac{ A(b-a)(b+a) + B(b-a)}{ b-a} = A(b+a) + B \\ 
  y' = 2Ax + B \\
  y'\left( \frac{ a+b}{2} \right) = A(a+b) + B 
\end{gathered}
\]
Thus the chord joining $a$ and $b$ has the same slope as the tangent line at the midpt.

\exercisehead{2} \emph{ The contrapositive of a theorem is always true.}  So the contrapositive of Rolle's Theorem is \\ 
If $\nexists$ at least one $c \in (a,b)$ s.t. $f'(c) = 0$, \medskip  \\
\phantom{ If at } then $f(a) \neq f(b)$.   \quad \\

\[
\begin{gathered}
  g' = 3x^2 - 3 = 3 (x^2 -1)  \Longrightarrow g'(\pm 1) = 0 \\
  \text{ Suppose } g(B) = 0, \quad \quad \, B \in (-1,1) \\
    \text{ then } \forall \, x \in (-1,1), \quad \, x \neq B, \quad \quad g(x) \neq g(B), \quad \text{ so } g(x) \neq 0 \text{ for } x \neq B \\
  \boxed{ \text{ so only at most one $B \in (-1,1)$ s.t. $g(B) = 0$ } }
\end{gathered}
\]

\exercisehead{3} $f(x)  = \frac{3-x^2}{2}$ if $x \leq 1$, \quad \, $f(x) = \frac{1}{x}$ if $x \geq 1$.  
\begin{enumerate}
\item See sketch.  
\item \[
\begin{gathered}
  f(x) = \begin{cases} \frac{ 3 - x^2}{2} & \text{ if } x \leq 1 \\
    1/x & \text{ if } x \geq 1 
\end{cases}  \quad \quad \, f(1) = 1 = f(1) = 1/1 \\
  f'(x) = \begin{cases} -x; \quad \, f'(1) = -1 & \text{ for } x \leq 1 \\
    -1/x^2; \quad \, f'(1) = -1 & \text{ for } x > 1 
\end{cases} 
\end{gathered}
\]
Then $f(x)$ is cont. and diff. on $[0,2]$.  \quad \\

For $0 \leq a < b \leq 1$
\[
\begin{gathered}
  \frac{ \frac{ 3 -b^2}{2} - \left( \frac{ 3 - a^2}{2} \right) }{ b -a } = \frac{ -(a+b)}{ 2 } = -c \\
  \text{ Note that $ - 1 \leq f' \leq 0 $ for $ 0 \leq x \leq 1 $ }
\end{gathered}
\]

For $1 \leq a < b \leq 2$
\[
\begin{gathered}
  \frac{ \frac{1}{b} - \frac{1}{a} }{ b- a } = \frac{ -1 }{ab} = \frac{-1}{c^2} \Longrightarrow c = \sqrt{ab} \\
  \text{ Note that $-1 \leq f' \leq -1/4 $ }
\end{gathered}
\]

For $ 0 \leq a \leq 1$ , \quad \quad $ 1 \leq b \leq 2$ 
\[
\begin{gathered}
  \frac{ \frac{1}{b} - \left( \frac{ 3 - a^2}{2} \right) }{ b-a} = \frac{ 2 - (3-a^2) b }{ 2b (b-a) } = -c \text{ or } \frac{-1}{c^2} \\
  \text{ depending upon if $0\leq c \leq 1$ or $ 1 \leq c \leq 2$, respectively }
\end{gathered}
\]
For instance, for $a=0 , \, b = 2$, then $\frac{ f(b) - f(a)}{ b-a} = -1/2$, so $c=1/2$ or $c= \sqrt{2}$  
\end{enumerate}

\exercisehead{4} 
\[
\begin{gathered}
  f(1) = 1 - 1^{2/3} = 0 = f(-1) = 1 - ((-1)^2)^3 = 0 \\
  f' = \frac{-2}{3} x^{-1/3} \neq 0 \quad \text{ for } |x| \leq 1
\end{gathered}
\]
This is possible since $f$ is not differentiable at $x=0$.  

\exercisehead{5}  $x^2 = x \sin{x} + \cos{x}$.  $g = xS + C - x^2$.  $g' = S + xC - S - 2x = xC - 2x = x(C- 2)$.  Since $|C| \leq 1$ then $(C-2)$ is negative for all $x$.  Then for $x \gtrless 0$, $g' \lessgtr 0$.  Since $g(0) = 1$ and for $x\to \pm \infty$, $g \to \mp \infty$, then we could conclude that $g$ must become zero between $0$ and $\infty$ and $-\infty$ and $0$.  

\exercisehead{6} 
\[
\begin{gathered}
  \frac{ f(b)- f(a)}{ b-a} = f'(c) \\
  \begin{gathered}
    b = x + h \\
    b-a = h \\
    a= x  \\
    x < x + \theta h < x + h 
  \end{gathered} \quad \, \Longrightarrow f(x+h) - f(x) = h f'(x+\theta h)   
\end{gathered}
\]
\begin{enumerate}
\item $f(x) = x^2, \quad \, f' = 2x$.  
\[
\begin{gathered}
  (x+h)^2 - x^2 = 2xh + h^2 =h(2(x+\theta h)) \\
  \frac{ 2x  + h }{ 2 } - x = \theta h \Longrightarrow \boxed{ \theta = \frac{1}{2}}  \quad \text{ so then } \lim_{h\to 0} \theta = \frac{1}{2} 
\end{gathered}
\]
\item $f(x) = x^3$, \quad \, $f'=3x^2$.  \\
 \[
\begin{gathered}
  (x+h)^3 - x^3 = 3x^2 h + 3xh^2 + h^3 = h3(x+\theta h)^2 \Longrightarrow \left( \sqrt{ \frac{3x^2 + 3xh +h^2 }{ 3 } } - x \right)/h = \theta \\
  \theta = \sqrt{ \frac{ 3x^2 + 3xh + h^2}{ 3h^2 } }- \frac{x}{h} = \frac{ \sqrt{ x^2 + xh + \frac{h^2}{3} }- x }{ h } \left( \frac{ \sqrt{ x^2 + xh + \frac{h^2}{3} } + x }{\sqrt{ x^2 + xh + \frac{h^2}{3} } + x } \right) = \\
  = \frac{ x + \frac{h}{3}}{ x + \sqrt{ x^2 + hx + \frac{h^2}{3} } } \\
  \Longrightarrow \lim_{h\to 0 }\boxed{  \theta = \frac{1}{2} }
\end{gathered}
\]
\textbf{ Notice the trick of multiplying by the conjugate } on top and bottom to get a way to evaluate the limit.  
\end{enumerate}

\exercisehead{7} $f(x) = (x-a_1)(x-a_2)\dots(x-a_r)g(x)$.  \\
\begin{enumerate}
\item $a_1 < a_2 < \dots < a_r$.   \\
Since $f(a_1) = f(a_2) = 0 $.  \quad \, $f'(c) = 0 \text{ for } c_1 \in (a_1,a_2)$.   \\
Consider that $f(a_2) = f(a_3) = 0$ as well as $f'(c_2) = 0 \text{ for } c \in (a_2, a_3)$.  \\
Indeed, since $f(a_j) = f(a_{j+1}) = 0$, $f'(c) = 0 $ for $c \in (a_j, a_{j+1})$.  \\
Thus, $\exists r- 1$ zero's.  \\
$f^{(k)}$ has $r-k$ zeros in $[a,b]$.  \\
$f^{(k)} = (x-a_1)(x-a_2)\dots(x-a_{r-k})g_k(x) $\\
Since $f(a_1) = f(a_2) = 0$, \quad \, $f^{(k+1)}(c_1) =0$ for $c_1 \in (a_1, a_2)$.  
\[
\begin{gathered}
  f^{(k)}(a_j) = f^{(k)}(a_{j+1}) = 0 , \quad \, f^{(k+1)}(c_j) = 0 \text{ for } c_j \in (a_j, a_{j+1}) \\
  \Longrightarrow f^{(k)}(x) \text{ has at least $r-k$ zeros in $[a,b]$ }
\end{gathered}
\]
We had shown the above by induction.  
\item We can conclude that there's at most $r+k$ zeros for $f$ (since $f^{(k)}$ has exactly $r$ zeros, the intervals containing the $r$ zeros are definite). 
\end{enumerate}

\exercisehead{8} Using the mean value theorem
\begin{enumerate}
\item 
\[
\begin{gathered}
  \frac{ \sin{x} - \sin{y}}{ x- y} = \cos{c} \quad \, \Longrightarrow \left| \frac{ \sin{x} - \sin{y} }{ x- y} \right| = |\cos{c} | \leq 1 \\
  \Longrightarrow | \sin{x} - \sin{y} | \leq |x-y| 
\end{gathered}
\]
\item $x \geq y > 0$.  \\
$f(z) = z^n$ is monotonically increasing for $n \in \mathbb{Z}$.  

By mean-value theorem, 
\[
\frac{ x^n -y^n }{ x-y} = nc^{n-1} \text{ for } y < c < x 
\]
Since $ 0 < y < c < x$; \quad \, $ n y^{n-1} \leq \frac{ x^n - y^n}{ x- y} \leq n x^{n-1}$.  
\end{enumerate}

\exercisehead{9} Let $g(x) = \left( \frac{ f(b) - f(a) }{ b-a } \right) x + \left( \frac{ b f(a) - a f(b) }{ b-a } \right)$.  
\[
\begin{gathered}
  f-g = h \quad \quad \begin{aligned}
    h(a) & = h(c) \\
    h(c) & = h(b) 
\end{aligned} \\
  \text{ so } \exists c_1 \in (a,c), \, c_2 \in (c,b) \text{ s. t. } h'(c_1) = h'(c_2) = 0 \text{ by Rolle's Thm. } \\
  \text{ Let } h' = H \\
  \text{ since } H(c_1) = H(c_2) = 0 \text{ and $H$ is cont. diff. on $(c_1,c_2)$.  then } \\
  \exists c_3 \in (c_1,c_2) \text{ s.t. } H'(c_3) = h''(c_3) = 0 \\
  \text{ Now } h'' = (f-g)'' = f'' \text{ so } f''(c_3) = 0 
\end{gathered}
\]
We've shown one exists; that's enough.  

\exercisehead{10} 
Assume $f$ has a derivative everywhere on an open interval $I$.  
\[
g(x) = \frac{ f(x) - f(a)}{ x- a} \text{ if } x \neq a; \quad \, g(a) = f'(a)
\]
\begin{enumerate}
\item $g = \left( \frac{1}{ x- a} \right) f - \frac{1}{ x- a} f(a)$.  $f$ is cont. on $(a,b]$ since $\exists \, f' \quad \, \forall x \in (a,b)$.  \\
  $\frac{1}{x-a}$ is cont. on $(a,b]$.  Then $g$ is cont. on $(a,b]$ (remember, you can add, subtract, multiply, and divide cont. functions to get cont. functions because the rules for taking limits allow so).  

$g$ is cont. at $a$ since $\lim_{x \to a } g = \lim_{ x \to a} \frac{ f(x) - f(a)}{ x-a} = f'(a)$.  \\

By mean value theorem,
\[
\left( \frac{ f(x) - f(a) }{ x- a} \right) = f'(c) = g(x) \quad \, \forall c \in (a,x) \quad \, \forall x \in (a,b]
\]
Then $\forall c \in [a,b]$, $f'(c)$ ranges from $f'(a)$ to $g(b)$ since $f'(c) = g(x)$ so whatever $g(x)$ ranges from and to, so does $f'(c)$.  
\item Let $h(x) = \frac{ f(x) - f(b)}{ x- b} $ if $x \neq b; \quad \, h(b) = f'(b)$.   \\
$h$ is cont. on $[a,b)$ since $\frac{1}{x-b}$ is cont., $f(x)$ is cont.  \\
  $\lim_{x\to b } h = \lim_{x\to b} \frac{f(x) - f(b)}{ x -b} = f'(b)$ so $h$ is cont. at $b$.   \\

$h$ is cont. on $[a,b] \to h$ takes all values from $h(a)$ to $f'(b)$ on $[a,b]$ (by intermediate value theorem).   \\

By mean value theorem,
\[
h(x) = \frac{f(b)-f(x)}{ b-x } = f'(c_2) \quad \, \text{ for } c_2 \in (x,b) \quad \, \forall \, x \in [a,b] 
\]
So then $f'$ ranges from $h(a)$ to $f'(b)$ just like $h$.   \\

$h(a) = g(b)$.  So then $f'$ must range from $f'(a)$ to $f'(b)$
\end{enumerate}

%-----------------------------------%-----------------------------------%-----------------------------------
\subsection*{ 4.19 Exercises - Applications of the mean-value theorem to geometric properties of functions, Second-derivative test for extrema, Curve sketching }
%-----------------------------------%-----------------------------------%-----------------------------------
 \quad \\

\exercisehead{1} $f(x) = x^2 - 3x + 2$
\begin{enumerate}
\item $f'(x) = 2x - 3 \quad \, x_0 = \frac{3}{2}$.  
\item $f'(x) \gtrless 0$ for $x \gtrless \frac{3}{2}$
\item $f'' = 2 > 0 $ for $\forall \, x \in \mathbb{R}$  
\item See sketch.  
\end{enumerate}

\exercisehead{2} $f(x) = x^3 - 4x$
\begin{enumerate}
  \item $f' = 3x^2 -4 $  \quad \, $x_c = \pm \frac{2}{ \sqrt{3}}$
  \item $f' \gtrless 0 $ when $|x| \gtrless \frac{2}{ \sqrt{3}}$  
  \item $f'' = 6x$  $f'' \gtrless 0 $ when $x \gtrless 0 $
  \item See sketch.  
\end{enumerate}

\exercisehead{3} $f(x) = (x-1)^2 (x+2) $
\begin{enumerate}
\item $f' = 3(x-1)(x+1)$ \quad \, $f'(x) =0 $ when $x = \pm 1$ 
\item $f' \gtrless$ when $|x| \gtrless 1$
\item $f'' = 3(2x) = 6x$ \quad \, $f'' \gtrless 0 $ when $x \gtrless 0$
\item See sketch.  
\end{enumerate}


\exercisehead{4} $f(x) = x^3 - 6x^2 + 9x + 5 $
\begin{enumerate}
\item $f' = 3x^2 - 12x + 9 = 3(x-3)(x-1)$  \quad \, $f'(x) = 0 $ when $x=3, 1$
\item \[
\begin{aligned}
  f'(x) > 0 \text{ when } x < 1, \, x >3 \\
  f'(x) < 0 \text{ when } 1 < x < 3 
\end{aligned}
\]
\item $f'' = 6x - 12 = 6(x-2)$ \quad \, $f'' \gtrless 0 $ when $x \gtrless 2$
\item See sketch.  
\end{enumerate}


\exercisehead{5} $f(x) = 2 + (x-1)^4 $
\begin{enumerate}
\item $f'(x) = 4 (x-1)^3$.  $f'(0) = 0 $ when $x=1$ 
\item $f'(x) \gtrless 0$ when $|x| \gtrless 1$
\item $f''(x) = 12(x-1)^2 > 0 \quad \, \forall \, x \neq 1$
\item See sketch.  
\end{enumerate}

\exercisehead{6} $ f(x) = 1/x^2 $
\begin{enumerate}
\item $f' = \frac{-2}{x^3}$    $f'(x) = 0$ for no $x$ 
\item $f' \gtrless 0$ when $x \lessgtr 0 $
\item $f'' = \frac{6}{x^4} > 0 $ \quad \, $\forall x \neq 0$  
\item See sketch.  
\end{enumerate}


\exercisehead{7} $f(x) = x + 1/x^2 $
\begin{enumerate}
\item $f' = 1 + \frac{ -2 }{x^3} $  $f'(x) = 0 = 1 - \frac{2}{x^3} \Longrightarrow x_c = 2^{1/3}$  
\item 
\[
\begin{aligned}
  & f'(x) > 0 \text{ when } x < 0, \, 0 < x < 2^{1/3} \\
  & f'(x) < 0 \text{ when } x > 2^{1/3}  
\end{aligned}
\]
\item $f'' = \frac{6}{x^4} > 0 \quad \, \forall x \neq 0$
\item See sketch.  
\end{enumerate}

\exercisehead{8} $f(x) = \frac{1}{ (x-1)(x-3)} $
\begin{enumerate}
\item $f' = \frac{-1}{ (x-1)^2(x-3)^2 }((x-3) + x-1) = \frac{ (-2)(x-2)}{ (x-1)^2 (x-3)^2 }$  \\
$f'(x) = 0 $ when $x=2$ 
\item $f' \gtrless 0$ when $x \lessgtr 2$
\item 
\[
\begin{gathered}
\begin{aligned}
  f'' & = (-2) \left( \frac{ (x-1)^2(x-3)^2 - (x-2) (2(x-1)(x-3)^2 + 2 (x-3)(x-1)^2 ) }{ (x-1)^4 (x-3)^4 } \right) = \\
  & = (6)\left( \frac{ x^2 - 4x + \frac{13}{3} }{ (x-1)^3 (x-3)^3} \right) 
\end{aligned} \\
x^2 - 4 x + \frac{13}{3} > 0 \text{ since } 144 - 4(-3)(-13) = 144 + 12(-13) < 0 \text{ so } \\
\begin{aligned}
  & f'' > 0 \text{ if } x > 3, \, x < 1 \\
  & f'' < 0 \text{ if } 1 < x < 3 
\end{aligned}
\end{gathered}
\]
\item See sketch.  
\end{enumerate}


\exercisehead{9} $f(x) = x/(1+x^2) $
\begin{enumerate}
\item 
\[
f' = \frac{ (1+x^2) - x(2x) }{ (1+x^2)^2 } = \frac{ 1 - x^2}{ (1+x^2)^2 }
\]
$f'(x)= 0$ when $x = \pm 1$
\item $f' \gtrless 0 $ when $|x| \lessgtr 1$
\item \[
\begin{gathered}
  f''  = \frac{ -2x (1+x^2)^2 - 2 (1+x^2)(2x)(1-x^2) }{ (1+x^2)^4 } = \frac{2x(x^2-3)}{ (1+x^2)^3 } \\
  \begin{aligned}
    & f'' > 0  \text{ when } x > \sqrt{3}  \\
    & f'' < 0 \text{ when } 0 < x < \sqrt{3} \\
    & f'' > 0 \text{ when } -\sqrt{3} < x < 0 \\
    & f'' < 0 \text{ when } x < -\sqrt{3} 
  \end{aligned}
\end{gathered}
\]
\item See sketch.  
\end{enumerate}


\exercisehead{10} $f(x) = (x^2 - 4)/(x^2 -9) $
\begin{enumerate}
\item 
\[
f' = \frac{ 2x(x^2- 9) - (x^2 -4) (2x) }{ (x^2 - 9)^2 } = \frac{ -10 x }{ (x^2 - 9)^2 } 
\]
$f'(0) = 0$
\item $f' \gtrless 0 $ when $x \lessgtr 0$, \quad \, $x \neq \pm 3$ 
\item 
\[
f'' = (-10) \left( \frac{ (x^2-9)^2 - 2(x^2 - 9) (2x) x }{ (x^2 - 9)^2 } \right) = (30) \frac{ (x^2 + 3) }{ (x^2 - 9)^3 } 
\]
$f'' \gtrless 0 $ when $|x| \gtrless 3$
\item See the sketch.  
\end{enumerate}


\exercisehead{11} $f(x) = \sin^2{x} $
\begin{enumerate}
\item $f' = \sin{2x}$  So then $f' = 0$ when $x = \frac{\pi }{2} n$  
\item 
\[
\begin{aligned}
  & f' > 0 \text{ when } \begin{aligned}
    0 < x & < \frac{ \pi}{2} \\
    \pi n < x  & < \frac{ \pi}{2} + \pi n 
\end{aligned} \\
  & f' < 0 \text{ when } \frac{\pi}{2} + \pi n < x < \pi(n+1) 
\end{aligned}
\]
\item 
\[
f'' = 2\cos{ 2x} \quad \quad \, \begin{aligned}
  & f'' > 0 \text{ when } \frac{ -\pi}{4} + \pi n < x < \frac{ \pi}{4}  + \pi n \\
  & f'' < 0 \text{ when } \frac{ \pi}{4} + \pi n < x < \frac{ 3\pi}{4}  + \pi n
\end{aligned}
\]
\item See sketch.  
\end{enumerate}


\exercisehead{12} $f(x) = x - \sin{x} $
\begin{enumerate}
\item $f' = 1 - \cos{x}$  \quad \quad \, $f' = 0$ when $x = 2\pi n$ 
\item $f' > 0 $ if $x \neq 2 \pi n$ 
\item 
  \[
  f'' = \sin{x} \quad \quad \, \begin{aligned}
    & f'' > 0 \text{ when } 2 \pi n < x < 2\pi n + \pi \\
    & f'' < 0 \text{ when } 2 \pi n + \pi < x < 2 \pi(n+1)
  \end{aligned}
  \]
\item See sketch.  
\end{enumerate}


\exercisehead{13} $f(x) = x+ \cos{x} $
\begin{enumerate}
\item $f' = 1 - \sin{x}$  \quad \quad \, $x = \frac{\pi}{2} + 2 \pi n$ \quad \quad \, $f'(x) = 0$
\item $f' > 0$ if $x \neq \frac{\pi}{2} + 2\pi n $
\item 
\[
f'' = -\cos{x} \quad \quad \, \begin{aligned}
  & f'' > 0 \text{ when } \frac{ - \pi}{2} + 2 \pi n < x < \frac{ \pi }{2} + 2\pi n  \\
  & f'' < 0 \text{ when } \frac{  \pi}{2} + 2 \pi n < x < \frac{ 3 \pi }{2} + 2\pi n 
\end{aligned}
\]
\item See sketch.  
\end{enumerate}


\exercisehead{14} $f(x) = \frac{1}{6}x^2 + \frac{1}{12} \cos{2x} $
\begin{enumerate}
\item $f' = \frac{1}{3} x + \frac{ -\sin{2x}}{6}$  \quad \quad \, $f'(0) = 0$
\item $f' \gtrless 0$ when $x \gtrless 0$
\item $f'' = \frac{1}{3} - \frac{ \cos{2x}}{ 3 } = \frac{ 1 - \cos{2x}}{3 }$  \\
$x = \pi n$ for $f'' =0$.  Otherwise $f'' > 0$ for $x \neq \pi n$
\item See sketch.  
\end{enumerate}

%-----------------------------------%-----------------------------------%-----------------------------------
\subsection*{ 4.21 Exercises - Worked examples of extremum problems }
%-----------------------------------%-----------------------------------%-----------------------------------
\quad \\
\exercisehead{1} 
\[
\begin{gathered}
  A = xy \\
  \begin{aligned}
    & P = 2(x+y) = 2 (x + \frac{A}{x} ) \\
    & P' = 2(1 - \frac{A}{x^2 } ) = 0 
  \end{aligned} \\
  x = \sqrt{A} \\
  P'' = \frac{4A}{x^3} > 0 \text{ for } x > 0 \text{ so $x = \sqrt{A}$ minimizes $P$ }
\end{gathered}
\]

\exercisehead{2} 
\[
\begin{gathered}
  A = xy \quad \quad \, L = 2x + y  \\
  A = x(L-2x) = Lx - 2x^2  \Longrightarrow \frac{dA}{dx} = L - 4x = 0 \text{ when } x = \frac{L}{4}  \quad \quad \, y = \frac{L}{2} \\
  A'' = -4 \text{ so } x = \frac{L}{4} \text{ maximizes } A 
\end{gathered}
\]

\exercisehead{3} 
\[
\begin{gathered}
  A = xy \quad \quad \, L = 2x + y = 2x + \frac{A}{x} \quad \quad \, \frac{dL}{dx} = 2 + \frac{-A}{x^2 } = 0 \text{ when } x = \frac{ \sqrt{A}}{ \sqrt{2}} \quad \quad \, y = \sqrt{2}\sqrt{A} \\
  L'' = \frac{2A}{x^3} > 0 \text{ for } x = \sqrt{ \frac{A}{2} } \quad \quad \, \text{ so $x$ minimizes $L$ }
\end{gathered}
\]

\exercisehead{4} $f = x^2 + y^2 = x^2 + (S-x)^2$ \\
$f' = 2x + 2 (S-x)(-1) = -2S + 4 x$ \quad \quad \, $\Longrightarrow x = \frac{S}{2}$ \\
$f'' = 4 > 0 $ so $x = \frac{S}{2}$ minimizes $f$ 

\exercisehead{5} $x^2 + y^2 = R >0$  \\
$f = x + y$ 
\[
\begin{gathered}
  f' = 1 +y' = 0 = 1 + \frac{-x}{y} = 0 \quad \quad \, \Longrightarrow \boxed{ y = x } \\
  f'' = y'' = \frac{ -1 - y'^2 }{ y } \text{ for $y > 0$, $f'' < 0$ so that $f$ is max. when $y = x$ } \\
 \text{ Note that } \quad \quad \, 
 \begin{gathered}
   2x + 2y y' = 0 \\
   y' = \frac{-x}{y} \\
   x + y y' = 0 
 \end{gathered}  \quad \quad \, 
 1 + y'^2 + y y'' = 0 \Longrightarrow y y'' = -1 - y'^2 
\end{gathered}
\]

\exercisehead{6}
\[
\begin{gathered}
  l^2 = (L-x)^2 + x^2 = L^2 - 2 Lx + 2x^2 = A \\
  \frac{dA}{dx} = -2L + 4 x = 0 \Longrightarrow x = \frac{L}{2} \\
  \frac{d^2 A}{dx^2 } = 4 > 0 \Longrightarrow A \text{ minimized } \\
  l(x = \frac{L}{2} ) = \frac{ L \sqrt{2}}{ 2 } 
\end{gathered}
\]

\exercisehead{7}
\[
\begin{gathered}
  (x + \sqrt{ L^2 - x^2 } )^2 = A \\
  A' = 2 (x + \sqrt{ L^2 - x^2 } )(  1 + \frac{ -x }{ \sqrt{ L^2 - x^2 } } ) = 0 \text{ when } L^2 - x^2 = x^2 \text{ or } x = \frac{L}{\sqrt{2}} \\
  \text{ so then the side of the circumscribing and area-maximized square is } \frac{ L}{\sqrt{2}} + \sqrt{ L^2 - \frac{L^2 }{2} } = \frac{2L}{\sqrt{2}}
\end{gathered}
\]

\exercisehead{8} 
\[
\begin{gathered}
  A = (2x)( 2 \sqrt{ R^2 - x^2 } ) = 4x \sqrt{ R^2 - x^2 } \\
  A' = 4 ( \sqrt{ R^2 - x^2 } + \frac{ -x^2 }{ \sqrt{ R^2 - x^2 } } ) = 4 \left( \frac{ R^2 - 2x^2 }{ \sqrt{ R^2 - x^2 } } \right) \Longrightarrow x = \frac{R}{\sqrt{2}} \\
  \text{ since } A' \gtrless 0 \text{ when } x \lessgtr \frac{R}{\sqrt{2}}, \text{ so $A$ is maximized at $x = \frac{R}{\sqrt{2}}$ } 
\end{gathered}
\]
$2x = \frac{2R}{\sqrt{2}}$; \quad \quad \, $2 \sqrt{ R^2 - x^2 } = \frac{2R}{\sqrt{2}}$ so then the rectangle that has maximum size is a square.  

\exercisehead{9} Prove that among all rectangles of a given area, the square has the smallest circumscribed circle.  

$A_0 = (2x)(2\sqrt{r^2 - x^2} ) = 4 x \sqrt{ r^2 - x^2 }$ (fix the area to be $A_0$) \\
$\left( \frac{A_0}{4x} \right)^2 = r^2 - x^2 \Longrightarrow x^4 - x^2 r^2 + \frac{A_0^2}{16} =  0 $ 
\[
\begin{gathered}
  \Longrightarrow 0 = 2xr^2 + x^2 2r \frac{dr}{dx} - 4 x^3 \\
  \frac{dr}{dx} = 0 \text{ (for extrema) } \Longrightarrow x = \frac{r}{\sqrt{2}} \text{ and } \sqrt{ r^2 - x^2 } = \frac{r}{\sqrt{2}}
\end{gathered}
\]
We could argue that we had found a minimum because at the ``infinity'' boundaries, the circumscribing circle would be infinitely large.  

\exercisehead{10} Given a sphere of radius $R$, find the radius $r$ and altitude $h$ of the right circular cylinder with the largest lateral surface area $2\pi r h$ that can be inscribed in the sphere.  
\[
\begin{gathered}
  R^2 = \left( \frac{h}{2} \right)^2 + r^2 \\
  A = 2\pi r h = 2 \pi r \sqrt{ 4(R^2 - r^2 ) } = 4 \pi r \sqrt{ R^2 - r^2 } \\
  \frac{dA}{dr} = 4\pi \left( \sqrt{ R^2 - r^2 } + \frac{ -r^2 }{ \sqrt{ R^2 - r^2 } } \right) = 4 \pi \left( \frac{ R^2 - 2r^2 }{ \sqrt{ R^2 - r^2 } } \right) \Longrightarrow \boxed{ r = \frac{ R}{\sqrt{2}} } \\
  \Longrightarrow \boxed{ h = \sqrt{2} R }
\end{gathered}
\]

\exercisehead{11} 
Among all right circular cylinders of given lateral surface area, prove that the smallest circumscribed sphere has radius $\sqrt{2}$ times that of the cylinder.  

\[
\begin{gathered}
  A_0 = 2 \pi R H \quad \quad \, \text{ ($A_0$ is the total lateral area of the cylinder) } \\
  r^2 = R^2 + \left( \frac{H}{2} \right)^2 = R^2 + \left( \frac{ A_0}{ 4 \pi R} \right)^2 = R^2 + \frac{ A_0^2 }{ 16 \pi^2 R^2 } \\
  2 r \frac{ dr}{dR} = 2R + \frac{ A_0^2}{ 16 \pi^2 } \left( \frac{ -2 }{R^3} \right) \Longrightarrow \frac{dr}{dR} = 0 \Longrightarrow R = \frac{ \sqrt{A_0}}{2 \sqrt{ \pi }} \Longrightarrow \frac{H}{2} = R \\
  r^2 = R^2 + R^2 = 2 R^2 \Longrightarrow \boxed{ r = \sqrt{2} R }
\end{gathered}
\]

\exercisehead{12} Given a right circular cone with radius $R$ and altitude $H$.  Find the radius and altitude of the right circular cylinder of largest lateral surface area that can be inscribed in the cone.  

$\frac{h}{R-r} = \frac{H}{R} = \alpha$ is the constraint (look, directly at the side, at the similar triangles formed)
\[
\begin{gathered}
  A = 2\pi r h = 2\pi r \alpha (R-r) = 2\pi \alpha (rR- r^2 ) \\
  \frac{dA}{dr} = 2 \pi \alpha (R- 2r) = 0 \Longrightarrow \boxed{ r = \frac{R}{2} }; \quad \, \boxed{ h =\frac{H}{2} } \\
  A'' = 2 \pi \alpha (R- 2r) \\
  \text{ since } \frac{dA}{dr} \gtrless 0 \text{ when } r \lessgtr \frac{R}{2}, \quad \, r = \frac{R}{2} \text{ maximizes lateral surface area }
\end{gathered}
\]

\exercisehead{13} Find the dimensions of the right circular cylinder of maximum volume that can be inscribed in a right circular cone of radius $R$ and altitude $H$.  

Constraint: $\frac{h}{R-r} = \frac{H}{R} = \alpha$ \\
\[
\begin{gathered}
  V = \pi r^2 h = \pi r^2 \alpha (R-r) = \pi r^2 \alpha (R-r) = \pi \alpha (Rr^2 - r^3) \\
  \frac{dV}{dr}  = \pi \alpha ( 2 R r  - 3 r^2 ) = r \pi \alpha ( 2R - 3r ) \quad \quad \, \boxed{ r = \frac{2R}{3} } \\
  \text{ since } \frac{dV}{dr} \gtrless 0 \text{ when } r \lessgtr \frac{2R}{3} , \quad r = \frac{2R}{3} \text{ maximizes volume } \\
  \boxed{ h = \frac{1}{3} H }
\end{gathered}
\]

\exercisehead{14} Given a sphere of radius $R$.  Compute, in terms of $R$, the radius $r$ and the altitude $h$ of the right circular cone of maximum volume that can be inscribed in this sphere.  

\[
\begin{gathered}
  V = \frac{ \pi r^2}{3} \left( R + \sqrt{ R^2 - r^2 } \right) \\
  \frac{dV}{dr} = \frac{ \pi }{3} \left( 2 rR + 2r \sqrt{ R^2 - r^2 } + \frac{ r^2 (-r) }{ \sqrt{ R^2 - r^2 } } \right) = \frac{ \pi }{3} r \frac{ (2R \sqrt{ R^2 - r^2 } + 2R^2 - 3r^2  )}{ \sqrt{ R^2 - r^2 } } = 0 \\
  \Longrightarrow \boxed{ r = \frac{ 2 \sqrt{2} R }{ 3 }; \quad \, h = \frac{4R}{3} } 
\end{gathered}
\]
Considering the geometric or physical constraints, since $\lim_{V \to \infty} V = \lim_{h \to \infty} V = 0$, so then $r = \frac{2\sqrt{2} R}{3} $ must maximize $V$.  

\exercisehead{15} Find the rectangle of largest area that can be inscribed in a semicircle, the lower base being on the diameter.  

\[
\begin{gathered}
  A = \sqrt{ R^2 - x^2 } x \\
  A' = \sqrt{ R^2 - x^2 } + \frac{ -x^2}{ \sqrt{ R^2 - x^2 }} = 0 \Longrightarrow \boxed{ x = \frac{R}{ \sqrt{2}} ; \quad \quad \, h = \frac{R}{ \sqrt{2}} }
\end{gathered}
\]

\exercisehead{16} Find the trapezoid of largest area that can be inscribed in a semicircle, the lower base being on the diameter.  

\[
\begin{gathered}
  A = \frac{1}{2} h ( 2 \sqrt{ R^2 - h^2 } + 2R ) \\
  \frac{dA}{dh} = \sqrt{ R^2 -h^2 } + R + h \left( \frac{-h}{ \sqrt{ R^2-h^2 } } \right) \\
  \frac{dA}{dh} = 0 \Longrightarrow \boxed{ h = \frac{ \sqrt{3} R }{ 2 } } \\
  \Longrightarrow A = \frac{ 5\sqrt{3}R^2 }{ 8 } \quad \quad \, \sqrt{ R^2 - h^2 } = 2 \sqrt{ R^2 - \frac{3}{4} R^2 } = 2 \frac{R}{2} = \boxed{ R }
\end{gathered}
\]

\exercisehead{17} An open box is made from a rectangular piece of material by removing equal squares at each corner and turning up the sides.  Find the dimensions of the box of largest volume that can be made in this manner if the material has sides (a) 10 and 10; (b) 12 and 18

\begin{enumerate}
\item \[
\begin{gathered}
  (x-2r)(Y- 2r)r = (xy - 2rx - 2ry + 4r^2)r = xyr - 2r^2 x - 2r^2 y + 4r^3 = V \\
  \frac{dV}{dr} = xy - 4rx - 4ry + 12r^2 = 0 \\
\Longrightarrow   \begin{aligned}
  r & = \frac{ 4 (x+y) \pm \sqrt{ 16(x+y)^2 - 4 (12) xy } }{ 2(12) } = \frac{ (x+y) \pm \sqrt{ x^2 + y^2 - xy }}{ 6 } 
\end{aligned} \\
\frac{d^2 V}{dr^2} = -4x - 4 y + 24 r = -4(x+y) + 24r
\end{gathered}
\]
We can plug in our expression for $r$ into the second derivative of $V$, the volume of the box, to find out that we want to pick the ``negative'' root from $r$, in order to maximize the box volume.  

Then for $x=10; \, y = 10$, we have $r = \frac{5}{3}$, so that the box dimensions are $\frac{5}{3} \times \frac{20}{3} \times \frac{20}{3}$.  
\item 12 and 18 \\
$\Longrightarrow 5- \sqrt{7} \times 2 + 2 \sqrt{7} \times 8 + 2 \sqrt{7}$
\end{enumerate}

\exercisehead{18} If $a$ and $b$ are the legs of a right triangle whose hypotenuse is 1, find the largest value of $2a+b$.  
\[
\begin{gathered}
  \begin{aligned} \\
  L & = 2a + b = 2a + \sqrt{ 1 - a^2 } \\
  L' & = 2 + \frac{-a}{ \sqrt{ 1 - a^2 } }
\end{aligned} \quad \quad \, L' = 0 \Longrightarrow \left( \frac{a}{2} \right)^2 = 1 - a^2  \Longrightarrow a = \frac{2}{ \sqrt{5}} \\
  L'' = (-1) \left( \frac{ \sqrt{ 1 - a^2 } - \frac{ -a}{ \sqrt{ 1 - a^2 } } a }{ 1 - a^2 } \right) = (-1)\left( \frac{1}{ (1-a^2)^{3/2} } \right) < 0 \text{ (so $a = \frac{2}{\sqrt{5}}$ maximizes $L$ ) } 
\end{gathered}
\]

\exercisehead{19} $2 + \frac{x^2}{600}$ gallons per hour.  $l_0 = 300 \, mi$  $x = $ constant speed.  $\frac{ l_0}{x} = $ time spent.  $K = \text{ gas cost } = 0.30$.  \\
$C = \text{ gas cost } + \text{ driver labor cost } = l_0 \left( \frac{ 2 K}{x} + \frac{ K x }{ 600 } + \frac{D}{x} \right)$\[
\begin{gathered}
  \frac{dC}{dx} = l_0 \left( \frac{-2K}{x^2 } + \frac{K}{600} - \frac{D}{x^2} \right) = 0 \xrightarrow{ \frac{dC}{dx} = 0 } x = \sqrt{ \frac{ 2 K + D}{ K } } 10 \sqrt{6 } \\
  \frac{d^2 C}{dx^2} = l_0 \left( (-2K - D) \left( \frac{-2}{x^3} \right) \right) = l_0 \left( \frac{ 2 (2K+D)}{x^3 } \right) > 0 \\
  \text{ Thus, $C$ is minimized if } x = \sqrt{ \frac{2K+D}{ K} } 10 \sqrt{6}  \\
  \Longrightarrow  C_{min} = (300) \left( \frac{ \sqrt{ 2 K + D} \sqrt{K}}{ 10 \sqrt{6}} + \frac{ \sqrt{K} \sqrt{ 2K +D } 10\sqrt{6} }{ 600 } \right) = \boxed{ 3 \sqrt{2} \sqrt{ 6+ 10D} }
\end{gathered}
\]
Remember that there is a \textbf{ speed limit } of $60 \, mi/hr$.  

\begin{enumerate}
\item $D =0$, \quad \, $x = 20 \sqrt{3}$ \quad \, $C = 6 \sqrt{3} \approx 10.39$
\item $D = 1$, \quad \, $x = 40 \sqrt{2}$ \quad \, $C = 12\sqrt{2} \approx 16.97$
\item $D=2$, \quad \, $x = 60$ (because of the speed limit) \quad \, $C = 300\left( \frac{2K}{60} + \frac{ K(60)}{600} + \frac{D}{ 60} \right) = 5 ( 2.4 + D) = 22.00$ \\
\item $D=3$, \quad \, $x = 60$ \quad \, $C = 27.00$
\item $D=4$, \quad \, $x = 60$ \quad \, $C = 32.00$
\end{enumerate}

\exercisehead{20}  $y = \frac{x}{x^2 + 1}$ Suppose the rectangle starts at $x_0$ on the $x$ axis.  Then its $y$ coordinate intersecting the curve, and thus the height of rectangle, must be $y_0 = \frac{x_0}{ x_0^2 + 1}$ 
\[
\begin{gathered}
  \Longrightarrow   x_0 = \frac{1}{ 2 y_0} \pm \sqrt{ \frac{1}{ (2y_0)^2 } - 1 } \\ 
  x_2 - x_1 = \sqrt{ \frac{1}{y_0^2} - 4 } 
\end{gathered}
\]
where $x_2 - x_1$ is going to be the base of the rectangle.  The volume of the cylinder, $V$, which is obtained from revolving the rectangle about the $x$ axis, is going to be
\[
\begin{gathered}
  V = \pi y_0^2 (x_2 - x_1) = \pi y_0^2 \left( \sqrt{ \frac{ 1}{y_0^2} - 4  } \right) = \pi y_0 \sqrt{ 1 - 4y_0^2 } \\
  \frac{dV}{dy_0} = \pi \left( \sqrt{ 1 -4 y_0^2 } + \frac{ y_0}{ 2 \sqrt{ 1 - 4 y_0^2 } } (-8y_0 ) \right) = \pi \left( \frac{ 1 - 8y_0^2 }{ \sqrt{ 1 - 4 y_0^2 } } \right) \quad \quad \, \Longrightarrow y_0 = \frac{1}{ 2 \sqrt{2}}
\end{gathered}
\]
We could argue that $V$ is maximized, since the ``infinite'' boundaries would yield a volume of $0$ (imagine stretching and squeezing the rectangle inside the curve).  

Then $V_{max} = \pi \frac{1}{8}  2 = \boxed{ \frac{ \pi}{4} }$

\exercisehead{21}  Draw a \textbf{ good diagram}.  Note how the right triangle that you folded is now \emph{ reflected backwards }, so that this triangle's right angle is on the left-hand side now.  

The constraint is that the crease touches the left edge.  
\[
\begin{aligned}
  w_0 & = l \sin{\alpha} + l \sin{\alpha} \cos{ (2\alpha)} = l \sin{\alpha} ( 1 + \cos{ (2\alpha) } ) = \\
  & = 2 l \sin{\alpha} \cos^2{\alpha}
\end{aligned}
\]

Note that we will obtain a minimum crease because by considering the ``physical infinite'' boundary, we could make a big crease along the vertical half of the paper or the horizontal half of the paper.  

So, isolating $l$, the length of the crease, and then taking the derivative,
\[
\begin{gathered}
  l  = \frac{ w_0 }{ 2 \sin{(\alpha)} \cos^2{\alpha} } = \frac{ w_0}{2} \csc{ (\alpha) } \sec^2{(\alpha) } \\
\begin{aligned} 
  \frac{dl}{d\alpha} & = \frac{w_0}{2} \left( -\cot{\alpha} \csc{\alpha} \sec^2{\alpha} + \csc{\alpha} 2 \sec{\alpha} \sec{\alpha} \tan{\alpha} \right) = \\
  & = \frac{w_0}{2} \left( \frac{ -C}{S} \left( \frac{1}{S} \right) \left( \frac{1}{C^2 } \right) + \frac{1}{S} 2 \left( \frac{1}{C^2} \right) \left( \frac{S}{C} \right) \right) = \frac{w_0}{2} \left( \frac{-1}{S^2 C} + \frac{2}{C^3} \right) 
\end{aligned} \\
\xrightarrow{ \frac{dl}{d\alpha} = 0 } \boxed{ \sin{ \alpha} = \frac{1}{\sqrt{3}} } \text{ or } \tan{\alpha} = \frac{1}{\sqrt{2}} 
\end{gathered}
\]
where $\alpha$ is the angle of the crease.  The corresponding minimum length of the crease will be 
\[
l = \frac{w_0}{2} \frac{1}{ \frac{ 1}{ \sqrt{3}} \frac{2}{3} } = \boxed{ \frac{ 9 \sqrt{3}}{2 } }
\]

\exercisehead{22} \begin{enumerate}
\item Consider the center of the circle $O$, the apex of the isosceles triangle that makes an angle $2\alpha$, $A$, and one of its other vertices, $B$.  Draw a line segment from $O$ to $B$ and simply consider the two triangles making up one half of the isosceles triangle.  Find all the angles.  

Angle $AOB$ is $\pi - 2\alpha$ by the geometry or i.e. inspection of the figure.  The complement of that angle is $2\alpha$.  Beforehand, we can get the length of the isosceles triangle leg from the law of cosines.  
\[
\begin{gathered}
  \cos{ (\pi - 2\alpha ) } = - \cos{ (2\alpha)} \\
  \begin{aligned}
    s^2 & = R^2 + R^2 -2R^2 \cos{ (\pi - 2\alpha)} + 2R^2 (1+ \cos{ (2\alpha) } ) = 2R^2 ( 2\cos^2{\alpha}) = 4R^2 \cos^2{\alpha} \\
    s & = 2R \cos{\alpha} 
  \end{aligned}
\end{gathered}
\]
\textbf{ The constraint equation } is
\begin{equation}\label{E:Constraint_Eqn_iso_triangle_in_circle}
  P = 4R \cos{\alpha} + 2R \sin{(2\alpha) } 
\end{equation}

So then
\[
\begin{gathered}
  P' = 4R ( -\sin{\alpha} ) + 4R \cos{(2\alpha) } = 0  \Longrightarrow \cos{2\alpha} = \sin{\alpha} \\
  \sin{\alpha} = \frac{ -\frac{1}{2} \pm \sqrt{ \frac{1}{4} - 4(1)(-\frac{1}{2} ) } }{ 2(1) } = \frac{1}{2} > 0 \\
  \Longrightarrow \boxed{ P = 3\sqrt{3} R }
\end{gathered}
\]

$P = 3 \sqrt{3}$ is a max because \medskip \\
\quad Look at the ``boundary conditions'' imposed on $P$  by the physical-geometry.  $\alpha =0$, triangle is completely flattened, $\alpha= \pi$, triangle ``completely disappears.''
\item I had originally thought to Reuse the constraint equation, Eqn. ( \ref{E:Constraint_Eqn_iso_triangle_in_circle}).  \emph{ This is wrong!} \medskip \\
Think about the problem directly and for what it actually is; less wishful thinking.  \\
Consider a fixed perimeter $L$ and imagine $L$ to be a string that can be stretched into an isoceles triangle.  A ``trivial'' isoceles triangle is a collapsed triangle with two sides of length $L/2$ only.  Then the radius of the disk needs to be $L/4$.   \bigskip \\

Consider a general isosceles triangle with $2\alpha$ as the vertex angle and isosceles sides of $h$.  The perimeter for this triangle, $P$, is then
\[
\begin{gathered}
  P = 2h + 2h \sin{\alpha} = 2h (1 + \sin{\alpha}) \\
  \Longrightarrow h = \frac{P}{ 2 ( 1+ \sin{\alpha}) } \\
  \frac{h}{2} = R\cos{\alpha}
\end{gathered}
\]
We could try to extremize this equation.
\[
\begin{gathered}
  \frac{ dR}{d\alpha} (4 \cos{\alpha} + 2 \sin{(2\alpha)} ) + R ( -4 \sin{\alpha} + 4 \cos{ (2\alpha) } ) = 0 \\
  \frac{dR}{d\alpha } = 0  \Longrightarrow \cos{(2\alpha)} = \sin{\alpha} \Longrightarrow \sin{\alpha} = \frac{1}{\sqrt{3}} \, \quad \quad \cos{\alpha} =\frac{\sqrt{2}}{\sqrt{3}} \\
   R = \frac{ 3P}{ 4 \sqrt{2} ( \sqrt{3} + 1 ) } 
\end{gathered}
\]
However, this is the \emph{minimized} $R$, \emph{minimized} radius for the smallest circle fitting a particular isosceles triangle of a fixed perimeter.  We want to smallest circle with a radius big enough to fit all the possible triangles.  Thus $R = \frac{L}{4}$
\end{enumerate}

\exercisehead{23} The constraint equation on perimeter is
\[
P = 2 h + W + \pi \left( \frac{W}{2} \right) = 2 h + W (1 + \frac{\pi}{2} )
\]
Then intensity function, ``normalized'' is given by
\[
I = Wh + \frac{ \pi}{2} \left( \frac{W}{2} \right)^2 \left( \frac{1}{2} \right) = \frac{ W p }{2} - \frac{W^2}{2} - \frac{3\pi}{16} W^2 
\]
So then 
\[
\frac{dI}{dW} = \frac{P}{2} - W - \frac{3\pi}{8} W = 0 \Longrightarrow W = \boxed{ \frac{ P}{ 2 + \frac{3\pi}{4} } }
\]
The height of the rectangle is
\[
h = \frac{ P - \frac{ P}{ 2+ \frac{3\pi}{4} } ( 1+ \frac{\pi}{2} ) }{ 2 } = P \left( \frac{ 4+\pi}{ 16 + 6 \pi } \right)
\]

\exercisehead{24} A log 12 feet long has the shape of a frustum of a right circular cone with diameters 4 feet and $(4+h)$ feet at its ends, where $h \geq 0$.  Determine, as a function of $h$, the volume of the largest right circular cylinder that can be cut from the log, if its axis coincides with that of the log.  

Remember to \textbf{ label your diagram carefully}.  
\[
\begin{gathered}
  \frac{ y}{x} = \frac{ \frac{h}{2} - y }{ l_0 - x } = \frac{h/2}{l_0} \Longrightarrow \frac{ h/2 - \left( \frac{ xh}{2l_0} \right) }{ l_0 -x } \\
  V = \pi (H+y)^2 (l_0 - x) = \pi (H + \frac{xh}{2l_0} )^2 (l_0 - x)  \\
  \text{ Note that } V(x=0) = \pi H^2 l_0 = \pi 4 (12) \\
  \begin{aligned}
    \frac{dV}{dx} & = \pi ( 2 ( H + \frac{hx}{2l_0} ) \frac{h}{2l_0} ( l_0 - x ) + (H + \frac{xh}{2l_0} )^2 (-1) ) = \\
    & = \pi (H+\frac{hx}{2l_0} )( h - H - \frac{3xh}{2l_0} ) 
\end{aligned} \quad \quad \, 
\xrightarrow{ \frac{dV}{dx} = 0 } x = \frac{ (h-H)2l_0}{ 3 h } \\
\begin{aligned}
  V(x= \frac{ (h-H) 2 l_0 }{ 3h } ) & = \pi \left( l_0 - \frac{ (h-H)2l_0}{ 3h} \right)\left( H + \frac{ h}{2l_0} \frac{ (h-H) 2l_0 }{ 3h } \right)^2  = \boxed{ \pi l_0 \frac{ (h + 2H)^3 }{ 27 h } }
\end{aligned} 
\end{gathered}
\]
where $H =2$, $l_0 = 12$

\exercisehead{25} 
\[
\begin{gathered}
  S = \sum_{k=1}^n (x-a_k)^2 \Longrightarrow \frac{dS}{dx} = \sum_{k=1}^n 2 (x-a_k) = 0 \\
  \Longrightarrow nx = \sum_{k=1}^n a_k \Longrightarrow x = \frac{ \sum_{k=1}^n a_k }{ n}
\end{gathered}
\]
Since $\lim_{x \to \pm \infty } S = + \infty$, $x = \frac{ \sum_{k=1}^n a_k }{ n } $ minimizes $S$.  

\exercisehead{26} \textbf{ Hint: draw a picture }.  Then observe that for $f(x) \geq 24$, $A$ must be greater than $0$ (we'll show that explicitly soon) and that $f$ must have a \emph{ minimum } somewhere.  

If $A<0$, then consider $f(x) = \frac{ 5 x^7 + A}{ x^5}$.  Consider $x = \frac{ -A^{1/7}}{ 6^{1/7} } > 0 $.  
\[
f(x= \frac{ -A^{1/7}}{ 6^{1/7}} = \frac{ A}{6x^5} < 0 
\]
Thus, $A>0$.  

\[
\begin{aligned}
  \frac{df}{dx} &= 10x - 5A x^{-6} = 5 \left( \frac{ 2x^7 -A }{ x^6 } \right) = 0 \\
  & x = \left( \frac{A}{2} \right)^{1/7} \\
  \frac{d^2 f}{dx^2} = 10 + 30 Ax^{-7} = 10 + 3- A \left( \frac{2}{A} \right) = 70 > 0  
\end{aligned}
\]
Thus $x=\left( \frac{A}{2} \right)^{1/7}$ minimizes $f$ for $A>0$.  
\[
\begin{gathered}
  f(x= \left( \frac{A}{2} \right)^{1/7} ) = \frac{ 5 \left( \frac{A}{2} \right) + A }{ \left( \frac{A}{2} \right)^{5/7} } = 24 \\
  \Longrightarrow \boxed{ A = 2 \left( \frac{24}{7} \right)^{7/2} }
\end{gathered}
\]


\exercisehead{27} Consider $f(x) = -\frac{x^3}{3} + t^2 x$ over $0 \leq x \leq 1$.  
\[
\begin{gathered}
  f(0) = 0 , \, f(1) = -\frac{1}{3} +t^2 \Longrightarrow f(1) \gtrless 0 \text{ if } t^2 \gtrless \frac{1}{3} \\
\begin{aligned}
  f'(x) & = -x^2 + t^2 = 0  \\
  & \Longrightarrow x^2 = t^2 \text{ but } x\geq 0, \, \text{ so } x = |t| 
\end{aligned} \\
f(x^2 = t^2) = -\frac{1}{3}t^2 (x) + t^2 x = \frac{2}{3} t^2 x > 0 \text{ for } 1 \geq x \geq 0 
\end{gathered}
\]
So the minimum isn't in the interior of $[0,1]$.  It's on the end points.  
\[
\boxed{ m(t) = 0 \text{ for } |t| > \frac{1}{3} \, m(t) = \frac{-1}{3} +t^2 \text{ for } |t| < \frac{1}{3} }
\]

\exercisehead{28} 
\begin{enumerate}
\item 
\[
\begin{gathered}
E(x,t) = \frac{ |t-x|}{x} \\
M(t) = \max{ \frac{ |t-x|}{ x} } \text{ as } x = a \to x = b
\end{gathered}
\]
\[
\begin{aligned}
  \frac{ |t-x|}{x} & = \begin{cases} 
    \frac{ t-x}{x} & \text{ if } t \geq x \\
    \frac{ x-t}{x} & \text{ if } t < x \end{cases} \\
  \frac{d}{dt} \left( \frac{ |t-x|}{x}  \right) & = \begin{cases} -\frac{t}{x^2} & \text{ if } t \geq x \\
    \frac{t}{x^2} & \text{ if } t < x 
  \end{cases} \\ 
\end{aligned}
\]
Now $t,x \geq a > 0$ (this is an important, given, fact ).  So $x=t$ should be a relative minimum.  \medskip \\
So the maximum occurs at either endpoints
\[
\frac{t-a}{a} = E(a,t), \, \frac{ b-t}{b} = E(b,t)
\]

By monotonicity on $\left[ a,t \right), \left(t,b \right] $, and having shown the relative minimum of $\frac{ |t-x| }{x}$ at $x=t$, the maximum occurs at $x=a$ or $x=b$, depending upon the relationship $E(a,t) \gtrless E(b,t)$.   
\item
\[
\begin{aligned}
M(t) & = \begin{cases} \frac{t-a}{a} & \text{ if } \frac{t-a}{a} > \frac{ b-t}{b} \text{ i.e. } t \left( \frac{b+a}{ab} \right) > 2 \\
  \frac{ b-t}{b} & \text{ if } \frac{ b-t}{b} > \frac{t-a}{a}  
\end{cases} 
\frac{dM}{dt} & = \begin{cases} \frac{1}{a} & \text{ if } t > 2 \frac{ ab}{a+b} \\
  \frac{-1}{b} & \text{ if } t < \left( \frac{ab}{a+b} \right) 2 
\end{cases} 
\end{aligned}
\]
Since $\frac{dM}{dt} \gtrless 0$ when $t \gtrless \frac{2ab}{a+b} $, $M$ is minimized for $\boxed{ t = \frac{2ab}{a+b} }$
\end{enumerate}

%-----------------------------------%-----------------------------------%-----------------------------------
\subsection*{ 4.23 Exercises - Partial Derivatives  }
%-----------------------------------%-----------------------------------%-----------------------------------
\quad \\

\exercisehead{8} $f(x,y) = \frac{x}{\sqrt{x^2 +y^2 } }$.
\[
\begin{aligned}
  f_x & = \frac{1}{\sqrt{x^2 + y^2 }} + \frac{ -x^2 }{ (x^2 + y^2 )^{3/2} } = \frac{ y^2 }{ (x^2 + y^2)^{3/2} } \\
  f_y  & = \frac{ -xy }{ (x^2 + y^2)^{3/2} } 
\end{aligned} \quad 
\begin{aligned}
  f_{xx} & = \frac{ -3y^2 x }{ (x^2 + y^2)^{5/2} }  \\
  f_{yy} & = (-x) \left( \frac{ x^2 - 2y^2 }{ (x^2 + y^2 )^{5/2} } \right)
\end{aligned}
\]
\[
\begin{aligned}
  f_{xy} & = (-y) \left( \frac{ (x^2 + y^2 )^{3/2} - x \frac{3}{2} (x^2 +y^2)^{1/2} (2x) }{ (x^2 +y^2 )^3 } \right) =   \\
  & = (-y) \left( \frac{ -2x^2 + y^2 }{ (x^2 + y^2 )^{5/2} } \right)
\end{aligned} \quad 
  f_{yx}  = \frac{ 2y }{ (x^2 +y^2 )^{3/2} } + \frac{ -3y^2 y}{ (x^2 + y^2 )^{5/2} } = \frac{ (-y)(-2x^2 + y^2 )}{ (x^2 + y^2)^{5/2} }
\]


\exercisehead{9} 
\begin{enumerate}
\item 
  \[
  \begin{aligned}
    & z = (x-2y)^2 \\
    & z_x = 2(x-2y) = 2 \sqrt{z} \\ 
    & z_y = 2 (x-2y)(-2) = -4 \sqrt{z}
  \end{aligned} \quad 
x(2z) - 4z y = (x-2y)2 \sqrt{z} = 2 z 
  \]
\item 
\[
\begin{aligned}
  & z = (x^4 + y^4)^{1/2} \\
  & z_x = \frac{1}{z} 2 x^3 \\
  & z_y = \frac{2y^3}{z} 
\end{aligned}
\quad x(2z ) - 4zy = (x-2y)2\sqrt{z} = 2z
\]
\end{enumerate}

\exercisehead{10} 
\[
f = \frac{xy }{(x^2 + y^2 )^2 }, \, f_x = \frac{ y}{ (x^2 + y^2 )^2 } + \frac{ -4 x^2 y }{ (x^2 + y^2)^3 } = \frac{ y^3 - 3x^2 y }{ (x^2 +y^2 )^3 } 
\]
So 
\[
\begin{aligned}
  f_{xx} & = \frac{ -6xy (x^2 + y^2 )^3  - 3 (x^2 + y^2 )^2 (2x) (y^3 -3x^2 y ) }{ (x^2 + y^2 )^6 } = \\
  & = \frac{ 12 xy (x^2 -y^2 )}{ (x^2 + y^2 )^4 } 
\end{aligned}
\]

By label symmetry, 
\[
f_{xx} + f_{yy} = \frac{ 12 xy (x^2 - y^2 )}{ (x^2 + y^2 )^4 } + \frac{ 12 yx (y^2 - x^2 )}{ (x^2 + y^2 )^4 } = 0 
\]

%-----------------------------------%-----------------------------------%-----------------------------------
\subsection*{ 5.5 Exercises - The derivative of an indefinite integral.  The first fundamental theorem of calculus, The zero-derivative theorem, Primitive functions and the second fundamental theorem of calculus, Properties of a function deduced from properties of its derivatve }
%-----------------------------------%-----------------------------------%-----------------------------------
\quad \\
Review the fundamental theorems of calculus, Thm. 5.1 and Thm. 5.3.  Note the \textbf{differences} between the two.  
\begin{theorem}[First fundamental theorem of calculus] \quad \\
Let $f$ be integrable on $[a,x]$ \quad $\forall x \in [a,b]$  \\
Let $c \in [a,b]$ and 
\begin{equation}
  A(x) = \int_c^x f(t) dt \quad \text{ if } a \leq x \leq b 
\end{equation}
Then $\exists \, A'(x) \quad \forall x \in \, (a,b)$ where $f$ is continuous at $x$ and 
\begin{equation}
  A'(x) = f(x)
\end{equation}
\end{theorem}

\begin{theorem}[Second fundamental theorem of calculus] \quad \\
Assume $f$ continuous on open interval $I$ \\
Let $P$ be any primitive of $f$ on $I$, i.e. $P' = f \quad  \forall x \in I$ \\
Then $\forall c, \, x \in I$
\begin{equation}
  P(x) = P(c) + \int_c^x f(t) dt 
\end{equation}
\end{theorem} 

\exercisehead{6} $\sqrt{2} \frac{2}{3} x^{3/2} + \sqrt{ \frac{1}{2} } \frac{2}{3} x^{3/2}  = \sqrt{2} x^{3/2}$ \medskip \\
$\int_a^b f = \sqrt{2} (b^{3/2} -a^{3/2} ) $

\exercisehead{7} $f = x^{3/2} - 3x^{1/2} + \frac{7}{2} x^{-1/2}$; \medskip \\
$P = \frac{2}{5} x^{5/2} - 2 x^{3/2} + 7 x^{1/2}$ \medskip \\
$\int_a^b f = \frac{2}{5} (b^{5/2} - a^{5/2}) - 2 (b^{3/2} - a^{3/2} ) + 7 ( b^{1/2} - a^{1/2} )$

\exercisehead{8} $P = \frac{3}{2} x^{4/3} - \frac{3}{2} x^{2/3}$; \quad \, $x > 0$

\exercisehead{9} $P = -3\cos{x} + \frac{x^6}{3}$ 

\exercisehead{10} $P = \frac{3}{7} x^{7/3} - 5 \sin{x}$

\exercisehead{11} $f'(x) = \frac{1}{x}$ 
\[
\begin{aligned}
  f &= \sum_{k=-\infty}^{\infty} a_k x^k \\
  f' & = \sum_{k=-\infty}^{\infty} ka_k x^{k-1} = \frac{1}{x}
\end{aligned}
\]
Comparing terms, only $k=0$ would work, but the coefficient is unequivocally $0$

\exercisehead{12} 
\[
\int_0^x |t| dt = \begin{cases} \int_0^x t dt & \text{ if } x \geq 0 \\
  \int_0^x -t dt & \text{ if } x < 0 
\end{cases} = \begin{cases} 
\frac{1}{2} x^2 & \text{ if } x \geq 0 \\
\frac{-1}{2} x^2 & \text{ if } x < 0 
\end{cases} = \frac{1}{2} x |x| 
\]

\exercisehead{13}
\[
\int_0^x (t+|t|)^2 dt = \begin{cases} 
\int_0^x (2t)^2 dt & \text{ if } x \geq 0 \\
0 \text{ if } x < 0 \end{cases} = \begin{cases} \frac{4}{3} x^3 & \text{ if } x \geq 0 \\
  0 & \text{ if } x < 0 
\end{cases} = \frac{2 x^2}{3} (x+|x|)
\]

\exercisehead{14}  Using 1st. fund. thm. of calc.  
\[
\begin{gathered}
\begin{aligned}
  \int_0^x f(t) dt & = A(x) - A(0) \\
A'(x) & = f(x) \\
& \Longrightarrow 2x + \sin{2x} + 2x \cos{2x} + - \sin{2x}
\end{aligned} \quad \quad \, 
\begin{aligned}
  f' & = 2 + 4 \cos{2x} + -4x \sin{2x} - 2 \cos{2x} \\
  & = 2 + 2 \cos{2x} - 4x \sin{2x} 
\end{aligned} \\
\boxed{ \begin{aligned}
f\left( \frac{\pi}{4} \right) & = \frac{\pi}{2} \\
f'\left( \frac{ \pi}{4} \right) & = 2 - \pi 
\end{aligned}
}
\end{gathered}
\]

\exercisehead{15} $\int_c^x f(t)dt = \cos{x} - \frac{1}{2} \quad f(x) = -\sin{x} \quad c = - \frac{\pi}{6} $.  

\exercisehead{16} Suppose $f(x) = \sin{x} - 1 $ and $c=0$.    
\[
\int_0^x t\sin{t}  - t = \left. \left( -t\cos{t} + \sin{t} - \frac{1}{2} t^2 \right) \right|_0^x = \sin{x} - x\cos{x} -\frac{1}{2} x^2 
\]
So $c=0$.  

\exercisehead{17} For  $f(x) = -x^2 f(x) + 2x^15 + 2x^17$ (found by taking the derivative of $\int_0^x f = \int_x^1 t^2 f + \frac{x^{16}}{8} + \frac{x^{18}}{9} + C$,) Suppose that $f = 2x^{15}$.  
\[
\begin{gathered}
\Longrightarrow \frac{x^{16}}{8} = - \frac{x^{18}}{ 9 } + - \frac{1}{9} + \frac{x^{16}}{ 8 } + \frac{x^{18}}{ 9 } + C \\
\Longrightarrow C = \frac{1}{9 }
\end{gathered}
\]

\exercisehead{18} By plugging in $x=0$ into the defined $f(x)$, $f(x) = 3 + \int_0^x \frac{ 1 + \sin{t}}{ 2+t^2 }dt$, we get for $p(x) = a + bx + cx^2$, 
\[
a = 3
\]
Continuing on, 
\[
\begin{gathered}
\begin{aligned}
  & f' = \frac{1+\sin{x}}{ 2+x^2} ; \\
  & f'(0) = \frac{1}{2} = b 
\end{aligned} \quad 
\begin{aligned}
  & f'' = \frac{ (\cos{x})(2+x^2) - 2x (1+ \sin{x}) }{ (2+x^2)^2 } \\
  & f''(0) = \frac{1}{2}+ 2c ; c=\frac{1}{4}
\end{aligned}
\end{gathered}
\]

\exercisehead{19} 
\[
\begin{gathered}
\begin{aligned}
  f(x) & = \frac{1}{2} \int_0^x (x-t)^2 g(t) dt = \frac{1}{2} \int_0^x (x^2 - 2xt + t^2) g(t) dt = \\
  & = \frac{1}{2} \left( x^2 \int_0^x g - 2x \int_0^x tg + \int_0^x t^2 g \right) \\
\end{aligned} \\
\begin{aligned}
  f' & = x \int_0^x g + \frac{x^2}{2} g(x) - \int_0^x tg - x (xg(x)) + \frac{1}{2} x^2 g(x) = \\
  & = x\int_0^x g - \int_0^x tg \\
  f'' & = \int_0^x g + xg - xg = \int_0^x g \quad \boxed{ f''(1) = 2 }  \\
  f''' &=g \quad \boxed{ f'''(1) = 5 }
\end{aligned}
\end{gathered}
\]

\exercisehead{20}
\begin{enumerate}
\item $( \int_0^x (1+t^2)^{-3} dt )' = (1+x^2)^{-3}$ 
\item $( \int_0^{x^2} (1+t^2)^{-3} dt )' = (1+x^4)^{-3} (2x) = \frac{2x}{ (1+x^4)^3 }$  
\item $( \int_{x^3}^{x^2} (1+t^2)^{-3} dt )' = (1+x^4)^{-3}(2x) - (1+x^6)^{-3}(3x^2) = \frac{ 2x}{ (1+x^4)^3 } - \frac{3x^2}{ (1+x^6)^3 }$ 
\end{enumerate}

\exercisehead{21}
\[
f'(x) = \left( \int_{x^3}^{x^2} \frac{t^6}{1+t^4} dt \right)' = \left( \frac{ x^{12}}{ 1+ x^8 } \right)(2x) - \left( \frac{ x^18}{ 1+ x^{12} } \right)3x^2
\]

\exercisehead{22} \begin{enumerate}
\item \[
\begin{aligned}
  f(x) & = 2x (1+x) + x^2 = 2x + 3x^2 \\
  f(2) & = 16
\end{aligned}
\]
\item $\frac{d}{dx} \left( \int_{a(x)}^{b(x)} f(t)dt \right) = f(b)b' - f(a)a'$ 
  \[
  \begin{aligned}
    2x + 3x^2 & = f(x^2) (2x) \\
    f(x^2) & = (1 + \frac{3x}{2} ) \\
    f(2) = 1 + \frac{3 \sqrt{2}}{2}
  \end{aligned}
  \]
\item $\int_0^{f(x)} t^2 dt = x^2(1+x) $
\[
\begin{gathered}
  2x + 3x^2  = (f(x))^2 f'(x) \\
  \Longrightarrow f^3(2) = 3(4)(3) = 9(4) = 36 \\
  f(2) = 36^{1/3}
\end{gathered}
\]
\item \[
\begin{gathered}
  \frac{d}{dx} \left( \int_0^{ x^2(1 +x)} f(t) dt \right) = 1 = f(x^2(1+x)) ( 2x (1+x) + x^2 ) \\
  x =1  \quad (f(2))(5) = 1 \Longrightarrow f(2) = \frac{1}{5}  
\end{gathered}
\]
\end{enumerate}

\exercisehead{23} 
\[
\begin{gathered}
a^3 - 2a \cos{a} + (2-a^2) \sin{a} = \int_0^a f^2(t) dt \\
3x^2 - 2 \cos{x} + 2x \sin{x} + -2x \sin{x} + (2-x^2 ) \cos{x} = 3x^2 -x^2 \cos{x} = f^2(x) \\
f(x) = x \sqrt{ 3 - \cos{x} } \\
\boxed{ f(a) = a \sqrt{ 3 -\cos{a}} }
\end{gathered}
\]

\exercisehead{24} $f(t) = \frac{t^2}{2} + 2 t \sin{t}$
\begin{enumerate}
\item 
\[
f' = 2t + 2 \sin{t} + 2t \cos{t} \quad f'(\pi) = 2\pi - 2\pi = 0 
\]
\item 
\[
\begin{gathered}
  f'' = 2 + 2 \cos{t} + 2 \cos{t} + -2t \sin{t} = 2 + 4 \cos{t} - 2t \sin{t}  \\
  f''\left( \frac{\pi}{2} \right) = 2 - \pi 
\end{gathered}
\]
\item $f''\left( \frac{3\pi}{2} \right) = 0 $
\item $f\left( \frac{ 5\pi}{2} \right) = \frac{ 25 \pi^2}{ 8 } + 5 \pi $
\item $f(\pi) = \frac{ \pi^2}{2} $
\end{enumerate}

\exercisehead{25} 
\begin{enumerate} 
\item
\[
\begin{gathered}
\begin{aligned}
  \frac{df}{dt} & = \frac{1 + 2 \sin{ \pi t }\cos{ \pi t } }{ 1+t^2 } =v(t) \\
  a(t) & = \frac{ 2 \pi (\cos{ (2\pi t )}) (1+t^2 ) - 2 t ( \sin{ ( 2 \pi t ) } ) }{ (1+t^2 )^2 } 
\end{aligned} \\
a(t=2) = a(t=1) = \frac{4\pi}{4} = \pi 
\end{gathered}
\]
\item $v(t = 1) \frac{1}{2}$
\item $v(t) = \pi (t-1) + \frac{1}{2} ; \, t > 1 $
 \item 
\[
f(t) - f(1) = \int_1^t v(t) dt = \int_1^t \pi (t-1) + \frac{1}{2} = \left. \left( \frac{\pi t^2 }{ 2 } - \pi t + \frac{1}{2} t \right) \right|_1^t = \frac{ \pi t^2 }{2} + -\pi t + \frac{t}{2} + \frac{ \pi}{2} - \frac{1}{2}
\]
\end{enumerate}

\exercisehead{26} 
\begin{enumerate}
\item \[
  \begin{gathered}
    f''(x) > 0 \, \forall x \quad f'(0) = 1 ; f'(1) = 0 \\
\int_0^1 f''(t)dt = f'(1) -f'(0) = 0 -1 < 0
\end{gathered}
\]
Thus, it's impossible, since $f''(x)>0$, so $\int_0^1 f''(t)dt > 0 $
\item \[
\begin{gathered}
  \int_0^1 \left( 3 - \frac{ \pi}{2} \sin{ \frac{ \pi x }{2} }\right) dx = \left. \left( 3x + \cos{ \frac{ \pi x }{2} } \right) \right|_0^1 = 3 - 1 =2 \\
f(x) = \frac{3 x^2 }{2} + \frac{2}{\pi } \sin{ \frac{ \pi x}{2} } + C
\end{gathered}
\]
\item $f''(0) > 0 \, \forall x \quad f'(0) = 1; \, f(x) \leq 100 \, \forall x > 0 $
\[
\begin{gathered}
  \int_a^b f''(t) dt = f'(b) -f'(a); \quad \int_c^k f'(t)dt = f(k) - f(c) \\
  \int_0^b f'' = f'(b) - f'(0) = f'(b) - 1 \gtrless 0 \, \text{ if } b \gtrless 0 \\
  \begin{aligned}
    \int_c^k (f'(b) - 1 ) db & = f(k) -f(c) - (k-c) > 0 \, \text{ if } k > c > 0 \\
& \begin{aligned}
f(k)  -f(c) & > k-c \\
f(k) - f(0) & > k - 0 \\
f(100) - f(0) & > 100 
\end{aligned}
\end{aligned}
f(x) \leq 100 \text{ is untrue for all $x>0$ }
\end{gathered}
\]
\item $f''(x) = e^x > 0 \quad \, f'(x) = e^x; \quad \, f'(0) = 1 \quad \, f(x)= e^x \quad \, \forall x < 0, e^x < 1$
\end{enumerate}

\exercisehead{27} $f''(t) \geq 6$.  \quad \quad $b-a = \frac{1}{2}$.  \quad \quad $f'(0) =0$
\[
\begin{gathered}
  \int_a^b f'' = f'(b) - f'(a) \geq 6(b-a) = 3 \quad \text{ since } b- a = \frac{1}{2} \\
  \int_0^a f'' = f'(a) - f'(0) = f'(a) \geq 6 (a-0) = 6a \\
  \text{ If } a = \frac{1}{2}, \, f'(1/2) \geq 3
\end{gathered}
\]
Then by intermediate value theorem, with $f$ being continuous and $f'(0)=0, \, f'(1/2) \geq 3$, \, $f'$ must take on the value of $3$ somewhere between $0$ and $3$.  Thus there is an interval $[a,b]$ of length $1/2$ where $f' \geq 3$.  

%-----------------------------------%-----------------------------------%-----------------------------------
\subsection*{ 5.8 Exercises - The Leibniz notation for primitives, Integration by substitution }
%-----------------------------------%-----------------------------------%-----------------------------------
\quad \\
\exercisehead{1} $\int \sqrt{ 2x +1 } dx = \frac{1}{3} (2x +1)^{3/2}$.  

\exercisehead{2} $\int x \sqrt{ 1+ 3x} = \frac{2x}{9} (1+3x)^{3/2} + - \frac{4}{135} (1+3x)^{5/2} $

\exercisehead{3} 
\[
\begin{gathered}
  \int x^2 \sqrt{ x + 1 } = \frac{ 2x^2 (x+1)^{3/2}}{ 3 } - \frac{ 8 x (x+1)^{5/2}}{ 15} + \frac{ 16(x+1)^{7/2}}{ 105} \\
  \text{ since } \\
\begin{aligned}
  \left( \frac{ 2x^2 (x+1)^{3/2}}{ 3 } \right)' & = x^2 ( x+1)^{1/2} + \frac{ 4x (x+1)^{3/2}}{ 3 } \\
  \left( \frac{ 8x (x+1)^{5/2}}{ 15 } \right)' & = \frac{4}{3} x (x+1)^{3/2} + \frac{ 8 (x+1)^{5/2}}{ 15} \\ 
  \left( \frac{ 16 (x+1)^{7/2}}{ 105} \right)' & = \frac{ 8 (x+1)^{5/2}}{ 15}
\end{aligned}
\end{gathered}
\]

\exercisehead{4}
\[
\begin{gathered}
  \int \frac{x dx}{ \sqrt{ 2 - 3x }} = \frac{ 2x (2-3x)^{1/2}}{ -3 } + \frac{ 4 (2-3x)^{3/2}}{ -27 } \\
  \int_{-2/3}^{1/3} \frac{ x dx}{ \sqrt{ 2-3x}} = -2/9 -4/27 - \left( 8/9 - 32/27 \right) = -2/27 
\end{gathered}
\]

\exercisehead{5}
\[
\int \frac{(x+1) dx}{ ((x+1)^2 + 1)^3 } = \frac{ ((x+1)^2 + 1 )^{-2} }{ -4 }
\]

\exercisehead{6} 
\[
\int \sin^3{x}  = \int \sin{x} (1- \cos^2{x}) = -\cos{x} + \frac{1}{3} \cos^3{x}
\]

\exercisehead{7} 
\[
\int x^{1/3} (1+x) = \frac{3}{4} x^{4/3} + \frac{3}{7} x^{7/3} = \frac{3}{4} (z-1)^{4/3} + \frac{3}{7} (z-1)^{7/3}
\]

\exercisehead{8} $\frac{\sin^{-2}{x}}{ -2 }$

\exercisehead{9} $ \left. \frac{ (4-\sin{2x})^{3/2}}{ -3 } \right|_0^{\pi/4} = \frac{ 3^{3/2} - 8 }{ -3 }$

\exercisehead{10} $(3+ \cos{x})^{-1}$

\exercisehead{11} $2\cos^{-1/2}{x}$

\exercisehead{12} $\left. 2\cos{\sqrt{x+1}} \right|_3^{8} = 2 (\cos{3} - \cos{2})$

\exercisehead{13} $-\frac{ \cos{x^n}}{ n } $

\exercisehead{14} $ \frac{ (1-x^6)^{1/2}}{ -3 }$ 

\exercisehead{15} 
\[
\int t(1+t)^{1/4} dt = \int (x-1) x^{1/4} dx = \frac{4}{9} x^{9/4} - \frac{4}{5} x^{5/4} = \frac{4}{9} (1+t)^{9/4} - \frac{4}{5} (1+t)^{5/4} 
\]

\exercisehead{16} $\int (x^2 + 1)^{-3/2} dx = ?$
\[
\left( \frac{ x}{\sqrt{x^2 + 1 } } \right)' = \frac{ \sqrt{x^2 + 1 } - x^2/ \sqrt{x^2 + 1 }}{ x^2 + 1} = \frac{1}{ (x^2 + 1)^{3/2} }
\]

\exercisehead{17} 
\[
(8x^3 + 27)^{5/3} \left( \frac{3}{5} \right)\left( \frac{1}{24} \right) = \frac{1}{40} (8x^3 + 27)^{5/3}
\]

\exercisehead{18} $\frac{3}{2} (\sin{x} -\cos{x})^{2/3}$

\exercisehead{19} 
\[
\begin{aligned}
  \int \frac{ x dx }{ \sqrt{ 1 + x^2 + (1+x^2)^{3/2} } } & = \int \frac{ \frac{1}{2} du }{ \sqrt{ u + u^{3/2} } } = \\
  & = \frac{1}{2} \int \frac{ du }{ \sqrt{ u } \sqrt{ 1 + u^{1/2}} } = 2 (1+u^{1/2})^{1/2}  = \\
  & = 2(1+\sqrt{ 1 + x^2 } )^{1/2} + C
\end{aligned}
\]

\exercisehead{20} 
\[
\int \frac{ (x^2 - 2x +1)^{1/5} dx }{ 1-x} = \int \frac{ - (x-1)^{2/5}}{ x-1} dx = - \int (x-1)^{-3/5} dx = -5/2 (x-1)^{2/5}
\]

\exercisehead{21} 
Thm. 1.18.  invariance under translation.  $\int_a^b f(x) dx = \int_{a+c}^{b+c} f(x-c) dx $.  \\ \\
Thm. 1.19.  expansion or contraction of the interval of integration.  
\[
\int_a^b f(x) dx = \frac{1}{k } \int_{ka}^{kb} f\left( \frac{x}{k} \right) dx 
\]

\[
\begin{gathered}
\begin{aligned}
y & x + c \\
dy & = dx 
\end{aligned}
\quad \int_a^b f(x) dx = \int_{a+c}^{b+c} f(y-c) dy \\ \\
\begin{aligned} 
y & = kx \\
dy & = k dx 
\end{aligned} \quad \int_a^b f(x) dx = \frac{1}{k} \int_{ka}^{kb} f\left( \frac{y}{k} \right) dy 
\end{gathered}
\]

\exercisehead{22}
\[
\begin{aligned}
  &  F\left( \frac{x}{a}, 1 \right) = \int_0^{x/a} \frac{ u^p}{ (u^2 + 1^2 )^q } du \quad \begin{aligned} u & = \frac{t}{a} \\ du & = \frac{dt}{a} \end{aligned} \\
  & F\left( \frac{x}{a}, 1 \right) = \frac{1}{a} \int_0^x \frac{ (t/a)^p dt }{ \left( \left( \frac{t}{a} \right)^2 + 1^2 \right)^q } = \\
  & \quad = a^{-p -1+2q } \int_0^x \frac{ t^p}{ (t^2 + 1^2)^q } dt = a^{-p-1+2q} F(x,a)
\end{aligned}
\]

\exercisehead{23}
\[
\begin{aligned}
\int_x^1 \frac{dt}{ 1+t^2 } & = F(1) - F(x) \\
\int_1^x \frac{dt}{ 1+t^2 } & = F(x) - F(1) \\
\int_1^{1/x} \frac{dt}{ 1+t^2 } & = F\left(\frac{1}{x}\right) - F(1) 
\end{aligned}
\quad
\begin{gathered}
\begin{aligned} 
u & = \frac{1}{t} \\
du & = \frac{-1}{t^2} dt , \, \frac{-1}{u^2} du = dt
\end{aligned} \\
\begin{aligned}
  \int_1^x \frac{dt}{ 1 +t^2 } &= \int_1^{1/x} \frac{ -du}{ u^2\left(1+ \frac{1}{u^2} \right) } = \\
    & = -\int_1^{1/x} \frac{du}{u^2+1} = \int_{1/x}^1 \frac{dt}{ t^2 + 1 }
\end{aligned}
\end{gathered}
\]

\exercisehead{24} 
\[
\int_0^1 x^m (1-x)^n dx = - \int_1^0 (1-u)^m (u^n) du = \int_0^1 (1-x)^m x^n dx \, \text{ using } \, \begin{aligned} u & = 1-x \\ x & = 1 - u \end{aligned}
\]

\exercisehead{25} 
\[
\begin{gathered}
  \cos^m{x} \sin^m{x} = \left( \frac{\sin{2x}}{2} \right)^m = 2^{-m} \sin^m{2x} \\
  \begin{aligned} 
    \int_0^{\frac{\pi}{2}} \cos^m{x}\sin^m{x} dx & = 2^{-m} \int_0^{\pi/2} \sin^m{2x}dx = 2^{-m} \int_0^{\pi} \frac{1}{2} \sin^m{x} dx = 2^{-m-1} \int_0^{\pi} \sin^m{x} dx = \\
    & = -2^{-m-1} \int_{\pi/2}^{-\pi/2} \sin^m{ \left( \frac{\pi}{2} - x \right) }dx = 2^{-m -1} \int_{-\pi/2}^{\pi/2} \cos^m{x} dx = 2^{-m} \int_0^{\pi/2} \cos^m{x} dx
  \end{aligned}
\end{gathered}
\]

\exercisehead{26} 
\begin{enumerate}
\item
\[
\begin{gathered}
  \begin{aligned}
    u & = \pi - x \\
    x & = \pi - u 
  \end{aligned} \quad 
\begin{aligned}
  \int_0^\pi x f(\sin{x}) dx & = \int_{\pi}^0 (\pi - u )f(\sin{ (\pi - u ) } ) (-du) = \int_0^{\pi} (\pi - u ) f(\sin{u}) du = \\
  & = \pi \int_0^{\pi} f(\sin{x}) dx - \int_0^{\pi} x f(\sin{x}) dx 
\end{aligned} \\
\Longrightarrow \int_0^{\pi} x f(\sin{x}) dx = \frac{\pi}{2} \int_0^{\pi} f(\sin{x}) dx 
\end{gathered}
\]
\item 
\[
\begin{gathered}
  \begin{aligned}
    u & = \cos{x} \\
    du & = -\sin{x} dx 
\end{aligned} \quad 
\begin{aligned}
  \int_0^{\pi} \frac{ x\sin{x}}{ 1 + \cos^2{x} } dx & = \int_0^{\pi} \frac{x\sin{x} }{ 2- \sin^2{x} } = \frac{\pi}{2} \int_0^{\pi} \frac{ \sin{x}}{ 2 - \sin^2{x}} dx = \frac{\pi}{2} \int_0^{\pi} \frac{\sin{x}}{ 1  + \cos^2{x}} dx = \\
  & = - \frac{ \pi}{2} \int_1^{-1} \frac{du}{ 1 + u^2} = \frac{\pi}{2} \int_{-1}^1 \frac{du}{ 1 + u^2 } = \pi \int_{-1}^1 \frac{dx}{ 1 + x^2 } 
\end{aligned}
\end{gathered}
\]
\end{enumerate}

\exercisehead{27} 
\[
\begin{aligned}
  x & = \sin{u}  \\
  dx & = \cos{u}
\end{aligned} \quad \, 
\int_0^1 (1-x^2)^{n-\frac{1}{2}} dx = \int_0^{\pi/2} (\cos^2{u})^{n-\frac{1}{2}} \cos{u} du = \int_0^{\pi/2} \cos^{2n}{u} du
\]

%-----------------------------------%-----------------------------------%-----------------------------------
\subsection*{ 5.10 Exercises - Integration by Parts }
%-----------------------------------%-----------------------------------%-----------------------------------
\quad \\
\exercisehead{1} $\int x \sin{x} = -x \cos{x} + \sin{x}$

\exercisehead{2} $\int x^2 \sin{x} = - x^2 \cos{x} + 2 x \sin{x} + 2 \cos{x}$ 

\exercisehead{3} $\int x^3 \cos{x} = x^3 \sin{x} + 3 x^2 \cos{x} - 6x \sin{x} + - 6 \cos{x}$

\exercisehead{4} $\int x^3 \sin{x} = -x^3 \cos{x} + 3x^2 \sin{x} + 6x \cos{x} - 6 \sin{x}$

\exercisehead{5} $\int \sin{x} \cos{x} = -\frac{1}{4} \cos{2x} = -\frac{1}{4} (\cos^2{x} - \sin^2{x})$ 

\exercisehead{6} $\int x \sin{x} \cos{x} dx = \int \frac{x}{2} \sin{2x} = - \frac{x \cos{2x}}{4} + \frac{ \sin{2x}}{8}$

\exercisehead{7} $\int \sin^2{x} = \int \sin{x} \sin{x} = - \sin{x} \cos{x} + \int \cos^2{x} $ \\
$\int \sin^2{x} dx = \frac{-1}{4} \sin{2x} + \frac{x}{2}$ 

\exercisehead{8} 
\[
\begin{gathered}
  \int \sin^n{x} dx = -\cos{x} \sin^{n-1}{x} + \int (n-1) \sin^{n-2}{x} \cos^2{x} \quad \, \begin{aligned} u & = \sin^{n-1}{x}  \\
    dv & = \sin{x} dx \end{aligned} \\
  \begin{aligned} 
    \int \sin^n{x} & = -\cos{x} \sin^{n-1}{x} + \int (n-1) \sin^{n-2}{x} (1- \sin^2{x}) = \\
    & = - \cos{x} \sin^{n-1}{x} + \int (n-1) \sin^{n-2}{x} - (n-1) \int \sin^n{x} 
  \end{aligned} \\
  \int \sin^n{x} = \frac{-1}{n} \sin^{n-1}{x} \cos{x} + \frac{ (n-1)}{n} \int \sin^{n-2}{x} 
\end{gathered}
\]

\exercisehead{9} 
\begin{enumerate}
  \item \[
\begin{gathered}
  \int \sin^2{x} = \frac{-1}{2} \sin{x} \cos{x} + \frac{1}{2} \int 1 = \frac{-1}{2} \sin{x} \cos{x} + \frac{1}{2} x \\
  \int_0^{\pi/2} \sin^2{x} dx = \frac{\pi}{4} 
\end{gathered}
\]
  \item $\int_0^{\pi/2} \sin^4{x} = \left. \frac{-1}{4} \sin^3{x} \cos{x}  \right|_0^{\pi/2} + \frac{3}{4} \int_0^{\pi/2} \sin^2{x} = \frac{3\pi}{16} $
  \item $ \int_0^{\pi/2} \sin^6{x} = \frac{5}{6} \int_0^{\pi/2} \sin^4{x}= \boxed{ \frac{5\pi}{32} } $
\end{enumerate}

\exercisehead{10} 
\begin{enumerate}
  \item \[
\begin{gathered}
  \int \sin^3{x} dx = \frac{-1}{3} \sin^2{x} \cos{x} + \frac{2}{3} \int \sin{x} = -\frac{1}{6} \sin{2x} \cos{x} - \frac{2}{3} \cos{x} = \frac{-3}{4} \cos{x} + \frac{1}{12} \cos{3x} \quad \text{ since } \\
  \begin{aligned}
  \frac{-3}{4} \cos{x} + \frac{1}{12} \cos{3x} & = \frac{-3}{4} \cos{x} + \frac{1}{12} (\cos{x} \cos{2x} - \sin{2x} \sin{x} ) = \\
  & =  -\frac{3}{4} \cos{x} + \frac{1}{12} (\cos{x} (1- 2 \sin^2{x}) + - 2 \sin^2{x} \cos{x} )  = \boxed{ \frac{-2}{3} \cos{x} - \frac{1}{3} \sin^2{x} \cos{x} }
  \end{aligned}
\end{gathered}
\]
  \item 
\[
\begin{gathered}
  \int \sin^4{x} dx = \frac{-1}{4} \sin^3{x} \cos{x} + \frac{3}{4} \int \sin^2{x} = \frac{-1}{4} \sin^3{x} \cos{x} + \frac{3}{4} ( \frac{x}{2} - \frac{ \sin{2x}}{4} ) = \frac{-1}{4} \sin^3{x} \cos{x} + \frac{3x}{8} - \frac{3\sin{2x}}{16} \\
  \text{ Now } \frac{1}{32} \sin{4x} = \frac{1}{32} (2\sin{2x} \cos{2x} ) = \frac{1}{8} (\sin{x} \cos{x} (1- 2 \sin^2{x}) = \frac{ \sin{2x}}{16} - \frac{1}{4} \sin^3{x} \cos{x} \\
  \Longrightarrow \frac{-1}{4} \sin^3{x} \cos{x} + \frac{3x}{8} - \frac{3\sin{2x}}{16} = \frac{3x}{8} - \frac{1}{4} \sin{2x} + \frac{1}{32} \sin{4x} 
\end{gathered}
\]
  \item \[
\begin{gathered}
  \begin{aligned}
    \int \sin^5{x} dx & = \int \sin^4{x} \sin{x} dx = -\cos{x} \sin^4{x} + \int \cos^2{x} 4 \sin^3{x} = \\
    & = -\cos{x} \sin^4{x} + 4 (\int \sin^3{x} - \int \sin^5{x} ) = -\cos{x} \sin^4{x} + 4 \int \sin^3{x} - 4 \int \sin^5{x}
\end{aligned} \\
  5 \int \sin^5 dx = -\cos{x} \sin^4{x} + 4 \int \sin^3{x} \\
  \begin{aligned}
    5 \int \sin^5 dx & = - \cos{x} (1- \cos^2{x})^2 + 4 ( \frac{-3}{4} \cos{x} + \frac{1}{12} \cos{3x} ) \\
    & = - \cos{x} (1- 2 \cos^2{x} + \cos^4{x} ) + -3 \cos{x} + \frac{1}{3} \cos{3x} \\
    & = -\cos{x} + 2 \cos^3{x} - \cos^5{x} - 3\cos{x} + \frac{1}{3} (\cos{x} \cos{2x} - \sin{x} \sin{2x} ) = \\
    & = -4 \cos{x} + 2 \cos^3{x} - \cos^5{x} + \frac{1}{3} (4 \cos^3{x} - 3 \cos{x} ) = -5 \cos{x} + \frac{ 10 \cos^3{x}}{ 3 } - \cos^5{x} 
\end{aligned} \\
  \int \sin^5{x} dx = - \cos{x} + \frac{2 \cos{3x}}{ 3 } - \frac{1}{5} \cos^5{x} 
\end{gathered}
\]
My solution to the last part of this exercise \textbf{ conflicts } with what's stated in the book.  
\end{enumerate}

\exercisehead{11} 
\begin{enumerate}
\item 
\[
\begin{gathered}
\begin{aligned}
  \int x \sin^2{x} dx & = ( \int \sin^2{x} )x - \int (\sin^2{t}) = \frac{x^2}{2} - \frac{ x \sin{2x}}{ 4 } - \left( \frac{x^2}{4} + \frac{ \cos{2x}}{8} \right) = \\
 & =   \frac{x^2}{8} - \frac{ x \sin{2x}}{4}  - \frac{ \cos{2x}}{8} 
\end{aligned} \\
\text{ we had used } \int \sin^2{x} = \frac{x}{2} - \frac{ \sin{2x}}{4} 
\end{gathered}
\]
\item 
\[
\begin{gathered}
\begin{aligned}
  \int x \sin^3{x} & = \frac{-3x}{4} \cos{x} + \frac{x}{12} \cos{3x} - \int -\frac{3}{4} \cos{x} + \frac{1}{12} \cos{3x} = \\
& = \frac{-3x}{4} \cos{x} + \frac{x}{12} \cos{3x} + \frac{3}{4} \sin{x} + \frac{-\sin{3x}}{36} 
\end{aligned} \\
\int \sin^3{x} = \frac{-3}{4} \cos{x} + \frac{1}{12} \cos{3x} 
\end{gathered}
\]
\item \[
\begin{aligned}
  \int x^2 \sin^2{x} dx & = x^2 \left( \frac{x}{2} - \frac{\sin{2x}}{4} \right) - \int 2x \left( \frac{x}{2} - \frac{ \sin{2x} }{4} \right) = \frac{x^3}{2} - \frac{ x^2 \sin{2x}}{4} - \frac{1}{3} x^3 + \int \frac{x \sin{2x}}{2} = \\
  & = \frac{x^3}{6} - \frac{ x^2 \sin{2x}}{ 4 } + \frac{1}{2} \left( \frac{ -x \cos{2x}}{2} + \frac{ \sin{2x}}{4} \right) = \\
  & = \boxed{ \frac{x^3}{6} - \frac{x^2 \sin{2x}}{ 4 } -  \frac{ x \cos{2x}}{4 } + \frac{ \sin{2x}}{8 } }
\end{aligned}
\]
\end{enumerate}

\exercisehead{12} 
\[
\begin{gathered}
  \begin{aligned}
    \int \cos^n{x} dx & = \int \cos^{n-1}{x} \cos{x} dx = \cos^{n-1}{x} \sin{x} + \int (n-1) \cos^{n-2}{x} \sin^2{x} = \\
    & = \cos^{n-1}{x} \sin{x} + (n-1) \int \cos^{n-2}{x} - \int (n-1) \cos^n{x} 
\end{aligned} \\
  \Longrightarrow \int \cos^n{x} = \frac{ \cos^{n-1}{x} \sin{x}}{ n} + \left( \frac{n-1}{n} \right) \int \cos^{n-2}{x} 
\end{gathered}
\]

\exercisehead{13}
\begin{enumerate}
\item $\int \cos^2{x} = \frac{ \sin{2x}}{5} + \frac{1}{2} x $  
\item $ \int \cos^3{x} = \frac{ \cos^2{x} \sin{x}}{3} + \frac{2}{3} \sin{x} = \frac{3}{4} \sin{x} + \frac{1}{12} \sin{3x}$ since
\[
  \begin{aligned}
    \frac{1}{12} \sin{3x} & = \frac{1}{12} (\sin{2x} \cos{x} + \sin{x} \cos{2x} ) = \frac{1}{6} \sin{x} \cos^2{x} + \frac{1}{12} \sin{x} (2 \cos^2{x} - 1 ) = \\
    & = \frac{1}{3} \sin{x} \cos^2{x} - \frac{1}{12} \sin{x}
\end{aligned} 
\]
\item 
\[
\begin{gathered}
  \int \cos^4{x} dx = \frac{ \cos^3{x} \sin{x}}{4} + \frac{3}{4} \left( \frac{1}{2} x + \frac{1}{4} \sin{2x} \right) = \frac{3}{8} x + \frac{3}{16} \sin{2x} + \frac{ \cos^3{x} \sin{x}}{4 } \\
\sin{4x} = 2 \sin{2x} \cos{2x} = 4 \sin{x} \cos{x} (2 \cos^2{x} - 1 ) = 8 \sin{x} \cos^3{x} - 2 \sin{2x} \quad \text{ then } \\
\int \cos^4{x} dx = \frac{3}{8} x + \frac{1}{4} \sin{2x} + \frac{1}{32} \sin{4x}
\end{gathered}
\]
\end{enumerate}

\exercisehead{14} 
\[
\begin{gathered}
  \int \sqrt{ 1 - x^2 } dx = x \sqrt{ 1 - x^2 } + \int \frac{ x^2}{ \sqrt{ 1- x^2 } } dx \\
  \int \frac{ x^2}{ \sqrt{ 1 - x^2 } } dx = \int \frac{ x^2 - + 1 }{ \sqrt{ 1 - x^2 + 1 - 1 } } = - \int \sqrt{ 1 - x^2 } + \frac{1}{ \sqrt{ 1 - x^2 } } \\
  x^2 = x^2 - 1 + 1 \\
  \Longrightarrow \int \sqrt{ 1 - x^2 } dx = \frac{1}{2} x \sqrt{ 1 - x^2 } + \frac{1}{2} \int \frac{1}{ \sqrt{ 1 - x^2 } }
\end{gathered}
\]

\exercisehead{15} 
\begin{enumerate}
\item 
\[
\begin{gathered}
  \int (a^2 - x^2 )^n dx  = x(a^2 - x^2)^n - \int n (a^2- x^2)^{n-1} (-2x) x dx = x (a^2 - x^2)^n + 2n \int x^2 ( a^2 - x^2)^{n-1} dx \\
  \int x^2 (a^2 - x^2)^{n-1} dx = \int ( (x^2 - a^2) + a^2)(a^2 - x^2)^{n-1} dx = \int -(a^2-x^2)^n + a^2(a^2- x^2)^{n-1} dx \\
  \Longrightarrow \int (a^2 - x^2)^n dx = \frac{ x(a^2 - x^2)^n }{ 2n+1 } + \frac{ 2a^2 n}{ 2n+1} \int (a^2 - x^2)^{n-1} dx 
\end{gathered}
\]
\item 
\[
\begin{gathered}
  \begin{aligned}
    \int (a^2- x^2) dx & = \frac{ x (a^2- x^2) }{ 3}  + \frac{ 2a^2}{3} x = \frac{ -x^3 }{3} + a^2 x \\
    \int (a^2- x^2)^{5/2} dx & = \frac{ x(a^2 - x^2)^{5/2}}{ 6} + \frac{a^2 5}{6} \int (a^2 - x^2)^{3/2} dx \\
    \int (a^2 - x^2)^{3/2} dx & = \frac{ x (a^2 - x^2)^{3/2} }{ 4 } + \frac{ 3a^2}{4} \int (a^2- x^2 )^{1/2} dx 
\end{aligned} \\
  \begin{aligned}
    \int (a^2 - x^2)^{1/2} & = a \int \sqrt{ 1 - \left( \frac{x}{a} \right)^2 } dx = a^2 \int \cos^2{\theta} d\theta = \\
& = a^2 \int \frac{ 1 + \cos{2\theta}}{ 2 } = a^2 \left( \frac{ \theta}{2} + \frac{ \sin{2\theta}}{4} \right) = a^2 \left( \arcsin{ \frac{x}{a} } + \frac{1}{2} \frac{x}{a} \sqrt{ 1 - \left( \frac{x}{a} \right)^2 } \right) 
\end{aligned} \quad \quad 
  \begin{aligned}
    \sin{\theta} & = \frac{x}{a} \\
    \cos{\theta} d\theta & = \frac{dx}{a} 
\end{aligned} \\
  \begin{aligned}
    \int_0^a \sqrt{ a^2 - x^2 } & = a^2 \left( \frac{ \pi}{2} - 0 \right) = \frac{ \pi a^2}{2} \\
    \int_0^a (a^2 - x^2 )^{3/2} dx & = \frac{3a^2}{4} \left( \frac{ \pi a^2}{2} \right) = \frac{ 3 \pi a^4}{8} \\
 \end{aligned} \\
\boxed{   \int_0^a (a^2 - x^2)^{5/2} dx  = \frac{ 5a^2}{6} \left( \frac{3\pi a^4}{8} \right) = \boxed{ \frac{5}{16} \pi a^6 } }
\end{gathered}
\]
\end{enumerate}

\exercisehead{16} $I_n(x) = \int_0^x t^n (t^2 + a^2)^{-1/2} dt $
\begin{enumerate}
  \item 
\[
\begin{aligned}
  I_n(x) & = (t^2 + a^2)^{1/2} t^{n-1} - \int (n-1)t^{n-2} (t^2 + a^2)^{1/2} = t^{n-1} (t^2 + a^2)^{1/2} - (n-1) \int \frac{ t^{n-2} (t^2+a^2) }{ (t^2 + a^2)^{1/2} } \\
  (n)I_n & = x^{n-1} (x^2 + a^2)^{1/2} - a^2 (n-1) \int \frac{ t^{n-2}}{ (t^2 + a^2)^{1/2} } = x^{n-1} \sqrt{ x^2 + a^2 } - (n-1)a^2 I_{n-2} 
\end{aligned}
\]
  \item $n=5; \, x=2; \, a = \sqrt{5}$. 
\[
\begin{aligned}
  I_1(2) & = \int_0^2 x (x^2 + 5)^{-1/2} dx = \left. (x^2 + 5)^{1/2} \right|_0^2 = 3 - \sqrt{5} \\
  5 I_5(2) & = \int_0^2 t^5 (t^2 + 5)^{-1/2} dt = 2^{5-1} (4+5)^{1/2} - 5(5-1) I_3(2) = 48 - 20 I_3(2) \\
  3 I_3(2) & = 2^2 \sqrt{ 4+5} - 5 (3-1) I_1(2) = 12 - 10 (3-\sqrt{5} ) \\
  I_5(2) & = \frac{1}{5} (48- 20 (-6 + \frac{ 10 \sqrt{5}}{ 3} ) ) = \boxed{ \frac{ 168}{5} - \frac{40 \sqrt{5}}{ 3 } }
\end{aligned}
\] 
\end{enumerate}

\exercisehead{17} 
\[
\begin{gathered}
  \int t^3 (c +t^3)^{-1/2} dt = \frac{ t 2 ( c+t^3)^{1/2} }{ 3 } - \int \frac{ 2 ( c+t^3)^{1/2}}{ 3 } \\
  \int_{-1}^3 t^3 ( 4  + t^3)^{-1/2} dt = \left. \frac{ t 2 (4+t^3)^{1/2} }{ 3 } \right|_{-1}^3 - \frac{2}{3} \int_{-1}^3 (4+ t^3)^{1/2}  = \boxed{ 2 \sqrt{31} + \frac{ 2 \sqrt{3}}{ 3 } - \frac{2}{3} (11.35) }
\end{gathered}
\]

\exercisehead{18}
\[
\begin{gathered}
  \int \frac{ \sin^{n+1}{x} }{ \cos^{m+1}{x} } dx = \int \sin^n{x} \left( \frac{ \sin{x}}{ \cos^{n+1}{x} } \right) dx = \frac{ \sin^n{x} }{ m \cos^m{x} } - \int \frac{ n \sin^{n-1}{x} }{ m \cos^{m-1}{x} } \\
  \Longrightarrow \int \frac{ \sin^{n+1}{x} }{ \cos^{m+1}{x} } dx = \frac{ \sin^n{x}}{ m \cos^m{x} } - \frac{n}{m} \int \frac{ \sin^{n-1}{x} }{ \cos^{m-1}{x} } 
\end{gathered}
\]

\exercisehead{19} 
\[
\begin{gathered}
\begin{aligned}
  \int \frac{ \cos^{m+1}{x} }{ \sin^{n+1}{x} } dx & = \int \cos^m{x} \left( \frac{ \cos{x} dx}{ \sin^{n+1}{x} } \right) = \cos^m{x} \frac{ 1}{ -n \sin^n{x} } - \int \frac{ m \cos^{m-1}{x} }{ -n \sin^m{x} } = \\
  & = \frac{ \cos^m{x} }{ -n \sin^n{x} } + \frac{-m}{n} \int \frac{ \cos^{m-1}{x}}{ \sin^{n-1}{x} } 
\end{aligned} \\
\int \cot^2{x} = \int \frac{ \cos^{1+1}{x} }{ \sin^{1+1}{x} } = \frac{-1}{1} \frac{ \cos^1{x}}{ \sin^1{x} } - \frac{1}{1} \int dx = -\cot{x} - x \\
\int \cot^4{x} dx = \int \frac{ \cos^{3+1}{x}}{ \sin^{3+1}{x} } = -\frac{1}{3} \frac{ \cos^3{x} }{ \sin^3{x} }- \int \frac{ \cos^{3-1}{x} }{ \sin^{3-1}{x} } = \frac{-1}{3} \cot^3{x} - (-\cot{x} - x ) = \boxed{ \frac{-1}{3} \cot^3{x} + \cot{x} + x }
\end{gathered}
\]

\exercisehead{20} 
\begin{enumerate}
\item \[
  \begin{gathered}
    \int_0^2 t f''(t) dt = 2 \int_0^1 t f''(2t) dt \quad \quad n =2 \\
\end{gathered}
\]
\item 
  \[
    \begin{aligned} 
    \int_0^1 x f''(2x) dx & = \frac{1}{2} \int_0^2 t f''(t) dt = \frac{1}{2} \left( \left. t f'(t) \right|_0^2 - \int_0^2 f'(t) dt \right) = \frac{1}{2} \left( 2 f'(2) - (f(2) - f(0) ) \right) = \boxed{4} 
    \end{aligned} 
    \]
\end{enumerate}

\exercisehead{21} 
\begin{enumerate}
\item Recall Theorem 5.5, the second mean-value theorem for integrals:
\begin{equation*}
\int_a^b f(x)g(x) dx = f(a) \int_a^c g(x)dx + f(b) \int_c^b g(x) dx  
\end{equation*}
\[
\begin{aligned}
  \int_a^b \sin{ \phi(t) } \left( \frac{ \phi'(t) }{ \phi'(t) } \right) dt & = \frac{1}{ \phi'(a) } \int_a^c \phi'(t) \sin{ \phi(t) } + \frac{1}{ \phi'(b) } \int_a^b \phi'(t) \sin{ \phi(t) } = \\
  & = \frac{1}{ \phi'(a) }\left. \cos{ \phi(t) } \right|_a^c + \frac{1}{ \phi'(b) } (\cos{ \phi(b) } - \cos{ \phi(a) } ) \leq \frac{4}{m} 
\end{aligned} \quad \quad \text{ where } 
\frac{1}{m} \geq \frac{1}{ \phi'(t) } \quad \forall t \in [a,b]
\]
\item  $\phi(t) = t^2$ ; \quad \quad \, $\phi'(t) = 2t > 2a$ if $t > a$ 
\[
\left| \int_a^x \sin{ t^2} dt \right| \leq \frac{4}{2a} = 2a
\]
\end{enumerate}

%-----------------------------------%-----------------------------------%-----------------------------------
\subsection*{ 5.11 Miscellaneous review exercises }
%-----------------------------------%-----------------------------------%-----------------------------------
\quad \\

\exercisehead{1} $g(x) = x^n f(x)$; \quad \quad $f(0)=1$ \medskip \\
$g'(x) = nx^{n-1}f(x) + x^n f'(x)$ ; \quad $g'(0) = 0$ especially if $n \in \mathbb{Z}^+$ (just note that if negative integer values are included, $g'(0)$ easily blows up)  \bigskip \\ 

$g^j(x) = \sum_{k=0}^h \binom{j}{k} \frac{n!}{ (n-k)! } x^{n-k} f^{j-k}(x)$ \medskip \\
If $j < n$ , then $g^j(0)$, since each term contains some power of $x$ \bigskip \\

If $j \geq n$, \[
\begin{aligned}
  & g^j(x) = \sum_{k=0}^n \binom{j}{k} \frac{ n!}{ (n-k)! } x^{n-k} f^{(j-k)}(x) \\
  & g^j(0) = \frac{j!}{(j-n)! }f^{(j-n)}(0) 
\end{aligned}
\]

If $j =n$, $g^n(0) = n!$

\exercisehead{2}
\[
\begin{gathered}
  \begin{aligned}
    P(x) & = \sum_{j=0}^5 a_j x^j \\
    P'(x) & = \sum_{j=1}^5 j a_j x^{j-1} \\
    P''(x) &  = \sum_{j=2}^5 j (j-1)a_j x^{j-2} 
  \end{aligned} \quad \quad \, 
  \begin{aligned}
    P(0) & = 1 = a_0 \\
    P'(0) & = 0 = a_1 \\
    P''(0) & = 0 = 2(1)a_2 
  \end{aligned}
 \quad \quad a_1 = a_2 = 0  \\
 \Longrightarrow 
\begin{aligned}
  P(x) & = a_5 x^5 + a_4 x^4 + a_3 x^3 + 1 \\
  P'(x) & = 5a_5 x^4 + 4a_4 x^3 + 3a_3 x^2  \\
  P''(x) & = 20a_5 x^3 + 12a_4 x^2 + 6a_3 x 
\end{aligned} \quad \quad \, 
\begin{aligned}
  P(1) & = a_5  + a_4  + a_3  + 1 =2 \\
  P'(1) & = 5a_5  + 4 a_4  + 3a_3  =0  \\
  P''(1) & = 20a_5  + 12a_4  + 6a_3  = 0  \\
\end{aligned} \\
\text{ Solve for the undetermined coefficients by Gauss-Jordan elimination process } \\
\left[ \begin{matrix} 
5 & 4 & 3 \\
20 & 12 & 6 \\
1 & 1 & 1 
\end{matrix} \right] \left[ 
  \begin{matrix}
    a_5 \\
    a_4 \\
    a_3 
\end{matrix} 
\right] = \left[ \begin{matrix} 
    0 \\
    0 \\
    1
\end{matrix}
\right]
\quad \quad 
\left[ \begin{matrix} 5 & 4 & 3 \\ 
 20 & 12 & 6 \\
 1 & 1 & 1 
\end{matrix} \right| \left. \begin{matrix} 0 \\ 0 \\ 1 \end{matrix} \right] = \left[ \begin{matrix} & 1 & 0 \\
    & 0 & 1 \\
    1 & 0 & 0 \end{matrix} \right| \left. \begin{matrix} -15 \\ 10 \\ 6 \end{matrix} \right]  \\
\Longrightarrow a_5 = 6 \quad \, a_4 = -15 \quad \, a_3 = 10  \quad \quad \, \boxed{ P(x) = 6x^5 - 15x^4 + 10 x^3 + 1 }
\end{gathered}
\]

\exercisehead{3} If $f(x) = \cos{x}$ and $g(x) = \sin{x}$, Prove that $f^{(n)} = \cos{(x+ \frac{ n \pi}{2}) }$ and $g^{(n)}(x) = \sin{ (x+\frac{n \pi }{2} ) }$

\[
\begin{gathered}
  \begin{aligned}
    f^{(n)}(x) & = \cos{ (x+ \frac{n \pi}{2} ) } = \begin{cases} 
    \sin{x} (-1)^{j+1} & \text{ if } n = 2j + 1 \\
    \cos{x} ( -1)^j & \text{ if } n = 2j 
    \end{cases} \\
      g^{(n)}(x) & = \sin{ (x+ \frac{n \pi}{2} ) } = \begin{cases} 
    \cos{x} (-1)^{j} & \text{ if } n = 2j + 1 \\
    \sin{x} ( -1)^j & \text{ if } n = 2j 
      \end{cases} \\
  \end{aligned} \\
  \begin{aligned}
    f(x) & = \cos{x} \\
    f'(x) & = -\sin{x} \\
    f''(x) & = -\cos{x} \\
    f'''(x) & = \sin{x} \\
    f''''(x) & = \cos{x} 
\end{aligned} \quad \quad \,
  \begin{aligned}
    & f^{(2j)}(x) = \cos{x} (-1)^j \\
    &  f^{ (2(j+1))}(x) = (\cos{x}(-1)^j)' = \cos{x} (-1)^{j+1} \\
    &  f^{(2j+1) }(x) = \sin{x}(-1)^{j+1} \\
    & f^{(2j+3)}(x) = (\sin{x}(-1)^{j+1} ) = \sin{x} (-1)^{j+2}
\end{aligned} \\
\quad \\
  \begin{aligned}
    g(x) & = \sin{x} \\
    g'(x) & = \cos{x} \\
    g''(x) & = -\sin{x} \\
    g'''(x) & = -\cos{x} \\
    g''''(x) & = \sin{x} 
\end{aligned} \quad \quad \,
  \begin{aligned}
    & g^{(2j)}(x) = \sin{x} (-1)^j \\
    &  g^{ (2(j+1))}(x) = (\sin{x}(-1)^j)'' = \sin{x} (-1)^{j+1} \\
    &  g^{(2j+1) }(x) = \cos{x}(-1)^{j} \\
    & g^{(2j+3)}(x) = (\cos{x}(-1)^{j+1} ) 
\end{aligned} 
\end{gathered}
\]

\exercisehead{4}
\[
\begin{gathered}
  \begin{aligned}
    & h'(x) = f'g + fg' \\
    & h''(x) = f''g + 2f' g' +fg''
  \end{aligned} \quad \quad \, h^{(n)} = \sum_{k=0}^n \binom{n}{k} f^{(k)}g^{(n-k)}  \\
\begin{aligned}
  h^{(n+1)} & = \sum_{k=0}^n \binom{ n}{k} \left( f^{(k+1)} g^{(n-k)} + f^{(k)} g^{ (n-k+1)} \right) = \\
  & = f^{(1)}g^{(n)} + fg^{(n+1)} + \sum_{k=1}^{n-1} \binom{ n}{k} \left( f^{(k+1)} g^{(n-k)} + f^{(k)} g^{ (n-k+1)} \right) + f^{(n+1)}g + f^{(n)}g^{(1)}  \\
  & = f^{(1)}g^{(n)} + fg^{(n+1)} + \sum_{k=2}^{n} \frac{ n!}{ (n-k+1)!(k-1)!} \left( f^{(k)} g^{(n-k+1)} + \sum_{k=1}^{n-1} \frac{ n!}{ (n-k)!k! } f^{(k)} g^{ (n-k+1)} \right) + \\
  & \quad + f^{(n+1)}g  + f^{(n)}g^{(1)} 
\end{aligned} \\
\text{ Now } \frac{ n!}{ (n-k+1)!(k-1)!} + \frac{n!}{ (n-k)!k!} = (k + (n-k +1) ) \left( \frac{ n!}{ (n+1 - k)!(k)! } \right) = \binom{n+1}{k} \text{ so then } \\
\begin{aligned}
  h^{(n+1)} & = fg^{(n+1)} + \sum_{k=1}^n \binom{ n+1}{k} f^{(k)}g^{(n+1-k)} + f^{(n+1)}g = \sum_{k=0}^{n+1} \binom{n+1}{k} f^{(k)}g^{(n+1-k)}
\end{aligned} 
\end{gathered}
\]
By induction, this formula is true.  

%\exercisehead{4} 
%\[
%\begin{gathered}
%\begin{aligned}
%  h^{(0)} & = f(x) g(x) \\
%  h^{(1)}(x) & = f'g + fg' \\
%  h'' & = f''g + 2f'g' + fg'' 
%\end{aligned} \\
%\text{ Suppose the $n$the case } h^{ (n) }(x) = \sum_{k=0}^{n} \binom{n}{k} f^{(k)}(x) g^{(n-k)}(x) \\
%\text{ Start from } h^{(n)}(x) = fg^{(n)} + (n)f^{(1)}g^{(n-1)} + \sum_{k=2}^{n-2} \binom{n}{k} f^{(k)}g^{(n-k)} + nf^{(n-1)}g^{(1)} + f^{(n)}g \\
%\begin{aligned}
%  h^{(n+1)} & = fg^{(n+1)} + (n+1)f^{(1)}g^{(n)} + nf^{(2)}g^{(n-1)} + \\ 
%  & + \sum_{k=2}^{n-2} \binom{n}{k} f^{(k+1)}g^{(n-k)} + \binom{n}{k} f^{(k)}g^{(n-k+1)} + nf^{(n-1)}g^{(2)} + (n+1)f^{(n)}g^{(1)} + f^{(n+1)}g \\
%  & = fg^{(n+1)} + (n+1)f^{(1)}g^{(n)} + nf^{(2)}g^{(n-1)} + \sum_{k=3}^{n-1} \frac{n! }{(k-1)!(n-k+1)!} f^{(k)}g^{(n-k+1)} + \\
%  & +  \sum_{k=2}^{n-2} \frac{n!}{ k!(n-k)! } f^{(k)}g^{(n-k+1)} + nf^{(n-1)}g^{(2)} + (n+1)f^{(n)}g^{(1)} + f^{(n+1)}g = \\
%  & = fg^{(n+1)} + (n+1)f^{(1)}g^{(n)} + nf^{(2)}g^{(n-1)} + \frac{ n(n-1)}{2} f^{(n-1)}g^{(2)}   + \sum_{k=3}^{n-2} \left( \frac{n!(k + (n-k+1)) }{(k)!(n-k+1)!} \right) f^{(k)}g^{(n-k+1)} + \\
%  & + \frac{n(n-1)}{2} f^{(2)}g^{(n-1)}  + nf^{(n-1)}g^{(2)} + (n+1)f^{(n)}g^{(1)} + f^{(n+1)}g = \\
%  & = fg^{(n+1)} + (n+1)f^{(1)}g^{(n)} +  \frac{n(n+1)}{2} f^{(2)}g^{(n+1-2)} f^{(2)}g^{(n-1)}+ \sum_{k=3}^{n-2} \left( \frac{n!(n+1) }{(k)!(n+1-k)!} \right) f^{(k)}g^{(n+1-k)} +  \\
%  & + \frac{n(n+1)}{2}f^{(n+1-2)}g^{(2)} + (n+1)f^{(n)}g^{(1)} + f^{(n+1)}g = \\
%  & = \sum_{k=0}^{n+1} \binom{n+1}{k} f^{(k)}(x) g^{(n+1-k)}(x)
% \end{aligned}
%\end{gathered}
%\]

\exercisehead{5} \begin{enumerate}
  \item 
\[
\begin{gathered}
  f^2 + g^2 = f(-g') + gf' \\
  Y = f^2 + g^2 \quad \quad Y' = 2ff' + 2gg' = 2(gf + 2g(-f) ) = 0 \Longrightarrow Y = C \\
  Y = C = f^2 + g^2 \quad \quad f(0) = 0 ; \quad g(0) = 1 \Longrightarrow \boxed{ C = 1 } 
\end{gathered}
\]
  \item 
\[
\begin{gathered}
\begin{aligned}
  h & = (F-f)^2 + (G-g)^2 \\ 
  & = f'(x) = g(x), \quad g'(x) = -f(x) ; \quad f(0) = 0 ; \quad g(0) = 1 \\
  h' & = 2 (F-f)(F' - f') + 2(G-g)(G'-g'); \quad h'(0) = 2(0) + 2(0) = 0 \\
  & = 2(F-f)(G-g) + 2 (G-g)(-F + f) = 0 \quad \forall x 
\end{aligned} \\
h(x) = C \Longrightarrow h(x) = (F(x) - f(x))^2 + (G(x) - g(x))^2 \\
h(0) = 0 \quad \text{ so } C = 0 \\
\Longrightarrow F =f; \, G =g 
\end{gathered}
\]
\end{enumerate}

\exercisehead{6} $\frac{df}{du} 2x = 3x^2$ \quad \quad $f'(4) = \frac{3x}{2} = 3$ where we had used the substitution \\
$u=x^2 \quad \quad u = 4; x = 2$

\exercisehead{7} $\frac{dg}{du} = u^{3/2}; \quad g(u) = \frac{2}{5} u^{5/2}$ \quad \quad $g(4) = \frac{2}{5} 2^5 = \boxed{ \frac{64}{5} }$ 

\exercisehead{8}
\[
\begin{gathered}
\begin{aligned}
  \int_0^x \frac{ \sin{t}}{ t+1 } dt & = \frac{1}{ 0 + 1 } \int_0^c \sin{t} dt + \frac{1}{ x+1 } \int_c^x \sin{t} dt = \\
  & = \left. -\cos{t} \right|_0^c + \frac{1}{x+1} \left. -\cos{t} \right|_c^x = -\cos{c} + 1 + \frac{-1}{x+1} (\cos{x} - \cos{c} ) = \\
& = \frac{ -x \cos{c} + x - \cos{c} + 1 - \cos{x} + \cos{c} }{ x+1 } = \frac{ x (1- \cos{c}) + ( 1 - \cos{c}) }{ x+1} > 0 
\end{aligned}
\end{gathered}
\]

\exercisehead{9} $y=x^2$ is the curve for $C$.  $y=\frac{1}{2}x^2$ is the curve for $C_1$.  
\[
\begin{gathered}
  \int_0^b (x^2 - \frac{1}{2} x^2 ) = \frac{1}{6} b^3 \quad \quad P: \, (b,b^2) \\
\text{ Assume $C_2$ is of the form $kx^2$ } \\
  \int_0^c (kx^2 - x^2) + \int_c^b (b^2 -x^2) = \frac{ (k-1)}{3} c^3 + b^2 (b-c) + - \frac{1}{3} (b^3 - c^3) \quad \quad \begin{aligned} 
    kc^2 & = b^2 \\
    k & = \frac{b^2}{c^2} 
\end{aligned} \\
\Longrightarrow \frac{ (b^2 - c^2)c }{ 3 } + b^3 - cb^2 - \frac{b^3}{3} + \frac{c^3}{3} = \frac{2b^3}{3} - \frac{2}{3} cb^2 = \frac{2}{3} b^2 (b-c) \\
\text{ Now } A(A) = A(B) \\
\Longrightarrow \frac{2}{3} b^2 (b-c)  = \frac{1}{6} b^3 \Longrightarrow b = \frac{4}{3}c \\
\text{ so then } \boxed{ kx^2 = \frac{16}{9}x^2 }
\end{gathered}
\]

\exercisehead{10} 
\begin{enumerate} \item $|Q(h) - 0| = \left| \frac{ f(h)}{h} \right| = \begin{cases} 
  \frac{h^2}{|h|} & \text{ if $x$ is rational } \\
  0 & \text{ if $x$ is irrational } \end{cases} $.  \\
For now, consider $0 < h < \delta$; \quad let $\delta(\epsilon; h = 0 ) = \epsilon$  \\
\[
|Q(h)  - 0 | = \left| \frac{ f(h)}{ h} \right| = \begin{cases} h & \text{ if $x$ is rational }  \\ 0 & \text{ if $x$ is irrational } \end{cases} < \epsilon 
\]
\item 
\[
\left| \frac{ f(h) -f(0) }{ h } - 0 \right| = \epsilon \Longrightarrow f'(0) = 0 
\]
\end{enumerate}

\exercisehead{11} $\int (2+3x) \sin{ bx} dx = \frac{-2}{5} \cos{5x} + - \frac{3x}{5} \cos{5x}  + \frac{3}{25} \sin{5x} $

\exercisehead{12} 
$\frac{4}{3} (1+x^2)^{3/2}$

\exercisehead{13} 
$ \left. \frac{ (x^2- 1 )^{10 } }{ 20 } \right|_{-2}^1 = \frac{ -3^{10}}{ 20 }$

\exercisehead{14} 
\[
\begin{aligned}
  \frac{1}{3} \int_0^1 \frac{6x + 7 + 2 }{ (6x+7)^3 } dx & = \frac{1}{3} \left. \left( \frac{ -(6x + 7)^{-1}}{6} + \frac{ - (6x + 7 )^{-2}}{ 6 } \right) \right|_0^1 = \frac{1}{3} \left( - \left( \frac{1}{13} \right) \left( \frac{1}{6} \right) + \frac{1}{42} + \frac{-1}{ (6)13^2 } + \frac{1}{ (6) 49 } \right) = \\
  & = \boxed{ \frac{37}{ 8281 } }
\end{aligned}
\]

\exercisehead{15} $\int x^4 ( 1 + x^5)^5 dx = \frac{ ( 1 + x^5)^6}{ 30 }$

\exercisehead{16} 
\[
\begin{gathered}
  \begin{aligned}
    \int_0^1 x^4 ( 1 - x)^{20} dx & = \quad \quad \quad \, \begin{aligned}
      u & = 1 -  x \\
      x & = 1 - u 
\end{aligned} \\
    & = - \int_1^0 (1-u)^4 u^{20} du = \int_0^1 (1-u)^4 u^{20} du = \int_0^1 (1 + 4 (-u) + 6u^2 + -4u^3 + u^4 )u^{20} du = \\
    & = \frac{ 1^{21} }{21} + \frac{ -4 1^22}{ 22 } + \frac{ 6 1^{23}}{ 23 } + \frac{ -4 1^{24}}{ 24} + \frac{1^{25}}{ 25 } =  \\
    & = \boxed{ \frac{1}{265650} }
\end{aligned} \\
\text{ Make sure to check your arithmetic.}
\end{gathered}
\]

\exercisehead{17} $\int_1^2 x^{-2} \sin{ \frac{1}{x} } dx = \left. \left( \cos{\frac{1}{x}} \right) \right|_1^2 = \cos{ \frac{1}{2}} - \cos{1} $

\exercisehead{18} $\int \sin{ (x-1)^{1/4} } dx $
\[
\begin{gathered}
  \begin{aligned}
    u & = (x-1)^{1/4} \\
    du & = \frac{1}{4} ( x-1)^{-3/4} dx = \frac{1}{4} \frac{1}{ u^3} dx 
  \end{aligned} \Longrightarrow \int \sin{ (x-1)^{1/4} } dx = \int (\sin{u})4 u^3 du = 4 \int u^3 \sin{u} du \\
\begin{aligned}  \int u^3 \sin{u} du & =  -u^3 \cos{u} + 3 u^2 \sin{u} + 6 u \cos{u} + 06 \sin{u}  \\
  & = -(x-1)^{3/4}\cos{ (x-1)^{1/4}} + 3 (x-1)^{1/2} \sin{( x-1)^{1/4}} + 6 (x-1)^{1/4}\cos{ (x-1)^{1/4}} - 6 \sin{(x-1)^{1/4}}
\end{aligned} \\
\begin{gathered} \sin{ (x-1)^{1/4} } dx  =  \\
 = \boxed{   -4 (x-1)^{3/4} \cos{ (x-1)^{1/4} } + 24 (x-1)^{1/4} \cos{ (x-1)^{1/4}} + 12 (x-1)^{1/4} \sin{ (x-1)^{1/4}} - 24 \sin{ (x-1)^{1/4} } }
\end{gathered}
\end{gathered}
\]

\exercisehead{19} $\int x \sin{x^2}\cos{x^2} dx = (1/4) \sin^2{x^2} + C$

\exercisehead{20} $\int \sqrt{ 1 + 3 \cos^2{x}} \sin{2x} dx = \int \sqrt{ 1 + 3 \cos^2{x}} 2 \sin{x} \cos{x} dx = \int u^{1/2} \frac{du}{-3} = \frac{2 u^{3/2}}{-9} = \boxed{ \frac{-2}{9} (1 + 3 \cos^2{x})^{3/2} }$, where we had used this substitution:
\[
u = 1 + 3 \cos^2{x} \quad \quad \, \begin{aligned}
  du & = -6 \cos{x} \sin{x} dx \\
  \frac{du}{-3} & = 2 \cos{x} \sin{x} dx 
\end{aligned}
\]  

\exercisehead{21} $\int_0^2 375 x^5 (x^2 + 1 )^{-4} dx$
\[
\begin{gathered}
  \begin{aligned}
    u & = x^2 + 1 \\
    du & = 2x dx \\
    (u-1)^2 & = x^4 
\end{aligned}  \quad  \, \Longrightarrow \begin{aligned}
    \int_0^2 \frac{375}{2} du (u-1)^2 u^{-4} & = \frac{375}{2} \int_1^5 du \frac{ (u^2 - 2 u +1)}{u^4} = \frac{375}{2} \int_1^5 du \left( \frac{1}{u^2} - \frac{2}{u^3} + \frac{1}{u^4} \right) = \\
    & = \frac{375}{2} \left. \left( \frac{-1}{u} + \frac{1}{u^2} + \frac{1}{-3u^3} \right) \right|_1^5 = \boxed{ 64 = 2^6}
\end{aligned}
\end{gathered}
\]

\exercisehead{22} $\int_0^1 (ax+b) (x^2 + 3x+2)^{-2} dx = \frac{3}{2}$  \quad \bigskip \\
Since $\left. \left( \frac{-1}{ x^2 + 3x + 2 } \right) \right|_0^1 = \frac{-1}{6} + \frac{1}{2} = \frac{2}{3}$, \medskip \\
then if $a = 9/2$, $b = \frac{27}{2}$, we'll obtain $3/2$

\exercisehead{23} $I_n = \int_0^1 (1-x^2)^n dx$
\[
\begin{gathered}
  \begin{aligned}
    I_n & = \int_0^1 (1-x^2)^n dx = \boxed{ \left. x (1-x^2)^n \right|_0^1 - \int_0^1 x n (1-x^2)^{n-1}(-2x) dx } = \\
    & = 2 \int_0^1 x^2 n (1-x^2)^{n-1} dx = 2n \int_0^1 ((x^2 - 1) + 1)(1-x^2)^{n-1} dx = \\
    & = 2n I_{n-1} -  2n I_n  
\end{aligned} \quad \, \Longrightarrow I_n = \left( \frac{2n}{ 2n + 1} \right) I_{n-1} \\
  \begin{aligned}
    I_2 & = \int_0^1 (1-x^2)^2 dx = \int_0^1 dx ( 1 - 2x^2 + x^4) = \left. \left( x - \frac{2 x^3}{3} + \frac{1}{5} x^5 \right) \right|_0^1 = \boxed{  \frac{8}{15} } \\
    I_3 & = \frac{ 6}{7} \frac{8}{15} \boxed{ \frac{ 16}{35} } \\
    I_4 & = \frac{8}{9}I_3 = \boxed{ \frac{ 128}{315} } \\
    I_5 & = \frac{10}{11} I_4 = \boxed{ \frac{ 256}{ 693} }
  \end{aligned}
\end{gathered}
\]

\exercisehead{24} $F(m,n) = \int_0^x t^m (1+t)^n dt$; \quad \quad $m > 0, \quad n >0$ \\
\[
\begin{gathered}
  F(m,n)  = \left. \frac{ t^{m+1}}{ m+1} (1+t)^n \right|_0^x - \int_0^x \frac{ t^{m+1}}{ m+1} n (1+t)^{n-1} dt = \frac{ x^{m+1} (1+x)^n }{ m+1 } - \frac{n}{m+1} F(m+1, n-1)  \\
  \boxed{ (m+1) F(m,n) + n F(m+1,n-1) = x^{m+1}(1+x)^n } \\
  \begin{gathered}
    F(11,1) = \int_0^x t^{11} (1+t)^1 dt = \int_0^x t^11 + t^{12} = \left. \left( \frac{ t^{12}}{12} + \frac{t^{13}}{13} \right) \right|_0^x = \frac{ x^{12}}{12} + \frac{ x^{13}}{ 13} \\
    11 F(10,2) + 2 \left( \frac{ x^{12}}{12} + \frac{ x^{13}}{13} \right) = x^{11} (1+x)^2 \\
    \boxed{ F(10,2) = \frac{ x^{13}}{13} + \frac{ x^{12}}{6} + \frac{ x^{11}}{11} }
  \end{gathered}
\end{gathered}
\]

\exercisehead{25} $f(n) = \int_0^{\pi/4} \tan^n{x} dx$ 
\begin{enumerate}
\item 
\[
\begin{gathered}
\text{  Use this extremely \textbf{ important } fact: } \boxed{ \int_a^b fg = f(b) \int_c^b g + f(a) \int_a^c g } \\
\boxed{ f(n+1) = \int_0^{\pi/4} \tan^n{x} \tan{x} = \int_c^{\pi/4} \tan^n{x} < \int_0^{\pi/4} \tan^n{x} = f(n)  }
\end{gathered}
\]
\item \[
\begin{aligned}
  f(n+2) + f(n) & = \int_0^{\pi/4} \tan^n{x} \tan^2{x} + \int_0^{\pi/4} \tan^n{x} = \int_0^{\pi/4} \tan^n{x} (\sec^2{x}) = \\
  & = \left. \frac{ \tan^{n+1}{x} }{n+1} \right|_0^{\pi/4} = \boxed{ \frac{1}{n+1} }
\end{aligned}
\]
\item 
\[
\begin{gathered}
  f(n+2) + f(n) = \frac{1}{ n+1} < f(n+1) + f(n) < 2 f(n) \\
  \frac{1}{ n-1} = f(n-2) + f(n) > f(n-1) + f(n) > 2 f(n) \\
  \Longrightarrow \frac{1}{n+1} < 2 f(n) < \frac{1}{n-1} 
\end{gathered}
\]
\end{enumerate}

\exercisehead{26} $f(0), \quad \, f(\pi) =2$ \quad \quad \, $\int_0^{\pi} (f(x) + f''(x)) \sin{x} dx = 5$ \\
\[
\begin{gathered}
  \int_0^{\pi} f'' \sin{x} = f' \sin{x} - \int f' \cos{x} = - \int f' \cos{x} = -f \cos{x} - \int f \sin{x} \\
  \int_0^{\pi} (f + f'') \sin{x} dx = \int f\sin{x} + - (f(\pi)(-1) - f(0) ) - \int f \sin{x} = 2 + f(0) = 5 \quad \quad \Longrightarrow \boxed{ f(0) = 3 }
\end{gathered}
\]

\exercisehead{27}
\[
\begin{aligned}
  \int_0^{\pi/2} \frac{ \sin{x} \cos{x}}{ x + 1 } dx & = \int_0^{\pi/2} \frac{ \sin{2x}}{ 2 x + 2 } dx = \frac{1}{2} \int_0^{ \pi } \frac{ \sin{x} }{ x + 2} dx = \frac{1}{2} \left( \frac{ -\cos{x} }{ x + 2 } - \int \frac{ \cos{x}}{ (x+2)^2 } \right) = \\
  & = \frac{1}{2} \left( \frac{1}{ \pi + 2 } + \frac{1}{2} - A \right) = \frac{ 4 + \pi }{ 4 (\pi + 2 ) } - \frac{A}{2} 
\end{aligned}
\]

\exercisehead{28} 
\[
\begin{gathered}
  \begin{aligned}
    & \int \frac{ dx}{ x \sqrt{ a + bx } } = \frac{ 2 \sqrt{ a + bx } }{ bx} + \frac{2}{b} \int \frac{ \sqrt{ a + bx} }{ x^2 } \\
    & \int \frac{ \sqrt{ a + bx}}{ x } = \frac{ \frac{2}{3b} (a+bx)^{3/2}}{ x } + \int \frac{ \frac{2}{3b} (a+bx)^{3/2} }{ x^2 } = \frac{ 2 }{ 3b} \sqrt{ a + bx} \left( \frac{ a}{x} + b \right) + \frac{2}{3b} \int \sqrt{ a+bx} \left( \frac{a}{x^2} + \frac{b}{x} \right)
  \end{aligned} \\
  \Longrightarrow \int \frac{ \sqrt{ a + bx} }{ x } = a \int \frac{ dx}{ x \sqrt{ a + bx} } + 2 \sqrt{ a + bx }
\end{gathered}
\]

\exercisehead{29} 
\[
\begin{gathered}
\begin{aligned}
  \int x^n \sqrt{ (ax+b) } dx & = \frac{ 2x^n (ax+b)^{3/2}}{ 3a} - \int \frac{ nx^{n-1} 2 (ax+b)^{3/2} }{ 3a} = \frac{ 2x^n (ax+b)^{3/2}}{ 3a} - \frac{2n}{3a} \int x^{n-1} (ax+b) \sqrt{ ax+b} = \\
  & = \frac{ 2x^n (ax+b) \sqrt{ ax+b} }{ 3a}  - \frac{ 2n}{3} \int x^n \sqrt{ ax+ b} - \frac{2nb}{3a} \int x^{n-1}\sqrt{ax+b} 
\end{aligned} \\
\boxed{ \int x^n \sqrt{ax+b} dx = \frac{ 2}{ (2n+3)a } \left( x^n ( ax+b)^{3/2} - nb \int x^{n-1} \sqrt{ ax+b} \right) + C }  \quad \quad n \neq \frac{-3}{2}
\end{gathered}
\]

\exercisehead{30} 
\[
\begin{gathered}
  \begin{aligned}
    \int \frac{ x^m}{ \sqrt{ a+bx} } dx & = \frac{ 2 x^m (a+bx)^{1/2} }{ b} - \int \frac{ m x^{m-1} 2 (a+bx)^{1/2} }{ b } \\
    & = \frac{2}{b} x^m (a+bx)^{1/2} - \frac{2m}{b} \int \frac{ x^{m-1} (a+bx) }{ \sqrt{ a+bx} } = \frac{2}{b} x^m (a+bx)^{1/2} - 2m \int \frac{ x^m}{ \sqrt{ a+bx} } - \frac{2ma}{b} \int \frac{ x^{m-1}}{ \sqrt{ a+bx} } 
  \end{aligned} \\
  \boxed{ \int \frac{ x^m}{ \sqrt{ a+bx } } dx = \frac{1}{2m+1} \frac{2}{b} x^m (a+bx)^{1/2} - \frac{2ma}{ b (2m+1) } \int \frac{ x^{m-1}}{ \sqrt{ a+bx} } }
\end{gathered}
\]

\exercisehead{31} 
\[
\begin{gathered}
  \begin{aligned}
    \int \frac{dx}{ x^n \sqrt{ ax+b} } & = 2 \frac{ \sqrt{ ax+b} }{ ax^n } + \int \frac{ n 2 \sqrt{ ax+ b}}{ ax^{n+1} } \sqrt{ \frac{ ax+b}{ ax+b} } = \frac{ 2 \sqrt{ ax+b}}{ ax^n } + \frac{2n}{a} \int \frac{ ax+b}{ x^{n+1} \sqrt{ ax+b} } = \\
    & = \frac{ 2 \sqrt{ ax+b}}{ ax^n } + 2n \int \frac{1}{ x^n \sqrt{ ax+b} } + \frac{2nb}{a} \int \frac{1}{ x^{n+1} \sqrt{ ax+b} } 
  \end{aligned} \\
  \Longrightarrow (1-2n) \int \frac{ dx}{ x^n \sqrt{ ax+b} } - \frac{ 2 \sqrt{ ax+b} }{ ax^n } = \int \frac{ b \left( \frac{2n}{a} \right) }{ x^{n+1} \sqrt{ ax+b} } \\
  \boxed{ \int \frac{1}{ x^n \sqrt{ ax+b} } = \frac{ - \sqrt{ ax+b}}{ (n-1) bx^{n-1} } - \frac{ (2n-3) a}{ (2n-2)b } \int \frac{1}{ x^{n-1} \sqrt{ ax+b} } }
\end{gathered}
\]

\exercisehead{32} I derived the formulas for this and Exercise 33 by doing the \textbf{ following trick}.   
\[
\begin{gathered} \boxed{ 
  \begin{aligned}
    (C^{m+1}S^{1-n})' & = (m+1) C^m (-S^{2-n}) + C^{m+2}(1-n)S^{-n} = -(m+1)C^m S^{2-n} + (1-n)C^m S^{-n}(1-S^2) = \\
    & = - (m+1 + 1 - n)C^m S^{2-n} + (1-n)C^m S^{-n} = -(m-n+2)C^mS^{2-n} + (1-n)C^m S^{-n}
  \end{aligned}  }\\
  \Longrightarrow \int C^m S^{-n} = \frac{ - (C^{m+1}S^{1-n} ) }{ n-1} - \frac{ (m-n+2)}{ n-1} \int C^m S^{2-n}
\end{gathered}
\]

\exercisehead{33} 
\[
\begin{gathered}
  \begin{aligned}
    (C^{m-1}S^{1-n})' & = (m-1)C^{m-2}(-S^{2-n}) + (1-n)C^m S^{-n} \\
    & = -(m-1)C^{m-2}(S^{-n})(1-C^2) + (1-n)C^m S^{-n} = \\
    & = -(m-1)C^{m-2}S^{-n} + (m-1) C^m S^{-n} + (1-n)C^m S^{-n} 
  \end{aligned} \\
  C^{m-1}S^{1-n} = - (m-1) \int C^{m-2}S^{-n} + (m-n)\int C^m S^{-n} \\
  \frac{m-1}{m-n} \int C^{m-2}S^{-n} + \frac{ C^{m-1}S^{1-n} }{ m-n} = \int C^m S^{-n}
\end{gathered}
\]

\exercisehead{34}
\begin{enumerate}
  \item $P'(x) - 3 P(x) = 4 - 5 x + 3x^2$
    \[
      \begin{aligned}
	P & = \sum_{j=0}^n a_j x^j \\
	P' & = \sum_{j=1}^n a_j jx^{j-1} = \sum_{j=0}^{n-1} a_{j+1}(j+1)x^j 
      \end{aligned} \quad \quad \, \Longrightarrow \sum_{j=0}^{n-1} (a_{j+1}(j+1) - 3a_j) x^j  = 4 - 5 x + 3x^2 \\
    \] 
Generally, we can say
\[
a_{j+1} = \frac{3}{j+1} a_j \quad \, \text{ if $ j \geq 3$ }
\]
We also have
\[
\begin{aligned}
  & a_1 (1) - 3a_0 = 4  \\
  & a_2(2) - 3a_1 = -5 \\
  & a_3(3) - 3a_2 = 3 \quad \Longrightarrow a_3 - a_2 = 1 
\end{aligned}
\]
Then let $a_2 = -1$ and $a_3 = 0$.  So we have $a_1 = 1$ and $a_0 = -1$.  $P(x) = -1 + x + -x^2$ is one possible polynomial and we were only asked for one.  

Suppose $Q$ s.t. $Q'-3Q = 4-5x+3x^2$ (another solution).  Then
\[
\begin{gathered}
  (P-Q)' - 3 (P-Q) = 0 \quad \, \forall \, x \\
  \Longrightarrow \frac{ (P-Q)' }{ P-Q } = 3  \quad \quad \, \Longrightarrow \ln{ (P-Q) } = 3 x \quad \Longrightarrow k e^{3x} = P - Q = k \sum_{j=0}^{\infty} \frac{ (3x)^j}{j! }  \\
  Q = -k \sum_{j=0}^{\infty} \frac{ (3x)^j}{j!} + P 
\end{gathered}
\]
Since we didn't specify what $Q$ has to be, we find that, in general, any $Q$ is $P$ plus some ``amount'' of the homogeneous solution, $ke^{3x}$.  
  \item If $Q(x)$ is a given polynomial, and suppose $P$ is a polynomial solution to $P'(x) - 3P(x) = Q(x)$.  Suppose $R$ is another polynomial solution such that $R'(x)-3R(x) = Q(x)$.  Then just like above, $P-R = ke^{3x}$.  If we wanted polynomial answers of finite terms, then $k$ must be zero.  Thus, there's at most only one polynomial solution $P$.  
\end{enumerate}

\exercisehead{35} \textbf{ Bernoulli Polynomials.}  
\begin{enumerate}
\item $P_1(x) = 1; \, P_n'(x) = n P_{n-1}(x); \, \int_0^1 P_n(x) dx = 0 , \quad \text{ if } n \geq 1 $  
\[
\begin{aligned}
  n=1 \quad & (1)(1) = P_1'  \quad  \int_0^1 ( x+ c ) = \left. ( \frac{1}{2} x^2 + Cx ) \right|_0^1 = \frac{1}{2} + C = 0 \quad  C = -1/2 \\
  & P_1 = x - 1/2 \\
  n=2 \quad & 2 (x- 1/2) = P_2' \quad \int_0^1 (x^2-  x + C) = \left. \left( \frac{1}{3} x^3 - \frac{1}{2} x^2 + Cx \right) \right|_0^1 = \frac{-1}{6} + C = 0 \quad  C = 1/6 \\
  &  P_2 = x^2 - x + 1/6 \\
  n=3 \quad & 3(x^2 - x + \frac{1}{6} ) = P_3' \quad  \int_0^1 (x^3 - \frac{3x^2}{2} + \frac{x}{2} + C ) = \frac{1}{4} - \frac{1^3}{2} + \frac{1}{4} + C = 0 \\ 
  &  P_3  = x^3 - \frac{3x^2}{2} + \frac{x}{2} \\
  n=4 \quad & 4 (x^3- \frac{3}{2} x^2 + \frac{x}{2} ) = P_4' \quad  \int_0^1 ( x^4 - 2x^3 + x^2 + C) = \frac{1}{5} - \frac{1^4}{2} + \frac{1}{3} 1^3 + C = 0 \quad  C = \frac{-1}{30} \\
  &  P_4 = x^4 - 2 x^3 + x^2 + \frac{-1}{30} \\
  n=5 \quad & 5(x^4 - 2x^3 + x^2 -\frac{1}{30} ) = P_5' \quad  \int_0^1 (x^5 - \frac{5x^4}{2} + \frac{5x^3}{3} - \frac{x}{6} + C ) = \frac{1}{6} - \frac{(1)^5}{2} + \frac{5 (1)^4}{12} + \frac{ - (1)^2}{12} + C = 0; \quad  C = 0 \\
  &  P_5  = x^5 - \frac{5x^4}{2} + \frac{5x^3}{3} - \frac{x}{6}
\end{aligned}
\]
\item The first, second, and up to fifth case has already been proven.  \\
Assume the $n$th case, that $P_n(t) = t^n + \sum_{j=0}^{n-1} a_j t^j$ (the general form of a polynomial of degree $n$).  
\[
P_{n+1}' = (n+1)(t^n + \sum_{j=0}^{n-1} a_j t^j ) \Longrightarrow P_{n+1} = t^{n+1} + (n+1) \sum_{j=0}^{n-1} \frac{ a_j t^{j+1}}{ j+1} +C 
\]
\item The first, second, and up to fifth case has already been proven.  \\
Assume the $n$th case, that $P_n(0) = P_n(1)$.  
\[
\begin{gathered}
  \int P_{n+1}' = P_{n+1}(1) - P_{n+1}(0) \quad \text{ (by the second fundamental theorem of calculus) } \\
  P_{n+1}' = (n+1) P_n \\
  \int_0^1 (n+1) P_n(t) = 0 \quad \text{ (by the given properties of Bernoulli polynomials) } \\
  \Longrightarrow P_{n+1}(1)  = P_{n+1}(0)
\end{gathered}
\] 
\item $P_n(x+1)-= P_n(x) = nx^{n-1}$ is true for $n=1,2$, by quick inspection (and doing some algebra mentally).  
\[
\begin{gathered}
  P_{n+1}' = (n+1) P_n \Longrightarrow \int P_{n+1}' = (n+1) \int P_n \\
  P_{n+1}(x+1) - P_{n+1}(x) = P_{n+1}(a_1) + (n+1)\int_{a_1}^{x+1} P_n(t) - ( P_{n+1}(a_2) + (n+1)\int_{a_2}^x P_n(t) ) \\
  a_1 = 1; \, a_2 =0; \, \text{ so } P_{n+1}(1) - P_{n+1}(0) = 0 \quad \text{ (from previous problems) } \\
  \Longrightarrow 
\begin{aligned} P_{n+1}(x+1) - P_{n+1}(x) & = (n+1) ( \int_1^{x+1} P_n(t) - \int_0^x P_n(t) ) = \\ 
  & = (n+1) \int_0^x P_n(t+1) - P_n(t) = (n+1) \int_0^x nt^{n-1} = (n+1)x^n 
\end{aligned}
\end{gathered}
\]
\item \[
\begin{gathered}
  \int_0^k P_n = \int_0^k \frac{P_{n+1}'}{ n+1} = \frac{ P_{n+1}(k) - P_{n+1}(0) }{ n+1} \\
  \begin{aligned}
    P_n(x+1) - P_n(x) & = nx^{n-1} \\
    \frac{ P_n(x+1) - P_n(x) }{ n} & = x^{n-1}
  \end{aligned}  \\ 
  \Longrightarrow 
  \begin{aligned} 
    \sum_{x=1}^{k-1} \frac{ P_{n+1}(x+1) - P_{n+1}(x) }{ n +1 } & = \sum_{x=1}^{k-1} x^n  = \sum_{r=1}^{k-1} \frac{ P_{n+1}(r+1) - P_{n+1}(r) }{ n+1} = \\ 
    & = \sum_{r=1}^{k-1} r^n  = \frac{ P_{n+1}(k) - P_{n+1}(0) }{ n+1} \quad \text{ (telescoping series and $P_{n+1}(1) = P_{n+1}(0)$ ) }
\end{aligned}
\end{gathered}
\] 
\item \textbf{This part was fairly tricky}.  A horrible clue was that this part will rely directly on the last part (because of the way this question is asked), which gave us $\sum_{j=1}^{x-1} j^n = \int_0^x P_n(t) dt = \frac{ P_{n+1}(x) - P_{n+1}(0)}{ n+1}$.  

Use induction.  It can be easily verified, plugging in, that $P_n(1-x) = (-1)^n P_n(x)$ is true for $n = 0 \dots 5$.  Assume the $n$th case is true.  

\[
\begin{gathered}
  \begin{aligned}
    u & = 1 - t \\
    du & = -dt 
  \end{aligned} \quad \quad \, 
  \begin{aligned}
\int_0^x P_n(t)dt & = \int_1^{1-x} -P_n(1-u)du = -\int_1^{1-x} P_n(u)(-1)^n du \\ 
& \text{ (since $P_n(1-x)=(-1)^n P_n(x)$, assumed $n$th case is true) } \\
& = (-1)^{n+1} \int_1^{1-x} P_n(t)dt = \\
& = (-1)^{n+1} \int_0^{1-x} P_n(t) dt \quad \, \text{ (since $\int_0^1 P_n = 0$ ) } 
\end{aligned} \\
  \Longrightarrow = (-1)^{n+1} \left( \frac{ P_{n+1}(1-x) - P_{n+1}(0) }{ n+1} \right) = \frac{ P_{n+1}(x) - P_{n+1}(0) }{ n +1} \\
  \Longrightarrow P_{n+1}(1-x) = (-1)^{n+1}P_{n+1}(x)
\end{gathered}
\]
In the second to last and last step, we had used $(-1)^{n+1} P_{n+1}(0) = P_{n+1}(0)$.  For $n+1$ even, this is definitely true.  If $n+1$ was odd, \medskip \\
Doing some algebra for the first five cases, we can show that $P_{2j-1}(0) = 0$ for $j=2,3$.  Assume the $j$th case is true.  Since $P_{2j-1}$ is a polynomial and $P_{2j-1}(0)=0$, then the form of $P_{2j-1}$ must be $P_{2j-1} = \sum_{k=1}^{2j-1} a_k x^j$.  Using $P_{n+1}' = (n+1)P_n$, 
\[
\begin{gathered}
  \begin{aligned}
  P_{2j}(x) - P_{2j}(0) & = 2j \int_0^x P_{2j-1} = 2j \sum_{k=1}^{2j-1} \int_0^x a_k t^k = 2j \sum_{k=1}^{2j-1} a_k \left. \frac{1}{ k+1} t^{k+1} \right|_0^x = \\
  & = 2j \sum_{k=1}^{2j-1} \frac{a_k}{k+1} x^{k+1} = 2j \sum_{k=2}^{2j} \frac{ a_{k-1} x^k}{ k } + P_{2j}(0) 
  \end{aligned} \\
  \begin{aligned}
  P_{2j+1}(x) - P_{2j+1}(0) & = (2j+1) \int_0^x \left( \sum_{k=2} (2j) \frac{a_{k-1}}{k}t^k + P_{2j}(0) \right) = (2j+1) \sum_{k=3}^{2j+1} \frac{ 2j a_{k-2} x^k }{ k(k-1) } + (2j+1)P_{2j}(0)x \\
  \end{aligned} 
\end{gathered}
\]
If we take the integral from $0$ to $1$, then we find that $P_{2j+1}(0) = 0 $ 
\item Using $P_n(1-x) = (-1)^n P_n(x)$, derived above, 
\[
\begin{gathered}
  \begin{aligned}
    & P_{2j+1}(0) = (-1)^{2j+1} P_{2j+1}(1) = (-1) P_{2j+1}(0) \\
    & \Longrightarrow P_{2j+1}(0) = 0 
  \end{aligned} \\
\begin{aligned}
  & P_{2j-1}(1-\frac{1}{2}) = (-1)^{2j-1} (P_{2j-1}(\frac{1}{2}) ) \\
  & \Longrightarrow P_{2j-1}(\frac{1}{2}) = 0 
\end{aligned}
\end{gathered}
\]
\end{enumerate}

\exercisehead{36} There's a maximum at $c$ for $f$, so $f'(c) =0$
\[
\begin{gathered}
  \begin{aligned}
    \int_a^x f''(t)dt & = f'(x) - f'(a) \\
    \int_0^c f''(t) dt & = f'(c) - f'(0) = -f'(0) \\
    \int_c^a f''(t) dt & = f'(a) - f'(c) = f'(a)
  \end{aligned} \quad \quad \, 
\begin{aligned}
  & |f'(0)| = |\int_c^0 f''(t) dt | \leq \int_0^c |f''|dt \leq mc \\
  & |f'(a)| \leq m (a-c )
\end{aligned} \\
\boxed{ |f'(0)| + |f'(a)| \leq ma }
\end{gathered}
\]

%-----------------------------------%-----------------------------------%-----------------------------------
\subsection*{ 6.9 Exercises - Introduction, Motivation for the definition of the natural logarithm as an integral, The definition of the logarithm.  Basic properties; The graph of the natural logarithm; Consequences of the functional equation $L(ab)=L(a)+L(b)$; Logarithm referred to any positive base $b\neq 1$; Differentiation and integration formulas involving logarithms;Logarithmic differentiation }
%-----------------------------------%-----------------------------------%-----------------------------------
\quad \\

\exercisehead{1} \begin{enumerate}
\item 
\[
\log{x} = c + \left. (\ln{|t|} ) \right|_e^x = c + \ln{|x|} - 1 \quad \, \Longrightarrow \ln{ \left( \frac{x}{|x|} \right) } = c - 1 
\]$\boxed{ c = 1}$
\item 
\[
\begin{gathered}
f(x) = \ln{ \frac{1+a}{1-a}} + \ln{ \frac{1+b}{1-b} } = \ln{ \left( \frac{ (1+a)(1+b) }{ (1-a)(1-b) } \right)}  = \ln{ \left( \frac{1+x}{1-x} \right) } \\
  \Longrightarrow \frac{ 1 + x }{ 1- x } = \frac{ (1+a)(1+b)}{ (1-a)(1-b) } = \frac{ 1 + a +b +ab }{ 1 - b-a + ab} 
\end{gathered}
\]
$\boxed{ x = \frac{a+b}{1 + ab} }$ 
\end{enumerate}

\exercisehead{2} 
\begin{enumerate}
  \item $\log{(1+x)} = \log{ (1-x) } $  \medskip \\
    $\Longrightarrow \boxed{ x =0 }$
  \item $ \log{(1+x)} = 1 + \log{ (1-x) }$ 
    \[
\begin{gathered}
    \ln{(1+x)} = 1 + \ln{ (1-x) } = \ln{ (e) } + \ln{ (1-x) } \ln{ e(1-x) } \\
    \Longrightarrow 1 + x = e - ex \quad \, \Longrightarrow \boxed{ x = \frac{ e-1}{ 1 + e }  }
\end{gathered}
\]
  \item $2 \log{x} =x \log{2}$ \medskip \\
    \[
    \ln{ x^2 } + \ln{ 2^{-x}} = 0 = \ln{1} = \ln{ x^2 2^{-x}} \quad \, \Longrightarrow \boxed{ x = 2}
    \]
  \item $\log{ ( \sqrt{x} + \sqrt{ x+1} ) } = 1 $ \medskip \\
    \[
    \begin{gathered}
      \sqrt{ x+1} = e - \sqrt{x} \Longrightarrow x+1 =e^2 - 2 \sqrt{x} e + x \\
      2\sqrt{x} e = e^2 - 1 \quad \, \Longrightarrow \boxed{ x = \left( \frac{e^2 - 1 }{ 2e } \right)^2 }
    \end{gathered}
    \]
\end{enumerate}

\exercisehead{3}
\[
\begin{gathered}
  \begin{aligned}
    f & = \frac{ \ln{x}}{x} \\
    f' & = \frac{ 1 - \ln{x}}{ x^2} \\
    f'' & = \frac{ -2}{x^3} - \left( \frac{ \frac{1}{x}x^2 - 2 x \ln{x}}{ x^4} \right) = \frac{ 2 \ln{x} - 3 }{ x^3 } 
  \end{aligned} \quad \quad \, 
  \begin{aligned}
    & \quad \\
    & f' < 0 \text{ when } x > e \quad \quad \, f'> 0 \text{ when } 0 < x < e  \\
    & \text{ for $x^3 > 0$, } \quad \, 2\ln{x} - 3 > 0 \Longrightarrow f''(x) < 0 \text{ (concave), when $0< x < e^{3/2}$ }; \\ 
    & \quad \quad \, f''(x) > 0 \text{ (convex), when $x> e^{3/2}$ }
  \end{aligned}
\end{gathered}
\]

\exercisehead{4} $f(x) = \log{ (1+x^2) }$ 
\[
f' = \frac{2x}{ 1 + x^2}
\]

\exercisehead{5} $f(x) = \log{ \sqrt{ 1 + x^2} }$
\[
f' = \frac{x}{1+x^2}
\]

\exercisehead{6} $f(x) = \log{ \sqrt{4-x^2 } }$
\[
f' = \frac{1}{2} \frac{-2x}{4-x^2} = \frac{-x}{4-x^2}
\]

\exercisehead{7} $f(x) = \log{ (\log{x}) }$
\[
f' = \frac{1}{\ln{x}} \left( \frac{1}{x}  \right) = \frac{1}{x \ln{x}}
\]

\exercisehead{8} $f(x) = \log{ x^2 \log{x} }$
\[
f' = \left( 2\log{x} + \log{ \log{x}} \right)' = \frac{2}{x} + \frac{1}{x \log{x}}
\]

\exercisehead{9} $f(x) = \frac{1}{4} \log{ \frac{x^2- 1}{x^2 + 1 } }$
\[
f' = \frac{1}{4} \left( \log{ x^2 - 1 } - \log{ x^2 + 1 } \right)' = \frac{1}{4} \left( \frac{2x}{ x^2 - 1 }  -\frac{2x}{ x^2 + 1 } \right) = x \left( \frac{1}{x^4 - 1 } \right)
\]

\exercisehead{10} $f(x) = (x+ \sqrt{ 1 + x^2 })^n$ 
\[
\begin{gathered}
  \ln{f} = n \ln{ (x+ \sqrt{ 1 +x^2 } ) } \\
  \frac{f'}{f} = n \left( \frac{1}{ x+ \sqrt{1 + x^2 } } \right)\left( 1 + \frac{ x}{ \sqrt{1+x^2 } } \right)= \frac{n}{\sqrt{ 1 + x^2} } \\
  f' = (x+ \sqrt{ 1 + x^2} )^n \frac{n}{ \sqrt{ 1 + x^2} }
\end{gathered}
\]

\exercisehead{11} $f(x) = \sqrt{ x+1} - \log{ (1+\sqrt{x+1} )} $
\[
f' = \frac{1}{2\sqrt{x + 1} } - \frac{ 1}{ 1 + \sqrt{x+1} } \left( \frac{1}{ 2 \sqrt{x+1}} \right) = \boxed{ \frac{1}{ 2 ( 1+ \sqrt{ x+1} ) } }
\]


\exercisehead{12} $f(x) = x \log{ (x+\sqrt{ 1 + x^2 } ) } - \sqrt{ 1 + x^2 } $
\[
f' = \log{ (x + \sqrt{ 1 + x^2 } ) } + \frac{x}{ x+\sqrt{1 + x^2}} \left( 1 + \frac{x}{ \sqrt{1+x^2} } \right) - \frac{x}{ \sqrt{ 1 + x^2} }
\]
\textbf{ Note to self}: \emph{Notice} how this had made some of the square root terms disappear.  

\exercisehead{13} $f(x) = \frac{1}{ 2 \sqrt{ab}} \log{ \frac{ \sqrt{a} + x\sqrt{b} }{ \sqrt{ a} - x\sqrt{b}} }$

\[
\begin{gathered}
  f = \frac{1}{ 2\sqrt{ab}} (\ln{ (\sqrt{a} + x \sqrt{b} ) }  - \ln{ (\sqrt{a} - x\sqrt{b} ) } ) \\
  f' = frac{1}{2 \sqrt{ab}} \left( \frac{1}{ \sqrt{a} +x\sqrt{b}} \sqrt{b} - \frac{1}{ \sqrt{a} - x \sqrt{b} } (-\sqrt{b}) \right) = \boxed{ \frac{ -x \sqrt{b}}{ \sqrt{a} (a-bx^2) } }
\end{gathered}
\]

\exercisehead{14} $f(x) = x (\sin{ (\log{x}) } - \cos{ \log{x}} )$
\[
f' = \sin{ (\ln{x})} - \cos{ (\ln{x})} + (\cos{ (\ln{x})} + \sin{ (\ln{x}) } ) = 2 \sin{(\ln{x}) }
\]

\exercisehead{15} $f(x) = \log^{-1}{x}$
\[
f' = \frac{-1}{x (\ln{x})^2} 
\]

\exercisehead{16} $\int \frac{dx}{ 2 + 3x}  = \frac{1}{3} \ln{ (2+3x)}$

\exercisehead{17} $\int log^2{x} dx$
\[
\begin{gathered}
  \begin{aligned}
(x\ln^2{x})' & = \ln^2{x} + 2 \ln{x} \\
  (x\ln{x} - x)' & = \ln{x} + 1 - 1 = \ln{x} \\
  \end{aligned} \quad \quad \, \Longrightarrow xln^2{x} - 2 (x\ln{2} - x) = x\ln^2{x} - 2 x\ln{x} + 2x 
\end{gathered}
\]

\exercisehead{18} $\int x \log{x} dx$
\[
\begin{gathered}
  \left( \frac{x^2 \ln{x}}{ 2 } \right)' = x\ln{x} + \frac{x}{2} \\
  \int x \ln{x} = \frac{ x^2 \ln{x}}{ 2 } - \frac{x^2}{4}
\end{gathered}
\]

\exercisehead{19} $\int x \log^2{x} dx$
\[
\begin{gathered}
  \left( \frac{x^2 \ln^2{x}}{ 2 } \right)' = x \ln^2{x} + x^2 \ln{x} \left( \frac{1}{x} \right) \\
  \Longrightarrow \int x \log^2{x} = \frac{ x^2 \ln^2{x} }{2} - \frac{x^2 \ln{x}}{2} - \frac{x^2}{4} 
\end{gathered}
\]

\exercisehead{20} $\int_0^{e^2-1} \frac{dt}{ 1 + t }$
\[
\int_0^{e^3-1} \frac{dt}{1+t} = \left. \ln{ (1+t)} \right|_0^{e^3-1} = 3 
\]

\exercisehead{21} $\int \cot{x} dx$
\[
\int \frac{ \cos{x} }{ \sin{x}} dx = \ln{ |\sin{x}| }
\]

\exercisehead{22} $\int x^n \log{(ax)} dx$ \emph{Solve the problem directly}.    
\[
\begin{gathered}
  \int x^n \log{ax} = \int x^n \log{a} + \int x^n \log{x} = \frac{ x^{n+1}}{n+1} \log{a} + \int x^n \log{x} \\
  \begin{aligned}
    \int x^n \ln{x} & = \frac{ x^{n+1}}{ n+1} \ln{x} - \int \frac{ x^{n+1}}{ n+1} \frac{1}{x}  = \frac{ x^{n+1}}{n+1} \log{x} - \frac{x^{n+1}}{(n+1)^2}
  \end{aligned}  \\
  \Longrightarrow \int x^n \log{ax} = \frac{ x^{n+1}}{ n+1} \log{a} + \frac{ x^{n+1}}{ n+1} \log{x} - \frac{x^{n+1}}{ (n+1)^2}
\end{gathered}
\]

\exercisehead{23} $\int x^2 \log^2{x} dx$
\[
\int x^2 \log^2{x} = \frac{1}{3}x^3 \ln^2{x} - \int \frac{x^2}{3} 2 \ln{x} = \frac{ x^3 \ln^2{x} }{ 3} - \frac{2}{3} \left( \frac{x^3 \ln{x}}{ 3 } - \frac{x^3}{9} \right) = \boxed{ \frac{x^3 \ln^2{x}}{ 3 } - \frac{ 2 x^3 \ln{x}}{9} + \frac{2x^3}{27} }
\]

\exercisehead{24} $\int \frac{dx}{ x \log{x}}$
\[
\int \frac{dx}{ x \ln{x}} = \ln{ (\ln{x})} + C
\]

\exercisehead{25} $\int_1^{1-e^{-2}} \frac{ \log{(1-t)} }{ 1-t} dt$
\[
\int_0^{1-e^2} \frac{ \ln{ (1-t)}}{ 1-t} dt = \left. - \frac{1}{2}(\ln{ (1-t) })^2 \right|_0^{1-e^{-2}} = \boxed{ -2 }
\]

\exercisehead{26} $\int \frac{ \log{|x|} }{ x \sqrt{ 1 + \log{ |x|} } } dx$
\[
\begin{aligned}
\int \frac{ \log{x}}{ x \sqrt{ 1 + \log{x}}} & = \int (2 (1+\log{x})^{1/2} )' \log{x} = 2 (1+ \log{x})^{1/2} \log{x} - \int \frac{ 2 (1+\log{x})^{1/2}}{ x} = \\
& = \boxed{ 2 \log{x} ( 1 + \log{x})^{1/2} -\frac{4}{3} (1+ \log{x})^{3/2} }
\end{aligned}
\]

\exercisehead{27} Derive 
\[
\int x^m \log^n{x} dx = \frac{ x^{m+1}}{ m+1} \ln^n{x} - \frac{n}{m+1} \int x^m \ln^{n-1}{x}
\]
By inspection, we just needed integration by parts.
\[
\int x^3 \ln^3{x} = \frac{ x^4}{4} \ln^3{x} - \frac{3}{4} \int x^3 \ln^2{x} = \frac{ x^4}{4} \ln^3{x} - \frac{3}{4} \left( \frac{x^4}{4} \ln^2{x} - \frac{2}{4} \int x^3 \ln{x} \right) = \frac{x^4 \ln^3{x}}{4} - \frac{ 3x^4 \ln^2{x}}{16} + \frac{3x^4 \ln{x}}{ 32 } - \frac{ 3x^4}{128} 
\]

\exercisehead{28} Given $x>0$, \quad \, $f(x) = x - 1 - \ln{x}$; \quad \, $g(x) = \ln{x} - 1 + \frac{1}{x}$
\begin{enumerate}
\item \[
\begin{gathered}
  \begin{aligned}
    & f' = 1 - \frac{1}{x} \\
    & g' = \frac{1}{x} - \frac{1}{x^2} = \frac{1}{x} f' 
  \end{aligned} \quad \quad \, xg' = f' \\
  \text{ so then if } f' > 0, \, g' > 0 ; \quad \, f' < 0, \quad g' < 0 \\
  \text{ For } f' < 0 \quad 0 < x < 1 \quad \quad \quad \, f' > 0 \quad \, x > 1 \quad \quad \quad f'(1) = g'(1) = 0 \\
  f(1) = g(1) = 0 \\
  \begin{aligned}
    & x - 1 - \ln{x} > 0 \text{ since } f(1) = 0 \text{ is a rel. min. } \\
    & 0 < \ln{x} - 1 + \frac{1}{x} \text{ since $g(0)$ is a rel. min. }
  \end{aligned} 
\end{gathered}
\]
\item See sketch.  
\end{enumerate}

\exercisehead{29} $\lim_{x\to 0} \frac{ \log{ (1+x)}}{x} = 1 $
\begin{enumerate}
\item $L(x) = \int_1^x \frac{1}{t} dt$ ;\quad \quad \, $L'(x) = \frac{1}{x}$; \quad \quad \, $L'(1) = 1$
\item Use this theorem.  
\begin{theorem}[Theorem I.31] \quad \\
If $3$ real numbers $a,x$, and $y$ satisfy the inequalities
\[
\begin{gathered}
a \leq x \leq a + \frac{y}{n} \\
\forall \, n \geq 1 , \quad \, n \in \mathbb{Z}, \, \text{ then } x = a
\end{gathered}
\]
\end{theorem}
\[
\begin{gathered}
  1 - \frac{1}{x} < \ln{x} < x - 1 \Longrightarrow 1 - \frac{1}{x+1} < \ln{x+1} < x \\
  1 - \frac{x}{1+x} < \frac{ \ln{x+1} }{x} < 1  \quad \quad \, \Longrightarrow \boxed{ \frac{ \ln{ (1+x)}}{x} = 1 }
\end{gathered}
\]
\end{enumerate}

\exercisehead{30} Using $f(xy) = f(x) +f(y)$, \\
\[
r = \frac{p}{q} \quad \quad \, \Longrightarrow f(a^{p/q}) = f((a^{1/q})^p) = p f(a^{1/q}) = \frac{p}{q} f(a)
\]
since
\[
\begin{gathered}
  \begin{gathered}
  f(a) = f(a); \quad \quad \, f(a^2) = f(aa) = 2 f(a) \\
  f(a^{p+1}) = f(a^p) + f(a) = (p+1)f(a)
\end{gathered} \quad \quad \, \begin{gathered} 
\text{ If $f(a) = f((a^{1/q})q) = q f(a^{1/q})$ } \\
  \text{ then } f(a^{1/q}) = \frac{1}{q} f(a) 
\end{gathered}
\end{gathered}
\]

\exercisehead{31} $\ln{x} = \int_1^{|x|} \frac{1}{t} dt$
\begin{enumerate}
\item $\ln{x} = \int_1^{|x|} \frac{1}{t} dt$  From this definition, then for $n$ partitions, $a_0 = 1, \, a_1 = 1 + \frac{b-1}{n}, \dots , a_n = b = x$ \medskip \\
$ \frac{b-a}{n} = \frac{b-1}{n}$; \quad \quad \, so if $a_k = 1 + k \left( \frac{b-1}{n} \right)$  
\[
\sum_{k=1}^n \left( \frac{ a_k -a_{k-1}}{ a_k } \right) < \log{x} < \sum_{k=1}^n \left( \frac{a_k - a_{k-1}}{ a_{k-1}} \right)
\]
\item $\log{x}$ is greater than the step function integral consisting of rectangular strips within $\frac{1}{x}$ and less than rectangular strips covering over $\frac{1}{x}$
\item $a_k = 1 + k$  \quad \quad \, $\Longrightarrow \frac{ a_k - a_{k-1}}{a_{k-1}} = \frac{1}{k}$ \\
  \[
  \begin{gathered}
    \sum_{k=1}^n \frac{1}{1+k} < \ln{ (n+1)} < \sum_{k=1}^n \frac{1}{k} \Longrightarrow \sum_{k=2}^{n+1} \frac{1}{k} < \ln{(n+1)} < \sum_{k=1}^n \frac{1}{k} \\
    \Longrightarrow \sum_{k=2}^n \frac{1}{k} < \ln{(n)} < \sum_{k=1}^{n-1} \frac{1}{k}
  \end{gathered}
  \]
\end{enumerate}

\exercisehead{32} 
\begin{enumerate}
\item $L(x) = \frac{1}{\ln{b} } \ln{x} = \log_b{x} $
\[
\begin{gathered}
  \log_a{x} = c \log_b{x} \quad \quad \, \Longrightarrow \log_a{a} = 1 \quad \quad \, \text{ (There must be a unique real number s.t. $L(a) = 1 $ ) } \\
  \log_a{x} = \frac{1}{ \log_b{a}} \log_b{x}  \quad \quad \, \Longrightarrow \log_b{x} = \log_b{a}\log_a{x}
\end{gathered}
\]
\item Changing labels for $a,b$: $\log_b{x} = \frac{ \log_a{x}}{ \log_a{b}}$
\end{enumerate}

\exercisehead{33}
$\log_e{10} = 2.302585$ \quad \\
\[
\begin{gathered}
  \log_{10}{e} = \frac{ \log_e{e}}{ \log_e{10}} \quad \quad \, \Longrightarrow \log_e{10} = \frac{1}{ \log_{10}{e}} = \frac{1}{ 2.302585} \simeq 0.43429
\end{gathered}
\]

\exercisehead{34} Given $\int_x^{xy} f(t)dt = B(y)$  \quad \quad $f(2)=2$, \quad \quad , We Want $A(x) = \int_1^x f(t)dt$ 
\[
\begin{gathered}
  \int_x^{xy} f(t) dt = B(xy) - B(x) \\
  \frac{d}{dx} \int_x^{xy} f(t)dt = \frac{d}{dx} \left( B(xy) - B(x) \right) = \frac{dB(xy)}{d(xy)} y - \frac{ dB(x)}{ dx} = f(xy)y - f(x) = 0 \\
  \Longrightarrow f(xy) = \frac{ f(x) }{y} \\
  A(x) = \int_1^x f(t) dt = \int_1^x f((2)\left( \frac{t}{2} \right) ) dt = \int_1^x \frac{f(2)}{(t/2)} dt = \boxed{ 4 \ln{x} }
\end{gathered}
\]

\exercisehead{35} Given $\int_1^{xy} f(t) dt = y \int_1^x f(t) dt + x \int_1^y f(t) dt $, and letting $F$ be the antiderivative of $f$,
\[
\begin{gathered}
  F(xy) - F(1) = y (F(x) -F(1)) + x (F(y) - F(1)) \\
  \xrightarrow{d/dx} f(xy)(y) = y (f(x)) + \int_0^y f(t) dt \xrightarrow{d/dx} f'(xy)y^2 = y(f'(x))  \\
  \xrightarrow{x=1} f'(y)  = \frac{ (f'(1)) }{y}  \xrightarrow{\int} f(y) = k \ln{y} + C \xrightarrow{y=1} f(1) = 0 + C = 3 
\end{gathered}
\]


 $ \int_1^{xy} (k \ln{t} + 3) = y\int_1^x (k\ln{t} +3)  + x \int_1^y (k\ln{t} + 3)  $
\begin{multline*}
  k \left( (xy)\ln{xy} - (xy) + 1\right) + 3(xy - 1)   = y \left( k \left( (x)\ln{x} - x +1 \right) + 3(x-1) \right) + x \left( k \left( y\ln{y} - y + 1 \right) + 3(y -1) \right) 
\end{multline*}
\[
\begin{gathered}
  \Longrightarrow k -3 = ky -3y  -kxy + kx + 3xy -3x \\
  y(3-k) + xy(k-3) + k-3 + (3-k)x = 0 \Longrightarrow k =3 \\
  \Longrightarrow f(x) = 3\ln{x} + 3 
\end{gathered}
\]

\exercisehead{36} 

%-----------------------------------%-----------------------------------%-----------------------------------
\subsection*{ 6.11 Exercises - Polynomial approximations to the logarithm }
%-----------------------------------%-----------------------------------%-----------------------------------
\quad \\ 

\exercisehead{1}
\begin{theorem}[Theorem 6.5] If $0<x<1$ and if $m\geq 1$, 
\[
\ln{ \frac{1+x}{1-x}} = 2 (x + \frac{x^3}{3} + \dots + \frac{x^{2m-1}}{2m-1} ) + R_m(x)
\]
where
\[
\begin{gathered}
  \frac{x^{2m+1}}{ 2m+1} < R_m(x) \leq \frac{2-x}{1-x} \frac{x^{2m+1}}{ 2m+1}  \\
  R_m(x) = E_{2m}(x) - E_{2m}(-x)
\end{gathered}
\]
where $E_{2m}(x)$ is the error term for $\log{1-x}$
\end{theorem}
$m=5$, \, $x=\frac{1}{3}$ 
\[
\ln{ \left( \frac{4/3}{2/3} \right)} = \ln{2} \simeq 2 \left( \frac{1}{3} + \frac{ \left( \frac{1}{3} \right)^3}{3} + \frac{ \left( \frac{1}{3} \right)^5}{ 5} + \frac{ \left( \frac{1}{3} \right)^7 }{ 7} + \frac{ \left( \frac{1}{3} \right)^9 }{9} \right) \simeq 0.693146047
\]
The error for $m=5$, \, $x = \frac{1}{3}$ is 
\[
\frac{ \left( \frac{1}{3} \right)^{11}}{11} \leq R_5(x) \leq \frac{5}{2} \frac{ \left( \frac{1}{3} \right)^{11} }{ 11 }
\]

\exercisehead{2} $\ln{ \left( \frac{1+x}{1-x} \right) } = \ln{ \frac{3}{2} } = \ln{3} - \ln{2}$  
\[
\begin{gathered}
  \begin{aligned}
    x & = \frac{1}{5} \\
    m & = 5 
  \end{aligned} \\
\ln{ \left( \frac{ 1 + x}{1-x} \right) } = 2 \left( \frac{1}{5} + \frac{ (1/5)^3}{3} + \frac{ (1/5)^7}{7} + \frac{ (1/5)^9}{9} \right) = 0.405465104 \\
 \frac{ (1/5)^{11}}{11} \simeq 0.000000002 < R_5(x) \leq \frac{9}{4} \frac{ \left( \frac{1}{5} \right)^{11} }{ 11 } \simeq 0.000000004 \\
\Longrightarrow \boxed{ \begin{gathered}
    \log{3} \simeq 1.098611 \\
    1.098611666 < \log{3} < 1.098612438 
\end{gathered} }
\end{gathered}
\]

\exercisehead{3} $x= \frac{1}{9}$  \quad \, $\ln{ \left( \frac{1 + \frac{1}{9} }{ 1 - \frac{ 1}{9} } \right) } = \ln{ \left( \frac{10}{8} \right)} = \ln{ \frac{5}{4} } = \ln{5} - 2 \ln{2}$ \medskip  \\
For $m=2$
\[
\begin{gathered}
2 \left( \frac{1}{9} + \frac{ \left( \frac{1}{9} \right)^5 }{ 3 } + \frac{ \left( \frac{1}{9} \right)^5}{ 5} \right) = 0.2231435 \\
\frac{ \left( \frac{1}{9} \right)^7 }{ 7 } \simeq 0.00000003 < R_3(x) \leq \frac{ \frac{17}{9}}{ \frac{8}{9} } \frac{ \left( \frac{1}{9} \right)^7 }{ 7 } \simeq 0.000000063 \\
1.609437 < \log{5} < 1.609438
\end{gathered}
\]

\exercisehead{4} $x=\frac{1}{6}$  \quad $\ln{  \left( \frac{ 1 + \frac{1}{6}  }{ 1 - \frac{1}{6} } \right) } = \ln{ \left( \frac{7}{5} \right) } = \ln{7} - \ln{5}$.  \medskip \\
For $m=3$, 
\[
\begin{gathered}
\ln{ \frac{ 1+x}{1-x} } \simeq 2 \left( \frac{1}{6} + \frac{ \left( \frac{1}{6} \right)^3}{3} + \frac{ \left( \frac{1}{6} \right)^5 }{ 5 } \right) = 0.336471193 \\
\text{ The error bounds are }  \begin{aligned}
  & \frac{ \left( \frac{1}{6} \right)^{2(3)+1} }{ 2(3)+1 } = 0.00000051 \\
  &  \frac{ \left( \frac{1}{6} \right)^{2(3)+1} }{ 2(3)+1 } \left( \frac{ 2 - \frac{1}{6} }{ 1 - \frac{1}{6} } \right) = 0.000001123 
\end{aligned} \\
  1.945908703 < \ln{7} < 1.945910316
\end{gathered}
\]

\exercisehead{5} 
\[
\begin{gathered}
  0.6931460 < \ln{2} < 0.6931476 \\
\begin{aligned}
\ln{3} & = 1.098614 \\ 
\ln{4} & =  1.386293 \\
\ln{5} & =  1.609436
\ln{6} & = \ln{2} + \ln{3} = 1.791700
\end{aligned}
\quad 
\begin{aligned}
  \ln{7} & = 1.945909  \\
  \ln{8} & = 3\ln{2} = 2.0794404  \\
  \ln{9} & = 2\ln{3} = 2.197228 \\
  \ln{10} & = \ln{5} + \ln{2} = 1.302577
\end{aligned}
\end{gathered}
\]

%-----------------------------------%-----------------------------------%-----------------------------------
\subsection*{ 6.17 Exercises - The exponential function, Exponentials expressed as powers of $e$, The definition of $e^x$ for arbitrary real $x$, The definition of $a^x$ for $a>0$ and $x$ real, Differentiation and integration formulas involving exponentials }
%-----------------------------------%-----------------------------------%-----------------------------------
\quad \\
\quad \\
\exercisehead{1} $f' = 3e^{3x-1}$
\exercisehead{2} $8xe^{4x^2}$
\exercisehead{3} $-2xe^{-x^2}$ 
\exercisehead{4} $\frac{1}{2\sqrt{x}} e^{\sqrt{x}}$
\exercisehead{5} $\frac{-1}{x^2} e^{-1/x}$
\exercisehead{6} $ \ln{2} 2^x$
\exercisehead{7} $(2x \ln{2})2^{x^2}$
\exercisehead{8} $\cos{x} e^{\sin{x}}$
\exercisehead{9} $-2\cos{x} \sin{x} e^{\cos^2{x}}$
\exercisehead{10} $\frac{1}{x} e^{\log{x}}$
\exercisehead{11} $e^x e^{e^x}$
\exercisehead{12} $e^{e^{e^{x}}} (e^x e^{e^x} )$ 
\exercisehead{13} $\int x e^x dx = xe^x -e^x$
\exercisehead{14} $\int xe^{-x} dx = -xe^{-x} + - e^{-x}$
\exercisehead{15} $ \int x^2 e^x = x^2 e^x - 2x e^x + 2e^x$
\exercisehead{16} $\int x^2 e^{-2x } dx = \frac{x^2 e^{-2x} }{-2} + \frac{xe^{-2x}}{-2} + \frac{-e^{-2x}}{4}$

\exercisehead{17} $ \int e^{\sqrt{x}} = e^{\sqrt{x}} (2\sqrt{x}) - \int \frac{1}{\sqrt{x}} e^{\sqrt{x}} = e^{\sqrt{x}} (2\sqrt{x}) - 2e^{\sqrt{x}} = \boxed{ 2(\sqrt{x} e^{\sqrt{x}} - e^{\sqrt{x}} ) }$

\exercisehead{18} $\int x^3 e^{-x^2}$.  
\[
\begin{gathered}
  \begin{aligned}
    x^2 e^{-x^2} & = -2x^3 e^{-x^2} + 2x e^{-x^2} \\
    e^{-x^2} & = -2x e^{-x^2}
  \end{aligned} \\
  \frac{1}{-2} (x^2 e^{-x^2} + e^{-x^2} )' = \frac{1}{-2} \left( 2xe^{-x^2} + -2x^3 e^{-x^2} + -2xe^{-x^2} \right) = \boxed{ \frac{x^2 e^{-x^2} + e^{-x^2} }{ -2} }
\end{gathered}
\]

\exercisehead{19} $e^x = b +\int_a^b e^t dt = b+e^x -e^a$, $\quad e^a = b$.  

\exercisehead{20}
\[
\begin{gathered}
\begin{aligned}
  A &= \int e^{ax} \cos{bx} dx \\
  B & = \int e^{ax} \sin{bx} dx 
\end{aligned} \\
\begin{aligned}
  A & = \frac{e^{ax}}{a} \cos{bx} - \int - \frac{b \sin{bx} e^{ax}}{ a} = \frac{e^{ax}}{ a} \cos{bx} + \frac{b}{a} B \quad \Longrightarrow aA + bB = e^{ax} \cos{bx} + C \\
  B & = \frac{ e^{ax}}{a} \sin{bx} - \int \frac{e^{ax}}{a} b \cos{bx} \Longrightarrow aB + bA = e^{ax} \sin{bx} + C 
  \end{aligned} \\
A = \frac{1}{a^2+b^2} \left( a e^{ax} \cos{bx} + be^{ax} \sin{bx} \right) \\
B = \frac{ -b e^{ax} \cos{bx} + ae^{ax} \sin{bx} }{ a^2 + b^2 }
\end{gathered}
\]

\exercisehead{21}
\[
\ln{f} = x \ln{x}; \quad \frac{f'}{f} = \ln{x} + 1; \quad f' = x^x (\ln{x} + 1 )
\]

\exercisehead{22} 
\[
\ln{\frac{f'}{f} } = \frac{1}{1+x} + \frac{1}{1+e^{x^2}} ( 2xe^{x^2} ), \quad f' = 1 + e^{x^2} + 2 (1+x)xe^{x^2}
\]

\exercisehead{23} $f = \frac{e^x - e^{-x}}{ e^x + e^{-x}}$.  
\[
\begin{gathered}
  \ln{f} = \ln{ (e^x - e^{-x}) } - \ln{ (e^x + e^{-x} )} \\
  \frac{f'}{f} = \frac{1}{ e^x - e^{-x}} (e^x + e^{-x}) - \frac{1}{ e^x + e^{-x}} (e^x - e^{-x}) \\
  f' = 1 - \left( \frac{e^x - e^{-x}}{ e^x + e^{-x} } \right)^2 
\end{gathered}
\]

\exercisehead{24} $f' = (x^{a^a})' + (a^{x^a})' + (a^{a^x})'$.  
\[
\begin{aligned}
  f_1 &= x^{a^a} \\
  \ln{f_1} & = a^a \ln{x} \\
  \frac{ f_1'}{f_1} & = \frac{a^a}{x} \\
  f_1' = x^{a^a-1}a^a
\end{aligned} \quad 
\begin{aligned}
  f_2 &= a^{x^a} \\
  \ln{f_2} & = x^a \ln{a} \\
  \frac{f_2'}{f_2} & = ax^{a-1} \ln{a} \\
  f_2' & = a^{x^a +1} x^{a-1} \ln{a} 
\end{aligned} \quad
\begin{aligned}
  f_3 & = a^{a^x} \\
  \ln{f_3} & = a^x \ln{a} \\
  \frac{ f_3'}{ f_3} & = (\ln{a})^2 a^x \\
  f_3' & = (\ln{a})^2 a^{x + a^x}
\end{aligned}
\]
$\Longrightarrow f' = x^{a^a - 1 }a^a + a^{x^a + 1}x^{a-1}\ln{a} + (\ln{a})^2 a^{x+a^x}$

\exercisehead{25}
\[
\begin{gathered}
  e^f = \ln{(\ln{x})}; \\
  f'e^f = \frac{1}{ \ln{x}} \left( \frac{1}{x} \right) \\
  f' = \frac{1}{ \ln{(\ln{x})} } \left( \frac{1}{ \ln{x}} \right) \left( \frac{1}{x} \right)
\end{gathered}
\]

\exercisehead{26} $e^f = e^x + \sqrt{ 1 + e^{2x}}$
\[
\begin{gathered}
  (e^f)f' = e^x + \frac{e^{2x}}{ \sqrt{ 1 + e^{2x}} } \\
  f' = \frac{ \sqrt{ 1 + e^{2x}} e^x + e^{2x} }{ \sqrt{ 1 + e^{2x}} (e^x + \sqrt{ 1 + e^{2x}} ) } = \frac{e^x}{ \sqrt{ 1 + e^{2x}} }
\end{gathered}
\]

\exercisehead{27}$\ln{f} = x^x \ln{x}$
\[
\begin{gathered}
  \frac{f'}{f} = (x^x)' \ln{x} + \frac{x^x}{x} = (x^x(\ln{x} + 1 ))\ln{x} + x^{x-1} \\
  f' = \boxed{ x^{x+x^x} (\ln{x} + 1 )\ln{x} + x^{x^{x^x} + x -1 } }
\end{gathered}
\]

\exercisehead{28} $\ln{f} = x \ln{ (\ln{x})}$
\[
\begin{gathered}
  \frac{f'}{f} = \ln{(\ln{x})} + \frac{1}{\ln{x}} \\
  f' = (\ln{x})^x (\ln{ (\ln{x}) } + \frac{1}{\ln{x}} )
\end{gathered}
\]

\exercisehead{29} $\ln{f} = (\ln{x}) \ln{x}$
\[
\begin{gathered}
  \frac{f'}{f} = \frac{2 \ln{x}}{ x} \\
 \boxed{  f' = 2x^{\log{x}-  1 }\ln{x} }
\end{gathered}
\]

\exercisehead{30} $\ln{f} = x \ln{(\ln{x})} - \ln{x} \ln{x} = x\ln{ \ln{x}} - (\ln{x})^2$
\[
\begin{gathered}
  \frac{f'}{f} = \ln{(\ln{x})} + \frac{1}{\ln{x}} - 2\frac{\ln{x}}{x} \\
  f' = \frac{ (\ln{x})^x }{ x^{\ln{x}}} \left( \ln{ \ln{x}} + \frac{1}{ \ln{x}} - \frac{2\ln{x}}{x} \right)
\end{gathered}
\]

\exercisehead{31}
\[
\begin{gathered}
\begin{aligned}
  \ln{f_1} & = \cos{x} \ln{\sin{x}} \\
  \frac{f_1'}{f_1} &= -\sin{x} \ln{\sin{x}} + \frac{ \cos^2{x}}{\sin{x}} \\
  f_1' & = -\sin{x} \cos{x} (\ln{ \sin{x}})^2 + \frac{\cos^3{x} \ln{\sin{x}} }{ \sin{x}} 
\end{aligned}
\quad
\begin{aligned}
  \ln{f_2} & = \sin{x} \ln{\cos{x}} \\
  \frac{f_2'}{f_2} & = \cos{x} \ln{\cos{x}} + \frac{ -\sin^2{x}}{ \cos{x}} \\
  f_2' & = \sin{x} \cos{x} (\ln{\cos{x}})^2 - \frac{ \sin^3{x} \ln{\cos{x}} }{ \cos{x}} 
\end{aligned} \\
\Longrightarrow f = \sin{x}\cos{x} (-(\ln{\sin{x}})^2 + (\ln{\cos{x}})^2 ) + \frac{ \cos^3{x} \ln{\sin{x}}}{ \sin{x} } + \frac{ -\sin^3{x} \ln{\cos{x}}}{ \cos{x}}
\end{gathered}
\]

\exercisehead{32} $\ln{f} = \frac{1}{x} \ln{x}$ \medskip \\
$\frac{f'}{f} = \frac{-1}{x^2} \ln{x} + \frac{1}{x^2}$ \quad \quad \, $\Longrightarrow f' = -x^{1/x-2} \ln{x} + x^{1/x - 2 }$

\exercisehead{33} $\ln{f} = 2 \ln{x} + \frac{1}{3} \ln{ (3-x)} - \ln{ (1-x) } + \frac{-2}{3} \ln{ (3+x) }$
\[
\begin{gathered}
  \frac{f'}{f} = \frac{2}{x} + \frac{-1}{3 (3-x)} - \frac{-1}{1-x} - \frac{2}{3} \frac{1}{3+x} \\
  \begin{aligned}
    f' & = 2 \frac{ x (3-x)^{1/3} }{ (1-x)(3+x)^{2/3} } + -\frac{1}{3} \left( \frac{ x^2 ( 3-x)^{-2/3}}{ (1-x)(3+x)^{2/3} } \right) + \frac{ x^2 (3-x)^{1/3}}{ (1-x)^2 (3+x)^{2/3} } - \frac{2}{3} \frac{ x^2 (3-x)^{1/3}}{ (1-x)(3+x)^{5/3} } \\
    & = \frac{ x (18-12x + \frac{4}{3} x^2 + \frac{2}{3} x^3 )}{ (1-x)^2 (3+x)^{5/3} (3-x)^{2/3} }
\end{aligned}
\end{gathered}
\]

\exercisehead{34} $\ln{f} = \sum_{i=1}^n b_i \ln{ (x-a_i) }$ 
\[
\frac{f'}{f} = \sum_{i=1}^n \frac{b_i}{ x-a_i} \quad \quad \, \Longrightarrow \boxed{ f' = \sum_{j=1}^n \frac{b_j}{ x-a_j} \prod_{i=1}^n (x-a_i)^{b_i} }
\]

\exercisehead{35}
\begin{enumerate}
\item Show that $f' = rx^{r-1}$ for $f = x^r$ holds for arbitrary real $r$.  
\[
\begin{gathered}
  x^r = e^{r \ln{x}} \\
  (e^{r \ln{x}})' = e^{r\ln{x}} \frac{r}{x} = rx^{r-1}
\end{gathered}
\]
\item For $x \leq 0$, by inspection of $x^r = e^{r \log{x}}$, then if $x^r > 0$, then the equality would remain valid.  So then $x^r = |x^r| = |x|^r$ and so 
\[
\begin{gathered}
  \ln{ |f(x)| } = r \ln{ |x| } \\
  \frac{ f'(x) }{ f(x) } = r \frac{1}{x} \Longrightarrow f'(x) = rx^{r-1}
\end{gathered}
\]  
\end{enumerate}

\exercisehead{36} Use the definition $a^x = e^{x \log{a}}$
\begin{enumerate}
\item $\log{a^x} = x \log{a}$ \bigskip \\
Taking the exponential is a well-defined inverse function to $\log$ so taking the $\log$ of both sides of the definition, we get $\log{ a^x} = x \log{a}$
\item $(ab)^x = a^x b^x$ \bigskip \\
\[
(ab)^x = e^{ x \log{ab}} = e^{x ( \log{a} + \log{b} )} = a^x b^x
\]
\item $a^x a^y = a^{x+y}$
\[
a^{x+y} = e^{ (x+y) \log{a} } = e^{x \log{a}}e^{y \log{a}} = a^x a^y 
\]
\item $(a^x)^y = (a^y)^x = a^{xy}$ 
\[
(a^x)^y = e^{xy \log{a}} = (a^y)^x = a^{xy}
\]
\item If $x = \log_a{y}$,  \\
Using the definition 
\[
\log_b{x} = \frac{ \log{x}}{ \log{b}} \quad \text{ if } b > 0, \, b \neq 1, \quad x > 0 
\]
so then
\[
  \log_a{y} = \frac{ \log{y}}{ \log{a}} = x \Longrightarrow \begin{aligned} 
    \log{y} & = x \log{a} \\
    e^{x \log{a}} & = e^{ \log{y}} = y = a^x 
\end{aligned}
\]
If $y = a^x$, \\
\[
\log{y} = x \log{a} \Longrightarrow x = \frac{ \log{y}}{ \log{a}} = \log_a{y}
\]
\end{enumerate}

\exercisehead{37} 
Let $f(x) = \frac{1}{2} (a^x + a^{-x}) $ if $a>0$.  
\[
\begin{gathered}
  f(x+y) = \frac{1}{2} (a^{x+y} + a^{-(x+y)})  \\
  f(x+y) +f(x-y) = \frac{1}{2} \left( a^{x+y} + a^{-(x+y)} + a^{x-y} + a^{-(x-y)} \right) \\
  f(x)f(y) = \frac{1}{4} (a^x + a^{-x})(a^y + a^{-y}) = \frac{1}{4} \left( a^{x+y} + a^{-x-y} + a^{-(x-y)} + a^{(x-y)} \right)
\end{gathered}
\]

\exercisehead{38} 
\[
\begin{gathered}
  f(x) = e^{cx}; \, f'(x) = ce^{cx}; \, \boxed{ f'(0) = c } \\
  \lim_{x\to 0} \frac{e^{cx} - 1}{x} = c\left( \lim_{x \to 0} \frac{ e^{cx} - 1 }{ cx} \right) = c \left( \lim_{cx \to 0} \frac{e^{cx} - 1 }{ cx } \right) = c \frac{ df(cx)}{ d(cx)}(0) = \frac{d}{dx} (e^{cx})(0) \\
  \lim_{x\to 0} \frac{ e^{cx} - 1 }{ x } = f'(0) = c
\end{gathered}
\]

\exercisehead{39}
\[
\begin{aligned}
  & g(x) = f(x) e^{-cx} \\
  & g'(x) = f' e^{-cx} + -c g = cg - cg = 0 \\
  & \boxed{ f = K e^{kx} }
\end{aligned}
\]

\exercisehead{40} Let $f$ be a function defined everywhere on the real axis.  Suppose also that $f$ satisfies the functional equation
\[
f(x+y) = f(x)f(y) \text{ for all $x$ and $y$ }
\]
\begin{enumerate}
\item 
\[
\begin{gathered}
  f(0) = f(0)f(0) = f^2(0) 
  \begin{aligned}
    & \text{ If $f(0) = 0$, then we're done } \\
    & \text{ If $f(0) \neq 0$ then } f(0) = 1 \text{ (by dividing both sides by $f(0)$ ) }
  \end{aligned} 
\end{gathered}
\] 
\item Take the derivative with respect to $x$ on both sides of the functional equation.
\[
\begin{gathered}
  \frac{df(x+y)}{d(x+y)}\frac{ d(x+y)}{dx}  = \frac{ df(x)}{ dx} f(y)  \quad \Longrightarrow \frac{ d(f(x+y))}{ d(x+y)} = f'(x) \frac{ f(x+y)}{f(x)} \\
  \boxed{ \text{ Let $y = -x+y$ } } \\
  \frac{df(y)}{dy} = f'(x) \frac{f(y)}{f(x) } \quad \, \Longrightarrow f'(x)f(y) = f'(y)f(x) 
\end{gathered}
\]
\item $\frac{f'(x)}{f(x)} = \frac{f'(y)}{f(y)} \quad \, \forall \, x,y$   \medskip \\
The only way they could do that for any arbitrary $x$, for any arbitrary $y$ they one could choose on \emph{either} side, is for them to both equal a constant \\
\quad \, $\Longrightarrow \frac{f'(y)}{f(y)} = c$
\item Referring to Exercise 39 of the same section, $f=e^{cx}$ since $f'(0)=1$
\end{enumerate}

\exercisehead{41}
\begin{enumerate}
\item 
\[
\begin{gathered}
  \begin{aligned}
    f & = e^x - 1 - x \\
    f' & = e^x - 1 \gtrless 0 \text{ if } x \gtrless 0 
  \end{aligned}
\quad \quad \, 
f(0) = e^0 - 1 - 0 = 0 
\quad \quad \, \begin{cases}
    e^x > 1 + x & \text{ for } x > 0  \\
   e^{-x} > 1 - x & \text{ for } x < 0 
\end{cases} 
\end{gathered}
\]
\item 
  \[
\begin{aligned}
  & \int_0^x e^t = e^x - 1 > x + \frac{1}{2}x^2 \quad \, \Longrightarrow e^{x} > 1 + x + \frac{1}{2} x^2 \\
  & -e^{-x} > -1 + x - \frac{x^2}{2} \quad \, \Longrightarrow e^{-x} < 1 - x + \frac{x^2}{2}
\end{aligned}
\]
\item
\[
\begin{aligned}
  & \int_0^x e^t = e^x - 1 > x + \frac{1}{2}x^2 + \frac{1}{3*2} x^3 \quad \, \Longrightarrow e^{x} > 1 + x + \frac{1}{2} x^2 + \frac{1}{3*2} x^3 \\
  & -e^{-x} > -1 + x - \frac{x^2}{2} + \frac{x^3}{3*2} \quad \, \Longrightarrow e^{-x} < 1 - x + \frac{x^2}{2} - \frac{x^3}{3*2}
\end{aligned}
\]
\item Suppose the $n$th case is true.  
\[
\begin{aligned}
&  e^x > \sum_{j=0}^n \frac{x^j}{j! } \quad \quad \, & e^{-x}  = \begin{cases} > \sum_{j=0}^{2m+1} \frac{x^j}{j!} & \\
    < \sum_{j=0}^{2m} \frac{x^j}{j! } & \end{cases} \\
& e^x > 1 + \sum_{j=0}^n \frac{x^{j+1}}{ (j+1)! } = 1 + \sum_{j=1}^{n+1} \frac{x^j}{j!} & = \sum_{j=0}^{n+1} \frac{x^j}{j!} \\ 
& -e^{-x} + 1 \begin{cases} > \sum_{j=0}^{2m+1} \frac{x^{j+1}}{ (j+1)! } & \\
    < \sum_{j=0}^{2m} \frac{x^{j+1}}{(j+1)!} & \end{cases} = & e^{-x} \begin{cases} < \sum_{j=0}^{2m+2} \frac{x^j}{j!} & \\
    > \sum_{j=0}^{2m+1} \frac{x^j}{j!} & \end{cases} 
\end{aligned}
\]
\end{enumerate}

\exercisehead{42} Using the result from Exercise 41,
\[
\left( 1 + \frac{x}{n} \right)^n = \sum_{j=0}^n \binom{n}{j} 1^{n-j} \left( \frac{x}{n} \right)^j = \sum_{j=0}^n \frac{n!}{ (n-j)!j! } \frac{x^j}{n^j} = \sum_{j=0}^n \frac{ n (n-1) \dots (n-j+1)}{ j!} \frac{x^j}{n^j} < \sum_{j=0}^n \frac{x^j}{j!} < e^x
\]

If you make this clever observation, the second inequality is easy to derive.
\[
\begin{gathered}
  x > 0 \quad \quad \, \frac{x}{n} > 0 \\
  e^{-\frac{x}{n}} > 1 - \frac{x}{n} \quad \, \Longrightarrow \left( e^{-x/n} \right)^n > \left( 1 - \frac{x}{n} \right)^n \\
  e^{-x} > (1- x/n)^n \quad \, \Longrightarrow e^x < (1 - x/n)^{-n}
\end{gathered}
\]

\exercisehead{43} $f(x,y) = x^y = e^{y \ln{x}}$
\[
\begin{aligned}
  & \partial_x f = x^y y /x \\
  & \partial_y f = x^y \ln{x}
\end{aligned}
\]

%-----------------------------------%-----------------------------------%-----------------------------------
\subsection*{ 6.19 Exercises - The hyperbolic functions } 
%-----------------------------------%-----------------------------------%-----------------------------------
\quad \\
\exercisehead{7} 
\[
2\sinh{x} \cosh{x} = 2 \frac{ e^x - e^{-x}}{ 2 } \frac{ e^x + e^{-x}}{ 2 } = \frac{1}{2} ( e^{2x} - e^{-x}) = \sinh{2x}
\]

\exercisehead{8} 
\[
\cosh^2{x} + \sinh^2{x} = \left( \frac{ e^x + e^{-x}}{ 2 } \right)^2 + \left( \frac{ e^x - e^{-x}}{ 2 } \right)^2  = \frac{1}{4} (e^{2x} + 2 + e^{-2x} + e^{2x} - 2 + e^{-2x} ) = \cosh{2x}
\]

\exercisehead{9}
\[
\cosh{x} + \sinh{x} = \frac{ e^x + e^{-x}}{ 2 } + \frac{ e^x - e^{-x}}{ 2 } = e^x 
\]

\exercisehead{10}
\[
\cosh{x}-\sinh{x} = \frac{ e^x + e^{-x}}{2} - \left( \frac{ e^x - e^{-x}}{ 2 } \right) = e^{-x} 
\]

\exercisehead{11} Use induction.  
\[
\begin{gathered}
  (\cosh{x} + \sinh{x})^2 = \cosh^2{x} + 2 \sinh{x} \cosh{x} + \sinh^2{x} = \cosh{2x} + \sinh{2x} \\
  \begin{aligned}
    (\cosh{x} + \sinh{x})^{n+1} & = (\cosh{x} + \sinh{x})(\cosh{nx}+\sinh{nx}) = \\ 
    & = \cosh{nx}\cosh{x} + \cosh{nx}\sinh{x} + \sinh{nx}\cosh{x} + \sinh{x}\sinh{nx} = \\
    & = 
    \begin{gathered} 
      \frac{ e^{nx} + e^{-nx}}{ 2 } \frac{ e^{x} + e^{-x}}{ 2 } + \frac{ e^{nx} + e^{-nx}}{ 2 } \frac{ e^{x} - e^{-x}}{ 2 } + \\
      + \frac{ e^{nx} - e^{-nx}}{ 2 } \frac{ e^{x} + e^{-x}}{ 2 } + \frac{ e^{nx} - e^{-nx}}{ 2 } \frac{ e^{x} - e^{-x}}{ 2 } 
      \end{gathered} \\
    & = \cosh{(n+1)x} + \sinh{(n+1)x} 
  \end{aligned}
\end{gathered}
\]

\exercisehead{12} 
\[
\cosh{2x} = \cosh^2{x} + \sinh^2{x} = 1 + 2\sinh^2{x}
\]

%-----------------------------------%-----------------------------------%-----------------------------------
\subsection*{ 6.22 Exercises - Derivatives of inverse functions, Inverses of the trigonometric functions }
%-----------------------------------%-----------------------------------%-----------------------------------
\quad \\
\exercisehead{1}
\[
 (\cos{x})' = -\sin{x} = - \sqrt{ 1 - \cos^2{x}} \, D\arccos{x} = \frac{ 1}{ - \sqrt{ 1 -x^2 }} \, -1 < x < 1 
\]

\exercisehead{2}
\[
\begin{gathered}
  (\tan{x})' = \sec^2{x} = \frac{ \sin^2{x} + \cos^2{x}}{ \cos^2{x}} = \tan^2{x} +1 \\
  D \arctan{x} = \frac{1}{1+x^2 }
\end{gathered}
\]

\exercisehead{3} 
\[
(\cot{x})' = -\csc^2{x} = - \frac{ (\sin^2{x} + \cos^2{x} )}{ \sin^2{x} } = -(1+\cot^2{x}) \Longrightarrow arccot{x} = -\frac{1}{1+x^2 }
\]

\exercisehead{4} 
\[
(\sec{y})' = \tan{y} \sec{y} = \sqrt{ \sec^2{y} - 1 } \sec{y}; |\sec{y}| > 1 \, \forall y \in \mathbb{R}
\]
If we choose to restrict $y$ such that $0\leq y \leq \pi$, then $(\sec{y})' > 0$.  Then we must make $\sec{y} \to |\sec{y}|$.  \medskip \\ 
$D arcsec{x} = \frac{1}{ |x| \sqrt{ x^2 -1 } }$

\exercisehead{5} 
\[
\begin{gathered}
  (\csc{y})' = -\cot{x}\csc{x} = -\csc{y}(\sqrt{ \csc^2{y} -1 } ) \\
  \text{ Let } y \text{ such that } \frac{ -\pi}{2} < y < \frac{ \pi}{2} \, (\csc{y}) <  0 \\
  D arccsc{x} = \frac{ 1}{ -|x| \sqrt{ x^2 -1 }}
\end{gathered}
\]

\exercisehead{6}
\[
\begin{gathered}
  \begin{aligned}
    &  (x arccot{x})' = arccot{x} - \frac{x}{ 1+x^2 } \\
    & \left( \frac{1}{2} \ln{ (1+x^2 )} \right)' = \frac{1x}{  (1+x^2) } \\
  \end{aligned} \\
  \int arccot{x} = x arccot{x} + \frac{1}{2} \ln{ (1+x^2) } + C 
\end{gathered}
\]

\exercisehead{7}
\[
\begin{gathered}
  \begin{aligned}
    & (x arcsec{x})' = arcsec{x} + \frac{ x }{ |x| \sqrt{ x^2 -1 } } \\
    & \left( \frac{x}{ |x| } \ln{ |x + \sqrt{ x^2 -1 } | } \right)' = \begin{cases} \frac{ 1 + \frac{ x}{ \sqrt{ x^2 -1 }} }{ |x + \sqrt{ x^2 -1 } | } & x >1 \\ \, & \, \\ - \frac{ -1 + \frac{x}{ \sqrt{ x^2 -1 }} }{ |x+ \sqrt{ x^2 -1 } | } & x < -1 \end{cases} =  \frac{ x }{ |x| \sqrt{ x^2 -1 } } 
  \end{aligned} \\  
  \Longrightarrow \int arcsec{x} dx = x arcsec{x} - \frac{x}{|x|} \log{ |x + \sqrt{ x^2 -1 } } | + C
\end{gathered}
\]

Take a note of this exercise.  \textbf{ When dealing with $(\mp x^2 \pm 1 )^{ \frac{2j+1}{2} }; j \in \mathbb{Z}$; try $x \pm \sqrt{ x^2 \pm 1 } $ combinations.}  It'll work out.  

\exercisehead{8} 
\[
\begin{gathered}
  \begin{aligned}
    & (x arccsc{x} )' =  arccsc{x} + \frac{ x}{ -|x| \sqrt{ x^2 - 1 }}  \\
    & \left( \frac{ x }{ |x| } \ln{ |x + \sqrt{ x^2 - 1 }| } \right)' = \begin{cases} 
      \frac{ 1 }{ x + \sqrt{ x^2 -1 } } \left( 1 + \frac{ x }{ \sqrt{ x^2 -1 }} \right)  = \frac{1}{ \sqrt{ x^2 - 1 }} \, & x >1 \\
      \frac{ -1 }{ \sqrt{ x^2 -1 }} & x < -1  \end{cases} 
  \end{aligned} \\
  \Longrightarrow \int arccsc{x} = x arccsc{x} + \frac{x}{|x|} \ln{ | x + \sqrt{ x^2 -1 } |  }
\end{gathered}
\]

\exercisehead{9} 
\[
\begin{gathered}
  \begin{aligned}
    & (x (\arcsin{x})^2 )' = (\arcsin{x})^2 + \frac{ 2x \arcsin{x}}{ \sqrt{ 1 - x^2 }}  \\
    & \left( \sqrt{ 1 - x^2 } \arcsin{x} \right)' = \frac{ -x}{ \sqrt{ 1 - x^2 }} \arcsin{x} +1    
  \end{aligned} \\
  \boxed{ \int (\arcsin{x})^2 = x (\arcsin{x})^2 + 2 \sqrt{ 1 - x^2 } \arcsin{x} -  2x  }
\end{gathered}
\]

\exercisehead{10} 
\[
\left( \frac{ - \arcsin{x} }{ x } \right)' = \frac{1}{x^2} \arcsin{x} - \frac{ 1 }{ x \sqrt{ 1-x^2 }}
\]
I would note how $x$ is in the denominator of the second term.  Again, reiterating,
\[
\begin{gathered}
  (\sqrt{ 1 -x^2 })' = \frac{ -x }{ \sqrt{ 1 - x^2 }} \\
  (y \pm \sqrt{ \pm 1 \mp x^2 })( y \mp \sqrt{ \pm 1 \mp x^2 } ) = y^2 - (\pm 1 \mp x^2 )
\end{gathered}
\]
Multiply by its ``conjugate.''  As we see, choose $y$ appropriately to get the desired denominator (that's achieved after differentiation).  Here, pick $y=1$.  
\[
\begin{gathered}
  (\ln{ (1+ \sqrt{ 1 - x^2 })} )' = \frac{ 1 }{ 1 + \sqrt{ 1 - x^2 }} \left( \frac{ -x }{ \sqrt{ 1 - x^2 }} \right) = \frac{ -x (1 - \sqrt{ 1 -x^2 }) }{ \sqrt{ 1 - x^2 } (x^2 )} = \frac{ - (1 - \sqrt{ 1 - x^2 })}{ x \sqrt{ 1 -x^2 } } \\
  \Longrightarrow \int \frac{ \arcsin{x} }{ x^2 } = \ln{ \left| \frac{ 1 - \sqrt{ 1 - x^2 }}{ x } \right| } - \frac{ \arcsin{x}}{ x } + C 
\end{gathered}
\]

\exercisehead{11} \begin{enumerate}
  \item 
\[
D\left( arccot{x} - \arctan{\frac{1}{x}} \right) = \frac{ -1}{ x^2 +  1} - \frac{1}{ 1 + \left( \frac{1}{x} \right)^2 } \left( \frac{ -1 }{ x^2 } \right) = 0 
\]
\item $ arccot{x} - \arctan{\frac{1}{x}} = C$ \medskip \\
  Now $arccot{x} = \frac{ \pi}{2} - \arctan{x}$.   
\[
\begin{gathered}
  \frac{ \pi}{2} - \arctan{x} - \arctan{ \frac{1}{x} } = C \Longrightarrow \frac{ \pi}{2} - C = \arctan{x} + \arctan{ \frac{1}{x} } \\
  x \to \infty \Longrightarrow \frac{ \pi}{2} - C = \frac{ \pi}{2} + 0 \Longrightarrow C = 0 \\
  \text{ but } x \to -\infty \\
  \frac{ \pi}{2} - C = - \frac{ \pi}{2} + 0 \, \Longrightarrow C = \pi 
\end{gathered}
\]
There are problems with the choice of brances for $arccot{x}, \arctan{ \frac{1}{x}}$, even though the derivatives work in all cases.   
\end{enumerate}

\exercisehead{12} 
\[
f' = \frac{1}{ \sqrt{ 1 - \left( \frac{x}{2} \right)^2 }} \left( \frac{1}{2} \right) 
\]

\exercisehead{13} 
\[
f' = \frac{ -1 }{ \sqrt{ 1 - \left( \frac{ 1 - x }{ \sqrt{2}} \right)^2 } } \left( \frac{ -1 }{ \sqrt{2}} \right) = \frac{ 1 }{ \sqrt{ 1 +2 x -x^2 } }
\]

\exercisehead{14} $f= \arccos{ \frac{1}{x}}$.  
\[
 f' = \frac{ -1}{ \sqrt{ 1 - \left( \frac{1}{x} \right)^2 } } \left( \frac{-1}{x^2 } \right) = \frac{ 1 }{ \sqrt{ x^2 - 1 } |x| }
\]

\exercisehead{15} \[
f(x) = \arcsin{ (\sin{x})} = \frac{1}{ \sqrt{ 1- \sin^2{x}} } \cos{x} = \frac{ \cos{x}}{ |\cos{x}| }
\]

\exercisehead{16} \[
\frac{1}{2\ sqrt{x}} - \frac{1}{ x+ 1 } \frac{1}{ 2 \sqrt{x}} = \frac{ \sqrt{x}}{ 2 (x+1) } 
\]

\exercisehead{17} \[
\frac{1}{1+x^2 } + \frac{ x^2 }{ 1+x^6}
\]

\exercisehead{18} 
\[
\frac{ 1 }{ \sqrt{ 1 - \left( \sqrt{ \frac{1-x^2 }{ 1 + x^2 } } \right)^2 }} \left( \frac{ -2x (2) }{ (1+x^2 )^2 } \right) = \frac{ \sqrt{ 1 + x^2 }}{ \sqrt{ 1 + x^2 - (1-x^2 )} } \frac{ -4x}{ (1+x^2 )^2 } = \frac{ -4 }{ ( 1+x^2 )^{3/2} \sqrt{2}} 
\]

\exercisehead{19} $f= \arctan{ \tan^2{x}}$
\[
\frac{1}{ 1 + \tan^4{x} } \left( 2 \tan{x} \sec^2{x} \right) = \frac{ 2 \tan{x} \sec^2{x}}{ 1 + \tan^4{x}}
\]

\exercisehead{20} 
\[
f' = \frac{1}{ 1 + (x + \sqrt{ 1 +x^2 })^2 } (1 + \frac{x}{ \sqrt{ 1 + x^2 } } ) = \frac{ \sqrt{1+x^2 } + x }{ 1 + (x+\sqrt{ 1+ x^2 })^2 }
\]

\exercisehead{21} 
\[
f' = \frac{1}{ \sqrt{ 1- (\sin^2{x} + \cos^2{x} - 2\sin{x}\cos{x})}} ( \cos{x} + \sin{x})  = \frac{ \cos{x} + \sin{x}}{ \sqrt{ 2 \sin{x} \cos{x}} }
\]

\exercisehead{22} 
\[
f' = (\arccos{ \sqrt{ 1 -x^2 }})' = \frac{ 1}{ - \sqrt{ 1 - (1-x^2 ) } } = - \frac{1}{ |x| }
\]

\exercisehead{23} 
\[
f' = \frac{ 1 }{ 1 + \left( \frac{1+x}{1-x} \right)^2 } \left( \frac{2}{ (1-x)^2 } \right) = \frac{ 2}{ (1-x)^2 + (1+x)^2 } = \frac{ 2 }{ (1-2x + x^2 + 1 + 2x + x^2 )} = \frac{1}{ 1 + x^2 }
\]

\exercisehead{24} $f = (\arccos{(x^2)})^{-2}$ 
\[
f' = -2 (\arccos{x^2 })^{-3} \left( \frac{-1}{ \sqrt{ 1 - x^2 } } \right) (2x) = \frac{ 4x (\arccos{x^2 })^{-3} }{ \sqrt{ 1 - x^4} } 
\]

\exercisehead{25} 
\[
f' = \left( \frac{1}{ \arccos{ \frac{1}{ \sqrt{x}} } } \right) \left( \frac{ -1 }{ \sqrt{ 1 - \frac{1}{x} } } \right) \left( \frac{-1}{ 2 x^{3/2} } \right) = \frac{ 1 }{ 2 \arccos{ \frac{1}{\sqrt{x}} } \sqrt{ x^3 - x^2 }} 
\]

\exercisehead{26} $\frac{dy}{dx} = \frac{ x+ y }{ x - y}$.  
\[
\begin{gathered} 
  \left( \arctan{ \frac{y}{x} } \right)' = \frac{ 1 }{ 1 + \left( \frac{y}{x} \right)^2 } \left( \frac{ y' x - y }{ x^2 } \right) = \left( \frac{1}{2} \ln{ (x^2 + y^2 )} \right)' = \frac{1}{2} \frac{1}{ (x^2 + y^2)} (2x + 2yy') \\
  \Longrightarrow y' = \frac{x+y}{ x-y}
\end{gathered}
\]

\exercisehead{27} 
\[
\begin{aligned}
  \ln{y} & = \ln{ (\arcsin{x}) } - \frac{1}{2} \ln{ (1-x^2 ) } \\
  \frac{y'}{y} & = \frac{1}{ \arcsin{ x }} \left( \frac{1}{ \sqrt{ 1 - x^2 } } \right) - \frac{1}{2} \frac{1}{ 1-x^2 } (-2x) = \frac{1}{ \arcsin{ x } \sqrt{1-x^2 } } + \frac{x}{ 1-x^2 } \\
    & y = \frac{ \arcsin{x}}{ \sqrt{ 1 - x^2 }}  \\
    y' & = \left( \frac{1}{ 1-x^2} \right) + \frac{ (\arcsin{x} )x }{ (1-x^2 )^{3/2} } = \frac{ \sqrt{ 1 - x^2 } + x (\arcsin{x})}{ (1-x^2 )^{3/2} } \\
    \ln{y'} & = \ln{ ( \sqrt{1-x^2 } + x \arcsin{x} )} - \frac{3}{2} \ln{ (1-x^2 )} \\
    \frac{ y''}{y'} & = \frac{1}{ \sqrt{ 1 - x^2 } + x \arcsin{x} }  \left( \frac{ -x}{ \sqrt{ 1 - x^2 }} + \arcsin{x} + \frac{x}{ \sqrt{ 1 - x^2 }} \right) - \frac{3}{2} \frac{ (-2x)}{ 1-x^2 }   \\
    y'' &= y' \frac{ \arcsin{x}}{ \sqrt{ 1 - x^2 } + x \arcsin{x} } + \frac{ y' 3x }{ 1-x^2 } = \boxed{ \frac{ \arcsin{x}}{ (1-x^2)^{3/2} } + \frac{ (\sqrt{ 1 - x^2 } + x \arcsin{x} )(3x) }{ (1-x^2)^{3/2} } }
\end{aligned}
\]

\exercisehead{28} 
\[
\begin{gathered}
f' = \frac{1}{ 1+x^2} -1 + x^2 = \frac{ 1 - 1- x^2 + x^4}{ 1+x^2 } = \frac{x^4}{ 1+x^2 } \geq 0 \, \forall x  \\
\text{ since } f(0) = \arctan{0} -0 + 0 = 0 , \quad \arctan{x} > x - \frac{x^3}{3}, \, \forall x > 0 
\end{gathered}
\]

\exercisehead{29} 
\[
\int \frac{dx}{ \sqrt{ a^2 - x^2 } }, \, a\neq 0  \, \Longrightarrow \boxed{ \arcsin{ \frac{x}{a} } }
\]

\exercisehead{30} 
\[
\int \frac{dx}{ \sqrt{ 2 - (x+1)^2 } } = \int \frac{ dx}{ \sqrt{2} \sqrt{ 1 - \left( \frac{x+1}{\sqrt{2} } \right)^2 } } = \arcsin{ \frac{ x+1}{\sqrt{2} } }
\]

\exercisehead{31} 
\[
\int \frac{dx}{ a^2 \left( 1 + \left( \frac{ x }{ a } \right)^2 \right)} = \frac{1}{a} \arctan{ \frac{x}{a} }
\]

\exercisehead{32} 
\[
\frac{dx}{ a (1 + \left( \frac{ \sqrt{b} a }{ \sqrt{a} } \right)^2 ) } = \frac{1}{ \sqrt{ ba}} \arctan{ \frac{ \sqrt{b} x }{ \sqrt{a} } } 
\]

\exercisehead{33} 
\[
\int \frac{dx}{ \left( x - \frac{1}{2} \right)^2 + \frac{7}{4} }  = \frac{4}{7} \int \frac{ dx}{ \left( \frac{2}{\sqrt{7} } \left( x-\frac{1}{2} \right) \right)^2 + 1 } = \boxed{ \frac{2}{ \sqrt{7}} \arctan{ \frac{ 2 \left(  x - \frac{1}{2} \right)}{ \sqrt{7} } } }
\]

\exercisehead{34} 
\[
\begin{gathered}
\begin{aligned}
  \left( \frac{ x^2 \arctan{x} }{ 2 } \right)' & = x \arctan{x} + \frac{x^2}{2} \frac{1}{ 1 +x^2 } = x \arctan{x} + \frac{1}{2} \left( 1 - \frac{1}{1+x^2 } \right) \\ 
  \left( \frac{1}{2} \left( x - \arctan{ x } \right) \right)' & = \frac{1}{2} \left( 1 - \frac{1}{1+x^2 } \right) 
\end{aligned} \\
\int x \arctan{x} = x \arctan{x} + -\frac{1}{2} (x - \arctan{x} )
\end{gathered}
\]

\exercisehead{35} 
\[
\begin{gathered}
  \begin{aligned}
    \left( \frac{x^3}{3} \arccos{x} \right)' & = x^2 \arccos{x} + \frac{x^3}{3} \frac{ -1}{ \sqrt{1 -x^2 } } \\
    (x^2 \sqrt{ 1 - x^2 })' & = 2x \sqrt{ 1 -x^2 } + -\frac{x^3}{ \sqrt{ 1 - x^2 }} \\
    ((1-x^2)^{3/2})' = \frac{3}{2} (-2x) (1-x^2)^{1/2} = -3x (1-x^2 )^{1/2}
  \end{aligned} \\
  \boxed{ \int x^2 \arccos{x} = \frac{x^3}{3} \arccos{x} - \frac{1}{3} x^2 \sqrt{ 1 -x^2} - \frac{9}{2} (1-x^2)^{3/2} }
\end{gathered}
\]

\exercisehead{36}
\[
\begin{gathered}
  \begin{aligned}
    \left( \frac{ x^2 (\arctan{x} )^2 }{ 2 } \right)' & = x (\arctan{x} )^2 + x^2 \left( \frac{1}{1+x^2 } \right) \arctan{x}  = x (\arctan{x} )^2 + \left( 1 - \left( \frac{1}{1+x^2 } \right) \right)\arctan{x} \\
    \left( \frac{ (\arctan{x} )^2 }{ 2 } \right)' & = \frac{ \arctan{x}}{ 1+x^2 } \\
    (x \arctan{x})' & = \arctan{x} + \frac{ x}{ 1+x^2 } 
\end{aligned} \\
  \int x (\arctan{x})^2 dx = \frac{ x^2 (\arctan{x})^2 }{  2} - \left( x\arctan{x} - \frac{ \ln{(1+x^2)} }{ 2} \right) + \frac{ (\arctan{x})^2 }{ 2 } 
\end{gathered}
\]

\exercisehead{37} 
\[
\begin{gathered}
  \begin{aligned}
    (\arctan{ \sqrt{x}})' & = \left( \frac{1}{1+x} \right) \left( \frac{1}{ 2\sqrt{x} } \right) \\
    (x \arctan{ \sqrt{x} })' & = \arctan{\sqrt{x}} + \frac{ \sqrt{x}}{ 2 (1+x) } = \arctan{ \sqrt{x}} + \frac{1}{2} \left( \frac{1}{\sqrt{x}} - \frac{1}{ \sqrt{x} (1+x) } \right) \\
    (x\arctan{ \sqrt{x}} + \arctan{ \sqrt{x}} + -x^{1/2} )' & = \arctan{ \sqrt{x}} + 0
  \end{aligned} \\
  \boxed{ \int \arctan{ \sqrt{x}} = x\arctan{ \sqrt{ x }} + \arctan{ \sqrt{x} } - x^{1/2} }
\end{gathered}
\]

\exercisehead{38} From the previous exercise,
\[
\int \frac{ \arctan{ \sqrt{x}}}{ \sqrt{x} (1+x) } dx = ( \arctan{ \sqrt{x}})^2 
\]

\exercisehead{39} Let $x = \sin{u}$
\[
\int \sqrt{ 1 -x^2} dx = \int \cos^2{u} du = \frac{ u}{2} + \frac{ \sin{ 2u}}{ 4} = \frac{ \arcsin{x}}{2} + \frac{ x \sqrt{ 1-x^2 }}{ 4 }
\]

\exercisehead{40} 
\[
\begin{gathered}
  \int \frac{ x e^{\arctan{x}}}{ (1+x^2)^{3/2 } } \\
  \begin{aligned}
    \left( \frac{ e^{\arctan{x}}}{ \sqrt{ 1 +x^2 } } \right)' & = \frac{ -x e^{\arctan{x}}}{ (1+x^2)^{3/2} } + \frac{ e^{\arctan{x}}}{ (1+x^2)^{3/2} } \\
    \left( \frac{ x e^{\arctan{x}}  }{ \sqrt{ 1 +x^2 }}\right)' &= \frac{ e^{\arctan{x}} }{ \sqrt{ 1 +x^2 } } + - \frac{ x^2 e^{\arctan{x}}}{ ( 1 +x^2 )^{3/2} } + \frac{ x e^{\arctan{x}}}{ ( 1 +x^2 )^{3/2} } = \frac{ e^{\arctan{x}} }{ ( 1 +x^2 )^{3/2} } + \frac{ x e^{\arctan{x}} }{ ( 1 +x^2 )^{3/2} } 
  \end{aligned} \\
  \frac{1}{2} \left( \frac{ x e^{\arctan{x}}}{ \sqrt{ 1+x^2 } } - \frac{ e^{\arctan{x}}}{ \sqrt{ 1 + x^2 } } \right)' = \frac{ xe^{\arctan{x}}}{ (1+x^2)^{3/2} } 
\end{gathered}
\] 

\exercisehead{41} From the previous exercise,
\[
\frac{1}{2} \left( \frac{ x e^{\arctan{x}}}{ \sqrt{ 1+x^2} } + \frac{ e^{\arctan{x}}}{ \sqrt{ 1+x^2 }} + C \right)
\]

\exercisehead{42} 
\[
\begin{gathered}
  \text{ Since } - \left( \frac{ x (1+x^2)^{-1} }{ 2 } \right)' = \frac{ x^2}{ (1+x^2)^2 } - \frac{ 1 }{ 2 (1+x^2) } \\
  \int \frac{ x^2}{ (1+x^2)^2} dx = \frac{ -x}{ 2(1+x^2)} + \frac{1}{2} \arctan{x}
\end{gathered}
\]

\exercisehead{43} $\arctan{e^x}$.  

\exercisehead{44} 
\[
\begin{gathered}
  \int \frac{ arccot{e^x}}{ e^x } dx \\
  \begin{aligned}
    & ( arccot{ e^x } )' = \frac{ -e^x}{ 1+e^{3x } } \\
    & -(e^{-x} arccot{ e^x } )' = e^{-x} arccot{ e^x } + \frac{ e^{-x} (-1) e^x }{ 1+ e^{2x} } = e^{-x} arccot{ e^x } + -\left( 1 - \frac{ e^{2x}}{ 1 + e^{2x} } \right) \\
    & (\ln{ (1+e^{2x} ) } )' = \frac{ 2 e^{2x} }{ 1 +e^{2x} } 
  \end{aligned} \\
  \boxed{ (-e^{-x} arccot{ e^x } + x - \frac{1}{2} \ln{ (1+e^{2x} ) } )' } = e^{-x} arccot{e^x }
\end{gathered}
\]

\exercisehead{45} 
\[
\int \sqrt{ \frac{ a+x}{ a-x } } dx = \int \frac{ a +x}{ \sqrt{ a^2 - x^2 } } dx = \int \left( \frac{ 1a }{ \sqrt{ 1 - \left( \frac{x}{a} \right)^2 } } + \frac{ x }{ \sqrt{ a^2 - x^2 } } \right) dx = a \arcsin{ \frac{x}{a} } + - \sqrt{ a^2 - x^2 }
\]

\exercisehead{46} 
\[
\begin{gathered}
  \begin{aligned} 
    \int \sqrt{ x -a} \sqrt{ b-x} dx & = \int \sqrt{ bx -ab -x^2 + ax } dx = \\
    & = \int \sqrt{ -\left( x - \left( \frac{ a+b}{2} \right) \right) \left( x - \left( \frac{ a+b}{2} \right) \right) + \frac{ a^2 + b^2 }{ 4} - \frac{ 2 ab}{4} } = \\ 
    & = \int \sqrt{ \left( \frac{ a -b}{2 } \right)^2 - \left( x - \left( \frac{ a+b}{2} \right) \right)^2 } = \left( \frac{ a-b}{2} \right) \int \sqrt{ 1 - \left( \frac{ x - \left( \frac{ a+b}{2 } \right) }{ \left( \frac{ a-b}{2} \right)} \right)^2 } dx = \\
    & = \left( \frac{ a-b}{2} \right)^2 \int \sqrt{ 1- u^2}= \\
    & = \boxed{  \left( \frac{ a-b}{2 } \right)^2 \frac{ \arcsin{ \left( \frac{ 2x - (a+b)}{ a-b } \right) } }{2 } + \frac{ 2x - (a+b) }{ 2 (a-b)^2 } \sqrt{ (a-b)^2 - (2x -(a+b))^2 } }
  \end{aligned} \\
  \text{ Since, recall, } \\
  \left( \frac{ \arcsin{x}}{ 2 } + \frac{1}{2} x \sqrt{ 1-x^2 } \right)' = \frac{1}{2} \frac{ 1 }{ \sqrt{ 1 -x^2 } } + \frac{ \sqrt{ 1-x^2 }}{ 2 } + \frac{1}{4} \frac{ x (-2x)}{ \sqrt{ 1-x^2 } } = \sqrt{ 1 -x^2 } 
\end{gathered}
\]

\exercisehead{47} \textbf{ Wow!} $\boxed{ \int \frac{dx}{ \sqrt{ (x-a)(b-x) }} } $
\[
\begin{gathered}
  \begin{aligned}
    x-a & = (b-a) \sin^2{u} \\
    dx & = (b-a)(2) \sin{u}\cos{u} du \\
    b-x & = (a-b)\sin^2{u} + b-a = \boxed{ (b-a)(\cos^2{u}) }
\end{aligned} \\
  \int \frac{dx}{ \sqrt{ (x-a)(b-x) } } = \int \frac{ (b-a)(2) \sin{u} \cos{u} du }{ \sqrt{ b-a} \cos{u} \sqrt{ b-a} \sin{u} } = 2 u = \boxed{ 2 \arcsin{ \sqrt{ \frac{ x-a}{b-a} } } }
\end{gathered}
\]

%-----------------------------------%-----------------------------------%-----------------------------------
\subsection*{ 6.25 Exercises - Integration by partial fractions, Integrals which can be transformed into integrals of rational functions }
%-----------------------------------%-----------------------------------%-----------------------------------
\quad \\
\exercisehead{1} $\int \frac{ 2x + 3}{ (x-2)(x+5) } = \int \left( \frac{1}{ x-2} \right) + \left( \frac{1}{ x+5} \right) = \ln{ (x-2)} + \ln{ (x+5) }$

\exercisehead{2} $\int \frac{ x dx }{ (x+1)(x+2)(x+3) }$ 
\[
\begin{gathered}
  \frac{A}{ x+1} + \frac{B}{ x+2} + \frac{C}{x+3} = A (x^2 + 5x + 6 ) + B(x^2 + 4x + 3 ) + C (x^2 + 3x + 2 ) \\
  \Longrightarrow \left[ \begin{matrix} 1 & 1 & 1 \\ 
       5 & 4 & 3 \\ 
       6 & 3 & 2 \end{matrix} \right]\left[ \begin{matrix} A \\ B \\ C \end{matrix} \right] = \left[ \begin{matrix} 0 \\ 1 \\ 0 \end{matrix} \right] \quad \, \Longrightarrow \left[ \begin{matrix} 1 & 1 & 1 \\ 
      5 & 4 & 3 \\
      6 & 3 & 2 \end{matrix} \right. \left| \begin{matrix} 0 \\ 1 \\ 0 \end{matrix} \right] = \left[ \begin{matrix} 1 & & 0 \\
      & 1 & 0 \\
      &  0 & 1 
\end{matrix} \right. \left| \begin{matrix} - 1/2 \\
      2 \\
      -3/2 \end{matrix} \right] \\
A = -1/2, \quad \, B = 2 , \quad \, C = -3/2 \\
\Longrightarrow \boxed{ \frac{-1}{2} \ln{ (x+1)} + 2 \ln{ (x+2)} + \frac{-3}{2} \ln{ (x+3) } }
\end{gathered}
\]

\exercisehead{3} $\int \frac{x}{ (x-2)(x-1)} = \int \frac{2}{x-2} + \frac{-1}{x-1} = 2 \ln{x-2} - \ln{(x-1)} $

\exercisehead{4} $\int \frac{x^4 + 2x - 6 }{ x^3 + x^2 -2x } dx$
\[
\begin{gathered}
  \frac{x^4 + 2x - 6 }{ x^3 + x^2 - 2x } = x-1 + \frac{ 3(x^2 -2 )}{ x^3 + x^2 - 2x } \quad \, \text{ (do long division) } \\
  \int x - 1 + \frac{ 3(x^2-2) }{ x(x+2)(x-1) } = \frac{1}{2} x^2 - x + 3 \int \frac{x^2 - 2}{ x(x+2)(x-1) } \\
  \int \frac{x^2 - 2}{ x(x+2)(x-1) } = \int \frac{1}{x} + \frac{1/3}{ x+2} + \frac{-1/3}{x-1} = \ln{x} + \frac{1}{3} \ln{x+2} - \frac{1}{3} \ln{x-1} \\
\Longrightarrow \boxed{ \frac{1}{2} x^2 -x + 3 \ln{x} + \ln{ x+2} - \ln{x-1} }
\end{gathered}
\]

\exercisehead{5}  $\int \frac{ 8x^3 +7}{ (x+1)(2x+1)^3 } dx $
\[
\begin{gathered}
  \frac{ 8x^3 + 7 }{ (x+1)(2x+1)^3 } = \frac{A}{ (2x+1)^3 } + \frac{B}{ (2x+1)^2 } + \frac{C}{ (2x+1)} + \frac{D}{(x+1) } \\
  8x^3 + 7 = A(x+1) + B(2x^2 + 3x + 1 ) + (4x^3 + 8x^2 + 5x + 1) C + D (8x^3 + 12x^2 + 6x + 1 ) \\
  \Longrightarrow \left[ \begin{matrix} 0 & 0 & 4 & 8 \\
      0 & 2 & 8 & 12 \\
      1 & 3 & 5 & 6 \\
      1 & 1 & 1 & 1 \end{matrix} \right] \left[ \begin{matrix} A \\ B \\ C \\ D \end{matrix} \right] = \left[ \begin{matrix} 8 \\ 0 \\ 0 \\ 7 \end{matrix} \right] \Longrightarrow \left[ \begin{matrix} 1 & 0 & 0 & 0 \\
      & 1 & 0 & 0 \\
      & & 1 & 0 \\
      & & 0 & 1 
\end{matrix} \left. \right| \begin{matrix} 12 \\ -6 \\ 0 \\ 1 \end{matrix} \right]  \\
  A = 12, \quad B = -6, \quad C = 0 , \quad D = 1 \\
  \int \frac{12}{ (2x+1)^3 } + \frac{-6}{(2x+1)^2 } + \frac{1}{x+1}  = \boxed{ \frac{ -6 (2x+1)^{-2}}{ 2 } + \frac{ 6 (2x+1)^{-1}}{ 2 } + \ln{(x+1)} }
\end{gathered}
\]

\exercisehead{6} $\int \frac{4x^2 + x + 1 }{ (x-1)(x^2 + x + 1 ) }$ 
\[
\begin{gathered}
  \frac{ 4x^2 + x + 1 }{ (x-1)(x^2 + x + 1 ) } = \frac{A}{x-1} + \frac{Bx + C}{ x^2 + x + 1 } \quad \Longrightarrow A(x^2 + x + 1 ) + (Bx+ C)(x-1) = 4x^2 + x + 1 \\
  \Longrightarrow \left[ \begin{matrix} 1 & 1 & \\ 
      1 & -1 & 1 \\
      1 & 0 & -1 \end{matrix} \right]\left[ \begin{matrix} A \\ B \\ C \end{matrix} \right] = \left[ \begin{matrix} 4 \\ 1 \\ 1 \end{matrix} \right] \quad \Longrightarrow A = 2, \quad B = 2 ,\quad C = 1  \\
  \Longrightarrow \int \frac{2}{x-1} + \frac{ 2x+1}{ x^2 + x + 1 } = \boxed{ 2 \ln{ |x-1| }  + \ln{ |x^2 + x + 1 | } }
\end{gathered}
\]

\exercisehead{7} $\int \frac{x^4 dx}{ x^4 + 5x^2 + 4 }$ \medskip \\
Doing the long division, $ \frac{x^4}{ x^4 + 5 x^2 + 4} = 1  + - \left( \frac{ 5x^2 +4 }{ (x^2 + 1)(x^2 +4) } \right)$
\[
\begin{gathered}
  \frac{ Ax + B}{ x^2 + 1 } + \frac{ Cx + D}{ x^2 +4 } = \frac{ 5x^2 + 4 }{ (x^2 + 1 )(x^2 +4) } \\
  \text{ It could be seen that } A +C = 0, \quad 4A + C = 0 \quad \, \text{ so } A = C = 0 \\
  \begin{aligned}
    B + D & = 5 \\
    4B + D & = 4 
  \end{aligned} \quad \quad \, \begin{aligned}
    B & = \frac{-1}{3} \\
    D & = \frac{16}{3} 
  \end{aligned} \\
\Longrightarrow \int 1 - \frac{ 5x^2 + 4 }{ (x^2 + 1)(x^2 +4) } = x - \int \frac{ -1/3}{ x^2 + 1 } + \frac{ 16/3}{ x^2 +4 } = \boxed{ x + \frac{1}{3} \arctan{x} + 4/3\arctan{x/2} + C }
\end{gathered}
\]

\exercisehead{8} $\int \frac{x+2}{x(x+1)} dx = \int \frac{1}{x+1} + \frac{2}{ x(x+1)} =\ln{ |x+1|} + 2 \int \frac{1}{x} - \frac{1}{x+1} = \boxed{ -\ln{ |x+1|} + 2 \ln{x} }$

\exercisehead{9} $\int \frac{dx}{ x (x^2 + 1)^2 } = \int \frac{A}{x} + \frac{Bx+C}{ (x^2+1)} + \frac{Dx + E }{ (x^2 + 1)^2 }$ \medskip \\
\[
\begin{gathered}
  \frac{ A(x^4 + 2x^2 + 1 ) + x (Bx+C) (x^2 + 1 ) + Dx^2 + Ex }{ x(x^2 +1)^2 } = \frac{ A(x^4 + 2x^2 + 1 ) + Bx^4 + Cx^3 + Bx^2 + Cx + Dx^2 + Ex }{ x (x^2 + 1)^2 } \\
  \Longrightarrow A = 1 ; \quad B = -1 ; \quad D = -1 ; \quad C = 0 ; \quad E = 0 \\
  \int \frac{1}{x} + \frac{ -x}{ x^2 + 1 } + \frac{-x}{(x^2 + 1)^2 } = \boxed{ \ln{x} +  - \frac{ \ln{ |x^2 + 1 | } }{2} + \frac{ (x^2 + 1)^{-1}}{2} } 
\end{gathered}
\]

\exercisehead{10} $\int \frac{dx}{ (x+1)(x+2)^2(x+3)^3}$ 

\exercisehead{11} $\int \frac{x}{ (x+1)^2 } dx $ 
\[
\begin{gathered}
\frac{x}{ (x+1)^2 } = \frac{A}{ x+1} + \frac{B}{ (x+1)^2 } \quad \quad \, \Longrightarrow \begin{gathered} 
  x = A(x+1) + B \\
  A=1; \quad \, B = -1 
\end{gathered} \\
\int \left( \frac{1}{ x + 1} + \frac{ -1}{ (x+1)^2 } \right) dx = \ln{x+1} + \frac{1}{ x+1} +C 
\end{gathered}
\]

\exercisehead{12} $\int \frac{dx}{ x (x^2 - 1 ) } = \int \frac{ dx}{ x (x-1)(x+1) } = \int \frac{A}{x} + \frac{B}{x-1} \frac{C}{ x + 1}$
\[
\begin{gathered}
A(x^2 - 1) + Bx(x+1) + Cx(x-1) = Ax^2 - A + Bx^2 + Bx + Cx^2 - Cx  \quad \quad \, \Longrightarrow A = -1, \, B = \frac{1}{2} = C  \\
\int \frac{-1}{x} + \frac{1/2}{ x-1} + \frac{1/2}{ x+1} = -\ln{ x } + \frac{1}{2} \ln{ |x-1| } + \frac{1}{2} \ln{ |x+1| }
\end{gathered}
\]

\exercisehead{13} $\int \frac{x^2 dx }{ x^2 + x -6 } = \int \frac{ x^2 dx }{ (x+3)(x-2) } $ \\
The easiest way to approach this problem is to notice that this is an improper fraction and to do long division: \\
$ \frac{x^2}{x^2 + x - 6} = 1 + \frac{ 6 - x }{ x^2 + x -6 }   $   \medskip \\
\[
\begin{gathered}
  \frac{A}{x+3} + \frac{B}{x-2} \Longrightarrow \begin{aligned}
    6 & = A(x-2) + B(x+3) \\
    & A = \frac{-6}{5} ; \quad \, B = \frac{6}{5} 
\end{aligned} \quad \quad \, 
\begin{aligned}
  -x & = (A+ B) x - 2A + 3 B \\
  2A & = 3 B \\
  & A = \frac{3B}{2} \quad \Longrightarrow \begin{aligned} 
    B & = -2/5 \\
    A & -3/5
    \end{aligned} 
\end{aligned} \\
\Longrightarrow \int 1 + \frac{ -6/5}{x+3} + \frac{ 6/5}{x-2} + \frac{-3/5}{ x+3} + \frac{ -2/5}{x-2} = \boxed{ \frac{-9}{5} \ln{ |x+3| } + \frac{4}{5} \ln{ |x-2| } + x + C }
\end{gathered}
\]

\exercisehead{14} $\int \frac{ x+2}{ (x-2)^2 } = \int \frac{ x - 2 + 4}{ (x-2)^2 } = \ln{ |x-2| } + \int \frac{4}{ (x-2)^2 } = \boxed{ \ln{ |x-2| } + -4 (x-2)^{-1} }$

\exercisehead{15} $\int \frac{dx}{ (x-2)^2 (x^2 - 4x + 5) }$ \\
Consider the denominator with its $x^2 - 4 x +5$.  Usually, we would try a partial fraction form such as $\frac{A}{x-2} + \frac{B}{(x-2)^2} + \frac{ Cx + D }{ x^2 - 4x + 5} $, but the algebra will get messy.  Instead, it helps to be clever here.

\[
\begin{gathered}
  \frac{1}{ (x-2)^2 (x^2 - 4x + 4 + 1 ) } = \frac{1}{ (x-2)^2 ( (x-2)^2 + 1 ) } = \frac{1}{ (x-2)^2 } - \frac{1}{ (x-2)^2 + 1 } \\
  \Longrightarrow \int \frac{dx}{ (x-2)^2 (x^2 - 4 x + 5) } = \int \frac{1}{ (x-2)^2 } - \frac{1}{ (x-2)^2 + 1 } = \boxed{ -(x-2)^{-1} - \arctan{ (x-2) } + C  }
\end{gathered}
\]

\exercisehead{16} $\int \frac{ (x-3) dx }{ x^3 + 3x^2 + 2x } = \int \frac{ (x-3)dx }{ x(x+2 )(x+1) }$
\[
\begin{gathered}
  \int \frac{ (x-3) dx}{ x (x+2)(x+1) } = \int \frac{1}{(x+2)(x+1) } + -3 \int \frac{1}{ x (x+2)(x+1) } \\
  \frac{1}{ (x+2)(x+1) } = \frac{ -1}{x+2} + \frac{ 1}{x+1} \\
  \frac{1}{ x(x+2)(x+1) } = \frac{A}{x} + \frac{ B}{x+2} + \frac{C}{x+1}
\end{gathered}
\]
Now to solve for $A,B,C$ in the last expression, it is useful to use Gaussian elimination for this system of three linear equations:
\[
\begin{gathered}
  \left[ \begin{matrix} 1 & 1 & 1 \\ 3 & 1 & 2 \\ 2 & 0 & 0 \end{matrix} \right] \left[ \begin{matrix} A \\ B \\ C \end{matrix} \right] = \left[ \begin{matrix} 0 \\ 0 \\ 1 \end{matrix} \right] \\
  \left[ \begin{matrix} 1 & 1 & 1 \\ 3 & 1 & 2 \\ 2 & & \end{matrix} \right| \left. \begin{matrix} 0 \\ 0 \\ 1 \end{matrix} \right] = \left[ \begin{matrix} 0 & 1 & 0 \\ 0 & 0 & 1 \\ 1 & 0 & 0 \end{matrix} \right| \left. \begin{matrix} 1/2 \\ -1 \\ 1/2 \end{matrix} \right] \\
  \Longrightarrow \frac{1}{ x ( x+2)(x+1) } = \frac{ 1/2}{ x } + \frac{ 1/2}{ x + 2 } + \frac{-1}{  x + 1 } \\
  \Longrightarrow  - \ln{ |x+2| } + \ln{ |x+1| } + -3/2 \ln{x} + -3/2 \ln{ |x+2| } + 3 \ln{ |x+1| }  = \boxed{ -5/2 \ln{ |x+2|} + 4 \ln{ |x+1| } - 3/2 \ln{x} }
\end{gathered}
\]

\exercisehead{17} Use partial fraction method to integrate $\int \frac{1}{ (x^2 -1)^2 }$.  Then \textbf{ build the sum}.  
\[
\begin{gathered}
  \frac{A}{ (x -1)^2 } +  \frac{B}{ (x + 1)^2 } +  \frac{C}{ (x -1) } +  \frac{D}{ (x +1) } \\
\text{ Now } 
  \begin{aligned}
    (x^2-1)(x-1) &= x^3 -x^2 -x + 1 \\
    (x^2 -1 )(x+1) & = x^3 + x^2 - x -1 
  \end{aligned} \\
  \Longrightarrow (x^2 - 1)(x+1) - (x^2 - 1)( x -1) = 2x^2 -2 \\
\begin{aligned}  
  & x^2 + 2x + 1 \\ 
  & x^2 - 2x + 1 \\
  & \Longrightarrow \text{ (summing the above two expressions we obtain) } 2x^2 + 2 
\end{aligned} \\
\boxed{ \frac{-1}{ x-1} + \frac{1/4}{ x+1} + \frac{1/4}{ (x-1)^2} + \frac{1/4}{ (x+1)^2 }  } \\
\boxed{ \int \frac{dx}{ (x^2-1)^2 } = \frac{1}{4} \ln{ \left| \frac{x+1}{x-1} \right| } + -\frac{1}{2} \left( \frac{ x}{ x^2 - 1 } \right) }
\end{gathered}
\]

\exercisehead{18} Use the method of partial fractions, where we find that
\[
\begin{gathered}
  \begin{aligned}
  \int \frac{ (x+1)}{ x^3 -1 } dx & = \int \frac{x+1}{ (x-1)(x^2 + x + 1 ) } dx = \int \frac{ -\frac{2}{3} x - \frac{1}{3} }{ x^2 + x +  1 } + \frac{ \frac{2}{3}}{ x-1 }  = \\
 &  = \boxed{ -\frac{1}{3} \ln{ |x^2 + x + 1 | } + \frac{2}{3} \ln{ |x - 1 | } }
  \end{aligned} \\
  \text{ where we had used the following partial fraction decomposition for the given integrand } \\
  \begin{gathered}
    \frac{ Ax + B}{ x^2 + x + 1 } + \frac{C}{ x-1} = \frac{ x +  1}{ x^3 - 1 } \\
    Ax^2 + Bx - Ax - B + Cx^2 + Cx + C = x + 1 
\end{gathered} \\
  \begin{gathered}
    2Ax + B - A + 2 Cx + C = 1 \quad \text{ (where we used the trick to take the derivative of the above equation) } \\
    \Longrightarrow A = -C \quad B-A - A = 1 \\
    -B + C =  1; \quad A = \frac{2}{-3} , \, C = \frac{2}{3} \, B = -\frac{1}{2}  
\end{gathered}
\end{gathered}
\]

\exercisehead{19} $\int \frac{ x^4 + 1 }{ x (x^2 + 1 )^2 }$ \\
Again, it helps to be clever here.  
\[
  \begin{aligned}
    \int \frac{ x^4 + 1 }{ x (x^2 + 1 )^2 } & = \int \frac{ x^4 + 2x^2 + 1 - 2 x^2 }{ x(x^2 + 1 )^2 } = \int \frac{ (x^2 + 1)^2 }{ x (x^2 + 1)^2 } + \frac{ - 2x}{ (x^2 + 1)^2 } = \\
    & = \ln{x} + (x^2 +1)^{-1} + C
  \end{aligned}
\]

\exercisehead{20} $\int \frac{dx}{ x^3 (x-2)}$ \\
Working out the algebra for the partial fractions method, we obtain
\[
\frac{1}{ x^3 (x-2) } = \left( \frac{ -1/2}{x^3} \right) + \frac{ -1/4}{x^2 } + \frac{ -1/8}{x } + \frac{ 1/8}{ x-2} 
\]
So then
\[
\boxed{ \int \frac{dx}{ x^3 (x-2) } = \frac{1}{4x^2 } + \frac{1}{4x} + \frac{-1}{8} \ln{x} + \frac{1}{8} \ln{ |x-2 | } + C }
\]

\exercisehead{21}
\[
\begin{aligned}
  \int \frac{ 1 - x^3}{ x (x^2 + 1 ) } & = - \int \frac{ x^3 - 1 }{ x(x^2 + 1 ) } = -\int \frac{x^2 }{ x^2 + 1 } + \int \frac{ 1 }{ x(x^2 + 1 ) } =  \\
  & = - \int \left( 1 - \frac{1}{ x^2 + 1 } \right) + \int \frac{1}{x} + \frac{-x}{ x^2 + 1 } = \\
  & = \boxed{ - x + \arctan{x} + \ln{x} - \ln{ |x^2 + 1 | } + C }
\end{aligned}
\]

\exercisehead{22} 
\[
  \begin{aligned}
    \int \frac{dx}{ x^4- 1 } & = \int \left( \frac{1}{ x^2 + 1 } \right) \left( \frac{1}{ x^2 - 1 } \right) = \int \frac{ 1/2}{ x^2 -1 } - \frac{1/2}{ x^2 + 1 } = \\
    & = \int \frac{1}{2} \left( \frac{1/2}{ x -1} - \frac{1/2}{ x + 1 }  \right) - \frac{1/2}{x^2 + 1 } = \\
    & = \boxed{ \frac{1}{4} \ln{(x-1)} - \frac{1}{4} \ln{(x+1)} - \frac{1}{2} \arctan{x} + C }
\end{aligned}
\]

\exercisehead{23} $\boxed{ \int \frac{dx}{ x^4 + 1 } }$ \medskip \\
\textbf{I} had to rely on complex numbers.  

Notice that with complex numbers, \emph{ you can split up polynomial power sums }
\[
\begin{gathered}
  \begin{aligned}
  x^4 + 1  = (x^2 + i)(x^2 - i ) & = (x + i e( i \frac{\pi}{4} ) )(x - i e( i \frac{\pi}{4} ) )(x +  e( i \frac{\pi}{4} ) )(x -  e( i \frac{\pi}{4} ) ) = \\ 
  & = (x+e(i\frac{3\pi}{4}))(x-e(i\frac{3\pi}{4}))(x+e(i\frac{\pi}{4}))(x-e(i\frac{\pi}{4})) 
\end{aligned} \\
\frac{A}{ (x+e(i\frac{3\pi}{4})) } + \frac{B}{ (x-e(i\frac{3\pi}{4})) } + \frac{C}{ (x+e(i\frac{\pi}{4})) } + \frac{D}{ (x-e(i\frac{\pi}{4}) ) 
} = \frac{1}{ x^4 + 1 } \\
A(x^2 - i )(x - e(i \frac{3\pi}{4}) ) + B(x^2 - i)(x+e(i\frac{3\pi}{4}) ) + C(x^2 + i )(x - e(i\frac{\pi}{4}) ) + D (x^2 + i )( x + e(i\frac{\pi}{4} ) ) = 1  \\
\xrightarrow{ \text{ do the algebra } } \quad \, \begin{aligned}
  & x^3: \, A + B + C + D = 0 \\
  & x^2: \, -A e(-\frac{3\pi}{4}) + Be(i\frac{3\pi}{4})  - C e(i \frac{\pi}{4}) + D e(i\frac{\pi}{4} ) = 0 \\
  & x^1: \, -i A - iB + iC + i D = 0  \\
  & x^0: \, -e(i\frac{\pi}{4})A + B e(i \frac{\pi}{4}) + C(-e(i \frac{3\pi}{4}) ) + D(e(i \frac{3\pi}{4})) = 1 
\end{aligned}  \\ 
\Longrightarrow \left[ \begin{matrix} 
     1 & 1 & 1 & 1 \\
     -e(i \frac{3\pi}{4}) & e(i\frac{3\pi}{4} ) & -e(i\frac{\pi}{4}) & e(i \frac{\pi}{4}) \\
     -i & -i & i & i \\
     -e(i \frac{\pi}{4}) & e(i\frac{\pi}{4}) & -e(i\frac{3\pi}{4}) & e(i\frac{3\pi}{4}) \\
     \end{matrix} \right]\left[ \begin{matrix} A \\ B \\ C \\ D \end{matrix} \right] = \left[ \begin{matrix} 0 \\ 0 \\ 0 \\ 1 \end{matrix} \right]
\end{gathered}
\]
\quad \\
 To do the complex algebra for the desired Gaussian elimination procedure, I treated the complex numbers as vectors and added them and rotated them when multiplied.  

\[
\begin{gathered}
 \begin{aligned} 
  \left[ \begin{matrix} 1 & 1 & 1 & 1 \\ 
      \searrow & \nwarrow & \swarrow & \nearrow \\
      \downarrow & \downarrow & \uparrow & \uparrow \\
      \swarrow & \nearrow & \searrow & \nwarrow 
\end{matrix} \right| \left. \begin{matrix} 0 \\ 0 \\ 0 \\ 1 \end{matrix} \right]  = \left[ \begin{matrix} 
       1 & 1 & 0 & 0 \\
       & 2 \nwarrow & & 2 \nearrow \\ 
       0 & 0 & 1 & 1  \\
       0 & 0 & 2 \searrow & 2 \nwarrow 
\end{matrix} \right| \left. \begin{matrix} 0 \\ 0 \\ 0 \\ 1 \end{matrix} \right] = 
  \left[ 
    \begin{matrix} 
      1 & 1 & 0 & 0 \\
      & 2 \nwarrow & & 2 \nearrow \\ 
      0 & 0 & 1 & 1 \\
      & & 0 & 4 \nwarrow \end{matrix} \right| \left. \begin{matrix} \\ \\ 0 \\ 1 \end{matrix} \right] 
  & = \left[ \begin{matrix} 1 & & & \\ & \nwarrow & &  \\ 0 & 0 & 1 & 0 \\ & & 0 & 1 \end{matrix} \right| \left. \begin{matrix} \frac{1}{4} \nwarrow \\ \frac{1}{4} \uparrow \\ \frac{1}{4} \nearrow \\ \frac{1}{4} \swarrow \end{matrix} \right] = \\
  & = \left[ \begin{matrix} 1 & &  & \\ & 1 & & \\ & & 1 & \\ & & & 1 \end{matrix} \right| \left. \begin{matrix} \frac{1}{4} \nwarrow \\ \frac{1}{4} \searrow \\ \frac{1}{4} \nearrow \\ \frac{1}{4} \swarrow \end{matrix} \right]
 \end{aligned} \\
\Longrightarrow \begin{aligned}
  & A = \frac{1}{4} e(\frac{i3\pi}{4}) \\
  & B = -\frac{1}{4} e(\frac{ i 3 \pi}{4} ) \\
  & C = \frac{1}{4} e(i\frac{\pi}{4}) \\
  & D = \frac{-1}{4} e(i\frac{  \pi}{4} )
\end{aligned}
\Longrightarrow 
  \int  \frac{ 1/4 e(\frac{i3\pi}{4}) }{ x + e(i\frac{3\pi}{4}) } + \frac{ -1/4 e(i \frac{3\pi}{4} ) }{ x - e(i\frac{3\pi}{4} ) } + \frac{ 1/4 e(i\frac{\pi}{4}) }{ x + e(i\frac{\pi}{4}) } + \frac{-1/4 e(i\frac{\pi}{4}) }{ x - e(i\frac{\pi}{4}) }  \\ 
\begin{gathered}
\left( \frac{1}{4} \right) \left( e(i\frac{ 3 \pi }{4} ) \ln{ (x + e(i\frac{3\pi}{4}) ) } - e(i\frac{ 3 \pi }{4} ) \ln{ (x -  e(i\frac{3\pi}{4}) ) } + e(i\frac{  \pi }{4} ) \ln{ (x + e(i\frac{\pi}{4}) ) } - e(i\frac{  \pi }{4} ) \ln{ (x - e(i\frac{\pi}{4}) ) } \right)
\end{gathered}
\end{gathered}
\]
After doing some complex algebra,
\[
\Longrightarrow \frac{1}{4 \sqrt{2}} \left( \ln{ \left| \frac{ x^2 + \sqrt{2}x + 1 }{ x^2 - \sqrt{2}x + 1 } \right| } - 2 \arctan{ \left( \frac{1}{ \sqrt{2} x - 1 } \right) } -  2 \arctan{ \left( \frac{1}{ \sqrt{2}x + 1 } \right) } \right)
\]
The computation could be done to do the derivative on this, so to check our answer and reobtain the integrand.  

\textbf{ Is there a way to solve this without complex numbers? }

\exercisehead{24} $\int \frac{ x^2 dx }{ (x^2 + 2x + 2 )^2 }$
\[
\begin{gathered}
  \int \frac{x^2 dx }{ (x^2 + 2x + 2 )^2 }  = \int \frac{ x^2 + 2x +2 - 2x -2 }{ (x^2 + 2x +2 )^2 } = \left( \int \frac{1}{ x^2 + 2x + 2 }\right) + (x^2 + 2x + 2 )^{-1} \\
  \int \frac{1}{ x^2 + 2x + 2 } = \int \frac{1}{ (x+1)^2 + 1 } = \arctan{ (x+1) } \\
  \boxed{ \int \frac{ x^2 dx }{ (x^2 +2x + 2 )^2 } = \arctan{ (x+1)} + \frac{1}{ x^2 + 2x + 2 } + C }
\end{gathered}
\]

\exercisehead{25} $ \int \frac{ 4x^5 - 1 }{ (x^5 + x + 1)^2 } dx$ 
\[
\begin{gathered}
\begin{aligned}
  & (-(x^5 +x + 1)^{-1} )' = (x^5 + x + 1)^{-2}(5x^4 + 1 ) \quad \, \text{ (doesn't work) } \\
  & (-x(x^5 + x + 1 )^{-1} )' = (x^5 + x + 1)^{-2}(5x^5 + x) + - (x^5  + x + 1)^{-1}  = (x^5 + x + 1)^{-2}(5x^5 + x - x^5 -x - 1 ) 
\end{aligned} \\
\Longrightarrow \int \frac{ 4x^5 - 1}{ (x^5 + x + 1)^2 } dx = \boxed{ -x(x^5 + x + 1)^{-2} }
\end{gathered} 
\]

\exercisehead{26} $ \int \frac{ dx}{ 2 \sin{x} - \cos{x} + 5 }$  \quad \, \emph{ (good example of the use of half angle substitution ) }
\[
\begin{gathered}
  \begin{aligned}
    \int \frac{dx}{ 2 \sin{x} + -\cos{x} + 5 } & = \int \frac{dx}{ 4 SC + -C^2 + S^2 + 5 } \left( \frac{ \frac{1}{C^2} }{ \frac{1}{C^2} } \right) = \int \frac{ \sec^2{ \frac{x}{2} } dx }{ 4 T - 1 + T^2 + 5 ( 1+T^2 ) } = \\
    & = \int \frac{  \sec^2{ \frac{x}{2} } dx }{ 6T^2 + 4T + 4 } = \int \frac{ \sec^2{ \frac{x}{2} } dx }{ 6 (T+ \frac{1}{3} )^2 + \frac{10}{3} } = \int \frac{ 2 du }{ 6 (u+ \frac{1}{3} )^2 + \frac{10}{3} } \quad \, \text{ (where $ \boxed{ \begin{aligned} 
	  u & = \tan{ \frac{x}{2} } \\
	  du & = \frac{ \sec^2{\frac{x}{2} }}{2} dx 
  \end{aligned} }$ ) } \\
    & = \frac{3}{5} \int \frac{ du }{ \frac{ 9(u + \frac{1}{3} )^2 }{5 } + 1 } = \boxed{ \frac{1}{\sqrt{5}} \arctan{ \left( \frac{ 3 ( \tan{ \frac{x}{2} } + \frac{1}{3} ) }{ \sqrt{5} } \right) } }
  \end{aligned}
\end{gathered}
\]

\exercisehead{27} $\int \frac{dx}{ 1 + a \cos{x} }$  $( 0 < a < 1)$
Again, using the half-angle substitution, $\boxed{ \begin{aligned} u & = \tan{\frac{x}{2}} \\
    du & = \frac{ \sec^2{ \frac{x}{2} } }{2 } dx 
\end{aligned} }$, 
\[
\begin{gathered}
  \begin{aligned}
    \frac{1}{a} \int \frac{dx}{ \frac{1}{a} + \cos{x} } & = \frac{1}{a} \int \frac{dx}{ \frac{1}{a} + C^2 - S^2 } = \frac{1}{a} \int \frac{ \sec^2{ \frac{x}{2} } dx }{ \frac{1}{a} \sec^2{\frac{x}{2} } + 1 - T^2 } = \\
    & = \frac{1}{a} \int \frac{ \sec^2{\frac{x}{2}} dx }{ \frac{1}{a} + 1 + T^2(\frac{1}{a} - 1 ) } = \frac{1}{a} \int \frac{ 2 du }{ \frac{1}{a} + 1 + u^2 \left( \frac{1}{a} - 1 \right) } = \frac{2}{ 1 + a} \int \frac{ du}{ 1 + \left( u \sqrt{ \frac{ 1 - a}{ 1 + a} } \right)^2 } = 
  \end{aligned} \\
\frac{2}{1 +a} \frac{ \arctan{ \sqrt{ \frac{1-a}{1+a} } u } }{ \sqrt{ \frac{1-a}{1+a} } }  = \Longrightarrow \boxed{ \frac{2}{ \sqrt{ 1 - a^2 }} \arctan{ \left( \sqrt{ \frac{ 1- a}{1+a} } \tan{ \frac{x}{2} } \right) } }
\end{gathered}
\]

\exercisehead{28} $\int \frac{dx}{ 1 + a \cos{x} }$ Half-angle substitution.  
\[
\begin{gathered}
  \begin{aligned}
    \int \frac{dx}{ 1 + a \cos{x}} & = \int \frac{dx}{ 1 + a (C^2 - S^2 ) } = \int \frac{ \sec^2{ \frac{x}{2} } dx }{ \sec^2{ \frac{x}{2} } + a (1- T^2) } = \\
    \begin{aligned}
      u & = \tan{\theta/2} = T \\
      du & = \sec^2{ \theta/2} \left( \frac{1}{2} \right) d\theta
\end{aligned} \Longrightarrow & = \int \frac{2 du }{ 1 + T^2 + a (1-T^2) } = 2 \int \frac{ du}{ (1-a)T^2 + (1+a) } = \frac{2}{1-a} \int \frac{ du }{ u^2 - \frac{a+1}{a-1} } = \\
    & = \frac{2}{1-a} \int \left( \frac{1}{ u - \sqrt{ \frac{ a+1}{ a-1} } } - \frac{1}{ u + \sqrt{ \frac{a+1}{a-1} } } \right) \frac{1}{ 2 \sqrt{ \frac{a+1}{a-1} } } = \\ 
    & = \sqrt{ \frac{a-1}{a+1} } \left( \frac{1}{ 1-a} \right) \left( \ln{ (u - \sqrt{ \frac{a+1}{a-1} } ) } - \ln{ (u + \sqrt{ \frac{a+1}{a-1} } ) } \right) = 
  \end{aligned} \\
  = \boxed{ \frac{-1}{ \sqrt{ a^2 - 1 } } \left( \ln{ \left( \frac{ \tan{ \frac{x}{2} } - \sqrt{ \frac{a+1}{a-1} } }{ \tan{ \frac{x}{2}} + \sqrt{ \frac{a+1}{a-1} } }  \right) } \right) }
\end{gathered}
\]


\exercisehead{29} $\int \frac{\sin^2{x}}{ 1 + \sin^2{x} }dx$
\[
\begin{gathered}
  \int \frac{ s^2 }{ 1 + s^2 } dx = \int \frac{ s^2 + 1 - 1 }{ 1 + s^2 } dx = x + - \int \frac{ dx}{ 1 + \sin^2{x} } \\
  \begin{aligned}
    \int \frac{dx}{ 1 + \sin^2{x} } & = \int \frac{dx}{ 1 + \left( \frac{ 1 - \cos{2x}}{ 2 } \right) } = \int \frac{2 dx}{ 3 - \cos{2x} } = \frac{2}{3} \int \frac{ dx}{ 1 - \frac{ \cos{2x}}{3} } = \frac{2}{3} \int \frac{dx}{ 1 - \left( \frac{c^2 - s^2 }{3 } \right) } = \frac{2}{3} \int \frac{ \sec^2{x} dx }{ \sec^2{x} - \left( \frac{1 - T^2}{3} \right) } = \\
    \begin{aligned} 
      & \\
      u & = \tan{x} \\
      du & = \sec^2{x} dx 
    \end{aligned} \Longrightarrow & = \frac{2}{3} \int \frac{ du }{ 1 + u^2 - \left( \frac{1-u^2 }{3 } \right) } = \frac{2}{3} \int \frac{du}{ \frac{2}{3} + \frac{4}{3} u^2 } = \int \frac{ du}{ 1 + (\sqrt{2} u)^2 } = \\
    & = \frac{1}{\sqrt{2}} \arctan{ \sqrt{2} \tan{x} } 
  \end{aligned} \\
\Longrightarrow \boxed{ \int \frac{ \sin^2{x}}{ 1 + \sin^2{x}} dx = x - \frac{1}{\sqrt{2}} \arctan{ (\sqrt{2} \tan{x} ) } }
\end{gathered}
\]
It seems like for here, when \emph{dealing with squares of trig. functions}, ``step up'' to double angle.  

\exercisehead{30} $\int \frac{dx}{ a^2 \sin^2{x} + b^2 \cos^2{x} } \quad (ab \neq 0)$ \emph{ Take note}, we \emph{need not} change the angle to half-angle or double-angle.  
\[
\begin{gathered}
  \begin{aligned}
    \frac{1}{ a^2 s^2 + b^2 c^2 } & = \frac{1}{ a^2 ( 1 -c^2) + b^2 c^2 } = \frac{1}{ a^2 + (b^2 -a^2)c^2 } = \frac{1}{ a^2 (1 + (kc)^2 ) } = \frac{ \sec^2 }{ a^2 ( \sec^2 + k^2 ) } = \frac{ \sec^2 }{ a^2 ( 1 + T^2 + k^2 ) } = \\
    \begin{aligned}
      & \\
      u & = \tan{x} \\
      du & = \sec^2{x}dx
    \end{aligned} \Longrightarrow & = \frac{du}{ a^2 ( 1 + u^2 + k^2 ) } = \frac{ 1/a^2 du }{ (1+k^2) ( 1 + \frac{u^2}{1+k^2} ) } = \frac{ \sqrt{ 1 + k^2}}{ a^2 (1+k^2 ) } \arctan{ \left( \frac{u}{ \sqrt{ 1 + k^2 } } \right) }
  \end{aligned} \\
   \Longrightarrow \int \frac{dx}{ a^2 \sin^2{x} + b^2 \cos^2{x} } = \boxed{ \frac{1}{ab} \arctan{ \left( \frac{a \tan{x}}{ b} \right) } }
\end{gathered}
\]

\exercisehead{31} $\int \frac{dx}{ (a\sin{x} + b \cos{x})^2 } \quad (a\neq 0)$ \medskip \\
\emph{ Note} it's a good idea to \emph{simplify}, cleverly, your constants as much as you can.  
\[
\int \frac{dx}{ (a\sin{x} +b \cos{x})^2 } = \frac{1}{a^2} \int \frac{dx}{ (\sin{x} + k \cos{x})^2 }
\]
Thus, only one constant, $k$, is only worried about.  
\[
\begin{gathered}
  \begin{aligned}
    \frac{1}{ (s+kc)^2 } & = \frac{1}{ s^2 + 2ksc + k^2 c^2 } = \frac{ 1 / c^2 }{ t^2 + 2kt + k^2 } = \frac{ \sec^2 }{ (t+k)^2 } = \\
    \begin{aligned}
      & \\
      u & = \tan{x} \\
      du & = \sec^2{x}dx 
    \end{aligned} \Longrightarrow & = \frac{du}{ (u+k)^2 } 
\end{aligned} \\ 
    \Longrightarrow \boxed{ \frac{1}{a^2} \int \frac{1}{ (s+\frac{b}{a}c)^2 } = \frac{-1}{ (a^2 \tan{x} + ab) } }
\end{gathered}
\]
Again, \emph{note}, \textbf{ we need not } always step up or step down a half angle in the substitution.  

\exercisehead{32} Note that we have a rational expression consisting of single powers of $\sin$ and $\cos$.  Then use the $\tan{ \frac{\theta}{2} }$ substitution.  
\[
\begin{gathered}
\int \frac{ \sin{x}}{ 1 + \cos{x} + \sin{x} }  = \int \frac{ 2 CS }{ 1 + 2 CS + C^2- S^2 } = \int \frac{ CS}{ C(S+C) } = \int \frac{ T}{(T+1 ) } \\
 \text{ where } \begin{aligned} C & = \cos{ x/2} \\
   S & = \sin{ x/2 } 
\end{aligned}  \quad \begin{aligned} u & = \tan{ \theta/2} \\ du & = \frac{ \sec^2{\theta/2} d\theta}{ 2} \end{aligned} \\
 \int \frac{ u }{ (u+1)} \left( \frac{ 2 du }{ u^2 + 1 } \right) = \int \frac{ 2 u du }{ (u^2 + 1 )( u+1) }  \\
 \begin{aligned}
   & \frac{ A}{ u+1 } + \frac{ Bu +C }{ u^2 + 1 } \\
   & Au^2 + A + Bu^2 + C u + Bu + C = u \\
   & A = -B \quad C+B = 1 \quad A+C = 0 \Longrightarrow C = \frac{1}{2}; \, B = \frac{1}{2} ; \, A = - \frac{1}{2} 
\end{aligned} \\
 \begin{aligned}
2 \int \left( \frac{ -1/2}{ u+1 } + \frac{ \frac{1}{2} (u+1 ) }{ u^2 +1 } \right) du & = \int \frac{-1}{ u+1 } + \frac{ u +1 }{ u^2 + 1 } \\
& = -\ln{ | u +1 | } + \frac{1}{2} \ln{ |u^2 +  1| } + \arctan{u } = \\
& = -\ln{ |\tan{ x/2} + 1 | } + \frac{1}{2} \ln{ |\sec^2{x/2 } | } + \frac{x}{2} \\
\end{aligned} \\
\Longrightarrow - \left( \ln{ |2| } \right) + \frac{1}{2} \ln{ |2| }+ \frac{\pi}{4} = \boxed{ -\frac{1}{2} \ln{ |2| } + \frac{\pi}{4} }
\end{gathered}
\]

\exercisehead{33}  $\int \sqrt{ 3 - x^2} dx$
\[
\begin{gathered}
    \int (x)' \sqrt{ 3 -x^2} dx  = x \sqrt{ 3 -x^2} - \int \frac{ x (-x)}{ \sqrt{ 3 - x^2 }} = x \sqrt{ 3 - x^2 } - \int \frac{ -x^2 + 3 - 3 }{ \sqrt{ 3 -x^2 } } = x \sqrt{ 3- x^2 }- \int \sqrt{ 3 - x^2 } + 3 \int \frac{1}{ \sqrt{3- x^2 } } \\
    \Longrightarrow 2 \int \sqrt{ 3 - x^2 } = x \sqrt{ 3 - x^2 } + \sqrt{3} \int \frac{1}{ \sqrt{ 1 - \left( \frac{x}{ \sqrt{3}} \right)^2 } } \\
    \Longrightarrow \boxed{ \int \sqrt{ 3 -x^2 } = \frac{x}{2} \sqrt{ 3 - x^2 } + \frac{3}{2} \arcsin{ \frac{x}{\sqrt{3}} } }
\end{gathered}
\]

\exercisehead{34} $\int \frac{1}{ \sqrt{ 3-x^2 }} dx = -(3-x^2)^{1/2} +C$.  
\[
\begin{aligned}
  \left( \arccos{ \frac{ x}{ \sqrt{3}} } \right)' & = \frac{1}{ \sqrt{3}} \frac{ -1}{ \sqrt{1-\frac{x^2}{3} } }  = - \frac{1}{ \sqrt{3-x^2 }}  \\
  \left( x \sqrt{3-x^2 } \right)' & = \sqrt{ 3- x^2 } + \frac{ - x^2}{ \sqrt{3-x^2 }} \\
  & \boxed{ \frac{ x \sqrt{3-x^2 } }{ 2 } + - \frac{3}{2} \arccos{ \frac{x}{ \sqrt{3}} } }
\end{aligned}
\]

\exercisehead{35} $ \int \frac{ \sqrt{ 3 - x^2 }}{ x} dx = \int \sqrt{ \frac{3}{x^2 } -1 } dx $.  
\[
\begin{gathered}
\begin{aligned}
  \frac{ \sqrt{3}}{ x } & = \sec{\theta} \\
  \sqrt{3} \cos{\theta} & = x \\
  dx = - \sin{\theta} \sqrt{3} 
\end{aligned} \\
\begin{aligned}
  \int \sqrt{ \sec^2{\theta} -1 } (-\sin{\theta}) \sqrt{3} & = \int \tan{\theta} \sin{\theta} (-\sqrt{3}) = -\sqrt{3} \int (\sec{\theta} - \cos{\theta} ) = \\
  & = -\sqrt{3} \ln{ |\sec{\theta} + \tan{\theta} |} + \sqrt{3} \sin{\theta} = \\
  & = \boxed{ -\sqrt{3} \ln{ \left| \frac{\sqrt{3}}{x} + \sqrt{ \frac{3}{x^2} -1 } \right| } + \sqrt{3} \sqrt{ 1 - \frac{x^2}{3} } }
\end{aligned}
\end{gathered}
\]

\exercisehead{36} $\int \sqrt{ 1 + \frac{1}{x} } dx $
\[
\begin{gathered}
  \left( x \sqrt{ 1 + \frac{1}{x} }\right)' = \sqrt{ 1 + \frac{1}{x}} + \frac{ x}{ 2 \sqrt{ 1 + \frac{1}{x} } } \left( \frac{-1}{x} \right) = \sqrt{ 1 + \frac{1}{x}} + \frac{ -1/2}{ \sqrt{ x^2 + x } } \\
    \left( \ln{ \left( x + \frac{1}{2} + \sqrt{ x^2 + x } \right) } \right)' = \frac{ 1}{ x+ \frac{1}{2} + \sqrt{ x^2 + x }} \left( 1 + \frac{ x + \frac{1}{2}}{ \sqrt{ x^2 + x }} \right) = \frac{1}{ \sqrt{ x^2 +x }} \\
    \Longrightarrow \int \left( 1 + \frac{1}{x} \right) dx = x \sqrt{ 1 + \frac{1}{x} } + \frac{1}{2} \ln{ \left( x + \frac{1}{2} + \sqrt{ x^2 + x } \right) } 
\end{gathered}
\]

\exercisehead{37}
\[
\begin{gathered}
  (x \sqrt{ x^2 +6 })' = \sqrt{ x^2 + 5} + \frac{ x^2 }{ \sqrt{ x^2 + 5 }} \\
  (\ln{ (x+ \sqrt{ x^2 + b }) } )' = \left(  \frac{ 1}{ x + \sqrt{ x^2 + b }}\right)\left( 1 + \frac{ x}{ \sqrt{ x^2 +b } } \right) = \frac{1}{ \sqrt{ x^2 + b} } \\
  \int \sqrt{ x^2 + 5 } = \frac{1}{2} \left( x \sqrt{x^2 +5} + 5 \ln{ (x+ \sqrt{ x^2 + 5} ) } \right)
\end{gathered}
\]

\exercisehead{38} 
\[
\begin{gathered}
  \left( \ln{ (x + \frac{1}{2} + \sqrt{ x^2 + x + 1 } )} \right)' = \frac{ 1 }{ x + \frac{1}{2} + \sqrt{ x^2 + x + 1 } } \left( 1 + \frac{ x + \frac{1}{2}}{ \sqrt{ x^2 + x + 1 } } \right) = \frac{ 1}{ \sqrt{ x^2 + x + 1 } } \\
    \int \frac{x}{ \sqrt{ x^2 + x + 1 }} = \int \frac{ x + \frac{1}{2}  - \frac{1}{2} }{ \sqrt{ x^2 + x + 1 }} = (x^2 + x + 1 )^{1/2} -\frac{1}{2} \ln{ (x + \frac{1}{2} + \sqrt{ x^2 + x + 1 } )}
\end{gathered}
\]
The trick is to note how I formed a ``conjugate-able'' sum from $x^2+x +1$'s derivative.  

\exercisehead{39}   
\[
\begin{gathered}
  \int \frac{dx}{ \sqrt{ x^2 + x} } = \int \frac{ dx}{ \sqrt{ \left( x + \frac{1}{2} \right)^2 - \frac{1}{4} } } = \int \frac{ 2 dx }{ \sqrt{ \left( 2 \left( x + \frac{1}{2} \right) \right)^2 - 1 } } \\
  \begin{gathered} 
    \left( \ln{ \left( \sqrt{ (2 (x+ 1/2))^2  - 1 } + 2 (x+ 1/2) \right) } \right)'  = \\ 
    = \frac{ 1 }{ \sqrt{  (2 (x+ 1/2))^2  - 1 } + 2 (x+ 1/2) } \left( 2 + \frac{ 2 (x+ 1/2) 2 }{ \sqrt{ (2 (x+1/2))^2 -1 } } \right) = \\
     = \frac{ 2}{ \sqrt{ (2 (x+1/2) )^2 - 1 } }
\end{gathered} \\
\boxed{  \int \frac{dx}{ \sqrt{ \left( x + \frac{1}{2} \right)^2 - \frac{1}{4} } } = \ln{ \left( 2 \left( x + \frac{1}{2} \right) + \sqrt{ (2 (x+ 1/2) )^2 -1 } \right) } + C }
\end{gathered}
\]

\exercisehead{40}

%-----------------------------------%-----------------------------------%-----------------------------------
\subsection*{ 6.26 Miscellaneous review exercises }
%-----------------------------------%-----------------------------------%-----------------------------------

\exercisehead{1}
\[
\begin{gathered}
  f(x) = \int_1^x \frac{ \log{t}}{ t + 1 } \\
  \begin{aligned}
  f\left( \frac{1}{x} \right) & = \int_1^{\frac{1}{x} } \frac{ \log{t} }{ t+1} dt = \int_1^x \frac{ -\ln{ (u)}}{ \frac{1}{u} + 1 } \left( \frac{ -1 }{ u^2 } du \right) = \\
  & = \int_1^x \frac{ \ln{(u)}}{ u+u^2 } du  \quad \begin{aligned} u & = \frac{1}{t} \\ du & = -\frac{1}{t^2} dt \end{aligned} 
  \end{aligned} \\
  f(x) + f\left( \frac{1}{x} \right) = \int_1^x \frac{ t\ln{t} + \ln{t} }{ t (t+1)} dt = \int_1^x \frac{ \ln{t}}{ t } dt = \left. \frac{ (\ln{t } )^2 }{ 2 } \right|_1^x = \boxed{ \frac{ (\ln{x})^2 }{ 2 } } \\
  \boxed{ f(2) + f\left( \frac{1}{2} \right) = \frac{1}{2} (\ln{2})^2 }
\end{gathered}
\]

\exercisehead{2} Take the derivative of both sides, using the (first) fundamental theorem of calculus.  
\[
  2ff' = f(x) \frac{ \sin{x}}{ 2 +\cos{x} }; \Longrightarrow 2f' = \frac{\sin{x}}{ 2 +\cos{x}} \\
\]
At this point, it could be very easy to evaluate the integral by guessing at the solution.  
\[
(-\ln{ 2 + \cos{x}} ) ' = \frac{ \sin{x} }{ 2 + \cos{x}} \Longrightarrow \boxed{ f = -\frac{ \ln{ |2+\cos{x} |}}{ 2 } + C }
\]
Otherwise, remember that for rational expressions involving single powers of $\sin$ and $\cos$, we can make a $u=\tan{\theta/2}$ substitution.  
\[
\begin{gathered}
  \begin{aligned}
    u & = \tan{ \frac{x}{2} } \\
    2 du & = \sec^2{ \frac{x}{2} } dx 
  \end{aligned}  \quad C = \cos{x/2}, \, S = \sin{x/2} \\
\begin{aligned}
  \int \frac{\sin{x}}{ 2 + \cos{x}} dx & = \int \frac{ 2 SC dx }{ 2 + C^2 - S^2 } = \int \frac{ 2T }{ 2 \sec^2{ x/2} + 1 - T^2 } \left( \frac{ 2 du }{ \sec^2{x/2} } \right) = \\
  & = 4 \int \frac{ u du }{ (1+u^2)(3+u^2) }= 2 \int \frac{T}{ T^2+1} - \frac{T}{ T^2 + 3 } \\
  & = 2 \left( \frac{1}{2} \ln{ T^2 + 1 } - \frac{1}{2} \ln{ T^2 +3 } \right) = \left( \ln{ \left( \frac{ T^2 + 1 }{ T^2 + 3 } \right) } \right) = \ln{ \left( \frac{ 2 }{ 4 + 2 \cos{x}} \right)} 
\end{aligned} \\
\text{ where } \tan^2{\frac{x}{2} } = \frac{ \sin^2{ \frac{x}{2} }}{ \cos^2{\frac{x}{2}} } = \frac{1-\cos{x}}{ 1+\cos{x}} \\
\end{gathered}
\]


\exercisehead{3} 
\[
\int \frac{ e^x}{x } dx = e^x - \int \frac{ e^x (x-1)}{ x^2 } x dx = e^x - \int \frac{ e^x (x-1)}{x } dx \dots
\]
No way.  

\exercisehead{4} $ \int_0^{\pi/2} \ln{ (e^{\cos{x}} ) } dx = \left. -\cos{x} \right|_0^{3/2} = \boxed{1}$.  

\exercisehead{5} 
\begin{enumerate}
\item 
\[
\begin{gathered}
  f = \sqrt{ 4x + 2 }{ x (x+1)(x+2) } \\
  \begin{aligned}
    \ln{f} & = \frac{1}{2} \ln{ \left( \frac{ 4(x+2)}{ x (x+1)(x+2) } \right) } = \frac{1}{2} \left( \ln{ (4x+2)} - \ln{x} - \ln{(x+1)} - \ln{ (x+2) } \right) \\
    \frac{ f' }{f} & = \frac{1}{2} \left( \left( \frac{4}{4x+2} \right) - \frac{1}{x} - \frac{1}{x+1} - \frac{1}{x+2} \right) \\
\Longrightarrow f' & = \frac{1}{2} \sqrt{\frac{  4x + 2 }{ x (x+1)(x+2) } } \left( \left( \frac{4}{4x+2} \right) - \frac{1}{x} - \frac{1}{x+1} - \frac{1}{x+2} \right)
  \end{aligned}
f'(1) = -\frac{7}{12}
\end{gathered}
\]
\item
  \[
  \begin{aligned}  
    \int_1^4 \pi \frac{ 4x +2}{ x(x+2)(x+1) }dx & = 2\pi \int_1^4 \frac{ (2x+1)}{ x (x+2)(x+1) } dx = \\ 
    & = 2\pi \left. \left( \frac{1}{2} \ln{x} + -\frac{3}{2} \ln{ |x+2|} + \ln{ |x+1|} \right) \right|_1^4 = \boxed{ \pi ln{ \frac{25}{8 } } }
    \end{aligned}
\]
since we can find the antiderivative through partial fractions:
\[
\begin{gathered}
  \frac{A}{x} + \frac{B}{ x+2} + \frac{C}{ x+1} = \frac{2x+1}{ x(x+2)(x+1) } \\
  A(x^2 + 3x+ 2 ) +B (x^2 + x) + C(x^2+2x) = 2x + 1 \\
  \left[ \begin{matrix} 1 & 1 & 1 \\ 3 & 1 & 2 \\ 2 & 0 & 0 \end{matrix} \right] \left[ \begin{matrix} A \\ B \\ C \end{matrix} \right] = \left[ \begin{matrix} 0 \\ 2 \\ 1 \end{matrix} \right] \Longrightarrow \left[ \begin{matrix} 0 &  1 & 0 \\ 0 & 0 & 1 \\ 1 & 0 & 0 \end{matrix} \right| \left. \begin{matrix} -3/2 \\ 1 \\ 1/2 \end{matrix} \right]
\end{gathered}
\]
\end{enumerate}

\exercisehead{6} 
\begin{enumerate}
\item \[
\begin{gathered}
\log{x} = \int_1^x \frac{1}{t} dt  \quad F(x) = \int_1^x \frac{e^t}{t} dt ; \, \begin{aligned} \text{ if } x & > 0 \\ e^t > 1 \text{ for } & t > 0 \end{aligned} \\
\text{ If } 0 < x < 1 \\
\log{x} = \int_1^x \frac{1}{t} dt = \int_x^1 \frac{-1}{t} dt > - \int_x^1 \frac{e^t}{t} = F(x) \\
\boxed{ \log{x} \leq F(x) \text{ for } x \geq 1  }
\end{gathered}
\]
\item \[
\begin{aligned}
  F(x+a)-F(1+a) &= \int_1^{x+a} \frac{e^t dt }{ t } - \int_1^{a+1} \frac{e^t dt }{ t } = \int_{1-a}^x \frac{ e^{t+a} dt }{ t+a} - \int_{1-a}^1 \frac{ e^{t+a} dt }{ t+a} \\
  & = e^a \int_1^x \frac{e^t}{ t+a} dt  
\end{aligned}
\]
\item \[
\begin{gathered}
\begin{aligned}
  \int_1^x \frac{ e^{at}}{t} dt & = \int_{a}^{ax} \frac{ e^t}{t} dt = \int_1^{ax} \frac{e^t}{t} + \int_a^1 \frac{e^t}{t} = \\
  & = \boxed{ F(ax) - F(a) }
\end{aligned} \\
\int_1^x \frac{e^t}{t^2} = -\frac{1}{t} e^t - \int -\frac{1}{t} e^t = \boxed{ - \left( \frac{e^x}{x} - e \right) + F(x) } \\
  \begin{aligned}
  \int_1^x e^{1/t} dt & = \int_1^{1/x} \frac{ -e^u }{ u^2 } du = - \left( \frac{-e^u }{u } - \int -\frac{e^u }{u } \right) = \\ 
&= \boxed{ x e^{1/x} - e - F(1/x)} \text{ where  we used the substitution } \begin{aligned} u & = \frac{1}{t} \\ du & = -\frac{1}{t^2} dt \end{aligned} 
\end{aligned}
\end{gathered}
\]
\end{enumerate}

\exercisehead{7}
\begin{enumerate}
\item \[
e^x = F(x) -F(0); \, F(x) = e^x + F(0) \Longrightarrow F(0) = 1 + F(0) 
\]
$0 \neq 1$.  False.  
\item \[
  \begin{gathered}
    \frac{d}{dx} \int_0^{x^2} f(t) dt = f(x^2) (2x) = -(2x)\ln{2}e^{x^2 \ln{2}} \quad f(x) = -\ln{2} e^{x \ln{2}} \\
    \int_0^{x^2} -\ln{2} e^{\ln{2}} dt = - \left. e^{t\ln{2}} \right|_0^{x^2} = -e^{x^2 \ln{2} } +1
  \end{gathered}
\]
\item \[
\begin{gathered}
  f(x) = 2f(x) f'(x); \Longrightarrow f(x) = \frac{x}{2} + C \\
  \int_0^x \left( \frac{1}{2} t + c \right) dt = \left. \left( \frac{t^2}{4} + ct \right) \right|_0^x = \frac{x^2}{4} + cx \\
  f^2(x) -1 = \frac{ x^2}{4} + Cx + C^2 -1 \\
  \Longrightarrow C=\pm 1, f(x) = \frac{x}{2} + \pm 1 
\end{gathered}
\]
\end{enumerate}

\exercisehead{8}
\begin{enumerate}
\item \[
\begin{gathered}
  \frac{ f(x+h) -f(x) }{ h} = \frac{f(x)f(h) -f(x) }{ h } = \frac{ f(x) ( h g(h))}{ h } = f(x) g(h) \\
  g(h) \to 1 \text{ as } h \to 0 \text{ so } \Longrightarrow f'(x) =f(x)  
\end{gathered}
\]
\item Since for $f(x) = e^x$, we defined $e^x$ such that $f'=f$, if 
\[
\begin{gathered}
(e^x + g)' = e^x + g' = e^x + g \\ 
\Longrightarrow e^x + g = Ce^x \, \Longrightarrow g = (C-1)e^x \text{ but } f'(0) =  1 \text{ so } \boxed{ g = e^x } 
\end{gathered}
\]
\end{enumerate}

\exercisehead{9} 
\begin{enumerate}
\item \[
\begin{aligned}
  g(2x) & = 2e^x g(x) \\
  g(3x) &= e^x g(2x) + e^{2x} g(x) = e^x 2 e^x g + e^{2x} g = 3e^{2x} g 
\end{aligned}
\]
\item \[
\begin{aligned}
  \text{ Assume } g(nx) & = n e^{(n-1)x } g \\
  g((n+1)x) & = e^x g(nx) + e^{(n+1)x} g(x) = ne^{nx} g(x) + e^{nx} g = (n+1) e^{nx} g 
\end{aligned}
\]
\item From $g(x+y) = e^y g(x) + e^x g(y)$, 
\[
g(0) = g(0) + g(0) \Longrightarrow g(0) = 0 
\]

\[
\begin{gathered}
  \frac{ g(x+h) - g(x)}{ h} = \frac{ e^h g(x) + e^x g(h) - g(x) }{ h} = g(x) \left( \frac{ e^h - 1 }{ h } \right) + \frac{ e^x g(h) }{ h } \\
  f'(0) = 2 = \lim_{h \to 0} \frac{ g(x+h) - g(x)}{ h } = \lim{ h \to 0} \frac{g(h)}{ h } 
\end{gathered}
\]
\item $g'(x) = g(x) + 2 e^x \, \boxed{ C=2 } $
\end{enumerate}

\exercisehead{10}
\[
\begin{gathered}
  \forall x \in \mathbb{R}, f(x+a) = bf(x); f(x+2a) = b f(x+a) = b^2 f(x) \\
  f(x + (n+1)a) = f(x+na +a) = b f(x+na) = b^{n+1} f(x) \\
  f(x+na) = b^n f(x) \\
  \boxed{ f(x) = b^{x/a} g(x) \text{ where $g$ is periodic in $a$ } }
\end{gathered}
\]

\exercisehead{11} 
\[
\begin{aligned}
  & (\ln{ (fg) })' = \frac{f'}{f} + \frac{g'}{g} \Longrightarrow  (fg)' = f'g +fg' \\
  & \left( \ln{ \left( \frac{f}{g} \right) } \right)' = \frac{f'}{f} - \frac{g'}{g} \Longrightarrow \frac{f'g -g'f}{ g^2 } = \left( \frac{f}{g} \right)'
\end{aligned}
\]

\exercisehead{12} $A = \int_0^1 \frac{e^t}{ t+ 1 } dt $
\begin{enumerate}
\item \[
  \begin{gathered}
    \int_{a-1}^a \frac{e^{-t}}{ t- a -1 } dt \\
\begin{gathered}
u =t -a 
\end{gathered} \quad \quad \Longrightarrow  \int_{-1}^0 \frac{ e^{-t-a}}{t-1} dt = -\int_1^0 \frac{e^{t-a}}{ -t - 1 } dt = -e^{-a} \int_0^1 \frac{ e^t}{t+1} = \boxed{ -e^{-a} A }
\end{gathered}
\]
\item $\int_0^1 \frac{t e^{t^2}}{ t^2 + 1 } dt = \int_0^1 \frac{ \frac{1}{2} du e^u }{ u + 1 } = \boxed{ \frac{1}{2} A } $ 
\item $ \int_0^1 \frac{e^t}{ (t+1)^2 } dt = \left. \frac{ -e^t }{ (t+1)} \right|_0^1 - \int_0^1 \frac{ -e^t}{ t+1}  = \boxed{ \frac{-e^1 }{2} + 1 + A } $
\item $\int_0^1 e^t \ln{ (1+t) } dt = e^t \ln{ (1+t) } - \int \frac{ e^t}{ 1 +t } = \boxed{ e\ln{2} - A }$
\end{enumerate}

\exercisehead{13}  
\begin{enumerate}
\item $p(x) = c_0 + c_1 x + c_2 x^2$; \quad \quad \, $f(x) = e^x p(x)$ \quad \quad \, $p' = c_1 + 2 c_2 x $ \quad \quad \, $p'' = 2 C_2 $ \\
$f' = f + e^x p'$ 
\[
\begin{gathered}
  f^{(n)}(x) = \sum_{j=0}^n \binom{n}{j} e^x (p(x))^j = f + e^x (c_1 + 2 c_2 x) + \left( \frac{ n!}{(n-1)! } \right) + e^x ( 2 c_2 ) \left( \frac{ n! }{ (n-2)! 2! } \right)  \\
  f^{(n)}(0) = c_0 + c_1 n + n(n-1) c_2 
\end{gathered}
\]
\item See generalization below.  
\item 
\[
\begin{gathered}
  p = \sum_{j=0}^m a_j x^j ; \quad \quad \, p(0) = a_0 ; \quad \quad \, p^{(k)}(x) = \sum_{j=0}^m a_j \frac{ j!}{ (j-k)!} x^{j-k} = p^{(k)}(0) = a_k k! \\
  f^{(n)}(x) = \sum_{j=0}^n \binom{n}{j} e^x p^{(j)}(x) \\
  f^{(n)}(0) = \sum_{j=0}^n \binom{n}{j} p^{(j)}(0) = \sum_{j=0}^m \binom{n}{j} a_j j! = \boxed{ \sum_{j=0}^m \frac{ n!}{ (n-j)!} a_j }
\end{gathered}
\]
So for $m=3$, then $f^{(n)}(0) = a_0 + n a_1 + n(n-1) a_2 + n(n-1)(n-2) a_3$
\end{enumerate}

\exercisehead{14}
$f(x) = x \sin{ax}$ ; \quad \quad \, $f^{(2)} = -a^2 x \sin{ax} +  2a \cos{ax}$ \medskip \\
$f^{(2n)}(x) = (-1)^n (a^{2n} x \sin{ax} - 2 n a^{2n-1} \cos{ax} )$
\[
\begin{aligned}
  f^{(2n+1)}(x) & = (-1)^n ( a^{2n+1} x \cos{ax} + a^{2n} \sin{ax} + 2na^{2n} \sin{ax} ) \\
  f^{(2n+2)}(x) & = (-1)^n (-a^{2n+2} x \sin{ax} + a^{2n+1} \cos{ax} (2n+2) )
\end{aligned}
\]

\exercisehead{15} 
\[
\begin{aligned}
 \sum_{k=0}^n (-1)^k \binom{n}{k} \frac{ 1 }{ k+m + 1 } & = \sum_{k=0}^n (-1)^k \binom{n}{k} \int_0^1 t^{k+m} dt = \int_0^1 \sum_{k=0}^n (-1)^k \binom{n}{k} t^{k+m} dt = \\
 & = \int_0^1 t^m \sum_{k=0}^n \binom{n}{k} (-t)^k dt = \int_0^1 t^m (1-t)^n dt = \\
 & = -\int_1^0 (1-u)^m u^n du = \int_0^1 (1-u)^m u^n du = \\
 & = \int_0^1 \sum_{j=0}6m \binom{m}{j} (-u)^j u^n du = \boxed{ \sum_{j=0}^m (-1)^j \binom{m}{j} \int_0^1 t^{j+n} dt } \\
 & \quad \quad \begin{aligned} u & 1 -t \\ du & = -dt \end{aligned}
\end{aligned}
\]

\exercisehead{16} $F(x) = \int_0^x f(t) dt $
\begin{enumerate}
\item \[
F(x) = \int_0^x (t+|t|)^2 = \begin{cases} \int_0^x (2t)^2 dt = \frac{4}{3} x^3 & \text{ if } t,x \geq 0 \\
  \int_0^x 0 dt = 0 & \text{ if } t,x < 0 \end{cases} 
\]
\item
\[
\begin{aligned}
  F(x) & = \int_0^x f(t) dt = \begin{cases} \int_0^x (1-t^2) dt & \text{ if } |t| \leq 1 \\ \int_0^x (1-|t|) dt & \text{ if } |t| > 1 \end{cases} = \\
  & = \begin{cases} \left. \left( t - \frac{1}{3} t^3 \right) \right|_0^x = x - \frac{x^3}{3} & \text{ if } |x| \leq 1 \\ \begin{cases} \frac{2}{3} + \int_1^x (1-t) dt = x - \frac{x^2}{2} + \frac{1}{6} & \text{ if } x > 1 \\ \frac{-2}{3} + \int_{-1}^x (1+t) dt = x + \frac{x^2}{2} - \frac{1}{6} & \text{ if } x < 1 \end{cases} & \text{ if } |x| \geq 1 \end{cases} \\
\end{aligned}
\]
\item $f(t) = e^{-|t| }$
\[
\begin{aligned}
  F(x) & = \int_0^x f(t) dt = \int_0^x e^{-|t|} dt = \\
  & = \begin{cases} \int_0^x e^{-t} dt = \left. e^{-t} \right|_x^0 = 1 - e^{-x} & \text{ if } x \geq 0 \\
    \int_0^x e^t dt = \left. e^t \right|_0^x = e^x -1 & \text{ if } x < 0 
\end{cases} 
\end{aligned}
\]
\item $f(t) = \text{ max. of $1$ and $t^2$ } $  
\[
\begin{aligned}
F(x) & = \int_0^x f(t) dt = \begin{cases} 
  \int_0^x 1 dt = x & \text{ if } |x| \leq 1 \\
  1 + \int_1^x t^2 dt & \text{ if } x > 1 \\
  -\int_0^x f & \text{ if } x < -1 
\end{cases} = \\
& = \begin{cases} 
x & \text{ if } |x| \leq 1 \\
1 + \left. \frac{1}{3} t^3  \right|_1^x = \frac{x^3}{3} + \frac{2}{3} & \text{ if } x > 1 \\
- \int_x^{-1} t^2 + - \int_{-1}^0 1 = \left. \frac{1}{3} t^3 \right|_{-1}^x -1 = \frac{x^3}{3} + \frac{2}{3} & \text{ if } x > 1 
\end{cases}
\end{aligned}
\]
\end{enumerate}

\exercisehead{17} $\int \pi f^2 = \pi \int_0^a f^2 = a^2 + a $.  
\[
\begin{gathered}
  \left( \frac{ x^2 + x }{ \pi} \right)' = \sqrt{ \frac{2x+ 1 }{ \pi} } \\
  \text{ for } \int_0^a \frac{ 2x + 1 }{ \pi } = \left. \left( x^2 + x \right) \right|_0^a = a^2 + a 
\end{gathered}
\]

\exercisehead{18} $f(x) = e^{-2x}$. 
\begin{enumerate}
  \item $A(t) = \int_0^t e^{-2x} dx = \left. \frac{ e^{-2x}}{ -2} \right|_0^t = \frac{ e^{-2t} -1 }{ -2 }$
  \item $V(t) = \pi \int_0^t e^{-4x} dx = \frac{ \pi}{ -4} ( e^{-4t} -1 )$
  \item 
\[
\begin{gathered}
  y = e^{-2x} \Longrightarrow \frac{ \ln{y}}{ -2} = x \\
  \begin{aligned}
  W(t) & = \pi \int_{ e^{-2t}}^1 \left( \frac{ \ln{y}}{-2} \right)^2 dy = \frac{ \pi}{4} \left. ( y(\ln{y})^2 - (2(y \ln{y} - y )) ) \right|_{ e^{-2t}}^1  = \\
  & = \frac{ \pi}{4} \left( 2 - \left( e^{-2t} 4 t^2 - (2 (e^{-2t} (-2t) - e^{-2t} ))  \right) \right) = \\
  & = \left( \frac{ \pi}{2} - \pi t e^{-2t} - \frac{ \pi}{2} e^{-2t} \right) 
  \end{aligned} \\
\text{ where the antiderivative used was } (y (\ln{y})^2 )' = (\ln{y})^2 + 2 \ln{y}
\end{gathered}
\]
  \item 
\[
\begin{gathered}
  \frac{ \frac{\pi}{4} ( 1 - e^{-4t } )}{ \left( \frac{ 1 - e^{-2t} }{ 2 } \right) } = \frac{ \pi}{2} \frac{ \left( \frac{ e^{-4t} - 1 }{ t } \right) }{ \frac{ e^{-2t } - 1 }{ t } } = \boxed{ \pi } \\
\text{ where we used the limit } \lim_{x\to 0} \frac{ e^{cx } - 1 }{ x } = c 
\end{gathered}
\]
\end{enumerate}

\exercisehead{19} $\sinh{ c } = \frac{3}{4} $
\begin{enumerate}
\item 
\[
\begin{gathered}
e^c = e^x + \sqrt{ e^{2x} + 1 } \\
\begin{aligned}
  \frac{ e^c - e^{-c}}{ 2 } & = \frac{ e^x + \sqrt{ e^{2x} + 1 } - \frac{ 1}{ e^x + \sqrt{ e^{2x} + 1 } } }{ 2 } = \\
  & = \frac{ e^{2x} + 2 e^x \sqrt{ e^{2x} + 1 } + e^{2x} + 1 - 1 }{ 2 ( e^x + \sqrt{ e^{2x} + 1 } ) } = e^x = \frac{3}{4} \\
  & \boxed{ x = \ln{3} - 2\ln{2}  }
\end{aligned} 
\end{gathered}
\]
\item 
\[
\begin{aligned}
 \frac{ e^c - e^{-c} }{ 2 } & = \frac{ e^x - \sqrt{ e^{2x} - 1 }  - \frac{ 1 }{ e^x - \sqrt{ e^{2x} - 1 } } }{ 2 } = \\
  & = \frac{ e^{2x} - 2 e^x \sqrt{ e^{2x} -1 } + e^{2x} - 1 - 1 }{ 2 ( e^x - \sqrt{ e^{2x} - 1 } ) } \\
 & = \frac{ e^x }  + \frac{ -1 }{ e^x - \sqrt{ e^{2x} - 1 }} = e^x - \frac{1}{ e^c} = \frac{3}{4} \\
 & \Longrightarrow \boxed{ x = \ln{5} - 2 \ln{2} }
\end{aligned}
\]
\end{enumerate}

\exercisehead{20}
\begin{enumerate}
\item True.  $\ln{ (2^{\log{5} } ) } = \ln{5^{\ln{2}}} = (\ln{2})\ln{5}$.  
\item   $\log_3{5} = \frac{ \log_2{5} }{ \log_2{3} }$ This is a true fact.  
\[
\begin{gathered}
 \frac{ \log_3{5} }{ \log_2{3}} = \frac{ \log_2{5}}{ (\log_2{3} )^2 } = \log_2{5}  \\
 \Longrightarrow 1 = \log_2{3} \quad \text{ False }
\end{gathered}
\]
\item \textbf{ Use induction } 
\[
\begin{aligned}
& n=1 \, 1^{-1/2} < 2 \sqrt{1} \\
& n=2 \, 1 + \frac{1}{\sqrt{2}} < 2 \sqrt{2} \\
  & n+1 \text{ case } \, \begin{aligned} \sum_{k=1}^{n+1} k^{-1/2} & = \sum_{k=1}^n k^{-1/2} + \frac{1}{ \sqrt{ (n+1) }} < \\
    & < 2 \sqrt{ n} \left( \frac{ \sqrt{ (n+1)} }{ \sqrt{ (n+1)} }\right) + \frac{1}{ \sqrt{ (n+1)} }
\end{aligned}
\end{aligned}
\]
Now $(n+\frac{1}{2})^2 = n^2 + n + \frac{1}{2} > n^2 + n$, certainly.  So then 
\[
\begin{gathered}
  n+\frac{1}{2} > \sqrt{ n^2 + n } \Longrightarrow n + 1 > \sqrt{ n^2 + n } \\
  \Longrightarrow \sum_{k=1}^{n+1} k^{-1/2} < 2 \sqrt{ n+1} 
\end{gathered}
\]
\item 
\[
f = (\cosh{x} - \sinh{x}  -1 ) = \frac{ e^x + e^{-x} - e^x + e^{-x}}{ 2 } - 1 = e^{-x} - 1 < 0 \text{ for } x > 0 
\]
False.  
\end{enumerate}

\exercisehead{21} For $0 < x < \frac{ \pi }{2}$, 
\[
\begin{gathered}
  (\sin{x})' = \cos{x} > 0 \text{ for } 0 < x < \frac{ \pi}{2} \\
  (x-\sin{x})' = 1 - \cos{x} \geq 0 \text{ for }  0 < x < \frac{ \pi}{2} \\
  (x-\sin{x})(x = 0) = 0 \Longrightarrow \sin{x} <x 
\end{gathered}
\]

\exercisehead{22} 
\[
\begin{gathered}
  \frac{1}{t} < \frac{1}{t} \left( \frac{ x+1}{t} \right) \quad \text{ if } 0 < x < t < x+1 \\
  \int_x^{x+1} \frac{1}{t} dt = \ln{ (x+1)} - \ln{x} ; \, \int_x^{x+1} \frac{x+1}{t} = \boxed{ \frac{1}{x} }
  \text{ So } \ln{ \frac{ x +1}{ x } } < \frac{1}{x} 
\end{gathered}
\]

\exercisehead{23} 
\[
\begin{gathered}
  (x-\sin{x})' = 1 - \cos{x} \geq 0 \, \forall x > 0 \\
  \text{ since } (x-\sin{x})(x=0) = 0 ; \, (x-\sin{x})'(x=0) = 0 , \, \text{ then } x - \sin{x} > 0 \text{ in general for } x > 0 \\
  (\sin{x} - \left( x-\frac{x^3}{6} \right) )'  = \cos{x} - 1 + \frac{x^2}{2} > 0 \, \forall x > 0 \\
  \Longrightarrow x - \frac{x^3}{6} < \sin{x} < x 
\end{gathered}
\]

\exercisehead{24} $(x^b + y^b)^{1/b} < (x^a + y^a)^{1/a}$ if $x> 0 , \, y > 0$ and $0 < a <b $ \medskip \\
$(x^n +y^n)^{1/n} = x ( 1 + \left( \frac{y}{x} \right)^n )^{1/n}$.  Without loss of generality, assume $x<y$.  \medskip \\
Consider $(1+A^n)^{1/n}$, $A$ constant.
\[
\begin{gathered}
  \begin{aligned}
    ((1 +A^n)^{1/n})' & = (\exp{ \left( \frac{1}{n} \ln{ (1+A^n) } \right) })' = (1+A^n)^{1/n} \left( \frac{-1}{n^2} \ln{ (1+A^n) } + \frac{1}{n} \left( \frac{1}{ 1 + A^n } \right)(\ln{A})A^n \right) = \\
    & = (1+A^n)^{1/n} \left( \frac{ -(1+A^n)\ln{ (1+A^n)} + n (\ln{A})A^n }{ n^2 ( 1 +A^n) } \right) = \\
    & = (1+A^n)^{1/n} \left( \frac{ - \ln{ (1+A^n ) } -A^n \ln{ (1+A^n) } + A^n \ln{A^n} }{ n^2 ( 1+A^n) } \right) = \\
    & = \frac{ (1+A^n)^{\frac{1-n}{n} } }{n^2} \left( \frac{ - \ln{(1+A^n)} + A^n \ln{ \left( \frac{A^n}{1+A^n} \right) } }{ n^2 ( 1 +A^n) } \right) < 0 \quad \, \text{ since $\ln{ \left( \frac{A^n}{ 1+A^n} \right) } <0 $ }
  \end{aligned} \\
  \Longrightarrow (x^b + y^b)^{1/b} < (x^a + y^a)^{1/a} \quad \text{ if $b>a$ }
\end{gathered}
\]

\exercisehead{25} 
\begin{enumerate}
\item \[
\int_0^x e^{-t} t = \left. \left( -te^{-t} - e^{-t} \right) \right|_0^x = - xe^{-x} -e^{-x} + 1 
\]
\item 
\[
\begin{aligned}
  \int_0^t t^2 e^{-t} dt &= \left. -t^2 e^{-t} \right|_0^x - \int - e^{-t} (2t) dt = -x^2 e^{-x} + 2 \int te^{-t} dt = \\
  & = -x^2 e^{-x} + -2xe^{-x} -2 e^{-x} +2 
\end{aligned}
\]
\item 
\[
\begin{aligned}
  \int_0^x t^3 e^{-t} dt & = \left. -t^3 e^{-t} \right|_0^x -3 \int_0^x t^2 (-e^{-t}) dt = -x^3 e^{-x} + 3 \int_0^x t^2 e^{-t} dt = \\
  & = -x^3 e^{-x} + 3(2) (e^{-x})\left( e^x - 1 -x -\frac{x^2}{2!} \right)
\end{aligned}
\]
\item Assume the induction hypothesis, that
\[
\begin{gathered}
  \int_0^x t^n e^{-t} dt = n! e^{-x} \left( e^x - \sum_{j=0}^n \frac{ x^j}{j!} \right) \\
  \begin{aligned}
    \int_0^x t^{n+1} e^{-t} dt & = \left. -t^{n+1} e^{-t} \right|_0^x - \int (n+1)t(-e^{-t}) = -x^{n+1} e^{-x} + (n+1)n! e^{-x} \left( e^x - \sum_{j=0}^n \frac{ x^j}{ j!} \right) \\
    & = \boxed{ (n+1)! e^{-x} \left( e^x - \sum_{j=0}^{n+1} \frac{x^j}{j!} \right) }
  \end{aligned}
\end{gathered}
\]
\end{enumerate}

\exercisehead{26} Consider the hint $a_1 \sin{x} + b\cos{x} = A(a\sin{x} + b\cos{x}) + B (a \cos{x} -b\sin{x} )$.  Solve for $A,B$ in terms of $a_1, b_1, a, b$.  Matching up term by term the coefficients for $\sin$ and $\cos$ separately,
\[
\begin{gathered}
  \begin{aligned}
    Aa^2 - abB & = aa_1 \\
    Ab^2 + Bab & = b_1 b \\
    \boxed{ A = \frac{ aa_1 + bb_1}{ a^2 + b^2 } } & 
  \end{aligned}
\quad 
\begin{aligned}
  -Aab + B b^2 & = - a_1 b  \\
  Aab + Ba^2 & = ab_1 \\
  & \boxed{ B = \frac{ ab_1 - a_1 b }{ a^2 + b^2 } }
\end{aligned} \\
\text{ So if not both $a,b = 0$ },  \\
\begin{aligned}
  \int \frac{ a_1 \sin{x} + b_1 \cos{x} }{ a \sin{x} + b\cos{x} } & = \int \frac{ A (a\sin{x} + b\cos{x} ) + B (a \cos{x} -b\sin{x} ) }{  a\sin{x} + b\cos{x} } = \\
  & = \boxed{ Ax + B \ln{ |a \sin{x} + b\cos{x} | } + C }
\end{aligned}
\end{gathered}
\]

\exercisehead{27} 
\begin{enumerate}
  \item \[
\begin{aligned}
  f'(x^2) & = \frac{1}{x} \\
  \frac{df}{du} &  = u^{-1/2} \\
  f(x^2) & = 2x - 1 
\end{aligned}
\]
  \item 
\[
f'(\sin^2{x}) = 1 - \sin^2{x} \, f'(u) = 1 - u \, f= u - \frac{1}{2} u^2 + C \Longrightarrow \boxed{ f(x) = x - \frac{x^2}{2} + \frac{1}{2} }
\]
  \item 
\[
f'(\sin{X}) = (1-\sin^2{x}) \, f(u) = u - \frac{1}{3}u^3 + C \, \boxed{ f(x) = x - \frac{x^3}{3} + \frac{1}{3}}
\]
  \item 
\[
\begin{gathered}
  f'(\ln{x})  = \begin{cases} 1 & \text{ for } x \leq 1 \\ x & \text{ for } x > 1 \end{cases} = \begin{cases} 1 & \text{ for } 0 < x \leq 1 \\ e^{\ln{x}} & x > 1 \end{cases}  \\
  \boxed{ f(y) = \begin{cases} 
      y & \text{ for } y < 0 \\
      e^y - 1 & \text{ for } y > 0 
\end{cases} }
\end{gathered}
\]
\end{enumerate}

\exercisehead{28} 
\begin{enumerate}
  \item 
\[
\begin{gathered}
  Li(x) = \int_2^x \frac{dt}{ \ln{t} } \text{ if } x \geq 2 \\
  Li(x) = \frac{x}{ \ln{x}} - 2 \frac{1}{ \ln{2} } - \int_2^x \frac{ -1}{ (\ln{x})^2 } dt = \frac{x}{\ln{x}} + \int_2^x \frac{dt}{ (\ln{x})^2 } - \frac{ 2 }{ \ln{2}} 
\end{gathered}
\]
  \item 
\[
\begin{gathered}
  \begin{aligned}
    Li(x) & = \frac{x}{\ln{x}} - \frac{2}{ \ln{x}} - \frac{2}{ (\ln{2})^2} + \frac{x}{ (\ln{x})^2 } - \int_a^x \frac{ -2}{ (\ln{t})^3 }dt \\
    Li(x) & = \frac{x}{\ln{x}} + \sum_{k=1}^{n-1} \frac{ k! x }{ \ln^{k+1}{x} } + n! \left( \frac{x}{ \ln^{n+1}{x}} - \int_2^x \frac{ -(n+1) dt }{ \ln^{n+2}{t} } \right) \\
    Li(x) & = \frac{x}{ \ln{x}} + \sum_{k=1}^n \frac{ k! x }{ \ln^{k+1}{x} } + (n+1)! \int_2^x \frac{dt}{ \ln^{(n+1)+1}{t} }
  \end{aligned}
\quad 
\begin{aligned}
  C_2 & = \frac{-2}{\ln{x}} + - \sum_{j=2}^2 \frac{ 2 (j-1)! }{ (\ln{2})^j } \\
  C_n & = -2 \frac{1}{\ln{2}} - \sum_{j=2}^n \frac{ 2 (j-1)!}{ (\ln{2})^j } \\
  C_{n+1} & = -2 \frac{1}{ \ln{2}} -  \sum_{j=2}^{n+1} \frac{ 2 (j-1)!}{ (\ln{2})^j }
\end{aligned}
\end{gathered}
\]
\item 
\[
\begin{gathered}
  Li(x) = \int_2^x \frac{dt}{ \ln{t}} \quad \begin{aligned} u & = \ln{t} \\ du & = \frac{1}{t} dt \\ e^u du & = dt \, e^u =t \end{aligned}  \\
  \boxed{ Li(x) =\int_{\ln{2}}^{\ln{x}} \frac{ e^t dt }{ t } }
\end{gathered}
\]
\item 
\[
\begin{gathered}
  c = 1 + \frac{1}{2} \ln{2}  \\
  t = u +1 \Longrightarrow \begin{aligned} \int_{c-1}^{x-1} \frac{ e^{2(u+1)} du }{ u }  & = e^2 \int_{c-1}^{x-1} \frac{ e^{2u}}{ u} du = \\ 
    & = e^{2} \frac{1}{2} \int_{ \frac{ 2 ( c-1)}{3 } }^{ \frac{ 2 (x-1)}{ 3 } } \frac{ e^t }{ \frac{t}{2}} dt = \\
    & = \boxed{ e^2 Li(e^{ 2 (x-1)} ) }
  \end{aligned} 
\end{gathered}
\]
\item 
\[
\begin{gathered}
\begin{aligned}
  f(x) & = e^4 Li(e^{2(x-2)} ) & - e^2 Li(e^{2(x-1)})  \\ 
  & = \int_c^x \frac{ e^{2t}}{ t-2} & - \int_c^x \frac{ e^{2t}}{ (t-1) } 
\end{aligned}
\Longrightarrow \boxed{ f'(x) = \frac{ e^{2x}}{ t-2} + - \frac{ e^{2x}}{ t-1} = e^{2x} \left( \frac{1}{ t^2 -3t +2 } \right)}
\end{gathered}
\]
\end{enumerate}

\exercisehead{29}$f(x) = \log{ |x| }$ if $x < 0$.  $\forall x < 0  \, \exists $ uniquely $\ln{ |x|}$ since $f' = \frac{1}{x} < 0 \, \forall x < 0$.  
\[
-e^y = x(y) = g(y) \quad \boxed{ D = \mathbb{R}}
\]

Recall Theorem 3.10.  
\begin{theorem}
Assume $f$ is strictly increasing and continuous on an interval $[a,b]$.  Let $c=f(a), d = f(b)$ and let $-g$ be the inverse of $f$.  \medskip \\
That is $\forall  y \in [c,d] $, Let $g(y)$ be that $x \in [a,b]$, such that $y = f(x)$.  \bigskip \\
Then 
\begin{enumerate}
\item $-g$ is strictly increasing on $[c,d]$ 
  \item $-g$ is continuous on $[c,d]$
\end{enumerate}
\end{theorem}

\exercisehead{30} 
$f(x) = \int_0^x (1+t^3)^{-1/2} dt $ if $x \geq 0$.  
\begin{enumerate}
\item \[
f'(x) = \frac{1}{ \sqrt{ 1 + x^3 } } > 0 \text{ for } x > 0 
\]
\item 
\[
\begin{gathered}
  g'(x) = \frac{1}{ f'(x)} = \sqrt{ 1 + x^3 } \, g''(x) = \frac{3 x^2}{ 2 \sqrt{ 1 + x^3} }
\end{gathered}
\]
\end{enumerate}

%\exercisehead{35} Given the Bernoulli polynomials, 
%\begin{equation}\label{E:Bernoulli_polynomials}
%P_0(x) = 1; \, P_n'(x) = n P_{n-1}(x) \, \text{ and } \, \int_0^1 P_n(x) dx = 0 \, \text{ if } n \geq 1 
%\end{equation}

%\begin{enumerate}
%\item 
%\[
%  P_1 = x - \frac{1}{2}; \, P_2 = x^2 - x + \frac{1}{6}; \, P_3 = x^3 - \frac{3}{2} x^2 + \frac{x}{2} + x; \, P_4 = x^4 - 2 x^3 + x^2 - \frac{1}{30} ; \, P_5= x^5 - \frac{5}{2} x^4 + \frac{5}{3}x^3 - \frac{x}{6} 
%\]  
%\item We already did the $n=0,1, \dots, 5 $ cases above.   
%\item 
%\end{enumerate}


%\exercisehead{36}





%%%%%%%%%%%%%%%%%%%%%%%%%%%%%%%%%%%%%%%%%%%%%%%%%%%%%%%%%%%%%%%%%%%%%%%%%%%%%%%%%%%%%%%%%%%%%%%%%%%%%%%%%%%%%%%%%%%%%%%%%%%%%%%%%%%%%%%%%%%%%%%%%%%%%%%%%%%%%%%%%%%%%%%%%%%%%%%%%%%%%%%%%%%%
\subsection*{ 7.4 Exercises - Introduction, The Taylor polynomials generated by a function, Calculus of Taylor polynomials }
%-----------------------------------%-----------------------------------%-----------------------------------

Use the following theorems for the following exercises.  
\begin{theorem}[Properties of Taylor polynomials, Apostol Vol. 1. Theorem 7.2.]
  \begin{enumerate}
    \item Linearity $T_n(c_1 f + c_2 g) = c_1 T_n(f) + c_2 T_n(g)$ 
    \item Differentiation $(T_n f)' = T_{n-1}(f') $ 
    \item Integration.  If $g(x) = \int_a^x f(t)dt$ \\
      $ T_{n+1} g(x) = \int_a^x T_n f(t) dt $
  \end{enumerate}
\end{theorem}

\begin{theorem}[ Substitution Property, Apostol Vol. 1. Theorem 7.3. ] Let $g(x) = f(cx)$, $c$ is a constant.  
  \begin{equation}
    T_n g(x; a) = T_n f(cx; ca) 
  \end{equation}
\end{theorem}

This theorem is useful for finding new Taylor polynomials without having to find the $j$th derivatives of the desired function.  
\begin{theorem}
$P_n$ is a polynomial of degree $n \geq 1$.  \\
Let $f,g$ be 2 functions with derivatives of order $n$ at $0$.  
\begin{equation}
  f(x) = P_n (x) + x^n g(x) 
\end{equation}
where $g(x) \mapsto 0$ as $x \mapsto 0$.  \\
Then $P_n = T_n(f,x=0)$.  
\end{theorem}

\exercisehead{3}
\[
\begin{gathered}
  T_n f(x) = \sum_{j=0}^n \frac{ f^{(j)} (a) }{ j!} (x-a)^j   \\
  \begin{aligned}
    a^x & = e^{x \ln{a} } \\
    (a^x)' & = (a^x) \ln{a} \\
    (a^x)^{(n+1)} & = (a^x (\ln{a})^n )' = a^x (\ln{a})^{ n+1 } 
  \end{aligned} \\
  T_n (a^x) = \sum_{j=0}^n \frac{ (\ln{a} )^j }{ j! } x^j
\end{gathered}
\]

\exercisehead{4}
\[
\begin{aligned}
& \left( \frac{1}{1+x} \right)' = \frac{-1}{(1+x)^2 }; \, \left( \frac{1}{ 1+x } \right)'' = \frac{ (-1)^2 2 }{ (1+x)^3 } \\
& \left( \frac{1}{1+x} \right)^{(n+1)} = \left( \frac{ (-1)^n n! }{ (1+x)^{n+1} } \right)' = \frac{ (-1)^{n+1} (n+1)! }{ (1+x)^{n+2} }\\
& T_n\left( \frac{1}{1+x} \right) = \sum_{j=0}^n (-1)^j x^j 
\end{aligned}
\]

\exercisehead{5} $\boxed{ \text{ Use Theorem 7.4. } }$.  Theorem 7.4 says for $f(x) = P_n (x) +x^n g(x)$, $P_n(x)$ is the Taylor polynomial.  
\[
\begin{gathered}
  \frac{1}{1-x^2} = \left( \sum_{j=0}^n (x^2)^j \right) + \frac{ (x^2)^{n+1} }{ 1-x^2 } = \left( \sum_{j=0}^n x^{2j} \right) + \frac{ x^n x^{n+2} }{ 1-x^2 } \\
  \frac{x}{1-x^2} = \left( \sum_{j=0}^n x^{2j+1} \right) + \frac{ (x^{2n+3}) }{ 1-x^2 } \\
  T_{2n+1} \left( \frac{x}{1-x^2 } \right) = \sum_{j=0}^n x^{2j+1}
\end{gathered}
\]

\exercisehead{6} 
\[
(\ln{ (1+x) })' = \frac{1}{1+x} \quad T_n \left( \frac{1}{1+x} \right) = \sum_{j=0}^n (-x)^j \quad T_n(\ln{1+x}) = \sum_{j=0}^n (-1)^j \frac{ x^{j+1}}{j+1} = \sum_{j=1}^n (-1)^{j+1} \frac{x^j}{ j}
\]

\exercisehead{7} 
\[
\begin{gathered}
  \left( \log{\sqrt{ \frac{1+x}{1-x} } } \right)' = \sqrt{ \frac{1-x}{1+x} } \sqrt{ \frac{1-x}{1+x} } \frac{1}{ (1-x)^2 } = \frac{1}{(1+x)(1-x) } = \frac{1}{1-x^2 } \\
  \int \frac{1}{1-x^2 } = \log{ \sqrt{ \frac{1+x}{1-x} } } \text{ so } \int \sum_{j=0}^n x^{2j} = \sum_{j=0}^n \frac{x^{2j+1}}{2j+1} = T_{2n+1}\left( \ln{ \sqrt{ \frac{1+x}{1-x} } } \right)
\end{gathered}
\]

\exercisehead{8}
\[
T_n \left( \frac{1}{2-x} \right) = T_n \left( \frac{ 1/2}{1-x/2} \right) = \frac{1}{2} T_n \left( \frac{1}{1- \left( \frac{1}{2} \right) x } \right) = \frac{1}{2} \left( \sum_{j=0}^n (\frac{1}{2} x)^j \right) = \sum_{j=0}^n \frac{ x^j }{ 2^{j+1} }
\]

\exercisehead{9}
We can show this in two ways.  \\
We could write out the actual polynomial expansion.
\[
(1+x)^{\alpha} = \sum_{j=0}^n \binom{ \alpha }{j} 1^{\alpha -j} x^j = \sum_{j=0}^n \binom{ \alpha }{ j } x^j 
\]
or determine each of the coefficients of the Taylor polynomial.  
\[
\begin{gathered}
  ((1+x)^{\alpha})' = \alpha(1+x)^{\alpha - 1} ; \, ((1+x)^{\alpha} )'' = \alpha(\alpha-1)(1+x)^{\alpha-2} \\
  ((1+x)^{\alpha})^{(n+1)} = \left( \frac{\alpha !}{ (\alpha - n )!} (1+x)^{ \alpha - n } \right)' = \frac{ \alpha ! }{(\alpha - (n+1) )! } (1+x)^{\alpha - (n+1) }
\end{gathered}
\]

\exercisehead{10}
Use the substitution theorem, Apostol Vol.1. Thm. 7.3., to treat $\cos{2x}$.  
\[
\begin{gathered}
T_{2n}(\cos{x} ) = \sum_{j=0}^n \frac{ (-1)^j x^{2j} }{ (2j)! } ; \, T_{2n}(\cos{2 x} ) = \sum_{j=0}^n \frac{ (-1)^j (2x)^{2j} }{ (2j)! } \\
T_{2n}(\sin{x}^2) = T_{2n}(\frac{1}{2} (1-\cos{2x} ) ) = \frac{1}{2} \left( 1 -  \sum_{j=0}^n \frac{ (-1)^j (2x)^{2j} }{ (2j)! } \right) = \sum_{j=1}^n \frac{ (-1)^{j+1} 2^{2j-1} x^{2j} }{ (2j)! }
\end{gathered}
\]

%-----------------------------------%-----------------------------------%-----------------------------------
\subsection*{ 7.8 Exercises - Taylor's formula with remainder, Estimates for the error in Taylor's formula, Other forms of the remainder in Taylor's formula }
%-----------------------------------%-----------------------------------%-----------------------------------

We will use Theorem 7.7, which we learn in the preceding sections, extensively.
\begin{theorem}\label{T:Error_on_Polynomial_Approximation}
  If for $j=1, \dots, n+1, m \leq f^{(j)}(t) \leq < \, \forall t \in I, $ $I$ containing $a$, 
  \begin{align}
  & m \frac{(x-a)^{n+1} }{ (n+1)! } \leq E_{n}(x) \leq M \frac{ (x-a)^{n+1}}{ (n+1)!}  \quad & \quad \text{ if } x > a \\
  & m \frac{(a-x)^{n+1} }{ (n+1)! } \leq (-1)^{n+1}E_{n}(x) \leq M \frac{ (a-x)^{n+1}}{ (n+1)!}  \quad & \quad \text{ if } x < a \\
  & E_{n}(x) = \frac{1}{n!} \int_a^x (x-t)^n f^{(n+1)}(t) dt 
\end{align}
\end{theorem}

\exercisehead{1} For $a=0$, $|\sin^{(j)}(x) | \leq 1 $ for $\forall x \in \mathbb{R}$.  
\[
\begin{gathered}
  E_n(x) \leq \frac{(x)^{2n+1} }{ (2n+1)! } \, \text{ if } x > 0; \, (-1)^{2n+1} E_{2n}(x) \leq (+1) \frac{ (-x)^{2n+1} }{ (2n+1)! } \\
  \Longrightarrow \boxed{ E_{2n}(x)| \leq \frac{ |x|^{2n+1}}{ (2n+1)!}}
\end{gathered}
\]

\exercisehead{2} 
\[
\begin{gathered}
  \cos{x} = \sum_{k=0}^n \frac{ (-1)^k x^{2k}}{ (2k)!} + E_{2n+1}(x) \, |\cos^{(j)}{(x)} | \leq 1 \\
  E_{2n+1}(x) \leq \frac{ x^{2n+2}}{ (2n+2)! } ; \, (-1)^{2n+2} E_{2n+1}(x) \leq (1) \frac{ (-x)^{2n+2}}{ (2n+2) } \\
  \Longrightarrow |E_{2n+1}(x) | \leq \frac{ |x|^{2n+2} }{ (2n+2)!} 
\end{gathered}
\]

\exercisehead{3} 
\[
\begin{gathered}
  \arctan{x} = \sum_{k=0}^{n-1} \frac{ (-1)^k x^{2k+1} }{ 2k +1} + E_{2n}(x) \\
  \sum_{k=0}^{n-1} \frac{ (-1)^k x^{2k+1} }{ 2k+1} \left( \frac{ (2k)!}{ (2k)!} \right) = \sum_{k=0}^{n-1} \frac{ (-1)^k (2k)! x^{2k+1} }{ (2k+1)!} 
  \Longrightarrow f^{(2k+1)}(0) = (-1)^k (2k)! \\ 
  \frac{ f^{(2n+1)}(0) x^{2n+1}}{ (2n+1)!} = \frac{ (-1)^n (2n)! x^{2n+1}}{ (2n+1)!} = \frac{ (-1)^n x^{2n+1}}{ 2n+1} \leq \frac{ x^{2n+1}}{ 2n+1}
\end{gathered}
\]

Note how $j$th derivative $(\arctan{x})^{(j)}$ changes sign with each differentiation for $f^{(2j+1)}(0)$.  Then we can always pick a small enough closed interval with $a=0$ as a left or right end point to make the $f^{(2j+1)}(0)$ value the biggest for $f^{(2j+1)}(t)$.   

\exercisehead{4} 
\begin{enumerate}
\item \[
\begin{gathered}
  x^2 = \sin{x} = x - \frac{x^3}{6} \Longrightarrow \frac{x^3}{6} + x^2 - x = \frac{x}{6} \left( x - \left( -3 +\sqrt{15} \right) \right) \left( x - (-3 - \sqrt{15} ) \right) \\
  \boxed{ x = \sqrt{15} - 3 }
\end{gathered}
\]
\item 
\[
\begin{gathered}
  E_4(r;0) = \frac{1}{4!} \int_0^r (r-t)^4 \cos{t} dt > 0 
  \sin{r}-r^2 = 0 + E_4(r) \leq \frac{r^5}{5!} < \frac{r}{5!} = \frac{ 3 }{ 5 (2)( 5)( 4)( 3) (2) 1 } = \frac{ \frac{3}{4} }{ (5)( 4)( 2) 5 } < \frac{1}{200} 
\end{gathered}
\]
\end{enumerate}

\exercisehead{5} 
\[
\begin{gathered}
  \arctan{r} - r^2 = r- \frac{r^3}{3} -r^2 + E_4(r;0) = 0 + E_4(r;0) \\
  E_4(r,0) = \frac{1}{4!} \int_0^r (x-t)^4 f^{(5)}(t)dt \leq \frac{ M (r^5)}{5!}
  \frac{ r^5 }{ 5!} = 0.065536 < \frac{ 7}{ 100 }
\end{gathered}
\]

\[
\begin{gathered}
  E_4(r,0) < E_j(r,0); j>4 \\
  \text{ the 5th degree term is } \frac{ f^{(5)}(0) }{ 5!} r^5 = \frac{24}{5!} r^5 > 0 \\
  \text{ so } r^2 -\arctan{r} = -E_4(r,0) < 0
\end{gathered}
\]

\exercisehead{6} Apply long division on the fraction in the integrand.  
\[
\begin{gathered}
  \int_0^1 \frac{ 1+x^{30}}{ 1+x^{60}} dx = \int_0^1 1 + \frac{ x^{30} - x^{60}}{ 1+x^{60}} dx = 1 + \int_0^1 x^{30} \left( \frac{ 1-x^{30}}{ 1 +x^{60}} \right) dx = \\
  = 1 + c \left. \frac{1}{31} x^{31} \right|_0^1 = 1 + \frac{c}{31} 
\end{gathered}
\]

\exercisehead{7} $\int_0^{1/2} \frac{1}{1+x^4} dx $.  
\[
\begin{gathered}
  \frac{1}{1+x^4 } = \sum_{j=0}^{\infty} (-x^4)^j = \sum_{j=0}^{n-1} (-x^4)^j + E_n = 1 + -x^4 + x^8 \dots \\
  \frac{16}{17} \frac{ x^{4n+1}}{ (n+1)!} \leq E_n(x;0) \leq \frac{1}{(n+1)! } (x^4)^{n+1} \\
  \Longrightarrow \int_0^{1/2} E_n = \frac{ \left( \frac{1}{2} \right)^{4n+5} }{ (n+1)!} \left( \frac{1}{4n+5} \right) \\
  \int_0^{1/2} \frac{1}{1+x^4} \simeq \frac{1}{2} + \frac{-1}{5} \left( \frac{1}{2} \right)^5 \\
  0.493852 < 0.49375 < 0.493858
\end{gathered}
\]

\exercisehead{8}
\begin{enumerate}
  \item 
    \[
    \begin{gathered}
      0 \leq x \leq \frac{1}{2} \, \sin{x} = x - \frac{ x^3}{3!} + E_4(x) \\
      |E_4(x)| \leq \frac{ M |x|^5 }{ 5! } = \frac{ \sin{\frac{1}{3}} |x|^5}{ 5!} \leq \frac{ 1 \left( \frac{1}{2} \right)^5}{ 5!} 
    \end{gathered}
    \]
  \item 
    \[
    \begin{gathered}
      \sin{x^2} = x^2 - \frac{ x^6}{6} + E_4(x^2) \\
      \int_0^{ \frac{ \sqrt{2}}{2} } \sin{x^2} = \left. \left( \frac{1}{3} x^3 - \frac{ x^7}{42} \right) \right|_0^{\sqrt{2}/2} = \sqrt{2} \left( \frac{1}{12} - \frac{1}{42(16)} \right) \\
      E_4(x^2) \leq \frac{ \sqrt{2}}{ 64 (5!)} \\
      \int_0^{ \frac{ \sqrt{2}}{2} } \sin{ x^2 } \leq \sqrt{2} \left( \frac{55}{672} + \frac{ 1}{ 64(5!)} \right) = 0.1159
    \end{gathered}
    \]
\end{enumerate} 

\exercisehead{9} 
\[
\begin{gathered}
  \sin{x} = x - \frac{ x^3}{6} + \frac{ x^5}{ 5!} \, E_6(x;0) \leq \frac{(1)x^7}{ 7! } \\
  \frac{ \sin{x}}{ x } = 1 - \frac{ x^2}{6} + \frac{ x^4}{5!} + \frac{ E_6(x;0)}{ x } \leq 1 - \frac{x^2}{6} + \frac{ x^4}{5!} + \frac{ x^6}{7!} \\
  \int_0^1 \frac{ \sin{x}}{ x } dx = 1 - \frac{1}{3} \left( \frac{1}{6} \right) + \frac{1}{5(5!)} + \frac{1}{7(7!)} = 0.9461 + \frac{1}{7(7!)} = 0.9461 + 0.0000283 
\end{gathered}
\]

\exercisehead{10} $\alpha = \arctan{ \frac{1}{5} }, \beta = 4 \alpha - \frac{ \pi}{4} $.  
\begin{enumerate}
  \item 
    \[
    \begin{aligned}
      \tan{ (A+B)} &= \frac{ (\tan{A} + \tan{B} )}{ 1 - \tan{A} \tan{B} } ; A=B = \alpha; \tan{ 2\alpha} = \frac{ 2\tan{\alpha}}{ 1- \tan^2{\alpha}} = \frac{ 2/5}{ 24/25} = 5/12 \\
      \tan{ 4\alpha} & = \frac{ 2 (\tan{ 2\alpha})}{ 1-\tan^2{(2\alpha)} } = \frac{ 2 (5/12)}{ 1 - (5/12)^2 } = \frac{ 10/12}{119/144} = \frac{120}{119}
      A=4\alpha, B = -\frac{\pi}{4} & \\
      \tan{ (4\alpha - \frac{ \pi}{4} )} & = \tan{ \beta} = \frac{ \tan{4\alpha} + \tan{ \left( -\frac{\pi}{4} \right) } }{ 1 - \tan{ 4\alpha} \tan{ \left( -\frac{ \pi}{4} \right) } } = \frac{ \frac{ 120}{ 119} + - \frac{ 119}{119} }{ 1 - \frac{ 120}{119} (-1) } = \frac{ 1}{ 239 } \\
      & \boxed{ 4 \arctan{ \frac{1}{5}} = \frac{ \pi }{4} + \arctan{ \frac{1}{239} } } \text{ This is incredible. }
    \end{aligned}
\]
\item \[
\begin{gathered}
    \begin{aligned} 
    T_{11}(\arctan{x}) & = \sum_{k=0}^{6-1} \frac{ (-1)^k x^{2k+1}}{ 2k+1 } + E_{2(6)}(x) \\ & = x + \frac{ -x^3}{3} + \frac{ x^5}{5} - \frac{x^7}{7} + \frac{x^9}{9} - \frac{x^{11}}{11} \dots \end{aligned} ; |E_{2(6)}(x) | \leq \frac{ x^{2(6)+1}}{ 2(6)+1 } \\
    \Longrightarrow 3.158328957 < 16 \arctan{ \frac{1}{5} } < 3.158328958 
\end{gathered}
\]
\item 
\[
\begin{gathered}
  T_3(\arctan{x}); x = \frac{1}{239} \\
  -0.016736304 < -4 \arctan{ \frac{1}{239} } < -0.016736304
\end{gathered}
\]
\item 
\[
\begin{gathered}
3.141592625 \\
3.158328972-0.016736300=3.141592672
\end{gathered}
\]
\end{enumerate}

%-----------------------------------%-----------------------------------%-----------------------------------
\subsection*{ 7.11 Exercises - Further remarks on the error in Taylor's formula.  The $o$-notation; Applications to indeterminate forms }
%-----------------------------------%-----------------------------------%-----------------------------------
\quad \\
\exercisehead{1} 
\[
2^x = \exp{ x \ln{2} } = \boxed{ 1 + (x \ln{2}) + \frac{ x^2 (\ln{2})^2 }{ 2! } } + o(x^2 )
\]

\exercisehead{2} 
\[
\begin{aligned}
  x (\cos{x}) & = ((x-1)+1) (\cos{x} ) = (x-1)\cos{1} + (-\sin{1})(x-1)^2 - \frac{ \cos{1} (x-1)^3 }{ 2 } + \cos{1} + \\
  & + (-\sin{1})(x-1) - \frac{ \cos{1}(x-1)^2 }{ 2 } + \frac{ \sin{1}(x-1)^3 }{3! } = \\ 
  & = \cos{1} + (\cos{1}-\sin{1})( x-1) + \left( -\sin{1} - \frac{ \cos{1}}{ 2 } \right) (x-1)^2 + \left( \frac{ \sin{1}- 3 \cos{1}}{ 3!} \right)(x-1)^3 + o(x-1)^3
\end{aligned}
\]

\exercisehead{3} Just treat the argument of $\sin{x-x^2 }$ just like $u$ with $u\to 0$.  
\[
\begin{gathered}
\sin{(x-x^2)} = (x-x^2) - \frac{ (x-x^2)^3}{ 3!} + \frac{ (x-x^2)^5}{5!} + o(x-x^2)^5 = \\
= (x-x^2) - \frac{1}{6} \left( x^3 -3 x^4 + 3 x^5 - x^6 \right) + \frac{1}{120} \left( x^5 - 5 x^6 + 10 x^7 - 10 x^8 + 5 x^9 + x^10 \right) = \\ 
= (x-x^2) - \frac{1}{6} x^3 - \frac{1}{2} x^4 + \frac{61}{120} x^5 - \frac{25}{120} x^6 
\end{gathered}
\]

\exercisehead{4} 
\[
\begin{gathered}
  \log{x} = \log{ (1 + (x-1)) } = (x-1) - \frac{ (x-1)^2 }{ 2} + \frac{ (x-1)^3}{ 3 } \\
  \Longrightarrow \boxed{ a = 0; b =1, c = \frac{-1}{2} }
\end{gathered}
\]

\exercisehead{5} 
\[
\begin{gathered}
  \cos{x} = 1 - \frac{1}{2} x^2 + o(x^3) \text{ as } x \to 0 \quad \begin{aligned} 1 - \cos{x} & = \frac{1}{2} x^2 + o(x^3) \\ \frac{ 1 - \cos{x}}{ x^2 } & = \frac{1}{2} + \frac{ o(x^3)}{ x^2 } \end{aligned} \\
  \boxed{ \text{ since } \frac{ 1 - \cos{x}}{ x^2 } = \frac{1}{2} + o(x), \, \frac{ 1 - \cos{x}}{x^2 } \to \frac{1}{2} \text{ as } x \to 0 } \\
  \cos{x} = 1 - \frac{1}{2} x^2 + \frac{ x^4}{4!} + o(x^5)  \Longrightarrow \cos{2x} = 1 - 2 x^2 + \frac{2}{3} x^4 + o(x^5) \\
  \frac{ 1 - \cos{2x} - 2x^2 }{ x^4} = \frac{ -\frac{2}{3} x^4 - o(x^5)}{ x^4} = \frac{-2}{3} - o(x) \to -\frac{2}{3} \text{ as } x \to 0
\end{gathered}
\]

\exercisehead{6} 
\[
\lim_{x\to 0} \frac{ \sin{ax}}{ \sin{bx} } = \lim_{x \to 0} \frac{ ax - \frac{ (ax)^3}{ 3! } + o(x^4) }{ bx + o(x^2) } = \frac{a}{b}
\]

\exercisehead{7} 
\[
\lim_{x \to 0} \frac{ \sin{2x}}{ \cos{2x} \sin{3x} } = \lim_{x \to 0} \frac{ (2x)+ \frac{ (2x)^3}{ 3! } + o(x^4) }{ \left( 1 - \frac{ (2x)^2 }{ 2! } + \frac{ (2x)^4}{ 4!} + o(x^5) \right) \left( (3x) - \frac{ (3x)^3}{3!} + o(x^4) \right) } = \boxed{ \frac{2}{3} }
\]

\exercisehead{8} 
\[
\lim_{x \to 0} \frac{ \sin{x} - x }{ x^3 } = \boxed{ -\frac{1}{6} }
\]

\exercisehead{9} 
\[
\lim_{x\to 0} \frac{ \ln{ 1 +x }}{ e^{2x} - 1 } = \lim_{x \to 0} \frac{ x - o(x)}{ 2x + o(x) } = \frac{1}{2}
\]

\exercisehead{10} Don't do the trig. identity.  
\[
\lim_{x\to 0} \frac{ 1 - \cos^2{x}}{ x \tan{x} } = \lim_{x\to 0} \frac{ 1 - \left( 1 - \frac{x^2}{2!} +o(x^2) \right)^2 }{ x \left( x + \frac{ x^3}{6!} +o(x^3) \right) } =  \lim_{x\to 0} \frac{ 1 -(1 + x^2 + o(x^2) )}{ x^2 + o(x^2 )} = 1
\]

\exercisehead{11} 
\[
\lim_{x\to 0} \frac{ x + -\frac{x^3}{6} + o(x^4) }{ x - \frac{x^3}{3} } = 1 
\]

\exercisehead{12} 
\[
\lim_{x\to 0} \frac{ e^{x \ln{a}} -1 }{ e^{x\ln{b}} -1 } = \lim_{x\to 0} \frac{ x\ln{a} + o(x) }{ x\ln{b} + o(x) } = \boxed{ \ln{a/b} }
\]

\exercisehead{13} 
\[
\lim_{x\to 1 } \frac{ (x-1) - \frac{ (x-1)^2}{2} + \frac{ (x-1)^3}{3} + o(x-1)^4 }{ (x+2)(x+1) } = \boxed{ \frac{1}{3} }
\]

\exercisehead{14} $\boxed{ 1 }$. 

\exercisehead{15} 
\[
\lim_{x\to 0} \frac{ x (e^x + 1 ) -2(e^x -1) }{ x^3 } = \lim_{x\to 0} \frac{ x (2 + x + \frac{ x^2 }{2 } ) - 2(x + x^2/2 + x^3/6)}{ x^3} = \lim_{x\to 0} \frac{ x^3 (\frac{1}{6} )}{ x^3 } = \boxed{ \frac{1}{6} }
\]

\exercisehead{16} 
\[
\lim_{x\to 0} \frac{ \ln{ (1+x)} -x }{ 1 -\cos{x}} = \lim_{x\to 0} \frac{ x - \frac{ x^2 }{2} + \frac{x^3}{3} +o(x^3) -x}{ x^3/2 } = \boxed{ -1 }
\]

\exercisehead{17} 
\[
\lim_{x\to \pi/2 } \frac{ \cos{x}}{ x - \frac{\pi}{2}} = \frac{ 0 + -1(x-\frac{\pi}{2} ) }{ x - \frac{\pi}{2} } = \boxed{ -1 } 
\]

\exercisehead{18} $ \boxed{ 1/6} $

\exercisehead{19} 
\[
\begin{aligned}
  \lim_{x\to 0 } \frac{ \cosh{x} - \cos{x}}{ x^2 } & =   \lim_{x\to 0 } \frac{ \frac{e^x + e^{-x}}{2} - \cos{x}}{ x^2 } = \\
  & =   \lim_{x\to 0 } \frac{ 1 + x + \frac{ x^2 }{2} - 2\left(1-\frac{x^2}{2} \right) + o(x^3) }{ x^2 } = \boxed{ 2 } 
\end{aligned}
\]

\exercisehead{20} 
\[
\begin{aligned}
  \lim_{x\to 0 } \frac{ 3\tan{3x} - 12 \tan{x} }{ 3 \sin{4x} - 12 \sin{x} } & =   \lim_{x\to 0 } \frac{ (4x) + \frac{ (4x)^3}{3} - 4 \left( x + \frac{x^3}{3} \right) + o(x^4) }{ (4x) - \frac{ (4x)^3 }{ 3!} - 4 \left( x - \frac{x^3}{3!} \right) + o(x^4) } = \\
  & = \frac{ 4^3 - 4}{ \frac{ -4^3 + 4}{ 2 } } = \boxed{ -2 } 
\end{aligned}
\]

\exercisehead{21}
\[
\begin{aligned}
    \lim_{x\to 0 } \frac{a^x - a^{sin{x}} }{ x^3 } & =   \lim_{x\to 0 } \frac{e^{x\ln{a}} - e^{\sin{x} \ln{a}}}{ x^3} = \\
    & =   \lim_{x\to 0 } \frac{ 1 + x \ln{a} + \frac{ (x \ln{a})^2 }{ 2! }  + \frac{ (x\ln{a})^3 }{ 6 } - \left( 1 + \sin{x}\ln{a} + \frac{ \sin^2{x} (\ln{a})^2 }{ 2 } + \frac{ \sin^3{x} (\ln{a})^3 }{ 3! } \right) + o(x^4) }{ x^3 } = \\
    & =   \lim_{x\to 0 } \frac{ \ln{a}(x - \left( x - \frac{x^3}{3!} \right) ) + \frac{ (x^2 \ln{a})^2 }{2!} - \frac{ (\ln{a})^2}{ 2 } (x^2 ) + \frac{ (ln{a})^3}{6} (x^3 - x^3 ) + o(x^4) }{ x^3 } = \boxed{ \frac{\ln{a}}{6} }
\end{aligned}
\]

\exercisehead{22} 
\[
\begin{aligned}
    \lim_{x\to 0 } \frac{ \cos{\sin{x}} - \cos{x}}{ x^4} & =   \lim_{x\to 0 } \frac{ 1 - \frac{\sin^2{x}}{2!} + \frac{ \sin^4{x}}{ 4!} + o(x^5) - \left( 1 - \frac{x^2}{2!} + \frac{x^4}{4!} \right) }{ x^4} = \\
    & =   \lim_{x\to 0 } \frac{ \frac{1}{2} \left( x^2 - (x^2 - \frac{x^4}{3} ) \right) +\left( \frac{x^4 - x^4}{4! }\right) + o(x^5) }{ x^4} = \boxed{ \frac{1}{6} }
\end{aligned}
\]

\exercisehead{23} 
\[
\lim{x\to 1 } x^{\frac{1}{1-x} } = \lim{x\to 1 } e^{\frac{1}{1-x} \ln{x}} = \exp{ \left( \lim{x\to 1 } \frac{\ln{x}}{ 1-x} \right) } = \exp{ \left( \lim{x\to 1 } \frac{ (x-1) + o(x-1)^2 }{ 1-x} \right) } = \boxed{ e^{-1} }
\]

\exercisehead{24}
\[
\begin{aligned}
  \lim_{x\to 0} (x+e^{2x})^{1/x} & = \exp{   \lim_{x\to 0} \frac{1}{x} \ln{ (x+e^{2x} ) }} \\
    &  \begin{aligned} \lim_{x\to 0} \frac{ \ln{(x+e^{2x})}}{ x } & =   \lim_{x\to 0} \frac{ \ln{ (1+x+e^{2x} - 1 )}}{ x } = \\
	 & =  \lim_{x\to 0} \frac{ x + e^{2x} - 1 + o(x^2 )}{ x}  = \lim_{x\to 0} \frac{ 3x + o(x^2)}{ x} = 3 \end{aligned}
    \\
  \Longrightarrow    \lim_{x\to 0} (x+e^{2x})^{1/x}  & = e^{3}
\end{aligned}
\]

\exercisehead{25} 
\[
\begin{aligned}
    \lim_{x\to 0} \frac{ (1+x)^{1/x} - e }{ x } & =   \lim_{x\to 0} \frac{ e^{\frac{1}{x} \ln{(1+x)}} - e }{ x } =   \lim_{x\to 0} \frac{ e^{ \frac{ x + -\frac{x^2}{2} + o(x^2)}{ x } } - e }{ x } = \\
    & =   \lim_{x\to 0} \frac{e^{1- \frac{x}{2} + o(x) } - e}{x } =   \lim_{x\to 0} \frac{ e(1+ -\frac{x}{2} + o(x)) - e }{ x } = \frac{-e}{2}  
\end{aligned}
\]

\exercisehead{26}
\[
  \lim_{x\to 0} \left( \frac{ (1+x)^{1/x}}{ e} \right)^{1/x} =   \lim_{x\to 0} \left( \exp{ \left( \frac{1}{x} \ln{ (1+x)} \right) } - 1 \right)^{1/x} =   \lim_{x\to 0} e^{ \frac{ x - \frac{x^2}{2} + o(x^3) - x }{ x^2 } } = e^{-1/2}
\]

\exercisehead{27}
\[
\begin{aligned}
  & \begin{aligned}
    (\arcsin{x})' & = \frac{1}{ \sqrt{1-x^2 }} \\
    (\arcsin{x})'' & = \frac{x}{ (1-x^2)^{3/2} } \\
    (\arcsin{x})''' & = \frac{1}{ (1-x^2)^{3/2} } + \frac{ 3x^2 }{ (1-x^2 )^{5/2} } 
\end{aligned} \\
  \exp{ \left(   \lim_{x\to 0} \frac{1}{x^2} \left( \ln{ \left( \frac{\arcsin{x}}{ x } \right) } \right)\right)}  &= \exp{   \lim_{x\to 0} \frac{1}{x^2} \ln{ \left( 1 + \left( \frac{ \arcsin{x}}{x}  -1 \right) \right) }} = \\
      =  \exp{ \left(   \lim_{x\to 0} \frac{1}{x^2 } \ln{ \left( 1 + \frac{ x + x^3/6 + o(x^4) - x }{ x } \right) } \right) } & = \exp{ \left(   \lim_{x\to 0} \frac{1}{x^2} \ln{ \left( 1 + \frac{x^2}{6} + o(x^3) \right) } \right) } = \\
      & = \exp{ \left(   \lim_{x\to 0} \frac{1}{x^2} \left( \frac{ x^2 + o(x^3)}{ 6 } \right) \right) } = \boxed{ e^{1/6} }
\end{aligned}
\]

\exercisehead{28} $\lim_{x\to 0 } \left( \frac{1}{x} - \frac{ 1}{ e^x - 1 } \right) = \lim_{x\to 0 }\left( \frac{ e^x - 1 -x }{ x (e^x - 1 ) } \right) = \lim_{x\to 0 }\frac{  \frac{ x^2}{2} + o(x^3) }{ x^2 + o(x^3)}  = \boxed{\frac{1}{2}} $


\exercisehead{29} 
\[
\begin{aligned}
  \lim_{x\to 1 } \left( \frac{1}{ \log{x}} - \frac{1}{x-1} \right) & =   \lim_{x\to 1 } \left( \frac{ (x-1)-\log{x}}{ (x-1)\log{x} } \right) = \\
  & =   \lim_{x\to 1 } \frac{ (x-1) - ((x-1)-\frac{ (x-1)^2}{ 2 } + o(x-1)^3 ) }{ (x-1)((x-1) + o(x-1)^2 ) } = \boxed{ \frac{1}{2} } 
\end{aligned}
\]

\exercisehead{30}
\[
\begin{gathered}
    \lim_{x\to 0} \frac{e^{ax} - e^x - x }{ x^2 } =     \lim_{x\to 0} \frac{ 1 + ax + \frac{(ax)^2}{2} + o(x^3) - 1 -x - \frac{x^2}{2} - x }{ x^2 }  \\
    \text{ if $a=2$, the limit is $\boxed{2}$ }
\end{gathered}
\]

\exercisehead{31} 
\begin{enumerate}
\item Prove $\int_0^x f(t) dt = o\left( \int_0^x g(t) dt \right) $ as $x\to 0$, given $f(x) = o(g(x))$.  \medskip \\
Consider $\lim_{x\to 0} \frac{ \int_0^x f(t) dt }{ \int_0^x g(t) dt } $.  

Since $f,g$ have derivatives in some interval containing $0$, $f,g$ continuous and differentiable for $|t| \leq x $.  

\[
    \lim_{x\to 0} \frac{ \int_0^x f(t) dt }{     \lim_{x\to 0} g(t)dt } =     \lim_{x\to 0} \frac{ \left( \frac{ A(x) - A(0)}{ x} \right) }{ \left( \frac{ B(x) - B(0)}{ x } \right) } = \frac{ f(0)}{g(0)} = \frac{     \lim_{x\to 0} f(x)}{     \lim_{x\to 0}g(x) } = 0  
\]
We can do the second to last step since $f,g$ have derivatives at $0$ and thus are continuous about $0$.  
\item Consider $\lim_{x\to 0} \frac{x}{e^x} = 0 $.  However, $    \lim_{x\to 0} \frac{1}{e^x } = 1 $.
\end{enumerate}

\exercisehead{32} 
\begin{enumerate}
\item Use long division to find that 
\[
\frac{1}{1+g(x)} = 1 - g(x) + g^2(x) + \frac{ -g^3(x)}{ 1+ g(x) } = 1 - g(x) + g^2(x) + o(g^2(x)) 
\]
\item
\end{enumerate}

\exercisehead{33} Given $\lim_{x\to 0} \left( 1 + x + \frac{f(x)}{x} \right)^{1/x} = e^3$, \textbf{ use the hint.}  
\[
\lim_{x\to 0} g(x) = A, \quad \text{ then $G(x) = A + o(1) $ as $x\to 0 $} 
\]
Then
\[
\begin{gathered}
  g(x) = e^3 + o(1) = \left( 1 + x + \frac{f}{x} \right)^{1/x} \\
  x \left( e^3 + o(1) \right)^x = x + x^2 + f(x) \, \Longrightarrow \boxed{ f(0) = 0 }   
\end{gathered}
\]
\[
\begin{gathered}
  x \exp{ x \ln{ (e^3 + o(1) )} } = x + x^2 + f(x) \\
  1 \exp{ x \ln{ (e^3 + o(1) )} } + x \exp{ x \ln{ (e^3 + o(1) )} } \left( \ln{e^3 + o(1)} + \frac{ x }{ e^3 + o(1)} (o'(1)) \right) = 1 + 2x + f'(0) \\
  1 + 3(0) + 0 = 1 + 0 + f'(0) \, \Longrightarrow f'(0) = 0 
\end{gathered}
\]
We need to assume that in general $o(1) = x + kx^2 + o(x^2)$.  
\begin{multline*}
  2 \exp{ x \ln{ (e^3 + o(1) )} } \left( \ln{ (e^3 + o(1) ) } + \frac{ x}{ e^3 + o(1)} (o'(1)) \right) + \\ 
  + x \exp{ x \ln{ (e^3 + o(1) )} } \left( \ln{ (e^3 + o(1))} + \frac{x}{e^3 + o(1)} o'(1) \right) + \\
  x\exp{ x \ln{ (e^3 + o(1) )} } \left( \frac{2 o'(1)}{ (e^3 + o(1)) } + \frac{ x o''(1)}{ e^3 + o(1) } + - \frac{x}{ (e^3 + o(1)) } (o'(1))^2 \right) = 2 + f''(x)  
\end{multline*}
\[
\xrightarrow{x\to 0} 2(3) = 2 + f'(0) \Longrightarrow f'(0) = 4 
\]

To evaluate $\lim_{x \to 0 } \left( 1 + \frac{f(x)}{ x } \right)^{1/x}$, consider a Taylor series expansion of $f$.  
\[
\begin{aligned}
  \lim_{x \to 0 } \left( 1 + \frac{f(x)}{ x } \right)^{1/x} & = \lim_{x \to 0 } \left( 1 + \frac{ 0 + 0 + 4 \frac{x^2}{2} + o(x^3)   }{ x } \right)^{1/x} = \lim_{x \to 0 } \left( 1 + x \left( 2 + o(x) \right)  \right)^{1/x} = \\
  & = \lim_{y\to 0} \lim_{x\to 0} \left( 1 + x \left( 2+o(y) \right) \right)^{1/x} = \lim_{y\to 0} \exp{ 2 + o(y)} = e^y
\end{aligned}
\]

%-----------------------------------%-----------------------------------%-----------------------------------
\subsection*{ 7.13 Exercises - L'Hopital's rule for the indeterminate form $0/0$ }
%-----------------------------------%-----------------------------------%-----------------------------------

\exercisehead{1} 
\[
\lim_{x\to 2 } \frac{ 3x^2 +2x -16}{ x^2 -x-2} = \lim_{x\to 2 } \frac{ (3x+8)(x-2)}{ (x-2)(x+1)} = \boxed{ \frac{14}{3} }
\]

\exercisehead{2}
\[
\lim_{x\to 3} \frac{ x^2 -4x + 3}{ 2x^2 - 13 x + 21 } = \lim_{x\to 3} \frac{ (x-3)(x-1)}{ (2x -1)(x-3) } = \boxed{ -2 }
\]

\exercisehead{3} 
\[
\lim_{x\to 0} \frac{ \sinh{x} - \sin{x}}{ x^3} = \frac{0}{0} = \lim_{x\to 0} \left( \frac{ \cosh{x} - \cos{x}}{ 3x^2 } \right) = \frac{0}{0} = \lim_{x\to 0} \left( \frac{ \sinh{x} + \sin{x}}{6x} \right) = \lim_{x\to 0} \left( \frac{ \cosh{x} + \cos{x}}{ 6} \right) = \boxed{ \frac{1}{3} }
\]

\exercisehead{4} 
\[
\lim_{x\to 0} \frac{ (2-x)e^x - x -2 }{ x^3 } = \lim_{x\to 0} \frac{ -e^x + (2-x) -1 }{ 3x^2 } = \lim_{x\to 0} \frac{ (2-x)(1+ x + x^2/2 + x^3/6 + o(x^3) ) }{ x^3 } = \boxed{ -\frac{1}{6} }
\]

\exercisehead{5} 
\[
\lim_{x\to 0} \frac{ \log{ (\cos{ax} )} }{ \log{ (\cos{bx} ) } } = \lim_{x\to 0} \frac{ - \cos{bx} \sin{ax} a }{ -\cos{ax} \sin{bx} b } = \lim_{x\to 0} \frac{ \cos{ax} a^2}{ \sin{bx} b^2 } = \frac{ a^2}{b^2}
\]

\exercisehead{6} When it doubt, \textbf{ Taylor expand}.  
\[
\begin{aligned}
  \lim_{x\to 0^+} \frac{ x- \sin{x}}{ (x\sin{x})^{3/2} } & = \lim_{x\to 0^+} \frac{ 1 - \cos{x}}{ \frac{3}{2} (x\sin{x})^{1/2} ( \sin{x} + x \cos{x}) } = \frac{2}{3} \lim_{ x\to 0^+} \frac{ 1 - (1- \frac{x^2}{2} + \frac{x^4}{24} + o(x^4) ) }{ e^{\frac{1}{2} \ln{ (x\sin{x}) } } (x - \frac{x^3}{6} + o(x^3) + x - \frac{x^3}{2} + o(x^3) ) } = \\
  & = \frac{2}{3} \lim_{x\to 0^+} \left( \frac{ \frac{x^2}{2} - \frac{x^4}{24} + o(x^4) }{ \sqrt{ x (x+ \frac{x^3}{6} + o(x^3) ) } ( 2x - \frac{2}{3} x^3 + o(x^3) ) } \right) =  \\
  & = \frac{2}{3} \lim_{ x\to 0^+} \left( \frac{ \frac{1}{2} - \frac{x^2}{24} + o(x^2) }{ \sqrt{ 1 + \frac{x^2}{6} + o(x^2) } (2 + \frac{-2}{3} x^2 + o(x^2) ) } \right)  = \frac{2}{3} \lim_{x\to 0^+} \left( \frac{ 1/2}{2} \right) = \boxed{ \frac{1}{6} }
\end{aligned}
\]
Notice in the third step how in general we deal with powers, $(x\sin{x})^{1/2}$, is to convert it into exponential form, $e^{\frac{1}{2} \ln{ (x\sin{x}) }}$, but it wasn't necessary.  

\exercisehead{7} Do L'Hopital's first.  
\[
\begin{aligned}
\lim_{x\to a^+} \frac{ \sqrt{x} - \sqrt{a} + \sqrt{x-a}}{ \sqrt{ x^2 -a^2 } } &= \lim_{x\to a^+} \frac{ \frac{ 1}{ 2 \sqrt{x}} + \frac{ 1 }{ 2 \sqrt{ x-a}} }{ \frac{ x }{ \sqrt{ x^2 -a^2 }} } = \frac{1}{2} \lim_{x\to a^+} \frac{ \sqrt{ x^2 -a^2}}{ x^{3/2}} + \frac{ \sqrt{ x+a} }{x } = \\
& = \frac{1}{2} \lim_{x\to a^+} \frac{ \sqrt{ x^2 -a^2 } + x^{1/2} \sqrt{ x +a} }{ x^{3/2}} = \frac{1}{2} \frac{ \sqrt{2} a }{ a^{3/2} } = \frac{ \sqrt{2}}{ 2 \sqrt{a}} 
\end{aligned}
\]

\exercisehead{8} Do L'Hopital's at the second step.  
\[
\begin{aligned}
  \lim_{x\to 1^+} \frac{ \exp{ (x \ln{x}) } -x }{ 1 - x + \ln{x} } & =  \lim_{x\to 1^+} \frac{ \exp{ x \ln{x}} (\ln{x} +1 ) -1}{ -1 + \frac{1}{x} } =  \lim_{x\to 1^+} \frac{ \exp{ (x\ln{x})} (\ln{x} +1)^2 + \frac{1}{x} \exp{ (x \ln{x} ) } }{ -1/x^2 } = \\
  & =  \lim_{x\to 1^+} x^2 \exp{ (x \ln{x} )} (\ln{x} +1)^2 + x \exp{ (x\ln{x}) } = 1 +1 = \boxed{ 2 }
\end{aligned}
\]

\exercisehead{9} Keep doing L'Hopital's. 
\[
\begin{gathered}
   \lim_{x\to 0} \frac{ \arcsin{2x} - 2 \arcsin{x} }{x^3}  =    \lim_{x\to 0} \frac{ \frac{ 2}{ \sqrt{ 1 - (2x)^3 } } - 2 \frac{ 1}{ \sqrt{ 1-x^2 } } }{ 3x^2 } = \\ 
   = \frac{2}{3} \lim_{x\to 0} \frac{ -\frac{1}{2} (1-(2x)^2 )^{-3/2} (-8x) - (-\frac{1}{2} )( 1 - x^2 )^{-3/2} (-2x) }{ 6x } = \\
    = \frac{1}{9}    \lim_{x\to 0} \frac{ 4 (1-(2x)^2 )^{-3/2} + 4x (- \frac{3}{2} )(1-(2x)^2 )^{-5/2} (-8x ) + -(1-x^2)^{-3/2} - x(-\frac{3}{2}) (1-x^2)^{-5/2}(-2x) }{1} = \\
    = \frac{1}{9}\frac{ 4-1}{1} = \boxed{ \frac{1}{3} }  
\end{gathered}
\]

\exercisehead{10} 
\[
\begin{aligned}
  \lim_{x\to 0} \frac{ x \cot{x} -1 }{ x^2 } & =    \lim_{x\to 0} \frac{ x \cos{x} - \sin{x}}{ x^2 \sin{x}} =    \lim_{x\to 0} \frac{ \cos{x} - x \sin{x} - \cos{x}}{ 2x \sin{x} + x^2 \cos{x} } =    \lim_{x\to 0} \frac{ - \sin{x}}{ 2 \sin{x} + x \cos{x} } = \\ 
  & = - \lim_{x\to 0} \frac{ \cos{x}}{ 2 \cos{x} + \cos{x} + -x \sin{x} } = - \lim_{x\to 0} \frac{ \cos{x}}{ 3 \cos{x} - x \sin{x} } = \boxed{ \frac{1}{3} }
\end{aligned}
\]

\exercisehead{11} 
\[
\lim_{x\to 1} \frac{ \sum_{k=1}^n x^k - n }{ x -1} = \lim_{x\to 1} \frac{ \sum_{k=1}^n kx^{k-1} }{ 1 } = \sum_{k=1}^n k = \frac{ n (n+1)}{2} 
\]

\exercisehead{12} 
\[
\begin{gathered}
    \lim_{x\to 0+} \frac{ 1 }{ x \sqrt{x}} \left( a \arctan{ \frac{ \sqrt{x}}{ a } } - b \arctan{ \frac{ \sqrt{x}}{ b } } \right)  =  \lim_{x\to 0+} \frac{ \left( a \frac{ 1 }{ 1 + \frac{x}{a^2 } } \left( \frac{1}{ 2 a \sqrt{x} } \right) - b \frac{1}{ 1 + \frac{ x }{b^2 } } \left( \frac{1}{ 2b\sqrt{x}} \right) \right)}{ \frac{3}{2} x^{1/2} } = \\
    = \frac{1}{3}  \lim_{x\to 0+} \frac{ a^2}{ a^2 + x } \left( \frac{1}{x} \right) - \frac{ b^2 }{ b^2 + x } \left( \frac{1}{x } \right) =  \frac{1}{3} \lim_{x\to 0+} \frac{ (b^2 + x)a^2 - b^2 (a^2 + x )}{ (a^2 + x )(x+b^2)x } = \\ 
    = \frac{1}{3}  \lim_{x\to 0+} \frac{ (a^2-b^2)}{ (a^2 -x)(b^2 +x) } = \boxed{ \frac{1}{3} \frac{ a^2 -b^2 }{ a^2 b^2 } }
\end{gathered}
\]

\exercisehead{13} 
\[
\begin{aligned}
  \frac{ (\sin{4x} )( \sin{3x} ) }{ x \sin{2x} } & = \frac{ (2\sin{2x} \cos{2x} )(\sin{3x} ) }{ x \sin{2x } } = \frac{ 2 ( -2\sin{2x} \sin{3x} + \cos{2x} 3 \cos{3x} )}{ 1} = \boxed{ 6 } \text{ as } x \to 0
\text{ otherwise } & \\
& \frac{ 2 \cos{2x} \sin{3x} }{ x } \to \frac{4}{\pi} \text{ as } x \to \frac{\pi}{2} 
\end{aligned}
\]
We used L'Hopital's at the second to last step for $x \to 0$.  

\exercisehead{14} 
\[
\begin{gathered}
  \lim_{x\to 0} (x^{-3} \sin{3x} + ax^{-2} +b ) = 0 \\
  \lim_{x\to 0} \frac{ \sin{3x} + ax + bx^3}{ x^3} = \frac{ 3 \cos{3x} + a + 3 bx^2 }{ 3x^2 } = \frac{ -9 \sin{3x} + 6bx }{ 6x } = \frac{ -27 \cos{3x} + 6b }{ 6 } = \frac{ -27 +6b }{ 6 } = 0 
\end{gathered}
\]
So $\boxed{ b = \frac{9}{2}, \, a = -3 }$.  

\exercisehead{15} 
\[
\begin{gathered}
  \lim_{x\to 0} \frac{ 1 }{ bx - \sin{x} } \int_0^x \frac{ t^2 dt }{ \sqrt{ a+t }} = \frac{ \frac{ x^2 }{ \sqrt{ a+x } } }{ b- \cos{x} } = \frac{ 2x }{ \frac{ 1 }{ 2 \sqrt{ a + x } } (1- \cos{x}) + \sqrt{ a+x} (\sin{x} ) } = \frac{ 2 x \sqrt{ a +x } }{ \frac{1}{2} (1-\cos{x}) + (a+x) \sin{x} } = \\
  = 2  \lim_{x\to 0} \frac{ \sqrt{ a+x} + \frac{ x }{ 2 \sqrt{ a + x }} }{ \frac{ \sin{x}}{2} + \sin{x} + (a+x) \cos{x} } = 2 \frac{ \sqrt{a}}{ a } = 1 \\
  \Longrightarrow \boxed{ a =4 }
\end{gathered}
\]
Note that we had dropped the limit notation in some earlier steps and applied L'Hopital's a number of times, and we also rearranged the denominator and numerator cleverly at each step.  

\exercisehead{16}
\begin{enumerate}
\item
\[
\begin{gathered}
  \text{ angle $ABC$ is } \frac{x}{2} , \, \text{ length $BC$ is } \tan{ \frac{x}{2}} \\
  2\tan{ \frac{x}{2}} \cos{ \frac{x}{2} } = 2 \sin{ \frac{x}{2} } \text{ is the base length of $ABC$ } \\
  \tan{ \frac{x}{2}} \sin{ \frac{x}{2} } \text{ is the height of triangle $ABC$ }  \\ 
  \Longrightarrow T(x) = \tan{ \frac{x}{2} } \sin^2{ \frac{x}{2} } = \frac{ 1 - \cos^2{ \frac{x}{2}}}{ \cos{ \frac{x}{2}} } \sin{ \frac{x}{2} } = \tan{ \frac{x}{2} } - \frac{1}{2} \sin{x}  
\end{gathered}
\]
\item \[
S(x) = \left( \frac{x}{ 2\pi } \right) (\pi (1)) - \frac{1}{2} \cos{ \frac{x}{2}} (2\sin{ \frac{x}{2} } ) = \frac{x}{2} - \frac{ \sin{2}}{ 2 } 
\]
\item Use L'Hopital's theorem.  
\[
\begin{aligned}
  \frac{ T(x)}{ S(x) } & = \frac{ \tan{\frac{x}{2} } - \frac{1}{2} \sin{x} }{ \frac{ x - \sin{x}}{2 } } \xrightarrow{\frac{d}{dx}} \frac{ \frac{1}{2} \sec^2{ \frac{x}{2} } - \frac{ \cos{x}}{2} }{ \frac{1-\cos{x}}{2} } \\ 
  & \xrightarrow{ \frac{d}{dx} } \frac{ \sec^2{ \frac{x}{2}} \tan{ \frac{x}{2}} + \sin{x}}{ \sin{x}} \xrightarrow{ \frac{d}{dx}} \frac{ \sec^2{\frac{x}{2}} \tan^2{\frac{x}{2}} + \frac{1}{2} \sec^4{ \frac{x}{2}} + \cos{x} }{ \cos{x}} \to \\
  & \xrightarrow{ x \to 0} \frac{3}{2} 
\end{aligned}
\]
\end{enumerate}

\exercisehead{17} Use L'Hopital's rule. 
\[
\begin{gathered}
I(t) = \frac{E}{R} ( 1 - e^{- \frac{Rt}{L} } ) \\
\lim_{R \to 0} I(t) = \lim_{R \to 0} \frac{ E(-1) (e^{-Rt/L } ) \left( \frac{ -t }{L} \right) }{ 1 } = \boxed{ \frac{Et}{L} } 
\end{gathered}
\]


\exercisehead{18}
\[
\begin{gathered}
  \begin{aligned}
    c- k & \to 0 \text{ since } c \to k \\
    k-c & = u \\
    k-u & =c 
\end{aligned} \\
  \begin{aligned}
    f(t) & = \frac{ A (\sin{ kt } - \sin{ ct } ) }{ c^2 - k^2 } = \frac{ A ( \sin{(kt)} - \sin{ (k-u) t } ) }{ -u (2k-u)} = \\
    & = \frac{ A (\cos{(k-u)t }) (t) }{ -(2k-u) + u} \to \frac{ -At \cos{kt}}{ 2k }
\end{aligned}
\end{gathered}
\]

%-----------------------------------%-----------------------------------%-----------------------------------
\subsection*{ 7.17 Exercises - The symbols $+\infty$ and $-\infty$.  Extension of L'Hopital's rule; Infinite limits; The behavior of $\log{x}$ and $e^x$ for large $x$ }
%-----------------------------------%-----------------------------------%-----------------------------------

\quad \\

\exercisehead{15} Use L'Hopital's at the second to last step. 
\[
\lim_{x \to 1^{-1}} (\ln{x}) (\ln{ (1-x)}) = \lim_{x \to 1^{-1}} \frac{ \ln{(1-x)}}{ \frac{1}{\ln{x}} } = \lim_{x \to 1^{-1}} \frac{ \left( \frac{1}{1-x} \right)(-1) }{ \frac{-1}{ (\ln{x})^2 } \left( \frac{1}{x} \right)} = \lim_{x \to 1^{-1}} \frac{x}{1-x} (\ln{x})^2 = \lim_{x \to 1^{-1}} \frac{ 2 \ln{x} \left( \frac{1}{x} \right)}{ (-1) } = \boxed{ 0}  
\]

\exercisehead{16} Persist in using L'Hopital's and trying all possibilities systematically.  
\[
\begin{aligned}
  \lim_{x \to 0^+} x^{x^x -1} &  =  \lim_{x \to 0^+} e^{ (x^x-1) \ln{x} } =  \lim_{x \to 0^+} e^{ (e^{x \ln{x} } -1 ) \ln{x} } = \exp{  \lim_{x \to 0^+} \frac{ \ln{x}}{ (e^{x\ln{x}}-1)^{-1} } } = \\
  & = \exp{  \lim_{x \to 0^+} \frac{ 1/x}{ (-1) (e^{x\ln{x}} -1)^{-2} (e^{x\ln{x}} (\ln{x}+1) ) }} = \exp{ - \lim_{x \to 0^+} \left( \frac{ e^{2 x \ln{x}  } - 2 e^{x\ln{x}} + 1}{ e^{x\ln{x}} (x\ln{x}+x)}  \right)}  = \\
      & = exp{ -  \lim_{x \to 0^+} \frac{ e^{x\ln{x} } (\ln{x} +1) + e^{ -x\ln{x}} (-\ln{x} -1)}{ (\ln{x} +1 +1 )}} = \exp{ -  \lim_{x \to 0^+} \frac{ e^{x\ln{x}} - e^{-x\ln{x}}}{ 1 + \frac{ 1}{ \ln{x}+1} } } = \boxed{1}
\end{aligned}
\]

\exercisehead{17} 
\[
\begin{aligned}
  \lim_{x \to 0^+} (x^{x^x}-1) &=   \lim_{x \to 0^+} \left( e^{x^x \ln{x}} -1 \right) =   \lim_{x \to 0^+} \left( e^{e^{x \ln{x} }\ln{x} } -1 \right) = \left( e^{  \lim_{x \to 0^+} e^{x \ln{x}} \ln{x} } -1 \right) = \\
  & = e^{e^{  \lim_{x \to 0^+} x \ln{x} }   \lim_{x \to 0^+} \ln{x} } -1 = 0 -1 = -1
\end{aligned}
\]
We used 
\[
  \lim_{x \to 0^+} x^{\alpha} \log{x} = 0 \, \forall \alpha > 0 
\]
since \medskip \\
\quad $t = \frac{1}{x}, \, x^{\alpha} \log{x} = \frac{ -\log{t}}{ t^{\alpha}} \to 0$ as $t\to \infty$.

\exercisehead{18} 
\[
\begin{aligned}
  \lim_{x\to 0^-} e^{\sin{x} \ln{(1- 2^x) } } & = \exp{ \left(   \lim_{x \to 0^-} \frac{ \ln{ (1-e^{x\ln{2}} )}}{ 1/\sin{x}} \right)} = \exp{ \left(   \lim_{x \to 0^-} \frac{ \frac{1}{1-e^{x\ln{2}} } \left( -\ln{2} e^{x\ln{2}} \right) }{ \frac{-1}{\sin^2{x}} \cos{x} } \right) } = \\
  & = \exp{ \left(   \lim_{x \to 0^-} \frac{ (\sin^2{x}) \ln{2} e^{x\ln{2}} }{ (1-e^{x\ln{2}} )\cos{x}} \right)} = \exp{ \left( (\ln{2})   \lim_{x \to 0^-} \frac{ 2 \sin{x} \cos{x}}{ -\ln{2} e^{x\ln{2}} } \right) } = \boxed{1}
\end{aligned}
\]

\exercisehead{19} \[
\lim_{x\to 0^+ } e^{\frac{1}{\ln{x} } \ln{x}} = \boxed{e}
\]

\exercisehead{20} At the end, L'Hopital's could be used to verify that indeed $\sin{x} \ln{ \sin{x}} \to 0$ as $x\to 0$.  
\[
\lim_{x \to 0^+ } e^{ \sin{x} \ln{ \cot{x}} } = e^{ \lim_{x \to 0^+ } \sin{x} (\ln{\cos{x}} - \ln{\sin{x}} ) } = e^{ \lim_{x \to 0^+ } -\sin{x} \ln{\sin{x} } } = \boxed{1}
\]

\exercisehead{21} Rewrite $\tan$ into $\sin$ and $\cos$ and use L'Hopital's.  
\[
\begin{aligned}
& \lim_{x \to \frac{\pi}{4} } (\tan{x})^{\tan{2x}} = \lim_{x \to \frac{\pi}{4} } e^{\tan{2x} \ln{\tan{x}} } = \exp{ \lim_{x \to \frac{\pi}{4} } \frac{1}{\cos{2x}} ( \ln{ \sin{x} } - \ln{ \cos{x} } ) } = \\
  & = \exp{ \lim_{x \to \frac{\pi}{4} } \frac{ \frac{1}{\sin{x}} \cos{x} - \frac{ 1}{ \cos{x} } (-\sin{x}) }{ -2\sin{2x} \cos{2x} } } = \exp{ \lim_{x \to \frac{\pi}{4} } \frac{1}{ -\sin^2{2 x } } } = \boxed{ e^{-1} } 
\end{aligned}
\]

\exercisehead{22} 
%\[
%\lim_{x \to 0^+} \exp{ 
%\]

\exercisehead{23} Use L'Hopital's theorem, taking derivatives of top and bottom.  
\[
\lim_{x\to 0^+} \exp{ \frac{ e}{ 1 +\ln{x}} \ln{x} } = \exp{ e \lim_{x \to 0} \frac{ \ln{x}}{ 1 + \ln{x} } } = \exp{ e\lim_{x \to 0 } \frac{ 1/x}{1/x} } = e^e
\]

\exercisehead{24} Rewrite $\tan$ into $\sin$ and $\cos$ and take out $\sin$ since we could do the limit before doing L'Hopital's.  
\[
\lim_{x\to 1 } (2-x)^{\tan{ (\pi x/2)}} = \lim_{x\to 1 } e^{\tan{ \frac{ \pi x}{2} } \ln{ (2-x)} } = e^{ \lim_{x \to 1 } \frac{ \sin{ \pi x /2 } \ln{ (2-x)} }{ \cos{ \pi x/2} } } = \exp{ \lim_{x\to 1} \frac{ \frac{ (-1)}{ 2-x} }{ \frac{\pi}{2} \sin{ \pi x/2} }} = \boxed{ \exp{ \frac{ -2}{\pi} } }
\]

\exercisehead{26}
\[
\begin{aligned}
  \lim_{x \to +\infty} \exp{ \left( x \ln{ \left( \frac{x+c}{x-c} \right) } \right) } & =  \exp{  \lim_{x \to +\infty} \frac{ \ln{ \left( \frac{ 1 + c/x }{ 1-c/x} \right) } }{ 1/x} } = \exp{  \lim_{x \to +\infty} \frac{ \frac{ 1}{ \left( \frac{ 1 + c/x}{ 1 -c/x} \right)} \left( \frac{ (x-c)-(x+c)}{ (x-c)^2} \right)}{ \frac{-1}{x^2} } } = \\
  & = \exp{ (2c)} = 4 \Longrightarrow \boxed{ c = \ln{2} }
\end{aligned}
\]

\exercisehead{27}
\[
(1+x)^c = \exp{ (c \ln{(1+x)} ) } = \exp{ (c (x-o(x)))} = 1 + c(x- o(x))+ o(x-o(x)) = 1 + cx + o(x)
\]
\[
x^2 \left( 1+ \frac{1}{x^2} \right)^{1/2} - x^2 = x^2 \left( 1 + \frac{1}{x^2} \right)^{1/2} -x^2 
\]
Let $x^2 = \frac{1}{t}$.  So $t\to 0$ as $x \to +\infty$.  
\[
\Longrightarrow \frac{ (1+t)^{1/2} -1}{t} = \frac{ 1+ \frac{1}{2} t - 1 + o(t)}{t} = \boxed{ \frac{1}{2} }
\]

\exercisehead{28}
\[
(x^5 + 7x^4 + 2)^c - x = x^5 \left( 1 + \frac{7}{x} + \frac{2}{x^5} \right)^c -x
\]
Let $\frac{1}{t} = x$ and guess that $c= \frac{1}{5}$
\[
\begin{aligned}
  \left( \frac{1}{t^5} + \frac{7}{t^4} +2 \right)^c -\frac{1}{t} & = \left( 1 + (7t + 2t^5) \right)^{1/5}/t - \frac{1}{t} = \\
  & = \frac{ 1 + \frac{1}{5} (7t + 2t^5) + o(t) -1 }{ t} = \boxed{ \frac{7}{5} } 
\end{aligned}
\]

\exercisehead{29}
\[
\begin{aligned}
  g(x) & = xe^{x^2}  \\
  g'(x) & = e^{x^2} + 2x^2 e^{x^2} \\ 
  g''(x) & = 2xe^{x^2} + 4x e^{x^2} + 4x^3 e^{x^2} = 6x e^{x^2} + 4x^3 e^{x^2} 
\end{aligned}
\quad 
\begin{aligned}
  f(x) & = \int_1^x g(t) \left( t+\frac{1}{t} \right) dt \\
  f'(x) & = g(x)\left( x + \frac{1}{x} \right) -g(1) 2 ; \\
  f''(x) & = g'(x) (x+1/x) + g(x) (1-1/x^2) 
\end{aligned}
\]

\[
\begin{aligned}
  \frac{ f''(x)}{ g''(x)} & = \frac{ (e^{x^2} + 2x^2 e^{x^2} )( x+1/x) + xe^{x^2} (1-\frac{1}{x^2} ) }{ 6xe^{x^2} + 4 x^3 e^{x^2} } = \frac{ 2xe^{x^2} + 2x^3 e^{x^2} + 2x e^{x^2} }{ (6xe^{x^2} + 4 x^3 e^{x^2} ) } = \\
  & = \frac{ 4x +2x^3}{6x+4x^3} =  \boxed{ \frac{1}{2} } \text{ as } x\to \infty
\end{aligned}
\]

\exercisehead{30}
\[
\begin{aligned}
g(x) &= x^c e^{2x}  \\
g'(x) & = cx^{c-1} e^{2x} + 2x^c e^{2x }
\end{aligned}
\quad 
\begin{aligned}
  f(x) & = \int_0^x e^{2t} (3t^2+1)^{1/2} dt \\
  f'(x) & = e^{2x} ( 3x^2+1)^{1/2} - 1 
 \end{aligned}
\]

Guessing that $c=1$
\[
\frac{ f'(x)}{g'(x)} = \frac{ e^{2x} (3x^2 +1)^{1/2} -1 }{ 2x^c e^{2x} + cx^{c-1} e^{2x} } = \frac{ \sqrt{3} x ( 1+ \frac{1}{3x^2} )^{1/2} -e^{-2x} }{ 2x+1} = \boxed{ \frac{ \sqrt{3}}{2} }
\]
So $\boxed{ c=1}$.  

\exercisehead{31}

\exercisehead{32} 
\begin{enumerate}
\item 
\[
P\left( 1+\frac{r}{m} \right), P\left( 1+\frac{r}{m} \right)^2, \dots P\left( 1+\frac{r}{m} \right)^m
\]
For each year, there are the just previously shown $m$ compoundings, so for $n$ years, 
\[
P\left( 1+\frac{r}{m} \right)^{mn}
\]
\item
\[
\begin{aligned}
  2 &= e^{rt} \\
  \frac{ \ln{2}}{r} = t = \boxed{ 11.55 years}
\end{aligned}
\]
\item 
\[
\begin{aligned}
  2P_0 & = P_0 \left( 1 + \frac{r}{m} \right)^{mt} \\
  \ln{2} & = mt \ln{ (1+ r/m)} \\
  t= \frac{\ln{2}}{ m \ln{ (1+r/m)} } = \frac{ \ln{2}}{ 4 \ln{ (1+ 0.06/4 )} } = \boxed{ 11.64 years }
\end{aligned}
\]
\end{enumerate}


%-----------------------------------%-----------------------------------%-----------------------------------
\subsection*{ 7.17 Exercises - The symbols $+\infty$ and $-\infty$.  Extension of L'Hopital's rule; Infinite limits; The behavior of $\log{x}$ and $e^x$ for large $x$ }
%-----------------------------------%-----------------------------------%-----------------------------------

\exercisehead{1} 
\[
\lim_{x\to 0 } \frac{ e^{-1/x^2}}{ x^{1000}} = \lim_{u\to \infty} e^{-u} u^500 = \lim_{u\to \infty } \frac{ u^{500}}{e^u} = 0 \quad \begin{aligned} u & = \frac{1}{x^2} \\ x^2 & = \frac{1}{u} \\ x^{1000} & = \frac{1}{u^500} \end{aligned}
\]

Where we had used Theorem 7.11, which are two very useful limits for $\log$ and $\exp$.
\begin{theorem}
If $a,b > 0$, 
\begin{align}
  & \lim_{x \to +\infty} \frac{ (\log{x})^b }{ x^a} = 0 \\
  & \lim_{x \to +\infty} \frac{ x^b}{ e^{ax} } = 0 
\end{align}
\end{theorem}

\begin{proof}
  Trick - use the definition of the logarithm as an integral.  

  If $c>0, t \geq 1, t^c \geq 1 \Longrightarrow t^{c-1} \geq t^{-1}$.  
\[
\begin{gathered}
  0 < \ln{x} = \int_1^x \frac{1}{t} dt \leq \int_1^x t^{c-1} dt = \frac{1}{c} (x^c -1) < \frac{ x^c}{c} \\
  \Longrightarrow 0 < \frac{ (\ln{x})^b }{ x^a } < \frac{ x^{cb-a}}{ c^b } \\
  \text{ Choose } c = \frac{a}{2b}, \frac{ x^{cb-a}}{ c^b } = \frac{ x^{-a/2}}{c^b } \to 0 \text{ as } x \to \infty  \\
  \text{ then } \frac{ (\ln{x})^b}{ x^a } \to 0 \text{ as } x \to 0
\end{gathered}
\]

For $\exp$, Let $t=e^x$.  $\ln{t} = x$.  $\frac{ x^b}{e^{ax} } = \frac{ (\ln{t})^b }{ t^a } \to 0 $ as $t \to \infty$ as $x \to \infty$.  
\end{proof}

\exercisehead{2}
\[
\lim_{x\to 0} \frac{ \sin{\frac{1}{x} }}{ \arctan{ \frac{1}{x} } } = \lim_{x\to 0} \frac{ \frac{1}{x} - o\left( \frac{1}{x} \right) }{ \frac{1}{x} - o\left( \frac{1}{x} \right) } = \boxed{ 1 }
\]

\exercisehead{3} Use L'Hopital's. 
\[
\lim_{x \to \frac{ \pi}{2} } \frac{ \tan{3x}}{ \tan{x}} = \lim_{x\to \frac{\pi}{2} } \frac{ \cos{x}}{ \cos{3x}} = \lim_{x \to \frac{\pi}{2} } \frac{ -\sin{x}}{ -3 \sin{3x}} = \boxed{ - \frac{1}{3}}
\]

\exercisehead{4} Use L'Hopital's.  
\[
\lim_{x\to \infty} \frac{ \ln{ (a+be^x)}}{ \sqrt{ a +bx^2 } } = \lim_{x\to \infty} \frac{ \frac{1}{a+be^x } (be^x ) }{ \frac{ bx }{ \sqrt{ a+bx^2 } } } = \lim_{x\to \infty} \frac{ \frac{1}{ ae^{-x} +b }}{ \frac{ 1 }{ \sqrt{ \frac{a}{x^2 } +b } } } = \frac{ 1}{\sqrt{ b}}
\]

\exercisehead{5} Make the substitution $x=\frac{1}{t}$.  
\[
\lim_{x \to \infty} x^4 \left( \cos{ \frac{1}{x} } -1 + \frac{1}{2x^2 } \right) = \lim_{t\to \infty} \frac{1}{t^4} \left( \cos{t} -1 + \frac{ t^2 }{2 } \right)  = \lim_{t\to \infty} \frac{ t^4/4! + o(t^4)}{ t^4 } = \frac{1}{4!} =\frac{1}{120}
\]

\exercisehead{6} 
\[
\lim_{x \to \pi } \frac{ \ln{ |\sin{x} | }}{ \ln{ |\sin{2x} | } } = \lim_{x\to \pi} \frac{ \frac{1}{\sin{x}} \cos{x} }{ \frac{2}{\sin{2x} } \cos{2x} } = -\frac{1}{2} \lim_{x\to \pi } \frac{ \sin{2x}}{ \sin{x}} = - \frac{1}{2} \lim_{x\to \pi} \frac{ 2\cos{2x}}{ \cos{x}} = \boxed{ 1 }
\]

\exercisehead{7} 
\[
\begin{gathered}
  \lim_{x\to \frac{1}{2}^- } \frac{ \ln{(1-2x)}}{ \tan{ \pi x }} = \lim_{x\to \frac{1}{2}^- } \frac{ \left( \frac{1}{1-2x} \right) (-2) }{ (\sec^2{\pi x } ) \pi } = \lim_{x\to \frac{1}{2}^- } \frac{ -2 (\cos^2{ \pi x } ) }{ (1-2x) \pi } = \\
  = -\frac{2}{\pi} \lim_{x\to \frac{1}{2}^- } \frac{ 2 \cos{ \pi x } \pi -\sin{ \pi x } }{ -2 } = 1 
\end{gathered}
\]

\exercisehead{8} 
\[
\lim_{x\to \infty} \frac{ \cosh{ x + 1 }}{ e^x } = \lim_{x\to \infty} \frac{ e^{x+1} + e^{-x -1 }}{ 2 e^x } = \frac{1}{2} \lim_{x \to \infty} e^1 + \frac{1}{ e^{2x +1 }} = \boxed{ \frac{e}{2} }
\]

\exercisehead{9} 
\[
\begin{gathered}
  \lim_{x\to \infty} \frac{ a^x}{x^b } = \lim_{x\to \infty} \frac{ e^{x \ln{a}}}{ x^b } \to \infty ; \, a >1 \\
  \text{ since } \lim_{x\to \infty} \frac{ x^b }{ e^{ax} } \, \text{ (in this case, $\ln{a} >0$ )}
\end{gathered}
\]

\exercisehead{10} 
\[
\lim_{x\to \frac{\pi}{2} }\frac{ \tan{x} -5 }{ \sec{x} +4} = \lim_{x\to \frac{ \pi}{2}} \frac{ \sec^2{x}}{ \tan{x} \sec{x} } = \lim_{x\to \frac{\pi}{2}} \frac{ \sec{x}}{ \tan{x}} = \lim_{x\to \frac{\pi}{2}} \frac{1}{\cos{x}} \frac{ \cos{x}}{ \sin{x}} = \boxed{ 1} 
\]

%%%%%%%%%%%%%%%%%%%%%%%%%%%%%%%%%%%%%%%%%%%%%%%%%%%%%%%%%%%%%%%%%%%%%%%%%%%%%%%%%%%%%%%%%%%%%%
\subsection*{ 8.5 Exercises - Introduction, Terminology and notation, A first-order differential equation for the exponential function, First-order linear differential equations }
%%%%%%%%%%%%%%%%%%%%%%%%%%%%%%%%%%%%%%%%%%%%%%%%%%%%%%%%%%%%%%%%%%%%%%%%%%%%%%%%%%%%%%%%%%%%%%
\quad \\
The ordinary differential equation theorems we will use are 

\begin{align}
  y' + P(x) y & = 0 \\ 
  A(x) & = \int_a^x P(t)dt \notag \\
  y = b e^{-A(x)} & 
\end{align}

Consider $y' + P(x)y = Q(x); A(x) = \int_a^x P(t)dt$.  \medskip \\
Let $h(x) = g(x) e^{A(x)}; g$ a solution.

\[
\begin{aligned}
  h'(x) & = (g'+Pg)e^A = Qe^A \\
  & \xrightarrow{ \text{ 2nd. fund. thm. of calc. } } h(x) =h(a) + \int_a^x Q(t) e^{A(t)} dt \\
  \text{ since } h(a) & = g(a) 
\end{aligned}
\]
\begin{equation}
  y = g(x) = e^{-A(x) } \left( \int_a^x Q(t) e^{A(t)} dt + b \right)
\end{equation}

We had done some of these problems previously, using an integration constant $C$, but following Apostol's notation for $y(a)= b$ for initial conditions is far more advantageous and superior as we seem clearly the dependence upon the initial conditions - so some of the solutions for the exercises will show corrections to the derived formula using Apostol's notation for $y(a)=b$ initial conditions.

\exercisehead{1} $A(x) = \int_0^x (-3)dt = -3x$
\[
y = e^{3x} \left( \int_0^x e^{2t} e^{-3t} dt + 0 \right) = e^{3x} \left. (-e^{-t}) \right|_0^x = \boxed{ -e^{2x} + e^{3x} }
\]

\exercisehead{2} $y' - \frac{2}{x} y = x^4$.  $A(x) =\int_1^x \left( \frac{-2}{t} \right) dt = -2\ln{x}$.  
\[
\begin{aligned}
  y & = e^{-A(x)} \left( \int_a^x Q(t) e^{A(t)} dt + b \right) = e^{2\ln{x} \left( \int_1^x t^4 e^{-2\ln{t}} dt + 1 \right)} = x^2 \left( \int_1^x t^2 dt +1 \right) = \\
    & = -x^2 + \frac{x^2}{3} (x^3 -1) = \frac{ x^5}{3} + \frac{2x^2}{3}
\end{aligned}
\]

\exercisehead{3} $y'+y\tan{x} = \sin{2x}$ on $\left( -\frac{\pi}{2}, \frac{\pi}{2} \right)$, with $y=2$ when $x=0$.  
\[
\begin{gathered}
  A(x) = \int_0^x P(t) dt = \int_0^x \tan{t} dt = -\left. \ln{ |\cos{t} | } \right|_0^x = -\ln{ \cos{x}} \\
  \begin{aligned}
    y & = e^{-A(x)} \left( \int_a^x Q(t) e^{A(t)} dt + b \right) = \cos{x} \left( \int_a^b \sin{2t} e^{-\ln{\cos{t}}} dt + 2 \right) = \\ 
	& = \cos{x} \left( \int_a^x 2\sin{t} dt + 2 \right) = -2\cos^2{x} + 4 \cos{x}
  \end{aligned}
\end{gathered}
\]

\exercisehead{4} $y'+xy =x^3$.  $y=0, x=0$.  
\[
\begin{gathered}
  A(x) = \frac{1}{2}x^2 \\
\begin{aligned}
  y &= e^{-\frac{1}{2} x^2 }\left( \int_0^x t^3 e^{ \frac{t^2}{2} }dt + 0 \right) = e^{\frac{ -x^2}{2} } \left. \left( t^2 e^{\frac{ t^2}{2} } - 2 e^{\frac{t^2}{2} } \right) \right|_0^x = \\
  & = x^2 - 2 + 2e^{ -\frac{x^2}{2} } 
\end{aligned}
\end{gathered}
\]

\exercisehead{5} $y' + y = e^{2t}$.  $y=1, t=0$.  
\[
\begin{gathered}
A = x 
\begin{aligned}
  y(x) & = e^{-x} \left( \int_0^x e^{2t} e^t dt + 1 \right) =e^{-x} \left( \left. \frac{e^{3t}}{3} \right|_0^x + 1 \right) \\
  & =\boxed{ \frac{e^{2x}}{ 3} + \frac{2}{3} e^{-x} }
\end{aligned}
\end{gathered}
\]

\exercisehead{6} $y'\sin{x} + y\cos{x} = 1; (0,\pi)$.  $\Longrightarrow y' + \cot{x} y = \csc{x}$
\[
\begin{gathered}
  A(x) = \int_a^x \cot{t} dt = \ln{ \left( \frac{ \sin{x} }{\sin{a}} \right) } \\
\begin{aligned}
  y(x) & = e^{-\ln{ \left( \frac{\sin{x} }{ \sin{a} } \right)} } \left( \int (\csc{t} )e^{\ln{ \left( \frac{ \sin{t}}{ \sin{a}} \right) } } dt + b \right) = \left( \frac{ \sin{a}}{ \sin{x} } \right) \left( \frac{ x -a}{ \sin{a} } + b \right) \\
    & \text{ indeed } y = \frac{ x-a}{ \sin{x} } + \frac{ b \sin{a}}{ \sin{x} }     
\end{aligned} \\
x\to 0 \text{ for } y = \frac{ x-a}{ \sin{x}} + \frac{ b \sin{a}}{ \sin{x}} = \frac{ x + b\sin{a} -a }{ \sin{x} } \\
\boxed{ b\sin{a} = a } \text{ for } x \to 0 \\
\boxed{ a-b\sin{a} = \pi } \text{ for } x \to \pi
\end{gathered}
\]

\exercisehead{7}
\[
\begin{gathered}
  x(x+1)y'+y = x(x+1)^2 e^{-x^2} \Longrightarrow y' + \frac{1}{x(x+1)} y = (x+1)e^{-x^2} \\
  A(x) = \int_a^x \left( \frac{1}{t} - \frac{1}{t+1} \right) dt = \ln{ \frac{x}{a} } - \ln{ \frac{x+1}{a+1} } \\
  e^{A(x)} = \frac{(a+1)x}{a(x+1)} \\
  \begin{aligned}
    y & = \frac{ a (x+1)}{ (a+1)x } \left( \int_a^x (t+1)e^{-t^2} \left( \frac{(a+1)t}{ a(t+1) } \right) dt  +b \right) = \frac{ (x+1)}{x} \left( \int_a^x t e^{-t^2} dt + b \frac{ a(x+1)}{ (a+1)x } \right) = \\
    & = \frac{ x+1}{2x} \left( e^{-a^2} - e^{-x^2} \right) + \frac{ a(x+1)b}{ (a+1)x }
  \end{aligned}
\end{gathered}
\]
It's easy to see that the last equation above goes to $0$ as $x \to -1$.  

\[
\begin{aligned}
  y & = (e^{-a^2} - e^{-x^2} ) (1/2)(1+\frac{1}{x} ) + \frac{ab}{ a+1} (1+ \frac{1}{x} ) = \\
  \lim_{x \to 0} y & = \frac{1}{x} \left( \frac{ e^{-a^2} - e^{-x^2} }{ 2} + \frac{ ab}{a+1} \right) \, \boxed{ a =0 }
\end{aligned}
\]

\exercisehead{8} $y' + y \cot{x} = 2 \cos{x}$ on $(0,\pi)$.  
\[
\begin{gathered}
  A(x) = \int_a^x \cot{t} dt = \ln{ \left( \frac{\sin{x}}{\sin{a}} \right) } \, e^{A(x)} = \frac{ \sin{x}}{\sin{a}} \\
  \begin{aligned}
    y & = \frac{\sin{a}}{\sin{x}} \left( \int 2 \cos{t} \frac{ \sin{t}}{\sin{a}} + b \right) = \frac{ \sin{a}}{\sin{x}} \left( \left. - \frac{ \cos{2t}}{ 2 \sin{a}} \right|_a^x + b \right) = \\
    & = \frac{ \cos{(2a)} + \cos{2x} }{ 2 \sin{x}} + \frac{ \sin{a}}{ \sin{x} } 
  \end{aligned}
\boxed{ y = \sin{x}} \, \boxed{ y = \frac{ \cos{(2a)} - \cos{(2x)}}{ 2 \sin{x}} + \frac{\sin{a}}{ \sin{x}}}
\end{gathered}
\]

\exercisehead{9} $(x-2)(x-3)y' + 2y = (x-1)(x-2)$.  $y' + \frac{2}{ (x-2)(x-3)} y = \frac{x-1}{x-3}$.  
\[
\begin{gathered}
  A(x) = \int_a^x \frac{ 2 dt }{ (t-2)(t-3)} = 2 \int_a^x \left( \frac{1}{ t-3} - \frac{1}{t-2} \right) dt = 2 \left. \left( \ln{ |t-3|} - \ln{ |t-2| } \right) \right|_a^x = 2 \ln{ \left| \frac{x-3}{a-3} \right| \left| \frac{ a-2}{ x-2} \right| } \\
  \begin{aligned}
   \text{ If } & (-\infty, 2 ), (3, +\infty), \begin{aligned} x - 3 & \lessgtr 0 \\ x -2 & \lessgtr 0 \end{aligned} \frac{3-x}{2-x} = \frac{ x-3}{x-2} \\
   \text{ If } & (2,3), x-3 < 0, \text{ but } x -2 > 0  
  \end{aligned} \\
  \begin{aligned}
    \text{ If } (-\infty, 2), (3,\infty) \, y & = b \left( \frac{x-2}{x-3} \right)^2 \left( \left| \frac{a-3}{ a-2} \right| \right)^2 + \left( \frac{x-2}{x-3} \right)^2 \left( x + \frac{1}{x-2} - a - \frac{1}{a-2} \right) 
  \end{aligned} 
\end{gathered}
\]
\[
  \begin{aligned}
    \text{ If } (2,3) \, y & = b \left( \frac{x-2}{x-3} \right)^2 \left( \left| \frac{a-3}{ a-2} \right| \right)^2 + \left( \frac{x-2}{3-x} \right)^2 \left| \frac{a-3}{a-2} \right|^2 \int \left( \frac{t-1}{t-3} \right) \left| \frac{a-2}{a-3 } \right|^2 \left( \frac{ 3-t}{t-2 } \right)^2 = \\
    & = b \left( \frac{x-2}{3-x} \right)^2 \left| \frac{a-3}{a-2} \right|^2 + \frac{ x-2}{3-x} \left( - \int \frac{(t-1)(3-t)}{(t-2)^2 } \right) = \\
    & = b \left( \frac{ x-2}{ x-3} \right)^2  \left| \frac{ a-3}{a-2} \right|^2 + \left( \frac{ x-2}{x-3} \right)^2 ( x + (x-2)^{-1} - a - (a-2)^{-1} )
  \end{aligned}
\]

\exercisehead{10} $s(x) = \frac{\sin{x}}{x}; x\neq 0 \, s(0)=1, \, T(x) = \int_0^x s(t) dt $\, $f(x) = xT(x)$ \\
\[
\begin{gathered}
  f' = T + x (s(x)) = T + \sin{x} \\
  xf' -f = x \sin{x} \\
\begin{gathered}
  y'-\frac{y}{x} = \sin{x} \\
  A(x) = \int_a^x \frac{ -1}{t} dt = -\ln{ \frac{x}{a} }, \, e^{-A(x)} = \frac{x}{a} \\
  y = \frac{ x}{a} \left( \int_a^x \sin{t} \frac{a}{t} + b \right) = xT + \frac{bx}{a} 
\end{gathered}  \\
y(0) = 0 \neq 1 \text{ since } P(x) = \frac{1}{x} \text{ is not continuous at } x = 0
\end{gathered} 
\]

\exercisehead{11} 
\[
\begin{gathered}
  f(x) = 1 + \frac{1}{x} \int_1^x f(t)dt \\
  x f(x) = x + \int_1^x f(t)dt \, \Longrightarrow f(x) + xf' = 1 + f(x) \Longrightarrow f' = \frac{1}{x} \\
  \Longrightarrow \boxed{ f(x) = \ln{ |x| } - C}  \\
  \begin{aligned}
\ln{|x|} -C & = 1 + \frac{1}{x} \int_1^x \left( \ln{ |t| } -c \right) dt = \\
  & = 1 + \frac{1}{x} \left. \left( t \ln{ |t| } -t -ct \right) \right|_1^x = \ln{ |x| } -C + \frac{1+C}{x} \Longrightarrow \boxed{ C = -1 } \\
  \end{aligned}
  \boxed{ f(x) = \ln{ |x|} +1 }
\end{gathered}
\]

\exercisehead{12} 
Rewrite the second property we want $f$ to have:
\[
\int_1^x f(t)dt = \frac{1-f(x)}{x} \Longrightarrow f(x) = -\frac{1}{x^2} - \left( \frac{ f' x - f }{ x^2 } \right)
\]
So then
\[
\begin{gathered}
  f' + \left( x- \frac{1}{x} \right) f = - \frac{1}{x}  \\
  \begin{aligned}
    & A(x) = \int P(t)dt = \int \left( t -\frac{1}{t} \right)dt = \left( \frac{1}{2} x^2 - \ln{x} \right) - \left( \frac{1}{2}a^2 -\ln{a} \right) \\
    & e^{A(x)} = e^{ \frac{ x^2 -a^2 }{ 2} } \frac{a}{x} ; e^{-A(x)} = \frac{x}{a} e^{\frac{ a^2-x^2}{2} } 
\end{aligned} \\
\boxed{ f(x) = \frac{x}{a} e^{ \frac{ a^2 -x^2}{ 2 } } \left( \int_a^x \frac{-1}{t^2} ae^{ \frac{ t^2 -a^2}{2} } dt + b \right) }
\end{gathered}
\]

\exercisehead{13} 
\[
\begin{gathered}
  \begin{aligned}
    v & = y^k \\ 
    v' & = ky^{k-1} y' 
\end{aligned} \quad v' + kPv = kQ = ky^{k-1} y' + kPy^k = kQ \text{ where } k =1-n  \\
  \Longrightarrow y' y^{-n} + P y^{1-n} =Q \Longrightarrow \boxed{ y' + Py = Q y^n }
\end{gathered}
\]

\exercisehead{14} $y'-4y=2e^x y^{1/2}$
\[
\begin{gathered}
  n=\frac{1}{2}\, k = 1 - \frac{1}{2} = \frac{1}{2} ; \, v = y^{1/2} ; \, v' + \frac{1}{2} (-4)v = \frac{1}{2} (2e^x) = v' -2v = e^x \\
  A(x) = \int_a^x P(t)dt =-2(x-a) = -2x \\
  v = e^{2x} \left( \int e^t e^{-2t} dt + b \right) = e^{2x} \left( -e^{-x} +b \right) = \boxed{ be^{2x}-e^x } \\
  \Longrightarrow y = (1+\sqrt{2})^2 e^{4x} -2 (1+\sqrt{2})e^{3x} + e^{2x} \\
  \begin{aligned}
    y(x) & = b^2-2b+1 = 2 \\
    & b = 1+\sqrt{2}
  \end{aligned}
\end{gathered}
\]

\exercisehead{15} 
$y'-y = -y^2 (x^2 + x +1), n=2 \, k=1-n =-1, \, v = y^k \, v = y^{-1}$.  
\[
\begin{gathered}
  v' + k P v = kQ \, v' + (-1)(-1)v = (-1)(-(x^2 +x+1)) = v' + v = x^2 + x +1 \\
  A(x) = \int P(t) dt = \int_0^x 1 dt = x \\
  v = e^{-x} \left( \int (t^2 + t+1) e^t dt +b \right) = e^{-x} \left. \left( t^2 e^{t} -t e^t + 2e^t \right) \right|_0^x + be^{-x} = (x^2 - x +2 ) - (2e^{-x}) + be^{-x} \\
  y = \frac{1}{ x^2 - x + 2 -2e^{-x}  }
\end{gathered}
\] 

\exercisehead{16} 
\[
\begin{gathered}
  v' + -\frac{1}{x} v = 2 x^2 \\
  v = e^{\ln{x}} \left( \int 2x^2 e^{-\ln{x} } dx + C \right) = x (x^2 + C) = x^3 + Cx 
\end{gathered}
\]
Then since $v=y^k, \, k = 1 - n$, 
\[
\begin{gathered}
  y = (x^3 + Cx)^2 = x^2 (x^2 + C)^2 ; \, x \neq 0  \\
\boxed{  y = x^2 (x^2 - 1)^2 }
\end{gathered}
\]
Check: 
\[
\begin{gathered}
  y' = 2 x (x^2 - 1)^2 + x^3 (4)(x^2 - 1)  \\
2x^2 + x^4 (4)(x^2 - 1) - 2 x^2 (x^2 -1)^2 = 4x^3  
\end{gathered}
\]

\exercisehead{17}$ xy' + y  = y^2 x^2 \log{x} $ on  $(0, +\infty )$  with $y = \frac{1}{2}$ when $x = \frac{1}{2}, x\neq 0 $.  
\[
\begin{gathered}
  y' + \frac{y}{x} =y^2 x \log{x} \\
  k = 1 - n = 1 -2 = -1, v= y^k = y^{-1}  \\
    v' + k P v = k Q; \quad v' + - \left( \frac{1}{x} \right) v = -1 x \log{x}  \\
  \begin{aligned}
v & = x \left( \int \frac{ - x \log{x} }{x} dt + C \right) = x \left( \int -\log{t} dt + C \right) = \\
& = x (- (x \log{x} - x ) + C ) = -x^2 \log{x} + x^2 + Cx  
  \end{aligned}
\boxed{ y = \frac{1}{ C x + x^2 - x^2 \log{x} } }  \quad C = -\frac{1}{2} (1 - \log{ \frac{1}{2} } ) + 2 \\
\end{gathered}
\]
Check: 
\[
\begin{gathered}
y' = (C + 2 x - 2 x \log{x} - x )(-y^2 )  \\
y^2 (-C + - 2 x + 2 x \log{x} + x + C + x -x \log{x} )
\end{gathered}
\]

\[
\begin{gathered}
  y=\frac{1}{2}; x =1 \, y(x=1) = \frac{1}{2} \, b=2 \\
  \boxed{ y = \frac{ 1}{ x (x\ln{x} -x +3 )} }
\end{gathered}
\]

\exercisehead{18}  $2 x y y' + (1+x)y^2 = e^x$ on $(0,+\infty)$, $y = \sqrt{e}$ when $x=1$; $y = - \sqrt{e}$ when $x=1$.  \\
If $x\neq 0, y \neq 0$,  
\[
\begin{gathered}
  y'  + \frac{ (1+x)}{2x} y = \frac{ e^2 }{2x} y^{-1}  \\
  k = 1 - n = 2;  \quad v = y^k = y^2 
\end{gathered}
\]
\[
\begin{gathered}
  v' + 2 P v = 2 Q; \, v' + 2 \left( \frac{ 1+x}{2x} \right) v = \frac{ 2 e^x}{2x} \quad \quad \Longrightarrow  v' + \frac{1+x}{x} v = \frac{e^x}{x} = v' + P_v = Q_v \\
A_v = \int_a^x P_v  = \int_a^x \frac{1+t}{t} dt = \ln{ \left( \frac{x}{a} \right) } + (x-a) ; \quad \quad \, e^{\int_a^x P_v dt} = e^{ \ln{ x/a} + x-a } = \frac{x}{a} e^{x-a} \\
  \begin{aligned}
    v & = \frac{ae^{-x+a}}{x} \left( \int_a^x \frac{e^t}{t} \frac{t}{a} e^{t-a} dt + b_v \right) = \frac{a}{x} e^{-x+a} \left(  \left. \frac{1}{2a} e^{2t-a} \right|_a^x + b_v  \right) \\ 
& = \frac{a}{x} e^{-x+a} \left( \frac{e^{2x-a}}{2 a } - \frac{e^a}{2a} + b_v \right) 
  \end{aligned} \\
y = \pm \sqrt{ \frac{a}{x} e^{-x+a} \left( \frac{e^{2x-a}}{2 a } - \frac{e^a}{2a} + b_v \right)  }
\end{gathered}
\]
%Check:
%\[
%\begin{gathered}
%  y' = \frac{1}{2} \left( \frac{ e^x + 2 C e^{-x} }{ 2x } \right)^{-1/2} \left( \frac{ (e^x - 2C e^{-x}) 2 x - 2 (e^x + 2 C e^{-x} ) }{ 4 x^2 } \right) = \frac{1}{2y} \left( \frac{ }{} \right)
%\end{gathered}
%\]

%\[
%\begin{gathered} 
%  y = \left( \frac{ e^x - e^{2a -x } }{ 2x } - \frac{ ba e^{a-x}}{ x } \right)^{1/2} ; y(x=1) = \left( \frac{ e^1 - e^{2a-1}}{ 2 } - \frac{ b (1) e^{a-1}}{ 1 } \right)^{1/2} \, \boxed{ b=-e } \\
%  b=0, a=0 \\
%  \boxed{ y = \left( \frac{ e^x - e^{-x}}{ 2x } \right)^{1/2} }
%\end{gathered}
%\]

Now $y^k = v$ \medskip \\
$y^2 = v = e = \frac{1}{1} e^{-1+1} \left( \frac{ e^{2-1} }{ 2(1) } - \frac{e^1}{2} + b_v \right) = b_v = e$
\begin{enumerate}
\item $b_v = e$  \quad $\Longrightarrow y = \sqrt{ \frac{1}{x} e^{-x+1} \left( \frac{e^{2x-1}}{2 } - \frac{e}{2} + e \right)  } $
\item $b_v =e $ \quad $\Longrightarrow y = - \sqrt{ \frac{1}{x} e^{-x+1} \left( \frac{e^{2x-1}}{2 } - \frac{e}{2} + e \right)  } $
\item If we could take $a=0$, then $\lim_{x\to 0} y = \pm \sqrt{ \frac{ e^x}{ 2x} - \frac{ e^{-x +2a }}{ 2x} + \frac{b_v ae^{-x+a}}{ x } } = \pm \sqrt{ \frac{ 1 + x - e^{2(0)}(1-x) + 2b_v 0 e^{-x+a} }{2x} } = \pm 1$ \medskip \\

If we consider $\lim_{x\to 0} y^2$, and let $a$ go to $0$, then 
\[
  v = \frac{1}{x} \left( e^{x} - e^{-x} \right) \Longrightarrow \frac{ \sinh{x}}{x}
\]
\end{enumerate}

\exercisehead{19} The Ricatti equation is $y' + P(x)y + Q(x) y^2 = R(x)$.  \medskip \\
If $u$ is a known solution, $y = u + \frac{1}{v}$ is also a solution if $v$ satisfies a first-order ODE.  
\[
\begin{gathered}
  (u + 1/v)' = u' + \frac{-1}{v^2} v' \\
  y' + P y + Qy^2 = R  \Longrightarrow u' + \frac{-v'}{v^2} + P(u+\frac{1}{v} ) + Q( u^2 + \frac{2u}{v} + \frac{1}{v^2} ) = R \\ \Longrightarrow \boxed{ v' -Pv = Q(2uv+1)  }
\end{gathered}
\]

\exercisehead{20} $y'+ y + y^2 =2, y =1,-2$.  
\begin{enumerate}
\item If $-2 \leq b < 1 $, 
\[
\begin{gathered}
  y'+y+y^2 = 2 \, P =1, Q =1, R=2 \\
  v' + (-P-2Qu) v  = Q \, y = u + \frac{1}{v}  
\end{gathered}
\]
\[
\begin{gathered}
  u=1 \\ 
  \begin{aligned} 
    &    v' + (-1-2(1)(1))v = v' -3v = Q =1 \\
    & v = e^{3x} \left( \int 1 e^{-3t} dt + b \right)  = e^{3x} \left( \left. \frac{ e^{-3t}}{ -3} \right|_a^x  + b \right) = b e^{3x} - \frac{1}{3} \left( 1 - e^{3x - 3a} \right)  \\
    y & = 1 + \frac{ 3 }{ 3be^{3x} - (1-e^{3x-3a})} \\
    y(0) & = 1 + \frac{ 3}{ 3b -(1-e^{-3a} ) } \Longrightarrow \begin{aligned} & b = 1 + \frac{1}{b} \\ & b^2 - b-1 = 0 \\ & b = \frac{ 1 \pm \sqrt{5}}{ 2 } \end{aligned} 
  \end{aligned}
\end{gathered}
\]
\[  
\boxed{ y  = 1 + \frac{3}{ 3be^{3x} - (1-e^{3x} ) } \quad \quad \, b = \frac{1- \sqrt{5}}{2} }
  \]
\item 
\[
\begin{gathered}
u =-2 \\
\begin{aligned}
  & v' + (-1-2(1)(-2))v = v' +3v = 1 \\
  & v =e^{-3x} \left( \int_a^x e^{3t} dt + b \right) = e^{-3x} \left( \frac{ e^{3x} - e^{3a} }{ 3} \right) + be^{-3x} = \frac{ 1 - e^{3a-3x}}{3} + be^{-3x} \\
  y & = -2 + \frac{3}{ 1 - e^{3a - 3x} + 3be^{-3x} }; y(0) = -2 + \frac{ 3 }{ 1 -e^{3a} +3b } \xrightarrow{a=0} y(0) = -2 + \frac{3}{3b} \Longrightarrow b = -1 \pm \sqrt{2} 
\end{aligned} \\
\boxed{ b \geq 1 \text{ or } b < -2 , y = -2 + \frac{ 3 }{ 1-e^{-3x} + 3b e^{-3x} }\, b = -1 \pm \sqrt{2} }
\end{gathered}
\]
\end{enumerate} 

%-----------------------------------%-----------------------------------%-----------------------------------
\subsection*{ 8.7 Exercises - Some physical problems leading to first-order linear differential equations }
%-----------------------------------%-----------------------------------%-----------------------------------
\quad \\
\quad \\
\exercisehead{3}
\begin{enumerate}
  \item $y' = -\alpha y(t)$.  \quad \quad $y(T) = y_0 e^{-\alpha T} = \frac{y_0}{n}$.  \quad \quad $n = e^{\alpha T}$ so the relationship between $T$ and $n$ doesn't depend upon $y_0$.  \quad \quad $\boxed{ \frac{1}{k} \ln{e} = T }$.  
  \item $f(a) = y_0 e^{-ka}; \quad f(b) = y_0 e^{-kb}$.  
\[
\begin{gathered}
  \frac{f(a)}{f(b)} = e^{-ka + kb} \quad \Longrightarrow \ln{ \frac{f(a)}{f(b)}} = -k(a-b); \quad \frac{1}{a-b} \ln{ \frac{f(a)}{ f(b)} } = -k \\
  f(t) = y_0 \exp{ \left( \frac{ \ln{ \frac{f(a)}{f(b)} } t }{ a-b} \right) }; \quad f(a) = y_0 \left( \frac{f(b)}{y_0} \right)^{a/b}; \quad \Longrightarrow \frac{f(a)}{ (f(b))^{a/b} } = y_0^{1- \frac{a}{b} } \\
  \begin{aligned}
    f(t) & = \left( \frac{f(a)}{ (f(b))^{a/b} } \right)^{\frac{b}{b-a} } \left( \frac{f(a)}{f(b)} \right)^{\frac{t}{a-b}} = \frac{ (f(a))^{ \frac{b}{b-a} } }{ f(b)^{\frac{a}{b-a}} } \frac{ f(a)^{-\frac{t}{b-a}} }{ f(b)^{\frac{t}{a-b}} } = \\
    & = \frac{f(a)^{\frac{b-t}{b-a}} }{ (f(b))^{\frac{a-t}{b-a}} } = \boxed{ f(a)^{\frac{b-t}{b-a}} f(b)^{\frac{t-a}{b-a}} }   
\end{aligned} \\
w(t) = \frac{ b-t}{b-a}; \quad 1 - w(t) = \frac{ b-a - (b-t) }{ b-a} = \frac{t-a}{b-a}
\end{gathered}
\]
\end{enumerate}

\exercisehead{4} $F=mv'= w_0 - \frac{3}{4} v$ \\
\[
\begin{gathered}
  \begin{aligned}
    w_0 & = 192 \\
    \frac{w_0}{g} & = 6 = m 
  \end{aligned} \quad \quad
v' = \frac{w_0}{m} - \frac{1}{8} v \Longrightarrow v' + \frac{1}{8} v = \frac{w_0}{m} = 32 \\
  v(t) = e^{-\frac{t}{8}} \left( \int_0^t \frac{w_0}{m} e^{ \frac{t}{8}} dt + b \right) = e^{\frac{-t}{8}} \left( \frac{ 8 w_0}{m} (e^{t/8} - 1) + 0 \right) = \left( 256 (1 - e^{-t/8}) \right) \\
  v(10) = 256(1-e^{-5/4}) = 256(1 - 37/128) = 182
\end{gathered}
\]

\[
\begin{gathered}
  F = mv' = w_0 - 12 v \quad \, v' + \frac{12}{m}v = \frac{w_0}{m} = v' + 2 v =32 \\
\begin{aligned}
  v(t) & = e^{-2t} \left( \int_{t_0}^t e^{2x} 32 dx + b \right) = e^{-2t} \left( 16 (e^{2t} - e^{2t_0}) + b \right)  = \left( 16 ( 1 - e^{2 (t_0 - t) } ) + b e^{-2t} \right) \\
  v(10) & = be^{-2(10)} = 182 \Longrightarrow b = 182 e^{20} 
\end{aligned} \\
v(t) = 16 + 166e^{20 - 2t }
\end{gathered}
\]
So then 
\[
v(t) = \begin{cases} 256(1- e^{-t/8}) & \text{ if } t < 10 \\ 16 + 166e^{20-2t} & \text{ if } t > 10 \end{cases} 
\]

\exercisehead{7} 
\begin{enumerate}
\item 
\[
\begin{gathered}
  y'(t) = (y - M)k = ky - kM \quad \quad y' + - ky = -kM \\
  y= e^{kt} (\int -kM e^{-kt} dt + b) = e^{kt} ( M(e^{-kt} - 1 ) + b )  \quad \quad M = 60^{\circ} \\
  y(0) = b = 200^{\circ}  \\
  y_f = e^{kT} (M (e^{-kT} - 1 ) + 200 ) = M + e^{kT} (200 - M)  \\
  \frac{1}{T} \ln{ \left( \frac{ y_f - M }{ b-M } \right) } = k = \frac{1}{T} (\ln{ (60) } - \ln{ (140) } ) = \frac{1}{T} \ln{ \left( \frac{3}{7} \right) } = \frac{1}{30} (\ln{3} - \ln{7} ) \\
  \boxed{ y(t) = 60 + 140 e^{ \frac{ \ln{3} - \ln{7} }{ 30 } t } }
\end{gathered}
\]
\item 
\[
\begin{gathered}
  y(t) = 60 + 140 e^{-kt}; \quad \, k = \frac{ (\ln{7} - \ln{3})}{ 30 } \\
  \ln{ \left( \frac{Y - 60}{140} \right) } = -kt \Longrightarrow t_f = \frac{ (\ln{ 140} - \ln{ (T - 60) } ) }{ k } \text{ for } 60 < T \leq 200 
\end{gathered}
\]
\item $t_f = \frac{ \ln{ (140) } - \ln{ (30) }}{ k } = \frac{30}{ \ln{ 7/3} } \ln{ \frac{14}{3} } = 54 \, \text{ minutes } $
\item $M = M(t) = M_0 - \alpha t \quad \quad \alpha = \frac{1}{10}$
\[
\begin{aligned}
  y & = e^{kt} \left( \int -k Me^{-kt} dt + b\right)  = e^{kt} \left( \int_0^t -k (M_0 - \alpha u ) e^{-ku} du + b\right) = \\
  & = -ke^{kt} \int_0^t ( M_0 e^{-ku} - \alpha u e^{ku} ) du + be^{kt} = \\
  & = -ke \left( \left. \frac{M_0 e^{-ku} }{ -k} \right|_0^t - \alpha \left. \left( \frac{ u e^{-ku}}{ -k} + \frac{ -e^{-ku}}{k^2 } \right) \right|_0^t \right) + be^{kt} = \\
  & = -ke^{kt} \left( \frac{ M_0 }{k} ( 1 - e^{-kt} ) - \alpha \left( \frac{ te^{-kt}}{-k} - \frac{ e^{-kt}}{k^2} + \frac{1}{k^2 } \right) \right) + be^{kt} = \\
  y(t) &= -M_0 e^{kt} + M_0 - \alpha t - \alpha /k + \alpha e^{kt}/ k + be^{kt} = (-M_0 + \alpha/k +b )e^{kt} + (M_0 - \alpha t - \alpha /k ) = \\
\end{aligned}
\]
\[
\boxed{ y(t) = (140 + \frac{3}{ (\ln{3} - \ln{7} ) } ) e^{-kt} + (60 - \frac{t}{10} - \left( \frac{3}{\ln{3/7} } \right) ) }
\]
\end{enumerate}

\exercisehead{8} $y'(t) = -k(y-M_0); \quad \quad y' + ky = kM_0$.  
\[
\begin{gathered}
  y = e^{-kt} \left( \int_{t_i}^{t_f} k M_0 e^{ku} du + b \right) = e^{-kt_f} \left( M_0 (e^{kt_f} - e^{kt_i} ) + b \right) = M_0 (1 - e^{-k (t_f - t_i)} ) + be^{-kt_f} \\
  \Longrightarrow \begin{aligned} y(t_f) - M_0 & = -M_0 e^{-k(t_f - t_i) } + be^{-kt_f} \\
    65 - M_0 & = -M_0 e^{-k(5)} + 75 e^{-k(5) } = (75 - M_0) e^{-5k} \Longrightarrow \ln{ \left( \frac{ 65 - M_0 }{ 75 - M_0 } \right)} = -5k ; \quad k = \frac{1}{5} \ln{ \left( \frac{75 - M_0}{ 65 -M_0 } \right) } \\
      60 - M_0 & = -M_0 e^{-k(5)} + 65 e^{-k(5) } = (65 - M_0) e^{-5k} \Longrightarrow \ln{ \left( \frac{ 60 - M_0 }{ 65 - M_0 } \right) }= -5k  
\end{aligned} \\
  \Longrightarrow \frac{ 65 - M_0}{ 75 - M_0} = \frac{ 60 - M_0 }{ 65 - M_0 } \\
  \boxed{ M_0 = 55 }
\end{gathered}
\]

\exercisehead{9} \quad \\
Let $y(t) = $ absolute amount of salt.  \\
Water is leaving according to $w(t) = w_0 + (3-2)t = w_0 + t $.   \\
Salt leaving $= \left( \frac{ 2 \, gal }{ min. } \right) \left( \frac{ y(t) \, \text{ salt }  }{ w(t) } \right) $ \\
So then
\[
y' = \frac{ -2y }{ w_0 + t } \Longrightarrow \ln{y } = -2 (\ln{ (w_0 + t) } - \ln{ w_0} ) = -2 \ln{ \left( \frac{w_0 + t}{ w_0} \right) }
\]
is the equation of motion given by the problem.  
\[
\begin{gathered}
  y(t) = Ce^{\ln{ \left( \frac{ w_0 +t }{ w_0 } \right)^{-2 } } } = C \left( \frac{ w_0 + t}{ w_0} \right)^{-2} \\
  \boxed{ y(t) = 50 \left( \frac{ 100}{ 100 + t } \right)^2 } \quad y(t=60 \, min. ) = 50 \left( \frac{100}{160} \right)^2 = 50 \frac{25}{64}  = \frac{625}{32} \simeq 19.53 
\end{gathered}
\]

\exercisehead{10}  \quad \\
Let $y$ be the dissolved salt (total amount of) at $t$ time.  The (total) amount of water at any given time in the tank is $w = w_0 + t$.  \\
There is dissolved salt in mixture that is leaving the tank at any minute.  There is also salt from undissolved salt in the tank that is ``coming into'' the dissolved salt, adding to the amount of dissolved salt in the mixture.  Thus
\[
y'(t) = (-2) \left( \frac{ y}{w} \right) + \alpha \left( \frac{y}{w} - 3 \right); \quad \alpha = \frac{-1 \, \text{ gal } }{ 3 \, \text{ min } }
\]
We obtained $\alpha$ easily by considering only the dissolving part and how it dissolves 1 pound of salt per minute if the salt concentration, $\frac{y}{w}$ was zero, i.e. water is fresh.  

\[
\begin{gathered}
  y'(t) = \frac{-7}{3} \frac{y}{w} + 1; \quad \quad y' = \frac{7}{3}\frac{y}{ w_0 +t } = 1 \\
  \begin{aligned}  
    P & = \frac{7/3}{ w_0 +t } \\
    \int P & = \int \frac{ 7/3}{ w_0 + t } = \frac{7}{3} \left. \ln{ (w_0 + t ) }  \right|_0^t = \frac{7}{3} \ln{ \left( \frac{ w_0 +t }{ w_0 } \right) } 
  \end{aligned} \\
  \begin{aligned}
 y & = e^{ - \ln{ \left( \frac{ w_0 + t}{ w_0 } \right)^{-7/3} } } \left( \int_0^t (1) e^{ \ln{ \left( \frac{ w_0 + u }{ w_0} \right)^{7/3} } } du + b \right) = \\
 & = \left( \frac{w_0}{ w_0 +t } \right)^{7/3} \left( \int_0^t \left( \frac{ w_0 + u }{ w_0} \right)^{7/3} du + b \right) = \left( \frac{w_0}{ w_0 +t } \right)^{7/3} \left( \frac{ 3 w_0}{ 10 } \left( \left( \frac{w_0 +t }{w_0} \right)^{10/3} - 1 \right) + b \right) = \\
 y & = \left( \frac{ 100}{ 100 + 60 } \right)^{7/3} \left( \frac{ 3 (100)}{ 10 } \left( \left( \frac{ 100 + 60}{ 100 } \right)^{10/3} - 1 \right) + 50 \right) \simeq 54.78 \, lbs.
  \end{aligned}
\end{gathered}
\]

\exercisehead{11} $LI'(t) +RI(t) = V(t)$ \quad \quad $V(t) = E\sin{\omega t}$.  
\[
\begin{aligned}
  I(t) & = I(0) e^{-Rt/L} + e^{-Rt/L} \int_0^t \frac{V(x)}{L} e^{Rx/L} dx \\
  I(t) & = I(0) e^{-Rt/L} + e^{-Rt/L} \frac{E}{L} \int_0^t \sin{ \omega x} e^{Rx/L} dx 
\end{aligned}
\]
Using $\int e^{ax} \sin{bx} dx = a e^{ax} \sin{bx} - be^{ax} \cos{bx} $
\[
I(t) = I(0) e^{\frac{-Rt}{L} } + \frac{E e^{ -Rt/L}}{ L } \left( \frac{ \frac{R}{L} e^{\frac{Rt}{L}} \sin{ \omega t} - \omega e^{\frac{Rt}{L} } \cos{ \omega t}}{ \left( \frac{R}{L} \right)^2 + \omega^2 } + \frac{ \omega }{ \left( \frac{R}{L} \right)^2 + \omega^2 } \right)
\]
$I(0) =0 $ \quad So 
\[
\begin{gathered}
  I(t)  = \frac{E}{L} \frac{ \frac{R}{L} \sin{ \omega t} - \omega \cos{ \omega t}}{ \left( \frac{R}{L} \right)^2 + \omega^2 } + \frac{ E \omega L}{ R^2 + (\omega L)^2 } e^{-Rt/L} = \frac{ E (R \sin{ \omega t } - \omega L \cos{ \omega t} ) }{ R^2 + (\omega L )^2 } + \frac{ E \omega L }{ R^2 + (\omega L)^2 } e^{ -Rt/L} = \\
  \sin{ \alpha } = \frac{ \omega L}{ \sqrt{ R^2 + (\omega L^2) } } \\
  \Longrightarrow I(t) = \frac{ E \sin{ (\omega t- \alpha )} }{ \sqrt{ R^2 + (\omega L )^2 } } + \frac{ E \omega L}{ (R^2 + \omega^2 L^2 )} e^{-Rt/L}  \\
  L = 0 \quad \sin{\alpha} = 0 
\end{gathered} 
\]

\exercisehead{12} 
\[
\begin{gathered}
  E(t) = \begin{cases} E & \text{ if } 0 < a \leq t < b \\
    0 & \text{ otherwise } \end{cases} \\
 \begin{aligned}
   I(t) & = e^{ -Rt/L} \int_0^t 0 = 0 \, \text{ for } t < a \\
   I(t) & = e^{-Rt/L} \int_a^t \frac{E}{L} e^{Rx/L} dx = \frac{E}{L} e^{-Rt/L} \frac{L}{R} \left( e^{Rt/L} -e^{Ra/L} \right) = \frac{E}{R} \left( 1 - e^{\frac{ R(a-t)}{L} } \right)  \\
   I(b) & = \frac{E}{R} (1 - e^{ R (a-b)/L } )  
\end{aligned} \\
 \text{ for } t > b, \quad I(t) = K e^{-Rt/L } \\
 \Longrightarrow I(t) = \frac{E e^{-Rt/L}}{ R } (e^{ \frac{ Rb}{L} } = e^{\frac{Ra}{l}} ) \quad \text{ for } I(b) = I(b)
\end{gathered}
\]

\exercisehead{13} From Eqn. 8.22, $\frac{dx}{dt} = kx(M-x)$
\[
\begin{gathered}
  \frac{dx}{dt} = kx(M-x) = kMx - kx^2; \quad \Longrightarrow \frac{dx}{ kMx - kx^2} = dt = \frac{ 1/k dx}{ x (M-x)} = dt \\
  \Longrightarrow k dt = \left( \frac{1}{x} + \frac{1}{ M-x} \right) \frac{1}{M} dx = \frac{ \ln{x} + - \ln{ (M-x) }}{ M } \\
  Mk(t-t_i) = \ln{ \frac{x}{ M-x}}; \quad e^{Mk (t- t_i) } (M-x) = x \\
  x(t) = \frac{ M e^{Mk (t-t_i)}}{ 1 + e^{Mk(t-t_i)} } = \boxed{ \frac{ M}{ 1 + e^{ -Mk (t-t_i) } } }
\end{gathered}
\]

\exercisehead{14} Note that we are given \emph{ three equally spaced times}.  
\[
\begin{gathered}
  M = x_2 + x_2 e^{ -\alpha (t_2 - t_0) }; \quad \frac{ M- x_2}{ x_2} = e^{ -\alpha (t_2 - t_0) } \\
  \frac{ M-x_2}{x_2} \left( \frac{ x_1}{ M-x_1} \right) = e^{- \alpha t_2 + \alpha t_0 + \alpha t_1 - \alpha t_0 } = e^{-\alpha (t_2 - t_1) } \\
  \frac{ M-x_3}{x_3} \left( \frac{ x_2}{ M-x_2} \right) = e^{- \alpha (t_3 - t_2) } = \frac{ M-x_2}{ x_2} \left( \frac{ x_1}{ M-x_1} \right)  \\
  (M-x_3)(M-x_1)x_2^2 = x_1 x_3 (M-x_2)^2 = x_1 x_3 (M^2 - 2Mx_2 + x_2^2) = x_2^2 (M^2 - M(x_1 + x_3) + x_1 x_3 ) \\
  (x_2^2 - x_1 x_3)M^2 = M (x_2^2 (x_1 + x_3) + - 2 x_2 x_1 x_3 ) = (-x_1 (x_3 - x_2) + x_3 (x_2 - x_1) ) x_2 \\
  \Longrightarrow \boxed{ M = x_2 \frac{ (x_3 (x_2 - x_1) - x_1 (x_3 - x_2) ) }{ x_2^2 - x_1 x_3 } }
\end{gathered}
\]

\exercisehead{15} 
\[
\begin{gathered}
  \frac{dx}{dt} = k(t) Mx - k(t) x^2 \quad \quad \frac{dx}{ Mx- x^2 } = k(t) dt \Longrightarrow M \int_{t_i}^t k(u) du = \ln{ \left( \frac{x}{M-x} \right) } \\
  \begin{aligned}
    \frac{x}{M-x} & = e^{ M \int_{t_i}^t k(u) du } \\
      x & = \frac{ Me^{ M \int_{t_i}^t k(u) du }}{ 1 + e^{ M \int_{t_i}^t k(u) du } } = \boxed{ \frac{ M }{ 1 + e^{ -M \int_{t_i}^t k(u) du } } }
\end{aligned}
\end{gathered}
\]

\exercisehead{16} 
\begin{enumerate}
  \item $M = 23 \frac{ 92 (23 - 3.9) - 3.9 (92 - 23) }{ 23^2 - 3.9 (92) } = 201 $ 
  \item \[
M = 122 \left( \frac{ 150 (122 - 92) - 92 (150 - 122) }{ (122)^2 - 92(150) } \right)  = 122 \left( \frac{ 150(30) - 92(28) }{ (122)^2 - 92 (150) } \right) = 216
\]
\item Reject.
\end{enumerate}

%-----------------------------------%-----------------------------------%-----------------------------------
\subsection*{ 8.14 Exercises - Linear equations of second order with constant coefficients, Existence of solutions of the equation $y'' + by = 0$, Reduction of the general equation to the special case $y'' + by = 0$, Uniqueness theorem for the equation $y'' + by = 0$, Complete solution of the equation $y'' + by =0 $, Complete solution of the equation $y'' + ay' +by =0$ }
%-----------------------------------%-----------------------------------%-----------------------------------
\quad \\

\exercisehead{1} $y'' - 4y = 0$ \quad $ y = c_1 e^{2x} + c_2 e^{-2x} $.  

\exercisehead{2} $y'' + 4y = 0$ \quad $ y = c_1 \cos{ (2x) } + c_2 \sin{ (2x) }$.  

Use Theorem 8.7.  
\begin{theorem}
Let $d = a^2 - 4 b$ be the discrimnant of $y'' + ay' + by =0 $.  \\
Then $\forall$ solutions on $(-\infty, \infty)$ has the form 
\begin{equation}
  y = e^{-ax/2} ( c_1 u_1(x) + c_2 u_2(x) ) 
\end{equation}
where 
\begin{enumerate}
  \item If $a=0$, then $u_1(x) = 1$ and $u_2(x) =x$ 
  \item If $d>0$, then $u_1(x) = e^{kx}$ and $u_2(x) = e^{-kx}$, where $k = \frac{\sqrt{d}}{2} $
  \item If $d < 0$, then $u_1(x) = \cos{kx}$ and $u_2(x) = \sin{kx}$; where $k= \frac{1}{2} \sqrt{ -d}$
\end{enumerate}
\end{theorem}

\exercisehead{3} $y'' - 4y' = 0; \quad a=-4$.  
\[
y = e^{2x} ( c_1 e^{2x} + c_2 e^{-2x} ) = \boxed{ c_1 e^{4x} + c_2 }
\]

\exercisehead{4} $y'' + 4y' = 0$ 
\[
y = e^{-2x} ( c_1 e^{2x} + c_2 e^{-2x} ) = \boxed{ c_1 + c_2 e^{-4x} }
\]

\exercisehead{5} $y'' - 2y' + 3y = 0$  \quad $d = 4 - 4(3) = -8$  \quad $\Longrightarrow \boxed{ y = e^{-x} (c_1 \sin{\sqrt{2}x } + c_2 \cos{ \sqrt{2}x } ) }$

\exercisehead{8} $y'' - 2y' + 5y = 0$ \quad $d = -16$ \quad $y = e^x (c_1 \cos{2x} + c_2 \sin{2x} )$

\exercisehead{9} $y'' + 2y' + y =0$  \quad $d = 4- 4(1)(1) = 0$ \quad $y = e^{-x} (1+x)$

\exercisehead{10} $y'' - 2y' + y = 0$ \quad $d = 4 - 4 (1)(1) = 0$ \quad $y = e^x (1 +x)$ 

\exercisehead{11} $y'' + \frac{3}{2} y' = 0$ \quad $y=1, \, y' = 1; \, x =0$  $d = \frac{9}{4} > 0$
\[
\begin{gathered}
  y = e^{ \frac{-3}{4} x} ( c_1 e^{ \frac{3x}{4} } + c_2 e^{ \frac{-3 x}{4} } ) = c_1 + c_2 e^{\frac{-3x}{2} } \\
c_2 = \frac{-2}{3} \Longrightarrow   y  = \frac{5}{3} + \frac{-2}{3} e^{ \frac{-3x}{2} }
\end{gathered}
\]

\exercisehead{12} $y'' + 25 y = 0 ; \quad y = -1 , \, y' =0, \, x = 3$.  
\[
\begin{aligned}
  y & = c_1 \sin{ 5x} + c_2 \cos{ 5x } \\
  y' & = 5 c_1 \cos{ 5x} + - 5c_2 \sin{5x} \\
  0 & = 5 c_1 \cos{15} - 5 c_2 \sin{15} 
  c_2 \sin{15} & = c_1 \cos{15} 
\end{aligned}
\quad \, 
\begin{aligned}
-1 & = c_1 \sin{15} + c_2 \cos{15} \\
-1 & c_1 (\sin{15} + \frac{ \cos^2{15}}{ \sin{15}} ) = c_1 \left( \frac{1}{\sin{15}} \right)
\end{aligned} \quad \, 
\begin{aligned}
c_1 & = -\sin{15} \\
c_2 & = -\cos{15}
\end{aligned}
\]
\[
\boxed{ y = -\sin{15} \sin{5x} - \cos{15} \cos{5x} }
\]

\exercisehead{13} 
$y'' - 4y' - y =0; \quad y = 2; \quad y' = -1 \, \text{ when } x = 1$ \\
$d = 10 -4(1)(-1) = 20 $
\[
\begin{gathered}
y = c_1 e^{ (2+\sqrt{5})x } + c_2 e^{ (2- \sqrt{5})x }  \\
\begin{aligned}
  y(x=1) & = c_1 e^{2+\sqrt{5}} + c_2 e^{ 2-  \sqrt{5}} = 2 \\
  y'(x=1) & = (2+\sqrt{5}) c_1 e^{ 2 + \sqrt{5} } + (2-\sqrt{5}) c_2 e^{ 2-\sqrt{5} } = -1 
\end{aligned} \quad \Longrightarrow (5+ 2 \sqrt{5} ) c_1 e^{2+\sqrt{5}} + (5 - 2 \sqrt{5} ) c_2 e^{2 - \sqrt{5}} = 0 \\
c_1 = \frac{ 2 \sqrt{5} - 5}{ 5 + 2 \sqrt{5}} c_2 e^{-2\sqrt{5}} \\
\left( \frac{2 \sqrt{5} - 5 }{ 5 + 2 \sqrt{5} } \right) c_2 e^{ 2 - \sqrt{5}} + c_2 e^{ 2- \sqrt{5}} = 2 = \frac{ 4 \sqrt{5} c_2 e^{ 2 - \sqrt{5}} }{ 5 + 2\sqrt{5}} \Longrightarrow \frac{5+ 2 \sqrt{5}}{ 2 \sqrt{5}} e^{ \sqrt{5} - 2 } = c_2 \quad c_1 = \frac{ 2 \sqrt{5} - 5 }{ 2 \sqrt{5}} e^{ -2 - \sqrt{5}} \\
\boxed{ y = \frac{ 2 \sqrt{5} - 5}{ 2 \sqrt{5}} e^{ -2 - \sqrt{5} } e^{ (2 + \sqrt{5}) x } + \frac{ 5  + 2 \sqrt{5}}{ 2 \sqrt{5}} e^{ \sqrt{5} - 2 } e^{ (2 - \sqrt{5}) x } }
\end{gathered}
\]

\exercisehead{14} $y'' + 4y' + 5 y = 0$,  with  $y = 2 $ and $y' = y''$ when  $x =0$ \\
$16 -4(1)(5) =-4$
\[
\begin{gathered}
\begin{aligned}
  y & = e^{-2x} (c_1 \sin{2x} + c_2 \cos{2x} ) \quad y(x=0) = c_2 = 2 \\
  y' & = -2 e^{-2x} (c_1 \sin{2x} + 2 \cos{2x} ) + 2 e^{-2x} (c_1 \cos{2x} - 2 \sin{2x} ) \\
  y'' & = 4e^{-2x} ( c_1 \sin{2x} + 2 \cos{2x} ) - 8 e^{-2x} (c_1 \cos{2x} - 2 \sin{2x} ) + 4 e^{-2x} (-c_1 \sin{2x} - 2\cos{2x} ) 
\end{aligned} \\
\begin{aligned}
  y'(0) & =  -2 (2) + 2 (c_1) = -4 + 2 c_1 \\
  y''(0) & = 4 (2) + (c_1) + 4 (-2)  = -c_1 = -4 + 2 c_1 \quad c_1 = \frac{4}{3}
\end{aligned} \\
\boxed{ y = e^{-2x} (\frac{4}{3} \sin{2x} + 2 \cos{2x} ) }
\end{gathered}
\]

\exercisehead{15} $y'' - 4y' + 29 y = 0$  \\
$ d = 16 - 4(1)(29) = -100$ \quad $\Longrightarrow u = e^{2x} (c_1 \sin{ 5x} + c_2 \cos{ 5x} )$.  
\[
\begin{gathered}
  v: \quad y'' + 4y' + 13 y = 0 \\
  d = 10 - 4(13) = -36 \quad \Longrightarrow v = e^{-2x} (b_1 \sin{3x} + b_2 \cos{3x} )  \\
  v(0) = b_2
\end{gathered}
\]
\[
\begin{gathered}
  \begin{gathered} 
    u(0) = 1 (0 + c_2 ) = c_2 = 0 \\
    \begin{aligned}
      u & = e^{2x} c_1 \sin{5x} \\
      u' & = 2 e^{2x} c_1 \sin{5x} + e^{2x} c_1 5 \cos{5x} \\
      u'\left( \frac{\pi}{2} \right) & = 1 = 2 e^{\pi} c_1 (1) \quad \, c_1 = \frac{1}{2e^{\pi}}
\end{aligned}  \\
    u'(0)  = \frac{1}{ 2 e^{\pi} } 5 \\
    u'(0) = v'(0) \Longrightarrow b_1 = \frac{5}{6 e^{\pi}}
\end{gathered} \quad \quad 
\begin{gathered}
  v = e^{-2x} b_1 \sin{ 3x} \\
  v'(0) = 3b_1  \\
\begin{aligned}
& \boxed{ u = \frac{1}{ 2e^{\pi}} e^{2x} \sin{ 5x} } \\
&  \boxed{ v = e^{-2x} \frac{5}{ 6 e^{\pi}} \sin{ 3x} }
\end{aligned}
\end{gathered}
\end{gathered}
\]

\exercisehead{16}
\[ 
\begin{gathered}
  \begin{aligned}
    y'' - 3 y' - 4 y & = 0 \quad \, & u \quad \, 9 - 4(1)(-4) = 25 \quad \, u & = e^{ \frac{3x}{2} } ( c_1 e^{ \frac{5x}{2} } + c_2 e^{ -\frac{5x}{2} } ) \\
    y'' +4 y' - 5 y & = 0 \quad \, & v \quad \, 16 - 4(1)(-5) = 36 \quad \, v & = e^{ -2x } ( b_1 e^{ 3x } + b_2 e^{-3x } ) \\
\end{aligned} \\
\begin{aligned}
  u(0) & = c_1 + c_2 = 0 
  v(0) & = b_1 + b_2 = 0 
\end{aligned} \Longrightarrow 
\begin{aligned}
  u & = 2 e^{ \frac{3x}{2} } c_1 ( \sinh{ \left( \frac{5x}{2} \right) } ) \\
  v & = 2 b_1 e^{-2x} ( \sinh{ (3x) } ) 
\end{aligned} \\
\begin{aligned}
  u' &= c_1 \frac{3}{2} e^{ \frac{3x}{2}} (e^{ \frac{5x}{2} } - e^{ \frac{-5x}{2} } ) + 2 e^{ \frac{3x}{2} } c_1 \frac{5}{2} \cosh{ \left( \frac{5x}{2} \right) }  \\
  u'(0) & = 5 c_1  
\end{aligned} \quad 
\begin{aligned}
  v' & = -4b_1 e^{-2x} \sinh{ (3x) } + 6 b_1 e^{-2x} \cosh{(3x) } \\
  v'(0) & = 6 b_1 
\end{aligned} \\
c_1 = \frac{6b_1}{5} 
\end{gathered}
\]

\exercisehead{17} 
\[
\begin{gathered}
  y'' + ky = 0 \\
  d = -4(k) \\
  \frac{ \sqrt{d}}{ 2 } = \frac{ \sqrt{ -4k}}{ 2 } = \sqrt{ - k }
\end{gathered} \quad \quad \quad 
\begin{gathered}
  \text{ Assume } k > 0 \\
  y= c_1 \sin{ \sqrt{k} x } + c_2 \cos{ \sqrt{k} x } \\
  y(0) = c_2 = 0 \\
  y(1) = c_1 \sin{ \sqrt{k} 1 } = 0 \Longrightarrow \sqrt{k} = n \pi
\end{gathered}
\]
\quad \\
\[
\begin{gathered}
  k < 0 ; \quad -k = \kappa > 0 \\
  y = c_1 e^{ \sqrt{ \kappa} x } + c_2 e^{ -\sqrt{ \kappa } x } ; \quad y(0) = c_1 + c_2 = 0 \quad y = c_1 \sinh{ \sqrt{\kappa} x } \\
  y = c_1 \sinh{ \sqrt{ \kappa } 1 } = 0 \, c_1 = 0 \\
\text{ so if $ k < 0$, there are no nontrivial solutions satisfying $f_k(0) = f_k(1) = 0$ }
\end{gathered}
\]

\exercisehead{18} $y'' + k^2y = 0 \quad \quad d = -4k^2 < 0 $ \quad \quad $\frac{ \sqrt{d}}{ 2 } = \frac{2k}{2} = k > 0 $ 
\[
\begin{gathered}
  \begin{aligned}
    y & = c_1 \sin{kx} + c_2 \cos{kx} \\
    y' & = kc_1 \cos{kx} - c_2 k \sin{kx} 
\end{aligned} \quad \quad \quad 
\begin{aligned}
  y(a) & = b = c_1 \sin{ka} + c_2 \cos{ka} \\
  y'(a) & = m = k c_1 \cos{ka} - c_2 k \sin{ka} 
\end{aligned} \\
\begin{aligned}
&  kb \cos{ka}  = k c_1 \cos{ka} \sin{ka} + c_2 k \cos^2{ka} \\
&  m \sin{ka}  = k c_1 \cos{ka} \sin{ka} - c_2 k \sin^2{ka}  \\
&  kb \cos{ka} - m \sin{ka} = c_2 k  \Longrightarrow c_2 = \frac{ kb \cos{ka} - m \sin{ka} }{ k } 
\end{aligned} \\
\begin{aligned}
  c_1 \sin{ka} & = b- c_2 \cos{ka} = \frac{kb}{k} - \left( \frac{ kb \cos{ka} - m \sin{ka} }{ k }\right) \cos{ka}  = \\
  & = \frac{ kb (1 - \cos^2{ka} ) + m \sin{ka} \cos{ka} }{ k } = \frac{ kb \sin^2{ka} + m \sin{ka} \cos{ka}}{ k } \\
  c_1 & = \frac{ kb \sin{ka} + m \cos{ka}}{ k } 
\end{aligned}  \\
y = \left( \frac{ kb \sin{ka} + m \cos{ka} }{ k}\right) \sin{kx} +  \left( \frac{kb\cos{ka} - m \sin{ka}}{ k} \right) \cos{kx}  \\
k = 0 \Longrightarrow y = mx - ma + b 
\end{gathered}
\]

\exercisehead{19} \begin{enumerate}
\item $y = k_1 \sin{x} + k_2 \cos{x} $
\[
\begin{gathered}
  \begin{aligned}
    b_2 & = k_1 \sin{(a_2)} + k_2 \cos{ (a_2) } = k_1 s_2 + k_2 c_2 \\
    b_1 & = k_1 \sin{ (a_1) } + k_2 \cos{ (a_1) } = k_1 s_1 + k_2 c_1 
\end{aligned} \quad \quad 
\begin{aligned}
  & b_2 c_1 = k_1 s_2 c_1 + k_2 c_2 c_1 \\
  & -(b_1 c_2 = k_1 s_1 c_2 + k_2 c_1 c_2 ) \\ 
  \Longrightarrow & b_2 c_1 - b_1 c_2 = k_1 (s_2 c_1 - s_1 c_2 ) 
\end{aligned} \Longrightarrow k_1 = \frac{ b_2 c_1 - b_1 c_2 }{ s_2 c_1 - s_1 c_2 } \\
\begin{aligned}
  & s_1 b_2 = k_1 s_1 s_2 + k_2 s_1 c_2 \\
  - & (s_2 b_1 = k_1 s_1 s_2 + k_2 c_1 s_2 ) 
\end{aligned} \Longrightarrow k_2 = \frac{ s_1 b_2 - s_2 b_1 }{ s_1 c_2 - c_1 s_2 } \\
\begin{aligned}
  y & = \left( \frac{ b_2 \cos{a_1} - b_1 \cos{a_2} }{ \sin{a_2} \cos{a_1} - \sin{a_1} \cos{a_2} } \right) \sin{x} + \left( \frac{ b_2 \sin{a_1} - b_1 \sin{a_2}}{ \sin{a_1} \cos{a_2} - \cos{a_1} \sin{a_2} } \right) \cos{x} \\
  y & = \boxed{ \frac{ b_2 \cos{ a_1} - b_1 \cos{a_2} }{ \sin{ (a_2 - a_1 ) } } \sin{x} + \frac{ b_2 \sin{ a_1} - b_1 \sin{a_2} }{ \sin{(a_1 - a_2 ) } } \cos{x} } \text{ if $a_2 - a_1 \neq \pi n $ } 
\end{aligned} \\
\text{ otherwise, if $a_2 - a_1 = \pi n$}; \quad \begin{aligned} b_2 c_1 - b_1 c_2 & = 0 \\ b_2 s_1 - b_1 s_2 & = 0 \end{aligned} \quad b_2 c_1 = b_1 (-1)^n c_1 ; \quad \text{ if } c \cos{(a_1) } \neq 0 , \quad b_2 = b_1 (-1)^n 
\end{gathered}
\]
\item It's true if $a_1 = a_2 = \frac{\pi}{4}; \quad \quad b_1 = b_2$.  
\item $y'' + k^2 y = 0$
\[
\begin{gathered}
  \begin{aligned}
    y & = A \sin{kx} + B \cos{kx} \\
    y(a_1) & = A \sin{ka_1} + B \cos{ka_1} = b_1 = AS_1 + BC_1 \\
    y(a_2) & = A \sin{ka_2} + B \cos{ka_2} = b_2 = AS_2 + BC_2 
  \end{aligned} \quad \, 
\begin{aligned}
  & \left[ \begin{matrix} S_1 & C_1 \\ S_2 & C_2 \end{matrix} \right] \left[ \begin{matrix} A \\ B \end{matrix} \right] = \left[ \begin{matrix} b_1 \\ b_2 \end{matrix} \right] \\
  & \left[ \begin{matrix} A \\ B \end{matrix} \right] = \left[ \begin{matrix} C_2 & -C_1 \\ -S_2 & S_1 \end{matrix} \right] \left( \frac{1}{ S_1 C_2 - C_1 S_2 } \right) \left[ \begin{matrix} b_1 \\ b_2 \end{matrix} \right] 
\end{aligned} \\
S_1 C_2 - C_1 S_2 = \sin{ka_1} \cos{ka_2} - \cos{ka_1} \sin{ka_2} = \sin{(k(a_1 - a_2 )) } \\
y = \frac{ b_1 \cos{ (ka_2) } - b_2 \cos{ (ka_1) } }{ \sin{ (k (a_1 - a_2 ) ) } } \sin{ka_1 } + \frac{ -b_1 \sin{(ka_2) } + b_2 \sin{(ka_1) } }{ \sin{ (k (a_1 -a_2) ) } } \sin{ka_2} \\
\text{ if } k(a_1 - a_2 ) \neq \pi n \quad k = 0 ; \quad \boxed{ y  = \left( \frac{ b_2 - b_1 }{ a_2 - a_1 } \right) (x-a_1) + b_1 }
\end{gathered}
\]
\end{enumerate}

\exercisehead{20} 
\begin{enumerate}
\item 
\[
u_1(x) = e^x ; \quad u_2(x) = e^{-x} \quad \quad u_2' = -e^{-x} \quad u_2'' = e^{-x} \Longrightarrow y'' - y =0
\] 
\item 
\[
\begin{gathered}
\begin{aligned}
u_1 & = e^{2x} \\
u_1' & = 2 e^{2x} \\
u_1'' & = 4 e^{2x} 
\end{aligned} \quad \quad \,
\begin{aligned}
u_2 & = xe^{2x} \\
u_2' & = e^{2x} + 2 x e^{2x} \\
u_2'' & = 2e^{2x} + 2e^{2x} + 4 x e^{2x} = 4 e^{2x} + 4 x e^{2x} = 4 e^{2x} ( 1 + x) 
\end{aligned} \\
u_2'' - 4 u_2' +4 u_2 =0 \Longrightarrow \boxed{ y'' - 4 y' +4 y = 0 }
\end{gathered}
\]
\item
\[
\begin{gathered}
\begin{aligned} u_1(x) & = e^{-x/2} \cos{x}; \\
  u_1' & = - \frac{1}{2} e^{-x/2} \cos{x} + -e^{-x/2} \sin{x} \\
  u_1'' & = \frac{1}{4} e^{-x/2} \cos{x} + e^{-x/2} \sin{x} + - e^{-x/2} \cos{x} \\
  & = \frac{-3}{4} e^{-x/2} \cos{x} + e^{-x/2} \sin{x}
\end{aligned} \quad \quad \, 
\begin{aligned}
  u_2(x) & = e^{-x/2} \sin{x} \\
  u_2' & = \frac{-1}{2} e^{-x/2} \sin{x} + e^{-x/2} \cos{x} \\
  u_2'' & = \frac{1}{4} e^{-x/2} \sin{x} + - e^{-x/2} \cos{x} - e^{-x/2} \sin{x} = \\
  & = \frac{-3}{4} e^{-x/2} \sin{x} - e^{-x/2} \cos{x} \\
  u_2'' &  + u_2' + \frac{5}{4} u_2 = 0 
\end{aligned} \\
\boxed{ y'' + y' + \frac{5}{4} y = 0 }
\end{gathered}
\]
\item $u_1(x) = \sin{ (2x+ 1) }; \quad \, u_2(x) = \sin{( 2x+2) }$   
\[
\begin{gathered}
\begin{aligned}
  u_1' & = 2 \cos{(2x+1) } \\
  u_1'' & = -4 \sin{(2x+1) }
\end{aligned} \quad \quad \,
\begin{aligned}
  u_2' & = 2 \cos{ (2x+2) } \\
  u_2'' & = -4 \sin{ (2x+2) }
\end{aligned} \\
\boxed{ y'' + 4y = 0 }
\end{gathered}
\]
\item 
\[
\begin{aligned}
  u_1 & = \cosh{x} \\
  u_1' & = \sinh{x} \\
  u_1'' & = \cosh{x} 
\end{aligned} \quad \boxed{ y'' - y = 0 } 
\]
\end{enumerate}

\exercisehead{21} $w = u_1 u_2' - u_2 u_1' $.  
\begin{enumerate}
\item $w=0 \quad \forall x \in \text{ open interval } I$, 
\[
\left( \frac{u_2}{u_1} \right)' = \frac{ u_2' u_1 - u_1' u_2 }{ u_1^2 } = 0 \Longrightarrow \frac{u_2}{u_1} = c
\]
If $\frac{u_2}{u_1}$ is not constant, then $w(0) \neq 0 $ for at least one $c$ in $I$ (otherwise, it'd be constant).  
\item $w' = u_1 u_2'' - u_2 u_1''$
\end{enumerate}

\exercisehead{22} 
\begin{enumerate}
\item $w' +aw = u_1 u_2'' - u_2 u_1'' + a (u_1 u_2' -u_2 u_1') = u_1 (-bu_2) + -u_2(bu_1) = 0 $ \\
  $w(x) = w(0)e^{-ax}$  \quad if $w(0) \neq 0 $, then $w(x) \neq 0 \quad \, \forall x$.  
\item $u_1 \neq 0$ If $w(0) = 0 , \quad w(x) = 0 \quad \forall x$, so $\frac{u_2}{u_1}$ constant.  If $\frac{u_2}{u_1}$ constant, $w(0) = 0$ since from the previous part.  
\end{enumerate}

\exercisehead{23} Recall the properties of the Wronskian.  
\begin{enumerate}
\item If $W(x) = v_1(x)v_2' - v_2 v_1' = v_1 v_2' - v_2 v_1' = 0 \quad \forall x \in I$, \\
  \phantom{ If } then $\frac{v_2}{v_1}$ constant on $I$ 
\item $W' = v_1 v_2'' - v_2 v_1'' $ 
\item $W' + a W = 0 $ if $v_1, v_2$ are solutions to $y'' + ay' + by = 0$ \\
  $W(x) = W(0) e^{-ax} $ \\
  So if $W(0) \neq 0 , \quad W(x) \neq 0 \quad \forall \, x$
\end{enumerate}
Consider adding together the solutions and the solution's derivatives into some function $f$.  By the linearity of the differential equation, we know that $f$ is also a solution since it is a linear superposition of solutions.  
\[
\begin{gathered}
  \begin{aligned}
    y(x) & = A v_1(x) + B v_2(x) \\
    f(0) & = A v_1(x) + B v_2(0) 
\end{aligned} \quad \quad 
\begin{aligned}
  y'(x) & = Av_1'(x) + Bv_2'(x) \\
  f'(0) & = Av_1'(0) + Bv_2'(0)
\end{aligned} \\
\left[ \begin{matrix} v_1(0) & v_2(0) \\ v_1'(0) & v_2'(0) \end{matrix} \right]\left[ \begin{matrix} A \\ B \end{matrix} \right] = \left[ \begin{matrix} f(0) \\ f'(0) \end{matrix} \right] \Longrightarrow \frac{1}{ W(0)} \left[ \begin{matrix} v_2'(0) & - v_2(0) \\ -v_1'(0) & v_1(0) \end{matrix} \right] \left[ \begin{matrix} f(0) \\ f'(0) \end{matrix} \right] = \left[ \begin{matrix} A \\ B \end{matrix} \right] \\
\text{ since } W(0) \neq 0 , \quad \text{ this division is allowed above }\\
\text{ so } \boxed{ y(x) = \left( \frac{ v_2'(0) f(0) - v_2(0)f'(0) }{ W(0)} \right) v_1(x) + \left( \frac{ v_1(0) f'(0) - v_1'(0) f(0) }{ W(0) } \right) v_2(x) }
\end{gathered}
\]
$f(0), \, f'(0)$ are initial conditions for $y$.  $f(0), \, f'(0)$ are arbitrary.   \\
But since $W(0) \neq 0$, \quad $W(0) = v_1(0)v_2'(0) - v_2(0)v_1'(0)$, we can do things like 
\[
f(0) = \frac{ v_2'(0) f(0) - v_2(0)f'(0)}{ v_1(0)v_2'(0) - v_2(0)v_1'(0) } v_1(0) + \frac{v_2(0)f'(0) - v_1'(0)f(0) }{ W(0)} v_2(0)
\]

%-----------------------------------%-----------------------------------%-----------------------------------
\subsection*{ 8.17 Exercises - Nonhomogeneous linear equations of second order with constant coefficients, Special methods for determining a particular solution of the nonhomogeneous equation $y'' + ay' +by = R$ }
%-----------------------------------%-----------------------------------%-----------------------------------
\quad \\

\exercisehead{1} $y'' - y = x$  \quad homogeneous solution $c_1 e^x  + c_2 e^{-x}$.  $y_p = -x$ \\
$y = c_1 e^x + c_2 e^{-x}- x$

\exercisehead{2} $y''-y' = x^2$ For the homogeneous solution
\[
y''  - y' = 0 \quad d= (-1)^2 - 4(1)(0) = 1 \quad \quad y_h = e^{\frac{x}{2} } \left( c_1 e^{\frac{x}{2} } + c_2 e^{-\frac{x}{2} } \right) = c_2e^x + c_1 
\]
\[
\begin{aligned}
y_p & = Ax^3 + Bx^2 + Cx + D \\
y_p' & = 3Ax^2 + 2Bx + C \\
y_p'' &  = 6 A x + 2B 
\end{aligned} \quad \quad 
\boxed{ y=  c_1 + c_2 e^x + \frac{-1}{3} x^3 + x^2 }
\]

\exercisehead{3} $y'' + y' = x^2 + 2x$
\[
\begin{gathered}
  e^{ - \frac{x}{2} } \left( c_1 e^{ - \frac{x}{2} } + c_2 e^{\frac{x}{2} } \right) = c_1 e^{-x} + c_2 \\
  \begin{aligned}
    & P = Ax^3 + Bx^2 + Cx + D; \quad \quad P' = 3Ax^2 + 2Bx + C \quad \quad P'' =6A x + 2B \\
    & 3Ax^2 + 2Bx + C + 6 Ax + 2B = x^2 + 2x \quad \quad A = \frac{1}{3} \quad B = 0 \quad C = 0
\end{aligned} \\
\boxed{   y = c_1 e^{-x } + c_2 + \frac{1}{3}x^3 } 
\end{gathered}
\]

\exercisehead{4} $y'' - 2y' + 3y = x^3$ \quad \quad $u = e^x ( c_1 \sin{ \sqrt{2}x} + c_2 \cos{ \sqrt{2} x} ) $ 
\[
\begin{gathered}
\begin{aligned}
  & 3 (Ax^3 + Bx^2 + Cx + D )  \\
  & 2(3Ax^2 + 2Bx + C ) \\
  & (6Ax + 2B) 
\end{aligned} \quad 
A = \frac{1}{3} \quad B= \frac{2}{3} \quad C = \frac{8}{9} \quad D = \frac{16}{27}  \\
y = C_1 e^x \sin{ \sqrt{2}x} + C_2 e^x \cos{ \sqrt{2} x } + \frac{1}{3} x^3 + \frac{2}{3} x^2 + \frac{8}{9} x + \frac{16}{27}
\end{gathered}
\]

\exercisehead{5} $y'' - 5y' + 4y = x^2 - 2x + 1 $\\
$ y_h = e^{ \frac{5x}{2} } \left( e^{\frac{3x}{2} } + e^{\frac{-3x}{2} } \right)  = c_1 e^{4x} + c_2 e^x $ \\
$d = \sqrt{25 - 4(4)} = 3$
\[
\begin{gathered}
  \begin{aligned}
  &   4(Ax^2 + Bx + C) \\
    & -5 (2Ax + B) \\
    & 2A 
\end{aligned} \quad \quad 
\begin{aligned}
  4A x^2 + 4Bx + 4C & \\
  -10 Ax - 5 B & \\
  2A & 
\end{aligned} \quad \quad A = \frac{1}{4} \quad B = \frac{1}{8} \quad \quad \begin{aligned} \frac{1}{2} - \frac{5}{8} + 4C = 1 & \\
  C = \frac{9}{32} & \end{aligned} \\
\boxed{ y = c_1 e^{4x} + c_2 e^x + \frac{1}{4} x^2 + \frac{1}{8} x + \frac{9}{32} }
\end{gathered}
\]

\exercisehead{6} 
\[
\begin{gathered}
  \begin{aligned}
    & y'' + y' - 6y = 2x^3 + 5x^2 - 7x + 2 \\
    & y_h = e^{\frac{-x}{2}} ( e^{ \frac{-5x}{2} } + e^{\frac{5x}{2}} ) = e^{-3x} + e^{2x} \\
    & d = \sqrt{ 1 - 4(-6) } =5
\end{aligned}  \\
\begin{aligned}
  y_p & = A x^3 + Bx^2 + Cx + D \\
  y_p' & = 3Ax^2 + 2Bx + C \\
  y_p'' & = 6Ax + 2B 
\end{aligned} \Longrightarrow 
\begin{aligned}
  -6A x^3 - 6Bx^2 - 6Cx - 6D & \\
  3Ax^2 + 2Bx + C & \\
  6Ax + 2B 
\end{aligned} \\
A = \frac{-1}{3} \quad B = -1 \quad C = \frac{1}{2} \quad D = \frac{-7}{12} \\
\boxed{ y = C_1 e^{-3x} + C_2 e^{2x} - \frac{1}{3} x^3 + -x^2 + \frac{1}{2} x - \frac{7}{12} }
\end{gathered}
\]

\exercisehead{7} 
\[
\begin{aligned}
  & y'' - 4y = e^{2x} \\
  & y_h = c_1 e^{2x} + c_2 e^{-2x} 
\end{aligned} \quad \quad 
\begin{aligned}
  v_1 & = e^{2x}  \quad \quad & v_2' = -2e^{-2x} \\
  v_1' & = 2 e^{2x} \quad & v_2 = e^{-2x} 
\end{aligned}
\]

Use Theorem 8.9.  
\begin{theorem}
  Let $v_1, \, v_2$ be solutions to $L(y) =0 $ where $L(y) = y'' + ay' + by$ \\
  Let $W = v_1 v_2' - v_2 v_1'$.  Then $L(y) = R$ where \\
$y_p = t_1 v_1 + t_2 v_2$ 
\begin{equation}
  t_1 = -\int v_2 \frac{R}{W(x) }dx ; \quad \quad t_2(x) = \int v_1 \frac{R}{W} dx 
\end{equation}
\end{theorem}

\[
\begin{gathered}
  \begin{aligned}
    t_1 & = - \int e^{-2x} \frac{ e^{2x}}{ -4 } = \frac{1}{4} x \\
    t_2 & = \int e^{2x} \frac{ e^{2x}}{ -4} = \frac{e^{4x}}{ -16}  
\end{aligned} \quad 
 w = e^{2x} (-2) e^{-2x} - e^{-2x} 2 e^{2x} = -4 \\
 y_p = \frac{x}{4} e^{2x} + \frac{e^{2x}}{-16} \\ 
 y = c_1 e^{2x} + c_2 e^{-2x} + \frac{x}{4} e^{2x} + \frac{e^{2x}}{-16} 
\end{gathered}
\]

\exercisehead{8} 
\[
\begin{gathered}
  \begin{aligned}
    & y'' + 4y = e^{-2x} \\
    & y_h = c_1 \sin{2x} + c_2 \cos{2x} 
\end{aligned} \quad \quad w = \sin{2x} - \sin{2x}(2) - 2\cos{2x} \cos{2x} = 2 \\
  \begin{aligned}
    t_1 & = - \int \frac{ \cos{2x} e^{-2x} dx }{ 2 } = \frac{-1}{2} \int e^{-2x} \cos{2x} dx \\
    t_2 & = \int \frac{ \sin{2x} e^{-2x}}{ 2 } = \frac{1}{2} \int e^{-2x} \sin{2x} dx 
\end{aligned} \quad \quad 
\begin{aligned}
(e^{-2x} \cos{2x} )' & = -2e^{-2x} \cos{2x} + -2e^{-2x} \sin{2x} \\
  (e^{-2x} \sin{2x} )' & = -2e^{-2x} \sin{2x} + 2e^{-2x} \cos{2x} 
\end{aligned} \\
\begin{aligned}
  \left( \frac{ e^{-2x} \sin{2x} - e^{-2x} \cos{2x} }{ 4} \right)' & = e^{-2x} \cos{2x} \\
  \left( \frac{ e^{-2x} \sin{2x} + e^{-2x} \cos{2x} }{ -4} \right)' & = e^{-2x} \sin{2x}  
\end{aligned}
\quad \quad 
\begin{aligned}
t_1 & = \frac{ e^{-2x} \cos{2x} - e^{-2x} \sin{2x} }{ 8 } \\
t_2 & = \frac{ e^{-2x} \sin{2x} + e^{-2x} \cos{2x} }{ -8 } 
\end{aligned} \\
y_p = \frac{ e^{-2x} \sin{2x} \cos{2x} - e^{-2x} \sin^2{2x} }{ 8 } + \frac{ e^{-2x} \sin{2x} \cos{2x} + e^{-2x} \cos^2{2x} }{ -8} = \boxed{ \frac{e^{-2x}}{ 8 } } \\
\boxed{ y = c_1 \sin{2x} + c_2 \cos{2x} + \frac{e^{-2x}}{8} }
\end{gathered}
\]

\exercisehead{9} $y'' + y' -2y  = e^x \quad \quad d^2 = 1-  (4)(1)(-2) = 9$
\[
\begin{gathered}
  y_h = e^{ \frac{-1x}{2}} \left( e^{ \frac{3x}{2} } + e^{\frac{-3x}{2}} \right) = e^x + e^{-2x} \\
  \begin{aligned}
    (xe^x)' & = e^x + xe^x \\
    + (xe^x)'' & = +( 2e^x + xe^x )  \\
\Longrightarrow 3e^x + 2 x e^x 
\end{aligned} \quad \quad 
\boxed{ y = c_1 e^x + c_2 e^{-2x} + \frac{1}{3} xe^2 }
\end{gathered}
\]

\exercisehead{10}
$y'' + y' -2y = e^{2x} $.  $y_h = e^{-\frac{x}{2}} \left( c_1 e^{\frac{3x}{2} } + c_2 e^{ - \frac{3x}{2} } \right) = c_1 e^x + c_2 e^{-2x} $.  

\[
\begin{gathered}
  W(x) = v_1 v_2' - v_2 v_1' = e^x (-2) e^{-2x} - e^{-2x}e^x = -3e^{-x} \\
  \begin{aligned}
t_1 & = -\int \frac{v_2 R}{ W} \quad & \quad t_2 & = \int \frac{v_1 R }{ W} \\
t_1 & = - \int \frac{ e^{-2x}e^{2x}}{ -3 e^{-x}} = \frac{1}{3} e^x \quad & \quad & t_2 = \int \frac{ e^x e^{2x}}{ -3 e^{-x}} = \frac{-1}{12} e^{4x } 
  \end{aligned} \\
  y_1 = t_1 v_1 + t_2 v_2 = \frac{1}{3} e^x e^x + - \frac{1}{12} e^{4x}e^{-2x} = \frac{1}{4} e^{2x} \\
  y = c_1 e^x + c_2 e^{-2x} + \frac{1}{4} e^{2x}
\end{gathered}
\]

\exercisehead{11} $y''+y'-2y = e^x + e^{2x}$.   \bigskip \\
Consider solutions to Exercise 9,10.  
\[
\begin{gathered}
  L(y_a)  = e^x ; \, L(y_b) = e^{2x} \, ; L(y_a+ y_b) = e^x + e^{2x} \\
  \Longrightarrow \boxed{ y = c_1 e^x + c_2 e^{-2x} + \frac{1}{3} x e^x + \frac{1}{4} e^{2x} }
\end{gathered}
\]

\exercisehead{12} $y''-2y'+y = x + 2x e^x$.  $d = 4-4(1) = 0$.  Recall the definition to be learned for this section of exercises:
\begin{theorem}
Let $d= a^2 -4b$ be the discriminant of $y''+ay'+by=0$.  Then every solution of this equation on $(-\infty,\infty)$ has the form 
\begin{equation}
  y = e^{-ax/2} (c_1 u_1(x) + c_2 u_2(x) )
\end{equation}
\begin{enumerate}
  \item  If  $d = 0$  then  $u_1 = 1 , u_2 = x$  
  \item If $d>0$, $u_1 = e^{kx}$; $u_2 = e^{-kx}, k =\frac{ \sqrt{d}}{ 2 } $ 
  \item If $d<0$, $u_1 = \cos{kx}, u_2 = \sin{kx}, k = \frac{ \sqrt{-d}}{2} $
\end{enumerate}
\end{theorem}

\[
\begin{gathered}
  y_h = e^x ( c_1 + c_2 x ) = c_1 e^x + c_2 xe^x \\
  t_1 = \int \frac{ -x e^x (2x e^x )}{ e^{2x}} = \frac{2x^3}{ -3} \, \quad t_2 = \int \frac{ e^x (2x e^x)}{ e^{2x} } = x^2 \\
  W(x) = e^x(e^x + xe^x ) -(xe^x)(e^x) = e^{2x} \\
  y_p = \frac{ 2x^3}{-3} e^x + x^3 e^x = \boxed{ \frac{ x^3 e^x}{ 3 } }\\
    \boxed{ y = \frac{ x^3 e^x}{3} + c_1 e^x + c_2 x e^x }
\end{gathered}
\]

\exercisehead{13} $y''+2y' +y = \frac{e^{-x}}{x^2 } $
\[
\begin{gathered}
  y_h = e^{-x} (c_1 + c_2 x ) = c_! e^{-x} + c_2 x e^{-x} \\
  t_1 = \int \frac{ -x e^{-x} \left( \frac{e^{-x}}{ x^2 } \right) }{ e^{-2x} } = -\ln{x} \quad t_2 = \int \frac{ e^{-x} \left( \frac{e^{-x}}{ x^2 } \right) }{ e^{-2x} } = \frac{-1}{x} \\
  W = e^{-x} (e^{-x} + -x e^{-x} ) - (-e^{-x})(xe^{-x}) = e^{-2x} \\
  y_p = -\ln{x} e^{-x} + xe^{-x} \left( \frac{ -1}{x} \right) = -\ln{x} e^{-x} -e^{-x} \\
  \boxed{ y = c_1 e^{-x} +c_2 x e^{-x} + (-\ln{x} -1)e^{-x}}
\end{gathered}
\]

\exercisehead{14} $y'' + y = \cot^2{x}$.  $y_h = c_1 \sin{x} + c_2 \cos{x}$.  
\[
\begin{gathered}
  \begin{aligned}
    t_1 &= \int \frac{ -\cos{x} }{ -1 } \cot^2{x} dx = \int \frac{ \cos{x} \cos^2{x}}{ \sin^2{x}} = \int \frac{ \cos{x}(1-\sin^2{x} ) }{ \sin^2{x}} = -\frac{1}{\sin{x}} + - \sin{x} \\
    t_2 & = \int \frac{ \sin{x}}{ -1} \cot^2{x} = -\int \frac{ \cos^2{x}}{ \sin{x}} = -\int \frac{ 1 - \sin^2{x}}{ \sin{x} } = \ln{ |\csc{x}+ \cot{x}| } + -\cos{x} 
  \end{aligned} \\
y_p = -1 - \sin^2{x} + \cos{x} \ln{ |\csc{x} + \cot{x}| } - \cos^2{x} = \boxed{ -2 + \cos{x} \ln{ |\csc{x} + \cot{x} | } } \\
\boxed{ y = c_1 \sin{x} + c_2 \cos{x} -2 + \cos{x} \ln{ |\csc{x} + \cot{x} | } }
\end{gathered}
\]

\exercisehead{15} $y'' - y = \frac{2}{1+e^x}$ \medskip \\
$y_h = c_1 e^x + c_2 e^{-x}$  \quad $\Longrightarrow W(x) = -e^x e^{-x} - e^x e^{-x} = -2$  

\[
\begin{gathered}
  \begin{aligned}
    t_1 & = - \int \frac{ e^{-x} \frac{2}{1+e^x} }{ -2} = \int \frac{e^{-x}}{1+e^x} = \\
    & = \int \frac{1}{e^x} - \frac{1}{ 1+e^x}  = -e^{-x} - \int \frac{1}{1+e^x} = -e^{-x} + \int \frac{-e^{-x}}{ e^{-x} + 1 } = -e^{-x} + \ln{(1+e^{-x}) } 
    \end{aligned} \\
  \begin{aligned}
t_2 = \int \frac{ e^x \frac{2}{1+e^x} }{ -2 } = -\ln{ |1+e^x | } 
\end{aligned} \\
  \boxed{ y = -1 + e^x \ln{ (1+e^{-x})} + -e^{-x} \ln{ (1+e^x) } + c_1 e^x + c_2 e^{-x} }
\end{gathered}
\]

\exercisehead{16} $y''  +y' - 2y = \frac{e^x}{1+e^x} $ \medskip \\
Discriminant: $ \frac{-1 \pm \sqrt{ 1^2 - 4 (-2) }}{ 2 } = -2, \, 1$ \quad $\Longrightarrow y_h = c_1 e^x + c_2 e^{-2x}$ \quad \, $\Longrightarrow W = e^x (-2)e^{-2x} - e^{-2x} e^x = -3 e^{-x} $ 
\[
\begin{gathered}
  \begin{aligned}
    t_1 = - \int \frac{e^{-2t} \left( \frac{e^t}{ 1 + e^t} \right) }{ -3 e^{-t} } = \frac{1}{3} \int \frac{1}{1 + e^t } = \frac{-1}{3} \ln{ (1+e^{-x})} 
  \end{aligned} \\
  \begin{aligned}
    t_2 & = \int \frac{e^t \left( \frac{e^t}{ 1 + e^t} \right) }{ -3e^{-t} } = \frac{1}{-3} \int \frac{e^{3t}}{ 1 + e^t} \xrightarrow{ u =e^t} \frac{-1}{3} \int \frac{ u^2 du }{ 1 + u } = \\ 
    & = \frac{-1}{3} \int u + \frac{-u}{u+1} = \frac{-1}{3} \int u + -1 + \frac{1}{u+1} = \frac{-1}{3} (\frac{1}{2} u^2 - u + \ln{u+1} ) \\
    & = \frac{-1}{6} e^{2x} + \frac{e^x}{3} + \frac{-1}{3} \ln{ (e^x + 1 ) } 
  \end{aligned} \\
  \boxed{ y_1 = \frac{-1}{3} e^x \ln{ (1+e^{-x}) } + \frac{-1}{6} + \frac{e^{-x}}{3} - \frac{e^{-2x}}{3} \ln{ (e^x+ 1 ) } + c_1 e^x + c_2 e^{-2x} }
\end{gathered}
\]

\exercisehead{17} $y' + 6y' + 9y = f(x)$; where $f(x) =1$ for $1 \leq x \leq 2$.  $f(x) = 0$ for all other $x$.  
\[
\begin{gathered}
  d = 36 - 4(1)(9) = 0 \\
  y_h = e^{-3x}(c_1 + c_2 x ) = c_1 e^{-3x} + c_2 x e^{-3x} \\
  \begin{aligned}
  W(x) & = e^{-3x} ( e^{-3x} -3x e^{-3x} ) - (xe^{-3x} )(-3e^{-3x}) = e^{-6x} \\
  t_1(x) & = \begin{cases} a < 1 < x & \int_1^x -te^{3t} dt = \left. \left( \frac{ -t e^{3t}}{ 3 } + \frac{ e^{3t}}{ 9 } \right) \right|_1^x = \frac{ -3x e^{3x} + e^{3x}}{ 9 } + \frac{2}{9}e^3 \\ 
    a < 1 < 2 < x & \int_1^2 -t e^{3t}dt = \frac{ -6e^6 + e^6}{ 9 } + \frac{ 2 e^3}{ 9} = \frac{ -5 e^6}{ 9} + \frac{2 e^3}{ 9 }  \\
    1 < a < x <2 & \int_a^x -t e^{3t} dt = \left. \left( \frac{ -t e^{3t}}{3 } + e^{3t}{ 9} \right) \right|_a^x  = \frac{ -x e^{3x}}{ 3 } + \frac{ e^{3x}}{ 9 } + \frac{ a e^{3a}}{ 3 } - \frac{ e^{3a}}{ 9 }  
\end{cases}  \\
  t_2(x) & = \int \frac{ e^{-3t} f(t) }{ e^{-6t} } = \int e^{3t} f(t) = \int_a^x e^{3t} = \frac{1}{3} \left( e^{3x} - e^{3a} \right) 
\end{aligned}\\
  y_1 = e^{-3x} \left( \frac{ -3x e^{3x} + e^{3x} }{ 9} + C \right) + \frac{x}{3} \\
  \boxed{ y = c_1 e^{-3x} + c_2 x e^{-3x} + \frac{1}{9} \text{ when } 1 \leq x \leq 2; \text{ otherwise } y =y_h }
\end{gathered}
\]

\exercisehead{18} Start from $y'' - k^2 y = R(x)$.  Suppose $L(y_p) = y_p'' - k^2 y_p = R(x)$.  \medskip \\
$y_h = c_1 \sinh{(kx)} + c_2 \cosh{ (kx)}$; \quad \, $L(y_h) = 0$ \medskip \\
So consider $L(y_p + y_h) = R(x)$; $y_p + y_h = y_1$ is a nother particular solution. \bigskip \\

The \emph{key to this problem} is to \emph{apply the integration directly on the ODE itself}, not to go the other way around by differentiating the supposed particular solution.  
\[
\begin{gathered}
\begin{aligned}
  \xrightarrow{ \int_0^x dt \sinh{ (k(x-t)) } } & \int_0^x dt \frac{d^2y}{dt^2}(t) \sinh{ (k(x-t) ) } -k^2 \int_0^x dt y(t) \sinh{ (k(x-t)) } = \int_0^x dt R(t) \sinh{ (k(x-t)) } dt 
\end{aligned} \\
\begin{aligned}
  \int_0^x y'' \sinh{(\kappa)} & = -y'(0) \sinh{ (kx)} - \int y' \cosh{(\kappa)}(-k) = \\ 
  & = -y'(0) \sinh{(kx)} + k(y(x) - y(0) \cosh{ (kx)} + k \int y(t) \sinh{(\kappa)} ) = \\
  & = -y'(0) \sinh{(kx)} + ky(x) - ky(0) \cosh{(kx)} + k^2 \int y(t)\sinh{(\kappa)}
\end{aligned} \\
\Longrightarrow y(x) - \frac{ y'(0) \sinh{(kx)}}{ k } - y(0) \cosh{(kx)} = \frac{1}{k} \int_0^x dt R(t) \sinh{ (k(x-t))}dt 
\end{gathered}
\]
Now note that $L(y_h)=0$, so applying $\int_0^x dt \sinh{(k(x-t))}$ results in $0$ still.  \medskip \\
With $y_p(x) = \frac{1}{k}\int_0^x dt R(t) \sinh{ (k(x-t))} dt + \frac{ y'(0) \sinh{(kx)} }{ k} + y(0) \cosh{ (kx)}$, we can add a homogeneous solution of $\frac{y'(0) \sinh{(kx)} }{ k} + y(0) \cosh{(kx)}$ to $y_p(x)$ to obtain 
\[
\boxed{ y_1(x) =\frac{1}{k} \int_0^x dt R(t) \sinh{ (k(x-t))} dt  }
\]

Now for $y'' - 9y = e^{3x}$,  
\[
\begin{gathered}
  \begin{aligned}
    y_1(x) & = \frac{1}{3} \int_0^x dt (e^{3t})\sinh{ (3(x-t) )} dt = \frac{1}{6} \int_0^x dt e^{3t} \left( \frac{ e^{3x- 3t} - e^{-3x + 3t} }{2} \right) = \frac{1}{6} \int_0^x dt (e^{3x} - e^{-3x + 6t} ) \\
    & = \frac{1}{6} \left( e^{3x}x - e^{-3x} \left. \frac{1}{6} e^{6t}  \right|_0^x \right) = \frac{1}{6} xe^{3x} - e^{-3x} \frac{ e^{6x} - 1 }{ 36} = \frac{1}{6} ( xe^{3x} - \frac{e^{3x}}{6} + \frac{e^{-3x}}{6} ) \\
    y_1' & = \frac{1}{6} (e^{3x} + 3x e^{3x} - \frac{3}{2} e^{3x} - \frac{3}{2} e^{-3x} ) \\
    y_1'' & = \frac{1}{6} \left( 3e^{3x} + 3e^{3x} + 9xe^{3x} - \frac{9}{2} e^{3x} + \frac{9}{2} e^{-3x} \right) = \frac{1}{4} e^{3x} + \frac{3}{4} e^{-3x} + \frac{3}{2} xe^{3x}  \\
    y_1'' - 9 y_1 & = \frac{1}{2} e^{3x} + \frac{1}{2} e^{-3x}
  \end{aligned} \\
\text{ Thus, we need to add homogeneous parts to our particular solution to make it work.  So if } \\
\boxed{ y_p = \frac{xe^{3x}}{6} - \frac{e^{3x}}{y} + \frac{e^{-3x}}{6} } \\
\text{ then it could be checked easily with some computation, that this satisfies the ODE. }
\end{gathered}
\]

\exercisehead{19} Start from $y'' + k^2 y = R(x)$ \smallskip \\
Again, note that if $L(y_p) = y_p'' + k^2 y_p = R(x)$, $L(y_p +y_h) = R(x) + 0 = R(x)$, so $y_1 = y_p + y_h$ is also a particular solution.

\[
\begin{gathered}
  \xrightarrow{ \int_0^x dt \sin{ k(x-t)} } \int_0^x dt \sin{k(x-t)} \frac{d^2y}{dt^2} + k^2 \int_0^x dt \sin{ k(x-t)y } = \int_0^x R(t) \sin{ k(x-t) } \\
  \begin{aligned}
    \int_0^x dt \sin{k(x-t)} \frac{d^2y}{dt^2} & = -y'(0) \sin{(kx)}  + k \int_0^x y'(t) \cos{(k(x-t))} dt = \\
    & = -y'(0) \sin{(kx)} + k (y(x) - y(0) \cos{(kx)} - k \int_0^x y(t) \sin{(k(x-t))} dt ) = \\
    & = ky(x) - ky(0) \cos{(kx)} - y'(0) \sin{(kx)} - k^2 \int_0^x y(t) \sin{(k(x-t))} dt 
  \end{aligned} 
\end{gathered}
\]
\[
\begin{gathered}
  \Longrightarrow y(x) = \frac{1}{k} \int_0^x dt R(t) \sin{k(x-t)} + y(0) \cos{(kx)} + \frac{ y'(0) \sin{(kx)} }{k} \\
  \text{ We can add $y_h$ with $c_1 = -y(0)$, \quad $c_2 = \frac{-y'(0)}{k} $ } \\
  y_1 = \frac{1}{k} \int_0^x dt R(t) \sin{k(x-t) }
\end{gathered}
\]

Now for $y'' + 9y = \sin{3x}$, then $k=3$, 
\[
\begin{gathered}
  y_1 = \frac{1}{3} \int_0^x \sin{3t} \sin{3(x-t)} dt \\
  \int_0^x \sin{3t} (\sin{3x} \cos{3t} - \cos{3x} \sin{3t} ) \\
  \begin{aligned}
    \int_0^x sc & = \int_0^x \frac{s(6t)}{2} dt  = \frac{-1}{12} (c(6x) - 1 ) \\
    \int_0^x s^2 & = \int_0^x \frac{ 1 - \cos{(6t)}}{2} = \left. \frac{x}{2} - \frac{ \sin{ (6t) }}{12} \right|_0^x = \frac{x}{2} - \frac{ \sin{(6x)}}{12 } 
  \end{aligned} \\
  y_1 = \frac{1}{3} \left( \sin{3x} \frac{-1}{12} (\cos{(6x)} - 1 ) - \cos{3x} \left( \frac{x}{2} - \frac{\sin{(6x)}}{12} \right) \right) = \boxed{ \frac{\sin{3x}}{18}  - \frac{ x \cos{3x} }{6}  }
\end{gathered}
\]
It could be shown with some computation that this particular solution satisfies the ODE without having to add or subtract parts of a homogeneous solution.  

\exercisehead{20} $y'' +y = \sin{x}$ \medskip \\
$y_h = c_1 \sin{x} + c_2 \cos{x}$  \quad \quad \, $\Longrightarrow W(x) = -s^2 - c^2 = -1$  
\[
\begin{gathered}
  t_1 = - \int \frac{ cs}{-1} = \frac{ - \cos{2x}}{4} ; \quad \, t_2 = \int \frac{ s s}{-1} = - \int \frac{1- \cos{2x}}{2} = - \left( \frac{x}{2} - \frac{ \sin{2x}}{4} \right) \\
  y_p  = - \frac{ \sin{x} \cos{2x}}{4} + \left( \frac{ \sin{2x} - 2x }{ 4 } \right) \cos{x} =  \frac{ \sin{x} \cos^2{x} + \sin^3{x} - 2x \cos{x}}{ 4 }   \\
\boxed{ y = c_1 \sin{x} + c_2 \cos{x} + \frac{ \sin{x} \cos^2{x} + \sin^3{x} - 2 x \cos{x}}{ 4 } }
\end{gathered}
\]

\exercisehead{21} $y''+y = \cos{x}$ \medskip \\
$y_h = c_1 \sin{x} + c_2 \cos{x} = c_1 S + c_2 C$ \quad \quad $W(x) = -1$
\[
\begin{gathered}
  t_1 = \int \frac{ - CC}{-1} = \int \frac{1 + \cos{2x}}{2} = \frac{ x + \frac{\sin{2x}}{2} }{2 } ; \quad \quad \, t_2 = \int \frac{ SC}{-1} = \frac{ \cos{2x}}{4} \\
  y_p = \frac{ x \sin{x}}{2} + \frac{ \sin{2x} \sin{x}}{4} + \frac{ \cos{x} \cos{2x}}{4} \\
\Longrightarrow \boxed{ y = \frac{ x\sin{x}}{2} + \frac{ \sin{2x} \sin{x}}{4} + \frac{ \cos{x} \cos{2x}}{4} + c_1 \sin{x} + c_2 \cos{x}} 
\end{gathered}
\]

\exercisehead{22} $y'' + 4y = 3x \cos{x}$ \medskip \\
$y_h = c_1 \sin{2x} + c_2 \cos{2x}$ \quad \quad \, $W(x) = -\sin^2{2x}(2) + -\cos^2{2x}(2) = -2 $  
\[
  \begin{aligned}
    t_1 & = \int \frac{ -\cos{(2x)} (3x \cos{x} )}{ -2} = \frac{3}{2} \int x \cos{x} \cos{ (2x) }  = \frac{3}{2} \int x c (1-2s^2 ) = \frac{3}{2} \int xc - 3 \int x cs^2 = \\ 
    & = \frac{3}{2} (xs+c) - 3 \int x \left( \frac{c^3}{3} \right)'  = \frac{3}{2} (xs+c) - (xs^3 -\int s^3 ) = \frac{3}{2} (xs+c) - xs^3 + \int s(1-c^2) = \\
    & = \frac{3}{2} (xs+c) - xs^3 + -c + \frac{1}{3}c^3 = \frac{3}{2} xs + \frac{c}{2} - xs^3 + \frac{1}{3} c^3 
  \end{aligned} 
\]
\[
\begin{gathered}
\begin{aligned}
  t_2 & = \int \frac{ \sin{(2x)} (3x\cos{x}) }{ -2} = -3 \int xsc^2 = \int x(c^3)' = xc^3 - \int c^3 = xc^3 - \int c(1-s^2) = \\
  & = xc^3 - s + \frac{1}{3} s^3 
\end{aligned} \\
\begin{aligned}
  y_p & = \left( \frac{3}{2} xs + \frac{c}{2} - xs^3 + \frac{1}{3} c^3 \right)(2sc) + (xc^3 - s + \frac{s^3}{3} )(1 - 2s^2) = \text{ (lots of algebra) } = \\
  & = \boxed{ xc^2 + \frac{2}{3}s }
\end{aligned} \\
\Longrightarrow \boxed{ y = c_1 \sin{2x} + c_2 \cos{2x} + x\sin{x} - \frac{2}{3} \cos{x} }
\end{gathered}
\]
Remember, \emph{ persistence is key } to work through the algebra, quickly.  

\exercisehead{23} $y'' +4 y = 3x\sin{x}$.  From the work above, we could guess at the solution.  
\[
\begin{gathered}
  \begin{aligned}
    (xs)' & = s + xc \\
    (xs)'' & = 2c + -xs 
  \end{aligned} \quad \quad \, 
  \begin{aligned}
    & (xs)'' + 4(xs) = 2c - xs + 4xs = 2c + 3xs
    & (c)'' + 4 c = 3c \Longrightarrow \left( \frac{-2}{3} c \right)'' + 4 \left( \frac{-2}{3} c \right) = -2c 
  \end{aligned} \\
  \Longrightarrow \boxed{ y_h = xs - \frac{2}{3} c } \\
  \boxed{ y = x\sin{x} - \frac{2}{3} \cos{x} + c_1 \sin{2x} + c_2 \cos{2x} }
\end{gathered}
\]

\exercisehead{24} $y'' - 3y' = 2e^{2x} \sin{x}$  Guessing and stitching together the solution seems easier to me.  
\[
\begin{gathered}
  \begin{aligned}
    (e^{2x} c)' & = 2e^{2x} c + -e^{2x} s \\
    (e^{2x} c)'' & = 4 e^{2x} c + - 4 e^{2x} s - e^{2x} c = 3e^{2x} c - 4 e^{2x} s \\
    (e^{2x} c)'' - 3(e^{2x} c)' & = 3e^{2x}c - 4 e^{2x} s - 6 e^{2x}c + 3e^{2x} s = \\
    & = -3e^{2x} c - e^{2x} s 
  \end{aligned}  \quad \,
  \begin{aligned}
    (e^{2x}s)' & = 2 e^{2x} s + e^{2x}c \\
    (e^{2x}s)'' & = 4 e^{2x} s + 4 e^{2x} c + -s e^{2x} = \\
    & = 3 e^{2x} s + 4 e^{2x} c \\
    (e^{2x}s)'' - 3(e^{2x}s)' & = 3 e^{2x} s + 4 e^{2x} c - 6 e^{2x}s - 3 e^{2x} c = \\ 
    & = -3e^{2x}s + e^{2x} c
  \end{aligned} \\
  (3e^{2x}s)'' - 3(3e^{2x}s)' + (e^{2x}c)'' - 3(e^{2x}c)' = -10 e^{2x} s \\
  \Longrightarrow \boxed{ y_p = \frac{e^{2x}(3\sin{x} + \cos{x})}{5} } \Longrightarrow  \boxed{ y = c_1 \sin{\sqrt{3}x} + c_2 \cos{\sqrt{3}x} + \frac{ e^{2x} (3\sin{x} + \cos{x}) }{-5} }
\end{gathered}
\]

\exercisehead{25} $y'' + y = e^{2x} \cos{3x}$.  

\[
\begin{gathered}
  \begin{aligned}
    (e^{2x} c(3x))'' & = 4 e^{2x} c(3x) + -12 e^{2x} s(3x) - 9 e^{2x} c(3x) = -5 e^{2x} c(3x) - 12 e^{2x} s(3x) \\
    (e^{2x}s(3x))'' & = 4e^{2x} s(3x) + 12 e^{2x} c(3x) + -9e^{2x} s(3x) = -5e^{2x} s(3x) + 12 e^{2x} c(3x) 
  \end{aligned} \\
  L(e^{2x} c(3x) - 3e^{2x} s(3x) ) = -40 e^{2x} \cos{(3x)} \\
  \boxed{ y = c_1 \sin{x} + c_2 \cos{x} + \frac{ e^{2x} \cos{(3x)} - 3 e^{2x} \sin{(3x)} }{-40 } }
\end{gathered}
\]



%-----------------------------------%-----------------------------------%-----------------------------------
\subsection*{ 8.19 Exercises - Examples of physical problems leading to linear second-order equations with constant coefficients }
%-----------------------------------%-----------------------------------%-----------------------------------
\quad \\
In exercises 1-5, a partcile is assumed to be moving in simple harmonic motion, according to the equation $y = C\sin{ (kx + \alpha) }$.  The velocity of the particle is defined to be the derivative $y'$.  The frequency of the motion is the reciprocal of the period.  (Period $= 2\pi/k$, frequency $= k/2\pi$ )

\exercisehead{1} Find the amplitude $C$ if the frequency is $1/\pi$ and if the initial values of $y$ and $y'$ (when $x=0$) are $2$ and $4$, respectively.
\[
\begin{gathered}
  \text{ frequency } = \frac{k}{2\pi} = \frac{1}{\pi} \quad \Longrightarrow k =2  \\
  \begin{aligned}
    y(x=0) & = C \sin{\alpha} \\
    y'(x=0) & = C\cos{\alpha}
  \end{aligned} \quad \Longrightarrow \frac{ y(x=0)}{ y'(x=0)} = \frac{1}{k} \tan{\alpha} = \frac{1}{2} \\
   \alpha = \frac{\pi}{4}  \text{ and } \boxed{ C = 2 \sqrt{2} }
\end{gathered}
\]

\exercisehead{2} Find the velocity when $y$ is zero, given that the amplitude is $7$ and the frequency is $10$.  
\[
\begin{gathered}
\begin{aligned}
  y & = C \sin{ (k x + \alpha )} \\
  y' & = C k \sin{ (kx + \alpha) }
\end{aligned} \quad \, C = 7 \quad \quad \, \frac{k}{2\pi} = 10 \Longrightarrow k = 20 \pi \\
y(x = \frac{-\alpha}{k} ) = 0 \Longrightarrow y'(x = \frac{-\alpha}{k} ) = \boxed{ 140 \pi }
\end{gathered}
\]

\exercisehead{3} 
\[
\begin{gathered}
  y = A \cos{ (mx + \beta) } \\
  \begin{aligned}
    y & = A \cos{ (mx + \beta) } = A \cos{\beta} \cos{ (mx) } - A \sin{\beta} \sin{(mx)} \\
    y & = C \sin{kx+\alpha} = C \cos{ \alpha} \sin{kx} + C \sin{\alpha} \cos{kx}
  \end{aligned} \quad \, \Longrightarrow \boxed{ k = m } \text{ (since $x$ is arbitrary )} \\
  \begin{aligned}
    -A \sin{\beta} & = C \cos{\alpha} \\
    A \cos{\beta} & = C \sin{\alpha} 
\end{aligned} \quad \, \Longrightarrow \tan{\alpha} = \pm \cot{\beta} = \mp \tan{ \left( \frac{\pi}{2} -\beta \right) } = \tan{ (\beta  - \frac{\pi}{2} ) } \\
  \Longrightarrow \boxed{ \alpha = \beta - \frac{ \pi}{2} }  \text{ and } \boxed{ |C| = |A| }
\end{gathered}
\]

\exercisehead{4} $\frac{2\pi}{T} = 4 \pi$ \medskip \\
$y= C \cos{ (kx +\alpha)} = C \cos{(kx)} = \boxed{ 3 \cos{ (4\pi x) } }$

\exercisehead{5} $y = C \cos{ (x+\alpha)}$  \quad $y_0 = C \cos{ (x_0 + \alpha) }$ \medskip \\
$y' = -C \sin{ (x+\alpha)} = \pm v_0$ \medskip \\
$v_0^2 + y_0^2 = C^2 \sin^2{(x+\alpha) } + C^2 \cos^2{ (x_0 + \alpha) } = C^2$  \quad $\Longrightarrow \boxed{ C = \sqrt{ v_0^2 + y_0^2 } }$

\exercisehead{6} 
\[
\begin{gathered}
  \begin{aligned}
    y & = C \cos{ (kx+\alpha) } \\
    y(0) & = C \cos{(\alpha)} = 1 \\
    y''(0) & = -k^2 C \cos{(\alpha)} = -12 
\end{aligned} \quad \, \begin{aligned}
    y' & = -kC \sin{ (kx +\alpha) } \\
    y'(0) & = -kC \sin{ (\alpha)} = 2 \\
    \frac{ y''(0)}{y(0)} & = -k^2 = \frac{-12}{1}  \Longrightarrow k = 2 \sqrt{3}
  \end{aligned} \\
  \frac{y'(0)}{y(0)} = \frac{ -kC \sin{(\alpha)}}{ C \cos{(\alpha)}} = \frac{2}{1} = 2 = -k\tan{(\alpha)}  \quad \Longrightarrow \boxed{ \alpha = \frac{-\pi}{6} } 
\end{gathered}
\]

\exercisehead{7} $k =\frac{2\pi}{3}$  \quad $y = -C \sin{(kx)}$  \medskip \\
$\boxed{ y = - C \sin{ \frac{2\pi x}{3} }}$; \quad \, $ C > 0$

\exercisehead{8}
Let's first solve the homogenous equation.  
\[
\begin{gathered}
  y'' + y = 0 \\
  \begin{aligned}
    y_h & = C_1 \sin{x} + C_2 \cos{x} \\
    W(x) & = -S^2 - C^2 = 1 
  \end{aligned} \quad \quad \begin{aligned}
    t_1 & = \int_0^x \frac{ -\cos{t} (1) }{ -1 } = \sin{x} \\
    t_2 & = \int_0^x \frac{ (\sin{t}) (1) }{ -1 } = (\cos{x} - 1 ) 
  \end{aligned} \\
  \text{ for } 0 \leq x \leq 2 \pi \text{ otherwise, for } x > 2 \pi, \, t_1 = 0, \, t_2 = 0 \\
\begin{aligned}
  y_1 & = \sin^2{x} + \cos^2{x} + (1-\cos{x} ) \\
  y'(x) & = C_1 \cos{x} + \sin{x} 
\end{aligned}  \quad \quad \, \begin{aligned} 
y(0) & = 0 = c_2 \\
y'(0) & = c_1 = 1 
\end{aligned} \\
y = \sin{x} + (1-\cos{x} ) \\
\Longrightarrow \boxed{ I(t) = \sin{t} + (1-\cos{t} ) } \quad \, 0 \leq t \leq 2 \pi 
\end{gathered}
\]

\exercisehead{9} 
\begin{enumerate}
  \item Consider large $t$.  Then $I(t) = F(t) + A \sin{ (\omega t + \alpha) } \to A \sin{ (\omega t + \alpha) }$
\[
\begin{gathered}
\begin{aligned}
  I & = A S(\omega t + \alpha )  = A ( S(\omega t ) C(\alpha ) + C(\omega t) S(\alpha ) ) \\
  I' & = \omega A C(\omega t + \alpha )  = \omega A ( C(\omega t ) C(\alpha ) - S(\omega t) S(\alpha ) ) \\
  I'' & = -\omega^2 A S(\omega t + \alpha)  = -\omega^2 A ( S(\omega t ) C(\alpha) + C(\omega t) S(\alpha) ) 
\end{aligned} \\
I'' + RI' + I = I'' + I' + I = \\
= A ((-\omega^2 C(\alpha) + - \omega S(\alpha) + C(\alpha) )S(\omega t) + (-\omega^2 S(\alpha) + \omega C(\alpha) + S(\alpha) ) C(\omega t) ) = S(\omega t ) \\
\Longrightarrow \tan{ (\alpha) } = \frac{-\omega }{1 - \omega^2 }  
\end{gathered}
\]
With $\tan{\alpha} = \frac{-\omega}{ 1 - \omega^2}$ and the trig identities $t^2 + 1 = \sec^2$, $\frac{1}{C^2} = \sec^2$, and $S^2 + C^2 = 1$, we can get  $\begin{aligned}
  C & = \frac{ 1 - \omega^2 }{ \sqrt{ 1 - \omega^2 + \omega^4} } \\
  S & = \frac{ - \omega}{ \sqrt{ 1 - \omega^2 + \omega^4 }}
\end{aligned}$ \\
Note that the sign of $S$ is fixed by $\tan$.  

\[
\begin{gathered}
A (-\omega^2 C(\alpha) + - \omega S(\alpha) + C(\alpha) )S(\omega t) = S(\omega t) \Longrightarrow A = \frac{1}{ (1-\omega^2 )C(\alpha) - \omega S(\alpha) } = \frac{D}{ (1 - \omega^2)^2 - \omega(-\omega) } \\
\Longrightarrow \boxed{ A = \frac{1}{ \sqrt{ \omega^4 - \omega^2 + 1 } } }
\end{gathered}
\]
We could immediately see that $\omega = \frac{1}{ \sqrt{2}}$, $f= \frac{1}{ 2 \pi \sqrt{2}}$ will maximize $A$.  
\item We could have, from the beginning, considered the problem with any $R$, in general.
\[
\begin{gathered}
  A ((-\omega^2 C(\alpha) + - R \omega S(\alpha) + C(\alpha) )S(\omega t) + (-\omega^2 S(\alpha) + \omega R C(\alpha) + S(\alpha) ) C(\omega t) ) = S(\omega t ) \\
  \tan{\alpha} = \frac{ - \omega R }{ 1 - \omega^2 } \quad \, \begin{aligned}
    S(\alpha) & = \frac{ - \omega R }{ \sqrt{ (\omega R)^2 + ( 1 - \omega^2 )^2 } } \\
    C(\alpha) & = \frac{ 1 - \omega^2 }{ \sqrt{ ( \omega R)^2 + (1-\omega^2)^2 } }
\end{aligned} \\
  \Longrightarrow A = \frac{1}{ -\omega^2 C(\alpha)  - \omega R S(\alpha) + C(\alpha) } = \frac{1}{ \sqrt{ (\omega R)^2 + ( 1 - \omega^2)^2 } } \\
(\omega R)^2 + ( 1- \omega^2)^2 = \omega^4 + \omega^2 ( -2 + R^2 ) + 1 \xrightarrow{ \frac{d}{d\omega } } 4\omega^3 + 2 \omega ( -2 +R^2) = 2\omega ( 2\omega^2 + (-2 + R^2) ) = 0 \\
  \Longrightarrow \begin{aligned}
    \omega & = 0 \\
    2 \omega^2 & = 2 - R^2 
\end{aligned} \xrightarrow{ \text{ to have resonances } } \boxed{ R < \frac{1}{ \sqrt{2}} }
\end{gathered}
\]
\end{enumerate}

\exercisehead{10} A spaceship is returning to earth.  Assume that the only external force acting on it is the action of gravity, and that it falls along a straight line toward the center of the earth.  The rocket fuel is consumed at a constant rate of $k$ pounds per second and the exhaust material has a constant speed of $c$ feet per second relative to the rocket.  

Let $M(t) = M$ be the mass of the $rocket+fuel$ combination at time $t$.  With $+y$ direction being towards earth, then the equation of motion is $F_g = +M(t) g$, where $g = 9.8 m/s^2$.  \medskip \\
$M(t)v(t) = Mv_R$ is the momentum of the rocket.  \\
$M(t+h) = M(t) - \Delta m = M - \Delta m$ is the change in mass of the rocket due to spent fuel.  \\
$v_e = $ velocity of the exhaust in the lab frame $= c + v_R(t)$ 
\[
  \begin{aligned}
    & \Delta p = \Delta m ( c + v_R) + (M- \Delta m)v_R(t+h) - Mv_R = M(v_R (t+h) - v_R) + -\Delta m (v_R(t+h) - v_R) + \Delta m c \\
    &  \frac{\Delta p}{ \Delta t } = M \left( \frac{ v_R(t+h) - v_R }{ \Delta t } \right) + -\Delta m \left( \frac{ v_R(t+h) - v_R }{ \Delta t } \right) + \left( \frac{ \Delta m }{ \Delta t } \right) c = M(t) g 
  \end{aligned} 
\]
\[
\begin{gathered}
  Mv_R' + \frac{kc}{g} = M(t) g \\
  \text{ Now } M(t) = M_0 - \frac{kt}{g}  \Longrightarrow v_R' = g - \frac{kc/g}{M} = g - \frac{kc/g}{ M_0 - \frac{kc}{g} } \\
  v_R = gt - \frac{kc}{g} \frac{g}{-k} \ln{ (M_0 - \frac{kt}{g} )} = gt + c \ln{ (M_0 - \frac{kt}{g} ) } \\
  \boxed{ y_R = \frac{gt^2}{2} + c \frac{g}{k} \left( \left( \frac{k}{g} t - M_0 \right) \ln{ \left( M_0 - \frac{kt}{g} \right) } - \frac{kt}{g} \right)  + \frac{M_0 cg }{k} \ln{M_0} }
\end{gathered}
\]

\exercisehead{11} 
\[
\begin{gathered}
  Mv_R' = \frac{-kc}{g} \\
  \begin{aligned}
    v_R' & = \frac{-kc}{g} \left( \frac{1}{ M_0 - \frac{kt}{g} } \right) \\
    v_R & = c \ln{ (M_0 - \frac{kt}{g} ) }
  \end{aligned}
 \quad \, \Longrightarrow y_R = c \frac{g}{k} \left( \left( \frac{kt}{g} - M_0 \right) \ln{ \left( M_0 - \frac{kt}{g} \right) } - \frac{kt}{g} \right) + \frac{M_0 cg \ln{M_0 }}{k } \\
 M_0 g = w \Longrightarrow \boxed{ y_R = c \frac{g}{k}  \left( \frac{ kt - w }{g} \ln{ \left( \frac{ w - kt }{g} \right) } - \frac{kt}{g} \right) + \frac{ wc}{k} \ln{ \frac{w}{g} } }
\end{gathered}
\]

We could've also solved this problem with an initial velocity of $v_0$ and gravity.  Then
\[
\begin{gathered}
  v_R(t) = gt + c \ln{(1- \frac{kt}{M_0 g} ) } + v_0 \\
  y(t) = v_0 t + \frac{1}{2} g t^2 + c \left( \left( t - \frac{M_0 g}{ k } \right) \ln{ \left( 1 - \frac{kt}{M_0 g} \right) - t} \right)
\end{gathered}
\]

\exercisehead{12} 
\[
\begin{gathered}
  Mv_R = (M- \Delta m)(v_R(t+h) ) + 0 \\
  M(v_R(t+h) - v_R(t) ) = (\Delta m )v_R(t+h) \\
  Mv_R' = \frac{k}{g} v_R  \Longrightarrow \frac{v_R'}{v_R} = \frac{k}{ g (M_0 - \frac{kt}{g} ) } = \frac{k}{ M_0 g (1 - \frac{kt}{M_0 g} ) } \\
  \begin{aligned}
    & \ln{ v_R} = (k/w) \ln{ (1 - \frac{kt}{w} ) }\left( \frac{-w}{k} \right) = -\ln{ (1- \frac{kt}{w} ) } \\
    & v_R = \frac{v_0}{1 - \frac{kt}{w} } \quad \, \Longrightarrow x(t) = v_0 \left( \frac{-w}{k} \right) \ln{ (1 - \frac{kt}{w} ) } = \frac{-v_0 w }{k} \ln{ ( 1 - \frac{kt}{w} ) } 
  \end{aligned}
\end{gathered}
\]

%-----------------------------------%-----------------------------------%-----------------------------------
\subsection*{ 8.22 Exercises - Remarks concerning nonlinear differential equations, Integral curves and direction fields }
%-----------------------------------%-----------------------------------%-----------------------------------
\quad \\
\exercisehead{1} $2x + 3y = C$  \quad \, $\Longrightarrow y' = \frac{-2}{3}$ 

\exercisehead{2} $y = Ce^{-2x}$ \quad \, $\Longrightarrow y' = -2y$ 

\exercisehead{3} $ x^2 - y^2 = c$  \quad \, $\Longrightarrow yy' =x \Longrightarrow y' = \frac{x}{y}; \quad y\neq 0$

\exercisehead{4} $xy = c$ \quad \, $\Longrightarrow y' = \frac{-y}{x}$; \quad $x \neq 0$

\exercisehead{5} $y^2 = cx$ \quad  $\Longrightarrow \frac{y^2}{x} = c  \Longrightarrow y' = \frac{y}{2x} \quad x \neq 0 $

\exercisehead{6} $x^2 + y^2 + 2 Cy = 1$ \quad \quad \[ \begin{gathered}
  \frac{x^2}{y} + y - \frac{1}{y} = -2C \\
  \frac{ 2xy - y'x^2 }{ y^2 } + y' + \frac{1}{y^2 }y' = 0 \\
  y' = \frac{ -2xy}{ 1 + y^2 - x^2 } 
\end{gathered} \]

\exercisehead{7} $ y = C(x-1)e^x$ 
\[
\begin{gathered}
  \frac{y}{ (x-1)e^x } = C \\
  \frac{ y' (x-1) e^x - (e^x + (x-1) e^x )y }{ (x-1)^2 e^{2x} } = 0 \\
  \boxed{ y' = \frac{ xy}{x-1} }
\end{gathered}
\]

\exercisehead{8} $y^4(x+2) = C(x-2)$
\[
\begin{gathered}
  \frac{ y^4 (x+2) }{x-2} = C \\
  \frac{ (4y^3 y' (x+2) + y^4)(x-2) - y^4 (x+2) }{ (x-2)^2 } = 0 \quad \, \Longrightarrow 4 y^3 y'(x+2) + y^4 = \frac{ y^4 (x+2)}{x-2}  \\
  \boxed{ y' = \frac{ y}{ (x-2)(x+2) } }
\end{gathered}
\]

\exercisehead{9} $y = c \cos{x}$  \quad \quad $\Longrightarrow y' = -\tan{x} y$

\exercisehead{10} $\arctan{y} + \arcsin{x} = C $
\[
\frac{1}{1+y^2} y' + \frac{1}{ \sqrt{ 1 - x^2} } = 0 \quad \, \Longrightarrow \boxed{ y' = \frac{ -(1+y^2)}{ \sqrt{ 1 - x^2 } } }
\]

 
\exercisehead{11} All circles through the points $(1,0)$ and $(-1,0)$.  \medskip \\
Start with the circle equation: $(x-A)^2 + (y-B)^2 = R^2$   \medskip \\
$(1,0)$: $(1-A)^2 + B^2 = R^2 \Longrightarrow -(1- 2A + A^2 + B^2 = R^2 )$ \\
$(-1,0)$: $(-1-A)^2 + B^2 = 1 + 2A + A^2 + B^2 = R^2 $ \\
$\Longrightarrow 4A = 0, \, A= 0$ \quad \,  $1 + B^2 = R^2$    \\

\[
\begin{gathered}
  x^2 + (y - \pm \sqrt{ R^2 - 1} )^2 = R^2 \\
  x^2 + y^2  - 2B y + (R^2 - 1 ) = R^2 \Longrightarrow x^2 +y^2 - 2By = 1 
\end{gathered}
\]
$B$ depends upon $R$, the radius of the circles, so we could use $B$ as the parameter for the family of circles.  

\[
\begin{gathered}
x^2 +y^2 - 1 = 2By  \\
\frac{x^2}{y} + y - \frac{1}{y} = 2B \quad \, \Longrightarrow \frac{ 2xy - y' x^2 }{ y^2} + y' + \frac{1}{y^2} y' = 0 \\
\boxed{ y' = \frac{2xy}{ y^2 - x^2 +  1 } }
\end{gathered}
\]

\exercisehead{12} \[
\begin{gathered}
  (x+A)^2 + (y+B)^2 = r^2 \\
  \begin{aligned}
    (1+A)^2 + (1+B)^2 & =  1 + 2A + A^2 + 1 + 2B + B^2 = r^2 \\
-\left(     (-1+A)^2 + (-1+B)^2 \right) & =  -\left( 1 - 2A + A^2 + 1 - 2B + B^2 = r^2  \right) \\
\Longrightarrow & 4A + 4B = 0 \Longrightarrow A= - B
  \end{aligned} \\
\begin{gathered}
  (x+ -B)^2 + (y+B)^2 = r^2 \\
  2(x-B) + 2(y+B)y' = 0 \\
  (y+B)y' = B-x 
\end{gathered} \quad \, \Longrightarrow y' = \frac{B-x}{y+B}  \\
(1-B)^2 + (1+B)^2 =r^2 \Longrightarrow \sqrt{2} \sqrt{ (1+B^2) } = r \text{ or } \sqrt{ \frac{r^2}{2} - 1 } = B \\
\Longrightarrow \text{ (so $B$ could be treated as a parameter for the family of curves) } 
\end{gathered}
\]

%-----------------------------------%-----------------------------------%-----------------------------------
\subsection*{ 8.24 Exercises - First-order separate equations }
%-----------------------------------%-----------------------------------%-----------------------------------

\quad \\
 
\exercisehead{1} $y' = x^3/y^2 $ \medskip \\
$\frac{1}{3} y^3 = \frac{1}{4} x^4 + C  \quad \Longrightarrow y^3 = \frac{3}{4} x^4 + C$
 
\exercisehead{2} $\tan{x}\cos{y} = -y' \tan{y}$ \medskip \\
$\ln{ |\cos{x}|} = \frac{1}{ \cos{y}}$
 
\exercisehead{3} $(x+1)y' + y^2 = 0$ \medskip \\
$\frac{1}{y} = \ln{ (x+1)} + c$
 
\exercisehead{4} $y' = (y-1)(y-2)$ 
\[
\begin{gathered}
  \left( \frac{1}{ y-2} + \frac{-1}{y-1} \right)y' = 1 \quad \, \Longrightarrow \ln{ (y-2)} - \ln{ (y-1) } = x \\
  \frac{y-2}{y-1} = e^x
\end{gathered}
\]
 
\exercisehead{5} $y \sqrt{ 1 - x^2} y' = x $ \medskip \\
$\frac{1}{2} y^2 = - \sqrt{ 1 - x^2 }$ \quad \quad \, $\Longrightarrow y^2 = -2\sqrt{ 1 - x^2 }$
 
\exercisehead{6} $(x-1)y' = xy$ 
\[
\begin{gathered}
  \ln{y} = \int 1 + \frac{1}{x-1} = x + \ln{ |x-1| } \\
  \boxed{ y = e^x ( x-1) + C } 
\end{gathered}
\]

\exercisehead{7} $(1-x^2)^{1/2}y' + 1 + y^2 = 0$ \medskip \\
$\arctan{y} = \arccos{x} + C$
 
\exercisehead{8} $xy(1+x^2)y' - (1+y^2) = 0$ 
\[
\begin{gathered}
  \begin{aligned} 
    \frac{1}{2} \ln{ (1+y^2) } & = \int \left( \frac{1}{x} - \frac{x}{1+x^2} \right) + C  = \ln{x} - \frac{1}{2 \ln{ |1+x^2| }}
    \end{aligned} \\
  y^2 = k \left( \frac{x}{ \sqrt{ 1 + x^2} } \right)^2 
\end{gathered}
\]
 
\exercisehead{9} $(x^2 - 4)y' = y$ \medskip \\
\[
\begin{gathered}
  \ln{y} = \frac{-1}{2} arctanh{ \left( \frac{x}{2} \right) } \\
  \begin{aligned}
  \text{ since } \int \frac{1}{x^2 - 4 } dx & = \frac{1}{4} \int \frac{dx}{ \left( \frac{x}{2} \right)^2 - 1 } = \\
  & = \frac{1}{2} \int \frac{du}{u^2 - 1 } \text{ ( where $u = \frac{x}{2} $ ) } 
\end{aligned} \\
  \begin{aligned}
    (\tanh{(u)})' & = \frac{ \cosh^2{u} - \sinh^2{u} }{ \cosh^2{u}} = 1 - \tanh^2{u} 
  \end{aligned} \\
\Longrightarrow \boxed{ y = k \exp{ (-\frac{1}{2} arctanh{ \left( \frac{x}{2} \right) } ) } }
\end{gathered}
\]
 
\exercisehead{10} $xyy' = 1 + x^2 +y^2 + x^2 y^2$ \medskip \\
$ \frac{1}{2} \ln{ (1+y^2)} = \ln{x} + \frac{1}{2} x^2 + C $ \quad \, $\Longrightarrow \boxed{ y^2  = kx^2 e^{x^2} - 1 }$
 
\exercisehead{11} $yy' = e^{x+2y}\sin{x}$
\[
\begin{gathered}
  \frac{ ye^{-2y}}{ -2}  - \frac{e^{-2y}}{ 4 } = \frac{e^x \sin{x} - e^x \cos{x} }{ 2 }  + C \\
  (2y+1)e^{-2y} = -2e^x (\sin{x} - \cos{x}) + C 
\end{gathered}
\]
 
\exercisehead{12} $xdx + y dy = xy (xdy - ydx) $
\[
\begin{gathered}
  y(1-x^2)dy = x(-y^2 - 1 )dx \\
  \frac{y dy}{ 1+y^2 } = \frac{x dx}{ x^2 - 1 } \Longrightarrow \frac{1}{2} \ln{ |1+y^2| } = \frac{1}{2} \ln{ |x^2 - 1 | } + C \\
   1 +y^2 = (x^2 - 1) K \Longrightarrow \boxed{ y^2 = K(x^2 - 1) - 1 }
\end{gathered}
\]
 
\exercisehead{13} $f(x) = 2 + \int_1^x f(t) dt$ 
\[
\begin{gathered}
  f'(x) = f(x) \quad \, \Longrightarrow f(x) = C e^x \\
  \Longrightarrow Ce^x = 2 + Ce^x - Ce^1 \quad \, C = \frac{2}{e} \\
  \boxed{ f(x) = \frac{2}{e} e^x }
\end{gathered}
\]
 
\exercisehead{14} $f(x)f'(x) = 5x$ \quad \, $f(0)=1$
\[
\begin{gathered}
  f(x)^2 = 5 x^2 + C \quad \, \Longrightarrow f(x) = \pm \sqrt{ 5x^2 +  C } \\
  \boxed{ f(x) = \sqrt{ 5 x^2 + 1 } }
\end{gathered}
\]
 
\exercisehead{15} $f'(x) + 2xe^{f(x)} = 0$ \quad \, $f(0) = 0$
\[
\begin{gathered}
  e^{-y}y' = -2x \, \Longrightarrow -e^{-y} = -x^2 + C \\
  y = \ln{ (x^2 + C)^{-1} } \quad \, \Longrightarrow \boxed{ y = - \ln{ (x^2 + 1 ) } }
\end{gathered}
\]

\exercisehead{16} $f^2(x) + (f'(x))^2 = 1$
\[
\begin{gathered}
  \begin{aligned}
    f & = -1 \\
    y'^2 & = 1 - y^2 
  \end{aligned} \quad \quad \, y' = \pm \sqrt{ 1 - y^2 } \\
  \pm \arcsin{(y)} = x + c \Longrightarrow \boxed{ f(x) = \pm \sin{ (x+c) } }
\end{gathered}
\]
 
\exercisehead{17} 
\[
\begin{gathered}
  \begin{aligned}
    &    \int_a^x f(t) dt = K(x-a) \\
    & f> 0 \quad \forall \, x \in \mathcal{R}
  \end{aligned} \quad \, \Longrightarrow \boxed{ f(x) = k > 0 }
\end{gathered}
\]

\exercisehead{18} 
\[
\begin{gathered}
  \int_a^x f(t) dt = k(f(x)-f(a)) \xrightarrow{ \frac{d}{dx} } f(x) = kf'(x) \\
  \Longrightarrow f(x) = Ce^{\frac{1}{k}x}; \quad \, C > 0 
\end{gathered}
\]

\exercisehead{19}
$\int_a^x (f(t)) dt = k (f(x) + f(a)) \Longrightarrow f(x) = Ce^{\frac{x}{k}}$ \medskip \\
$k C e^{\frac{x}{k}} - k Ce^{\frac{a}{k}} = k Ce^{\frac{x}{k}} + k C e^{a/k} \Longrightarrow 2k Ce^{a/k} = 0 \Longrightarrow C = $ \\
$\boxed{ f = 0 }$

\exercisehead{20} $\int_a^x f(t) dt = k f(x)f(a)$; \quad \quad \, $f(x) = k f'(x)f(a)$
\[
\begin{gathered}
  \frac{1}{ k f(a)} = \frac{ f'(x) }{ f(x) } \Longrightarrow \ln{ f(x) } = \left( \frac{1}{kf(a) } \right) x + C ; \Longrightarrow f(x) = C \exp{ \left( \frac{x}{ k f(a) } \right) } \\
  \int_a^x f(t) dt = \left. \left( kf(a) C e^{ \frac{t}{ k f(a) } } \right) \right|_a^x = kf(a) C e^{ \frac{x}{kf(a) }} - k f(a) Ce^{ \frac{a}{ k f(a) } } = k C e^{ \frac{x}{kf(a)} } C e^{ \frac{a}{ kf(a) } } \\
  f(a) \left( e^{ \frac{x}{k f(a) }} - e^{ \frac{a}{ k f(a) } } \right) = Ce^{ \frac{x}{k f(a)} } e^{ \frac{a}{ k f(a) } } \\
  \xrightarrow{x=a} 0 = C e^{ \frac{2a}{ k f(a) } } \Longrightarrow C = 0 \\
\Longrightarrow  \boxed{ f = 0 }
\end{gathered}
\]
 
%-----------------------------------%-----------------------------------%-----------------------------------
\subsection*{ 8.26 Exercises - Homogeneous first-order equations }
%-----------------------------------%-----------------------------------%-----------------------------------
 \quad \\
\exercisehead{1} $f(tx,ty) = f(x,y)$ homogeneity (or homogeneity of zeroth order).  
\[
\begin{gathered}
  y' = f(x,y) = \left( \frac{x}{v} \right)' = \frac{v-xv'}{v^2} = f(x,\frac{x}{v} ) = f(1,\frac{1}{v}) \\
  v-v^2 f(1,\frac{1}{v}) = xv' \Longrightarrow \boxed{ \ln{x} = \int \frac{ dv}{ v - v^2 f(1,\frac{1}{v} ) } }
\end{gathered}
\]

\exercisehead{2} $y' = \frac{-x}{y}$  \quad \, $\Longrightarrow \frac{1}{2}y^2 = -\frac{1}{2} x^2 + C$ $ \Longrightarrow \boxed{ y^2 = -x^2 + C }$

\exercisehead{3} $y' = 1 + \frac{y}{x}$ \medskip \\
$\frac{y}{x} = v$ \quad $\Longrightarrow y' = v+ xv' = 1 + v \Longrightarrow v = \ln{x}$ \medskip \\
$\boxed{ y = x(\ln{x} +C ) }$

\exercisehead{4} $y' = \frac{x^2 + 2y^2 }{ xy } $ 
\[
\begin{gathered}
  y' = \frac{ x^2 + 2y^2 }{ xy }  = \frac{x}{y} + \frac{2y}{x} \\
  v = \frac{y}{x} \quad \, \Longrightarrow v+xv' = \frac{1}{v} + 2v \quad \, \Longrightarrow \frac{ v' }{ \frac{1}{v} + v } = \frac{1}{x} \\
  \frac{1}{2} \ln{ |1+v^2| } = \ln{x} + C \quad \, \Longrightarrow \boxed{ y^2 = (Cx^2 - 1 )x^2 }
\end{gathered}
\]

\exercisehead{5} $(2y^2 - x^2)y' + 3xy = 0$
\[
\begin{gathered}
  \text{ if $2y^2 \neq x^2$ }, y' = \frac{3xy}{ x^2 - 2y^2 } \\
  \begin{aligned}
    y & = vx \\ 
    y' & = v'x + v 
  \end{aligned} \quad \, \Longrightarrow y' = v'x + v  = \frac{ 3vx^2 }{ x^2 - 2v^2 x^2 } = \frac{3v}{ 1- 2v^2 } \\
  \Longrightarrow \frac{1-2v^2}{ 2v( 1+v^2) } v' = \frac{1}{x} \\
  \frac{1}{2} \left( \frac{1}{v} + \frac{-3v}{ 1 + v^2 } \right) v' = \frac{1}{x} \Longrightarrow \frac{1}{2} \ln{v} + \frac{-3}{2} \ln{ (1+v^2) } = \ln{x} + C \Longrightarrow \frac{v}{ (1+v^2)^3 } = Cx^2 = \frac{ y/x}{ \left( \frac{x^2 +y^2}{x^2} \right)^3 } \\
  \boxed{ yx^3 = C (x^2 + y^2)^3 }
\end{gathered}
\]

However,
\[
\begin{gathered}
  y' = \frac{3xy}{x^2 - 2y^2} \\
  \begin{aligned}
    v & = \frac{y}{x} \\
    y' & = v'x + v 
  \end{aligned} \quad \, \Longrightarrow v'x + v = \frac{3x^2 v}{ x^2 - 2v^2 x^2 } = \frac{3v}{ 1 - 2 v^2 }  \\
  v'x = \frac{3v}{ 1 - 2v^2 } - \frac{ (v-2v^3)}{ 1-2v^2 } = \frac{ 2(v +v^3) }{ 1-2v^2 } \Longrightarrow \left( \frac{1}{v} + - \frac{ 3v}{1+v^2 } \right)v' = \frac{2}{x} \Longrightarrow \ln{v} + -\frac{3}{2} \ln{ |1+v^2| } = 2 \ln{x} + C \\
  \frac{v}{ (1 + v^2)^{3/2} } = Cx^2 \quad \Longrightarrow \frac{ y^2/x^4}{ (x^2 + y^2)^3 } = Cx^4 \\
  \Longrightarrow \boxed{ y^2 = C(x^2 + y^2)^3 }
\end{gathered}
\]

\exercisehead{6} $xy' = y - \sqrt{ x^2 +y^2}$
\[
\begin{gathered}
  \Longrightarrow y' = \frac{y}{x} = \sqrt{ 1 + \left( \frac{y}{x} \right)^2 } \\
  \begin{aligned}
    &   \quad \\
     v & = \frac{y}{x} \\
    vx & = y 
  \end{aligned} \quad \, v'x + v = v - \sqrt{ 1 + v^2}  \Longrightarrow \begin{gathered} 
    \quad \\ 
    v'x = -\sqrt{ 1 +v^2 } \\
    \frac{ - v' }{ \sqrt{ 1 + v^2} } = \frac{1}{x} \\
    \Longrightarrow \ln{ (v+ \sqrt{ 1 + v^2 }) } = \ln{x} + C \text{ since } \\
    (\ln{ (v+\sqrt{ 1 + v^2} ) })' = \frac{1}{ v+ \sqrt{1+ v^2} } \left( 1 + \frac{v}{ \sqrt{ 1 + v^2} } \right) = \frac{1}{\sqrt{ 1 + v^2 }}
  \end{gathered} \\
  v + \sqrt{ 1 + v^2 } = Cx \quad \, \Longrightarrow 1 + v^2 = C^2 x^2 - 2vCx + v^2 \\
  \Longrightarrow v = \frac{ Cx}{2} - \frac{1}{ 2 C x } \quad \, \Longrightarrow \boxed{ y = \frac{Cx^2}{2} - \frac{1}{2C} }
\end{gathered}
\]

\exercisehead{7} $x^2 y' + xy + 2y^2 = 0 $
\[
\begin{gathered}
  x^2 y' = -2y^2 - xy \quad \Longrightarrow y' = \frac{-2y^2}{x^2} - \frac{y}{x} \quad \, v = \frac{y}{x} \Longrightarrow  \begin{aligned} 
    v' x + v & = -2v^2 - v \\
    v'x & = -2(v^2 + v) 
  \end{aligned} \\
  \frac{v'}{ v(v+1)} = \frac{-2}{x} = \int v' \left( \frac{1}{v} - \frac{1}{v+1} \right) = -2\ln{x} + C \Longrightarrow \frac{v}{v+1} = \frac{C}{x^2} \\
  \boxed{ y = \frac{-Cx}{C-x^2} }
\end{gathered}
\]

\exercisehead{8} $y^2 + (x^2 - xy + y^2)y' = 0$
\[
\begin{gathered}
  \left( \frac{y}{x} \right)^2 + \left( 1 - \frac{y}{x} + \left( \frac{y}{x} \right)^2 \right)y' = 0 \\
  \text{ Let } v= \frac{y}{x} \quad \, \Longrightarrow \frac{-v^2}{ 1 - v +v^2 } = v'x + v \\
  v' x = \frac{ -v^2}{ 1- v +v^2 } - v = \frac{ -v(1+v^2 )}{ 1 - v + v^2 }  \quad \Longrightarrow \frac{ v^2 - v + 1 }{ v(1+v^2) } v' = \frac{-1}{x} = \left( \frac{1}{v} + \frac{-1}{v^2 + 1 } \right) v' = - \frac{1}{x}  \\
  \Longrightarrow \ln{v} - \arctan{v} = -\ln{x} + C \Longrightarrow \ln{(vx)} = \arctan{x} + C  \quad \, \boxed{ \ln{y} = \arctan{ \frac{y}{x} }  + C } 
\end{gathered}
\]

\exercisehead{9} $y' = \frac{ y (x^2 + xy + y^2) }{ x(x^2 + 3xy + y^2 ) }$
\[
\begin{gathered}
  y' = \frac{ y (x^2 + xy + y^2) }{ x(x^2 + 3xy +y^2 ) } = \left( \frac{y}{x} \right) \left( \frac{ 1 + \frac{y}{x} + \frac{y^2}{x^2 } }{ 1 + \frac{3y}{x} + \frac{y^2}{x^2} } \right) \xrightarrow{ v = \frac{y}{x} } v'x + v = v \left( \frac{1 + v+ v^2 }{ 1 + 3v + v^2 } \right) = v + \frac{-2v^2}{ v^2 + 3v + 1 } \\
  v' ( 1 + \frac{3}{v} + \frac{1}{v^2} ) = \frac{-2}{x} \Longrightarrow v + 3 \ln{v} + \frac{-1}{v} = -2\ln{x} + C \\
  \boxed{ \frac{y}{x} + 3 \ln{y} - \frac{x}{y} = \ln{x} + C } 
\end{gathered}
\]

\exercisehead{10} $y' = \frac{y}{x} + \sin{ \frac{y}{x} }$
\[
\begin{gathered}
  \begin{aligned}
    \frac{y}{x} & = x \\
    y' & = v + v'x
  \end{aligned} \quad \, \Longrightarrow \begin{gathered}
    v + v'x = v + \sin{v}  \\
    \frac{v'}{\sin{v}} = \frac{1}{x} 
\end{gathered}  \\
  -\ln{ \csc{v} + \cot{v} } = \ln{x} + C \quad \, \Longrightarrow \boxed{ \csc{v}+\cot{v} = \frac{K}{x} }
\end{gathered}
\]

\exercisehead{11} $x ( y+ 4x)y' + y(x+4y) =0 $
\[
\begin{gathered}
  y' = \frac{ -y(x + 4y)}{ x ( y+4x) } = \frac{ - \frac{y}{x} ( 1 + \frac{4y}{x} ) }{ \frac{ y}{x} + 4 } \xrightarrow{ v = \frac{y}{x}} v+xv' = \frac{ -v (1+4v) }{ v + 4 } \\
  xv' = \frac{ -5 v(1+v)}{ v+ 4 } \Longrightarrow \frac{-5}{x} = \frac{ v+4}{v(1+v)}v' = \left( \frac{4}{v} + \frac{-3}{1+v} \right)v' \\
  \xrightarrow{\int} 4 \ln{v} - 3 \ln{(1+v)} = -5 \ln{x} + C \Longrightarrow \boxed{ (yx)^4 = (x+y)^3 C }
\end{gathered}
\]

%-----------------------------------%-----------------------------------%-----------------------------------
\subsection*{ 8.28 Miscellaneous review exercises - Some geometrical and physical problems leading to first-order equations }
%-----------------------------------%-----------------------------------%-----------------------------------
\quad \\
\exercisehead{1} 
\[
2x + 3 y = C \quad y'= -\frac{2}{3} \quad g' = \frac{3}{2} \Longrightarrow g - \frac{3}{2} x = C 
\]

\exercisehead{2} 
\[
xy = C \xrightarrow{ d/dx} y + xy' = 0 \Longrightarrow y' = -y/x \, x \neq 0 \Longrightarrow g' = x/g \Longrightarrow \frac{1}{2} g^2 = \frac{1}{2} x^2 + C 
\]

\exercisehead{3} $x^2 + y^2 + 2Cy = 1 $
\[
\begin{gathered}
  x + yy' + C y' = 0 \Longrightarrow y'(y+C) = -x \\
  y' = \frac{-x}{y+C} = \frac{-x}{ y + \frac{ 1 - x^2 - y^2 }{ 2y } } = \frac{-2xy}{ y^2 - x^2 + 1 } \\
  \xrightarrow{ orthogonal curves } y' = \frac{ y^2 - x^2 + 1 }{ 2xy} = \left( \frac{1}{2x} \right) y + \frac{1}{2}\left( \frac{1}{x} - x \right) y^{-1} 
\end{gathered}
\]

Recognize that this is a \emph{ Ricatti equation } and we know how to solve them.  

\[
\begin{gathered}
  y' + \frac{-1}{2x} y = y^{-1} \left( \frac{-x}{2} + \frac{1}{2x} \right) \quad \quad \, \begin{aligned} 
    \quad & \\
    n & = -1 \\
    k & = 1 - n = 1 - (-1) = 2  \end{aligned} \\
  v = y^k = y^2 \quad \quad \quad \,   v' + 2 \left( \frac{-1}{2x} \right) v = \frac{2}{2} \left( \frac{1}{x} - x \right)  \\
  A(x) = \int_a^x P(t) dt = \int_a^x \frac{-1}{t} = \ln{\frac{a}{x} } \quad \quad \,   \int_a^x Qe^A = \int_a^x \left( \frac{1}{t} - t \right) \frac{a}{t} = \frac{-a}{x} + 1 - a(x-a) \\
\boxed{  y^2 = v = -1 + \frac{x}{a} - x(x-a) + \frac{bx}{a} }
\end{gathered}
\]

\exercisehead{4} $y^2 = Cx$.  
\[
\begin{gathered}
  \frac{y^2}{x } = C \xrightarrow{ d/dx} \frac{ 2y y' x - y^2 }{ x^2 } = 0 \\
  y' = \frac{ y }{2x } \Longrightarrow y' = \frac{1}{ \left( \frac{-y}{2x } \right) } = \frac{-2x}{y} \\
  \Longrightarrow \boxed{ y^2 + 2x^2 =C }
\end{gathered}
\]

\exercisehead{5} $x^2 y = C$.  
\[
\begin{gathered}
  2xy + x^2 y' = 0 \quad y' = -\frac{2y}{x} \Longrightarrow y' = \frac{x}{2y} \\
  \frac{1}{2} y^2 = \frac{x^2}{4} + C \\
  \boxed{ 2y^2 -x^2 = C }
\end{gathered}
\]

\exercisehead{6} $y = C e^{-2x} $ 
\[
\begin{gathered}
  e^{2x} y = C \Longrightarrow 2e^{2x} y + e^{2x} y' = 0 \\
  y' = -2y \xrightarrow{ \text{ invert} } y' = \frac{1}{ 2 y } \\
  \Longrightarrow \boxed{ y^2 = x + C }
\end{gathered}
\]

\exercisehead{7} $x^2 - y^2 =C$ 
\[
\begin{gathered}
  2x - 2y y' = 0 \\
  y' = \frac{x}{y} \Longrightarrow y' = \frac{ -y}{ x} \\
  \Longrightarrow \ln{y}  = - \ln{x} + C \Longrightarrow \boxed{ y = \frac{C}{x} } 
\end{gathered}
\]

\exercisehead{8} $y\sec{x} = C$
\[
\begin{gathered}
  y'\sec{x} + y \tan{x} \sec{x} = 0 \\
  y' = - y\tan{x} \xrightarrow{ \text{ (invert) }} y' = \frac{1}{ y \tan{x}}  \\
  \boxed{ \frac{1}{2} y^2 = \ln{ |\sin{x}| } + C }
\end{gathered}
\]

\exercisehead{9} All circles through the points $(1,0)$ and $(-1,0)$
From Sec. 8.22, Ex.10, we had obtained $y' = \frac{2xy}{ y^2 - x^2 + 1 }$ \medskip \\
$\Longrightarrow y' = \frac{x^2 -y^2 - 1 }{ 2xy} = \frac{x^2-1}{2yx} - \frac{y}{2x}$ \quad $\Longrightarrow y' + \frac{1}{2x} y = \frac{x^2 - 1}{2x} y^{-1}$ \bigskip \\
Recognize this is a \emph{ Ricatti equation}.  \bigskip \\
For $y'+Py = Qy^n$, in this case, $n = -1$, and so $k=1-n = 1 - (-1) = 2$.  \\
Then $v= y^k$ and $v' + kPv = kQ$.  In this case, \\
$v' + 2 \left( \frac{1}{2x} \right) v = 2 \left( \frac{ x^2 - 1 }{ 2x} \right) = v' + \frac{1}{x} v = \frac{ x^2 - 1 }{x} = x - 1/x$.  
\[
\begin{gathered}
  A(x) = \int_a^x P(t)dt = \int_a^x \frac{1}{t} = \ln{ \frac{x}{a} } \\
  \int (t - \frac{1}{t}) \exp{ \left( \ln{ \frac{t}{a} } \right) } dt = \int \left( \frac{t^2 }{a} - \frac{1}{a} \right) dt = \left. \left( \frac{ \frac{1}{3} t^3 }{a} - \frac{t}{a} \right) \right|_a^x \\
  e^{-\ln{ \frac{x}{a} } } = \frac{a}{x} \\
  \Longrightarrow y^2 = v = \frac{ \frac{1}{3} x^3 - x - \frac{1}{3}a^3 + a }{x} + \frac{ba}{x} 
\end{gathered}
\]

\exercisehead{10} All circles through the points $(1,1)$ and $(-1,-1)$.  
\[
\begin{gathered}
  (x+A)^2 + (y+B)^2 = r^2 \\
  \begin{aligned}
    (1+A)^2 + (1+B)^2 & =  1 + 2A + A^2 + 1 + 2B + B^2 = r^2 \\
-\left(     (-1+A)^2 + (-1+B)^2 \right) & =  -\left( 1 - 2A + A^2 + 1 - 2B + B^2 = r^2  \right) \\
\Longrightarrow & 4A + 4B = 0 \Longrightarrow A= - B
  \end{aligned} \\
\begin{gathered}
  (x+ -B)^2 + (y+B)^2 = r^2 \\
  2(x-B) + 2(y+B)y' = 0 \\
  (y+B)y' = B-x 
\end{gathered} \quad \, \Longrightarrow y' = \frac{B-x}{y+B}  \\
(1-B)^2 + (1+B)^2 =r^2 \Longrightarrow \sqrt{2} \sqrt{ (1+B^2) } = r \text{ or } \sqrt{ \frac{r^2}{2} - 1 } = B \\
\Longrightarrow \text{ (so $B$ could be treated as a parameter for the family of curves) } \\
\begin{gathered}
  y' = \frac{-1}{ \left( \frac{B-x}{y+B} \right)} = \frac{ y+B}{ x-B } \\
  \frac{y'}{y+B} = \frac{1}{x-B} \Longrightarrow \boxed{ y = C(x-B) - B }
\end{gathered}
\end{gathered}
\]


\exercisehead{11} With $(0,Y) = Q$ the point that moves up wards along the positive $y$-axis and \medskip \\
$P= (x,y)$ being the point $P$ that pursues $Q$, \bigskip \\
$y' = \frac{ Y - y}{ X - x } =  \frac{ Y-y}{0-x}$ is the slope of the tangent line on a point on the trajectory of $P$.  

The condition given, that the distance of $P$ from the $y$-axis is $k$ the distance of $Q$ from the origin, is \medskip \\
\quad \quad $kY = x$.  

\[
\begin{gathered}
  y' = \frac{ \left( \frac{1}{k} \right) x - y }{ x }  = f(x,y) \\
\text{ $f(x,y) $  is homogeneous of zero order } \Longrightarrow y = vx \quad \text{ (try this substitution) } \\
y' = v' x + v = \frac{1}{k} - v \Longrightarrow \frac{ v'}{ \frac{1}{k} -2 v } = \frac{1}{x} \\
\Longrightarrow -\frac{1}{2} \ln{ \left( \frac{1}{k} - 2 v \right) } = \ln{x} + C \Longrightarrow y = \frac{x}{2k} - \frac{1}{2C^2 x } \\
\xrightarrow{ (1,0) } \boxed{ y = \frac{x}{2k} = \frac{1}{ 2kx} } \, \quad k = \frac{1}{2} \quad y = x - \frac{1}{x} 
\end{gathered}
\]

\exercisehead{12} 
\[
y = \frac{x}{2k} - \frac{1}{2kx}
\]

\exercisehead{13} $y=f(x)$.  
\[
\begin{gathered}
  n \int_0^x f(t) dt = x f(x) - \int_0^x f(t) dt ; (n +1) \int_0^x f(t) dt =  x y \\
  \Longrightarrow (n_1) y = y + x y' \Longrightarrow \begin{aligned} n y & = x y ' \\ \frac{n}{x} & = \frac{y'}{y} \\ n\ln{x} & = \ln{y} \end{aligned} \\
  \boxed{ y = C x^n } \text{ of } y = Cx^{1/n}
\end{gathered}
\]

\exercisehead{14} 
\[
\begin{gathered}
  n \int_0^x \pi f^2(t) dt = \int_0^x (\pi (y(x))^2 - \pi (f(t))^2 ) dt \\
  (n+1)\int_0^x \pi f^2(t) dt = xy^2(x) = xy^2; \quad (n+1) f^2(x) = y^2 + 2xy y' \\
  y' = \frac{ny}{2x} \quad \Longrightarrow \ln{ y} = \frac{n}{2} \ln{x} + C \\
  \boxed{ y = C x^{n/2} } \text{ or } y = C x^{1/2n} 
\end{gathered}
\]

\exercisehead{15} 
\[
\pi \int_0^x f^2(t) dt = x^2 f(x) \, \quad \pi f^2(x) = 2xf + x^2 f'  \Longrightarrow f' = \frac{ \pi f^2 - 2xf }{ x^2 } 
\]

The left hand side of the last expression shown is homogeneous.  Thus do the $y=vx$ substitution.  
\[
\begin{gathered}
  v'x + v = \frac{ \pi v^2 x^2 - 2x^2 v }{ x^2 } = \pi v^2 - 2v \\
  \frac{ v'x }{ \pi v^2 - 3v } =1 \Longrightarrow \frac{ v' }{ \pi (v^2 - \frac{3v}{\pi} ) }  = \frac{1}{x} = \frac{1}{3} \left( \frac{1}{ v - \frac{3}{\pi} } -  \frac{1}{v} \right) v' \\
  \begin{aligned}
    \ln{ (v- \frac{3}{\pi} )} - \ln{v} & =\ln{ \left( \frac{v - 3/\pi}{ v } \right) } & = 3 \ln{x} + C \\
    vx - \frac{3}{\pi} x & Cvx^4 
\end{aligned} \\
  y - \frac{3}{\pi}x = C y x^3 \Longrightarrow \boxed{ y = \frac{ 3x/\pi}{ 1 + \frac{x^3}{2} } }
\end{gathered}
\]

\exercisehead{16} $ A = \int_0^a f ; \quad B = \int_a^1 f $ \medskip \\
$A-B = \int_0^a f + \int_1^a f = 2 f(a) + 3 a +b $ \medskip \\
$\xrightarrow{ d/dx} 2f(a) = 2f'(a) + 3 \Longrightarrow 1 = \frac{ f'(a)}{ f(a) - \frac{3}{2} } $

So then
\[
\begin{gathered}
  a+C = \ln{ (y - \frac{3}{2} ) }; \quad f(a) = Ce^a + \frac{3}{2} \\
  f(1) = 0 = Ce^1 + \frac{3}{2}; \quad \Longrightarrow C = - \frac{3}{2e } \\
  \boxed{ f(x) = \frac{-3}{2} e^{x- 1 } + \frac{3}{2}  }
\end{gathered}
\]

To find $b$, 
\[
\begin{gathered}
  2 \left( \frac{-3}{2} e^{a-1} + \frac{3}{2} a \right) + \frac{3}{2} e^{-1} + \frac{3}{2} - \frac{3}{2} = 2 f(a) + 3a + b = \\
  = 2 \left( \frac{-3}{2} e^{a-1} + \frac{3}{2} \right) + 3a + b \\
  \Longrightarrow \boxed{ b = \frac{3}{2} e^{-1} -3 } 
\end{gathered}
\]

\exercisehead{17}
\[
\begin{gathered}
  \begin{aligned}
    A(x) & = \int_0^x \left( f(t) - \left( \left( \frac{y(x) - 1 }{ x } \right)t + 1 \right) \right) dt = x^3 = \\
    & = \int_0^x f + -\left( \frac{ y(x) - 1 }{ x } \right) \frac{1}{2} x^2 - x = \int_0^x f + -\left( \frac{ y(x) - 1 }{ 2 } \right) x -x 
\end{aligned} \\
  \xrightarrow{ d/dx} f(x) + -\frac{1}{2} (y'x + y ) - \frac{1}{2} = 3x^2 \\
  y' = \frac{ -2(3x^2 + \frac{1}{2} - y/2)}{ x } = -6x - x^{-1} + \frac{y}{x} 
\end{gathered}
\]

As a leap of faith, try $y=vx$ substitution to solve $y' = -6x -x^{-1} + \frac{y}{x} $.  
\[
\begin{gathered}
  v'x + v = -6x - x^{-1} + v \quad v' = -6 -x^{-2}; \quad v = -6x + x^{-1} + C  \\
  \boxed{ y = -6x^2 + 5x + 1 }
\end{gathered}
\]

\exercisehead{18} Assuming no friction at the orifice and energy conservation.  
\[
mgh = \frac{1}{2} m v_f^2 
\]
(imagine how the top layer of water is now at the bottom of the tank (final ``potential energy configurations''))

$V_f = \sqrt{ 2gh }$ (how fast water is rushing out) \medskip \\
$ A_0 = $ cross-sectional area of the orifice.  

\[
\begin{gathered}
  \frac{dV}{dt} = A \frac{dh}{dt} = -c\sqrt{ 2g h } A_0 \\
  \left. 2h^{1/2} \right|_{h_i}^{h_f} = -c\sqrt{ 2g } \frac{A_0}{A} t \\
  \Longrightarrow T = \frac{ \sqrt{2} A}{ c \sqrt{ g} A_0 } \left( \sqrt{ h_f} - \sqrt{ h_i} \right) = 59.6 sec
\end{gathered}
\]
Note that we included the discharge coefficient $C=0.6$.  

\exercisehead{19} $\frac{dV}{dt} = -c\sqrt{ 2gh} A_0 + \gamma_0 = A \frac{dh}{dt} = -\kappa h^{1/2} + \gamma_0$.   $\kappa = c \sqrt{2g} A_0$.  

\[
\begin{gathered}
\frac{ A dh}{ \gamma_0 - \kappa h^{1/2}} = dt = (A/\gamma_0) \frac{ dh}{ 1 - \frac{ \kappa}{\gamma_0} h^{1/2} }  \\
\begin{aligned}
  & \left( \ln{ \left( 1 - a h^{1/2} \right)} \right)' = \frac{1}{ 1 - a h^{1/2} } \left( - \frac{a}{2} \frac{1}{ h^{1/2} } \right) \quad \text{ (where $a = \frac{ \kappa}{\gamma_0} $ ) } \\
  & (h^{1/2})'  = \frac{1}{2h^{1/2}} \left( \frac{ 1 - a h^{1/2}  }{  1 - a h^{1/2} } \right) = \frac{ \left( \frac{1}{2} - \frac{a}{2} h^{1/2} \right)}{ h^{1/2} \left( 1 - a h^{1/2} \right) } 
\end{aligned} \\
\begin{aligned}
  \Longrightarrow \int (A/\gamma_0) \frac{ dh}{ 1 - \frac{\kappa}{ \gamma_0} h^{1/2} } & = - \left. (A/\gamma_0) \left( \frac{ 2 \gamma_0^2 }{ \kappa^2 } \ln{ \left( 1 - \frac{\kappa}{\gamma_0} h^{1/2} \right) } + \frac{ 2 \gamma_0 h^{1/2}}{ \kappa} \right) \right|_{h_i}^{h_f} = \\
    & = T 
\end{aligned} \\
\Longrightarrow \exp{ \left( \frac{-\gamma_0}{A} t - \frac{2\gamma_0}{\kappa} \left( h_f^{1/2} - h_i^{1/2} \right) \right)} \frac{\kappa^2}{ 2\gamma_0^2} = \frac{  1 - \frac{\kappa}{\gamma_0} h_f^{1/2} }{  1 - \frac{\kappa}{\gamma_0}  h_i^{1/2} } \\
  \xrightarrow{ t \to \infty } h_f = \frac{ \gamma_0^2}{ \kappa^2} = \frac{ (100 in^3/s)^2}{ c^2 (2) (32 ft/s^2) (5/3 in^2 )^2 (12in /1ft) }  = \left( 25/24 \right)^2
\end{gathered}
\]

\exercisehead{20}
\[
\begin{aligned}
  & V_0 = \frac{1}{3} \pi R_0^2 H_0 \\
  & V(h) = V_0 - \frac{1}{3} \pi h \left( h \frac{R_0}{ H_0} \right)^2 = V_0 - \frac{1}{3} \pi \frac{ R_0^2 }{ H_0^2 } h^3 = V_0 - \alpha h^3  \\
  & mg (H_0 -h) = \frac{1}{2} mv_f^2 ; \, \sqrt{ 2 g (H_0 - h) } = v_f \quad \text{ (energy conservation) } \\
  & cA_0 v_f = cA_0 \sqrt{ 2 g H_0 \left( 1 - \frac{h}{H_0} \right) } = \beta \sqrt{ 1 - \frac{h}{H_0} } \\
  \frac{dV}{dt } & = -\beta \sqrt{ 1 - h/H } = -3 \alpha h^2 \frac{dh}{dt} \, \Longrightarrow \frac{dh}{dt} = \frac{ \beta}{ 3 \alpha h^2 } \sqrt{ 1 - \frac{h}{H} } \\
\end{aligned}
\]
\[
\begin{aligned}
    \int \frac{ h^2 /H_0^2 }{ \sqrt{ 1 - \frac{h}{H_0} } } & = \frac{ \beta/H_0^2 }{ 3 \alpha} T = \\
    & = H_0 \int \frac{u^2 du }{ \sqrt{ 1 - u} } = H_0 \int \frac{ (1-y)^2 (-dy) }{ \sqrt{ y } } = - H_0 \int \frac{ 1 - 2 y +y^2 }{ \sqrt{ y }} = \\
    & = -H_0 \left. \left( 2 y^{1/2} - \frac{2}{3} y^{3/2} + \frac{2}{5} y^{5/2} \right) \right|_{h_i}^{h_f} = \\
    & = -H_0 \left. \left( 2 \left( 1 - \frac{h}{H_0} \right)^{1/2} - \frac{4}{3} \left( 1 - \frac{h}{H_0} \right)^{3/2} + \frac{2}{5} \left( 1 - \frac{h}{H_0} \right)^{5/2} \right) \right|_{h_i}^{h_f} = \\
    & = \frac{ cA_0 \sqrt{ 2 g H_0}/H_0^2 }{ 3 \left( \frac{1}{3} \pi \frac{R_0^2}{ H_0^2 } \right) } T  
\end{aligned}   
\]
For $h_i = 0 , \quad h_f = H$, 
\[
H_0(2(1) - 4/3(1) + 2/5) = H_0 (16/15) = cA_0 \sqrt{2gH_0} T / ( \pi R_0^2 ) ; T = \frac{ \frac{16}{15} \sqrt{H_0} \pi R_0^2 }{ cA_0 \sqrt{2g}} = \frac{2}{9} \frac{ \pi R_0^2 \sqrt{ H_0}}{ A_0 }
\]

\exercisehead{21} $m^2 x - m + (1-x) = 0 \quad \Longrightarrow (m^2 - 1 ) x + 1 - m = 0, \, \quad \Longrightarrow m = 1 $

\exercisehead{22} Given $x+y^3 + 6xy^2 y' = 0$ \medskip \\
Notice that $(2xy^3)' = 2(y^3 + 3xy^2 y') = 2y^3 + 6xy^2 y'$.  So then
\[
x +y^3 + 6xy^2 y' = (2xy^3)' + -y^3 + x = 0
\]
$\boxed{ \text{ Let } \, u = 2xy^3 }$.  So then $y^3 = \frac{u}{2x}$.  Thus, we have
\[
u' - \frac{ u}{2x} = -x
\]
This equation is linear because it is a linear combination of $y'$ and $y$.  Now we can use the formula for solving first-order linear differential equations of the form $u' + Pu =Q$.
\[
\begin{gathered}
  u = e^{-A} \left( \int_{x_0}^{x_f} Q e^A dx + u(x_0) \right) \quad \, \text{ for } A = \int P dt \\
  \Longrightarrow A = \int \frac{-1}{2t} dt = \frac{-1}{2} \ln{t} \text{ and } u = x^{1/2} \left( \frac{-2}{3} (x^{3/2} - x_0^{3/2} ) + u(x_0) \right) = \frac{-2}{3}x^2 + Cx^{1/2} \\
  \Longrightarrow \boxed{ y^3 = \frac{-x}{3} + \frac{C}{x^{1/2}} } \text{ for } x > 0 \text{ or } y^3 = \frac{-x}{3} \, \forall \, x
\end{gathered}
\]


\exercisehead{23} Given $(1+ y^2 e^{2x}) y' + y =0$, let $\boxed{ y = ue^{mx}}$, $u$ unknown.  
\[
\begin{gathered}
  y' = u' e^{mx} + mu e^{mx} \\
  (1+ u^2 e^{2mx} e^{2x})(u' e^{mx} + mu e^{mx} ) + ue^{mx} =0 = u' e^{mx} + u'(u^2 e^{3mx+2x}) + mu e^{mx} + mu^3 e^{3mx} e^{2x} + ue^{mx} 
\end{gathered}
\]
\[
\begin{aligned}
  & u' (e^{mx} + u^2 e^{(3m+2)x} ) + mue^{mx} + mu^3 e^{3mx +2x} + ue^{mx} = 0 \\
  & u' = \frac{ -(m+1)u - mu^3 e^{(2m+2) x} }{ 1 + u^2 e^{2(m+1)x } } 
\end{aligned}
\]
$ \boxed{ \text{Let} \, m=-1}$.  
\[
\begin{gathered}
  \Longrightarrow u' = \frac{ u^3}{ 1 + u^2} \Longrightarrow 
  \frac{ 1 + u^2 }{u^3 } du = \left( \frac{1}{u^3} + \frac{1}{u} \right)du = dx \Longrightarrow u^{-2}/-2 + \ln{u} - C = x  \\
  y = ue^{-x} \text{ or } u =ye^x \\
\Longrightarrow \boxed{ \ln{y} = x + \frac{e^{-2x}}{2y^2} +C }  
\end{gathered}
\]  

\exercisehead{24} Given $f$ s.t. $2f'(x) = f\left( \frac{1}{x} \right)$ if $x >0$, $f(1) = 2$ and $x^2 y'' + axy' + by = 0$ \\
\begin{enumerate}
\item 
\[
\begin{gathered}
2f''(x) = \frac{d}{dx} f\left( \frac{1}{x} \right) = (f')\left( \frac{-1}{x} \right) \\
\Longrightarrow x^2 y'' = \frac{-1}{2} f' = \frac{-1}{4} f\left( \frac{1}{x} \right) \\
\boxed{ a = 0; \quad \, b = \frac{1}{4} }
\end{gathered}
\]
\item  \[
\begin{gathered}
  \begin{aligned}
    f(x) & = Cx^n \\
    f' & = n C x^{n-1} \\
    f'' & = n(n-1) Cx^{n-2} 
  \end{aligned} \quad \quad \, \Longrightarrow n(n-1)Cx^n = \frac{-1}{4} Cx^n \\
  n^2 - n + \frac{1}{4} \Longrightarrow \boxed{ n = \frac{1}{2} } 
\end{gathered}
\] 
\end{enumerate}

\exercisehead{28} Choose the units for time to be in days first - we can convert into years later.  \medskip \\
If no one died from accidental death, then the population will grow by $e$.  That means, with $x=x(t)$ being the population at time $t$ 
\[
\frac{dx}{dt} = x \quad \, \Longrightarrow x = Ce^t
\]
which makes sense because if $C$ is the original population number, then after 1 year, $x=Ce$.   \\

With $t$ in days, we have a decrease of $\frac{1}{100} x$ in population each day due to death.  Add up the changes from the decrease due to deaths and the increase due to growth for the DE:
\[
\begin{gathered}
\frac{dx}{dt} = \frac{1}{365}x - \frac{1}{100} x = \frac{ 100 - 365}{36500} x = \frac{-265}{36500} \\
\Longrightarrow x = Ce^{ \frac{-265}{36500} t }
\end{gathered}
\]
Change $t$ units to years by multiplying the ``time constant'' $\frac{-265}{36500}$ by $365$ days.
\[
\boxed{ x = 365 \exp{ \left( -2.65 t \right) } }
\]
To get the total fatalities, simply integrate the deaths during each year.  
\[
\int_0^t y = \frac{365}{100} \frac{365}{2.65} \left( -\exp{ (-2.65 t) }  + 1 \right)
\]

\exercisehead{29} For constant gravity, $\Delta K = - \Delta U$ \medskip \\
$\Longrightarrow - (0 - mgh) = \frac{1}{2} mv_f^2$ \medskip \\
$v_f = \sqrt{ 2 gh} = (6.37 \times 10^8 \, cm) \left( \frac{1 \, m }{ 2.54 \, cm } \right) \left( \frac{1 \, ft }{ 12 \, in } \right) = 6.93 \, \frac{ mi }{sec} = 24940 \, \frac{ mi }{ hr} $

The constant energy formula could also be obtained by considering  \medskip \\
$F = \frac{G M_e m}{ -r^2 } = m \frac{d^r }{dt^2 } = - \partial_r U$

\exercisehead{30} Let $y= f(x)$ be the solution to $y' = \frac{ 2y^2 + x }{ 3y^2 + 5 }$  $f(0)=0$
\begin{enumerate}
\item $y'(0)=0$ as easily seen.  Now $y'' = \frac{ (4yy'+1)(3y^2 + 5) - (6yy')(2y^2 + x) }{ (3y^2 + 5)^2 }$, so then \\
$y''(0) = \frac{1}{5} > 0$.  It is a minimum.  
\item $f'(x) \geq 2/3 \quad \forall \, x \geq 10/3$.  $a=2/3$ since $f$ will be above this tangent line.  \\
Suppose, in the ``worst case,'' $f'(x) = 0$ for $0 \leq x \leq 2/3$.  Then $f(x) = 0$ for $0 \leq x \leq 2/3$.  Then the tangent line must be at $y=0$ at $x=10/3$ to remain below the graph of $f(x)$.  \medskip \\
$\Longrightarrow \frac{2}{3}x - 20/9 < f(x)$
\item Since $f'(x) \geq \frac{2}{3}$ for each $x \geq \frac{10}{3}$, then $f \to \infty$ for $x \to \infty$ (otherwise, $f$ would have to decrease somewhere, which would contradict the given fact about $f$).  Rewrite the DE for $y'$ to be 
\[
y' = \frac{2y^2 + x }{ 3y^2 + 5} \Longrightarrow (3y^2 + 5 )y' = 2y^2 + x \Longrightarrow (3 + \frac{5}{y^2}) y' = 2 + \frac{x}{y^2}
\]
Consider 
\[
y' = \frac{2y^2 + x }{ 3y^2 + 5 } = \frac{ 2 + \frac{x}{y^2} }{ 3 + \frac{5}{y^2} }
\]
specifically, $\frac{x}{y^2}$.  Now $y$ must, at the very least, have some linear increase because we had already shown that $y' \geq \frac{2}{3}$.  So $y^2$ would go to infinity faster than linear $x$.  Thus $\lim_{x\to \infty} y' = \frac{2}{3}$.  So then $(3 + \frac{5}{y^2})y' \xrightarrow{ x\to \infty} (3 + 0 )\frac{2}{3} = 2 = 2 + \frac{x}{y^2}$.  
\[
\boxed{ 0 = \frac{x}{y^2} }
\]
\item $y' = \frac{2 + \frac{x}{y^2}}{ 3 + \frac{5}{y^2} } \xrightarrow{x\to \infty} \frac{2}{3}$.  $\Longrightarrow y = \frac{2}{3} x$ or $\frac{y}{x} = \frac{2}{3}$.  
\end{enumerate}

\exercisehead{31} Given a function $f$ which satisfies the differential equation $xf''(x) + 3x (f'(x))^2 = 1 - e^{-x}$
\begin{enumerate}
\item $c\neq 0 $ for an extrenum.  \\
$cf''(c) + 3 c(f'(c))^2 = cf''(c) = 1 - e^{-c} \quad \, \Longrightarrow f''(c) = \frac{1-e^{-c}}{c} > 0 $
\item \emph{Cleverly}, consider the \emph{limit}.  
\[
  xf''(x) + 3x (f'(x))^2 = 1 - e^{-x} \Longrightarrow f''(x) + 3 (f'(x))^2 = \left( \frac{1-e^{-x}}{x} \right) \xrightarrow{ x \to 0 } f''(0) + 0 = 1 
\]
So a critical point at $x=0$ would be a minimum.  
\item We'll have to ``cheat'' a little and use the idea of power series early on here. \bigskip \\

$f'' + 3(f')^2 = \frac{1-e^{-x}}{x}$ suggests that we consider the Taylor series of $e^{-x}$.  
\[
\frac{1-e^{-x}}{x} = \frac{ - \sum_{j=1}^{\infty} \frac{(-x)^j}{j! } }{ x } = \sum_{j=0}^{\infty} \frac{ (-1)^j x^j }{ (j+1)! }
\]
This further suggests that $f$ itself has a power series representation because its first and second order derivatives are simply a combination of infinitely many terms containing powers of $x$.  \medskip \\
\phantom{ This furt} Then suppose $f = \sum_{j=0}^{\infty} a_j x^j$.  
\[
\begin{gathered}
f'' + 3 (f')^2  = \frac{1-e^{-x}}{x} \Longrightarrow \begin{aligned}
  f' & = \sum_{j=0}^{\infty} (j+1) a_{j+1} x^j \\
  f'' & = \sum_{j=0}^{\infty} (j+2)(j+1) a_{j+2} x^j  
\end{aligned} \\
\Longrightarrow \sum_{j=0}^{\infty} (j+2)(j+1) a_{j+2} x^j + 3 \sum_{j=0}^{\infty} \sum_{k=0}^{\infty} (j+1)(k+1) a_{j+1}a_{k+1} x^{j+k} = \sum_{j=0}^{\infty} \frac{ (-1)^j x^j }{ (j+1)! } \\
\begin{aligned}
  \text{ If } f(0) = 0 & \quad \, a_0 = 0 \\
  \text{ If } f'(0) = 0 & \quad \, a_1 = 0 
\end{aligned} 
\end{gathered}
\]
Then $f = a_2 x^2 + \sum_{j=3}^{\infty} a_j x^j$.  Consider the $x^0$ terms in the DE.  $(f')^2$ doesn't contribute, because $f'$'s leading order term is $x^1$.  So then
\[
2(1) a_2 + 0 = 1 \Longrightarrow \boxed{ a_2 = \frac{1}{2} } \quad \, \text{ i.e. } f = \frac{1}{2} x^2 + \sum_{j=3}^{\infty} a_j x^j 
\]
$\boxed{A = \frac{1}{2}} $ in order for $f(x) \leq Ax^2$
\end{enumerate}


%-----------------------------------%-----------------------------------%-----------------------------------
\subsection*{ 9.6 Exercises - Historical introduction, Definitions and field properties, The complex numbers as an extension of the real numbers, The imaginary unit $i$, Geometric interpretation.  Modulus and argument }
%-----------------------------------%-----------------------------------%-----------------------------------
\quad \\
\exercisehead{6} Let $f$ be a polynomial with real coefficients.  
\begin{enumerate}
\item Since $\overline{z_1 z_2}$ \medskip \\
  \quad \quad \quad \, $\overline{ (z_1^{n+1}) } = \overline{ z_1^n z_1 } = \overline{ z_1^n } \overline{ z_1} = \overline{z}_1^{n+1}$
\[
\overline{f(z)} = \overline{ \sum a_j z^j } = \sum a_j \overline{z}^j = f(\overline{z})
\]
\item If $f(z) = 0$, then $\overline{f(z)} = f(\overline{z}) = 0 $ as well.  
\end{enumerate}

\exercisehead{7} The three ordering axioms are 
\[
\begin{aligned}
  & \text{ Ax. 7 } \text{ If } x,y \in \mathbb{R}^+, \, x+y, xy \in \mathbb{R}^+ \\
  & \text{ Ax. 8 } \forall \, x \neq 0 , \, x \in \mathbb{R}^+ \text{ or } -x \in \mathbb{R}^+ \text{ but not both } \\
  & \text{ Ax. 9 } 0 \notin \mathbb{R}^+
\end{aligned}
\]
$x<y$ means $y-x$ positive.   \\

\noindent Suppose $i$ positive: $i(i)=-1$ but $-1$ is not positive.  \\
Suppose $-i$ is positive.  $-i(-i) = -1$ but $-1$ is not positive.  \\
$i$ is neither positive nor negative so Ax. 8 is not satisfied.  

\exercisehead{8} Ax. 8 , Ax. 9 are satisfied.  \medskip \\
$(a+ib)(c+id) = (ac-bd, ad+bc)$ so Ax. 7 might not be satisfied.  

\exercisehead{9} Ax. 7, Ax. 8 , Ax. 9 are trivially satisfied (all are positive).  

\exercisehead{10} $x>y$  $x>y$ is well defined.  Ax. 8 is satisfied.  \bigskip \\

For Ax. 7, $(\frac{3}{2},1), (1,\frac{1}{2})$ contradicts Ax. 7 since we required the product to be positive as well if the factors are positive.  We found this particular counterexample by considering factors $(a,b), (c,d)$, so the product of the two is $(ac-bd, ad+bc)$ and so we need $ac-bd - ad -bc  = a(c-d) -b(c+d) < 0$

\exercisehead{11} See sketch.  

\exercisehead{12} 
\[
\begin{gathered}
\begin{aligned}
  w & = \frac{az+b}{cz+d}  \\
  w & = \frac{ (az+b)(c\overline{z} + d) }{ (cz+ d)(c\overline{z} + d) } = \frac{ ac |z|^2 + adz + bc\overline{z} + bd }{ c^2 |z|^2 + cd(z+\overline{z}) + d^2 } \\
  w+ - \overline{w} & = \frac{ ac |z|^2 + adz + bc \overline{z} + bd - ac|z|^2 - ad\overline{z} -bcz - bd}{ |cz+d|^2} = \boxed{ \frac{ (ad-bc)(z-\overline{z})}{ |cz+d|^2 } }
\end{aligned} \\
\text{ If $ad-bc > 0$ } \\
w - \overline{w} = 2 Im{w} = \frac{ad- bc}{ |cz+d|^2 }2 Im{z}; \quad \frac{ad-bc}{ |az+d|^2} > 0 
\end{gathered}
\]
So $Im{w}$ has the same sign as $Im{z}$
%-----------------------------------%-----------------------------------%-----------------------------------
\subsection*{ 9.10 Exercises - Complex exponentials, Complex-valued functions, Examples of differentiation and integration formulas }
%-----------------------------------%-----------------------------------%-----------------------------------
\quad \\ 
\exercisehead{7}
\begin{enumerate}
\item 
\[
\begin{aligned}
  & \text{ if $m\neq n$ }, \int_0^{2\pi} e^{i x ( n - m ) } dx = \left. \frac{ e^{ix (n-m)}}{ i (n - m )} \right|_0^{2\pi } = \frac{ 1 -1}{ i (n-m)} = 0 \\
  & \text{ if $m=n$ }, \int_0^{2\pi} e^{i x (0) } dx = 2\pi
\end{aligned}
\]
\item \[
    \begin{aligned}
      \int_0^{2\pi} e^{inx} e^{-i mx } dx & = \int_0^{2\pi} (\cos{nx} + i \sin{nx})( \cos{mx} -i \sin{mx}) = \\
      & = \int_0^{2\pi} \cos{nx}\cos{mx} +\sin{nx}\sin{mx} + i (\sin{nx}\cos{mx} - \sin{mx} \cos{nx} )  \\
      \int_0^{2\pi} e^{-inx} e^{-i mx } dx & = \int_0^{2\pi} (\cos{nx} - i \sin{nx})( \cos{mx} -i \sin{mx}) = \\
      & = \int_0^{2\pi} \cos{nx}\cos{mx} -\sin{nx}\sin{mx} + i (-\sin{nx}\cos{mx} - \sin{mx} \cos{nx} )  
    \end{aligned} 
\]
\[
\begin{gathered}
    \text{ Summing the two equations above } \\ 
    \begin{gathered}
      0 = \int_0^{2\pi} 2\cos{nx}\cos{mx} + 2 \int_0^{2\pi} i -\sin{mx}\cos{nx} \\
      \Longrightarrow \int_0^{2\pi} \cos{nx} \cos{mx} = 0 , \quad \int_0^{2\pi} \sin{mx}\cos{nx} = 0 
    \end{gathered} 
\end{gathered}
\]
\[
    \begin{aligned}
      \int_0^{2\pi} e^{inx} e^{-i mx } dx & = \int_0^{2\pi} (\cos{nx} + i \sin{nx})( \cos{mx} -i \sin{mx}) = \\
      & = \int_0^{2\pi} \cos{nx}\cos{mx} +\sin{nx}\sin{mx} + i (\sin{nx}\cos{mx} - \sin{mx} \cos{nx} )  \\
      \int_0^{2\pi} e^{inx} e^{i mx } dx & = \int_0^{2\pi} (\cos{nx} + i \sin{nx})( \cos{mx} +i \sin{mx}) = \\
      & = \int_0^{2\pi} \cos{nx}\cos{mx} -\sin{nx}\sin{mx} + i (\sin{nx}\cos{mx} + \sin{mx} \cos{nx} )  
    \end{aligned} 
\]
\[
\begin{gathered}
    \text{ Subtract the two equations above } \\  
    \begin{gathered}
      \int_0^{2\pi} \sin{nx}\sin{mx} -i \int_0^{2\pi} \sin{mx} \cos{nx} = 0 \\
      \Longrightarrow \int_0^{2\pi} \sin{nx}\sin{mx} = 0 
    \end{gathered}
\end{gathered}
\]

\[
\begin{gathered}
  \begin{aligned}
    \int_0^{2\pi} e^{inx}e^{-inx} & = 2\pi = \int_0^{2\pi} (\cos{nx} + i\sin{nx})(\cos{nx} - i \sin{nx}) = \\
    & = \int_0^{2\pi} \cos^2{nx} + \sin^2{nx} 
\end{aligned} \\
  \begin{aligned}
    \int_0^{2\pi} e^{inx}e^{inx} & = 0 = \int_0^{2\pi} \cos^2{nx} - \sin^2{nx} + i (2\cos{nx}\sin{nx}) \\
    \Longrightarrow & \int_0^{2\pi} \cos^2{nx} - \sin^2{nx} = 0 \, \int_0^{2\pi} \cos{nx}\sin{nx} = 0  
  \end{aligned} 
\end{gathered}
\]
\[
\begin{gathered}
  \text{ Summing the two results above, we obtain } \quad \quad \quad  
  \begin{gathered}
    2\pi = \int_0^{2\pi} 2 \cos^2{nx}  \\
    \Longrightarrow \int_0^{2\pi} \cos^2{nx} = \pi 
  \end{gathered} \\
\text{ Then also, } \int_0^{2\pi} \sin^2{nx} = \pi 
\end{gathered}
\]
\end{enumerate}

\exercisehead{8} 
\[
\begin{gathered}
  z = re^{i\theta} = re^{i(\theta + 2\pi m ) }, \, m \in \mathbb{Z} \\
  z^{1/n} = r^{1/n} e^{i(\theta/n + 2 \pi m/n) } \, m = 0, 1, \dots n-1  \\
  \Longrightarrow z^{1/n} = Re^{i \alpha}\epsilon^m = z_1 \epsilon^m \\
  \text{ The roots are spaced equally by an angle } \, 2\pi /n 
\end{gathered}
\]
\[
\begin{gathered}
  i = e^{ i \pi/2 + i2\pi n } \Longrightarrow i^{1/3} = e^{i\pi/6}, \, e^{i 5\pi/6}, \, e^{i 3\pi/2 } \\ 
  i^{1/4} = e^{i \pi/8 }, \, e^{5 i \pi /8 }, \, e^{9i \pi/8}, \, e^{13i \pi /8 } \\
  -i = e^{-i\pi/2 + i 2\pi n } \Longrightarrow (-i)^{1/4} = e^{-i\pi/8}, \, e^{3i\pi/8}, \, e^{7i \pi /8}, \, e^{11i \pi/8} 
\end{gathered}
\]

\exercisehead{9}
\[
\begin{gathered}
\begin{aligned}
  e^{iu}e^{iv} & = e^{i(u+v)} = \cos{u+v} + i \sin{u+v} = \\
  & = (\cos{u}+i\sin{u})(\cos{v}+i\sin{v}) = \cos{u}\cos{v} - \sin{u}\sin{v} + i ( \cos{v}\sin{u} + \cos{u}\sin{v}) 
\end{aligned} \\
\Longrightarrow \sin{ u +v} = \cos{v}\sin{u} + \cos{u}\sin{v} \\
\Longrightarrow \cos{u+v} = \cos{u}\cos{v} - \sin{u}\sin{v} \\
\begin{aligned}
  \sin^2{z} + \cos^2{z} & = \left( \frac{ e^{iz} - e^{-iz} }{ 2i } \right)^2 + \left(  \frac{ e^{iz} + e^{-iz} }{ 2 } \right)^2 = \\
  & = \frac{ -(e^{2iz} + e^{-2iz} -2) + (e^{2iz} + 2 + e^{-2iz} ) }{ 4} = 1 
\end{aligned} \\
\begin{aligned}
  \cos{iy} & = \frac{ e^{i iy} + e^{-i iy }}{ 2 } = \\
  & = \frac{ e^{-y} + e^{y}}{ 2 } = \cosh{y} 
\end{aligned} \quad
\begin{aligned}
  \sin{iy} & = \frac{ e^{i iy } - e^{ -i iy }}{ 2 i } = \\
  & = \frac{ e^{-y} - e^{ y }}{ 2 i } = i \sinh{y}
\end{aligned} \\
\begin{aligned}
  e^{iz} & = e^{i (x+iy)} = e^{ix}e^{-y} = (\cos{x} + i \sin{x})e^{-y} \\
  e^{-iz} & = e^{-i (x+iy)} = e^{-ix}e^{y} = (\cos{x} - i \sin{x})e^{y}    
\end{aligned} \\
\text{ Thus it is clear, by mentally adding and subtracting the above results that  } \\
\Longrightarrow \begin{aligned}
  \cos{z} & = \cos{x} \cosh{y} - i \sin{x} \sinh{y} \\
  \sin{z} & = i \cos{x} \sinh{y}  + \sin{x} \cosh{y}
\end{aligned}
\end{gathered}
\]
 


%\begin{aligned}
%  \cos{z} & = \frac{ e^{i(x+iy)} + e^{-i(x+iy)}}{ 2} = \\ 
%  & = (1/2) (e^{ix}e^{-y} + e^{ix}e^{+y } - e^{ix}e^{y} + (e^{ix}e^{-y} - e^{ix}e^{-y}) + e^{-ix}e^{y} + e^{-ix}e^{-y} - e^{-ix}e^{-y} )  \\
%  & = (1/2)(e^{ix} 2\cosh{y} -e^{ix}2\sinh{y} +e^{-y}(-2i)\sin{x} + e^{-ix} 2 \cosh{y} = \\
%  & =  
%\end{aligned}

\exercisehead{10} 
\begin{enumerate}
\item $Log{(-1)} = i \pi$  \quad \quad \, $\log{(i)} = \ln{1} + i \frac{\pi}{2} = i \frac{\pi}{2} $ 
\item $ Log{(z_1 z_2 )} = Log{ (|z_1||z_2| e^{i (\theta_1 + \theta_2) } ) } = \ln{ |z_1||z_2| } + i ( \theta_1 + \theta_2 + 2 n \pi ) = Log{z_1} + Log{z_2} + i 2\pi n $ 
\item $Log{ (z_1/z_2) } = Log{ (|z_1|/|z_2| e^{i (\theta_1 - theta_2) } ) } = \ln{ \frac{|z_1|}{|z_2|} } + i ( \theta_1 - \theta_2 + 2 n \pi )  = Log{z_1} - Log{z_2} + i 2\pi n $
\item  $\exp{ (Log{z})} = \exp{ (\ln{|z|} + i \theta + i 2 \pi n ) } = z$
\end{enumerate}

\exercisehead{11} 
\begin{enumerate}
\item 
\[
\begin{aligned}
  & 1^i = e^{i Log{1}} = e^{ i ( i 2\pi n ) } = e^{-2\pi n } = 1 \text{ if $n=0$ } \\
  & i^i = e^{i Log{i}} = e^{ i (i \frac{\pi}{2} + i 2\pi n ) } = e^{ -\frac{\pi}{2} - 2\pi n } = e^{-\pi/2} \text{ if $n=0$ } \\
  & (-1)^i = e^{i Log{-1}} = e^{i (i\pi + i 2\pi n ) } = e^{-\pi - 2 \pi n } = e^{-\pi }
\end{aligned}
\]
\item $z^a z^b = e^{a Log{z}} e^{b Log{z}} = e^{ a Log{z} + b Log{z} } = e^{(a+b) Log{z} } = z^{a+b} $x
\item 
\[
\begin{gathered}
\begin{aligned}
  & (z_1 z_2)^w = e^{ w Log{z_1 z_2 } } = e^{w (Log{z_1} + Log{z_2} + 2 \pi m i ) } \\
  & (z_1^w z_2^w ) = e^{ w Log{z_1} }e^{ w Log{z_2} } = e^{ w (Log{z_1} + Log{z_2} ) }
\end{aligned} \\
m = 0 \text{ is the condition required for equality. }
\end{gathered}
\]
\end{enumerate}

\exercisehead{12} 
\[
\begin{aligned}
  & \begin{gathered} 
    \text{ if } L(u) = P, \, L(v) = Q, \\
    L(u+iv) = (u+iv)'' + a (u+iv)' + b(u+iv) = u''+au' +bu + i(v'' + av' + bv) = L(u) + iL(v) = P+iQ = R 
  \end{gathered} \\
  & \begin{gathered}
    \text{ if } L(f)  = R \\
    L(u+iv) = L(u) + iL(v) = P + iQ  \\
    \text{ then, equating real and imaginary parts, $L(u) = P, \, L(v) = Q$ }
  \end{gathered}
\end{aligned}
\]

\exercisehead{13} 
\[
L(y) = -\omega^2 y + ai \omega y + by = A e^{i\omega x} \Longrightarrow (- \omega^2 + ai \omega + b )B = A
\]

We cannot let $(-\omega^2 + a i \omega b ) = 0$ for a nontrivia solution.  Thus $b\neq \omega^2$ or $a\omega \neq 0$.  
\[
B = \frac{A}{ -\omega^2 + ai\omega +b}
\]

\exercisehead{14} \[
\begin{gathered}
  L(\widehat{y}) = ce^{i\omega x }; \, \widehat{y} = B e^{i\omega x} = \frac{c}{ -\omega^2 + ai\omega + b} e^{i\omega x } \\
  \Longrightarrow \widehat{y} = \frac{c}{ \sqrt{ (b-\omega^2)^2 + (a\omega)^2 }} e^{i(\omega x - \alpha)} \text{ where } \tan{ \alpha} = \frac{a\omega}{ b-\omega^2} \\
  \Longrightarrow \Re{\widehat{y}} = \frac{c}{ \sqrt{ (b-\omega^2)^2 +(a\omega)^2 }} \cos{(\omega x - \alpha)}
\end{gathered}
\]

\exercisehead{15} 
\[
\begin{gathered}
  \Im{ (\widehat{y}) } = \frac{ c}{ \sqrt{ (b-\omega^2)^2 + (a\omega)^2 } } \sin{ (\omega x + \alpha)} \\
  \Longrightarrow A = \frac{c}{ \sqrt{ (b-\omega^2)^2  + (a\omega)^2 }} ; \quad -\tan{\alpha} = \frac{a\omega}{ b-\omega^2}
\end{gathered}
\]

%-----------------------------------%-----------------------------------%-----------------------------------
\subsection*{ 10.4 Exercises - Zeno's paradox, Sequences, Monotonic sequences of real numbers }
%-----------------------------------%-----------------------------------%-----------------------------------

\exercisehead{1} Converges to $0$.  
\[
f(n) = \frac{n}{ n+1} - \frac{n+1}{n} = \frac{n^2 - (n^2 + 2n + 1)}{ n(n+1)} = \frac{ -2n -1}{ n^2 + n} = \frac{ \frac{-2}{n} - \frac{1}{n^2} }{ 1+ \frac{1}{n} } \xrightarrow{n\to \infty} 0 
\]

\exercisehead{2} Converges to $-1$.  \[
f(n) = \frac{ n^3 - (n^3+n+n^2 + 1 )}{ (n+1)n } = \frac{ n^2 - n-1}{ n(n+1)} = \frac{ -1 -\frac{1}{n} - \frac{1}{n^2}}{ 1+\frac{1}{n}} \xrightarrow{n\to \infty}  -1
\]

\exercisehead{3}
Diverges since
\[
|cos{ \frac{n \pi}{2} } - L | \geq \left| \left|  1 \cos{ \frac{ n\pi}{2} }  \right| - |L| \right| \geq |1 - |L||
\]
Choosing $\epsilon_1 = \frac{ |1-|L||}{ 2 } $, \, $|\cos{ \frac{n\pi}{2} - L} | > \epsilon_1$ for $n = 4m$.  

\exercisehead{4} $f(n)= \frac{1}{5} + \frac{3}{5n} - \frac{2}{5n^2} \to \lim_{n\to \infty} f(n) = \frac{1}{5}$

\exercisehead{5} $f(x) = \frac{x}{2^x} = \frac{x}{ \exp{ (x \ln{2}) } } \to 0 $ since $\lim_{x\to \infty} \frac{x^{\alpha}}{ (e^x)^{\beta} }$.  

\exercisehead{6} $f(n) = 1 + (-1)^n = 0 \text{ of } 1$.  \medskip \\
Thus, choosing $\epsilon_1 = \frac{ |1- |L||}{2}$; 
\[
|f(n) - L | \geq | |f(n)| - |L|| = |1- |L|| > \epsilon_1 \quad \text{ for any $n=2m$ }
\]

\exercisehead{7} $f(n) = \frac{1 + (-1)^n}{ n}$.  \medskip \\
Suppose $\epsilon = \frac{3}{N}$. \bigskip \\
So for $n>N, \, \frac{1}{N} > \frac{1}{n}$, $n \geq N = N(\epsilon) = 3/\epsilon$.  
 
\[
|f(n)| = \left| \frac{ 1 + (-1)^n}{ n} \right| \leq \frac{2}{n} < \frac{3}{n} < \frac{3}{N} = \epsilon
\]

\exercisehead{8} $f(n) = \frac{ (-1)^n}{n } + \frac{1+ (-1)^n }{ 2 }$ 
\[
\begin{aligned}
  |f(n) - L | & = \left| \frac{ (-1)^n}{ n} + \frac{ 1 + (-1)^n }{ 2} - L \right| \geq \left| \left| L - \frac{ (-1)^n }{ n} \right| - \left| \frac{ 1 + (-1)^n}{2} \right| \right| \geq \\
  & \geq \left| \left| |L| - \left| \frac{ (-1)^n}{ n } \right| \right| - \left| \frac{ 1 + (-1)^n}{ 2 } \right| \right| = \left| \left| |L| - \frac{1}{n} \right| = \left| \frac{ 1 + (-1)^n}{ 2 } \right| \right| \geq \\
  & \geq \left| \left| |L| - \frac{1}{n} \right| - \frac{1}{2} \right|
\end{aligned}
\]
Thus, consider 
\[
\left| \left| |L| - \frac{1}{n} \right| - \frac{1}{2} \right| > \left| \left| \frac{1}{N} - |L| \right| - 1 \right| = \epsilon_0 \, \text{ for } n > N
\]

\exercisehead{9} $f(x) = \exp{ \left( \frac{1}{x} \ln{2} \right) }$; \quad $\lim_{x\to \infty} f(x) = 0$.

\exercisehead{10} 
\[
|f(n) - L | = |n^{(-1)^n} - L | \geq | |n^{(-1)^n}| - |L| | = ||n| - |L|| = |n| - |L| > N - |L|
\]
Thus, for $n>N$, $N(\epsilon) = \epsilon + |L| $, so then $|f(n)- L| > \epsilon$.

\exercisehead{11} $f(n) = \frac{ n^{2/3} \sin{ n!} }{ n+1}$.  
\[
|f(n) | = \left| \frac{n^{2/3} \sin{(n!)}}{ n+1} \right| = \left| \frac{ \sin{ (n!)}}{ n^{1/3} + n^{-2/3} } \right| \leq \left| \frac{1}{n^{1/3}} \right|
\]
Thus, for $n> N$, \quad $N(\epsilon) = \frac{1}{ \epsilon^3}$, \quad $|f(n)| < \epsilon$.  

\exercisehead{12} Converges, since
\[ 
\begin{gathered}
\begin{aligned}
  f(n) - \frac{1}{3} & = \frac{ 3^{n+1} + 3(-2)^n - 3^{n+1} - (-2)^{n+1} }{ 3( e^{n+1} + (-2)^{n+1} ) } = \left( \frac{ (-2)^n (3+2)}{ 3(3^{n+1} + (-2)^{n+1} ) } \right) = \\
  & = \left( \frac{5}{3} \frac{ (-2)^n }{ 3^{n+1} + (-2)^{n+1} } \right) 
\end{aligned} \\
\begin{aligned}
  \left| f(n) - \frac{1}{3} \right| & = \frac{5}{3} \left| \frac{(-2)^n }{ 3^{n+1} +(-2)^{n+1}} \right| \leq \left| \frac{ - (-2)^{n+1}}{ 3^{n+1} + (-2)^{n+1} } \right| = \\
  & = \left| \frac{-1}{ \left( \frac{3}{-2} \right)^{n+1} + 1 } \right| = \left| \frac{1}{ 1 + \left( \frac{3}{-2} \right)^{n+1}} \right| < \\
    & < \frac{1}{ \left( \frac{3}{2} \right)^{n+1} } < \frac{1}{ \left( \frac{3}{2} \right)^n } 
\end{aligned}
\end{gathered}
\]
For $n > N$, consider $\epsilon = \left( \frac{2}{3} \right)^{-N}$, i.e. $N = \frac{ -\ln{\epsilon}}{ \ln{ 2/3} } = N(\epsilon)$.  Thus \medskip \\
$\boxed{ L= \frac{1}{3};}$

\exercisehead{13} 
\[
\begin{aligned}
  & f(n) = \sqrt{ n+1} - \sqrt{n} \\
  & f(n) = \left( \sqrt{n+1} - \sqrt{n} \right) \left( \frac{ \sqrt{ n+1} + \sqrt{n} }{ \sqrt{ n+1} + \sqrt{n} } \right) = \frac{ n+1 - n}{ \sqrt{ n+1} + \sqrt{n} } = \sqrt{ 1}{ \sqrt{n+1} + \sqrt{n}} \\ 
  & |f(n)| = \left| \frac{1}{ \sqrt{n+1} + \sqrt{n} } \right| \leq \frac{1}{ 2 \sqrt{n}}; 
\end{aligned}
\]
So then $\forall \epsilon$, we have $\epsilon = \frac{1}{ 2 \sqrt{N}}$ and for $n>N$, $\frac{1}{2\sqrt{n}} < \frac{1}{2\sqrt{N}} = \epsilon$.

Thus $f(n)$ converges to $0$.  

\exercisehead{14} 
\[
f(n) = na^n = n\exp{ n \ln{a}} = \frac{n}{ \exp{ n\ln{\frac{1}{a} } } } \to 0 \, \text{ since } \lim_{x\to \infty} \frac{x^a}{ (e^x)^b } = 0
\]

\exercisehead{15} $f(n) = \frac{ \log_a{n}}{ n}, a >1$.  $\lim_{n\to \infty} f(n) = 0 $ \text{ since } $\lim_{x\to \infty} \frac{ (\log{x})^a}{x^b} = 0 $ for $a> 0, b > 0$ 

\exercisehead{16} $\lim_{n\to \infty} f(n) = 0$

\exercisehead{17} $\lim_{n\to \infty} f(n) =e^2$.  

\exercisehead{18} 
\[
\begin{aligned}
  \left| 1 + \frac{n}{n+1} \cos{ \frac{n\pi}{2} } - L \right| & \geq \left| \left| 1 + \frac{n}{n+1} \cos{ \frac{n\pi}{2} } \right| - |L| \right| = \left| \left| 1 + \frac{-n}{n+1} \right| - |L| \right| = \\
  & = \left| \frac{1}{n+1} - |L| \right| > |L|
\end{aligned}
\]
Choose $\epsilon_0 = \frac{ |L|}{2}$.  For any $N$, for $n >N$, $|f(n)-L| > \epsilon_0$.  

\exercisehead{19} 
\[
\begin{aligned}
  1 + \frac{i}{2} & = \frac{ \sqrt{5}}{2} e^{i\alpha} \\
  1 + \left( \frac{1}{2} \right)^2 & = \frac{5}{4} \\
  \tan{\alpha} & = \frac{1}{2} 
\end{aligned}
\quad 
\begin{aligned}
  \left( \frac{\sqrt{5}}{2} e^{i\alpha} \right)^{-n}, & \quad \left( \frac{2}{\sqrt{5}} \right)^n e^{-ni\alpha} \\
  \lim_{n \to \infty} \left( \frac{2}{\sqrt{5} } \right)^n e^{-ni \alpha} & = 0
\end{aligned}
\]

\exercisehead{20} $\lim_{n \to \infty} f(n) = \lim_{n \to \infty} e^{-\pi i n/2}$ diverges since 
\[
|e^{i\pi i n/2} - L | \geq | |e^{-i\pi i n/2} | - |L| | = |1 - |L| | > |L|
\]
So for $\epsilon_0 = \frac{|L|}{2}$, for $n > N$, $|f(n) - |L|| > \epsilon_0$

\exercisehead{21}
$f(n) = \frac{1}{n} e^{-i\pi n /2}$ \quad $|f(n)| = \frac{1}{n}$.  \medskip \\
Suppose $N(\epsilon) = \frac{1}{\epsilon}$; then for $n>N$, $|f(n)| < \epsilon_0$

\exercisehead{22} $|f(n) - L| \geq | | ne^{-\pi i n/2} | - |L| | = |n-|L||$ \medskip \\
Consider $\epsilon_0 = |1 - |L|| $ for $n>N >1$, $|f(n) - L| > \epsilon_0$.

\exercisehead{23} $a_n = \frac{1}{n}$.  
\[
\begin{gathered}
  |a_n | = \left| \frac{1}{n} \right| < \frac{1}{N} ; \quad N(\epsilon) = \frac{1}{ \epsilon} \\
  \begin{aligned} 
    \epsilon & = 1, 0.1, 0.01, 0.001, 0.0001 \\
    N & = 1, 10, 100, 1000, 10000
  \end{aligned}
\end{gathered}
\]

\exercisehead{24} $|a_n - 1 | = \left| \frac{n+1 - n}{ n+ 1 } \right| = \frac{1}{n+1} < \frac{1}{n} $ \medskip \\
$N(\epsilon) = \frac{1}{\epsilon} = 1,10, 100, 1000, 10000$.

\exercisehead{25} $ |a_n|= \frac{1}{n}$ \medskip \\
$N(\epsilon) = 1,10,100, 1000,10000$.  

\exercisehead{26} $|a_n| = \left| \frac{1}{n!} \right| \leq \frac{1}{ \exp{ n\ln{n}}} < \frac{1}{ \exp{n}} $ \medskip \\ 
For $n>N$, $\frac{1}{\exp{N}} = \epsilon$, so that $N = \ln{ 1/\epsilon}$.  
\[
N(\epsilon) = 1, 2, 4, 6, 9
\]

\exercisehead{27} $a_n = \frac{2n}{n^2+1}$; $|a_n|  = \left| \frac{2}{ n^2 + 1/n} \right| \leq \left| \frac{2}{n^2} \right|$.  
\[
N(\epsilon) = \frac{ \sqrt{2}}{ \sqrt{\epsilon}} = 1,4,14, 44, 141
\]

\exercisehead{28} $|a_n| = \left| \frac{9}{10} \right|^n = \left( \frac{9}{10} \right)^n = e^{n\ln{9/10}}$
\[
\begin{aligned}
  N(\epsilon) & = \frac{ \ln{\epsilon}}{ \ln{ \left( \frac{9}{10} \right)} } = \frac{ -\ln{ 1/\epsilon}}{ \ln{ (9/10) }} = \\
  & = 1,21,43, 65, 87
\end{aligned}
\]

\exercisehead{30} 
If $\forall \epsilon >0$, \, $\exists N \in \mathbb{Z}^+$ such that $n>N$, $|a_n| <\epsilon$.  
\[
\begin{aligned}
  |a_n|^2 & < |a_n| \epsilon < \epsilon^2 \\
  |a_n^2| & <\epsilon^2
\end{aligned}
\]
So for $\forall \epsilon_1 > 0$, $\epsilon_1 = \epsilon^2$ and $\exists N = N(\epsilon) = N(\epsilon_1)$, so that $|a_n^2| < \epsilon_1$.  

\exercisehead{31} 
\[
|a_n + b_n - (A+B)| = ||a_n -A| |b_n -B| | \leq |a_n - A| + |b_n -B | < \epsilon + \epsilon = 2 \epsilon
\]

\[
\begin{gathered}
  \forall \epsilon > 0, \, \exists N_A, N_B \in \mathbb{Z}^+, \, |a_n - A |< \epsilon \quad \text{ if } n > N_A; |b_n -B | <\epsilon \text{ if } n > N_B \\
  \text{ Consider } \max{ (N_A, N_B) } = N_{A+B} \\
  |a_n + b_n - (A+B) | < 2 \epsilon \\
\begin{aligned}  
  & \forall \epsilon_1 > 0, \, \epsilon_1 = 2\epsilon, \, \text{ then } \exists N_{A+B} = N_{A+B}(\epsilon_1) \in \mathbb{Z}^+ \text{ such that } \\
  & \left| (a_n +b_n) - (A+B) \right| < \epsilon_1 \text{ if } n > N_{A+B}
\end{aligned} 
\end{gathered}
\]

\[
\begin{gathered}
  |ca_n - cA| = c|a_n - A| < c\epsilon \\
  \forall \epsilon_1 > 0, \, \epsilon_1 = c\epsilon; \quad \text{ then } \exists N_{cA} = N(\epsilon) = N(\epsilon_1) \in \mathbb{Z}^+ \text{ such that } \\
  |ca_n -cA| < \epsilon_1 \text{ for } n > N(\epsilon_1)
\end{gathered}
\]

\exercisehead{32} Given $\lim_{n\to \infty} a_n = A$, 
\[
\begin{gathered}
  \lim_{n\to \infty} (a_n -A)(a_n+A) = \lim_{n\to \infty} (a_n -A) \lim_{n\to \infty} (a_n+ A) = 0 (2A) = 0 \\
  2a_n b_n = (a_n +b_n)^2 - a_n^2 - b_n^2 \\
  2\lim_{n\to \infty} a_n b_n = \lim_{n\to \infty} (a_n +b_n)^2 - \lim_{n \to \infty}a_n^2 - \lim_{n \to \infty}b_n^2 = \left( \lim_{n\to \infty} (a_n+b_n)\right)^2 - A^2 - B^2 = 2AB \\
  \Longrightarrow \lim_{n\to \infty} a_n b_n = AB
\end{gathered}
\]

\exercisehead{33} $\binom{ \alpha}{n} = \frac{ \alpha (\alpha-1)(\alpha-2) \dots (\alpha - n +1) }{ n!}$
\begin{enumerate}
\item 
\[
\begin{aligned}
\binom{ -\frac{1}{2}}{ 1! } & = \frac{-1/2}{1} \\  
\binom{ -\frac{1}{2}}{ 2! } & = \frac{ \left( -\frac{1}{2} \right) \left( \frac{-3}{2} \right)}{ 2!} = \frac{5}{8 } \\
\binom{ -\frac{1}{2}}{ 3! } & = \frac{ \left( -\frac{1}{2} \right)\left( \frac{-3}{2} \right) \left( \frac{-5}{2} \right)}{ 3! } = -\frac{5}{16}
\end{aligned}
\quad 
\begin{aligned}
  \binom{ -\frac{1}{2}}{4} & = \frac{ \left( -\frac{1}{2} \right) \left( \frac{-3}{2} \right) \left( \frac{-5}{2} \right) \left( \frac{-7}{2} \right)}{ 4!} = \frac{35}{128} \\
  \binom{ -\frac{1}{2} }{ 5} & = \frac{ \left( -\frac{1}{2} \right)\left( \frac{-3}{2} \right)\left( \frac{-5}{2} \right)\left( \frac{-7}{2} \right) \left( \frac{ -9}{2} \right)}{5!} = \frac{-63}{256}
\end{aligned}
\]
\item $a_n = (-1)^n \binom{ \frac{-1}{2}}{ n} $.  \medskip \\
  $a_1 = \frac{1}{2} > 0$.  $\quad a_2 = \frac{3}{8} > 0$  
\[
\begin{aligned}
  a_{n+1} & = (-1)^{n+1} \binom{ -\frac{1}{2}}{ n+1} = (-1)^{n+1} \binom{ \alpha}{n} \left( \frac{ -\frac{1}{2} - (n+1)+1}{ n+1} \right) = \frac{a_n (-1) \left( \frac{-1}{2} - n \right) }{ n+1} = \\
  & = \frac{ a_n (n+1/2) }{ n+1} > 0 
\end{aligned}
\]
\[
a_{n+1} = \left( \frac{ n+1/2}{n+1} \right)a_n < a_n
\]
\end{enumerate}

\exercisehead{34}
\begin{enumerate}
\item \[
\begin{aligned}
  t_n - s_n & = \frac{1}{n} \sum_{k=1}^n f\left( \frac{k}{n} \right) -  \frac{1}{n} \sum_{k=0}^{n-1} f\left( \frac{k}{n} \right) = \frac{1}{n} \left( f(1) +  \sum_{k=1}^{n-1} f\left( \frac{k}{n} \right) - \left( f(0)  +  \frac{1}{n} \sum_{k=1}^{n-1} f\left( \frac{k}{n} \right)  \right) \right)  = \\
  & = \frac{1}{n} (f(1)-f(0))
\end{aligned}
\]

Since $f\left( \frac{k}{n} \right) \leq f(t) \leq f\left( \frac{k+1}{n} \right)$ for $\frac{k}{n} \leq t \leq \frac{k+1}{n}$, by $f$ being monotonically increasing.  
\[
\begin{gathered}
  \Longrightarrow s_n \leq \int_0^1 f(x)dx \leq t_n \quad \text{ (from definition of integral) } \\
  0 \leq \int_0^1 f(x) dx -s_n \leq t_n - s_n = \frac{1}{n} (f(1) - f(0))
\end{gathered}
\]
\item Use Theorem 1.9.  
\begin{theorem}
Every function $f$ which is bounded on $[a,b]$ has a lower integral $\underline{I}(f)$ and an upper integral $\overline{I}(f)$ satisfying
\[
\int_a^b s(x) dx \leq \underline{I}(f) \leq \overline{I}(f) \leq \int_a^b t(x) dx 
\]
for all step functions $s$ and $t$ with $s\leq f \leq t$.  The function $f$ is integrable on $[a,b]$ iff its upper and lower integrals are equal,
\[
\int_a^b f(x) dx = \underline{I}(f) = \overline{I}(f)
\]
\end{theorem}
Since $f(x)$ is integrable, then $\lim_{n \to \infty} { s_n } = \lim_{n\to \infty} { t_n } = \int_0^1 f(x) dx $
\item $\left( \frac{ b-a}{n} \right) = \Delta$, $s_n = \frac{1}{ \Delta} \sum_{k=0}^{n-1} f(a+k\Delta)$, $t_n = \frac{1}{\Delta} \sum_{k=1}^n f(a+k\Delta)$

So by increasing monotonicity of $f$, $s_n \leq \int_a^b f(x) dx \leq t_n$.  
\[
\begin{gathered}
  t_n - s_n = \frac{1}{\Delta} \left( f(b) + \sum_{k=1}^{n-1} f(a+k\Delta) = \sum_{k=1}^{n-1} f(a+k\Delta) -f(a) \right) = \frac{ f(b)-f(a)}{\Delta} \\
  0 \leq \int_a^b f(x) dx \leq \frac{f(b) - f(a)}{\Delta}
\end{gathered}
\]
\end{enumerate}

\exercisehead{35}
\begin{enumerate}
\item $\lim_{n\to \infty} \frac{1}{n} \sum_{k=1}^n \left( \frac{k}{n} \right)^2 = \int_0^1 t^2 dt = \frac{1}{3}$ 
\item $\lim_{n\to \infty} \sum_{k=1}^n \frac{1}{n+k} = \lim_{n\to \infty} \frac{1}{n} \sum_{k=1}^n \frac{1}{ 1+ \frac{k}{n} }= \int_0^1 \frac{1}{x} dx = \ln{2}$ 
\item $\lim_{n \to \infty} \frac{1}{n} \sum_{k=1}^n \frac{1}{ 1 + \left( \frac{k}{n} \right)^2 } = \int_0^1 \frac{1}{1+x^2} dx = \left. \arctan{x} \right|_0^1 = \boxed{ \frac{\pi}{4} }$ 
\item $\lim_{n \to \infty} \sum_{k=1}^n \frac{1}{ \sqrt{ n^2 +k^2 }} = \lim_{n\to \infty} \frac{1}{n} \sum_{k=1}^n \frac{1}{ \sqrt{ 1 + \left( \frac{k}{n} \right)^2 }}  = \int_0^1 \frac{1}{ \sqrt{ 1 + x^2 }} dx = \left. \ln{(x+ \sqrt{ 1 +x^2 } ) } \right|_0^1 = \ln{ (1+ \sqrt{2}) }$
\item $\lim_{n\to \infty} \sum_{k=1}^n \frac{1}{n} \sin{ \frac{k\pi}{n} } = \lim_{n\to \infty} \frac{1}{n} \sum_{k=1}^n \sin{ \frac{k\pi}{n} } = \int_0^1 \sin{ \pi x } dx = \left. \left( \frac{ -  \cos{\pi x }}{ \pi} \right) \right|_0^1 = \frac{ -(-1-1)}{ \pi} = \boxed{ \frac{2}{\pi} }$
\item $\lim_{n\to \infty} \sum_{k=1}^n \frac{1}{n} \sin^2{ \frac{k\pi}{n}} = \int_0^n \sin^2{x\pi} = \boxed{ \frac{1}{2} }$
\end{enumerate}

%-----------------------------------%-----------------------------------%-----------------------------------
\subsection*{ 10.9 Exercises - Infinite series, The linearity property of convergent series, Telescoping series, The geometric series }
%-----------------------------------%-----------------------------------%-----------------------------------
\quad 

\exercisehead{1} $\sum_{n=1}^{\infty} \frac{1}{ (2n-1)(2n+1)} = \sum_{n=1}^{\infty} \frac{ 1/2 }{ 2n-1 } - \frac{ 1/2}{ 2n+1} = \frac{1}{2}$

\exercisehead{2} $\sum_{n=1}^{\infty} \frac{2}{ 3^{n-1}} = 2 \sum_{n=0}^{\infty} \frac{1}{3^n} = 2 \frac{1}{ 1 - 1/3} = \boxed{ 3 } $

\exercisehead{3} $\sum_{n=2}^{\infty} \frac{1}{n^2- 1 }  = \sum_{n=2}^{\infty} \frac{ 1/2}{ n-1} - \frac{ 1/2}{ n+1} = \sum_{n=2}^{\infty} \left( \frac{ 1/2}{ n-1} - \frac{ 1/2}{ n } \right) + - \left( \frac{1/2}{ n+1} - \frac{ 1/2}{n} \right) = \frac{1}{2} + \frac{1}{4} = \frac{3}{4}$

\exercisehead{4} $\sum_{n=1}^{\infty} \frac{ 2^n + 3^n}{ 6^n}  = \sum_{n=1}^{\infty} \left( \frac{1}{3} \right)^n  + \sum_{n=1}^{\infty} \left( \frac{1}{2} \right)^n = \frac{ 1/3}{ 1 - 1/3} + \frac{ 1/2}{ 1-1/2} = \frac{1}{2} + 1 = \frac{3}{2}$

\exercisehead{5} $ \sum_{n=1}^{\infty} \frac{ \sqrt{n+1} - \sqrt{n}}{ \sqrt{ n^2 + n }} = \sum_{n=1}^{\infty} \frac{1}{ \sqrt{n}} - \frac{1}{ \sqrt{n+1}} = 1 $ 

\exercisehead{6} 
\[
\begin{aligned}
  \sum_{n=1}^{\infty} \frac{n}{ (n+1)(n+2)(n+3)} & = \sum_{n=1}^{\infty} \frac{3/2}{ (n+2)(n+3)} + \frac{ -1/2}{ (n+1)(n+2)} = \sum_{n=1}^{\infty} \frac{1}{ (n+2)(n+3) }  + \frac{1}{2} \left( \frac{-1}{6} \right) = \\
  & = \sum_{n=1}^{\infty} \frac{1}{1+2} - \frac{1}{ n+3} -\frac{1}{12} = \frac{1}{3} - \frac{1}{12} = \boxed{ \frac{1}{4} }
\end{aligned}
\]

\exercisehead{7} $\sum_{n=1}^{\infty} \frac{2n+1}{ n^2(n+1)^2} = \sum_{n=1}^{\infty} \frac{1}{n^2} - \frac{1}{ (n+1)^2} = 1$

\exercisehead{8} $\sum_{n=1}^{\infty} \frac{ 2^n + n^2+n }{ 2^{n+1} n (n+1)} = 1 = \sum_{n=1}^{\infty} \frac{1}{ 2n (n+1)} + \frac{1}{ 2^{n+1}} = \frac{1}{2} \sum_{n=1}^{\infty} \frac{1}{n} - \frac{1}{n+1} + \frac{ 1/4}{ 1-1/2} = \frac{1}{2} + \frac{1}{2} =1$

\exercisehead{9} $\sum_{n=1}^{\infty} \frac{(-1)^{n-1}(2n+1)}{ n(n+1)} = \sum_{n=1}^{\infty} (-1)^{n-1} \left( \frac{1}{n} + \frac{1}{n+1} \right)$.  
\[
\begin{gathered}
  (-1)^{n-1} \left( \frac{1}{n} + \frac{1}{n+1} \right) + (-1)^n \left( \frac{1}{n+1} + \frac{1}{ n+2} \right) = (-1)^n \left( \frac{-1}{n} + \frac{1}{n+2} \right) \\
  \sum_{j=1}^{\infty} \left( \frac{-1}{2j -1} + \frac{1}{2j + 1 } \right) = \frac{1}{2} \sum_{j=1}^{\infty} \left( \frac{ -1}{ (j-1/2)} + \frac{1}{ (j+1/2)} \right) = \frac{1}{2} \frac{-1}{ 1/2} = -1
\end{gathered}
\]

\exercisehead{10} 
\[
\begin{aligned}
  \sum_{n=2}^{\infty} \frac{ \log{ ((1+\frac{1}{n} )^n (1+n) ) } }{ (\log{n^n} )( \log{(n+1)^{n+1} } ) } & = \sum_{n=2}^{\infty} \frac{ \log{ \left( \left( \frac{n+1}{n} \right)^n (1+n) \right) }}{ \log{ (n+1)^{n+1} } \log{n^n } } = \\
  & = \sum_{n=2}^{\infty} \frac{ log{ (n+1)^{n+1} } - \log{n^n} }{ \log{ (n+1)^{n+1} } \log{n^n} } = \sum_{n=2}^{\infty} \frac{1}{ \log{n^n}} - \frac{1}{ \log{ (n+1)^{n+1} } } = \\
  & = \frac{1}{ 2 \log{2}} = \log_2{\sqrt{e}}
\end{aligned}
\]
since if $\frac{1}{ 2 \log{2}} = y$, then $y = \log_2{\sqrt{2}}$.  

\exercisehead{11} $\sum_{n=1}^{\infty} nx^n = x \frac{d}{dx} \sum_{n=1}^{\infty} x^n = x \left( \frac{x}{1-x} \right)' = \frac{x}{ (1-x)^2}$.  

\exercisehead{12} 
\[
\begin{aligned}
  \sum_{n=1}^{\infty} n^2 x^n & = x \frac{d}{dx} \sum_{n=1}^{\infty} nx^n = x \left( \frac{x}{(1-x)^2} \right)' = \\
  & = x \frac{ (1-x)^2 + 2(1-x)x}{ (1-x)^4 } = \frac{ x ( 1 -x^2)}{ (1-x)^4} = \frac{ x(1+x)}{(1-x)^3} 
\end{aligned}
\]

\exercisehead{13}
\[
\begin{aligned}
  \sum_{n=1}^{\infty} n^3 x^n & = x \frac{d}{dx} \sum_{n=1}^{\infty} n^2 x^n = x \left( \frac{x+x^2}{(1-x)^3} \right)' = \\
  & = x \frac{ (1+2x)(1-x)^3 + 3(1-x)^2 (x)(x+1)}{ (1-x)^6} = x \frac{ (1+2x)(1-x)+3x(x+1)}{ (1-x)^4} = \frac{ x (x^2 + 4x + 1 )}{ (1-x)^4} 
\end{aligned}
\]

\exercisehead{14} $ \sum_{n=1}^{\infty} n^4 x^4 = x \frac{d}{dx} \sum_{n=1}^{\infty} n^3 x^3 = x \left( \frac{ x^3 + 4x^2 + x }{ (1-x)^4 } \right)' $.
\[
\begin{gathered}
  \ln{ \left( \frac{ x^3 + 4 x^2 + x }{ (1-x)^4} \right) } = \ln{ (x^3 + 4x^2 + x )} - 4 \ln{ (1-x)} \\
  (\ln{f})' = \frac{1}{f}f' = \frac{ 3x^2 + 8x + 1 }{ x^3 + 4x^2 + x } + 4 \frac{1}{1-x}; \\
  \begin{aligned}
    f' & = \frac{ (3x^2 + 8x + 1)(1-x)}{ (1-x)^5} + \frac{ 4 ( x^3 + 4 x^2 + x )}{ (1-x)^5} = \\
    & = \frac{ 3x^2 + 8x+1 -3x^3 - 8x^2 - x + 4x^3 + 16x^2 + 4x }{ (1-x)^5} = \frac{ x^3 + 11x^2 + 11x + 1 }{ (1-x)^5} 
\end{aligned} \\
  \Longrightarrow \frac{x^4 + 11x^3 - 11x^2 + x }{ (1-x)^5}
\end{gathered}
\]

\exercisehead{15} $\sum_{n=1}^{\infty} \frac{x^n}{ n} = x \sum_{n=1}^{\infty} \int_0^x t^{n-1} dt = \int_0^x dt \sum_{n=1}^{\infty} t^{n-1} = \int_0^x \frac{1}{1-t} = -\ln{(1-x)}$.  

\exercisehead{16} 
\[
\begin{aligned}
  \sum_{n=1}^{\infty} \frac{x^{2n-1}}{2n-1} & = \sum_{j=1}^{\infty} \int_0^x t^{2j-2} dt = \int_0^x dt \sum_{j=1}^{\infty} (t^2)^{j-1} = \\
    & = \int_0^x \frac{dt}{ 1-t^2} = \int_0^x dt \left( \frac{1/2}{ 1-t} + \frac{1/2}{ 1+t} \right) = \frac{1}{2} \ln{ \left( \frac{1+x}{1-x} \right) }
\end{aligned}
\]

\exercisehead{17} $\sum_{n=0}^{\infty} (n+1) x^n = \sum_{n=0}^{\infty} \frac{d}{dx} x^{n+1} = \frac{d}{dx} \left( \frac{x}{1-x} \right) = \frac{1}{ (1-x)^2 }$  

\exercisehead{18} 
\[
\begin{aligned}
  \sum_{n=0}^{\infty} \frac{(n+1)(n+2)}{ 2!} x^n & = \sum_{n=0}^{\infty} \left( \frac{x^{n+2}}{2} \right)'' = \frac{d^2}{dx^2} \frac{x^2}{2} \left( \frac{1}{1-x} \right) = \frac{d}{dx} \left( \frac{x}{ 1-x} + \frac{x^2}{ 2(1-x)^2} \right) \\
  & = \frac{1}{1-x} + \frac{x}{(1-x)^2} + \frac{x}{(1-x)^2} + \frac{x^3}{ (1-x)^3} = \frac{ 1-2x + x^2 + 2x -2 x^2 + x^2 }{ (1-x)^3} = \frac{1}{(1-x)^3}
\end{aligned}
\]

\exercisehead{19} 
\[
\begin{aligned}
  \sum_{n=0}^{\infty} \frac{(n+1)(n+2)(n+3)}{ 3!} x^n & = \sum_{n=0}^{\infty} \frac{d^3}{dx^3} \frac{x^{n+3}}{3!} = \frac{1}{3} \frac{d}{dx} \sum_{n=0}^{\infty} \frac{d^2}{dx^2} \frac{x^{(n+1) +2} }{ 2} = \\
  & = \frac{1}{3} \frac{d}{dx} \sum_{n=1}^{\infty} \frac{d^2}{dx^2} \frac{x^{n+2}}{2} = \frac{1}{3} \frac{d}{dx} \left( \sum_{n=0}^{\infty} \frac{d^2}{dx^2} \frac{ x^{n+2}}{ 2} - \frac{d^2}{dx^2} \left( \frac{x^2}{2} \right) \right) = \\
  & = \frac{1}{3} \frac{d}{dx} \left( \frac{1}{ (1-x)^3} -1 \right)= \frac{1}{3} \frac{ -3(-1)}{ (1-x)^4} = \frac{1}{(1-x)^4}
\end{aligned}
\]

\exercisehead{20} $\sum_{n=1}^{\infty} n^k x^n = \frac{P_k(x) }{ (1-x)^{k+1}}$  
\[
\begin{aligned}
  \sum_{n=1}^{\infty} n^{k+1} x^n & = x \frac{d}{dx} \sum_{n=1}^{\infty} n^k x^n = x \frac{d}{dx} \left( \frac{ P_n(x)}{ (1-x)^{k+1}} \right) = x \left( \frac{ P_k'(x) (1-x)^{k+1} + (k+1)(1-x)^k P_k(x) }{ (1-x)^{2k+2} } \right) = \\
  & = x \left( \frac{ P_k'(x)(1-x) + (k+1)P_k(x) }{ (1-x)^{k+2} }\right) = \frac{ (k+1) x P_k(x) + x (1-x) P_k'(x) }{ (1-x)^{k+2} } 
\end{aligned}
\]

$((k+1)P_k(x) + (1-x)P_k'(x))x $ has $x$ as its lowest degree term from $xP_x'(x)$ and \\
$(k+1)x^{k+1} + -kx^{k+1} = x^{k+1}$ highest degree term is obtained from $(k+1)P_k(x) + -xP_k'(x)$.  

\exercisehead{21} $\sum_{n=0}^{\infty} \binom{n+k}{k} x^n = \frac{1}{(1-x)^{k+1}} = \frac{d^k}{dx^k} \sum_{n=0}^{\infty} \frac{x^{n+k}}{k!}$.  

\[
\begin{aligned}
  \sum_{n=0}^{\infty} \binom{ n+k+1}{ k+1} x^n & = \sum_{n=0}^{\infty} \frac{d^{k+1}}{ dx^{k+1}} \frac{x^{n+k+1}}{ (k+1)! } = \frac{1}{ (k+1)} \frac{d}{dx} \sum_{n=0}^{\infty} \frac{d^k}{dx^k} \frac{x^{(n+1) +k}}{k!} = \\
  & = \frac{1}{k+1} \frac{d}{dx} \sum_{k=1}^{\infty} \frac{d^k}{dx^k} \frac{x^{n+k}}{k!} = \left( \frac{1}{k+1} \right) \frac{d}{dx} \sum_{n=0}^{\infty} \frac{d^k}{dx^k } \frac{x^{n+k}}{ k!} - \frac{d^k}{dx^k} \frac{x^k}{k!} = \\
  & = \left( \frac{1}{k+1} \right) \frac{d}{dx} \left( \frac{1}{ (1-x)^{k+1}} - 1 \right) = \frac{1}{ (1-x)^{k+2} }
 \end{aligned}
\]

\exercisehead{22} 
\begin{enumerate}
\item $\sum_{n=2}^{\infty} \frac{n-1}{n!} = \sum_{n=2}^{\infty} \frac{1}{ (n-1)!} - \sum_{n=2}^{\infty} \frac{1}{n!} = \sum_{n=1}^{\infty} \frac{1}{n!} - \sum_{n=2}^{\infty} \frac{1}{n!} = \boxed{1}$.  
\item $\sum_{n=2}^{\infty} \frac{n}{n!} + \sum_{n=2}^{\infty} \frac{1}{n!} = \sum_{n=2}^{\infty} \frac{1}{(n-1)!} + \sum_{n=0}^{\infty} \frac{1}{n!} - 1 -1 = \sum_{n=1}^{\infty} \frac{1}{n!} + \sum_{n=0}^{\infty} \frac{1}{n!} -2 = \sum_{n=0}^{\infty} \frac{2}{n!} - 3 = \boxed{ 2e -3}$.  
\item \[
\begin{aligned}
  \sum_{n=2}^{\infty} \frac{ (n-1)(n+1)}{n!} & = \sum_{n=2}^{\infty} \frac{n^2}{n!} + \sum_{n=2}^{\infty} \frac{-1}{n!} = \sum_{n=2}^{\infty} \frac{n}{ (n-1)!}  + \sum_{n=2}^{\infty} \frac{-1}{n!} = \\
  & = \sum_{n=1}^{\infty} \frac{n+1}{n!} + \sum_{n=2}^{\infty} -\frac{1}{n!} = \sum_{n=1}^{\infty} \frac{1}{ (n-1)!} + 1 = \sum_{n=0}^{\infty} \frac{1}{n!} + 1 = \boxed{ e+1 }
\end{aligned}
\]
\end{enumerate}

\exercisehead{23} \begin{enumerate}
\item
$x\frac{d}{dx} \left( x \frac{d}{dx} \sum_{n=1}^{\infty} \frac{x^n}{n! } \right) = x \frac{d}{dx} \left( x \sum_{n=1}^{\infty} \frac{nx^{n-1}}{ n! } \right) = x \frac{d}{dx} \sum_{n=1}^{\infty} \frac{nx^n}{n!} = \sum_{n=1}^{\infty} \frac{ n^2 x^n }{n!} = x \frac{d}{dx} \left( x \frac{d}{dx} e^x \right) = x^2 e^x + xe^x$
\item $x \frac{d}{dx} \left( \sum_{n=1}^{\infty} \frac{n^2 x^n}{ n!} \right) = \sum_{n=1}^{\infty} \frac{n^3 x^n}{n!} = x \frac{d}{dx} \left( (x^2 + x)e^x \right) = x \left( (2x+1)e^x + (x^2 + x)e^x \right) = (x^3 + 3x^2 + x)e^x$ \medskip \\
  x = 1 \quad \boxed{ k=5}
\end{enumerate}

\exercisehead{24} 

\begin{enumerate}
  \item $\sum_{n=2}^{\infty} (-1)^n = \sum_{n=2}^{\infty} (-1)^n (n-(n-1))$.  Identical.  
  \item $\sum_{n=2}^{\infty} (1-1) = \sum_{n=2}^{\infty} (-1)^n$.  Not identical.  
  \item Not identical.  $\sum_{n=2}^{\infty} (-1)^n$ vs. $ \left( \sum_{n=2}^{\infty} (-1+1) \right) + 1$.  
  \item Identical.  $\sum_{n=0}^{\infty} \left( \frac{1}{2} \right)^n = 1 + \sum_{n=1}^{\infty} \left( \left( \frac{1}{2} \right)^{n-1} - \left( \frac{1}{2} \right)^n \right) = \sum_{n=1}^{\infty} \left( \frac{1}{2} \right)^n (2-1) = \sum_{n=0}^{\infty} \left( \frac{1}{2} \right)^n $  
\end{enumerate}


\exercisehead{25}  
\begin{enumerate}
  \item \[
\begin{gathered}
  1 + x^2 + x^4 + \dots + x^{2n} + \dots = \frac{1}{1-x^2} \quad \text{ if } |x| <1 \\
  \Longrightarrow 1 + 0 + x^2 + 0 + x^4 + \dots = \frac{1}{ 1-x^2 } \quad \text{ if } |x| <1 
\end{gathered}
\]
  \item Thm. 10.2.  $\sum_{n=1}^{\infty} (\alpha a_n + \beta b_n) = \alpha \sum_{n=1}^{\infty} a_n + \beta \sum_{n=1}^{\infty} b_n$.  So then
\[
\sum_{j=0}^{\infty} x^j  - \sum_{j=0}^{\infty} \frac{ x^j + (-x)^j}{ 2 } = \sum_{j=0}^{\infty} \frac{x^j - (-x)^j}{ 2 } = \sum_{j=0}^{\infty} x^{2j+1} = \frac{1}{1-x} - \frac{1}{1-x^2} = \frac{x}{1-x^2}
\]
  \item 
\[
\sum_{j=0}^{\infty} (x^2)^j + \sum_{j=0}^{\infty} x^j = \sum_{j=0}^{\infty} (x^j - x^{2j} ) = \frac{x}{1-x^2}
\]
\end{enumerate}

%-----------------------------------%-----------------------------------%-----------------------------------
\subsection*{ 10.14 Exercises - Tests for convergence, Comparison tests for series of nonnegative terms, The integral test }
%-----------------------------------%-----------------------------------%-----------------------------------

We'll be using the integral test.  

\begin{theorem}[Integral Test] \quad \\
Let $f$ be a positive decreasing function, defined for all real $x \geq 1$.  \\
For $\forall n \geq 1$, let $s_n = \sum_{k=1}^n f(k)$ and $t_n = \int_1^n f(x) dx$.  \\
Then both sequences  $\{ s_n \}$ and $\{ t_n \}$ converge or both diverge.  
\end{theorem}

\exercisehead{1} 
\[
\begin{gathered} 
\begin{gathered}
  \frac{3}{4j-3} + \frac{-1}{4j-1} = \frac{ 3(4j-1) + (-1)(4j-3) }{ (4j-3)(4j-1) } = \frac{ 8j}{ (4j-3)(4j-1) } \\  
\end{gathered} \\
\sum_{j=1}^n \frac{j}{ (4j-3)(4j-1) } = \sum_{j=1}^n \left( \frac{ 3/8}{ 4j -3} + \frac{ -1/8}{ 4j-1} \right) \\
\begin{aligned}
  \int_1^n \left( \frac{ 3/8}{ 4x-3} + \frac{ -1/8}{4x-1} \right) dx & = \left. \left( (3/8) \frac{ \ln{ (4x -3 ) }}{ 4} + (-1/8) \frac{ \ln{(4x -1) }  }{ 4 } \right) \right|_1^n = \\
  & = \frac{1}{32} \left. \ln{ \frac{ (4x-3)^3}{ 4x-1}  } \right|_1^n = \frac{1}{32} \ln{ \left( \frac{3 (4n-3)^3 }{ 4n -1 } \right) } 
\end{aligned} \\
\lim_{n \to \infty} \frac{1}{32} \ln{ \left( \frac{3 (4n-3)^3 }{ 4n - 1 } \right) } = \lim_{n\to \infty} \int_1^n \frac{ x dx }{ (4x-3)(4x-1) } dx = \infty, \quad \text{ so  $\sum_{j=1}^n \frac{j}{ (4j-3)(4j-1) } $ diverges as well }
\end{gathered}
\]

\exercisehead{2} 
\[
\begin{gathered}
  \sum_{j=1}^{\infty} \frac{ \sqrt{2j-1} \log{ (4j + 1 )} }{ j(j+1) } = \sum_{j=1}^{\infty} a_j \\
  \begin{aligned}
    a_j & \leq \frac{ \sqrt{4j+1} \log{ (4j + 1 )} }{ j(j+1/4) } = \frac{ 4 \log{ (4j+1)} }{ j(4j+1)^{1/2} } \left( \frac{ (4j+1)}{ (4j+1)} \right) = 4 \frac{ \log{(4j+1)} (4+ 1/j) }{ (4j+1)^{3/2} } \leq \\
    & \leq \frac{ 16 \log{ (4j+1)}}{ (4j+1)^{3/2} } = b_j
  \end{aligned} \\
  \text{ Now use the integral test on $\sum b_j$ to determine the convergence of $\sum b_j$.   } \\
  \begin{aligned}
    \int_1^n \frac{ \log{ (ax+1) }}{ (ax+1)^{3/2} } dx & = \int \left( \frac{ (-2) }{ a(ax+1)^{1/2} }\right)' \log{ (ax+1)} dx = \\
    & = \frac{-2}{ a(ax+1)^{1/2} } \log{(ax+1)} - \int \frac{ -2}{a(ax+1)^{1/2} } \left( \frac{a}{ax+1} \right) = \\
    & = \frac{-2}{a(ax+1)^{1/2}} \log{ (ax+1)} + \frac{ -4}{a(ax+1)^{1/2}} \\
    \lim_{n\to \infty} \int_1^n \frac{ log{(ax+1)}}{(ax+1)^{3/2}} dx & = \lim_{n\to \infty} \left( \frac{-2\log{(an+1)}}{a(an+1)^{1/2}}  + \frac{2 \log{(a+1) }  }{a(a+1)^{1/2}}  + \frac{-4}{a} \left( \left( \frac{1}{an+1} \right)^{1/2} - \frac{1}{ (a+1)^{1/2} } \right) \right) = \\
      & = \frac{2\log{(a+1)}}{ a(a+1)^{1/2} } + \frac{4}{ a(a+1)^{1/2} }
  \end{aligned}
\end{gathered}
\]
Then by integral test, $\sum b_j$ converges.  Since $\sum b_j$ converges, then $\sum a_j$ converges by comparison test.  

\exercisehead{3} $\sum_{j=1}^{\infty} \frac{j +1}{ 2^j} $. 
\[
\begin{aligned}
  \int_1^n \frac{ x+1}{ e^{x\ln{2}} } dx & = \int_1^n (xe^{-x\ln{2}} + e^{-x \ln{2}} ) dx = \left. \left( \frac{x e^{-x\ln{2}}}{ -\ln{2}} + \frac{ e^{-x\ln{2}} }{ -\ln{2}} + \frac{ -e^{-x\ln{2}}}{ (\ln{2})^2 } \right) \right|_1^n \\
  & = \frac{ ne^{-n\ln{2}}}{ -\ln{2}} + \frac{e^{-\ln{2}}}{ \ln{2}} + - \left( \frac{1}{ (\ln{2})^2 } + \frac{1}{ \ln{2}} \right) e^{-n \ln{2}} + \left( \frac{1}{ (\ln{2})^2 } + \frac{1}{ \ln{2} } \right) e^{-\ln{2}} \\
  \lim_{n\to \infty} \int_1^n \frac{x+1}{ e^{x\ln{2}}} dx & = \boxed{ \left( \frac{2}{\ln{2}} + \frac{1}{ (\ln{2})^2 } \right)\left( \frac{1}{2} \right) }
\end{aligned}
\]  
By integral test, $\sum_{j=1}^{\infty} \frac{j+1}{2^j}$ converges.  

\exercisehead{4} $\sum_{j=1}^{\infty} \frac{j^2}{ 2^j }$.  
\[
\begin{aligned}
  \int_1^n \frac{x^2}{2^x} dx & = \int_1^n \frac{x^2}{ e^{x\ln{2}} } = \left. \left( \frac{x^2 e^{-x\ln{2}}}{ -\ln{2}} + \frac{2xe^{-x\ln{2}}}{ - (-\ln{2})^2 } + \frac{ 2 e^{-x\ln{2}}}{ (-\ln{2})^3 } \right) \right|_1^n \\
  \lim_{n\to \infty} \int_1^n \frac{x^2}{ 2^x } dx & = e^{-\ln{2}} \left( \frac{1}{ \ln{2}} + \frac{2}{ (\ln{2})^2 } + \frac{2}{ (\ln{2})^3 } \right) = \frac{1}{2} \left( \frac{1}{ \ln{2}} + \frac{2}{ (\ln{2})^2 } + \frac{2}{ (\ln{2})^3 } \right)
\end{aligned}
\]
By integral test, $\sum_{j=1}^{\infty} \frac{j^2}{2^j}$ converges.  

\exercisehead{5} 
\[
\sum_{j=1}^{\infty} \frac{ |\sin{jx} |}{ j^2} = \sum_{j=1}^{\infty} a_j \leq \sum_{j=1}^{\infty} \frac{1}{j^2}
\]
$\sum_{j=1}^{\infty} \frac{1}{j^2}$ converges since 
\[
\lim_{n\to \infty} \int_1^n \frac{1}{x^s} dx = \lim_{n\to \infty} \left. \frac{ x^{-s +1}}{ -s+1} \right|_1^n = \lim_{ n\to \infty} \left( \frac{1}{ 1-s} \right) \left( \frac{1}{ n^{s-1} } -1 \right) = \frac{1}{ s-1} \text{ if } s> 1
\]
$\sum_{j=1}^{\infty} \frac{ |sin{jx} |}{ j^2 }$ converges by comparison test and integral test.  

\exercisehead{6}
\[
\sum_{j=1}^{\infty} \frac{ 2 + (-1)^j}{ 2^j } = \sum_{j=1}^{\infty} \left( \frac{1}{ 2^{2j-1}} + \frac{3}{ 2^{2j} } \right) = \frac{2 (1/4)}{ 1-1/4} + \frac{3(1/4)}{ 1-1/4} = \boxed{ \frac{4}{3} }
\]

\exercisehead{7} $\sum_{j=1}^{\infty} \frac{j!}{ (j+2)!} $.  
\[
a_j = \frac{j!}{(j+2)!} = a_j = \frac{1}{ (j+1)(j+2)} \leq \frac{1}{j^2} = b_j
\]
Since $\sum b_j$ converges, $\sum a_j $ converges, by comparison test.  

\exercisehead{8} $\sum_{j=2}^{\infty} \frac{ \log{j}}{ j\sqrt{j+1}} = \sum_{j=2}^{\infty} a_j \leq \sum_{j=2}^{ \infty} \frac{ \log{j}}{ j^{3/2} }$
\[
\begin{aligned}
  \int_2^n \frac{ \log{x}}{ x^{3/2}} dx & = \int_2^n (-2 x^{-1/2})' \log{x} = \left. (-2x^{-1/2} \log{x} ) \right|_2^n - \int_2^n - \frac{2x^{-1/2}}{ x } dx = \\
  & = \left. (-2) \left( \frac{ \log{n}}{ n^{1/2}} - \frac{ \log{2}}{ 2^{1/2}} \right) + -4x^{-1/2} \right|_2^n \\
  \lim_{n\to \infty} \int_2^n \frac{ \log{x}}{ x^{3/2} } dx &= 2^{1/2} \log{2} + \frac{4}{\sqrt{2}}
\end{aligned}
\]
So $\sum a_j$ converges by comparison test.  

\exercisehead{9} $\sum_{j=1}^{\infty} \frac{1}{ \sqrt{ j(j+1) }} = \sum_{j=1}^{\infty} a_j$.  
Let $b_j = \frac{1}{j}$.
\[
\lim_{j\to \infty} \frac{a_j}{b_j} = \lim_{j\to \infty} \frac{j}{ \sqrt{ j(j+1)} } = \lim_{j\to \infty} \frac{1}{ \sqrt{ 1 + 1/j}} = 1 
\]
By limit comparison test, since $\sum b_j$ diverges, $\sum a_j$ diverges.  

\exercisehead{10} $\sum_{j=1}^{\infty} \frac{ 1 + \sqrt{j}}{ (j+1)^3 - 1 } = \sum_{j=1}^{\infty} a_j$ \bigskip \\
$b_j = \frac{1}{j^{5/2} }$ 
\[
\lim_{j\to \infty} \frac{a_j}{b_j} = \lim_{j\to \infty} \left( \frac{ 1 + \sqrt{j}}{ (j+1)^3 -1 } \right) j^{5/2} = \lim_{j\to \infty} \frac{ j^3 + j^{5/2}}{ (j+1)^3 -1 } = \lim_{j\to \infty} \frac{ 1 + 1/j^{1/2} }{ (1+ 1/j)^3 - \frac{1}{j^3} } 
\]
By limit comparison test, since $\sum b_j$ converges, $\sum a_j$ converges.  

\exercisehead{11} $\sum_{j=2}^{\infty} \frac{1}{ (\log{j})^s } = \sum a_j$ \medskip \\
$ \boxed{ \text{ If } s \leq 0, \sum a_j \text{ diverges since $\lim_{j\to \infty} a_j \neq 0 $ }  }$

$\text{ If $0 < s \leq 1, \quad \frac{1}{ (\log{j})^s } > \frac{1}{j^s}$, and since $\sum_{j=2}^{\infty} \frac{1}{j^s}$ diverges for $0<s <1$, so does $\sum \frac{1}{ (\log{j})^s }$ }$

\[
\int (\log{x})^{-s} = \int \left( \frac{1}{x} (\log{x})^{-s} \right) x = \int \left( \frac{ (\log{x})^{-s+1} }{ (1-s)} \right)' x = \frac{ (\log{x})^{-s+1}}{ (1-s) } x - \int \frac{ (\log{x})^{1-s} }{ 1-s }
\]
Thus, if $s>1$ has any decimal part, or is an integer, its integral will diverge, so that by integral test, the series diverges.  

\exercisehead{12} $\sum_{j=1}^{\infty} \frac{ |a_j|}{ 10^j }; \quad |a_j| < 10$.  
\[
\sum_{j=1}^{\infty} \frac{ |a_j| }{ 10^j } < \sum_{j=1}^{\infty} \frac{10}{10^j} = \sum_{j=0}^{\infty} \frac{1}{ 10^j } = \frac{1}{ 1- 1/10} = \boxed{ \frac{10}{9} }
\]

\exercisehead{13} $\sum_{j=1}^{\infty} \frac{1}{ 1000j +1 } < \sum_{j=1}^{\infty} \frac{1}{ 1000j} = \frac{1}{1000} \sum_{j=1}^{\infty} \frac{1}{j}$ \\
The series diverges since $\sum \frac{1}{j}$ diverges.  

\exercisehead{14} 
\[
\begin{gathered}
  \sum_{j=1}^{\infty} \frac{ j \cos^2{ (j\pi/3 ) } }{ 2^j } \leq \sum_{j=1}^{\infty} \frac{j}{2^j } = \sum_{j=1}^{\infty} \frac{j }{ e^{j \ln{2} } } \\
  \int_1^{\infty} \frac{x}{ e^{kx} } = \int_1^{\infty} x e^{-kx} = \left. \left( \frac{xe^{-kx}}{ -k } - \frac{ e^{-kx}}{ (-k)^2 } \right) \right|_1^{\infty} = \frac{e^{-k}}{ k} + \frac{ e^{-k}}{ (-k)^2 }
\end{gathered}
\]

\exercisehead{15} $\sum_{j=3}^{\infty} \frac{1}{ j \log{j} (\log{ (\log{j}) } )^s }$ 
\[
\int \frac{1}{ x \log{x} ( \log{ (\log{x} ) } )^s } = \int \left( \frac{ (\ln{ (\ln{x}) } )^{-s+1} }{ -s +1 } \right)' = \begin{cases} \frac{ (\ln{ (\ln{x}) } )^{-s+1}}{ -s +1 } & \text{ if } s > 1, \, s<1, \, s\neq 0 \\
  \ln{ (\ln{ ( \ln{x} ) } ) } & \text{ if } s = 1 \\
  \ln{ (\ln{x} )} & \text{ if } s = 0 
\end{cases} 
\]
Converges, by integral test, for $s>1$

\exercisehead{16} Converges by integral test since
\[
\int_1^{\infty} x e^{-x^2 } = \left. \left( \frac{ e^{-x^2 }}{ -2} \right) \right|_1^{\infty} = 0 + \boxed{ \frac{e^{-1}}{ 2 } }
\]

\exercisehead{17} \textbf{ I drew a picture to help me see what's going on. } 
\[
\begin{gathered}
  \frac{\sqrt{x}}{ 1+x^2 } \leq \frac{ \sqrt{x} }{ 1 } \\
  \int_0^{1/n} \sqrt{x} dx = \left. \frac{2}{3} x^{3/2} \right|_0^{1/n} = \frac{2}{3} \left( \frac{1}{n} \right)^{3/2} \\
  \sum_{j=2}^{\infty} \frac{2}{3} \left( \frac{1}{j} \right)^{3/2} = \frac{2}{3} \sum_{j=2}^{\infty} \frac{1}{j^{3/2}}
\end{gathered}
\]
So $\sum_{1}{n} \int_0^{1/n} \frac{ \sqrt{x}}{ 1+x^2 } dx $ converges by comparison test.  

\exercisehead{18} 
\[
\begin{gathered}
\begin{aligned}
  (e^{-\sqrt{x}})' & = e^{-\sqrt{x}} \frac{ -1 }{ 2 \sqrt{x} } \\
  (\sqrt{x} e^{-\sqrt{x}} )' & = \frac{ -e^{-\sqrt{x}} }{ 2 } + \frac{1}{2\sqrt{x}} e^{-\sqrt{x}}
  (\sqrt{x} e^{-\sqrt{x}} + e^{-\sqrt{x}} )' & = \frac{e^{-\sqrt{x}}}{ 2 } 
\end{aligned} \\
\begin{aligned}
\int_n^{n+1} e^{-\sqrt{x}} dx & = 2 \left. (\sqrt{x} + 1 ) e^{-\sqrt{x}} \right|_n^{n+1} = 2 \left. \left( \frac{ \sqrt{x}}{ e^{\sqrt{x}} } + \frac{1}{ e^{\sqrt{x}} } \right) \right|_n^{n+1} = \\
& = 2 \left( \frac{ \sqrt{ n+1}}{ e^{\sqrt{n+1}} } + \frac{1}{ e^{\sqrt{n+1} } } - \frac{ \sqrt{n}}{ e^{\sqrt{n}} } - \frac{1}{ e^{\sqrt{n}} } \right) 
\end{aligned} \\
2\sum_{j=1}^{\infty} \left( \frac{ \sqrt{j+1} }{ e^{\sqrt{j+1}} } - \frac{ \sqrt{j}}{ e^{\sqrt{j}} } + \frac{1}{ e^{\sqrt{j+1}}} - \frac{1}{ e^{\sqrt{j}} } \right) = \boxed{ -\frac{1}{ e } }
\end{gathered}
\]
\textbf{ Note the use of telescoping sum in the last step.}  The series converges.  

\exercisehead{19} 
$f$ is nonnegative and increasing.  $\xrightarrow{ \text{ $f$ increasing } } f(k) \leq f(x) \leq f(k+1)$, $k \leq x \leq k+1$  \\
\[
\begin{gathered}
  \int_k^{k+1} f(k) dx \leq \int_k^{k+1} f(x) dx \leq \int_k^{k+1} f(k+1) dx \Longrightarrow f(k) \leq \int_k^{k+1} f(x) dx \leq f(k+1) \xrightarrow{ \sum} \\
  \xrightarrow{\sum} \sum_{k=1}^{n-1} f(k) \leq \sum_{k=1}^{n-1} \int_k^{k+1} f(x) dx = \int_1^n f(x)dx \leq \sum_{k=1}^{n-1} f(k+1) = \sum_{k=2}^n f(k) 
\end{gathered}
\]\\

Note that for the summation on the integral, we simply relied on the properties of integrals to get the final integral.  \\

$\int_1^n f(x) dx = \int_1^n \log{x} dx = \left. (x \ln{x} -x ) \right|_1^n = n \ln{n} - n +1 $.  
\[
\begin{gathered}
  \sum_{k=1}^{n-1} \ln{k} \leq \int_1^n \ln{x} \leq \sum_{k=2}^n \ln{ (k) } \Longrightarrow   \sum_{k=1}^{n-1} ln{k} \leq n \ln{n} - n +1  \leq \sum_{k=2}^n \ln{ (k) } \\ 
\begin{aligned}
  \exp{ \left( \sum_{k=1}^{n-1} \ln{ k } \right) } & = (n-1)! \leq n^n e^{-n+1} \\
  \exp{ \left( \sum_{k=2}^{n} \ln{ k } \right) } & = (n)! \geq n^n e^{-n+1} 
\end{aligned} \\
\Longrightarrow \frac{e^{1/n}}{e} < \frac{(n!)^{1/n}}{ n } < \frac{ e^{1/n} n^{1/n} }{ e } 
\end{gathered}
\]

%-----------------------------------%-----------------------------------%-----------------------------------
\subsection*{ 10.16 Exercises - The root test and the ratio test for series of nonnegative terms }
%-----------------------------------%-----------------------------------%-----------------------------------

\exercisehead{1} $\sum_{j=1}^{\infty} \frac{ (j!)^2 }{ (2j)!} $
\[
\frac{ ((j+1)!)^2 }{ (2j+2) } \left( \frac{ (2j)!}{ (j!)^2 } \right) = \frac{ (j+1)^2 }{ (2j+2)(2j+1)} = \frac{ j^2 + 2j +1 }{ 4j^2 + 6j + 2 } \xrightarrow{ j\to \infty } \frac{1}{4}
\]
Converges by ratio test.  

\exercisehead{2} $\sum_{j=1}^{\infty} \frac{ (j!)^2 }{ 2^{j^2 } }$.  
\[
\frac{ ((j+1)! )^2 }{ 2^{(j+1)^2 } } \frac{ 2^{j^2 } }{ (j!)^2 }  = \frac{ (j+1)^2 2^{j^2 }}{ 2^{j^2 + 2j + 1 } } = \frac{ j^2 + 2j + 1 }{ 2 e^{ j \ln{2} } } \xrightarrow{ j \to \infty } 0 
\]
Converges by ratio test.

\exercisehead{3} $\sum_{j=1}^{\infty} \frac{2^j j! }{ j^j }$ 
\[
\frac{ 2^{j+1} (j+1)! }{ (j+1)^{j+1} } \frac{ j^j }{ 2^j j! } = \frac{ 2(j+1) }{ (j+1) } \left( \frac{ 1 }{ 1 + 1/j } \right)^j \xrightarrow{ j\to \infty} \frac{2}{e} < 1 
\]
Converges by ratio test. 

\exercisehead{4} $\sum_{j=1}^{\infty} \frac{ 3^j j! }{ j^j }$
\[
\frac{ 3^{j+1} (j+1)! }{ (j+1)^{j+1 } } \left( \frac{j^j }{ 3^j j! } \right) = 3 \left( \frac{1}{ (1+1/j)^j } \right) \xrightarrow{ j \to \infty } \frac{3}{e} > 1 
\]
Diverges by ratio test. 

\exercisehead{5} $\sum_{j=1}^{\infty} \frac{j!}{ 3^j }$.  
\[
\frac{ (j+1)! }{ 3^{j+1 } } \frac{3^j }{ j! } = \frac{j+1}{3}
\]
Diverges by ratio test. 

\exercisehead{6} $\sum_{j=1}^{\infty} \frac{j!}{ 2^{2j } }$
\[
\frac{(j+1)!}{ 2^{2(j+1)} } \frac{ 2^{2j} }{ j! } = \frac{ (j+1) }{ 4 } 
\]
Diverges.  

\exercisehead{7} $\sum_{j=2}^{\infty} \frac{1}{ (\log{j})^{1/j } } $ \bigskip \\
Draw a picture to see what's going on.  
\[
\begin{gathered}
  \sum_{j=2}^{\infty} \frac{1}{ (\log{j})^{1/j} } = \sum_{j=2}^{\infty} \exp{ \left( \frac{-1}{j} \ln{ \log{j} } \right) } \\
  0 < \frac{ \ln{ \log{j} } }{ j } \quad \text{ for $j >3 $ } \\
 \frac{ \ln{ (\log{j}) }}{ j } < \frac{ \ln{j} }{ j } \quad \Longrightarrow \lim_{j\to \infty} \frac{ \ln{ (\log{j} ) }}{ j } < \lim_{j\to \infty} \frac{ \ln{j}}{ j } = 0 \\
 \Longrightarrow \lim_{j\to \infty} \frac{-1}{j} \ln{ (\log{j} ) } = 0 \quad \text{ so then } \\
 \lim_{j\to \infty} \exp{ \left( \frac{-1}{j} \ln{ (\log{j} ) } \right) } = 1
\end{gathered}
\]
 $\sum_{j=2}^{\infty} \frac{1}{ (\log{j})^{1/j } } $ diverges because the $a_j$ term doesn't go to zero.  

\exercisehead{8} $\sum_{j=1}^{\infty} (j^{1/j} - 1)^j $
\[
((j^{1/j}-1)^j )^{1/j} = (e^{ \frac{1}{j} \ln{j} } -1 ) \xrightarrow{j\to \infty} 0
\]
Converges by root test. 

\exercisehead{9} $\sum_{j=1}^{\infty} e^{-j^2 } $
\[
(e^{-j^2 })^{1/j} = e^{-j} \xrightarrow{ j\to \infty} 0
\]
Converges by root test.

\exercisehead{10} I systematically tried ratio test and then root test.  Both were inconclusive.  \bigskip \\
Consider comparison with $\sum \frac{1}{j}$.  
\[
\left( \frac{ e^{j^2} - j }{ je^{j^2 } } \right) j = \frac{ e^{j^2 } - j }{ e^{j^2 } } =1 - \frac{j}{ e^{j^2 } } \xrightarrow{j\to \infty} 1
\]
By limit comparison test, since $\sum \frac{1}{j}$ diverges, so does $\sum \left( \frac{1}{j} - \frac{1}{e^{j^2 } } \right)$.  

\exercisehead{11} $\sum_{j=1}^{\infty} \frac{ (1000)^j}{ j! } = e^{1000}$

\exercisehead{12} $\sum_{j=1}^{\infty} \frac{ j^{j+1/j}}{ (j+1/j)^j }$.  
\[
a_j^{1/j} = \frac{ j^{1/j^2 }}{ 1 + \frac{1}{j^2} } \quad \lim_{j\to \infty} a_j^{1/j} = \lim_{j\to \infty} \frac{ \exp{ \left( \frac{1}{j^2} \ln{j} \right)} }{ 1 + \frac{1}{j^2 } } = 1 
\]
Note that root test is inconclusive.

\[
\begin{gathered}
  a_j = \frac{ \exp{ \left( \frac{1}{j} \ln{j} \right) } }{ \left( 1 + \frac{1}{j} \right)^j } \geq \frac{ \exp{ \left( \frac{1}{j} \ln{j} \right)} }{ \left( 1 + \frac{1}{j^2 } \right)^{j^2 } } \\
  \lim_{j\to \infty} a_j \geq \lim_{j\to \infty} \frac{ \exp{ \left( \frac{1}{j} \ln{ j} \right)}}{ \left( 1 + \frac{ 1 }{ j^2 } \right)^{j^2 } } = \frac{1}{e} > 0 
\end{gathered}
\]
Diverges since $\lim_{j\to \infty} a_j >0$.  

\exercisehead{13} $\sum_{j=1}^{\infty} \frac{ j^3( \sqrt{2} + (-1)^j )^j }{ 3^j }$.  
\[
\left( \frac{j^3 (\sqrt{2} + (-1)^j )^j }{ 3^j } \right)^{1/j} = \frac{ j^{3/j} (\sqrt{2} + (-1)^j ) }{ 3 } = \frac{ e^{ \frac{3}{j} \ln{j} } (\sqrt{2} + (-1)^j ) }{ 3 } \xrightarrow{ j \to \infty} \frac{ (\sqrt{2} + (-1)^j ) }{ 3 } < 1
\]
Converges by root test.  

\exercisehead{14} $\sum_{j=1}^{\infty} r^j |\sin{jx} |$.  \medskip \\
If $0<r<1$.  
\[
\begin{gathered}
  \sum_{j=1}^{\infty} r^j |\sin{jx}| < \sum_{j=1}^{\infty} r^j \quad \\
  \text{ so by comparison test, $\sum_{j=1}^{\infty} r^j |\sin{jx}| $ converges for $0 <r <1$}
\end{gathered}
\]
If $r \geq 1$,
\[
\lim_{j\to \infty} r^j | \sin{jx} | \neq 0 \text{ so } \sum_{j=1}^{\infty} r^j |\sin{jx} | \text{ diverges, unless $jx = \pi j$ }
\]

\exercisehead{15} 
\begin{enumerate}
  \item $c_j = b_j - \frac{ b_{j+1} a_{j+1}}{ a_j } > 0 \quad \forall j \geq N$.  Then there must be a positive number $r$ that's in between $c_j$ and $0$.  \medskip \\
    \[
\begin{gathered}
  a_j b_j - a_{j+1}b_{j+1} \geq r a_j \\
  r \sum_{j=N}^n a_j \leq \sum_{j=N}^n (a_j b_j - a_{j+1}b_{j+1} ) = a_N b_N - a_{n+1}b_{n+1} \leq a_N b_N \\
  \Longrightarrow \sum_{j=N}^n a_j \leq \frac{ a_N b_N}{ r } 
\end{gathered}
\]
  \item $c_n <0$
\[
\begin{gathered}
\begin{gathered}
  b_j - \frac{ b_{j+1} a_{j+1} }{ a_j } < 0 \\
  a_j b_j < b_{j+1} a_{j+1} \Longrightarrow \frac{ b_j }{b_{j+1}} < \frac{ a_{j+1} }{ a_j }
\end{gathered} \quad 
\begin{gathered}
  \sum \frac{1}{b_j} \text{ diverges, so } \\
  \lim_{j\to \infty} \frac{b_j}{ b_{j+1} } \geq 1 \text{ by ratio test }
\end{gathered} \\
\xrightarrow{ j\to \infty} 1 \leq \frac{b_j}{ b_{j+1} } < \frac{ a_{j+1} }{ a_j}
\end{gathered}
\]
So by ratio test, $\sum a_j$ diverges.  
\end{enumerate}

\exercisehead{16} $b_{n+1} = n; \, b_n = n-1$.  \medskip \\
$c_n = n-1 - \frac{n a_{n+1}}{ a_n} \geq r \Longrightarrow \frac{ a_{n+1}}{ a_n } \leq 1 - \frac{1}{n} - \frac{r}{n}$.  \medskip \\
Using Exercise 15, $\sum a_n$ converges.  

$\sum \frac{1}{b_n}$ diverges since $\sum \frac{1}{b_n}$ is a harmonic series of $s=1$.  \medskip \\
\[
n-1 - \frac{n a_{n+1}}{ a_n } \leq 0 \Longrightarrow 1 - \frac{1}{n} \leq \frac{a_{n+1}}{ a_n}
\]

\exercisehead{17} For some $N \geq 1, \, s>1, \, M > 0$, and given that 
\[
\frac{a_{n+1}}{ a_n } = 1 - \frac{A}{n} + \frac{f(n)}{ n^s} = 1  - \left( \frac{ A - \frac{f(n)}{ n^{s-1}} }{ n } \right)
\]
Consider $A - \frac{f(n)}{n^{s-1}}$.  \medskip \\
\phantom{ Consider} Since $|f(n)| < M$, $f(n)$ is finite, so consider $s$ larger than $1$ and $n$ going to infinity so that $\frac{f(n)}{n^{s-1}} \to 0$.  \medskip \\
Using Exercise 16, for $\sum a_j$ to converge, $A - \frac{f(n)}{ n^{s-1}} = 1 +r$ where $r>0$, for all $n \geq N$, where $N$ is some positive number.  Let $r = \frac{ M }{ N^{s-1}}$ so that 
\[
A = 1 + r + \frac{f(n)}{ n^{s-1}} > 1
\]
If $A>1$, then $\sum a_n$ converges.  

If $A=1$, then consider using Exercise 15 and $b_n = n \log{n}$.  
\[
\begin{gathered}
  \begin{aligned}
c_n & = b_n - b_{n+1} \frac{a_{n+1}}{ a_n } = (n-1) \log{ (n-1) } - n \log{n} \left( \frac{a_{n+1}}{ a_n } \right) \\
& = (n-1)\log{ (n-1)} - n \log{n} \left( 1 - \frac{1}{n} + \frac{f(n)}{ n^s} \right) = (n-1)\log{ \left( \frac{ (n-1) }{ n } \right) } - n\log{n} \left( \frac{f(n)}{ n^s } \right)  = \\
& = -(n-1) \log{ \left( \frac{ n }{ (n-1) } \right) } - n\log{n} \left( \frac{f(n) }{n^s} \right)
\end{aligned} \\
  \text{ since $\frac{ \log{n}}{ n^{s-1} } \xrightarrow{n\to \infty} 0$, } \\
   -(n-1) \log{ \left( \frac{ n }{ (n-1) } \right) } - n\log{n} \left( \frac{f(n) }{n^s} \right) < 0 \quad \text{ for $n$ large enough } \\
   \text{ since $c_n < 0$ for $n\geq N$ for some $N>0$, then by Exercise 15, $\sum a_n$ is divergent. }
\end{gathered}
\]

Given that $A<1$, then for $A - \frac{f(n)}{ n^{s-1} }$, choose $N >0$ so that $\frac{M}{n^{s-1}} < \epsilon <1$ and that $A - \frac{f(n)}{n^{s-1}} \leq A + \frac{M}{n^{s-1}} = A + \epsilon \leq 1$.  We can always choose $\epsilon$ small enough because there's always a real number in between $A$ and $1$ (Axiom of Archimedes).  
\[
\begin{gathered}
  A - \frac{ f(n)}{ n^{s-1} } \leq 1 \quad \Longrightarrow - \left( A - \frac{ f(n)}{ n^{s-1}} \right) \geq -1  \\
\Longrightarrow \text{ using Exercise 16, } \frac{ a_{n+1}}{ a_n } = 1 - \frac{ A + \frac{ f(n)}{ n^{s-1} } }{ n } \geq 1 - \frac{1}{n } \, \text{ for all $n \geq N$ } 
\end{gathered}
\]

\exercisehead{18}
\[
\left( \frac{ 1 \cdot 3 \cdot 5 \dots (2n+1) }{ 2 \cdot 4 \cdot 6 \dots (2n+2) } \cdot \frac{ 2 \cdot 4 \cdot 6 \dots (2n) }{ 1 \cdot 3 \cdot 5 \dots (2n-1) } \right)^k = \left( \frac{2n+1}{ 2n+2} \right)^k \xrightarrow{ n\to \infty} 1 
\]
Ratio test fails.  
\[
\begin{gathered}
  \begin{aligned} 
    \frac{a_{n+1}}{ a_n } & = \left( \frac{ 2n+1}{ 2n+2} \right)^k = \left( 1 + \frac{-1}{2n+2 } \right)^k = \left( 1 + \frac{-1/2}{ n+1} \right)^k = \\
    & = \sum_{j=0}^k \binom{k}{j} \left( \frac{ -1/2}{ n+1} \right)^j = 1  + k \left( \frac{-1/2}{ n+1} \right) + \sum_{j=2}^{k} \binom{k}{j} \left( \frac{-1/2}{ n+1} \right)^j 
  \end{aligned} \\
  \text{ Note that for $k< \infty, \sum_{j=2}^k \binom{k}{j} \left( \frac{-1/2}{ n+1 } \right)^j < \infty$ } \\
  \text{ Let $\left| \sum_{j=2}^k \binom{k}{j} \left( \frac{-1/2}{ n+1} \right)^j \right| \leq M$ } \\
  \text{ $k/2 = A > 1$ or $k >2$ means $\sum a_j$ converges} \quad \text{ $k/2 =A \leq 1$ or $k \leq 2$ means $\sum a_j$ diverges }
\end{gathered}
\]

%-----------------------------------%-----------------------------------%-----------------------------------
\subsection*{ 10.20 Exercises - Alternating series, Conditional and absolute convergence, The convergence tests of Drichlet and Abel }
%-----------------------------------%-----------------------------------%-----------------------------------

We will be using Leibniz's test alot, initially.
\begin{theorem}[Leibniz's Rule]
If $a_j$ is a monotonically decreasing sequence with limit $0$, \medskip \\
$\quad \quad \sum_{j=1}^{\infty} (-1)^{j-1} a_j$ converges. \bigskip \\
If $S = \sum_{j=1}^{\infty} a_j, \quad s_n = \sum_{j=1}^n (-1)^{j-1} a_j, $ 
\[
0 < (-1)^j (S-  s_j) < a_{j+1}
\]
\end{theorem}

\exercisehead{1} $\sum_{j=1}^{\infty} \frac{ (-1)^{j+1}}{ \sqrt{j}}$.  $\lim_{j\to \infty} \frac{1}{ \sqrt{j}} = 0 $ Converges conditionally.  

\exercisehead{2} $\sum_{j=1}^{\infty} (-1)^j \frac{ \sqrt{j}}{ j+100 }$  \quad \quad $\lim_{j\to \infty} \frac{ \sqrt{j}}{ j+100 }=0$.  Converges by Leibniz's test.  

$\frac{1}{ \sqrt{j} + \frac{100}{ \sqrt{j} } } \geq \frac{1}{ 101 \sqrt{j }} $, so by comparison test, the series diverges absolutely.  So the alternating series converges conditionally by comparison test.  

\exercisehead{3} $\sum_{j=1}^{\infty} \frac{ (-1)^{j-1} }{ j^s }$  If $s>1$, then the series absolutely converges.  $\lim_{j\to \infty} \frac{1}{j^s} = 0$ if $s>0$.  Converges conditionally for $0<s<1$.  Otherwise, if $s<0$ the series diverges absolutely.

\exercisehead{4} $\sum_{j=1}^{\infty} (-1)^j \left( \frac{ 1 \cdot 3 \cdot 5 \dots (2j-1) }{ 2 \cdot 4 \cdot 6 \dots (2j) } \right)^3$.  
\[
\frac{ a_{j+1}}{ a_j } = \frac{ 1 \cdot 3 \cdot 5 \cdot (2j+1) }{ 2 \cdot 4 \cdot 6 \dots (2j+2) } \frac{ 2 \cdot 4 \cdot 6 \dots (2j) }{ 1 \cdot 3 \cdot 5 \dots (2j-1) } = \frac{ 2j+1}{ 2j+2}
\]
Absolutely converges.  

\exercisehead{5} $\sum_{j=1}^{\infty} \frac{ (-1)^{j(j-1)/2} }{ 2^j}$ converges since $\lim_{j\to \infty} \frac{ (-1)^{j(j-1)/2} }{ 2^j } =0; \quad \sum_{j=1}^{\infty} \frac{1}{2^j } = \frac{ 1/2}{ 1- 1/2} = \boxed{1} $.  Absolutely converges.  

\exercisehead{6} $\sum_{j=1}^{\infty} (-1)^j \left( \frac{ 2j + 100}{ 3j+1} \right)^j$.  
\[
\begin{gathered}
\begin{aligned}
  \exp{ \left( j \ln{ \left( \frac{ 2j+100}{ 3j+1 } \right) } \right) } & = \exp{ \left( j \ln{ \left( \frac{2}{3} + \frac{ 298}{ 9j+3 } \right) } \right) } \leq \exp{ \left( j \left( \ln{ \left( \frac{2}{3} \right) } + \frac{1}{ \frac{2}{3} } \left( \frac{ 298}{ 9j +3 } \right) \right) \right) } = \\
  & = \exp{ \left( j \ln{ \frac{2}{3} } + \frac{ 146}{ 3 + 1/j} \right) }
\end{aligned} \\
0 \leq \lim_{j \to \infty} \exp{ \left( j \ln{ \left( \frac{2j +100}{ 3j + 1 } \right) } \right) } \leq \lim_{j\to \infty} \\exp{ \left( j \ln{ \frac{2}{3} } + \frac{146}{3 + 1/j} \right) } = 0 \\
\Longrightarrow \lim_{j\to \infty} \exp{ \left( j \ln{ \left( \frac{2j+100}{ 3j+1 } \right) } \right) } = 0 
\end{gathered}
\]
So the alternating series converges.  
\[
\begin{gathered}
  \left( \frac{2j+100}{ 3j+1 } \right) < \frac{2j+100}{ 3j } < \frac{2.5j}{ 3j } = \frac{5}{6} \quad (\text{ for } j \geq 200 ) \\
  \Longrightarrow \left( \frac{2j+100}{ 3j +1 } \right)^j  < \left( \frac{5}{6} \right)^j  \, \text{ for } j \geq 200
\end{gathered}
\]
So the series absolutely converges by comparison with a geometric series.  

\exercisehead{7} $\sum_{j=2}^{\infty} \frac{ (-1)^j }{ \sqrt{j} + (-1)^j }$.  
\[
\begin{gathered}
  \lim_{j\to \infty} \frac{ 1 }{ \sqrt{j} + (-1)^j } \text{ doesn't exist since ???} \\
\end{gathered}
\]
To show divergence, we usually think of either \emph{taking the general term and finding the limit} (and if it goes to a nonzero constant, then it diverges), or we use ratio, root, comparison test on the general term.  Since this is an alternating series, I've observed that the general term \emph{ is a sum of two adjacent terms, one even and one odd}.  
\[
\begin{gathered}
  \frac{ (-1)^j }{ \sqrt{j} + (-1)^j } \\
  \begin{aligned}
    \frac{ (-1)^{2j} }{ \sqrt{2j} + (-1)^{2j}} + \frac{ (-1)^{2j + 1} }{ \sqrt{ 2j+1} + (-1)^{2j+1} } & = \frac{1}{ \sqrt{2j} + 1 } + \frac{-1}{ \sqrt{2j+1} + 1 } = \frac{ \sqrt{2j+1}  - 1 - (\sqrt{2j} + 1 ) }{ (\sqrt{2j} + 1 )(\sqrt{ 2j+1} - 1 ) } = \\ 
    & = \frac{ \sqrt{ 2j+1} - \sqrt{2j} - 2 }{ (\sqrt{2j} + 1 )( \sqrt{2j+1} - 1 ) } = \frac{ \sqrt{2j} \sqrt{ 1 + \frac{1}{2j} } - \sqrt{2j} - 2 }{ (\sqrt{2j} + 1 )(\sqrt{2j} \sqrt{ 1 + \frac{1}{2j} } - 1 ) } = \\
    \xrightarrow{ \text{ for $j$ large } } & \simeq \frac{ \sqrt{2j} \left( 1 + \frac{1}{4j} \right) - \sqrt{2j} - 2 }{ (\sqrt{2j} + 1 )( \sqrt{2j} \left( 1 + \frac{1}{4j} \right) - 1 ) } = \frac{-2}{j} \left( \frac{ 1 - \frac{1}{4 \sqrt{2j}} }{ 2 - \frac{1}{2j} + \frac{1}{ 2\sqrt{2} j^{3/2} } } \right) 
\end{aligned}
\end{gathered}
\]
Every term, since we considered any $j$, will contain $-2$.  So we factor it out.  Then
\[
\frac{1}{j} \left( \frac{ 1 - \frac{1}{4 \sqrt{2j} } }{ 2 - \frac{1}{2j} + \frac{1}{ 2 \sqrt{2j} j^{3/2}} } \right) > \frac{1}{j} \left( \frac{ 1-  \frac{1}{ 4\sqrt{2} j} }{ 4 - \frac{1}{ \sqrt{2} j } } \right) = \frac{1}{4j}
\]
By comparison test to $\frac{1}{j}$ the series diverges.  

\exercisehead{8} Using the theorem
\begin{theorem} \quad \\
  Assume $\sum |a_j| $ converges \medskip \\
  \phantom{ Assu } Then $\sum a_j$ converges and $|\sum a_j | \leq \sum |a_j |$.  
\end{theorem}
So using the contrapositive,  \medskip \\
If $\sum a_j$ diverges, \\
\phantom{ If } $\sum |a_j|$ diverges.  

\[
\frac{1}{ j^{1/j}} = \frac{1}{ e^{\frac{1}{j} \ln{j } } } \quad
\lim_{j\to \infty} \frac{1}{ j^{1/j} } = \frac{1}{ \exp{ \left( \lim_{j\to \infty} \frac{1}{j} \ln{j} \right) } } = 1 
\]
Diverges absolutely.  

\exercisehead{9} $\sum_{j=1}^{\infty} (-1)^j \frac{j^2}{ 1 + j^2 }$  Diverges absolutely.  
\[
\begin{gathered}
\begin{aligned}
  \frac{ (2j)^2 }{ 1 + (2j)^2 } - \frac{ (2j-1)^2 }{ 1 + (2j-1)^2 } & = \frac{ 4j^2 }{ 1 + 4j^2 } \left( \frac{ 4j^2 -4j + 2 }{ 4j^2 - 4j + 2 } \right) - \frac{ (4j^2 - 4j +1 ) }{ (4j^2 - 4j +2 ) } \frac{ (1+4j^2 ) }{ 1 + 4j^2 } \\
  & = \frac{ 4j-1}{ 2 (1+4j^2) (2j^2 - 2j + 1 ) }
\end{aligned} \\
\frac{ 4j - 1 }{ 2 ( 1 + 4j^2 ) ( 2j^2 - 2j + 1 ) } (j^3 ) = \frac{ 4 - 1/j }{ 2(4+1/j^2 )(2 - 2/j + 1/j^2 ) } = \frac{1}{4}
\end{gathered}
\]
By limit comparison test, with $\sum \frac{1}{j^3 }$, $\sum_{j=1}^{\infty} \frac{j^2}{1+j^2}$ converges.  

\exercisehead{10} $\sum_{n=1}^{\infty} \frac{ (-1)^n}{ \log{ (e^n + e^{-n} ) } }$
\[
\lim_{n\to \infty} \frac{1}{ \log{ (e^n + e^{-n} ) } } = \lim_{n\to \infty} \frac{1}{n} = 0
\]
The series converges.  

\[
\lim_{ n \to \infty} \frac{n}{ \log{ (e^n + e^{-n} ) } } = \lim_{n \to {\infty} } \frac{ n}{ \log{e^n } + \log{ (1+ e^{-2n} ) } } = \lim_{n\to \infty} \frac{1}{ \left( \frac{ n + \log{ (1+ e^{-2n} ) } }{ n } \right) } = 1
\]
Since $\sum \frac{1}{n}$ diverges, $\sum \frac{1}{ \log{ (e^n + e^{-n} ) } }$ diverges.  

\exercisehead{11} $\sum_{j=1}^{\infty} \frac{ (-1)^j }{ j \log^2{(j+1) } }$  

$\lim_{j\to \infty} \frac{1}{ j \log^2{(j+1) } } = 0 $ so by Leibniz's test, the alternating series.  

\[
\begin{gathered}
  \frac{1}{ n \log^2{ (n+1) } } < \frac{1}{ n \log^2{(n) }} \\
  \int \frac{1}{ n \log^2{n} } = \int \left( \frac{-1}{ \log{n} } \right)' = \frac{-1}{ \log{n} } \xrightarrow{n\to \infty} \frac{1}{\log{2}}
\end{gathered}
\]
Converges by comparison test to $\frac{1}{n \log^2{n}}$, which converges by integral test.  So the series absolutely converges.  

\exercisehead{12} $\sum_{j=1}^{\infty} \frac{(-1)^j}{ \log{ (1 + 1/j) } }$ diverges absolutely.  
\[
  \begin{aligned}
    \frac{ (-1)}{ \log{ \left( 1 + \frac{1}{ 2j - 1 } \right) } }  + \frac{1}{ \log{ \left( 1 + \frac{1}{2j } \right) } } & = \frac{ -1}{ \log{ \left( \frac{2j}{ 2j - 1 } \right) } } + \frac{1}{ \log{ \left( \frac{2j+1}{ 2j } \right) } } = \frac{ -\log{ \left( \frac{2j+1}{ 2j } \right) } + \log{ \left( \frac{2j}{ 2j -1} \right) } }{ \log{ \left( \frac{2j}{ 2j-1} \right) } \log{ \left( \frac{2j+1}{ 2j } \right) } } = \\
    & = \frac{ \log{ \left( \frac{ (2j)^2 }{ 4j^2 - 1 } \right) } }{ \log{ \left( \frac{2j}{2j-1} \right)} \left( \log{ \left( 1 + \frac{1}{2j} \right) } \right) } = \frac{ \log{ \left( 1 + \frac{1}{ 4j^2 - 1 } \right) } }{ \log{ \left( 1 + \frac{1}{ 2j - 1 } \right) } \log{ \left( 1 + \frac{1}{2j} \right) } } = \\
    & = \frac{ \frac{1}{ 4j^2 -  1 } + o\left( \frac{1}{4j^2 - 1 } \right) }{ \left( \frac{ 1 }{ 2j - 1 } + o\left( \frac{1}{ 2j - 1 } \right) \right)\left( \frac{1}{2j} + o\left( \frac{1}{2j} \right) \right)} \approx  \\
    & \approx \frac{4j^2 - 2j }{ 4j^2 -1} = \frac{ 1 - \frac{1}{2j} }{ 1 - \frac{1}{4j^2 } } \xrightarrow{j\to \infty} 1 
    \end{aligned}
\]
So the alternating series diverges.  

\exercisehead{13} $\sum_{j=1}^{\infty} \frac{(-1)^j j^{37} }{ (j+1)! }$  Use the ratio test.  
\[
\frac{a_{j+1}}{ a_j } = \frac{(j+1)^37 }{ (j+2)! }\frac{ (j+1)! }{ j^37} = \left( \frac{1}{ j+2} \right)\left( 1 + \frac{1}{j} \right)^{37} \to 0
\]
Converges for $\sum |a_j|$.  Then $\sum a_j$ converges.  The series absolutely converges.  

\exercisehead{14} $\sum_{n=1}^{\infty} (-1)^n \int_n^{n+1} \frac{e^{-x}}{ x } dx $
\[
\begin{aligned}
  \int_n^{n+1} \frac{e^{-x}}{x } dx \leq \int_n^{n+1} \frac{1}{e^{2x}} dx = \left. \frac{e^{-2x}}{ -2} \right|_n^{n+1} = \left( \frac{-1}{2} \right) \left( \frac{1}{ e^{2(n+1)}}  - \frac{1}{e^{2n}} \right) = \\
  & = \frac{e^2 - 1 }{ 2 e^{2n+2} } < 1 
\end{aligned}
\]
Converges absolutely.  

\exercisehead{15} $\sum_{j=1}^n \sin{ (\log{j})}$ 

$\lim_{j\to \infty} \sin{ (\log{j}) }$ doesn't exist.  So the series is divergent.  

\exercisehead{16} $\sum_{j=1}^{\infty} \log{ \left( j \sin{\frac{1}{j} } \right) }$
Note that 
\[
\begin{aligned}
  \log{\left( j \sin{\frac{1}{j} } \right) } & = \log{ \left( \frac{ \sin{1/j}}{ 1/j} \right) } \\
  \lim_{j\to \infty} \log{ \left( \frac{\sin{1/j} }{ 1/j } \right) } & = \log{ \left( \lim_{j\to \infty} \frac{ \sin{1/j}}{ 1/j } \right) } = \log{1} = 0 
\end{aligned}
\]
\[
\begin{gathered}
  \sin{ \frac{1}{j} } = \sum_{k=0}^{\infty} \frac{ \left( \frac{1}{j} \right)^{2k+1} }{ (2k+1)!} (-1)^k \\
  \begin{aligned}
    \log{ \left( j \sum_{k=0}^{\infty} \frac{ \left( \frac{1}{j} \right)^{2k+1} (-1)^k }{ (2k+1)! } \right) } & = \log{ \left( 1 + \frac{-1}{ 6j^2 } + \sum_{k=2}^{\infty} \frac{ \left( \frac{1}{j} \right)^{2k} (-1)^k }{ (2k+1)! } \right) } \geq \\
    & \geq \log{ \left( 1 + - \frac{1}{6j^2 } \right) } \geq \frac{-1}{6j^2 }
  \end{aligned}
\end{gathered}
\]
The series absolutely converges.  

\exercisehead{17}  $\sum_{j=1}^{\infty} (-1)^j \left( 1 - j \sin{\frac{1}{j} } \right)$  
\[
\begin{gathered}
\begin{aligned}
  \left( 1 - x \sin{\frac{1}{x} } \right)' & = - \sin{\frac{1}{x}} - x \cos{\frac{1}{x}} \left( \frac{-1}{x^2} \right) = \\
  & = -\sin{ \frac{1}{x} } + \frac{1}{x} \cos{ \frac{1}{x} } = \frac{ -x \sin{\frac{1}{x} } + \cos{\frac{1}{x}} }{ x } 
\end{aligned} \\
\begin{aligned}
  \sin{\frac{1}{x} } & = \sum_{j=0}^{\infty} \frac{ \left( \frac{1}{x} \right)^{2j+1} (-1)^j }{ (2j +1)! } \\
  \frac{ -x \sin{\frac{1}{x} } + \cos{\frac{1}{x}} }{ x } & = \\
  & = \frac{ -1 + \frac{ \left( \frac{1}{x} \right)^3 (+1) }{ 3! } + \sum_{j=2}^{\infty} \frac{ \left( \frac{1}{x} \right)^{2j+1} (-1)^j }{ (2j+1)! } + 1 - \left( \frac{1}{x} \right)^2/2 + \sum_{j=2}^{\infty} \frac{ \left( \frac{1}{x} \right)^{2j} (-1)^j }{ (2j)! } }{x } < 0 \, \text{ for $x$ large enough }
\end{aligned}
\end{gathered}
\]
$\sum_{j=1}^{\infty} (-1)^j \left( 1 - j \sin{\frac{1}{j} } \right)$ converges since $a_j = 1 - j \sin{\frac{1}{j}}$ is monotonically decreasing sequence with limit $0$.

\[
\begin{gathered}
  1 - j \sin{\frac{1}{j} } = 1 - j \sum_{k=0}^{\infty} \frac{ \left( \frac{1}{j} \right)^{2k+1} (-1)^k }{ (2k+1)! } = 1 - j \left( \frac{1}{j} + \sum_{k=1}^{\infty} \frac{ \left( \frac{1}{j} \right)^{2k+1} (-1)^k }{ (2k+1)! } \right) = \\
  = 1 - \left(1+ \sum_{k=1}^{\infty} \frac{ \left( \frac{1}{j} \right)^{2k} (-1)^k }{ (2k+1)! } \right) = \sum_{k=1}^{\infty} \frac{ \left(\frac{1}{j} \right)^{2k} (-1)^{k+1} }{ (2k+1)! } \leq \frac{1}{6j^2 } 
\end{gathered}
\]
The series converges absolutely since the term itself is a series that is dominated by $\frac{1}{6j^2}$, so that by comparison test, the series must converge.  

\exercisehead{18} $\sum_{j=1}^{\infty} (-1)^j \left( 1 - \cos{ \frac{1}{j} } \right)$.  
\[
  (\cos{ \frac{1}{x} })' = \left( -\sin{\frac{1}{x} } \left( \frac{-1}{x^2} \right) \right) = \frac{-1}{x^2} \sin{ \frac{1}{x} } < 0 
\]
$\sum_{j=1}^{\infty} (-1)^j (1- \cos{\frac{1}{j}} ) $ converges since $a_j = (1- \cos{\frac{1}{j} }) $ is monotonically decreasing to $0$

\[
(1- \cos{\frac{1}{j} }) = 1 - \sum_{k=0}^{\infty} \frac{ (1/j)^{2k} (-1)^k }{ (2k)! } = \sum_{k=1}^{\infty} \frac{ (1/j)^{2k} (-1)^{k+1} }{ (2k)! } \leq \frac{1}{2j^2 } 
\]
So the series converges absolutely, by comparison test with $\sum \frac{1}{j^2 }$ which converges.  

\exercisehead{19} $\sum_{j=1}^{\infty} (-1)^j \arctan{ \frac{1}{2j+1} }$.  
\[
(\arctan{ \left( \frac{1}{2j+1} \right) })' = \frac{1}{ 1 + \left( \frac{1}{2j+1} \right)^2 } \left( \frac{-1}{ (2j+1)^2 } \right) (2) = \frac{ -2 }{ (2j+1)^2 + 1 } < 0  
\]
$\sum_{j=1}^{\infty} (-1)^j \arctan{ \frac{1}{2j+1} }$ converges, since $a_j = \arctan{ \frac{1}{2j+1} }$ is monotonically decreasing to $0$

\[
\begin{gathered}
  \begin{aligned}
    \frac{1}{ 1+x^2} & = (\arctan{x})' = \sum_{j=0}^{\infty} (-x^2)^j = \sum_{j=0}^{\infty} (-1)^j x^{2j} \\
    & \Longrightarrow \arctan{x} = \sum_{j=0}^{\infty} (-1)^j \frac{ x^{2j+1}}{ 2j+1} 
  \end{aligned} \\
\begin{aligned}
  \arctan{ \frac{1}{ 2j+1} } &= \sum_{k=0}^{\infty} \frac{ (-1)^k \left( \frac{1}{2j+1} \right)^{2k+1} }{ (2k+1) } =  \frac{1}{ 2j+1} + (-1)\left( \frac{ \frac{1}{(2j+1)^3} }{ 3  } \right) + \sum_{k=2}^{\infty} \frac{ (-1)^k \left( \frac{1}{ 2j+1} \right)^{2k+1} }{ 2k+1 } > \\
  & > \frac{1}{2j + 1 } + (-1)\frac{1}{ 3 (2j+1)^3 } = \frac{ 3 (4j^2 + 4j + 1 ) + (-1) }{ 3 (2j+1)^3 } = \frac{ 12 j^2 + 12 j + 2 }{ 3(2j+1)^3 } > \frac{ 2j }{ (2j+1)^2 } > \frac{2}{9} \frac{1}{ j} \, \text{ for $j>2$ }
\end{aligned}
\end{gathered}
\]
So by comparison test to $\sum \frac{1}{j}$, $\sum \arctan{\frac{1}{2j+1} }$ diverges absolutely.  The series is conditionally convergent.  

\exercisehead{20} $\sum_{j=1}^{\infty} (-1)^j \left( \frac{\pi}{2} - \arctan{ \log{j} } \right)$  \\
\[
\left( \frac{\pi}{2} - \arctan{ \log{n} } \right)' = \left( \frac{-1}{ 1 + (\log{n})^2 } \right)\left( \frac{1}{n} \right)
\]
\[
\begin{gathered}
  \frac{\pi}{2} - \arctan{ \log{n}} \geq \frac{\pi}{2} - \arctan{ (n-1)} \to \frac{\pi}{2} - \arctan{n} \quad \text{ just change indices} \\
\begin{aligned}
  \int_0^n \left( \frac{\pi}{2} - \arctan{x} \right) dx & = \left. \frac{\pi}{2} x - \left( x \arctan{x} - \frac{1}{2} \ln{ (1+x^2 ) } \right) \right|_0^n \\
  & = \frac{\pi}{2} n - \left( n \arctan{n} - \frac{1}{2} \left( \ln{ \left( 1 + n^2 \right) } \right) \right) = n \left( \frac{\pi}{2} - \arctan{n} \right) + \frac{1}{2} \ln{ (1+n^2) } 
\end{aligned} \\
\lim_{n\to \infty} \left( n \left( \frac{\pi}{2} - \arctan{n} \right) + \frac{1}{2} \ln{ (1 + n^2 ) } \right) \to \infty 
\end{gathered}
\]
Then by the integral test, $\sum \frac{\pi}{2} - \arctan{ \log{n} }$ diverges absolutely.  So the alternating series is conditionally convergent.  

\exercisehead{21} $\sum_{j=1}^{\infty} \log{ \left( 1 + \frac{1}{ |\sin{j} | } \right)}$  \\
$\lim_{j\to \infty} \log{ \left( 1 + \frac{1}{ |\sin{j} | }  \right)}$ doesn't exist and $\log{ \left( 1 + \frac{1}{ |\sin{j} | }  \right) }> 0 \quad \forall j$ so the series diverges.  

\exercisehead{22} $\sum_{j=2}^{\infty} \sin{ \left( j\pi + \frac{1}{ \log{j} }\right) }$
\[
\begin{gathered}
  \sin{ \left( 2j \pi + \frac{1}{ \log{2j} } \right) } + \sin{ \left( (2j+1) \pi + \frac{1}{ \log{ (2j+1) } } \right) } = \\
  = \sin{ (2j\pi )}\cos{ \frac{1}{ \log{2j} } } + \sin{ \left( \frac{1}{ \log{2j } } \right) } \cos{ (2\pi j)} + \sin{ (2j+1) } \pi \cos{ \frac{1}{ \log{(2j+1) } } } + \cos{ (2j+1) \pi} \sin{ \left( \frac{1}{ \log{2j+1} } \right) } = \\
    = \sin{ \frac{1}{\log{2j} }} - \sin{ \frac{1}{ \log{ 2j+1} } } = \\
      = \sum_{k=0}^{\infty} \frac{ \left( \frac{1}{ \log{2j} } \right)^{2k+1} (-1)^k }{ (2k+1)!}  -\sum_{k=0}^{\infty} \frac{ \left( \frac{1}{ \log{ (2j+1) } } \right)^{2k+1} (-1)^k }{ (2k+1)!} 
\end{gathered} 
\]
\[
\begin{gathered}
\begin{aligned}
  \sin{ \left( \frac{1}{ \log{2j} } \right) } - \sin{ \left( \frac{1}{ \log{2j+1} } \right) } & = \\
  & = \sum_{k=0}^{\infty} \frac{(-1)^k}{ (2k+1)! } \left( \left( \frac{1}{ \log{2j} } \right)^{2k+1} - \left( \frac{1}{ 2j+1} \right)^{2k+1} \right)
\end{aligned} \\
0 < \log{ 2j } < \log{2j+1} \text{ so } \left( \frac{1}{ \log{2j} } \right)^{2k+1} - \left( \frac{1}{ \log{2j+1} } \right)^{2k+1} > 0 
\end{gathered}
\]
\[
\begin{gathered}
\text{ and since for $j>1$, $\left( \frac{1}{\log{2j}} \right)$ and $\left( \frac{1}{ \log{2j+1}} \right)$ are $<1$ and so we are adding smaller and smaller amounts } \\
\begin{aligned}
   & <  \frac{1}{ \log{2j} } - \frac{\log{2j+1} } = \\
  & = \frac{ \log{(2j+1) } - \log{2j} }{ \log{2j} \log{2j+1} } \leq \frac{ \log{ \left( 1 + \frac{1}{2j } \right) } }{ (\log{(2j) })^2  } \leq \frac{  \frac{1}{2j} }{ ( \log{(2j) } )^2 } 
\end{aligned} \\
  \int_1^n \frac{1}{ 2j (\log{2j})^2 } =  \left. -\frac{1}{ \log{2j} } \right|_1^n \xrightarrow{ n\to \infty} \frac{1}{\log{2}}
\end{gathered}
\]
So the series converges by using integral test, showing that $\sum \frac{1}{ 2j (\log{2j})^2} $ converges, so by comparison test, the series converges.  

\exercisehead{33} $  \sum_{n=1} n^n z^n $
\[
\begin{gathered}
\left|   \sum_{n=1} n^n z^n  \right| = \left| \sum_{j=1}^{\infty} (jz)^j \right| = \sum \left( e^{ \ln{j} } z \right)^j \\
\sum_{j=1}^n (e^{\ln{j}} z )^j = \\
= \frac{ e^{\ln{j}} z - ( e^{\ln{j}} z )^n }{ 1 - e^{\ln{j}} z } \longrightarrow \infty 
\end{gathered}
\]
So $z=0$

\exercisehead{34} $\sum_{j=1}^{\infty} \frac{ (-1)^j z^{3j}}{ j!} = \sum_{j=1}^{\infty} \frac{ (-z^3)^j }{ j! } = e^{-z^3} $.  $\boxed{ \mathbb{C} }$.  

\exercisehead{35} $\sum_{j=0}^{\infty} \frac{z^j}{3^j} = \sum_{j=0}^{\infty} \left( \frac{1}{3} \right)^j z^j$  \\
$\sum z^j$ be convergent or $\sum_{j=1}^n z^j$ bounded.  \\
$\boxed{ |z| < 3 }$ and $|z| = 3$ if $z\neq 3$

\exercisehead{36} $\sum_{j=1}^{\infty} \frac{ z^j}{j^j}$  $ \boxed{ \{ z \} = \mathbb{C} }$ since \\
\[
\left| \left( \frac{z}{j} \right) \right| < 1 \text{ for } j \geq N > |z|
\]

\exercisehead{37} $\sum_{j=1}^{\infty} \frac{(-1)^j }{ z+j}$ \\
By Leibniz's Rule, since $\frac{1}{ z + j } \xrightarrow{ j \to \infty } 0 $, then the series converges.  However, $z$ cannot be equal to any negative integer since one term in the series will then blow up.  

\exercisehead{38} $\sum_{j=1}^{\infty} \frac{z^j}{ \sqrt{j}} \log{ \left( \frac{2j+1}{j} \right) }$.  
\[
\frac{ z^j}{ \sqrt{j}} \log{ \left( 2 + \frac{1}{j} \right)} = z^j \frac{ \log{ \left( 2+\frac{1}{j} \right) } }{ \sqrt{j}}
\]
Since $\lim_{j\to \infty} \frac{ \log{ \left( 2 + \frac{1}{j} \right) } }{ \sqrt{j}} = 0$ so that $\frac{ \log{ \left( 2 + \frac{1}{j} \right) } }{ \sqrt{j}} $ is a monotonically convergent sequence.  

Then by Dirchlet's test, $\sum_{j=1}^n z^j$ must be bounded.  $|z| > 1$, and $|z| = 1$ if $z \neq 1$. 

\exercisehead{39} $\sum_{j=1}^{\infty} \left( 1 + \frac{1}{ 5j+1} \right)^{j^2} |z|^{17j} = \sum_{j=1}^{\infty} \left( 1 + \frac{1}{ 5j +1 } \right)^{j^2} (|z|^{17})^j $
\[
\begin{gathered}
  \left( \left( 1 + \frac{1}{ 5j + 1 } \right)^j |z|^{17} \right)^j \\
  \lim_{j\to \infty} \left( 1 + \frac{1}{ 5j + 1 } \right)^j \leq \lim_{j\to \infty} \left( 1 + \frac{ 1/5}{j} \right)^j = e^{1/5} \\
  e^{1/5} |z|^{17} < 1 \Longrightarrow \boxed{ |z| < e^{ -1/85} }
\end{gathered}
\]

\exercisehead{40} $\sum_{j=0}^{\infty} \frac{ (z-1)^j }{ (j+2)! }$
\[
\left| \sum_{j=0}^{\infty} \frac{ (z-1)^j}{ (j+2)! } \right| \leq \sum_{j=0}^{\infty} \frac{ |(z-1)|^j}{ (j+2)! } \leq \sum_{j=0}^{\infty} \frac{ |(z-1)|^{j+2} }{ (j+2)! } = \boxed{ e^{|z-1|} - 1 - \frac{ |z-1|}{1!} }
\]
The series converges $\forall \, z$.  


\exercisehead{41} 
\[
\sum_{j=1}^{\infty} \frac{ (-1)^j (z-1)^j }{ j } = \sum_{j=1}^{\infty} \frac{ (1-z)^j}{ j } = \log{ (1- (1-z) ) } = \boxed{ \log{z} }
\]
So the series converges $\forall \, z$ except for $z=0$.

\exercisehead{42} $\sum_{j=1}^{\infty} \frac{ (2z+3)^j }{ j \log{(j+1) }}$ 
\[
\begin{gathered}
\lim_{j\to \infty} \frac{1}{ j \log{ (j+1) }} = 0 \text{ so } \frac{ 1 }{ j \log{ (j+1) }} \text{ is a monotonically convergent sequence } \\
\sum (2z +3)^j \quad \quad \begin{aligned}
 |2z + 3 | & < 1 \\
 |z + \frac{3}{2} | & < \frac{1}{2} 
\end{aligned} \quad \quad \text{ then } \sum (2z+3)^j \text{ converges }  \\
\left( \frac{1}{ x \log{ (x+1) }}\right)' = \frac{-1}{ (x\log{ (x+1) } )^2 } \left( \log{ (x+1)} + \frac{x}{x+1} \right) < 0; \quad \text{ for } x > 0 
\end{gathered} 
\]
By Dirichlet's test, $\sum_{j=1}^{\infty} \frac{ (2z+ 3)^j }{ j \log{ (j+1) } } $ converges for $\boxed{ |z+\frac{3}{2} | \leq \frac{1}{2}; \quad z \neq -1 }$.  

\exercisehead{43} $\sum_{j=1}^{\infty} \frac{ (-1)^j }{ (2j-1) } \left( \frac{ 1-z}{ 1+z} \right)^j = \sum_{j=1}^{\infty} \frac{ (-1)^j/2 }{ j - \frac{1}{2} } \left( \frac{ 1-z}{ 1+z} \right)^j $ \\
$\Longrightarrow \sum_{j=1}^{\infty} \frac{ \left( \frac{z-1}{z+1} \right)^j }{ 2j - 1 } = \lim_{j\to \infty} \frac{1}{ 2j-1} = 0 $ and $\frac{1}{2j-1}$ is monotonically decreasing.  \\
So $\frac{1}{2j-1} $ is a monotonically decreasing convergent sequence of real terms.  

For $\left| \frac{ z-1}{ z+1} \right| < 1$, 
\[
\begin{gathered}
  \left| \frac{ z-1}{z+1} \right| < 1 \Longrightarrow |z-1| < |z+1| \\
  \begin{aligned} 
  (u-1)^2 + v^2 & < (u+1)^2 + v^2  \\
    u^2 - 2u + 1 & < u^2 + 2u + 1 \\
    -u & < u 
\end{aligned} \quad \quad \boxed{ \Re{(z)} > 0; \quad z \neq 0 } \\
\end{gathered}
\]

\exercisehead{44} 
\[
\begin{gathered}
  \sum_{j=1}^{\infty} \left( \frac{z}{2z+1} \right)^j = \sum_{j=1}^{\infty} \left( \frac{1}{2} + \frac{ -1/2}{ 2z+1} \right)^j = \sum_{j=1}^{\infty} \left( \frac{1}{2} \right)^j \left( 1 - \frac{1}{ 2z + 1 } \right)^j \\
  \left( \frac{1}{2} \right)^j \text{ is a monotonically decreasing, convergent sequence } \\
  \text{ Now we want } \left| 1 - \frac{1}{ 2z + 1 } \right| = \left| \frac{ 2 z + 1 - 1 }{ 2 z + } \right| = \left| \frac{2z}{ 2z+ 1 } \right| < 1 \\
  \begin{aligned}
    |2z| & < |2z+ 1 | \\
    |z| & < |z + 1/2 | \\
    u^2 +v^2 & < (u+1/2)^2 + v^2 = u^2 + u + 1/4 + v^2 \\
    \frac{-1}{4} & < u \Longrightarrow \boxed{ \Re{(z)} > -\frac{1}{4} }
\end{aligned}  \quad \quad 
\begin{gathered}
  \text{ if } \Re{(z)} = \frac{-1}{4} \\
  \frac{ 2 \left( - \frac{1}{4} + iv \right) }{ 2 \left( \frac{-1}{4} + iv \right) + 1 } = \frac{ \frac{-1}{2} + 2iv }{ \frac{1}{2} + 2iv } \\
\Longrightarrow \boxed{ z \neq \frac{-1}{4} } 
\end{gathered}
\end{gathered}
\]

\exercisehead{45} $\sum_{j=1}^{\infty} \frac{j}{j+1} \left( \frac{z}{2z+1} \right)^j$.  
\[
\begin{gathered}
  \begin{aligned}
    \left( \frac{x}{x+1} \right)' & = \frac{ (x+1) - x }{ (x+1)^2} = \frac{1}{ (x+1)^2 } > 0  \\
    \frac{j}{j+1} & \text{ is a monotonically increasing and convergent sequence }
\end{aligned} \\
\frac{1}{2} + \frac{-1}{ 2 (2z+1) } = \left| \frac{ z}{ 2z+1 } \right| < 1 \\
\left| \frac{ 2z+1}{ z} \right| > 1 \Longrightarrow \boxed{ |2+ \frac{1}{z} | > 1 }
\end{gathered}
\]

\exercisehead{46} $\sum_{j=1}^{\infty} \frac{1}{ (1+ |z|^2)^j } = \sum_{j=1}^{\infty} \left( \frac{1}{ 1+ |z|^2} \right)^j $ 
\[
\begin{aligned}
  \frac{1}{ 1+ |z|^2} & < 1 \\
  1 & < 1  + |z|^2 \\ 
  0 & < |z|^2 
\end{aligned} \quad \quad \forall z \, \text{ except $z=0$ }
\]

\exercisehead{47} $\sum_{j=1}^{\infty} (-1)^j \frac{ 2^j \sin^{2j}{x} }{ j} $
%\[
%\begin{gathered}
%\begin{aligned}
%  \sin^{2j}{x} & = \left( \frac{ e^{ix} - e^{-ix}}{ 2 } \right)^{2j} = \frac{1}{2^{2j}} \sum_{k=0}^{2j} \binom{2j}{k} (e^{ix})^k (e^{-ix})^{2j-k} \\
%  & = \frac{1}{ 2^{2j} } \sum_{k=0}^{2j} \binom{ 2j}{k} e^{ikx} e^{0ix2j + ixk} (-1)^{2j-k} = \frac{1}{ 2^{2j}} \sum_{k=0}^{2j} \binom{2j}{k} e^{ i2kx - 2ixj}(-1)^{2j-k} 
%\end{aligned} \\
%\begin{aligned}
%  \sum_{j=1}^{\infty} (-1)^j \frac{ 2^j \sin^{2j}{x} }{ j} & = \sum_{j=1}^{\infty} \frac{ (-2)^j}{j} \frac{1}{ 2^{2j}} \sum_{k=0}^{2j} \binom{2j}{k} e^{ix (2k-2j)} (-1)^{2j-k} =  \\
%  & = \sum_{j=1}^{\infty} \sum_{k=0}^{2j} \frac{ \left( \frac{-1}{2} \right)^j }{ j } \binom{2j}{k} (-1)^{2j - k} e^{ix (2k-2k) } \\
%\boxed{ \forall x \in \mathbb{R} }
%\end{gathered}
%\]

Use Dirichlet's Test.  $\frac{1}{j}$ is a monotonically decreasing sequence converging to zero.  Consider $(-2)^j \sin^{2j}{x}$.  The condition for convergence is
\[
\begin{gathered}
  |(-2 \sin^2{x})|  < 1 \quad \text{ for $j \geq N$ for some $N$ } \\
  \Longrightarrow x \in ( \frac{ -\pi}{4} + n \pi, \frac{ \pi}{4} + n \pi ), \quad n \in \mathbb{Z} 
\end{gathered}
\]


\exercisehead{49} $\sum a_j$ converges.  \\
$\sum a_j \frac{1}{ a_j}$ diverges.  \\
\quad \quad Then since $\sum a_j$ is a convergent series (by Abel's test), $\frac{1}{a_j}$ is a divergent sequence.   \\
\quad Then $\sum \frac{1}{a_j}$ is divergent (since $\lim_{j\to \infty} \frac{1}{a_j}$ doesn't exist ).  

\exercisehead{50} $\sum |a_j|$ converges. \\
$\sum |a_j|$ converges, then $\sum a_j$ converges.  \\
$|a_j|^2 = a_j^2$.  $|a_j|$ converges, then 
\[
\begin{aligned}
  \lim_{j\to \infty} |a_j| & = 0 \\
  \lim_{j\to \infty} |a_j|^2 & = 0 
\end{aligned} \quad \quad 
  \lim_{j\to \infty} \frac{ |a_j|^2}{a_j^2}  = 1 \\
\]
By limit comparison test, $\sum a_j^2$ converges.  

Counterexample:  $\sum \left( \frac{1}{j} \right)^2 $ converges, but $\sum \frac{1}{j}$ diverges.  

\exercisehead{51} Given $\sum a_j, \quad a_j \geq 0$.  $\sum a_j$ converges.  
\[
\begin{gathered}
  \sum \sqrt{ a_j} \frac{1}{ (j)^p } \quad \quad \begin{aligned} 
    \lim_{j\to \infty} a_j & = 0 \\
    \lim_{j \to \infty} \sqrt{ a_j } & = \left( \lim_{j \to \infty} a_j \right)^{1/2} = 0 
  \end{aligned} \\
    \int \sum_{j=0}^{n-1} x^j  = \sum_{j=0}^{n-1} \frac{x^{j+1}}{ j+1} = \sum_{j=1}^n \frac{x^j}{j} = \int \frac{1-x^n}{ 1-x}  \\
\end{gathered}
\]

A counterexample would be $\frac{ \sqrt{a_j}}{ j^p } = \sqrt{ \frac{ a_j}{j}}$. 

\exercisehead{52}
\begin{enumerate}
\item 
$\sum a_j$ converges absolutely, then if $\sum |a_j|$ converges, $\sum a_j^2$ converges.  
\[
\begin{gathered}
  \frac{a_j^2}{ 1+ a_j^2} = 1 + \frac{-1}{1+a_j^2 } \\
  \frac{ a_j^2}{1 + a_j^2} \leq a_j^2 \text{ since } \sum a_j^2 \text{ converges }, \sum \frac{a_j^2}{ 1+ a_j^2} \text{ converges }
\end{gathered}
\]
\item $\sum a_j$ converges absolutely, $\lim_{j\to \infty} |a_j| = 0$ \\
$\sum \left| \frac{a_j}{1+a_j} \right| = \sum |a_j| \left( \frac{1}{ |1+a_j | } \right)$.  \\
By Abel's test, since
\[
\lim_{j\to \infty} \frac{ 1}{ |1+ a_j| } = \frac{ 1 }{ |1 + \lim_{j\to \infty} a_j | } = 1 \quad \text{ shows that } \frac{1}{ |1+ a_j | } \geq 0 \text{ is a monotonically convergent sequence }
\]
By Abel's test, $\sum \frac{a_j}{1 + a_j}$ is convergent.  
\end{enumerate}

%-----------------------------------%-----------------------------------%-----------------------------------
\subsection*{ 10.22 Miscellaneous review exercises - Rearrangements of series }
%-----------------------------------%-----------------------------------%-----------------------------------

\exercisehead{1} 
\begin{enumerate}
\item \[
\begin{aligned}
  a_j & = \sqrt{ j+ 1 } - \sqrt{j} = \sqrt{j} \sqrt{ 1 + \frac{1}{j} } - \sqrt{j} = \sqrt{j} \left( 1 + \frac{1}{2} \left( \frac{1}{j} \right) + o\left( \frac{1}{j} \right) + - 1 \right) = \\
  & = \sqrt{j} \left( \frac{1}{2} \left( \frac{1}{j} \right) + o\left( \frac{1}{j} \right) \right) \\
  \lim_{j\to \infty} a_j & = 0 
\end{aligned}
\]
\item \[
\begin{gathered}
  \begin{aligned}
    a_j & = (j+1)^c - j^c = j^c \left( \left( 1 + \frac{1}{j} \right)^c - 1 \right) = (j^c) \left( 1 + c \left( \frac{1}{j} \right) + o\left( \frac{1}{j} \right) - 1 \right) = \\
    & = j^c \left( c\left( \frac{1}{j} \right) + o\left( \frac{1}{j} \right) \right) = \left( c j^{c-1} + j^c o\left( \frac{1}{j} \right) \right)
  \end{aligned} \\
\boxed{   \begin{aligned}
    \text{ if } c > 1, \quad & a_j \text{ diverges } \\
    \text{ if } c = 1, \quad & \lim_{j\to \infty} a_j = 1 \\
    \text{ if } c < 1, \quad & \lim_{j \to \infty} a_j  = 0 
\end{aligned} }
\end{gathered}
\]
\end{enumerate}

\exercisehead{2} 
\begin{enumerate}
\item
\[
(1+x^n)^{\frac{1}{n}} = \exp{ \left( \frac{1}{n} \ln{ (1+x^n ) } \right) } = \exp{ \left( \frac{1}{n} \sum_{j=1}^{\infty} \frac{ (x^n)^j (-1)^{j-1}}{ j } \right) }\xrightarrow{n \to \infty} \boxed{1}
\]
\item $\lim_{n\to \infty} (a^n + b^n)^{1/n} = \lim_{n\to \infty} a\left( 1 + \left( \frac{b}{n} \right)^n \right)^{1/n} = a$ if $a>b$.  
\end{enumerate}

\exercisehead{3} $a_{n+1} = \frac{ a_n + a_{n-1}}{2} = \frac{ a_{n-1} + a_{n-2}}{ 2^2} + \frac{ a_{n-2} + a_{n-3}}{2^2}$  



%-----------------------------------%-----------------------------------%-----------------------------------
\subsection*{ 10.24 Exercises - Improper integrals }
%-----------------------------------%-----------------------------------%-----------------------------------
\quad \\
\exercisehead{1} $\int_0^{\infty} \frac{x}{ \sqrt{ x^4 + 1 } } dx$
\[
\lim_{ x\to \infty} \left( \frac{x}{ \sqrt{ x^4+1 }} \right)\left( \frac{1}{1/x} \right) = \lim_{x\to \infty} \frac{x^2}{ x^2 \sqrt{ 1 + 1/x^4 } } = 1 
\]
Since $\int_1^{\infty} \frac{1}{x}$ diverges, so does $\int_0^{\infty} \frac{x}{ \sqrt{ x^4 + 1 } } dx$

\exercisehead{2} 
\[
\begin{gathered}
  \int_{-\infty}^{\infty} e^{-x^2} dx  = \int_0^{\infty} e^{-x^2} dx + \int_0^{-\infty} e^{-x^2} dx = \int_0^{\infty} e^{-x^2} dx + - \int_{\infty}^0 e^{-x^2} dx = 2 \int_0^{\infty} e^{-x^2} dx \\
  \int_0^{\infty} e^{-x^2} dx \leq \int_0^{\infty} e^{-x} dx = \left. -e^{-x^2} \right|_0^{\infty} = \boxed{ 1 } 
\end{gathered}
\]
Converges by theorem.  

\exercisehead{3} $\int_0^{\infty} \frac{1}{ \sqrt{ x^3 + 1 } } dx$

\exercisehead{4} $\int_0^{\infty} \frac{1}{ \sqrt{ e^x}} dx $

\exercisehead{5} $\int_{0^+}^{\infty} \frac{ e^{-\sqrt{x}}}{\sqrt{x}} dx $

\exercisehead{6} $\int_{0^+}^1 \frac{ \log{x}}{ \sqrt{x}} dx$ 

\exercisehead{7} $\int_{0^+}^{1-} \frac{\log{x}}{ 1-  x} dx$

\exercisehead{8} $\int_{-\infty}^{\infty} \frac{x}{ \cosh{x}} dx$

\exercisehead{9} $\int_{0^+}^{1-} \frac{dx}{ \sqrt{x} \log{x}}$

\exercisehead{10} $\int^{\infty} \frac{dx}{x (\log{x})^s}$

%-----------------------------------%-----------------------------------%-----------------------------------
\subsection*{ 11.7 Exercises - Pointwise convergence of sequences of functions, Uniform convergence of sequences of functions, Uniform convergence of sequences of functions, Uniform convergence and continuity, Uniform convergence and integration, A sufficient condition for uniform convergence, Power series.  Circle of convergence }
%-----------------------------------%-----------------------------------%-----------------------------------

\exercisehead{1} $\sum_{j=0}^{\infty} \frac{z^j}{2^j} = \sum_{j=0}^{\infty} \left(\frac{z}{2} \right)^j$ \bigskip \\
Using the \emph{comparison test}, \\
\quad \quad $\left| \frac{z}{2} \right|^j \leq t^j$;\quad \quad $\left| \frac{z}{2} \right| <1 \Longrightarrow |z| < 2$ \\

Suppose $|z| = 2$, $z \neq 2$  $\sum_{j=0}^{\infty} \left( \frac{z}{2} \right)^j = \sum_{j=0}^{\infty} ( e^{2i\theta})^j $ \medskip \\
Now $\sum_{j=0}^n (e^{2i\theta})^j \leq \frac{1}{\sin{\theta}} + 1 $, since
\[
\begin{gathered}
  \sum_{j=1}^n (e^{2i\theta})^j = \frac{ e^{2i\theta} - e^{2i\theta(n+1)} }{ 1 - e^{2i\theta} } = \frac{ e^{in\theta} - e^{-i\theta n} e^{i \theta n + i \theta } }{ -e^{-i\theta} + e^{i \theta} } = \frac{ \sin{n\theta} e^{i\theta n + i \theta} }{ \sin{\theta}} < \frac{1}{ \sin{\theta} } 
\end{gathered}
\]
So $\sum_{j=0}^{\infty} \left( \frac{z}{2} \right)^j $ converges for $|z| = 24$, $z\neq 2$


\exercisehead{2} $\sum_{j=0}^{\infty} \frac{z^j}{(j+1)2^j }$ \medskip \\

Use \emph{ ratio test }.  
\[
\frac{a_{j+1}}{a_j} = \frac{z^{j+1}}{ (j+2)2^{j+1} } \frac{ (j+1)2^j }{ z^j } = \frac{z}{2} \frac{ (j+1)}{(j+2)} = \frac{z}{2} \frac{ (1+1/j)}{(1+2/j)} \xrightarrow{ j \to \infty} \frac{z}{2}
\]
If $|z| <2$, $\sum_{j=0}^{\infty} a_j $ converges, if $|z| > 2$, $\sum a_j$ diverges.  

If $|z| = 2$, $\sum \frac{z^j}{(j+1)2^j}  = \sum (e^{2i \theta})^j \left( \frac{1}{j+1} \right)$ \medskip \\
Now $\frac{1}{j+1}$ is a monotonically decreasing sequence of real terms.  \\
\phantom{No} $\sum (e^{2i\theta})^j $ is a bounded series.  \medskip \\
By Dirichlet's test, $\sum a_j$ converges if $|z| = 2$, \, $z\neq 2$


\exercisehead{3} $\sum_{j=0}^{\infty} \frac{(z+3)^j }{ (j+1)2^j }$ \\
Use ratio test: 
\[
\frac{ a_{j+1}}{a_j} = \left( \frac{ (z+3)^{j+1}}{ (j+2)2^{j+1} } \right) \frac{ (j+1) 2^j }{ (z+3)^j } = \frac{ (z+3)}{2} \left( \frac{j+1}{j+2} \right) = \frac{(z+3)}{2} \left( \frac{1 + 1/j}{ 1+2/j} \right) \xrightarrow{j\to \infty} \frac{z+3}{2}
\]
Using Theorem 11.7, 
\begin{theorem}[Existence of a circle of convergence]
\[
\begin{gathered}
\text{ Assume $\sum a_j z^j$ } \,  \begin{aligned} & \text{ converges for at least $z_1 \neq 0$ } \\
  & \text{ diverges for at least one $z_2 \neq 0 $ } \end{aligned} \\
\exists \, r > 0, \quad \text{ such that } \sum a_j z^j \begin{aligned} & \text{ absolutely converges for $|z| < r$ } \\
  & \text{ diverges for $|z| > r $ } 
\end{aligned}
\end{gathered} 
\]
\end{theorem}

We can plug in real numbers to satisfy the condition $\frac{ |z+3|}{2} < 1$ for convergence.  \\
$\sum a_j$ converges for $|z+3| < 2$; \quad diverges for $|z+3| > 2$.  

Consider $|z+3| = 2$; \quad $z \neq 1$
$\sum a_j = \sum (e^{2i \theta})^j \left( \frac{1}{j+1} \right)$.  Since $\frac{1}{j+1}$ is a monotonically decreasing sequence of real numbers and $\sum (e^{2i \theta})^j$ is a bounded series, by Dirichlet's test, \\
$\sum a_j$ converges for $|z+3| = 2; \quad z\neq -1$.  

\exercisehead{4} $\sum_{j=1}^{\infty} \frac{ (-1)^j 2^{2j} z^{2j}}{ 2j } = - \sum_{j=1}^{\infty} \frac{ (-1)^{j-1} (2z)^{2j}}{ (2j) }$.  Look at what the terms look like.  

Consider using Leibniz's Rule, Theorem 10.14.  
\begin{theorem}[Leibniz's rule]
  If $a_j$ is a monotonic decreasing sequence and $\lim_{j\to \infty} a_j=0$, \\
  \phantom{ If } then $\sum_{j=1}^{\infty} (-1)^{j-1} a_j $ converges.  
\end{theorem}
\[
\begin{gathered}
  \frac{ (-1)^j 2^{2j} z^{2j} }{ 2j} = \frac{ (-1)^j (2z)^{2j} }{ 2j } \\
  \text{ Consider } \left| \frac{ (2z)^{2j} }{ 2j } \right| = \frac{ |2z|^{2j}}{ 2j } = \frac{ (2|z|)^{2j} }{2j } \\
  \text{ Consider } 2|z| = M^{1/2} < \infty \\
  \Longrightarrow \frac{ (2|z|)^{2j}}{ 2j} = \frac{ M^j }{ 2j } = \frac{ e^{j \ln{ M}}}{ 2j } \\
  \text{ converges for } 0 < 2 |z| = M \leq 1 \Longrightarrow |z| \leq \frac{1}{2} 
\end{gathered}
\]
(we use Theorem 11.6 at this point, because real numbers are included in complex numbers).  

\begin{theorem}
  Assume $\sum a_j z^j$ converges for some $z=z_1 \neq 0 $.  \\
  \phantom{ Then } 
\begin{enumerate}
\item $\sum a_j z^j$ converges absolutely $\forall z $ with $|z| < |z_1|$.  
\item $\sum a_j z^j$ converges uniformly on every circular disk with center at $0$ and $R < |z_1|$
\end{enumerate}
\end{theorem}

We had \emph{first used Leibniz's test} to find $az_1$ on the real line.  

\[
\frac{ 2^{4j} z^{4j} (4j (1 - \frac{1}{ 4z^2 } ) - 2 ) }{ 4 j (4j-2) } = \frac{ 2^{4j} z^{4j} ((1- \frac{1}{4z^2 } ) - \frac{2}{4j } ) }{ (4j-2) }  \xrightarrow{ j \to \infty} \frac{ 2^{4j} z^{4j} (1 - \frac{1}{4z^2 } ) }{ (4j-2) }
\]
If $|z| > \frac{1}{2}$, the series diverges (by $a_j$th general term test).  

\exercisehead{5} $\sum_{j=1}^{\infty} (1 - (-2)^j )z^j $.  \\
\textbf{ Try limit comparison test }.  \\
\textbf{ The first step } is to test \emph{ absolute convergence } first; it's easier.  
\[
\frac{ \left| (1- (-2)^j ) z^j \right| }{ \left| (-2z)^j \right| } = \left| \left( \frac{1}{-2} \right)^j -1 \right| \xrightarrow{ j\to \infty} 1 
\]
According to limit comparison test, for $\sum (1-(-2)^j)z^j $ to converge, $\sum (-2z)^j$ must converge.  \\
\quad So if $|z| < \frac{1}{2}$, then $\sum_{j=1}^{\infty} (1-(-2)^j)z^j$ absolutely converges.  

If $|z| = \frac{1}{2}, \quad z \neq \frac{-1}{2}$, 
\[
\begin{aligned}
  \sum (1- (-2)^j) z^j & = \sum z^j - \sum (-e^{2i\theta} )^j = \\
  & = 0 - \sum (e^{2i\theta + \pi i })^j < \frac{1}{ \sin{ (\theta + \pi )} }
\end{aligned}
\]
If $z = \frac{-1}{2}$,
\[
\sum \left( - \frac{1}{2} \right)^j - \sum 1^j \to \infty
\]

\exercisehead{6} $\sum_{j=1}^{\infty} \frac{ j! z^j }{ j^j }$  \\
A \textbf{ very big hint } is to use Exercise 19 on pp. 399, in the section for Exercises 10.14.  \\
\text{ Since } $\sum_{j=1}^{n-1} f(j) \leq \int_1^n f(x) dx $
\[
\begin{gathered}
  \begin{aligned}
    \sum_{j=1}^{n-1} \ln{j} & \leq \int_1^n \ln{x} = n\ln{n} - n +1 & \leq \sum_{j=2}^n \ln{j} \\
    (n-1)! & \leq n^n e^{-n} n & \leq n! 
  \end{aligned} \\
  \frac{ n! }{ n^n } \geq n e^{-n} \\
  \begin{aligned}
  \lim_{n \to \infty} \frac{ n! }{ n^n } & \geq \lim_{n \to \infty} n e^{-n} = 0  \\
  \lim_{n \to \infty} \frac{ n! }{ n^n } & \leq \lim_{ n \to \infty} \frac{ n^2 }{ e^n} = 0 
  \end{aligned} \\
  \Longrightarrow \boxed{ \lim_{n \to \infty} \frac{n!}{n^n }  =0 }
\end{gathered}
\]
So then since $\frac{ n!}{ n^n } $ is a monotonically decreasing convergent sequence of real terms; if $\sum z$ is a bounded series, then by Dirichlet's test, $\sum \frac{n! z^n}{ n^n }$ is convergent.  

$|z| < 1; \quad |z| = 1 \, \text{ if } z \neq 1 $

Try the ratio test, because it's clear from the results of the ratio test where convergence and divergence begins and ends.
\[
\begin{aligned}
  \frac{ (n+1)! |z|^{n+1} }{ (n+1)^{n+1} } \frac{ n^n }{ n! |z|^n } & = \left( \frac{n}{n+1} \right)^n |z| = \\
  & = \left( \frac{1}{ 1+ 1/n } \right)^n |z| \xrightarrow{ n \to \infty} \frac{1}{e} |z|
\end{aligned}
\]
Converges for $ \boxed{ |z| < e }$.  

Now try plugging in a complexified $e$:
\[
\frac{e^j}{ \left( \frac{j^j}{j! } \right) } = \frac{ j! e^j e^{i2 \theta j } }{ (j)^j} \geq \frac{ j^j e^{-j}j e^j e^{i 2 \theta j }}{ j^j } e^{i 2 \theta j } \to \infty
\]
So the series diverges for $|z| = e$.  

\exercisehead{7} $\sum_{j=0}^{\infty} \frac{ (-1)^j (z+1)^j }{ j^2 + 1 } $ \bigskip \\

By the ratio test,
\[
\frac{ |z+1|^{j+1} }{ j^2 + 2j + 2 } \frac{j^2 + 1 }{ |z+1|^j } = |z+1| \frac{1  + 1/j^2 }{ 1 + 2/j + 2/j^2 } \xrightarrow{ j\to \infty} |z+1|
\]
The series absolutely converges for $|z+1| < 1$.  

For $z=0$, $\sum \frac{(-1)^j}{ j^2 + 1 }$ converges since $\frac{1}{j^2 + 1 } \xrightarrow{ j \to \infty} 0 $ \bigskip \\

For $z=-2$, $\sum \frac{ (-1)^j (-1)^j }{j^2 + 1} = \sum \frac{1}{ j^2 + 1}$ and $\frac{1}{j^2+1} < \frac{1}{j^2}$, but $\sum \frac{1}{j^2}$ is convergent (by integral test).  So the series converges for $z=-2$.  

By Dirichlet's test, the series converges as well, if we treat $a_j = (-1)^j (z+1)^j$ and $b_j = \frac{1}{j^2+1}$ to be the monotonically decreasing sequence.  

$\Longrightarrow  \boxed{ |z+1| \leq 1 } \text{ for convergence } $x

\exercisehead{8} $\sum_{j=0}^{\infty} a^{j^2} z^j$, $0 <a < 1$ \medskip \\
Use the \emph{ root test }.  
\[
(a^{j^2}z^j )^{1/j} = a^j z \xrightarrow{ j \to \infty} 0 
\]
So the series converges $\forall \, z \in \mathbb{C}$ 

\exercisehead{9} $\sum_{j=1}^{\infty} \frac{ (j!)^2}{ (2j)!} z^j$  \bigskip \\
Use the \emph{ ratio test } \\
\[
\begin{gathered}
\frac{ a_{j+1}}{ a_j } = \frac{ ((j+1)! )^2 z^{j+1} }{ (2(j+1))! } \frac{ (2j)!}{(j!)^2 z^j } = \frac{ (j+1)^2 z }{ (2j+2)(2j+1) } \xrightarrow{ j \to \infty} \frac{ 1 (z)}{ 4 } \\
\begin{aligned}
  & \text{ for $|z| < 4$, $\sum a_j$ absolutely converges } \\
  & \text{ for $|z| > 4$, $\sum a_j$ diverges }
\end{aligned}
\end{gathered}
\]

Let's test the boundary for convergence.  
\[
  \begin{aligned}
    \frac{(j!)^2}{(2j)!} 4^j = \frac{(j!)^2}{ (2j)! } e^{ j \ln{4}} & = \frac{ (j!)^2 }{ (2j-2)! } \frac{ e^{j\ln{4}} }{ (2j)(2j-1) } \geq \frac{ (j!)^2 }{ (2j-2)! } \frac{ e^{j \ln{4}} }{ (2j)^2 } \geq \\
    & \geq \frac{(j!)^2 }{ (2j-2)! } \frac{j^{2j} }{ (j!)^2 } = \frac{j^{2j}}{ (2j-2)! } \to \boxed{ \infty }
\end{aligned}
\]
where we had used
\[
\begin{gathered}
  (n-1)! \leq n^n e^{-n} n \leq n! \quad \quad \, \Longrightarrow ne^{-n} \leq \frac{n!}{n^n } \\
  \frac{n}{e^n } \leq \frac{n!}{n^n } \quad \quad \, \Longrightarrow \frac{n^n}{n!} \leq \frac{e^n}{n}
\end{gathered}
\]

\exercisehead{10}  $\sum_{j=1}^{\infty} \frac{3^{\sqrt{j}} z^j}{ j } = \sum_{j=1}^{\infty} \frac{e^{\sqrt{j} \ln{3} } z^j }{ j }$
\[
\begin{gathered}
  \frac{e^{\sqrt{j+1} \ln{3}} z^{j+1} }{ j+1 } \frac{ j}{ e^{\sqrt{j} \ln{3}} z^j } = \left( \frac{1}{j+1} \right) e^{ (\sqrt{j+1} - \sqrt{j}) \ln{3} } z \\
  \sqrt{j+1} - \sqrt{j} = \sqrt{ 1 + \frac{1}{j}} - 1 \simeq 1 + \frac{1}{2} \left( \frac{1}{j} \right) - 1 = \frac{1}{2j} \quad \quad \, \text{ (for large $j$ ) } \\
\text{ (for large $j$ ) }\quad \quad \,   \left( \frac{j}{j+1} \right) e^{ \frac{1}{2j} } z \to 0 \\
\Longrightarrow \text{ Converges } \quad \forall \, z \in \mathbb{C}
\end{gathered}
\]

\exercisehead{11} $\sum_{j=1}^{\infty} \left( \frac{ 1 \cdot 3 \cdot 5 \dots (2j-1) }{ 2 \cdot 4 \cdot 6 \dots (2j) } \right)^3 z^j $
\[
\begin{gathered}
  \frac{a_{j+1} }{ a_j} = \left( \frac{ 1 \cdot 3 \cdot 5 \dots (2j+1) }{ 2 \cdot 4 \cdot 6 \dots (2j+2) } \right)^3  z^{j+1} \left( \frac{2 \cdot 4 \cdot 6 \dots (2j) }{ 1 \cdot 3 \cdot 5 \dots (2j-1) } \right)^3 \frac{1}{z^j }  = \left( \frac{ 2j + 1 }{ 2j+2} \right)^3 z = \left( 1 + \frac{ -1/2}{j+2} \right)^3 z \\
  \text{ If $|z| < 1$, it converges by ratio test, \quad if $|z|= 1$, then it converges by Gauss test } \\
\begin{aligned}
  \frac{a_{j+1}}{a_j} & = \left( 1 + \frac{-1/2}{j+2} \right)^3 z = \sum_{k=0}^3 \binom{3}{k} \left( \frac{ -1/2}{ j+2 } \right)^k |z| =     \\
  & = |z| \left( 1 + 3 \left( \frac{ -1/2}{ j+2} \right) + 3 \left( \frac{ -1/2}{ j+2} \right)^2 + \left( \frac{ -1/2}{ j+2} \right)^3 \right) \xrightarrow{ j\to \infty} |z| 
\end{aligned} \\
\boxed{ \text{ diverges for $|z| > 1$ (by ratio test ) } }
\end{gathered}
\]

\exercisehead{12}   $\sum_{j=1}^{\infty} \left( 1 + \frac{1}{j} \right)^{j^2} z^j $   \\
\[
\begin{gathered}
  \left( \left( 1 + \frac{1}{j} \right)^{j^2} z^j \right)^{1/j} = \left( 1 + \frac{1}{j} \right)^j z \xrightarrow{ j \to \infty} e^1 z \\
  \begin{aligned}
    & |z| < \frac{1}{e}, \quad \sum (1 + \frac{1}{j} )^{j^2} z^j \text{ converges by root test }  \\
    & |z| > \frac{1}{e}, \quad \sum (1 + \frac{1}{j} )^{j^2} z^j \text{ diverges by root test } 
  \end{aligned} \\
\boxed{ \frac{1}{e} = r }
\end{gathered}
\]

\exercisehead{13} $\sum_{j=0}^{\infty} ( \sin{aj})z^j$ \quad \quad \, $a > 0$  \\
$|\sin{ (aj)} z^j | \leq |z|^j$ \\
By comparison test, $\sum_{j=0}^{\infty} (\sin{aj})z^j $ converges, since $\sum_{j=0}^{\infty} |z|^j$ converges absolutely, for $|z|<1$.   \bigskip \\
Note that if $a=\pi$, the series is zero.  

$\sum_{j=0}^{\infty} (\sin{aj}) \to \infty$ for $a=2\pi$ so \boxed{ $r=1$ indeed.   }

\exercisehead{14} $\sum_{j=0}^{\infty} (\sinh{aj})z^j = \sum_{j=0}^{\infty} \left( \frac{ e^{aj} - e^{-aj}}{ 2 } \right) z^j = \frac{1}{2} \left( \sum_{j=0}^{\infty} (e^a z)^j - \sum_{j=0}^{\infty} \left( \frac{ z}{e} a \right)^j \right)$ ; \quad \quad $a>0$ \medskip \\
If $|z| < \frac{1}{e^a}$, then $\sum \sinh{aj }z^j$ converges.  So then the radius of convergence is $r= \frac{1}{e^a}$

\exercisehead{15}$\sum_{j=1}^{\infty} \frac{ z^j}{ a^j + b^j }$.  Assume $a<b$ 
\[
\begin{gathered}
  \frac{ z^j}{ b^j ( 1 + \left( \frac{a}{b} \right)^j ) }  \\
  \text{ (ratio test) } \frac{ z^{j+1}}{ b^{j+1} (1 + \left( \frac{a}{b} \right)^{j+1} ) } \left( \frac{ b^j \left( 1 + \left( \frac{a}{b} \right)^j \right) }{ z^j } \right) = \frac{z}{b} \left( \frac{ 1 + \left( \frac{a}{b} \right)^j }{ 1 + \left( \frac{a}{b} \right)^{j+1} } \right) \xrightarrow{ j \to \infty} \left( \frac{z}{b} \right)  
\end{gathered}
\]
So then $|z| \lessgtr b $ converges (diverges) by ratio test.  

If $a=b$,
\[
\sum_{j=1}^{\infty} \frac{z^j}{2a^j} = \frac{1}{2} \sum_{j=1}^{\infty} \left( \frac{z}{a} \right)^j 
\]
By comparison test with $x^j$, if $|z| \lessgtr |a|$, the series converges (diverges).  

\exercisehead{16} $\sum_{j=1}^{\infty} \left( \frac{a^j}{j} + \frac{b^j}{j^2} \right) z^j $ Use ratio test on each of the sums, separately.  
\[
\begin{gathered}
\begin{gathered}
  \frac{ (a|z|)^{j+1}}{ j+1} \frac{ j}{ (a|z|)^j } = \frac{ a|z| }{ 1 + \frac{1}{j} } \xrightarrow{ j \to \infty} a |z| \\
  \Longrightarrow |z| < \frac{1}{a} \text{ then the series converges } 
\end{gathered} \\
\begin{gathered}
  \frac{ (b|z|)^{j+1}}{ (j+1)^2 } \frac{j^2}{ (b|z|)^j } = b|z| \left( \frac{1}{ 1 + \frac{1}{j}} \right)^2 \xrightarrow{ j\to \infty} b|z| \\
  \Longrightarrow |z| < \frac{1}{b}
\end{gathered}
\end{gathered}
\]
$\boxed{ \text{ So if $a \gtrless b$, then $r = a;(b)$ } }$

\exercisehead{17} $\int_0^1 f_n(x) = \int_0^1 nx e^{-nx^2} = \left. \frac{ e^{-nx^2}}{ -2 } \right|_0^1 = \frac{ e^{-n}-1}{ -2 } \xrightarrow{n \to \infty} \frac{1}{2} $ \\
However, \\
$\lim_{n\to \infty} nxe^{-nx^2} = 0$

This example shows that the operations of integration and limit cannot always be interchanged.  We need uniform convergence.  
\exercisehead{18} $f_n(a) = \frac{ \sin{nx}}{n}$  \quad \quad \, $\lim_{n \to \infty} \frac{\sin{nx}}{n} = 0$ \\
$f(x) = \lim_{n\to \infty} f_n(x)$ \\
$f_n'(x) = \frac{ n \cos{nx} }{ n } \quad \, \Longrightarrow \lim_{n\to \infty} f_n'(0) = 1$

This example shows that differentiation and limit cannot always be interchanged.  

\exercisehead{19}  $\sum_{j=1}^{\infty} \frac{ \sin{jx}}{j^2} = f(x)$ 
\[
\begin{gathered}
  \frac{ \sin{jx}}{ j^2} \leq \frac{1}{j^2 } \quad \quad \, \forall \, x \in \mathbb{R} \\
  \text{ by comparison test, $\sum_{j=1}^{\infty} \frac{ |\sin{jx} | }{ j^2 } $ converges, so $\sum_{j=1}^{\infty} \frac{ \sin{jx}}{ j^2 } $ converges } \\ 
  \left| \frac{ \sin{jx}}{ j^2} \right| \leq \frac{1}{ N^2 } \quad \quad \, \forall \, x \in \mathbb{R}; \quad \quad \, \forall j \geq N 
\end{gathered}
\]
$\sum{N} \frac{1}{N^2}$ converges, so $\sum \frac{ \sin{jx}}{j^2}$ uniformly converges.  \medskip \\
Then by Thm., since $\frac{ \sin{jx}}{j^2}$ are continuous, $\sum \frac{\sin{jx}}{j^2}$ is continuous.  \bigskip \\

Since $\sum \frac{ \sin{jx}}{ j^2}$ uniformly converges. 
\[
\begin{aligned}
  \int_0^{\pi} \sum \frac{ \sin{jx}}{j^2} & = \sum \int_0^{\pi} \frac{ \sin{jx}}{j^2} = \sum \left. \frac{ \cos{jx}}{ -j^3} \right|_0^{\pi} = \sum \frac{ (-1)^j - 1 }{ -j^3 } = \\
  & = \boxed{ 2 \sum_{j=1}^{\infty} \frac{1}{ (2j-1)^3 } }
\end{aligned}
\]

\exercisehead{20}
It is known that $\sum_{j=1}^{\infty} \frac{ \cos{jx}}{j^2} = \frac{ x^2}{4} - \frac{ \pi x }{2} + \frac{ \pi^2}{ 6}$ if $0 \leq x \leq 2 \pi$

\begin{enumerate}
\item $x = 2 \pi$ \\
  \[
  \sum_{j=1}^{\infty} \frac{1}{j^2} = \frac{ (2\pi)^2 }{4} - \frac{ 2 \pi^2}{2} + \frac{ \pi^2}{6} = \frac{ \pi^2}{6 }
\]
\item As shown in Ex. 19, $\sum \frac{ \cos{ jx}}{ j^2} $ is uniformly convergent on $\mathbb{R}$ 
\[
\begin{gathered}
  \sum \int \frac{ \cos{jx}}{ j^2} = \sum \left. \left( \frac{ \sin{jx}}{ j^3} \right) \right|_0^{\pi/2} = \sum \frac{ (-1)^{j+1} }{ (2j-1)^3 } \\
\begin{aligned}
  \int \frac{ x^2}{4} - \frac{ \pi x }{2 } + \frac{ \pi^2}{6} & = \left. \left( \frac{ 1}{3} \frac{ x^3}{4}  - \frac{ \pi x^2}{ 4 } + \frac{ \pi^2}{6} x \right) \right|_0^{\pi/2} =  \\
  & = (\pi)^3 \left( \frac{1}{ 12 (8) } - \frac{1}{16} + \frac{1}{12} \right) = \boxed{ (\pi)^3\frac{1}{32} }
\end{aligned}
\end{gathered}
\]
\end{enumerate}

%-----------------------------------%-----------------------------------%-----------------------------------
\subsection*{ 11.13 Exercises - Properties of functions represented by real power series, The Taylor's series generated by a function, A sufficient condition for convergence of a Taylor's series, Power-series expansions for the exponential and trigonometric functions, Bernstein's Theorem }
%-----------------------------------%-----------------------------------%-----------------------------------

Sufficient Condition for convergence.  
\begin{theorem}[Bernstein's Theorem]
Assume $\forall x \in [0,r], \, f(x), \, f^{(j)}(x) \geq 0 \quad \forall j \in \mathbb{N}$.  \\
Then if $0 \leq x < r$
\begin{equation}
\sum^{\infty} \frac{f^{(k)}(0)}{k! }x^k \quad \text{ converges }
\end{equation}
\end{theorem}

\begin{proof} If $x=0$, we're done.  Assume $0< x< r$.  
\[
\begin{aligned}
  f(x) & = \sum_{k=0}^n \frac{ f^{(k)}(0) }{ k! } x^k + E_n(x) \\
  E_n(x) & = \frac{ x^{n+1}}{n! } \int_0^1 u^n f^{(n+1)}(x-xu) du \\
  F_n(x) & = \frac{ E_n(x) }{ x^{n+1}} = \frac{1}{n!} \int_0^1 u^n f^{(n+1)}(x-xu) du 
  \begin{aligned}
    & \text{ Since } f^{(n+1)} > 0, \quad f^{(n+1)}(x(1-u)) \leq f^{(n+1)}(r(1-u)) \\
    & \quad \Longrightarrow F_n(x) \leq F_n(r) \Longrightarrow \frac{ E_n(x)}{x^{n+1}} \leq \frac{E_n(r)}{ r^{n+1 }}
  \end{aligned}
\end{aligned}
\]
  For $f(x) = \sum_{j=0}^n \frac{ f^{(j)}(0)}{j! }x^j + E_n(x) \Longrightarrow E_n(x) \leq \left( \frac{x}{r} \right)^{n+1} E_n(r)$ 
\[
f(r) = \sum_{j=0}^n \frac{ f^{(j)}(0)}{j!}r^j + E_n(r) \geq E_n(r) \text{ since } f^{(j)}(0) \geq 0 \quad \forall j 
\]
So then $0 \leq E_n(x) \leq \left( \frac{x}{r} \right)^{n+1} f(r) $ \\
\quad \quad $n \to \infty$ and $f(t)$ will be some non-infinite value, so $E_n(x) \xrightarrow{ n\to \infty} 0$.  
\end{proof}

\exercisehead{1} $\sum_{j=0}^{\infty} (-1)^j x^{2j}$ \\
Consider absolute convergence.  $\lim_{j\to \infty} (x^2)^j = 0 $ If $x^2 <1$  \bigskip \\
If $|x| = 1$, then consider
\[
\begin{gathered}
  x^{2(2j)} - x^{2 ( 2j+1) } = x^{4j} (1-x) \\
  \sum_{j=0}^{\infty} (1-x)x^{4j} = (1-x) \sum_{j=0}^{\infty} (x^4)^j \\
\boxed{  \text{ Indeed $\sum_{j=0}^{\infty} (-1)^j x^{2j} $ converges for $|x| \leq 1$ } }
\end{gathered}
\]

\exercisehead{2} $\sum_{j=0}^{\infty} \frac{x^j}{ 3^{j+1}} = \frac{1}{3} \sum_{j=0}^{\infty} \left( \frac{x}{3} \right)^j $ 
The series converges for $ \boxed{ |x| < 3  }$

\exercisehead{3} $\sum{j=0}^{\infty} j x^j$  
\[
\begin{gathered}
  \int_0^x \sum_{j=0}^{\infty} j t^{j-1}  = \sum_{j=0}^{\infty} x^j 
\end{gathered}
\]
So the series converges for $|x|<1$.  Note that we had used the integrability of power series.  


\exercisehead{4} $\sum_{j=0}^{\infty} (-1)^j j x^j$.  
\[
\begin{aligned}
  jx^j & j e^{j \ln{x}} \\
  \lim_{j \to \infty} j e^{j \ln{x}} &= \begin{cases} \infty & \text{ if } x > 1 \\ 0 & \text{ if } 0 < x < 1 \end{cases} 
\end{aligned}
\]
If $x=1$, 
\[
\begin{gathered}
  (2j)x^{2j}  - (2j+1) x^{2j+1} = x^{2j} (2j + -(2j +1) x ) = \boxed{ -1 }
  \sum (-1) = \infty
\end{gathered}
\]
So $\sum (-1)^j j x^j $ converges only for $0 \leq x < 1 , \quad |x| < 1$.  

\exercisehead{5} $\sum_{j=0}^{\infty} (-2)^j \frac{ j+2}{j+1} x^j = \sum_{j=0}^{\infty} (-1)^j \left( \frac{ j+2}{j+1} \right) (2x)^j $
\[
\lim_{j \to \infty} \left( \frac{ j+2}{ j+1} \right) (2x)^j = \lim_{j\to \infty} (2x)^j \quad \text{ if } |2x| < 1 
\]
So when $|x| < \frac{1}{2} $, the series converges.  

\[
\begin{gathered}
  \left( \frac{ 2j + 2 }{ 2j+1 } \right) - \left( \frac{2j+3}{ 2j + 2 } \right) = \frac{ (2j+2)(2j+2) - (2j+3)(2j+1) }{ (2j+1)(2j+2) } \\
  \frac{ 4j^2 + 8j + 4 - (4j^2 + 5j +3 ) }{ 4j^2 + 6j + 2 } = \frac{ 3j + 1 }{ 4j^2 + 6j + 2 } = \frac{ (3j+1)/2 }{ 2j^2 + 3j+1 } = \frac{ (3j +1 )/2 }{ (2j+1)(j+1) } \\
  \frac{ 3j + 1 }{ 4j^2 + 6j + 2 } < \frac{ 3j +1 }{ 4j^2 + \frac{4}{3} } < \frac{ 3 (j+1/3) }{ 4 (j^2 + 1/3) } < \frac{3}{4} \left( \frac{1}{j} \right) + \frac{1}{12j^2 }
\end{gathered}
\]
Thus, it diverges, by comparison test with $\frac{1}{j}$ for $x=\frac{1}{2}$.  

\begin{theorem} Let $f$ be represented by $f(x) = \sum_{j=0}^{\infty} a_j (x-a)^j$ in the $(a-r,a+r)$ interval of convergence 
\begin{enumerate}
\item $\sum_{j=1}^{\infty} j a_j (x-a)^{j-1}$ also has radius of convergence $r$.  
\item $f'(x)$ exists $\forall x \in (a-r, a+r)$ and 
\begin{equation}
  f'(x) = \sum_{j=1}^{\infty} j a_j (x-a)^{j-1}
\end{equation}
\end{enumerate}
\end{theorem}

\exercisehead{6} $\sum_{j=1}^{\infty} \frac{(2x)^j}{j} = \sum_{j=1}^{\infty} \frac{e^{j \ln{2x}}}{ j }$ \quad \quad \, $\Longrightarrow \boxed{ |x| < \frac{1}{2} }$ it'll converge, since by comparison test, $\frac{e^{j \ln{2x}} }{ j} < \frac{1}{j^2}$ if $ 0 < x <\frac{1}{2}$.  

\exercisehead{7} $\sum_{j=0}^{\infty} \frac{(-1)^j}{ (2j+1) } \left( \frac{x}{2} \right)^{2j}$.  
\[
\begin{gathered}
  \sum_{j=0}^{\infty} \frac{ x^{2j}}{ 2 (j+1/2) 2^{2j} } < \sum_{j=0}^{\infty} \frac{ x^{2j}}{ 2^{2j+1} j } = \frac{1}{2} \sum_{j=0}^{\infty} \left( \frac{x}{2} \right)^{2j} \left( \frac{1}{j} \right) = \frac{1}{2} \sum_{j=0}^{\infty} \frac{e^{2j \ln{\frac{x}{2} } } }{ j} \\
  \text{ since by comparison test, } \quad \frac{ e^{2j \ln{ \frac{x}{2}} } }{ j } < \frac{1}{j^2} \quad \, \text{ if $0<x<2$ } \\
  \text{ If $x = \pm 2$ },  
  \begin{aligned}
    \frac{1}{ 2 (2j) + 1 } - \frac{ 1 }{ 2 ( 2j+1) + 1 } & = \frac{1}{ 4j + 1 } - \frac{1}{ 4j + 3 } = \frac{ 2 }{ (4j+1)(4j+3) } \leq  \\ 
    & \leq \frac{1}{8} \frac{1}{j^2 } \text{ (converges by comparison test to $\sum \frac{1}{j^2} $ ) }
  \end{aligned}
\end{gathered}
\]
For $\boxed{ |x| < 2 }$ , $\sum_{j=0}^{\infty} \frac{ (-1)^j }{ (2j+1) } \left( \frac{x}{2} \right)^{2j} $ converges.  

\exercisehead{8} $\sum_{j=0}^{\infty} \frac{(-1)^j x^{3j}}{ j! } = \sum_{j=0}^{\infty} \frac{ (-x^3)^j}{j!} = e^{-x^3}$, which converges $\forall \, x \in \mathbb{R}$ 

\exercisehead{9} $\sum_{j=0}^{\infty} \frac{x^j}{(j+3)!} = \frac{1}{x^3} \sum_{j=0}^{\infty} \frac{x^{j+3}}{ (j+3)! } = \frac{1}{x^3} \sum_{j=+3}^{\infty} \frac{x^j}{j! } = \frac{1}{x^3} ( e^x - x^2/2 - x - 1 ) $ Thus, it converges for $\forall \, x \in \mathbb{R}$.  

\exercisehead{10} $\sum_{j=0}^{\infty} \frac{ (x-1)^j }{ (j+2)! } = \frac{1}{ (x-1)^2} \sum_{j=0}^{\infty} \frac{ (x-1)^{j+2} }{ (j+2)! } = \frac{1}{ (x-1)^2 } \left( \sum_{j=2}^{\infty} \frac{ (x-1)^j }{ j! } \right) = \frac{1}{(x-1)^2 } \left( e^{x-1} - (x-1) - 1 \right) = \boxed{ \frac{ e^{x-1} -x }{ (x-1)^2 } }$

\exercisehead{11} $a^x = e^{x \log{a}} = \sum_{j=0}^{\infty} \frac{ (\log{ax})^j }{ j! }$ \\
$\frac{ (\log{ax})^{j+1}}{ (j+1)! } \frac{j!}{ (\log{ax})^j } = \frac{ (\log{ax})}{ j+1} \xrightarrow{ j\to \infty} 0 $.  
By ratio test, $\sum_{j=0}^{\infty} \frac{ (\log{ax})^j }{ j! }$ converges for all $x$.  

\exercisehead{12} 
\[
\begin{gathered}
  \sinh{x} = \frac{ e^x - e^{-x}}{ 2 } = \frac{1}{2} \left( \sum_{j=0}^{\infty} \frac{x^j}{j!}  - \sum_{j=0}^{\infty} \frac{ (-x)^j }{ j! } \right) = \sum_{j=0}^{\infty} \frac{x^{2j+1}}{ (2j+1)! } \\
  \frac{x^{2j+3}}{ (2j+3)! } \frac{ (2j+1)!}{ x^{2j+1} } = \frac{x^2}{ (2j+3)(2j+2) } \xrightarrow{ j\to \infty} 0 
\end{gathered}
\]

\exercisehead{13} 
\[
\begin{gathered}
  \sin^2{x} = \frac{1 - \cos{2x}}{ 2 } = \frac{ 1 - \sum_{j=0}^{\infty} \frac{ (2x)^{2j}}{ (2j)! } (-1)^j }{ 2 } = \sum_{j=1}^{\infty} \frac{ 2^{2j-1} x^{2j} (-1)^{j+1} }{ (2j)! } \\
  \frac{ 2^{2j+1} x^{2j+2} }{ (2j+2)!} \frac{ (2j)!}{ 2^{2j-1} x^{2j} } = \frac{ 4 x^2 }{ (2j+2)(2j+1) } \xrightarrow{ j \to \infty} 0 
\end{gathered}
\]
So the series converges $\forall \, x$

\exercisehead{14} $\frac{1}{ 1 - \frac{x}{2} } = \sum_{j=0}^{\infty} \left( \frac{x}{2} \right)^j$ \quad \quad $\frac{1}{2-x} = \sum_{j=0}^{\infty} \frac{x^j}{2^{j+1} }$ 
\[
\begin{gathered}
  \frac{ x^{j+1} }{ 2^{j+2} } \frac{2^{j+1} }{ x^j } = \frac{ x}{2} < 1 \quad \, \Longrightarrow \text{ (the series converges for $|x| <2$, by ratio test) } \\
  \text{ If $x=2$, the series would diverge } \\
  \text{ If $x=-2$ } \frac{1}{2} \sum_{j=0}^{\infty} \frac{ (-2)^j }{ 2^j } = \frac{1}{2} \sum_{j=0}^{\infty} (-1)^j  = 0 \text{ but $\frac{1}{ 2 - (-2) } = \frac{1}{4}$x }
\end{gathered}
\]

\exercisehead{15} $e^{-x^2} = \sum_{j=0}^{\infty} \frac{ (-1)^j x^{2j}}{ j!}$  \\
$ \frac{ x^{2j+2}}{ (j+1)!} \frac{j!}{ x^{2j} } = \frac{ x^2}{j+1} \xrightarrow{ j\to \infty} 0 $

\exercisehead{16} $\sin{3x} = \sin{2x}\cos{x} + \sin{x} \cos{2x} = 3 \sin{x} - 4 \sin^3{x}$.  
\[
\begin{aligned}
  \sin^3{x} & = \frac{3 \sin{x} - \sin{3x} }{4} = \frac{3}{4} \left( \sum_{j=0}^{\infty} \frac{ x^{2j+1} (-1)^j }{ (2j+1)! } - \sum_{j=0}^{\infty} \frac{ (3x)^{2j+1} (-1)^j }{ (2j+1)! } \right) = \\
  & = \frac{3}{4} \left( \sum_{j=0}^{\infty} \frac{ (-1)^j x^{2j+1} (1-3^{2j+1} ) }{ (2j+1)! } \right) = \frac{3}{4} \left( \sum_{j=0}^{\infty} \frac{ (-1)^{j+1} (3^{2j+1} - 1) x^{2j+1} }{ (2j+1)! } \right)
\end{aligned}
\]


\exercisehead{17} $\log{ \sqrt{ \frac{1+x}{1-x} }} = \frac{1}{2} ( \log{ ( 1 + x)} - \log{ (1-x) } ) = \frac{1}{2} \left( \sum_{j=1} \frac{(+x)^j }{j} (-1)^{j-1} - \sum_{j=1} \frac{ (x^j)(-1)^j }{ j } \right)$
\[
\begin{gathered}
  \ln{(1+x) } = \sum_{j=0}^{\infty} \frac{ (-x)^{j+1}}{j+1} (-1) = \sum_{j=1}^{\infty} \frac{ (-x)^j}{j} (-1) \quad \quad \, -\ln{(1-x)} = \sum_{j=0}^{\infty} \frac{ x^{j+1}}{ j+1}  = \sum_{j=1}^{\infty} \frac{x^j}{j} \quad \Longrightarrow - \sum_{j=1}^{\infty} \frac{ ((-1)^j + 1 ) x^j }{ j } \\
  \Longrightarrow \sum^{\infty} \boxed{ \frac{ x^{2j+1}}{ 2j+1 }  } \\
  \frac{ x^{2j+3} }{ 2j+3} \frac{ 2j + 1 }{ x^{2j+1} } \xrightarrow{ j\to \infty} x^2 \\
  |x^2| < 1 , \quad \quad \, \text{ converges, with radius of convergence of $1$ }
\end{gathered}
\]

\exercisehead{18} $\frac{3x}{ 1 + x - 2x^2} = \frac{1}{1- x} - \frac{1}{1+2x} = \sum_{j=0}^{\infty} \frac{x^j}{j} - \sum_{j=0}^{\infty} \frac{ (-2x)^j }{ j} = \sum_{j=0}^{\infty} \frac{x^j}{j} ( 1 - (-2)^j )$ 
\[
\begin{gathered}
  \frac{ x^{j+1} | (1- (-2)^{j+1} ) | }{ j+1} \frac{ j}{ |x^j (1- (-2)^j ) | } = x \left| \frac{ \left( \frac{1}{ (-2)^j } + 2  \right) }{ \left( \frac{1}{ (-2)^j } - 1 \right) } \right| \xrightarrow{ j \to \infty} 2 x < 1 \\
  |x| < \frac{1}{2} \\
  \begin{gathered}  
    \text{ For $x=\frac{1}{2}$ } \\
    \frac{ x^{2j}}{ 2j } ( 1 - (-2)^{2j} ) + \frac{ x^{2j+1} }{ 2j+1} (1- (-2)^{2j+ 1 } ) = \frac{ x^{2j} ((2j+1)(1-2^{2j}) + x(1+2^{2j+1})(2j) ) }{ (2j)(2j+1) } \\
     \frac{ \left( \frac{1}{2} \right)^{2j} ( 2 j + 1 - (2j+1)2^{2j} + 2j x + 2^{2j+2} jx ) }{ (2j)(2j+1) } \xrightarrow{ j\to \infty} \frac{ - (2j+1) + 4j }{ (2j)(2j+1) } = \frac{ 2j - 1 }{ (2j)(2j+1) } \to 0
  \end{gathered}
\end{gathered}
\]
So $\frac{3x}{ 1 + x- 2x^2}$ converges for $|x| \leq \frac{1}{2}$

\exercisehead{19} $\frac{12- 5x}{ 6 -5 x - x^2 } = \sum_{j=0}^{\infty} \left( 1 + \frac{ (-1)^j }{ 6^j } \right) x^j$  \quad $(|x| <1)$.  
\[
\begin{gathered}
  \frac{12- 5x}{ 6 -5x - x^2 } = \frac{ 5x - 12 }{ (x+6)(x-1) } =   \frac{1}{1-x} +\frac{6}{ 6+x} = \sum_{j=0}^{\infty} x^j + \sum_{j=0}^{\infty} \left( \frac{-x}{6} \right)^j = \sum_{j=0}^{\infty} x^j (1 + \left( \frac{-1}{6} \right)^j ) 
\end{gathered}
\]
$|x| <1$ since for $x=1$, $\lim_{j\to \infty} (1 + \left( \frac{-1}{6} \right)^j ) = 1 $.  

\exercisehead{20} $\frac{1}{ x^2 + x + 1 } = \frac{2}{ \sqrt{3}} \sum_{j=0}^{\infty} \sin{ \frac{ 2\pi (j+1) }{3} } x^j $ \quad $(|x| < 1)$

\[
  \begin{gathered}
    x^3 - 1 = (x-1)(x^2 + x + 1 ) \\
    \frac{1}{ x^2 + x + 1 } = \frac{ x- 1}{ x^3 - 1 } = \frac{1-x}{ 1 - x^3 }
  \end{gathered} \quad \quad \, 
\begin{aligned}
  \frac{1}{ 1 - x^3} & = \sum_{j=0}^{\infty} (x^3)^j \\ 
  \frac{1-x}{ 1 - x^3 } & = \sum_{j=0}^{\infty} \left( (x^3)^j - x^{3j+1} \right) \\ 
\end{aligned}
\]
By induction, it could be observed that $x^{3j}, -x^{3j+1}, 0$ appear in sequences of $3$ terms.  \medskip \\
\quad \quad $\frac{2}{\sqrt{3}} \sin{ \frac{ 2 \pi (j+1)}{ 3 } }$ accounts for this.  \bigskip \\
$\Longrightarrow \frac{2}{\sqrt{3}} \sum_{j=0}^{\infty} \sin{ \frac{2 \pi (j+1)}{ 3 } x^j }$ \quad \, $|x| < 1$ \bigskip \\

$\sin{ \frac{ 2\pi (j+1)}{ 3 } x^j } < x^j$ (so by comparison test to $\sum x^j$, the radius of convergence is $1$ )

\exercisehead{21} 
\[
\begin{gathered}
  \frac{1}{1-x} = \sum_{j=0}^{\infty} x^j ; \quad \quad \left( \frac{1}{1-x} \right)' = \frac{1}{ (1-x)^2} = \sum_{j=1}^{\infty} j x^{j-1} = \sum_{j=0}^{\infty} (j+1)x^j \\
  \frac{1}{ (1-x)(1-x^2) } = \frac{1/4}{ 1-x} + \frac{ 1/2}{ (1-x)^2 } + \frac{ 1/4}{ 1+x} = \frac{1}{4} \left( \sum_{j=0}^{\infty} ( x^j + (-x)^j ) + 2 \sum_{j=1}^{\infty} j x^{j-1} \right)  \\
  \frac{x}{ (1-x)(1-x^2)} = \frac{1}{4} \left( \sum_{j=0}^{\infty} x^{j+1} (1+ (-1)^j ) + 2 \sum_{j=1}^{\infty} jx^j \right) = \frac{1}{4} \sum_{j=1}^{\infty} x^j ( 1 + (-1)^{j-1} + 2 j )
\end{gathered}
\]

\exercisehead{22} 
\[
\begin{gathered}
  \sin{ \left( 2x + \frac{ \pi}{4} \right) } = \sum_{j=0}^{\infty} \frac{ (2x)^{2j+1} }{ (2j+1)! } (-1)^j ; \quad \quad \cos{2x} = \sum_{j=0}^{\infty} \frac{(2x)^{2j} }{ (2j)!} (-1)^j  \\
  \begin{aligned}
    \sin{2x} & = \sum_{j=0}^{\infty} \frac{ (2x)^{2j+1}}{ (2j+1)! } (-1)^j \\
    \cos{2x} & = \sum_{j=0}^{\infty} \frac{ (2x)^{2j}}{ (2j)! } (-1)^j 
  \end{aligned} \quad \quad \sum_{j=0}^{\infty} a_j x^j = \frac{ \sqrt{2}}{ 2} (\sin{2x} + \cos{2x} ) \\
  \text{ For } j = 98, \quad a_j = \frac{ 2^{98}  (-1)^{49}}{ 98! } \left( \frac{ \sqrt{2}}{ 2 } \right) = \boxed{ \frac{ -2^{98}}{ 98! }\left( \frac{\sqrt{2}}{ 2 } \right) }
\end{gathered}
\]

\exercisehead{23} 
\[
\begin{gathered}
\begin{aligned}
  f(x) & = (2+x^2)^{5/2} \\
  f'(x) & = \frac{5}{2} (2+x^2)^{3/2} (2x) = 5x (2+x^2)^{3/2} \\
  f''(x) & = 5 (2+x^2)^{3/2} + \frac{15}{2} x (2+x^2)^{1/2} (2x) = 5 (2+x^2)^{3/2} + 15 x^2 (2+x^2)^{1/2} \\
  f'''(x) & = \frac{15}{2} (2+x^2)^{1/2} (2x) + 30 x (2 +x^2)^{1/2} + \frac{ 15x^2}{2} (2+x^2)^{-1/2} (2x) \\
  f''''(x) & = 15 (2+x^2)^{1/2} + \frac{15}{2} (2+x^2)^{-1/2} (2x) x + 30 (2+x^2)^{1/2} + \frac{30x}{2} (2+x^2)^{1/2} (2x) + \\
  & + 45 x^2 (2+x^2)^{-1/2} + 15 x^3 \left( - \frac{1}{2} \right) (2+x^2)^{-3/2} (2x) 
\end{aligned} \\
\boxed{ 2^{5/2} + 0 x + \frac{ 5(2^{3/2}) x^2}{ 2! } + \frac{ 0x^3}{3!} + \frac{45}{4!} \sqrt{2} x^4 }
\end{gathered}
\]

\exercisehead{24} $f(x) = e^{-1/x^2}$ if $x\neq 0$ and let $f(0) = 0 $
\begin{enumerate}
\item \[
\begin{gathered}
  f(x) = \sum_{j=0}^{\infty} \frac{ \left( \frac{-1}{x^2} \right)^j }{j! } = 1 + \frac{-1}{x^2} + \sum_{j=2}^{\infty} \frac{ \left( \frac{-1}{x^2 } \right)^j }{j! } = \sum_{j=0}^{\infty} \frac{ (-1)^j x^{-2j} }{j! } \\
  f^{(k)} = \sum_{j=0}^{\infty} (-1)^j \frac{ (-2j)(-2j-1)\dots (-2j - (k-1)) }{ j! } x^{-2j} = \sum_{j=0}^{\infty} (-1)^j \frac{ (-2j)!}{ (-2j-k)! j! } x^{-2j} 
\end{gathered}
\]

Use \emph{ ratio test }:
\[
\frac{ (-2(j+1))! }{ (-2j-2 -k)! (j+1)! } \frac{j! (-2j-k)!}{ (-2j)! } \frac{ x^{-2(j+1)}}{ x^{-2j}} = \frac{ (-2j-k)(-2j - k -1)}{ (j+1)(-2j)(-2j-1) }x^{-2} \xrightarrow{ j \to \infty} 0 
\]
Thus, by ratio test, every order of derivative exists for every $x$ on the real line since the series representing the derivative converges for every $x$.  
\item $f(x) = \sum_{j=0}^{\infty} \frac{-x^{-2j}}{j!}$.  There are no nonzero terms of positive power, i.e. no $x^j$; \quad \, $ j \geq 1$.  \medskip \\
  $\Longrightarrow f^{(j)}(0) = 0 \quad \, \forall \, j \geq 1$
\end{enumerate}

%-----------------------------------%-----------------------------------%-----------------------------------
\subsection*{ 11.16 Exercises - Power series and differential equations, binomial series } 
%-----------------------------------%-----------------------------------%-----------------------------------
 
\quad \\

\exercisehead{1} For $(1-x^2)y'' - 2xy' + 6y = 0$,
\[
\begin{gathered}
  \begin{aligned}
    & y = \sum_{j=0}^{\infty} a_j x^j \\
    & y'  = \sum_{j=1}^{\infty} j a_j x^{j-1} \\
    & y'' =  \sum_{j=2}^{\infty} j (j-1)a_j x^{j-2} = \sum_{j=0}^{\infty} (j+2)(j+1) a_{j+2} x^j 
  \end{aligned} \quad \quad \, 
  f(0) = 1 \Longrightarrow a_0 = 1 \quad \quad \, f'(0) = 0 \Longrightarrow a_1 = 0  \\
  \begin{aligned}
    &    2(1) a_2 + 3 (2) a_3 x + -2(1) a_1 x + 6a_0 + 6a_1 x + \sum_{j=2}^{\infty} ((j+2)(j+1) a_{j+2} - j(j-1)a_j - 2j a_j + 6a_j ) x^j = \\ 
    & = 2a_2 + 6 + 6a_3 x + \sum_{j=2}^{\infty} ( (j+2)(j+1) a_{j+2} + a_j (j+3)(j-2) ) x^j 
  \end{aligned} \\
  \Longrightarrow \begin{aligned}
    a_2 & = -3 \\
    a_3 & = 0 
  \end{aligned} \quad \quad \, \boxed{ a_{j+2} = \frac{ (j+3)(j-2) }{ (j+2)(j+1) } a_j } 
\end{gathered}
\]
For $j=2, \quad a_4 = 0$, so then $a_{j+2} = 0$ for $j=2,4, \dots$.  Likewise, since $a_3 = 0$, then $a_{j+2} =0$ for $j=3,5, \dots$.   \\
$\Longrightarrow f(x) = 1 - 3 x^2$

\exercisehead{2}
Using the work from above, then for $(1-x^2)y'' - 2xy' + 12 y = 0$
\[
\begin{gathered}
    f(0) = 0 \Longrightarrow a_0 = 0 \quad \quad \, f'(0) = 2 \Longrightarrow a_1 = 2  \\
  \begin{aligned}
    &    2(1) a_2 + 3 (2) a_3 x + -2(1) a_1 x + 12a_0 + 12a_1 x + \sum_{j=2}^{\infty} ((j+2)(j+1) a_{j+2} - j(j-1)a_j - 2j a_j + 12a_j ) x^j = \\ 
    & = 2a_2 +  6a_3 x + -4x + 0 + 24 x + \sum_{j=2}^{\infty} ( (j+2)(j+1) a_{j+2} - a_j (j+4)(j-3) ) x^j 
  \end{aligned} \\
  \Longrightarrow \begin{aligned}
    a_2 & = 0 \\
    a_3 & = -10/3 
  \end{aligned} \quad \quad \, \boxed{ a_{j+2} = \frac{ (j+4)(j-3) }{ (j+2)(j+1) } a_j }
\end{gathered}
\]
For $j=3, \quad a_5 = 0$, so then $a_{j+2} = 0$ for $j=3,5, \dots$.  Likewise, since $a_2 = 0$, then $a_{j+2} =0$ for $j=2,4, \dots$.   \\
$\Longrightarrow f(x) = -10/3x^3 +2   $

\exercisehead{3} $f(x)  = \sum_{j=0}^{\infty} \frac{ x^{4j}}{ (4j)! }; \quad \, \frac{ d^4 y}{ dx^4} = y$
\[
\begin{gathered}
\begin{aligned}
  & (x^4)' = 4 x^3 \\
  & (x^4)'' = 12 x^2 \\
  & (x^4)''' = 24 x \\
  & (x^4)'''' = 24 
\end{aligned}
\quad \quad \, 
\begin{aligned}
  & \left( \frac{ x^{4j}}{ (4j)! } \right)' = \frac{ x^{4j-1} }{ (4j-1)! } \\
  & \left( \frac{ x^{4j}}{ (4j)! } \right)'' = \frac{ x^{4j-2}}{ (4j-2)! } \\
  & \left( \frac{ x^{4j}}{ (4j)! } \right)''' = \frac{ x^{4j-3} }{ (4j-3)! } \\
  & \left( \frac{ x^{4j}}{ (4j)! } \right)'''' = \frac{ x^{4j -4 }}{ (4j-4)! } 
\end{aligned} \quad \quad \, 
\sum_{j=1}^{\infty} \frac{ x^{4j-4}}{ (4j-4)! } = y'''' = \boxed{ \sum_{j=0}^{\infty} \frac{ x^{4j}}{ (4j)! } } = f(x) \\
\quad \\ \text{ To test convergence, use the ratio test } \\
\frac{ x^{4j+4}}{ (4j+4)! } \frac{ (4j)! }{ x^{4j} } = \frac{ x^4 }{ (4j+4)(4j+3)(4j+2)(4j+1) } \xrightarrow{ j\to \infty} 0 \quad \, \forall \, x \in \mathbb{R}
\end{gathered}
\]
So the series converges on $\mathbb{R}$.  

\exercisehead{4}$f(x) = \sum_{j=0}^{\infty} \frac{x^j}{ (j!)^2 }$ \quad \quad \, $xy'' + y' - y =0$
\[
\begin{gathered}
  \begin{aligned}
  y' & = \sum_{j=1}^{\infty} \frac{ j x^{j-1} }{ (j!)^2 } = \sum_{j=0}^{\infty} \frac{ (j+1)x^j }{ ((j+1)! )^2 } \\
  y'' & = \sum_{j=1}^{\infty} \frac{ (j+1)j x^{j-1} }{ ((j+1)! )^2 } 
\end{aligned} \\
  \sum_{j=1}^{\infty} \left( \frac{ (j+1) j }{ ((j+1)! )^2 } + \frac{ (j+1)}{ ((j+1)!)^2 } - \frac{1}{ (j!)^2 } \right)x^j + \frac{1}{1!} - 1 = \sum_{j=1}^{\infty} \left( \frac{1}{ (j!)^2 } - \frac{1}{ (j!)^2 } \right) = 0
\end{gathered}
\]



\exercisehead{5} $f(x) = 1 + \sum_{j=1}^{\infty} \frac{ 1\cdot 4 \cdot 7 \dots (3j-2) }{ (3j)!} x^{3j}; \quad y'' = x^a y + b \quad \text{ (Find $a$ and $b$ ) } $ 
\[
\begin{aligned}
  f' & = \sum_{j=1}^{\infty} \frac{ 1 \cdot 4 \cdot 7 \dots (3j-2) }{ (3j-1)! } x^{3j-1} \\
  f'' & = \sum_{j=1}^{\infty} \frac{ 1 \cdot 4 \cdot 7 \dots (3j-2) }{ (3j-2)! } x^{3j-2} = \sum_{j=1}^{\infty} \frac{ 1 \cdot 4 \cdot 7 \dots (3j-5) }{ (3j-3)! } x^{3j-2} = \\
  & = x + \sum_{j=2}^{\infty} \frac{ 1 \cdot 4 \cdot 7 \dots (3j-5) }{ (3j-3)! } x^{3j-2} = x + \sum_{j=1}^{\infty} \frac{ 1 \cdot 4 \cdot 7 \dots (3j-2) }{ (3j)!} x^{3j+1} \\
  x^a f & = -x^a + \sum_{j=1}^{\infty} \frac{1 \cdot 4 \cdot 7 \dots (3j-2) }{ (3j)! } x^{3j+a}
\end{aligned}
\]
So then $\boxed{ a = 1 ; \quad b =0 }$.  
\[
\frac{ 1 \cdot 4 \cdot 7 \dots (3j+1) }{ (3j+3)! } x^{3j+3} \left( \frac{ (3j)! }{ 1 \cdot 4 \cdot 7 \dots (3j-2) } \right) \frac{1}{ x^{3j}} = \frac{1(3j+1)}{ (3j-2)(3j+3)(3j+2)(3j+1) } x^3 \xrightarrow{j \to \infty} 0 
\]
So the series converges for all $x$.  

\exercisehead{6} $f(x) = \sum_{j=0}^{\infty} \frac{ x^{2j}}{j!}; \quad y' = 2 xy$.  
\[
f' = \sum_{j=1}^{\infty} \frac{ 2j x^{2j-1}}{ j! } =2 \sum_{j=1}^{\infty} \frac{ x^{2j-1}}{ (j-1)! } = 2 \sum_{j=0}^{\infty} \frac{ x^{2j+1}}{ j!} = 2 x f
\]
\[
\frac{ x^{2j+2}}{(j+1)!} \frac{j!}{x^{2j}} = \frac{x^2}{j+1} \xrightarrow{j\to \infty} 0 \quad \forall x
\]
By ratio test, $f$ converges $\forall \, x \in \mathbb{R}$. 

\exercisehead{7} $f(x) = \sum_{j=2}^{\infty} \frac{x^j}{j!} \quad y' = x+ y $
\[
f' = \sum_{j=2}^{\infty} \frac{x^{j-1}}{ (j-1)! } = \sum_{j=1}^{\infty} \frac{x^j}{j!} = x + y 
\]
\[
\frac{ x^{j+1}}{ (j+1)! } \frac{j!}{x^j} = \frac{x}{j} \xrightarrow{j\to \infty} 0 
\]
So the series converges $\forall \, x \in \mathbb{R}$ by ratio test.  

\exercisehead{8}
\[
\begin{gathered}
  \begin{aligned}
    f(x) & = \sum_{j=0}^{\infty} \frac{(-1)^j (kx)^{2j}}{ (2j)! } \\
    f' & = \sum_{j=1}^{\infty} \frac{ (-1)^j (kx)^{2j-1} k }{ (2j-1)! } = \sum_{j=0}^{\infty} \frac{ (-1)^{j+1} (kx)^{2j+1} k }{ (2j+1)! } \\
    f'' & = \sum_{j=1}^{\infty} \frac{ (-1)^{j+1} (kx)^{2j} }{ (2j)! } k^2 
  \end{aligned} \quad \quad f'' -k^2 f = 0 \\
  \frac{ (kx)^{2j+1}}{ (2j+1)! } \frac{ (2j)! }{ (kx)^{2j} } = \frac{ kx}{2k+1} \xrightarrow{ j\to \infty} 0 \quad \text{ by ratio test, $f$ converges $\forall x \in \mathbb{R}$.  }
\end{gathered}
\]

\exercisehead{9} 
\[
\begin{gathered}
  f'' = \sum_{j=1}^{\infty} \frac{ (3x)^{2j-1} }{ (2j-1)! } 9 = \sum_{j=0}^{\infty} \frac{ 9 (3x)^{2j+1}}{ (2j+1)! } \\
  9 (f-x) = 9 (x + \sum_{j=0}^{\infty} \frac{ (3x)^{2j+1}}{ (2j+1)! } - x ) \\
  \frac{ (3x)^{2j+3}}{ (2j+3)! } \frac{ (2j+1)! }{ (3x)^{2j+1}} = \frac{ 9x^2 }{ (2j+3)(2j+2) } \xrightarrow{j\to \infty} 0 \\
\text{ (by ratio test, $f$ converges $ \forall x \in \mathbb{R}$ ) }
\end{gathered}
\]

\exercisehead{10} $J_0(x) = \sum_{j=0}^{\infty} (-1)^j \frac{ x^{2j}}{ (j!)^2 2^{2j}}$ \quad \quad $J_1(x) = \sum_{j=0}^{\infty} (-1)^j \frac{ x^{2j+1}}{ j! (j+1)! 2^{2j+1} }$.  
\begin{enumerate}
\item
\[
\begin{gathered}
  \frac{ x^{2j+2}}{ ((j+1)!)^2 2^{2j+2} } \frac{ (j!)^2 2^{2j}}{ x^{2j}} = \frac{ x^2}{ (j+1)^2 4 } \xrightarrow{ j\to \infty} 0 \quad \quad \text{ by ratio test, $f$ converges $\forall x \in \mathbb{R} $ } \\
  \frac{ x^{2j+3}}{ (j+2)! 2^{2j+ 3} } \frac{ j! 2^{2j+1}}{ x^{2j+1}} = \frac{x^2}{ (j+2)(j+1)4 } \xrightarrow{j\to \infty} 0 \quad \quad \text{ by ratio test, $f$ converges $\forall x \in \mathbb{R}$ }    
\end{gathered}
\]
\item \[
\begin{gathered}
  J_0'(x) = \sum_{j=1}^{\infty} (-1)^j \frac{ x^{2j-1}}{ (j-1)! (j!) 2^{2j-1}} = \sum_{j=0}^{\infty} (-1)^{j+1} \frac{ x^{2j+1}}{ j! (j+1)! 2^{2j+1} } = -J_1(x) 
\end{gathered}
\]
\item
\[
\begin{gathered}
  j_0(x) = x J_0(x) = \sum_{j=0}^{\infty} (-1)^j \frac{ x^{2j+1}}{ (j!)^2 2^{2j}} \quad \quad \begin{aligned} j_1(x) & = x J_1(x) = \sum_{j=0}^{\infty} (-1)^j \frac{ x^{2j+2}}{ j! (j+1)! 2^{2j+1} } \\
    j_1'  &= \sum_{j=0}^{\infty} \frac{ (-1)^j x^{2j+1} }{ (j!)^2 2^{2j} } \end{aligned}\\
    \Longrightarrow j_0 = j_1' 
   \end{gathered}
\]
\end{enumerate}


\exercisehead{11} $x^2 y'' + xy' + (x^2 - n^2) y = 0$.   \\
$n=0 \Longrightarrow x^2 y'' + xy' + (x^2) y = 0 $
\[
\begin{gathered}
  \begin{aligned}
    J_0 & = \sum_{j=0}^{\infty} (-1)^j \frac{ x^{2j}}{ (j!)^2 2^{2j}} = \sum_{j=1}^{\infty} (-1)^{j-1} \frac{ x^{2j-2}}{ ((j-1)!)^2 2^{2j-2}}; \\
    J_0' & = \sum_{j=1}^{\infty} (-1)^j \frac{ x^{2j-1}}{ j! (j-1)! 2^{2j-1}}; \\
    J_0'' & = \sum_{j=1}^{\infty} (-1)^j \frac{ x^{2j-2}}{ j! (j-1)! } \frac{ (2j-1) }{ 2^{2j-1}} 
\end{aligned} \\
\sum_{j=1}^{\infty} (-1)^j \left( \frac{ (2j-1)}{ j! (j-1)! 2^{2j-1}} + \frac{1}{ j! (j-1)! 2^{2j-1} } + \frac{ -2j}{ ((j-1)!)j! 2^{2j-1} } \right) = 0 
\end{gathered}
\]

$n=1 \Longrightarrow x^2 y'' + xy' + (x^2 - 1 ) y = 0$
\[
\begin{gathered}
\begin{aligned}
  J_1(x) & = \sum_{j=0}^{\infty} \frac{ (-1)^j x^{2j+1}}{ j! (j+1)! 2^{2j+1} } = \frac{x}{2} + \sum_{j=1}^{\infty} \frac{ (-1)^j x^{2j+1}}{ j! (j+1)! 2^{2j+1}} = \frac{x}{2} + \sum_{j=1}^{\infty} \frac{ (-1)^{j-1} x^{2j-1}}{ (j-1)!(j)! 2^{2j-1}} \\
  J_1' & = \frac{1}{2} + \sum_{j=1}^{\infty} \frac{ (-1)^j (2j+1) x^{2j} }{ j! (j+1)! 2^{2j+1} } \\
  J_1'' & = \sum_{j=1}^{\infty} \frac{ (-1)^j (2j+1)(2j) x^{2j-1} }{ (j!)(j+1)! 2^{2j+1} }
\end{aligned} \\
x^2 J_1'' + x J_1' + (x^2- 1)J_1 = \\
\begin{aligned}
 = \sum_{j=1}^{\infty}  & x^{2j+1} \left( \left( \frac{ (-1)^j (2j+1) }{ (j!)(j+1)! 2^{2j+1} } \right) ((2j)+ (1) ) + \frac{ (-1)^j(-1) }{ (j-1)!j! } \left( \frac{ j+1}{j+1} \right) \left( \frac{j}{j} \right) \left( \frac{1}{ 2^{2j-1}} \right)\left( \frac{2^2}{2^2} \right) - \frac{ (-1)^j }{ (j+1)!(j!)2^{2j+1} }  \right) + \\
 & + \frac{x}{2} -\frac{x}{2} =  \\
 = \sum_{j=1}^{\infty} & \left( \frac{ (-1)^j x^{2j-1} }{ (j!)(j+1)! 2^{2j+1} } \right) \left( (2j+1)(2j+1) + (-1)(2j)(2j+2) - 1\right) = 0
\end{aligned}
\end{gathered}
\]

\exercisehead{12} $y' = x^2 + y^2$ ; \quad \, $y= 1$ when $x=0$.  
\[
\begin{gathered}
 y'(0) = 0 +1^2 = 1   \quad \quad \, 
  \begin{gathered}
    y = a_0 + a_1 x + a_2 x^2 + \sum_{j=3}^{\infty} a_j x^j \\
    \begin{aligned}
      y^2 & = a_0^2 + a_1^2 x^2 + a_2^2 x^4 + \left( \sum_{j=3}^{\infty} a_j x^j \right)^2 + \\ 
      & + 2 a_0 a_1 x + 2 a_0 a_2 x^2 + 2a_0 \sum_{j=3}^{\infty} a_j x^j + \\ 
      & + 2a_1 a_2 x^3 + 2a_1 \sum_{j=3}^{\infty} a_j x^{j+1} + 2a_2 \sum_{j=3}^{\infty} a_j x^{j+2}
    \end{aligned}
  \end{gathered} \\
  y' = a_1 + 2a_2 x + \sum_{j=3}^{\infty} j a_j x^{j-1} \\
  \boxed{ a_1 = 1  } \text{ since } y'(0) = 1 \\
  \text{ Consider the first few terms of $x^2 + y^2$ } \\
  a_0^2 + 2a_0 a_1 x + a_1^2 x^2 + 2a_0 a_2 x^2 + x^2  = a_1 + 2a_2 x + 3 a_3 x^2 \quad \Longrightarrow \begin{gathered}
    a_1 = 1 = a_0^2 \Longrightarrow \boxed{ a_0 = 1 } \\
    2a_2 = 2a_0 a_1 \Longrightarrow \boxed{ a_2 = 1 } \\
    3a_3 = a_1^2 + 2a_0 a_2 + 1 = 4 \quad \Longrightarrow \boxed{ a_3 = \frac{4}{3} }
\end{gathered}
\end{gathered}
\]

\exercisehead{13} $y' = 1 + xy^2$ with $y=0$ when $x=0$  $\quad \quad \, \Longrightarrow a_0 = 0$  
\[
\begin{gathered}
  \begin{aligned}
    y & = \sum_{j=1}^{\infty} a_j x^j \\
    y & = \sum_{j=1}^{\infty} j a_j x^{j-1} = \sum_{j=0}^{\infty} (j+1)a_{j+1} x6j 
  \end{aligned} \quad \quad \, 
\begin{aligned}
  y & = a_1 x + a_2 x^2 + a_3 x^3 + \sum_{j=4}^{\infty} a_j x^j \\
  y^2 & = a_1^2 x^2 + a_2^2 x^4 + a_3 x^6 + \left( \sum_{j=4}^{\infty}a_j x^j \right)^2 + \\
  & + 2 a_1 a_2 x^3 + 2 a_1 a_3 x^4 + 2a_1 \sum_{j=4}^{\infty} a_j x^{j+1} + \\
  & + 2a_2 a_3 x^5 + 2a_2 \sum_{j=4}^{\infty} a_j x^{j+2} + 2a_3 \sum_{j=4}^{\infty} a_j x^{j+3} 
\end{aligned} \\
\boxed{ a_1 = 1  } \\
\begin{aligned}
  & x: \quad 2a_2 = 0 \quad \Longrightarrow a_2 = 0  
  & x^2: \quad 3 a_3 = 0 \quad \Longrightarrow a_3 = 0 \\
  & x^3: \quad 4a_4 = 1^2 \quad \Longrightarrow \boxed{ a_4 = \frac{1}{4} }  \\
  & x^4: \quad 5a_5 = 0 \quad \Longrightarrow a_5 = 0 
  & x^5: \quad 6a_6 = 0  \quad \Longrightarrow a_6 = 0 \\
  & x^6: \quad 7a_7 = 2a_1 a_4 + 2a_2 a_3 \quad \Longrightarrow \boxed{ a_7 = \frac{1}{14} } \\
  & x^7: \quad 8a_8 = 0 + 2a_2 a_4 = 0 \quad \Longrightarrow a_8 = 0 
  & x^8: \quad 9a_9 = 0 \quad \Longrightarrow a_9 = 0 \\
  & x^9: \quad 10a_{10} = \left( \frac{1}{4}\right)^2 + 2 (1) \frac{1}{14} \quad \Longrightarrow \boxed{ a_{10} = \frac{23}{1120} } \\
\end{aligned}
\end{gathered}
\]

\exercisehead{14} $y' = x + y^2$ \quad \, $y = 0 \text{ when } x = 0$ \quad \, $\Longrightarrow a_0 = 0 $ \\
$y'(0) = 0+ 0 = 0 \quad \Longrightarrow a_1 = 0$ 
\[
\begin{gathered}
  y = \sum_{j=2}^{\infty} a_j x^j  \quad \quad \, y' = \sum_{j=2}^{\infty} j a_j x^{j-1} = \sum_{j=1}^{\infty} (j+1)a_{j+1} x^j \\
  \begin{aligned}
    y^2 & = \left( a_2 x^2 + a_3 x^3 + a_4 x^4 + \sum_{j=5}^{\infty} a_j x^j \right)^2 = \\
    & = a_2 x^4  + a_3^2 x^6 + a_4^2 x^8 + \left( \sum_{j=5}^{\infty} a_j x^j \right)^2 + \\
    & + 2 a_2 a_3 x^5 + 2a_2 a_4 x^6 + 2 a_2 x^2 \sum_{j=5}^{\infty} a_j x^j + 2 a_3 a_4 x^y + 2 a_3 x^3 \sum_{j=5}^{\infty} a_j x^j + 2 a_4 x^4 \sum_{j=5}^{\infty} a_j x^j 
  \end{aligned} \\
y' = x + y^2 \\
\begin{aligned}
  & x: 2a+2 = 1 + 0 \Longrightarrow \boxed{ a_2 = \frac{1}{2} } \\
  & x^2: 3a_3 = 0 \Longrightarrow a_3 = 0 
  & x^3: 4a_4 = 0 \Longrightarrow a_4 = 0 \\
  & x^4: 5a_5 = a_2^2 \Longrightarrow \boxed{ a_5 = \frac{1}{20} }
  & x^5: 6a_6 = 0 \Longrightarrow a_6 = 0 
  & x^6: 7a_7 = 0 \Longrightarrow a_7 = 0 \\
  & x^7: 8a_8 = 2(\frac{1}{2} )(\frac{1}{20} ) \Longrightarrow \boxed{ a_8 = \frac{1}{160} } \\
  & x^8: 9a_9 = 0 \Longrightarrow a_9 = 0 
  & x^9: 10a_{10} = 0 \Longrightarrow a_{10} = 0 \\
  & x^{10}: 11a_{11} = 2(\frac{1}{2})(\frac{1}{160}) + \left( \frac{1}{20} \right)^2 \Longrightarrow \boxed{ a_{11} = \frac{ 7}{8800} }
\end{aligned}
\end{gathered}
\]

\exercisehead{15} $y' =\alpha y$ \quad \\
$\Longrightarrow \sum_{j=0}^{\infty} (j+1)a_{j+1}x^j = \alpha \sum_{j=0}^{\infty} a_j x^j $ \medskip \\
\quad \quad \quad $a_{j+1} = \frac{ \alpha a_j}{ (j+1) }$ \medskip \\
\quad \quad \quad $\boxed{ a_j = \frac{\alpha^j}{j!} a_0 }$x (by induction) 

\exercisehead{16} $y'' = xy$
\[
\begin{gathered}
\begin{aligned}
  y'' & = \sum_{j=0}^{\infty} (j+2)(j+1) a_{j+2} x^j  = \\
  & = 2a_2 + \sum_{j=0}^{\infty} (j+3)(j+2)a_{j+3} x^{j+1}
\end{aligned} = \sum_{j=0}^{\infty} a_j x^{j+1} \\
\Longrightarrow a_2 = 0 \text{ and } a_{j+3} = \frac{a_j}{(j+3)(j+2) } \\
\begin{aligned}
  & j =0  \quad  \, & a+3 = \frac{a_0}{3\cdot 2 } \\
  & j = 3 \quad  \, & a_6 = \frac{a_3}{6\cdot 5 }
\end{aligned} \quad \quad \, 
\begin{aligned}
  & j= 1  \quad  \, & a_4 = \frac{a_1}{4\cdot 3 } \\
  & j =4  \quad \, & a_7 = \frac{a_1}{ 7\cdot 6 \cdot 4 \cdot 3 }
\end{aligned} \\
\boxed{ a_{3j} = \frac{ a_0}{ (3j)! } \prod_{k=0}^{j-1} (3k+1) } ; \quad \quad \, \boxed{ a_{3j+1} = \frac{a_1}{(3j+1)!} \prod_{k=0}^{j-1} (3k+2) }
\end{gathered}
\]

\exercisehead{17} $y'' + xy' +y = 0$
\[
\begin{gathered}
  \begin{aligned}
    y &  = \sum_{j=0}^{\infty} a_j x^j \\
    y' & = \sum_{j=1}^{\infty} j a_j x^{j-1} \\
    y'' & = \sum_{j=2}^{\infty} j (j-1) a_j x^{j-2} \\
    & = \sum_{j=0}^{\infty} (j+2)(j+1) a_{j+2}x^j 
  \end{aligned} \quad \quad \, y'' + xy' + y = \sum_{j=1}^{\infty} x^j ((j+2)(j+1)a_{j+2} + ja_j + a_j ) = 0 \Longrightarrow a_{j+2} = \frac{-a_j}{(j+2) } \\
  \begin{aligned}
    a_2 & = \frac{-a_0}{2} \\
    a_4 & = \frac{-a_2}{4}
  \end{aligned} \quad \quad \, \begin{aligned}
    a_3 & = \frac{-a_1}{3} \\
    a_5 & = \frac{-a_3}{5} = \frac{a_1}{15}
  \end{aligned} \\
  \boxed{ a_{2j} = \frac{ (-1)^j a_0}{(2j)!! } \quad \quad \, a_{2j+1} = \frac{ a_1 }{ (2j+1)!!}(-1)^j } \text{ (could be shown by induction) }
\end{gathered}
\]

\exercisehead{18} Recall that \[
\begin{aligned}
  y & = \sum_{j=0}^{\infty} a_j x^j \\
  y' & = \sum_{j=1}^{\infty} j a_j x^{j-1} \\
  & = \sum_{j=0}^{\infty} (j+1)a_{j+1} x^j
\end{aligned}
\]
Knowing this, we could \emph{cleverly observe} that $e^{-2x} = \sum_{j=0}^{\infty} (2a_j + (j+1)a_{j+1})x^j$ is actually a \emph{differential equation}!!!
\[
\Longrightarrow e^{-2x} = y' + 2y
\]
Solving this ODE using $y(x) = e^{-A(x)} \left( \int_a^x Q(t) e^{A(t)} dt + y(a) \right)$ where $A(x) = \int_a^x P(t)dt$, 
\[
\boxed{ y = e^{-2x}(x+1) }
\]
We had obtained the necessary initial conditions to solve this ODE from the information given, that $a_0 = 1$, so that $y(0)=1$.  

By doing some simple computation and comparison of powers with $e^{-2x}$, then $a_1  =2$, $a_2 = -2$, $a_3 = 4/3$

\exercisehead{19} $\cos{x} = \sum_{j=0}^{\infty} a_j (j+2) x^j$ for $f(x) = \sum_{j=0}^{\infty} a_j x^j$.  \bigskip \\

Using $\cos{x} = \sum_{j=0}^{\infty} \frac{(x)^{2j}}{ (2j)!} (-1)^j $ representation, we can immediately conclude that for odd terms, $a_{2j+1} = 0$ and by matching powers of $x$,
\[
\begin{aligned}
  a_{2j}(2j+2) & = (-1)^j \frac{1}{(2j)!} \\
  a_5 & = 0 \\
  a_6 (6+2) & = \frac{(-1)^3 }{6!} \Longrightarrow \boxed{ a_6 = \frac{-7}{8!}}
\end{aligned}
\]
Now notice that for $\cos{x} = \sum_{j=0}^{\infty} a_j (j+2) x^j = \sum_{j=1}^{\infty}ja_j x^j + 2 \sum_{j=0}^{\infty} a_j x^j$ is actually a differential equation, $\cos{x} = xy' + 2y$.  We can solve this first-order ODE using \medskip \\
$y(x) = e^{-A(x)} \left( \int_a^x Q(t) e^{A(t)} dt + y(a) \right)$ where $A(x) = \int_a^x P(t)dt$.  Then solving $y' + \frac{2y}{x} =\frac{\cos{x}}{x} $, 
\[
y = \frac{1}{x^2} ( x \sin{x} + \cos{x} - (a\sin{a} + \cos{a}) + b) 
\]
Plugging $0$ as a good guess back into the ODE, $\cos{0} = 1 = y(0) (2)$ \quad $\Longrightarrow y(0) = \frac{1}{2}$  With this initial condition, we get
\[
\boxed{f(x) = \frac{ \sin{x}}{x} + \frac{ \cos{x} - 1 }{x^2} } \text{ if } x \neq 0 
\]
So $f(0)=\frac{1}{2}$ and $f(\pi) = \frac{-2}{\pi^2 }$ 

\exercisehead{20}
\begin{enumerate}
\item 
\[
\begin{aligned}
  (1-x)^{-1/2} & = \sum_{j=0}^{\infty} \binom{ -1/2}{j} (-x)^j  = \\
  & = 1 + \frac{1}{2}x + \frac{ \left( \frac{-1}{2} \right) \left( \frac{-3}{2} \right) }{2} x^2 + \frac{ \left( \frac{-1}{2} \right) \left( \frac{-3}{2} \right) \left( \frac{-5}{2} \right) }{ 3! } x^3 + \frac{ \left( \frac{-1}{2} \right) \left( \frac{-3}{2} \right) \left( \frac{-5}{2} \right) \left( \frac{-7}{2} \right) }{ 4! } x^4 + \\
  & \quad + \frac{ \left( \frac{-1}{2} \right) \left( \frac{-3}{2} \right) \left( \frac{-5}{2} \right) \left( \frac{-7}{2} \right) \left( \frac{-9}{2} \right)}{5! } x^5 + \dots = \\
  & = 1 + \frac{1}{2} x + \frac{3}{8} x^2 + \frac{5}{8} x^3 + \frac{35}{128} x^5 + \frac{63}{256} x^5 + \dots
\end{aligned}
\]
\item To make the notation clear, $(1-x)^{-1/2} = \sum_{j=0}^{\infty} \binom{ -1/2}{j} (-x)^j = \sum_{j=0}^{\infty} b_j x^j = \sum_{j=0}^{\infty} a_j$ \medskip \\
Now
\[
\frac{ \binom{\alpha}{j+1} }{ \binom{\alpha}{j} } = \frac{ \alpha (\alpha -1) \dots ( \alpha - (j+1)+1) }{ (j+1)! } \frac{j!}{ \alpha(\alpha - 1 ) \dots (\alpha -j +1) } = \frac{ (\alpha - j)}{ (j+1) }
\]
So for $\alpha = \frac{-1}{2}$, 
\[
\frac{ a_{j+1}}{a_j} = - \left( \frac{ 1/2 + j }{ j+1} \right) (-x) < x 
\]
Using this, we further find that 
\[
\begin{gathered}
  b_{j+1} < b_j \frac{1}{50} \\
b_{j+2} < b_{j+1} \frac{1}{50} < b_j \left( \frac{1}{50} \right)^2 
\end{gathered}
\]
For  $x = \frac{1}{50}$.  So by induction, $b_{n+j} < b_n \left( \frac{1}{50} \right)^j$ 
\[
\begin{gathered}
  r_n = \sum_{j=1}^{\infty} a_{n+j} < \sum_{j=1}^{\infty} a_n \left( \frac{1}{50} \right)^j = a_n \frac{ 1/50}{ 1 - 1/50} = \frac{ a_n}{49} \\
\boxed{  r_n < \frac{a_n}{49} }
\end{gathered}
\]
\item Note that $\left( 1 - x \right)^{-1/2} = \left( 1 - \frac{1}{50} \right)^{-1/2} = \left( \frac{49}{50} \right)^{-1/2} = \frac{5\sqrt{2}}{7} $
\[
\begin{gathered}
  \begin{aligned}
    \frac{7}{5} \left( 1 - \frac{1}{50} \right)^{-1/2} & = 1 + \frac{1}{100} + \frac{3}{2} \left( \frac{1}{2(50)} \right)^2 + \frac{5}{2} \left( \frac{1}{ 2(50)} \right)^3 + \frac{35}{8} \left( \frac{1}{2 (50)} \right)^4 + \frac{63}{8} \left( \frac{1}{ 2(50) } \right)^5 
  \end{aligned}\\ 
\boxed{  \sqrt{2} \simeq 1.4142135624 }
\end{gathered}
\]
\end{enumerate}

\exercisehead{21}
\begin{enumerate}
\item $\frac{1732}{1000} \left( 1 - \frac{176}{3000000} \right)^{-1/2} = \frac{ 1732}{1000} \left( \frac{ 3000000}{2999824} \right)^{1/2}$ \\
Obviously, $(3000000)^{1/2} = 1000\sqrt{3}$ so that we have $1732 \left( 3/2999824 \right)^{1/2}$.  \\
With long multiplication, we could show easily that $1732*1732 = 2999824$ (it's harder to divide).  So then
\[
\boxed{ \frac{1732}{1000} \left( 1 - \frac{ 176}{3000000} \right)^{-1/2} = \sqrt{3} }
\]
\item 
\end{enumerate}

\exercisehead{22} $\arcsin{x} = \int \frac{1}{ \sqrt{ 1 - x^2} }$ \medskip \\
$(1-x^2)^{-1/2} = \sum_{j=0}^{\infty} \binom{\alpha}{j}(-x^2)^j = 1 + \sum_{j=1}^{\infty} \binom{\alpha}{j} (-x^2)^j$ 
\[
\begin{gathered}
  \Longrightarrow \arcsin{x} = x + \sum_{j=1}^{\infty} \binom{ -1/2}{j} \frac{ (-1)^j}{(2j+1)} x^{2j+1} \\
  \frac{ \left( \frac{-1}{2} \right) \left( \frac{-3}{2} \right) \dots \left( \frac{-1}{2} - j +1 \right) }{ j (j-1) \dots (2)(1) } = (-1)^j\frac{ (1)(3) \dots ( 1 + 2j - 2)}{ (2j)!!} = (-1)^j\frac{(2j-1)!!}{(2j)!!} \\
  \Longrightarrow \boxed{ \arcsin{x} = x + \sum_{j=1}^{\infty} \frac{ (2j-1)!!}{(2j)!!} \frac{x^{2j+1}}{2j+1} }
\end{gathered}
\]


%-----------------------------------%-----------------------------------%-----------------------------------
\subsection*{ 12.4 Exercises - Historical introduction, The vector space of $n$-tuples of real numbers, Geometric interpretation for $n \leq 3$ } 
%-----------------------------------%-----------------------------------%-----------------------------------
\quad \\
\exercisehead{1} \quad \\ $\begin{aligned}
  A = & (1,3,6) \\
  B = & (4,-3,3) \\
  C = & (2,1,5) 
\end{aligned}$
\begin{enumerate}
\item $(5,0,9) = A+B$
\item $(-3,6,3)$
\item $(3,-1,4)$
\item $7A - 2B - 3C = (7,21,42) - (8,-6,6)-(6,3,15) = (-1,26,36) - (6,3,15) = (-7,24,21)$
\end{enumerate}

\exercisehead{2} See sketches.

\exercisehead{3} See sketches.  

\exercisehead{4}
\begin{enumerate}
\item See the sketch.  
\item I think it'll be a line.  
\item It'll fill a parallelogram with $A$ and $B$ as the sides.  
\item It'll fill the entire real, 2-dim. plane.  
\end{enumerate}

\exercisehead{5} \quad \\ 
$\begin{aligned}
  & A = (2,1) \\
  & B = (1,3) \\
  & C = (c_1,c_2)
\end{aligned}$ \medskip \\
$xA + yB = (2x,x) + (y,3y) = (2x+y, x+3y) = (c_1,c_2)$  \quad \quad \, $\begin{aligned}
  x & = \frac{ c_2 - 3c_1}{-5} \\
  y & = \frac{ c_1 - 2c_2 }{-5}
\end{aligned}$

\exercisehead{6}  \quad \\ 
$\begin{aligned}
  A = & (1,1,1) \\
  B = & (0,1,1) \\
  C = & (1,1,0)
\end{aligned}$  \quad \quad $D = xA + yB + zC$
\begin{enumerate}
  \item $D = (x+z,x+y+z,x+y)$ 
  \item $0 = x+z = x+y+z \Longrightarrow y=0$  \\
    $x = 0 \quad z=0$
  \item $D = (1,2,3)$ \\
    $x+y+z = 2 = 3 + z$  \quad \quad $z=-1$ \\
    $x-1=1$  \quad $x=2$ \quad $y=1$
\end{enumerate}

\exercisehead{7} \quad \\ 
$\begin{aligned}
  A = & (1,1,1) \\
  B = & (0,1,1) \\
  C = & (2,1,1)
\end{aligned}$  \quad \quad $D = xA + yB + zC$
\begin{enumerate}
  \item $D = (x+2z, x+y+z,x+y+z)$
  \item $x=2, \, z=-1, \, y = -1$ and $D = 0$ 
  \item $x+y+z =2$ but $x+y+z=3$
\end{enumerate}

\exercisehead{8} \quad \\
$\begin{aligned}
  A = & (1,1,1,0) \\
  B = & (0,1,1,1) \\
  C = & (1,1,0,0)
\end{aligned}$  \quad \quad $D = xA + yB + zC$
\begin{enumerate}
  \item $D = (x+z,x+y+z,x+y,y)$
  \item $y=0 \Longrightarrow x=0 \Longrightarrow z=0$
  \item $(1,5,3,4) = D$ $\Longrightarrow y = 4, \, x = -1, \, z=2$
  \item $(1,2,3,4) = D$ $\Longrightarrow y= 4, \, x = -1, \, z = -1 $ but $x+z=1$ 
\end{enumerate}

\exercisehead{9} \quad \\
$A,B$ parallel $\Longrightarrow A = tB$ \quad \quad $t \neq 0$ \\
$C,B$ parallel $\Longrightarrow C = sB$ \quad \quad $s \neq 0$ 

$A = t B = \frac{ tC}{s}$ and $\frac{t}{s} = k \neq 0$.  So $A$, $C$ are parallel.  

\exercisehead{10} Given \\
$\begin{aligned}
  & C = A +B \\
  & A = sD
\end{aligned}$ \\
If $C \parallel D$ so that $C = tD$ \\
\quad \quad $C- A = (t-s)D = B$.  If $t-s \neq 0$ then $B \parallel C$.  Even if $t=s$, then that meant that $C=A$ and $B=0$ anyways.   \\
If $B \parallel D$, so that $B = uD$ \\
\quad \quad $c = (s+u)D \Longrightarrow C \parallel D$  

\exercisehead{11}  
\begin{enumerate}
\item Theorem 12.1 says that vector addition is commutative and associative.  

Let $A,B \in V_n$; $A = (a_1,\dots, a_n)$, $B = (b_1, \dots, b_n)$

Use this definition,
\begin{definition} \quad \\
  $A= B$ iff $a_1 =b_1 \dots a_n = b_n$ \\
  $A+B = (a_1 + b_1, \dots, a_n + b_n)$ \\
  $cA = (ca_1 , \dots , ca_n)$ 
\end{definition}

\[
\begin{gathered}
  A + B  = (a_1 + b_1, \dots, a_n + b_n) = (b_1 + a_1 , \dots , b_n + a_n) = B+A \\
  \begin{aligned}
    A+ (B+C) & = (a_1 + (b_1 +c_1) , \dots, a_n + (b_n + c_n) ) = ((a_1 +b_1) +c_1 , \dots, (a_n +b_n) + c_n) = \\
    & = (A+B) + C 
  \end{aligned} \\
  c(dA) = (c(da_1), \dots, c(da_n) ) = ((cd)a_1, \dots, (cd)a_n) = (cd) A \\
  c(A+B) = (c(a_1+ b_1), \dots , c(a_n+b_n)) = (ca_1 + cb_1, \dots , ca_n + cb_n) = cA + cB \\
  (c+d)A = ((c+d)a_1, \dots, (c+d)a_n) = (c a_1 + da_1 , \dots, ca_n +da_n) = cA + dA
\end{gathered}
\]

So commutativity and associativity follows from the associativity, commutativity, and distributivity of the one-dim. reals (or even complex numbers).  
\item See sketch.  
\end{enumerate}

\exercisehead{12}  \quad \\
$
\begin{aligned}
  A = & B- C  \\  
  C = & B- A  \\
  B = & A + C 
\end{aligned}$
\quad \quad 
$\begin{gathered}
  A + \frac{1}{2} C - \frac{1}{2} A = \\
  \begin{aligned}
    = & \frac{1}{2} A + \frac{1}{2} C = \frac{1}{2} (A +C ) \\
    = & \frac{1}{2} B
  \end{aligned}
\end{gathered}$

It reminds me of how the diagonals of a parallelogram bisect each other.  

%-----------------------------------%-----------------------------------%-----------------------------------
\subsection*{ 12.8 Exercises - The dot product, Length or norm of a vector, Orthogonality of vectors } 
%-----------------------------------%-----------------------------------%-----------------------------------
\quad \\
\exercisehead{1} \quad \\
$\begin{aligned}
  & A = (1,2,3,4)  \\
  & B = (-1,2,-3,0)  \\
  & C = (0,1,0,1)
\end{aligned}$
\begin{enumerate}
  \item $A\cdot B = -6$
  \item $B\cdot C = 2$
  \item $A\cdot C = 6$
  \item $A\cdot ( B+C) = A \cdot ( -1,3,-3,1) = 0$
  \item $(A-B)\cdot C = (2,0,6,4)\cdot C = 4$
\end{enumerate}

\exercisehead{2} \quad \\
$
\begin{aligned}
  & A = (2,4,-7) \\
  & B = (2,6,3) \\
  & C = (3,4,-5)
\end{aligned}
$
\begin{enumerate}
  \item $(A\cdot B)C = 7(3,4,-5) = (21,28,-35)$
  \item $A\cdot (B+C) = A\cdot (5,10,-2) = 64$
  \item $(A+B)\cdot C = (4,10,-4) \cdot C = 72 $
  \item $A(B\cdot C) = (30,60,-105)$
  \item $A/(B\cdot C) = \left( \frac{2}{15}, \frac{4}{15}, \frac{-7}{15} \right)$
\end{enumerate}

\exercisehead{3} Given $A \cdot B = A \cdot C$, $A \neq 0$, \\
Then $\| A \| \| B \| \cos{ \theta_{AB} } = \| A \| \| C \| \cos{ \theta_{AC} } $ \\
Let $\| B \| = \| C \|$ but $\theta_{AB} = -\theta_{AC}$.  Then $B \neq C$.  

\exercisehead{4}  Given $A\cdot B = 0$ \quad $\forall \, A$, \\
$ \| A \| \| B \| \cos{ \theta_{AB}} = 0$, choose $\| A \| \neq 0$, \quad $ \cos{ \theta_{AB}} \neq 0$ \\
$ \| B \| =0$

\exercisehead{5} \quad \\
$\begin{aligned}
  & A = (2,1,-1) \\
  & B = (1,-1,2)
\end{aligned}$; find a nonzero $C$ in $V_3$ s.t. $A\cdot C = B\cdot C = 0$  \smallskip \\
$C = (c_1,c_2,c_3)$ without loss of generality, let $c_1 =1$ \smallskip \\
$\begin{aligned}
  &  2 + c_2 - c_3  = 0  \\ 
  & 1 + -c_2 +2c_3 = 0 
\end{aligned}$ \quad $ \Longrightarrow \begin{aligned}
 & c_2 = -5 \\
 &  c_3 = -3 
\end{aligned}$

\exercisehead{6}  \quad \\
$\begin{aligned}
  & A = (1,-2,3) \\
  & B = (3,1,2)
\end{aligned}$ \quad $C = xA + yB$ \quad \quad $\begin{aligned}
  & B \cdot B = 14 \\
  & C \cdot B = 0 
\end{aligned}$ \bigskip \\
$\begin{aligned}
  C \cdot B = (xA + yB) \cdot B = x (A \cdot B) + y B\cdot B & = 0  \\
  x (7) + y (14) & = 0 \Longrightarrow x + 2y = 0 \\
  y = -1, \, x = 2 \quad \quad C = (-1,-5,4) 
\end{aligned}$

\exercisehead{7} \quad \\
Given $\begin{aligned}
  & A = (2,-1,2) \\
  & B = (1,2,-2)
\end{aligned}$ \quad \quad $\begin{aligned}
  & A = C+D \\
  & B\cdot D = 0 
\end{aligned}$ \quad \quad $C \parallel B$ \quad \quad $B = tC$ \medskip \\
$tC \cdot D = 0 $ \quad \quad $t \neq 0 $ \\
$A \cdot B = B\cdot C = -4$  \\
$B \cdot B = tB \cdot C = 9$ \quad \quad \, $ \Longrightarrow t = 9/-4$  \\
$C = \frac{-4}{9} (1,2,-2) $ \\
$D = A - C = (2,-1,2) - \left( \frac{4}{9},\frac{8}{9},\frac{-8}{9} \right) = \left( \frac{22}{9}, \frac{-1}{9}, \frac{10}{9} \right)$

\exercisehead{8} Given that if \quad \\
$\begin{aligned}
  & A = (1,2,3,4,5) \\
  & B = (1,1/2,1/3,1/4,1/5)
\end{aligned}$ \quad \quad $C,D \in V_5$  \\
$B = C + 2D$ \quad $D \cdot A = 0 $, \quad $C \parallel A$; \quad $C = tA$ 

$\Longrightarrow B = tA + 2D$ \\
$A\cdot B = 5 = tA \cdot A + 2 A \cdot D = t(55)$  \quad $t = 1/11$ \\
$C = \frac{1}{11} (1,2,3,4,5)$  \\
$D = \frac{1}{2} \left( B - \frac{A}{11} \right) = \frac{1}{2} \left( (1,1/2,1/3,1/4,1/5) - \left( \frac{1}{11}, \frac{2}{11}, \frac{3}{11}, \frac{4}{11}, \frac{5}{11} \right) \right)$  \\
$D = \left( \frac{5}{11}, \frac{7}{44}, \frac{1}{33}, \frac{-5}{88}, \frac{-7}{55} \right)$

\exercisehead{9} \quad \\
$\begin{aligned}
  & A = (2,-1,5) \\
  & B = (-1,-2,3) \\
  & C = (1,-1,1)
\end{aligned}$ \quad \quad $\begin{aligned}
  & \| A+ B \| = \| (1,-3,8) \| = \sqrt{74} \\
  & \| A - B \| = \| (3,1,2) \| = \sqrt{14} \\
  & \| A + B - C \| = \| (0,-2,7) \| =\sqrt{53} \\
  & \| A - B  + C \| = \| (4,0,3) \| = 5 
\end{aligned}$

\exercisehead{10} $B \cdot A = 0$; \quad $ \| B \| = \| A \| $ 
\begin{enumerate}
  \item $B = (1,-1)$ 
  \item $A = (1,-1)$, $B=(1,1)$ 
  \item $A = (2,-3)$, $B=  (3,2)$ 
  \item $A = (a,b)$, $B = (b,-a)$
\end{enumerate}

\exercisehead{11} \quad \\
$\begin{aligned}
  & A = (1,-2,3) \\
  & B = (3,1,2) 
\end{aligned}$
\begin{enumerate}
  \item $A+ B = (4,-1,5)$  \quad \quad $\Longrightarrow \boxed{ C = \frac{ (4,-1,5) }{ \sqrt{42} } }$  
  \item $A - B = (-2,-3,1)$ \quad \quad $\boxed{ C = \frac{ (-2,-3,1) }{ \sqrt{14} } }$ 
  \item $ A + 2B = (7,0,7)$ \quad \quad $\Longrightarrow \boxed{ C = (1/\sqrt{2}, 0, 1/\sqrt{2} ) }$ 
  \item $2A - B = (-1,-5,4)$ \quad \quad $\boxed{ C = \frac{ (-1,-5,4) }{ \sqrt{42} } }$  
\end{enumerate}

\exercisehead{12}  \quad \\
$\begin{aligned}
  & A = (4,1,-3) \\
  & B = (1,2,2) \\
  & C = (1,2,-2) \\
  & D = (2,1,2) \\
  & E = (2,-2,1) 
\end{aligned}$ \quad \quad $\boxed{ \begin{aligned}
  & A \cdot B = 0 \\ 
  & C \cdot D = 0 \\
  & C \cdot E = 0 \\
  & D \cdot E = 0 
\end{aligned} }$

\exercisehead{13} The answer is easy to see geometrically.  If you do the algebra, the condition of $\|A \| = \|B \|$ fixes the length of the vector and $A\cdot B$ orthogonality fixes the direction, up to $\pi$ or $180^{\circ}$.  
\begin{enumerate}
  \item $A= (1,2)$  $\Longrightarrow \boxed{ B = (-2,1), (2,-1) }$ 
  \item $A = (1,-2)$ $\Longrightarrow \boxed{ B = (2,1), (-2,-1) }$
  \item $A = (2,-1)$ $\Longrightarrow \boxed{ B = (1,2), (-1,-2) }$  
  \item $A = (-2,1)$ $\Longrightarrow \boxed{ B = (1,2), (-1,-2) }$ 
\end{enumerate}

\exercisehead{14} \quad \\
$\begin{aligned}
  & A = (2,-1,1) \\
  & B = (3,-4,-4) \\
  & B-A = (1,-3,-5)
\end{aligned}$ \quad \quad $
\begin{aligned}
  & A \cdot B = 6 \\
  & A+B = (5,-5,-3)
\end{aligned}$ \medskip \\
We have the condition that we have one right angle for a right triangle: $(A - C ) \cdot (C-B) = A \cdot C - C^2 - A\cdot B + B\cdot C = (A+B)\cdot C - C^2 -6 = 0$ \smallskip \\
$\begin{aligned}
  (5,-5,-3)\cdot C & = 6 + C^2 \\
  5c_1 - 5c_2 + 0 & = 6 + c_1^2 + c_2^2  \\
  \xrightarrow{c_1=2} 10 - 5c_2 & = 10 + c_2^2 
\end{aligned}$ \quad \quad $\Longrightarrow c_2 =-5$

We just need to find one $C$.  Let $c_3 = 0$ for $C = (c_1,c_2,c_3)$.  Then we get the above last two statements if we also choose $c_1=2$.  \smallskip \\
So we have $\boxed{ C = (2,-5,0) }$

So in summary, we get \\
$\begin{aligned}
  & B-A = (1,-3,-5) \\
  & A-C = (0,4,1) \\
  & C-B = (-1,-1,4)
\end{aligned}$ and indeed $\begin{aligned}
  & \| A- C \| = \sqrt{17} \\
  & \| C- B \| = \sqrt{18} \\
  & \| B-A \| = \sqrt{35} 
\end{aligned}$ and $(A-C) \cdot (C-B) = 0$, as a right triangle should.  

\exercisehead{15} \quad \\
$\begin{aligned}
  & A = (1,-1,2) \\
  & B = (2,1,-1)
\end{aligned}$ \quad \, $\Longrightarrow C = (1,-5,-3)$

\exercisehead{16} Given that \\
$\begin{aligned}
  & A = (1,2) \\
  & B = (3,4)
\end{aligned}$ \quad \quad $\begin{gathered}
  A = P+Q \\
  P \parallel B \Longrightarrow P = sB \\
  Q \perp B
\end{gathered}$ \medskip \\
$\Longrightarrow \boxed{ P = \frac{11}{25} (3,4) }, \quad \quad Q = A - P = \boxed{ \left( \frac{-8}{25}, \frac{6}{25} \right) } $

\exercisehead{17} Given $A = (1,2,3,4)$, \quad $B = (1,1,1,1)$, then for $A = P+Q$, $P \parallel B$, $Q \cdot B = 0$, \\
$\boxed{ P = \frac{5}{2} (1,1,1,1) }$, $Q = A- P = \boxed{ \left( \frac{-3}{2}, \frac{-1}{2}, \frac{1}{2}, \frac{3}{2} \right) }$

\exercisehead{18} Given \\
$\begin{aligned}
  & A = (2,-1,1) \\
  & B = (1,2,-1) \\
  & C = (1,1,-2)
\end{aligned}$ \quad \quad $\begin{gathered}
  D = x B + yC \\
  A\cdot D = 0 \\
  \| D \| = 1
\end{gathered}$ \medskip \\
then $D = \pm \frac{1}{\sqrt{2}} B + \mp \frac{1}{\sqrt{2}} C$

\exercisehead{19} $\| A + B \|^2 - \| A - B \|^2 = A^2 + 2A\cdot B + B^2 - (A^2 -  2A\cdot B + B^2) = 4 A\cdot B$

\exercisehead{20} $\| A + B \|^2 + \| A - B \|^2 = A^2 + 2A\cdot B + B^2 + (A^2 -  2A\cdot B + B^2) = 2A^2 + 2B^2$ 

The sum of the square of the sides of a parallelogram is equal to the sum of the square of the diagonals of a paralellogram.  

\exercisehead{21} \quad \\ 
``The sum of the squares of the sides of any quadrilateral exceeds the sum of the squares of the diagonals by four times the square of the length of the line segment which connects the midpoints of the diagonals.''
\[
\begin{aligned}
  &  
\begin{aligned} 
  \text{ Sum of the squares of the sides } & = A^2 + (C-A)^2 + (C-B)^2 + B^2  = \\
  & = A^2 + C^2 - 2AC + A^2 + C^2 - 2CB+B^2  + B^2 
\end{aligned} \\
& \begin{aligned}
  \text{ sum of the squares of the diagonals } & = C^2 + (A-B)^2  = \\
  & = C^2 + A^2 - 2AB + B^2 
\end{aligned} \\
& \begin{aligned}
  & \text{ four times the square of the length of the line segment which connects the midpoints of the diagonals } = \\ 
  & = 4 \left( \frac{C}{2} - \left( \frac{ A+B}{2} \right) \right)^2 = \\
  & C^2 - 2 C ( A+B) + A^2 + 2AB + B^2 
\end{aligned}
\end{aligned}
\]
So then if we take the \emph{ sum of the squares of the sides } - \emph{ sum of the squares of the diagonals } = \\
= \text{ four times the square of the length of the line segment which connects the midpoints of the diagonals}, then we get the desired equality.  

\exercisehead{22} Given that $\| A \| = 6$ and for any pair of scalars, $x,y$, \\
$(xA + yB) \cdot (4yA - 9xB ) = 4xyA^2 - 9x^2 A\cdot B + 4y^2 A\cdot B - 9xyB^2 = 0$ $\Longrightarrow 144xy + ( 4y^2 -9x^2 )A\cdot B = 9xyB^2 $ \medskip \\
\quad \quad \quad $x =2, \, y=3$ without loss of generality.  \\
\quad \quad \quad \quad $144 xy = 9xy B^2$ $\Longrightarrow 16 = B^2, \, \boxed{ \| B \| = 4 }$

This implies that $A\cdot B$ in order for the above statements to be true $\forall $ \, $x,y$

$ \| (2A + 3B) \|^2 = 4A^2 + 12 A\cdot B + 9B^2 = 144 + 144 = (12\sqrt{2})^2 $ $\Longrightarrow \| 2A + 3B \| = 12 \sqrt{2}$

\exercisehead{23} Given that \\
$\begin{aligned}
  & A = (1,2,3,4,5) \\
  & B = (1,1/2,1/3,1/4,1/5)
\end{aligned}$ \quad \quad $\begin{gathered}
  B = C+D \\
  D \perp A \\
  C \parallel A \text{ or } C = tA 
\end{gathered}$ 

$B\cdot A = 5 = C\cdot A + D \cdot A = tA^2 = 55 t$

$\boxed{ C = \frac{1}{11} (1,2,3,4,5) }$

$D = B-C = (1,1/2,1/3,1/4,1/5) - \frac{1}{11} (1,2,3,4,5) = \boxed{ \left( \frac{10}{11}, \frac{7}{22}, \frac{2}{33}, \frac{-5}{44}, \frac{-14}{55} \right) }$

\exercisehead{24} Given that \\
$\begin{gathered}
  A \cdot B \neq 0 \\
  C \parallel A \quad \quad C = tA \\
  D \perp A \text{ so that } A \cdot D = 0 \\
  B = C+D
\end{gathered}$ \quad since $\| A \| neq 0$, if $t \neq 0$, then $C$ exists.  

$A\cdot B = C \cdot A + D \cdot A = tA^2 $ \quad $\Longrightarrow t = \frac{A \cdot B}{A^2 }$ \quad $ \boxed{ C  = \frac{ A \cdot B}{A^2} A }$ \medskip \\
$\boxed{ D = B - C = B - \frac{ A \cdot B}{A^2 } A }$

\exercisehead{25} 
\begin{enumerate}
\item Given that $A \cdot B = 0$, $\| A + xB \|^2 = A^2 + 2xA \cdot B + x^2 B^2 = A^2 + x^2 B^2 \geq A^2 $ since $x^2 B^2 $ is positive.  
\item Given that $\| A + xB \| \geq \| A \|$, then $A^2 + 2x A \cdot B  + x^2 B^2 \geq A^2$.  \medskip \\
  \quad \quad $x (2A\cdot B + xB^2) \geq 0$
\[
\begin{gathered}
\begin{gathered}
  \text{ If } x > 0 \\
  2A \cdot B + xB^2 \geq 0 \\
  xB^2 \geq - 2A \cdot B \\
  \text{ then } A\cdot B \geq 0 \\
\end{gathered}
\quad \quad \quad 
\begin{gathered}
  \text{ If } x < 0 \\
  2 A \cdot B + xB^2 \leq 0 \\
  xB^2 \leq - 2A \cdot B \\
  \text{ then } A\cdot B \leq 0 
\end{gathered}  \\
\Longrightarrow A \cdot B = 0 
\end{gathered}
\]
We relied heavily upon the fact that $x$ was any number, so we could make $x$ as small or as big as we want.  
\end{enumerate}


%-----------------------------------%-----------------------------------%-----------------------------------
\subsection*{ 12.11 Exercises - Projections.  Angle between vectors in $n$-space, The unit coordinate vectors } 
%-----------------------------------%-----------------------------------%-----------------------------------
\quad \\

\exercisehead{1} Given \\
$\begin{aligned}
  & A = (1,2,3) \\
  & B = (1,2,2)
\end{aligned}$ \quad \quad $\begin{aligned}
  & A \cdot B = 11 \\
  & B \cdot B = 9
\end{aligned}$ \\
So the projection of $A$ on $B$ is $\boxed{ \frac{ A\cdot B}{B \cdot B} B = \frac{11}{9} (1,2,2) }$  

\exercisehead{2} Given \\
$\begin{aligned}
  & A = (4,3,2,1) \\
  & B = (1,1,1,1)
\end{aligned}$ \quad \quad $\begin{aligned}
 & B \cdot B =4 \\
  & A \cdot B = 10 
\end{aligned}$ \\
So the projection of $A$ on $B$ is $ \boxed{ \frac{5}{2} (1,1,1,1) }$

\exercisehead{3} 
\begin{enumerate}
  \item $A = (6,3,-2)$.  $\boxed{ \begin{aligned}
    & \cos{a} = \frac{6}{7} \\
    & \cos{b} = \frac{3}{7} \\
    &  \cos{c} = \frac{-2}{7}  
\end{aligned} }$ 
  \item $\frac{1}{7} (6,3,-2)$
\end{enumerate}

\exercisehead{4} Given \\
$\begin{aligned}
 & A = (1,2,1) \\
 & B = (2,1,-1)
\end{aligned}$ \quad \quad $\begin{aligned} 
  & C = (1,4,1) \\
  & D = (2,5,5) 
\end{aligned}$ \\
$\begin{aligned}
 & |A| = \sqrt{6} \\
  & |B| = \sqrt{6} 
\end{aligned}$ \quad \quad $\begin{aligned}
  & |C| = \sqrt{18} = 3 \sqrt{2} \\
  &  |D| = \sqrt{54} = 3 \sqrt{6}
\end{aligned}$ \quad \quad 
$\begin{aligned}
  & \frac{ A \cdot B}{ |A| |B| } = \frac{1}{2}  \Longrightarrow \theta_{AB} = \boxed{ \frac{\pi}{6} } \\
  &  \frac{ C \cdot D}{ |C||B| } = \frac{ \sqrt{3}}{2} \Longrightarrow \theta_{CD} = \boxed{ \frac{ \pi}{3} }
\end{aligned}$

\exercisehead{5} If we let  \\
$\begin{aligned}
  & A = (2,-1,1) \\
  & B = (1,-3,-5) \\
  & C = (3,-4,-4)
\end{aligned}$ \quad \quad $\begin{aligned}
  & B-A = x_1 = (-1,-2,-6) \quad \quad & |x_1| = \sqrt{41} \\
  & C-B = x_2 = (2,-1,1) \quad \quad & |x_2 | = \sqrt{6} \\
  & A-C = x_3 = (-1,3,5) \quad \quad & |x_3| = \sqrt{35} 
\end{aligned}$  \quad \quad $\begin{aligned}
  & x_1 \cdot x_2 = -6 \\
  & x_1\cdot x_3 = -35 \\
  & x_2 \cdot x_3 = 0
\end{aligned}$ \quad $\boxed{ \begin{aligned}
    & \theta_{12} = -6/(\sqrt{ 246 } ) \\
    & \theta_{13} = -35/(\sqrt{ 1435 } )  \\
    & \theta_{23} = 0 
\end{aligned} }$

\exercisehead{6} Given that $\| A \| = \| C \| =5$, \quad $\| B \| = 1$, $\| A - B + C \| = \| A + B + C \|$, and $\frac{ A \cdot B}{ |A||B| } = \cos{ \frac{ \pi }{8} }$, then consider 
\[
\begin{gathered}
  (A- B + C ) \cdot (A -B + C) = (A+B+C) \cdot (A+B+C) \\
  ((A+C) - B)^2 = (A+C+B)^2 \Longrightarrow -2(A+C)\cdot B = 2 (A+C)\cdot B \\
  0 = A\cdot B + B \cdot C \Longrightarrow -A \cdot B = B\cdot C \\
  \Longrightarrow \frac{ -5 (1) \cos{ \frac{ \pi }{8 } } }{ (1) 5 } = \cos{ \theta_{BC} } \Longrightarrow \boxed{ \theta_{BC} = \frac{ 9 \pi }{8 } } 
\end{gathered}
\]

\exercisehead{7} Given that $\frac{ A \cdot C}{ |A||C| } = \cos{ \theta_{AC} } = \frac{ B \cdot C}{ |B||C| } = \cos{ \theta_{BC} }$, then $\frac{ A \cdot C}{ |A| } = \frac{ B \cdot C}{|B| }$ or $|B|A\cdot C = |A|B\cdot C$.  \\
So $C \cdot (|B|A - |A| B) = |B| (A \cdot C) - |A| B\cdot C = 0 $

\exercisehead{8} Given that \\
$\begin{aligned}
  & A = (1,1,\dots, 1 ) \\
  & B = (1,2,\dots, n )
\end{aligned}$ then $\begin{aligned}
  & |A|^2 = n \\
  & |B|^2 = \sum_{j=1}^n j^2  \\
  & A\cdot B = \sum_{j=1}^n j = \frac{ n (n+1) }{2} 
\end{aligned}$ 

To find $\sum_{j=1}^n j^2$, recall this trick:
\[
\begin{gathered}
  S = \sum_{j=1}^n j^3 = \sum_{j=0}^{n-1} (j+1)^3 = \sum_{j=0}^{n-1} j^3 + 3j^2 + 3j + 1 = \\
  = S - n^3 + 3 \sum_{j=1}^{n-1} j^2 + 3 \left( \frac{ (n-1)n }{ 2 } \right) + n \\
  \Longrightarrow \sum_{j=1}^n j^2 = \frac{ (n+1)(2n+1)n}{6 } 
\end{gathered}
\]

So now we could get a remarkable limit:
\[
\cos{ \theta_{AB}} = \frac{ A \cdot B}{|A||B|} = \frac{ \frac{ n (n+1)}{2} }{ \sqrt{n} \sqrt{ \frac{ (n+1)(2n+1) n }{6} }} = \frac{ \sqrt{3}}{2} \frac{ 1 + 1/n}{ \sqrt{ (1+1/n)(1+1/2n ) } } \xrightarrow{ n \to \infty} \boxed{ \frac{ \sqrt{3}}{2} } \text{ or } \theta_{AB} = \pi/3
\]

\exercisehead{9} Given that \\
$\begin{aligned}
  & A = (2,4,6,\dots, 2n ) \\
  & B = (1,3,5, \dots, 2n-1) 
\end{aligned}$ \quad \quad then 
$\begin{aligned}
  & |A|^2 = \sum_{j=1}^n (2j)^2 \\
  & |B|^2 = \sum_{j=1}^n (2j-1)^2 \\
  & A\cdot B = \sum_{j=1}^n 2j(2j-1) 
\end{aligned}$

Doing the math,
\[
\begin{gathered}
  |A|^2 = \sum 4j^2 = 4 \sum_{j=1}^n j^2 = \frac{4 (n+1)(2n+1) n }{6} \\
  \begin{aligned} 
    |B|^2 & = \sum_{j=1}^n (2j-1)^2 =  \sum_{j=1}^n (4j^2 - 4j +1) = 4 \frac{ (n+1)(2n+1)n }{6} - 4 \frac{ n(n+1)}{2} + n = \\
    & = n \frac{ (4n^2 + 6n +2 - 6n -6 +3 ) }{3 } = \frac{ n (4n^2 - 1 ) }{3 } 
    \end{aligned} \\
  \begin{aligned}
    A\cdot B & = \sum_{j=1}^n 2j (2j-1) = \sum_{j=1}^n 4j^2 - 2 j = 4 \frac{ (n+1)(2n+1) n }{6} - 2 \frac{ n(n+1)}{2} = \\
    & = \frac{ ( 2 (2n^2 + 3n + 1 ) - 3 n -3 ) n }{ 3 } = \frac{ (4n^2 + 3n-1) n }{3 } 
  \end{aligned} \\
  \begin{aligned}
    \frac{ A \cdot B}{|A||B|} = \frac{ n (4n^2 + 3n-1)/3 }{ \sqrt{ \frac{ 2 ( 2n^2 + 3n+1) n }{3} \left( \frac{ n (4n^2 - 1 ) }{3 } \right) } } = \frac{ 4 + 3/n - 1 /n^2 }{ \sqrt{ 2 ( 2 + 3/n + 1/ n ) (4-  1/n^2 ) } } \xrightarrow{ n \to \infty} \frac{ 4 }{ \sqrt{ 2 (2) 4 } } = \boxed{ 1 } 
\end{aligned}
\end{gathered}
\]
It seems to make sense that $\theta_{AB} \xrightarrow{ n\to \infty} 0$, as the two vector getting closer and closer.  

\exercisehead{10}
\begin{enumerate}
\item $\begin{aligned}
  & A = (c\theta,-s\theta) \\
  & B = (s\theta, c\theta)
\end{aligned}$ \quad \quad $\begin{gathered}
  c^2 + s^2 = 1 \\
  A\cdot B = cs - cs = 0 
\end{gathered}$
\item $(x,y) \in V_2$, \\
  $(x,y) = xA + yB = x(C,-S) + y(S,C) = (xC + yS, -xS + yC)$  \quad \quad $\Longrightarrow \begin{aligned}
  & x = x C + y S \\
  & y = -xS + y C 
\end{aligned}$ \quad \, $\Longrightarrow \begin{aligned}
  & x(1-C) = yS \\
  & y(1-C) = -xS
\end{aligned}$ \\
If $\theta \neq 2 \pi j$ and $y\neq 0 $, then \\
$\begin{aligned}
  xy ( 1 - C ) & = y^2 S \\
  x(-xS) & = y^2 S  \\
  -x^2 S = y^2 S 
\end{aligned}$ 

If $\theta \neq \pi j$, then contradiction.  If $\theta = (2j-1) \pi$, $2x = 0, \, y =0$.  If $\theta = 2j\pi$, then $\forall \, x , y \in \mathbb{R}$  
\end{enumerate}

\exercisehead{11} For a rhombus, with adjacent sides $A,B$, $|A| = |B|$.  The diagonals are given by $A+B$ and $A-B$.  \\
$(A+B)\cdot (A-B) = A^2 - B^2 = 0$

\exercisehead{12} $(\cos{a}, \sin{a} ) \cdot (\cos{b},\sin{b}) = \cos{a}\cos{b} + \sin{a}\sin{b}$.  But $(\cos{a},\sin{a})\cdot (\cos{b},\sin{b}) = \cos{ a-b }/(1)(1)$ because the angle between the two vectors is $a-b$.  

\exercisehead{13} 
\[
\| A-B \|^2 = (A-B)\cdot(A-B) = A^2 + B^2 - 2 \|A \| \| B \| \cos{ theta} 
\]

\exercisehead{14} Given $A\cdot B = \sum_{k=1}^n |a_k b_k |$, 
\[
\begin{gathered}
  B\cdot A = \sum_{j=1}^n |b_j a_j | = \sum_{j=1}^n |a_j b_j| = A\cdot B \\
  A\cdot (cB) = \sum_{j=1}^n |a_j c b_j | = \sum_{j=1}^n |c a_j b_j | = (cA)\cdot B = |c| \sum_{j=1}^n |a_j b_j | = |c| (A\cdot B) \\
  A\cdot A = \sum_{j=1}^n |a_j|^2 \geq 0, \quad \, A \cdot A =0 \text{ only if } A = 0 
\end{gathered}
\]

To prove the Cauchy-Schwarz inequality for this metric, recall the proof for Thm. 1.41.  
\[
\begin{gathered}
  \sum_{j=1}^n (|a_j|x + |b_j|)^2 = \sum_{j=1}^n |a_j|^2 x^2 + 2 |a_j||b_j|x + |b_j|^2 = x^2 \sum_{j=1}^n |a_j|^2 + 2x \sum_{j=1}^n |a_j b_j | + \sum_{j=1}^n |b_j|^2 \geq 0 \\
  x^2 A' + 2xB' + C' \geq 0 \text{ where} \begin{aligned}
    & A' = \sum_{j=1}^n |a_j|^2 = A\cdot A \\
    & B' = \sum_{j=1}^n |a_j b_j| = A \cdot B \\
    & C' = \sum_{j=1}^n |b_j|^2 = B \cdot B
\end{aligned} \quad \Longrightarrow A' \left( x+ \frac{B'}{A'} \right)^2 + \frac{ A'C' - B'^2}{A'} \geq 0 \\
  \text{ Let } x = \frac{-B'}{A'} \\
  \Longrightarrow A'C' \geq B'^2 \text{ or } \boxed{ \sum_{j=1}^n |a_j|^2 \sum_{j=1}^n |b_j^2 | \geq (\sum_{j=1}^n |a_j b_j|)^2 }
\end{gathered}
\]

\exercisehead{15} Given $A\cdot B = 2a_1 b_1 + a_2 b_2 + a_1 b_2 + a_2 b_1 $
\[
\begin{gathered}
  B\cdot A = 2 b_1 a_1 + b_2 a_2 + b_1 a_2 + b_2 a_1 = A\cdot B = 2a_1 b_1 + a_2 b_2 + a_1 b_2 + a_2 b_1 \\
  \begin{aligned} 
    A\cdot B + A\cdot C & = 2a_1 b_1 + a_2 b_2 + a_1 b_2 + a_2 b_1 + 2a_1 c_1 + a_2 c_2 + a_1 c_2 + a_2 c_1 = \\
    &  = 2 a_1 (b_1 + c_1 ) + a_2 (b_2 +c_2) + a_1 (b_2 + c_2 ) + a_2 (b_1 +c_1) = A \cdot (B+C) 
  \end{aligned} \\
  \begin{aligned}
    (cA)\cdot B & = 2(ca_1)b_1 + ca_2 b_2 +ca_1 b_2 + ca_2 b_1 = 2a_1 (cb_1 ) + a_2 (cb_2) + a_1 (cb_2) + a_2 (cb_1) = \\
    & = c(2a_1 b_1 + a_2 b_2 + a_1 b_2 + a_2 b_1) = A \cdot (cB) = c(A\cdot B) 
\end{aligned} \\
  A\cdot A = 2a_1^2 + a_2^2 + a_1 a_2 + a_2 a_1 = a_1^2 + (a_1 + a_2)^2 \geq 0, \quad \, A\cdot A = 0 \text{ only if } \quad \begin{aligned}
    & a_1 = 0 \\
    & a_1 + a_2 = 0 
\end{aligned} \quad \Longrightarrow A = 0 
\end{gathered}
\]
For the Cauchy-Schwarz inequality, 
\[
\begin{gathered}
  A^2 B^2 = (a_1^2 + (a_1+a_2)^2 )(b_1^2 + (b_1+b_2)^2 ) \\
  \begin{aligned}
  (A\cdot B)^2 & = (a_1 b_1 +a_2(b_1 + b_2) + a_1(b_1+b_2))^2 = (a_1 b_1 + (a_1+a_2)(b_1+b_2))^2 = \\
    & = (a_1 b_1)^2 + 2a_1 b_1 (a_1 + a_2)(b_1+b_2) + (a_1+a_2)^2(b_1 +b_2)^2 
  \end{aligned} \\
  \quad \\
  \text{ Now } (a_1(b_1+b_2) - b_1(a_1+a_2))^2 \geq 0 \\
  a_1^2 (b_1+b_2)^2 - 2a_1 b_1(b_1+b_2)(a_1+a_2) + b_1^2 (a_1+a_2)^2 \geq 0 \\
  \text{ This fact shows that  } A^2 B^2 \geq (A\cdot B)^2 
\end{gathered}
\]

\exercisehead{16}
Given $A\cdot B = 2a_1 b_1 + a_2 b_2 + a_3 b_3 + a_1 b_3 + a_3 b_1 $
\[
\begin{gathered}
  B\cdot A = 2 b_1 a_1 + b_2 a_2 + b_3 a_3 + b_1 a_3 + b_3 a_1 = A\cdot B = 2a_1 b_1 + a_2 b_2 + a_3 b_3 + a_3 b_1 + a_1 b_3 \\
  \begin{aligned} 
    A\cdot B + A\cdot C & = 2a_1 b_1 + a_2 b_2 + a_3 b_3 + a_1 b_3 + a_3 b_1 + 2a_1 c_1 + a_2 c_2 + a_3 c_3 + a_1 c_3 + a_3 c_1 = \\
    &  = 2 a_1 (b_1 + c_1 ) + a_2 (b_2 +c_2) + a_3 (b_3 + c_3) + a_1 (b_3 + c_3 ) + a_3 (b_1 +c_1) = A \cdot (B+C) 
  \end{aligned} \\
  \begin{aligned}
    (cA)\cdot B & = 2(ca_1)b_1 + (ca_2) b_2 + (ca_3) b_3 + (ca_1) b_3 + (ca_3) b_1 = 2a_1 (cb_1 ) + a_2 (cb_2) + a_3(cb_3) + a_1 (cb_3) + a_3 (cb_1) = \\
    & = c(2a_1 b_1 + a_2 b_2 + a_3 b_3 + a_1 b_3 + a_3 b_1) = A \cdot (cB) = c(A\cdot B) 
\end{aligned} \\
  A\cdot A = 2a_1^2 + a_2^2 + a_3^2 + a_1 a_3 + a_3 a_1 = a_1^2 + a_2^2 + (a_1 + a_3)^2 \geq 0, \quad \, A\cdot A = 0 \text{ only if } \quad \begin{aligned}
    & a_1 = 0 \\
    & a_2 = 0 \\
    & a_1 + a_3 = 0 
\end{aligned} \quad \Longrightarrow A = 0 
\end{gathered}
\]
For the Cauchy-Schwarz inequality, 
\[
\begin{gathered}
  A^2 B^2 = (a_1^2 + a_2^2 + (a_1+a_3)^2 )(b_1^2 + b_2^2+ (b_1+b_3)^2 ) \\
  \begin{aligned}
  (A\cdot B)^2 & = (a_1 b_1 + a_2 b_2+ a_3(b_1 + b_3) + a_1(b_1+b_3))^2 = (a_1 b_1 + a_2 b_2 + (a_1+a_3)(b_1+b_3))^2 = \\
    & = (a_1 b_1)^2 + a_2^2 b_2^2 + 2a_1 b_1 a_2 b_2 + 2(a_1 b_1 +a_2 b_2)(a_1 + a_3)(b_1+b_3) + (a_1+a_3)^2(b_1 +b_3)^2 
  \end{aligned} \\
\end{gathered}
\]
As you could see, a direct proof has messy algebra.  Instead, since this dot product satisfies all the usual dot product properties, then we can just reuse the vector method proof of the Cauchy-Schwarz inequality, without reference to a specific coordinate system, to conclude that the Cauchy-Schwarz inequality holds.  \emph{That's the trick} - think in terms of generality, of vectors in general, and not specific coordinate systems.  

\exercisehead{17} Given that $\| A \| = \sum_{j=1}^n |a_j|$  
\begin{enumerate}
  \item 
\[
\begin{gathered}
  \sum |a_j | >0 , \quad \, \sum |a_j| = 0 \text{ if } a_j =0 \quad \forall \, j \\
  \| c A \| = \sum |c a_j | = |c| \sum |a_j | = |c| \| A \| \\
  \| A + B \| = \sum |a_j +b_j | < \sum |a_j | + |b_j | = \| A \| + \| B \| 
\end{gathered}
\]
  \item See sketch.  Looks like a diamond.  
  \item Given $\| A \| = \left| \sum_{j=1}^n a_j \right|$
\[
\begin{gathered}
  \| A \| > 0 \text{ if } A \neq 0 \, \text{ but } \| A \| = 0 \text{ for } a_j = (-1)^j \text{ and } n = 2 m \\
  \| c A \| = \left| \sum_{j=1}^n c a_j \right| = |c| \left| \sum_{j=1}^n a_j \right| = |c| \| A \| \\
  \| A +B \| = \left| \sum_{j=1}^n (a_j +b_j) \right| \leq \left| \sum_{j=1}^n a_j \right| + \left| \sum_{j=1}^n b_j \right| = \| A \| + \| B \| 
\end{gathered}
\]
\end{enumerate}

\exercisehead{18} Given that $\| A \| = \max_{1 \leq j \leq n } |a_j|$
\begin{enumerate}
\item $\| A \| \geq 0 $ since $|a_j| \geq 0$ for any $a_j \in \mathbb{R}$ \\
  If $A \neq 0 $ then $\exists$ \, at least one $a_j$ s.t. $a_j \neq 0$ \\
$|a_j| > 0$, so $\| A \| = 0$ only if $A=0$.  \\

$\| c A \| = \max_{ 1 \leq j \leq n} |ca_j | = \max_{ 1 \leq j \leq n } |c| |a_j|$ \\

Consider $a_J = \max_{1\leq j \leq n } |a_j|$ \\
$|a_J| \geq |a_j|$ \quad \, $\forall \, j \neq J$ \\
So $|c||a_J| \geq |c| |a_j| $ is valid.  $|c| |a_j| = |ca_j|$.  Then \\
$\max_{1 \leq j \leq n }|c| |a_j| = |c| \max_{1\leq j \leq n } |a_j|$   \\

$\| A +B \| = \max_{ 1 \leq j \leq n } |a_j + b_j | \leq \max_{1\leq j \leq n } |a_j | + |b_j| \leq \max_{1\leq j \leq n} |a_j| + \max_{1 \leq k \leq n } |b_k| = \| A \| + \| B \|$ \\
\item See sketch.  Looks like a square.  
\end{enumerate}

\exercisehead{19} Given $A = (a_1, \dots, a_n)$, $\|A \|_1 = \sum_{j=1}^n |a_j|$, \quad $\| A \|_2 = \max_{1 \leq j \leq n} |a_j|$, \quad $\| A \| = \sqrt{ \sum_{j=1}^n (a_j)^2 }$.  \\

Let $ \| A \|_2 = \max_{1 \leq j \leq n} |a_j | = |a_J|$.  \medskip \\
$\| A \|^2 = \sum_{j=1}^n a_j^2 $, \quad \, $\|A \|^2$ sum obvious includes $a_J$ term, and all $a_j^2$ terms are positive.  \medskip \\
$\Longrightarrow \|A \|^2 \geq \| A \|_2^2 $ or $\| A \| \geq \| A\|_2$  \\

$\left( \sum_{j=1}^n |a_j| \right)^2 \geq \sum_{j=1}^n (a_j)^2 $, because of the cross terms, which are positive from the absolute value function on each individual term.   \medskip \\
So then $\| A\|_1 \geq \|A \|$ \\

Note that we had used the fact that if $b^2 > a^2$ and $a,b>0$, then $b>a$.  This is because \\
If $a=b$, then $a^2 = b^2$.  Contradiction. \\
If $a <b$, then $a\cdot a < a\cdot b < b^2 $, from $ab < b^2$.  Contradiction.   \\

$\| A \|_2 \leq \| A \| \leq \| A \|_1$ means that $2$-norm will ``norm'' or ``expand'' a given coordinate pair the least, assign to it the smallest norm, so coordinates of bigger values are needed for $2$-norm.  In contrast, the $1$-norm assigns bigger values for the norm of a given pair of coordinates.  The $1$-norm requires smaller values for the coordinates to get the same norm value as the $2$-norm and usual spherical norm.  

Notice how going from $2$ norm, $\| A\|$, and $1$ norm, we go from a square, to a circle, and finally to a diamond.  

\exercisehead{20} Given that $d(A,B) = \| A - B \|$,
\[
\begin{gathered}
  d(B,A) = \| B - A \| = \| (-1) (A-B) \| = |-1| \| A - B \| = \|A - B \| \\
  d(A.B) = \| A - B \| = 0 \text{ only if } A - B = 0 \Longrightarrow A = B \\
  d(A,B) = \| A - B \| = \| A - C + C  - B \| \leq \|A - C \| + \|C - B \|  = d(A,C) + d(C,B) 
\end{gathered}
\]


%-----------------------------------%-----------------------------------%-----------------------------------
\subsection*{ 12.15 Exercises - The linear span of a finite set of vectors, Linear independence, Bases } 
%-----------------------------------%-----------------------------------%-----------------------------------

\exercisehead{1} $x(i-j) + y(i+j)$
\begin{enumerate}
  \item $i$; $x = 1/2$, $ y = 1/2$ \\
  \item $j$; $x = -1/2$, $y=1/2$  \\
  \item $3i - 5j$; $x=4$, $y=-1$  
  \item $7i + 5j$; $x=1$, $y=6$  
\end{enumerate}

\exercisehead{2} \quad \\
$\begin{aligned}
  & A = (1,2) \\
  & B = (2,-4) \\
  & C = (2,-3) 
\end{aligned}$ \quad $C = xA + yB$  $\Longrightarrow x=1/4, \quad y = 7/8$  

\exercisehead{3} \quad \\
$\begin{aligned}
  & A = (2,-1,1) \\
  & B = (1,2,-1) \\
  & C = (2,-11,7)
\end{aligned}$ $C= xA + yB$  
\[
\begin{gathered}
  \left[ 
    \begin{matrix} 
      2 & 1 \\
      -1 & 2 \\
      1 & -1 
    \end{matrix} \right. \left| \begin{matrix} 2 \\ -11 \\ 7 \end{matrix} \right] \Longrightarrow \left[ \begin{matrix} 0 & 1 \\
-1 & 0 \\
0 & 1 \end{matrix} \right. \left| \begin{matrix} -4 \\
-3 \\
-4 \end{matrix} \right] \\
  \Longrightarrow \boxed{ \begin{aligned} x & =3 \\
      y & = -4 \end{aligned} }
\end{gathered}
\]

\exercisehead{4}  If $C= (2,11,7)$,
\[
\left[ \begin{matrix} 
    2 & 1 \\
    -1 & 2 \\
    1 & -1 
\end{matrix} \right. \left| \begin{matrix} 2 \\
    11 \\ 7 \end{matrix} \right]
 = \left[ \begin{matrix} 
     0 & 5 \\
     -1 & 2 \\
     0 & 3 \end{matrix} \right. \left| \begin{matrix} 
24 \\
11 \\
29 \end{matrix} \right]
\]

\exercisehead{5} 
\begin{enumerate}
\item $A \parallel B$ $\Longrightarrow A = tB$, $t\neq 0$ \\
  $A- tB = 0$ and $t\neq 0$.  So $A,B$ linearly dependent.  
\item $A \nparallel B$ \\
  $A-tB \neq 0$ since $\nexists \, t \in \mathbb{R}$ s.t. $A=tB$.  Then $A,B$ linearly independent.  
\end{enumerate}

\exercisehead{6} Given $(a,b)$, $(c,d)$, the following statements must be true.  
\[
\begin{gathered}
  \left[ \begin{matrix} a & c \\
      b & d \end{matrix} \right] \left[ \begin{matrix} x \\ y \end{matrix} \right] = \left[ \begin{matrix} 0 \\ 0 \end{matrix} \right] \\
  \left( \frac{1}{ ad - bc } \right) \left[ \begin{matrix} d & -c \\ 
      -b & a \end{matrix} \right] \left[ \begin{matrix} a & c \\
      b & d \end{matrix} \right] \left[ \begin{matrix} x \\ y \end{matrix} \right] = \left[ \begin{matrix} x \\ y \end{matrix} \right] = \left[ \begin{matrix} 0 \\ 0 \end{matrix} \right] 
\end{gathered}
\]
If $ad- bc \neq 0$, then we could apply this inverse matrix to obtain $x=y=0$.  \smallskip \\
Suppose $x=y=0$ and $ad-bc =0$.  Then $ad=bc$ and we could rewrite $(c,d)$ to be $\left( \frac{ad}{b}, \frac{bc}{a} \right)$.  Then \smallskip \\
\phantom{ Suppose } $\frac{-d}{b} (a,b) + (c,d) = 0$.  But we had said that $x=y=0$ were the only coefficients.  Then $ad-bc \neq 0$


\exercisehead{7} Given $(1+t,1-t), (1-t,1+t)$, and from above,  \\
$(1+t)(1+t) - (1-t)(1-t) = ad-bc = 4t \neq 0$.  So for $t \neq 0$, the two vectors are linearly independent.  

\exercisehead{8} $i,j,k$ are linearly independent.  $i+j+k = L( \{ i,j,k \} )$.  By thm, $\{ i,j,k,i+j+k \}$ are linearly dependent (Apostol's Thm. 12.8, any set of $k+1$ vectors is dependent, $S$ independent, $|S| = k$ ).  

$i,j,k$ are linearly independent and pairwise linearly independent.  $i+j+k$ is not in the span of any pair from $i,j,k$ because $i+j+k$ is a linear combination of one unit coordinate vector not included in the pair.  Then the 3 vectors, including $i+j+k$, are linearly independent (if $i+j+k$ is not in the span of two other independent vectors, then there's no linear combination of the three that would give a nontrivial representation of $0$).  

\exercisehead{9} 
\begin{enumerate}
  \item $c_1 i + c_2(i+j) = 0 $ $c_1 = -c_2$  $c_2 = 0$  Then $c_1 =0 $ 
  \item $i+j - i = j \in L(S)$ 
  \item $3 i - 4 j = -4 (i+j) + 5i$ 
  \item $i, i+j$ linearly independent $|S| =2$ by Thm. 12.10, since we have 2 linearly independent vectors in $V_2$, $S$ forms a basis for $V_2$.  
\end{enumerate}

\exercisehead{10}  Given $A = i, B = i+j, C = i + j + 3k$ in $V_3$
\begin{enumerate}
\item $xA + yB + zC = 0$.  Then $z=0$.  Then $y=0$.  $x=0$.  $A,B,C$ linearly independent.  
\item $j = B-A$  $k = \frac{ C - B}{3}$  
\item $2i - 3j + 5k = \boxed{ \frac{ -14B + 5C }{3} + 5A }$  
\item $\{ A,B,C \}$ are linearly independent.  Since $| \{ A,B,C \} | = 3 =n$, for $V_n = V_3$, then by Thm. 12.10, $\{ A , B , C \}$ form a basis for $V_3$.  
\end{enumerate}

\exercisehead{11}  Let $A=(1,2), B=(2,-4), C=(2,-3), D=(1,-2)$.  \\
Note that $B=2D$ ($B,D$ are parallel).  

$A,B,C,D$ are all each linearly independent, by themselves.  

$\boxed{ \{ A,D \}, \{ A,B \}, \{ A,C \}, \{ B,C \}, \{ D, C \} }$

\exercisehead{12}  \quad \\
Let $ \begin{aligned}
  & A = (1,1,1,0) \\
  & B = (0,1,1,1) \\
  & C = (1,1,0,0)
\end{aligned}$ 
\begin{enumerate}
\item $A,B,C$ are linearly independent since for $xA + yB + zC = 0$, $y=0$, $x=0$ and so $z=0$.  
\item $-(0,1,2,1) = D$ would make the set dependent.  
\item $E = (1,0,0,0)$
\item $(1,2,3,4) = 4B + -A - C + 3E$
\end{enumerate}

\exercisehead{13} 
\begin{enumerate}
  \item $x (\sqrt{3}, 1 , 0) + y(1,\sqrt{3},1) + z(0,1,\sqrt{3}) = 0$.  Then $\begin{aligned}
    y & = -\sqrt{3} z \\
    x\sqrt{3} & = -y \\
    x+\sqrt{3}y + z & = 0 
\end{aligned}$ $\Longrightarrow \frac{ -y}{\sqrt{3}} + \sqrt{3} y + \frac{y}{ -\sqrt{3}} = 0$ \quad So $y=0$.  \\
    Then $x=z=0$
  \item $(\sqrt{2},1,0) - \sqrt{2} (1,\sqrt{2},1) = (0,-1,-\sqrt{2})$.  Dependent.  
  \item $(t,1,0) + -t(1,t,1) = (0,1,t)$  $\Longrightarrow 1 - t^2 = 1$ so that $\boxed{ t=0 }$.  
\end{enumerate}

\exercisehead{14} 
\begin{enumerate}
\item $\begin{aligned}
  & (1,0,1,0) \\
  & (1,1,1,1) \\
  & (2,0,-1,0)
\end{aligned}$ \quad \quad \\
  Note that we cannot get any bigger set, since $(0,1,0,1)$ is a linear combination of $(1,0,1,0)$ and $(1,1,1,1)$ \\
\item $(1,1,1,1), (1,-1,1,1)$ are immediately linearly independent, since they're not parallel to each other.  \\

$x(1,1,1,1) + y(1,-1,1,1) = (x+y,x-y,x+y,x+y)$, so the 3rd. and 4th. coordinate of any linear combination of these two vectors will be equal.  \medskip \\
  Thus $\begin{aligned}
    & (1,1,1,1) \\
    & (1,-1,1,1) \\
    & (1,-1,-1,1)
\end{aligned}$ are linearly independent.   \\

$\begin{aligned}
    & x(1,1,1,1) + \\
    & y(1,-1,1,1) + \\
    & z(1,-1,-1,1)
\end{aligned} = (x+y+z,x-y-z,x+y-z, x+y+z)$ \medskip \\
So any linear combination of the these three vectors, will always have the 1st. and 4th. coordinates equal.  Then \smallskip \\
$(1,1,1,1), (1,-1,1,1), (1,-1,-1,1),(1,-1.-1.-1)$ are linearly independent.  
\item $\begin{aligned}
  & (1,1,1,1) \\
  & (0,1,1,1) \\
  & (0,0,1,1) \\
  & (0,0,0,1)
\end{aligned}$ are linearly independent.  
\end{enumerate}

\exercisehead{15} 
\begin{enumerate}
\item $x(A+B) + y(B+C) + z(A+C) = (x+y+z)A + (x+y)B + (y+z)C = 0 $ \\
  $\Longrightarrow x = -z = y = -x $ since $A,B,C$ are linearly independent.  Then $x=y=z =0$.  
\item $x(A-B) + y(B+C) + z(A+C) = (x+z)A + (-x +y)B + (y+z)C =0$  So we have $\begin{aligned}
  x & = y \\
  y & = -z \\
  x & = -z 
\end{aligned}$ \medskip \\
  If $x,y\neq 0$, $z \neq 0$ and condition for linear independence of $A,B,C$ is still satisfied.  Thus, $A-B, B+C, A+C$ lienarly dependent.  
\end{enumerate}

\exercisehead{16} 
\begin{enumerate}
\item \quad \\
  If $S = \mathcal{B}_{V_3}$, \smallskip \\
  \phantom{ If } $S = \mathcal{B}_{V_3}$ means $\forall \, x \in V_3$, $x \in L(S)$.  
  \phantom{ If } Since $i,j,k \in V_3$, $i,j,k \in L(S)$ \\

  If $i,j,k \in L(S)$, then $L(\{ i,j,k \} ) \subseteq L(S)$ (since every linear combination of $i,j,k$ can be reexpressed as a linear combination of vectors in $S$, since each $i,j,k$ is a linear combination of vectors in $S$).   \smallskip \\
  \phantom{ If } Then since $\forall \, x \in V_3$, $x \in L(\{ i, j , k \} )$, $x \in L(S)$  \\
  \phantom{ If } So $S$ forms a basis for $V_3$.  
\item \quad \\
  If $S = \mathcal{B}_{V_n}$, \smallskip \\
  \phantom{ If } $S = \mathcal{B}_{V_n}$ means $\forall \, x \in V_n$, $x \in L(S)$.  
  \phantom{ If } Since $e_1,\dots,e_n \in V_n$, $e_1,\dots,e_n \in L(S)$ \\

  If $e_1,\dots,e_n \in L(S)$, then $L(\{ e_1,\dots,e_n \} ) \subseteq L(S)$ (since every linear combination of $e_1,\dots,e_n$ can be reexpressed as a linear combination of vectors in $S$, since each $e_1,\dots,e_n$ is a linear combination of vectors in $S$).   \smallskip \\
  \phantom{ If } Then since $\forall \, x \in V_n$, $x \in L(\{ e_1, \dots , e_n \} )$, $x \in L(S)$  \\
  \phantom{ If } So $S$ forms a basis for $V_n$. 
\end{enumerate}

\exercisehead{17} $\{ (0,1,1),(1,1,1),(0,0,1) \}$ and $\{ (0,1,1),(1,1,1),(0,0,-1) \}$

\exercisehead{18} $\{ (0,1,1,1), (1,1,1,1), (0,0,0,1), (0,0,1,0) \}$ and $\{ (0,1,1,1), (1,1,1,1), (0,0,0,-1), (0,0,-1,0) \}$ 

\exercisehead{19} Given that \\
$\begin{aligned}
  & S = \{ (1,1,1), (0,1,2), (1,0,-1) \} \\
  & T = \{ (2,1,0), (2,0,-2) \} \\
  & U = \{ (1,2,3), (1,3,5) \} 
\end{aligned}$
\begin{enumerate}
\item Consider $x(2,1,0) + y(2,0,-2) = (2x+2y, x,-2y)$ 
\[
\begin{gathered}
  \left[ \begin{matrix} 1 & 0 & 1 \\
      1 & 1 & 0 \\
      1 & 2 & -1 \end{matrix} \right] \left[ \begin{matrix} a \\
b \\
c \end{matrix} \right] = \left[ \begin{matrix} 2(x+y) \\
x \\
-2y \end{matrix} \right] \\
  \left[ \begin{matrix} 1 & 0 & 1 \\
      1 & 1 & 0 \\
      1 & 2 & -1 \end{matrix} \right. \left| \begin{matrix} 
      2(x+y) \\
      x \\
      -2y \end{matrix} \right] = \left[ \begin{matrix} 1 & 0 & 1 \\
      & 1 & -1 \\
      & 2 & -2 \end{matrix} \right. \left| \begin{matrix} 
      2(x+y) \\
      -x-2y \\
      -2x - 4 y 
      \end{matrix} \right] \Longrightarrow \begin{aligned}
    &  a+c = 2(x+y) \\
    &  b-c = -x-2y 
\end{aligned} \\
  \quad  \\
  \boxed{ 2(x+y)(1,1,1) + (-x-2y)(0,1,2) }
\end{gathered}
\]
\item 
\[
\begin{gathered}
  \left[ \begin{matrix} 
      1 & 0 & 1 \\
      1 & 1 & 0 \\
      1 & 2 & -1 \end{matrix} \right. \left| \begin{matrix} a \\ b \\ c \end{matrix} \right] = \left[ \begin{matrix} 0 \\ 0 \\ 0 \end{matrix} \right] \Longrightarrow \left[ \begin{matrix} 1 & 0 & 1 \\
      & 1 & -1 \\
      & 2 & -2 \end{matrix} \right. \left| \begin{matrix} a \\ b \\ c \end{matrix} \right] \quad \Longrightarrow 
  \begin{aligned}
     a & = -c \\
     b & = c 
\end{aligned} \quad \text{ so $S$ is linearly dependent }  \\
\quad \\
\begin{aligned}
  & 2(2,1,0) - \frac{3}{2} (2,0,-2) = (1,2,3) \\
  & 3(2,1,0) - \frac{5}{2} (2,0,-2) = (1,3,5) 
\end{aligned} \quad \quad \Longrightarrow L(U) = L(T)  \\
\begin{aligned}
  & (1,1,1) + (0,1,2) = (1,2,3) \\
  &  (1,1,1) + 2(0,1,2) = (1,3,5) 
\end{aligned} \quad \Longrightarrow L(S) = L(U) = L(T) 
\end{gathered}
\] 
\end{enumerate}

\exercisehead{20} $A,B$ are finite subsets.  
\begin{enumerate}
\item Consider $x_A \in L(A)$, $x_A = \sum c_j A_J$.  Since $\forall \, A_j \in B$, $x_A \in L(B)$.  Then $L(A) \subseteq L(B)$  
\item Consider $c_j \in A \bigcap B$ so that $c_j \in A$ and $c_j \in B$ \smallskip \\
  $\sum c_j C_j \in L(A) $ \text{ since } $\forall \, C_j \in A$ and  $\sum c_j C_j \in L(B) $ \text{ since } $\forall \, C_j \in B$ \\
  $\Longrightarrow \sum c_j C_j \in L(A) \bigcap L(B) \Longrightarrow L(A \bigcap B) \subseteq L(A) \bigcap L(B)$
\item Now $L(A \bigcap B) \subseteq L(A) \bigcap L(B)$.  \\
  If $\begin{aligned}
    A & = \{ (1,1,0), (1,0,1) \} \\
    B & = \{ (1,0,0), (0,1,1) \} 
\end{aligned}$ \medskip \\
  then we could see that $A \bigcap B = \emptyset$, but $L(A) \bigcap L(B)$ is nonempty; for instance, it'll include $(2,1,1) = (1,1,0) + (1,0,1) = 2(1,0,0) + (0,1,1)$.  
\end{enumerate}


%-----------------------------------%-----------------------------------%-----------------------------------
\subsection*{ 12.16 Exercises - The vector space $V_n(\mathbb{C})$ of $n$-tuples of complex numbers }
%-----------------------------------%-----------------------------------%-----------------------------------
\quad \\
\exercisehead{1} Given $\begin{aligned}
  A = & (1,i) \\
  B = & (i,-i) \\
  C = & (2i,1)
\end{aligned}$
\begin{enumerate}
\item $A\cdot B = -i -1$
\item $B\cdot A = \overline{ A \cdot B} = -1 + i $
\item $(iA) \cdot B = i (A \cdot B) = 1-i$
\item $A\cdot (iB) = -i (A \cdot B) = -1 + i$
\item $(iA) \cdot (iB) = i (-i)A\cdot B = -i -1$
\item $(i,-1)\cdot (-1,1) = B\cdot C = 2 + -i$
\item $A\cdot C = -2i + i =-i $
\item $(B+C)\cdot A = (3i,1-i)\cdot A = 3i - i -1 = 2i -1$ 
\item $(A-C)\cdot B= (1-2i,i-1)\cdot B = (-i-2,-1-i) = -3 -2i$ 
\item $(A-iB)\cdot (A+iB) = (A- (-1,1))\cdot ((1,i)+(-1,1)) = (2,-1+i)\cdot (0, 1+i) = \boxed{ 2i }$
\end{enumerate}

\exercisehead{2} Given \\
$\begin{aligned}
  & A= (2,1,-i) \\
  & B= (i,-1,2i)
\end{aligned}$ and $V_3(\mathbb{C})$

$\begin{aligned}
  & A\cdot C = (2,1,-i) \cdot (\overline{c}_1, \overline{c}_2, \overline{c}_3 ) = 2 \overline{c}_1 + \overline{c}_2 - i \overline{c}_3 =0 \\
  & B\cdot C = (i,-1,2i)\cdot (\overline{c}_1, \overline{c}_2, \overline{c}_3 ) = i \overline{c}_1 + - \overline{c}_2 + 2i \overline{c}_3 = 0 
\end{aligned}$  \quad $\Longrightarrow \begin{aligned}
  & c_2 = c_1 (-4 +i) \\
  & c_3 = c_1 (-2i -1)
\end{aligned}$ \\

$\boxed{ C = (1,i-4, -1-2i) }$

\exercisehead{3} 
\[
\begin{aligned}
  \| A + B \|^2 & = (A+B) \cdot (A+B) = (A+B) \cdot A + (A+B) \cdot B = \overline{ A\cdot (A+B) } + \overline{ B\cdot (A+B) } = \\
  & = \overline{ A\cdot A + A \cdot B } + \overline{ B\cdot A + B\cdot B } = \|A \|^2 + \| B\|^2 + \overline{A\cdot B} + \overline{ B\cdot A } = \\
  & = \boxed{ \| A \|^2 + \| B \|^2 + \overline{ A \cdot B} + A \cdot B }
\end{aligned}
\]

\exercisehead{4} 
\[
\begin{aligned}
  \| A + B \|^2 - \| A - B \|^2 & = (A+B) \cdot (A+B) - (A-B)\cdot (A-B) = \\
  & = (A+B) \cdot A + (A+B)\cdot B - (A-B) \cdot A + (A-B)\cdot B = \\
  & = \overline{ A\cdot (A+B) } + \overline{ B \cdot (A+B) } - \overline{ A \cdot (A-B ) } + \overline{ B\cdot (A-B) } = \\ 
  & = \overline{ A\cdot A} + \overline{ A\cdot B} + \overline{ B\cdot A} + \overline{ B\cdot B} - \overline{ A\cdot A} + \overline{ A\cdot B} + \overline{B\cdot A} - \overline{ B\cdot B} = \\
  &= \boxed{ 2 (A\cdot B + \overline{ A\cdot B } ) }
\end{aligned}
\]

\exercisehead{5}
\[
\begin{aligned}
  \| A + B \|^2 + \| A - B \|^2 & = (A+B) \cdot (A+B) + (A-B)\cdot (A-B) = \\
  & = (A+B) \cdot A + (A+B)\cdot B + (A-B) \cdot A - (A-B)\cdot B = \\
  & = \overline{ A\cdot (A+B) } + \overline{ B \cdot (A+B) } + \overline{ A \cdot (A-B ) } - \overline{ B\cdot (A-B) } = \\ 
  & = \overline{ A\cdot A} + \overline{ A\cdot B} + \overline{ B\cdot A} + \overline{ B\cdot B} + \overline{ A\cdot A} - \overline{ A\cdot B} - \overline{B\cdot A} + \overline{ B\cdot B} = \\
  &= \boxed{ 2 \| A\|^2 + 2 \| B \|^2 }
\end{aligned}
\]

\exercisehead{6} $A,B \in V_n(\mathbb{C})$  
\begin{enumerate}
\item $ \overline{ ( A\cdot B + \overline{ A\cdot B} ) } = \overline{A\cdot B} + \overline{ \overline{ A\cdot B}} = \overline{ A \cdot B } + A\cdot B = A\cdot B + \overline{ A\cdot B}$  So then $A\cdot B + \overline{ A\cdot B}$ must be real.  
\item 
\[
\begin{gathered}
  \text{ Now } \| A + B \|^2 = \| A \|^2 + \| B \|^2 + A\cdot B + \overline{ A\cdot B}, \text{ so then } \\
  \| \frac{ A}{ \| A \| } + \frac{ B }{ \| B \| } \|^2 = 1 + 1 + \frac{ A \cdot B + \overline{ A \cdot B } }{ \| A \| \|B \| } \geq 0  \\
  \Longrightarrow \frac{ A \cdot B + \overline{ A \cdot B} }{ \| A \| \| B \| } \geq - 2 \\
\quad \\
\text{ Now } \| A - B \|^2 = (A-B)\cdot (A-B) = (A-B)\cdot A - (A-B) \cdot B = \| A \|^2 - B\cdot A - A\cdot B + \| B \|^2 \geq 0 \medskip \\
 \text{ so then } \| \frac{A}{ \| A \| } - \frac{ B}{ \| B \| } \|^2 = 2 - \frac{ (A\cdot B + \overline{ A\cdot B } )}{ \| A \| \| B \| } \geq 0 \\
 \Longrightarrow 2 \geq \frac{ A \cdot B + \overline{ A\cdot B } }{ \| A \| \| B \| }
\end{gathered}
\]
\end{enumerate}

\exercisehead{7} With this definition, that $\theta = \arccos{ \frac{ \frac{1}{2} (A \cdot B + \overline{ A\cdot B } ) }{ \| A \| \| B \| } }$,  \medskip \\
We had already showed that \\
$ \| A - B \|^2 = (A-B)\cdot (A-B) = (A-B)\cdot A - (A-B) \cdot B = \| A \|^2 - B\cdot A - A\cdot B + \| B \|^2 \geq 0 $ \smallskip \\
Thus $\| A - B \|^2 = \| A \|^2 + \| B \|^2 - (A\cdot B + \overline{ A\cdot B} ) = \| A \|^2 + \| B \|^2 - 2 \| A \| \| B \| \cos{ \theta}$

\exercisehead{8} $V_n(\mathbb{C})$ \\
$\begin{aligned}
  & A = (1,0,i,i,i) \\
  & B = (i,i,i,0,i)
\end{aligned}$ \quad \quad $\theta = \arccos{ \frac{ \frac{1}{2} (A\cdot B + \overline{ A\cdot B } ) }{ \| A \| \| B \| } }$

$A \cdot B = -i + 1 + 1 = 2 - i $  \quad \quad $\overline{ A\cdot B} = 2 + i$ \quad $\Longrightarrow A \cdot B + \overline{ A\cdot B } = 2 - i + 2 + i = 4 $

\noindent $\| A \|^2 = (1,0,i,i,i) \cdot ( 1,0 , i ,i,i ) = 1 + 1 + 1 + 1 = 4 $ \\
$\| B \|^2 = (i,i,i,0,i) \cdot (i,i,i,0,i) = (1+1+1+1) = 4 $

$\theta = \arccos{ \frac{ \frac{1}{2} 4 }{ 2 \cdot 2 } } = \arccos{1/2}$  \quad $\Longrightarrow \cos{\theta} = 1/2 \text{ or } \boxed{ \theta = \pi/3 } $

\exercisehead{9}  \quad \\
Given that $\begin{aligned}
  & A = (1,0,0) \\
  & B = (0,i,0) \\
  & C = (1,1,i)
\end{aligned}$
\begin{enumerate}
  \item $aA + bB + cC = (x,y,z)$  \\
    $c = z/i = -iz$  Then $b = -i(y + iz)$ and $a = x+iz$.  \\

If $x=y=z=0$, then $c=0=b=a$.  So $A,B,C$ does form a basis.  
  \item $(5,2-i,2i) = 2C + 3A -B$
\end{enumerate}

\exercisehead{10}
\[
\begin{gathered}
  \sum_{j=1}^n c_j e_j = (x_1, \dots, x_n) = X \\
  \text{ then the $c_j$'s simply equal the complex coordinates of $X$ } c_j = x_j \\
  \quad \\
 \text{ If $X =0 $, then each $c_j =0 $.  So the $e_j$'s are linearly independent }.  
 \Longrightarrow \text{ $e_j$'s form a basis for $V_n(\mathbb{C})$ }
\end{gathered}
\]

%-----------------------------------%-----------------------------------%-----------------------------------
\subsection*{ 13.5 Exercises - Introduction, Lines in $n$-space, Some simple properties of straight lines, Lines and vector-valued functions }
%-----------------------------------%-----------------------------------%-----------------------------------
\quad \\
\exercisehead{1}
Recall that if $X_0 \in L(P,A)$, and $P,Q \in L(P,A)$, then $\begin{aligned} 
  X_0 = P + tA & = P + t(Q-P) \\
  X_0 - P & = t(Q-P) 
\end{aligned}$.   \smallskip \\
$Q-P = (4,0)$, $P=(-3,1)$, $Q=(1,1)$  
\begin{enumerate}
  \item $(0,0)$ no.  
  \item $(0,1)$ yes.  
  \item $(1,2)$ no 
  \item $(2,1)$ yes
  \item $(-2,1)$ yes
\end{enumerate}

\exercisehead{2} $P=(2,-1)$; $Q=(-4,2)$, $Q-P = (-6,3)$
\begin{enumerate}
  \item $(0,0)$ yes
  \item $(0,1)$ no 
  \item $(1,2)$ no 
  \item $(2,1)$ no 
  \item $(-2,1)$ yes.  
\end{enumerate}

\exercisehead{3} $P = (-3,1,1)$ \quad $A = (1,-2,3)$
\begin{enumerate}
  \item $(0,0,0)$ no 
  \item $(2,-1,4)$ no 
  \item $(-2,-1,4)$ yes
  \item $(-4,3,-2)$ yes
  \item $(2,-9,16)$ yes
\end{enumerate}

\exercisehead{4} $\begin{aligned}
  & P = (-3,1,1) \\
  & Q = (1,2,7)
\end{aligned}$ and $Q-P = (4,1,6)$
\begin{enumerate}
  \item $(-7,0,5)$ no 
  \item $(-7,0,-5)$ yes 
  \item $(-11,1,11)$ no 
  \item $(-11,-1,11)$ no.  
  \item $(-1,3/2,4)$ yes.  
  \item $(-5/3,4/3,3)$ yes.
  \item $(-1,3/2,-4)$ no.  
\end{enumerate}

\exercisehead{5} $P=(2,1,1)$, $Q=(4,1,-1)$, $R = (3,-1,1)$
\begin{enumerate}
  \item $Q-P = (2,0,-2)$ \quad \, $R-P = (1,-2,0)$.  No.  
  \item $P=(2,2,3)$, \quad $Q=(-2,3,1)$, \quad $R=(-6,4,1)$ \\
  $Q-P = (-4,1,-2)$, \quad $R-P = (-8,2,-2)$.  \quad No.  
  \item $P=(2,1,1)$; \quad $Q = (-2,3,1)$, \quad $R=(5,-1,1)$  \\
    $Q-P = (-4,2,0)$ \quad \quad $R-P = (3,-2,0)$. \quad No.  
\end{enumerate}

\exercisehead{6} \quad \\
$\begin{aligned}
  & A = (2,1,1) \\
  & B = (6,-1,1) \\
  & C = (-6,5,1) 
\end{aligned}$ \quad \, Since $\begin{aligned}
& B-A = (4,-2,0) \\
& C-A = (-8,4,0) 
\end{aligned}$  \quad \, $\begin{aligned}
  & D= (-2,3,1) \\
  & D-A = (-4,2,0) \\
  & F-A = (-6,3,0)
  \end{aligned}$ \quad \quad $\begin{aligned}
  & E = (1,1,1) \\
  & F = (-4,4,1) \\
  & G = (-13,9,1) \\
  & H = (14,-6,1) 
\end{aligned}$  \quad \\
$\{ A,B,C,D,F \}$

\exercisehead{7} $P=(1,1,1)$ \quad \, $A = (1,2,3)$  \smallskip \\
$\begin{gathered}
  P+tA = Q+uB \\
  (1,1,1) + t(1,2,3) = (2,1,0) + u (3,8,13) 
\end{gathered}$  \quad \, $\begin{aligned}
  1+t & = 2 + 3u \\
  1 + 2t & = 1 + 8u 
\end{aligned} \quad \Longrightarrow u = 1, \quad t =4$ \\
$\boxed{ (5,9,13) }$ is an intersection point.  

\exercisehead{8} 
\begin{enumerate}
\item $X = P + tA = Q+u B$  $\Longrightarrow P - Q = uB - tA $  \\
  $P-Q = aA + bB$  $\Longrightarrow P + (-a)A = Q + bB$ and $\begin{aligned}
  & P + (-a)A \in L(P;A) \\ 
  & Q + bB  \in L(Q;B)
\end{aligned}$  \\
\item $\begin{aligned}
  L & = \{ (1,1,-1) + t(-2,1,3) \} 
  L' & = \{ (3,-4,1) + t(-1,5,2) \} 
\end{aligned}$ \smallskip \\
 $ (1,1,-1) + t(-2,1,3) = (3,-4,1)  + u (-1,5,2) $ \smallskip \\
  \[
  \begin{aligned}
   & 1+ -2t = 3 - u 
    & 1+t = -4 + 5u 
\end{aligned} \quad \quad 3 = -5 + 9u \quad \quad \begin{aligned}
    & u = 8/9 \\
    & t = -5/9
\end{aligned}
\]
But $t = -5/9$, $u=8/9$, doesn't work for the 3rd. coordinate.  
\end{enumerate}

\exercisehead{9} $X(t) = P + tA$  \quad \, $P=(1,2,3)$  \quad \, $A= (1,-2,2)$, \quad \, $Q=(3,3,1)$
\begin{enumerate}
\item \[
\begin{aligned}
  \| Q - X(t) \|^2 & = (Q-X)\cdot (Q-X) = Q^2 - 2Q\cdot X + X^2 = 19 - 2 Q\cdot (P+tA) + P^2 + 2 t P\cdot A + t^2 A^2 = \\
  & = 19 -24 + -2t(-1) + 14 + 2t(3) + t^2 9 = 9 + 8t + 9t^2 
\end{aligned}
\]
\item $X(t_0)$  $\| Q - X(t) \|$ min.  $\frac{d}{dt} \| Q-X(t) \|^2 = 18t + 8 =0$  $t = -4/9$  \\
  $\| Q - X(t) \| = \sqrt{ 9 + \frac{-32}{9} + \frac{16}{9} } = \sqrt{ \frac{ 81-16}{9} } = \sqrt{ \frac{65}{9} } = \frac{ \sqrt{13\cdot 5}}{3} $
\item 
\[
\begin{aligned}
  (Q-X(t_0)) \cdot A  & = Q\cdot A - X\cdot A = Q\cdot A - (P+tA) \cdot A = Q\cdot A - P\cdot A - tA \cdot A = \\
  & = -1 -3 - \left( \frac{-4}{9} \right)(9) = 0 
\end{aligned}
\]
\end{enumerate}

\exercisehead{10} $Q \notin L(P;A)$
\begin{enumerate}
\item $f(t) = \| Q - X(t) \|^2$ \quad \, $X(t) = P+tA$
\[
\begin{aligned}
  (Q-X)\cdot (Q-X) & = Q^2 - 2Q \cdot X + X^2 = Q^2 -2Q\cdot (P+tA) + P^2 + 2tP \cdot A + t^2 A^2 = \\
  & = Q^2 - 2Q\cdot P - 2tQ\cdot A + P^2 + 2tP\cdot A + t^2 A^2 
\end{aligned}
\]
$\frac{d}{dt} (\| Q - X(t) \|^2) = 2tA^2 + 2(P\cdot A - Q\cdot A) = 0$ \quad \quad \, $t= \frac{(Q-P)\cdot A }{A^2} $
\item $(Q-X)\cdot A = Q\cdot A - (P+tA)\cdot A = Q\cdot A - P\cdot A - tA^2 = (Q-P)\cdot A - \frac{ (Q-P)\cdot A}{A^2} A^2 = 0 $
\end{enumerate} 

\exercisehead{11} $L(P;A)$, $L(Q, A)$ in $V_n$ \medskip \\
Consider $x_1 = P+tA$.  \\
\quad \quad $X_1 = Q + (P-Q) + t_1 A$ \\
\quad \quad \quad if $P-Q = u_1 A$, then $X_1 \in L(Q,A)$  \\

Likewise, if $\begin{aligned}
  X_2 & = Q + tA \\
  X_2 & = P + (Q-P) + t_2 A 
\end{aligned}$  \medskip \\
\quad \quad \quad If $P-Q = -u_2 A$, then $X_2 \in L(P;A)$ \\

If $P-Q \parallel A$, then $X_1 \in L(Q,A)$ and $X_2 \in L(P,A)$, so that $L(Q,A) = L(P,A)$ \\
If $P-Q \nparallel A$, then $X_1 \notin L(Q,A)$ and $X_2 \notin L(P;A)$, $\forall \, X_1 \in L(P,A)$, $\forall \, X_2 \in L(Q,A)$

$X_1 = Q+ (P-Q) +t_1 A \neq Q + (u_1 + t_1)A \Longrightarrow P-Q \neq u_1 A $ so that $X_1 notin L(Q,A)$

\exercisehead{12} $L(P;A), L(Q;B)$, $A\nparallel B$ \\

Consider $\begin{gathered}
  \begin{aligned}
    & X_1 = P + tA \\
    & X_2 = Q + uB
\end{aligned} \\
X_1 - X_2 = P-Q + tA - uB
\end{gathered}$ \quad \quad $\begin{gathered}
  X_1 - X_2 = 0 \text{ only if } P-Q = u_1 B - t_1 A \\
  \text{ If } P-Q = u_2 B - t_2 A, \text{ then } (u_1 - u_2)B = (t_1 - t_2) A \\
  \text{ but } A \nparallel B \text{ so } t_1 = t_2, u_1 = u_2 
\end{gathered}$ \\

(only one intersection point).  Otherwise, if $\nexists \, u_1, t_1$ s.t. $P-Q = u_1 B - t_1 A$, then there's no intersection.  

%-----------------------------------%-----------------------------------%-----------------------------------
\subsection*{ 13.8 Exercises - Planes in Euclidean $n$-space, Planes and vector-valued functions }
%-----------------------------------%-----------------------------------%-----------------------------------
\quad \\
\exercisehead{1} Let $M = \{ P + sA + tB \}$, $P=(1,2,-3), A = (3,2,1), B = (1,0,4)$.  Remember, if $\begin{aligned}
  & X = P + sA + tB \\
  & X- P = sA + tB 
\end{aligned}$
\begin{enumerate}
\item $(1,2,0)$ $X-P = (0,0,3)$; no 
\item $(1,2,1)$ $X-P = (0,0,4)$; no 
\item $(6,4,6)$ $X-P = (5,2,9) = A + 2B$, yes
\item $(6,6,6)$ $X-P = (5,4,9)$ no
\item $(6,6,-5)$ $X-P = (5,4,-2) = 2A + -B$ yes.  
\end{enumerate}

\exercisehead{2} \quad \\
$\begin{aligned}
  & P = (1,1,-1) \\
  & Q = (3,3,2) \\
  & R = (3,-1,-2)
\end{aligned}$ so that $\begin{aligned}
  & Q-R = (0,4,4) \\
  & P-R = (-2,2,1)
\end{aligned}$ 
\begin{enumerate}
  \item $(2,2,1/2)$ \quad $X- R = (-1,3,5/2)$ yes 
  \item $(4,0,-1/2)$ \quad $X- R = (1,1,3/2)$  yes 
  \item $(-3,1,-3)$ \quad $X- R = (-6,2,-1)$ yes 
  \item $(3,1,3)$ \quad $X- R = (0,2,5)$ no  
  \item $(0,0,0)$ \quad $X- R = (-3,1,2)$ no 
\end{enumerate}

\exercisehead{3}
\begin{enumerate}
  \item  $(1,2,1) + s(0,1,0) + t(1,1,4) = (x,y,z)$ $\Longrightarrow \begin{aligned}
    & x = 1+ t \\
    & y = 2 + s+ t\\
    & z = 1+4t
  \end{aligned}$
  \item $(0,1,0) + s(1,1,1) + t(1,0,4)$
\end{enumerate}

\exercisehead{4} $(1,2,0) + s(1,1,2) + t(-2,4,1)$ or $N = (-7,-5,6)$ (by finding the cross product of the two spanning vectors), so that $-7x - 5y + 6z = -17$.  
\begin{enumerate}
\item $(0,0,0)$ no. 
\item $(1,2,0)$ yes.
\item $(2,-3,-3)$ yes.  
\end{enumerate}

\exercisehead{5} $M = \{ P + s(Q-P) + t(R-P) \}$
\begin{enumerate}
  \item $P + s(Q-P) + t(R-P) = (1-s-t)P + sQ + tR$ \smallskip \\
    Since $1-s-t=p$, for $pP$, $1 = p+s+t$
  \item $\begin{aligned}
    X &  = P + s(Q-P) + t(R-P) = \\
    & = (1-s-t)P + sQ + tR 
\end{aligned}$ \\
$p = 1-s-t$, $p + s+t = 1$
\end{enumerate}

\exercisehead{6} $ax + by +cz = d$
\begin{enumerate}
  \item $(2,3,1)$ spanned by $(3,2,1)$ and $(-1,-2,-3)$, 
\[
    \begin{aligned} 
    X & = P + sA + tB  \\
    \left( \begin{matrix} x \\ y \\ z \end{matrix} \right) & = \left( \begin{matrix} 2 \\ 3 \\ 1 \end{matrix} \right) + s \left( \begin{matrix} 3 \\ 2 \\ 1 \end{matrix} \right) + t \left( \begin{matrix} -1 \\ -2 \\ -3 \end{matrix} \right) \quad \, \Longrightarrow \begin{aligned}
      & x = 2 + 3s - t \\
      & y = 3 + 2s -2t \\
      & z = 1 + s - 3t 
\end{aligned}
    \end{aligned}
\]
$\Longrightarrow x - 2y + z = -3$ \\
\item If $\begin{aligned}
  P & = (-2,-1,-3) \\
  Q-P & = (4,4,4) \\
  R-P & = (6,4,4)
\end{aligned}$ \quad $(x,y,z) = (-2,-1,-3) + s(1,1,1) + t(3,2,2)$  \medskip \\
  $\begin{aligned}
  x & = -2 + s + 3t \\
  y & = -1 + s + 2t \\
  z & = -3 + s + 2t 
\end{aligned}$ \quad $\Longrightarrow  y + 1 = z +3$ or $y - z = 2$ 
\item $(2,3,1)$ parallel to the plane through the origin, spanned by $(2,0,-2)$ and $(1,1,1)$  
\[
\begin{gathered}
  \begin{aligned}
    & P + & s A + & t B \\
    \left( \begin{matrix} x \\ y \\ z \end{matrix} \right) = & \left( \begin{matrix} 2 \\ 3 \\ 1 \end{matrix} \right) + & s \left( \begin{matrix} 2 \\ 0 \\ -2 \end{matrix} \right) + & t \left( \begin{matrix} 1 \\ 1 \\ 1 \end{matrix} \right) \end{aligned} \quad \\
    \Longrightarrow \begin{aligned}
      x & = 2 + 2 s + t \\
       y & = 3 + t \\
       z & = 1 - 2s + t 
\end{aligned} \quad \, \Longrightarrow \boxed{ x + 2 y + z = -3 }
\end{gathered}
\]
\end{enumerate}

\exercisehead{7} $M$ s.t. $3x - 5y + z = 9$
\begin{enumerate}
\item Out of $\begin{aligned}
  & (0,-2,-1) \\
  & (-1,-2,2) \\
  & (3,1,-5)
\end{aligned}$, $\boxed{ \begin{aligned}
    & (0,-2,-1) \\
    & (-1,-2,2)
\end{aligned} }$
\item $\begin{aligned}
  & (0,-2,-1) - (2,1,8) = (-2,-3,-9) \\
  & (-1,-2,2) - (2,1,8) = (-3,-3,-6) 
\end{aligned}$ \quad $\Longrightarrow \boxed{ (2,1,8) + s(2,3,9) + t(1,1,2) } $
\end{enumerate}

\exercisehead{8} Given two planes, $M = \{ P + sA + tB \}$, $M' = \{ Q + sC + tD \}$, and the specific coordinates, \\
$\begin{aligned}
  & P = (1,1,1) \\
  & A = (2,-1,3) \\
  & B = (-1,0,2)
\end{aligned} \quad \quad 
\begin{aligned}
  & Q = (2,3,1) \\
  & C = (1,2,3) \\
  & D = (3,2,1) 
\end{aligned}$ \\
It's easiest to find the Cartesian coordinate equations for each of the planes, to find the intersection, and the easiest way to find the Cartesian coordinate equations is to find the normal to each of the planes, using the cross product on $A,B$ and on $C,D$.  $A\times B = (-2,-7,-1)$ and $C\times D = (-4,8,-4)$.  Then 
\[
\begin{aligned}
   M:  & 2x + 7y + z = 10 \\
   M': & -4x + 8y + -4z = 12
\end{aligned}
\]
It's easy to check that these are the correct equations because we could plug in $P,Q$ respectively to check the equation and plug in $A,B$ and $C,D$, respectively to check that we get $0$ (because adding arbitrary amounts of $A,B$ or $C,D$, respectively, should not affect each respective equation).  \\

Thus, solving for both $M$ and $M'$, we get $x+9y=13$.  Then two distinct points include $\boxed{ (4,1,-5), (-5,2,6)}$

\exercisehead{9} 
\begin{enumerate}
\item Given a plane that \\
$P = (2,3,1), \quad \begin{aligned}
& A = (1,2,3) \\
& B = (3,2,1) 
\end{aligned}$ and $x-2y + z = 0$, so that (with a Cartesian coordinate equation, we could easily plug in numbers to find 3 points that this plane contains) \\
$\begin{aligned}
& P = (2,1,0) \\
& Q = (3,2,1) \\
& R = (-1,-1,-1)
\end{aligned} \quad \quad \begin{aligned}
  & Q - P = (1,1,1) \\
  & R - P = (-3,-2,-1) \\
  & X = \left( \begin{matrix} 2 \\ 1 \\ 0 \end{matrix} \right) + s \left( \begin{matrix} 1 \\ 1 \\ 1 \end{matrix} \right) + t \left( \begin{matrix} -3 \\ -2 \\ -1 \end{matrix} \right)
\end{aligned}$
So since these two planes have $L(A,B) = L(Q-P,R-P)$ (since $(1,2,3)+(3,2,1) = 4(1,1,1)$ and $(3,2,1) = -(-3,-2,-1)$, then the two planes are parallel.)  \\
\item Given that \\ 
$\begin{aligned}
 & M' : x- 2 y + z & = 0  \\
 & M : x + 2y + z  & = 0  
\end{aligned}$ \quad then to find the intersection, solve for the two equations to get $2x + 2z =0 \Longrightarrow x = -z$.  Then two points in the intersection, $M' \bigcap M''$ are $\boxed{ (1,0,-1), (-1,0,1) }$
\end{enumerate}

\exercisehead{10} \[
\begin{aligned}
  L & = \left( \begin{matrix} 1 \\ 1 \\ 1 \end{matrix} \right) + s \left( \begin{matrix} 2 \\ -1 \\ 3 \end{matrix} \right) \\
  M & = \left( \begin{matrix} 1 \\ 1 \\ -2 \end{matrix} \right) + t \left( \begin{matrix} 2 \\ 1 \\ 3 \end{matrix} \right) + u \left( \begin{matrix} 0 \\ 1 \\ 1 \end{matrix} \right)
\end{aligned}
\]
Condition for intersection:
\[
\left( \begin{matrix} 1 \\ 1 \\ 1 \end{matrix} \right) + s \left( \begin{matrix} 2 \\ -1 \\ 3 \end{matrix} \right) = \left( \begin{matrix} 1 \\ 1 \\ -2 \end{matrix} \right) + t \left( \begin{matrix} 2 \\ 1 \\ 3 \end{matrix} \right) + u \left( \begin{matrix} 0 \\ 1 \\ 1 \end{matrix} \right)
\]
Then
\[
\begin{gathered}
  \left( \begin{matrix} 0 \\ 0 \\ -3 \end{matrix} \right) = s \left( \begin{matrix} 2 \\ -1 \\ 3 \end{matrix} \right)  -t \left( \begin{matrix} 2 \\ 1 \\ 3 \end{matrix} \right) - u \left( \begin{matrix} 0 \\ 1 \\ 1 \end{matrix} \right) = \left[ \begin{matrix} 2 & -2 & 0 \\ -1 & -1 & -1 \\ 3 & -3 & -1 \end{matrix} \right]\left[ \begin{matrix} s \\ t \\ u \end{matrix} \right] \\
  \left[ \begin{matrix} 2 & - 2 & 0 \\ -1 & - 1 & -1  \\ 3 & -3 & -1 \end{matrix} \right| \left. \begin{matrix} 0 \\ 0 \\ -3 \end{matrix} \right] = \left[ \begin{matrix} 2 & 0 & 0 \\ 0 & -2 & 0 \\ 0 & 0 & 1 \end{matrix} \right| \left. \begin{matrix} -3 \\ 3 \\ 3 \end{matrix} \right] \quad \, \Longrightarrow \begin{aligned} s & = \frac{-3}{2} \\ t & = \frac{3}{-2} \\ u & = 3 \end{aligned}  \quad \, \Longrightarrow \boxed{ X = \left( \begin{matrix} -2 \\ 5/2 \\ -7/2 \end{matrix} \right) }
\end{gathered}
\]

\exercisehead{11} $L  = \{ (1,1,1) + t (2,-1,3) \} $
\begin{enumerate}
  \item $M = \{ (1,1,-2) + s(2,1,3) + t(3/4,1,1) \}$  \\
    Suppose $(2,-1,3) = s(2,1,3) + t(3/4,1,1)$.  Then $\begin{aligned}
      & 2 = 2s + 3/4 t \\ 
      & -1 = s +t 
\end{aligned}$.  Doing the algebra, we get $t = -16/5, s = 11/5$.  But the third coordinate doesn't work, $33/5 - 16/5 = 17/5$.  So $M,L$ are not parallel since $(2,-1,3) \notin L(\{ (2,1,3), (3/4,1,1) \} )$ \\
  \item $(1,1-2), (3,5,2), (2,4,-1)$.   $\begin{aligned}
    & C = (2,4,4) \\
    & D= (1,3,1)
\end{aligned}$ \quad \, $sC + tD = (2,-1,3)$.  Doing the algebra, we get $t= -5, s =7/2$ from the first two coordinates, but it doesn't agree with the third coordinates, so that $7/2 C + -5D \neq (2,-1,3)$.  No, not parallel.  
\item From, $x + 2y + 3z = -3$, we get the spanning vectors easily by plugging in numbers that'll make the LHS equal to zero: $(-3,0,1), (1,1,-1)$.  Again, we find that $(2,-1,3) \neq (-1)(1,1,-1) + (-1)(-3,0,1)$.  No, not parallel.  
\end{enumerate}

\exercisehead{12} $P,Q \in M$.  \\
Consider $M = \{ P + SA + tB \}$.  $Q \in M$, so that $P + sA + tB = Q$. $\Longrightarrow Q - P = sA + tB$.  \\
Now $c_1 (sA + tB) + c_2 A = (c_1 s + c_2)A + c_1 tB = 0$ \\
$A,B$ linearly independent, so $c_1 = 0$, then $c_2 = 0$.  \\
So $sA + tB = Q -P$, $A$ are linearly independent, and $Q-P = sA + tB, A$ span the same space as $A,B$, so by theorem, \\
$M = \{ P + s(Q-P) + tA \}$ \\

For $X \in L$, $X = P + s(Q-P) = P + s(Q-P) + 0A$ so $L \subseteq M$.  

\exercisehead{13} For $L$ containing $(1,2,3)$ and $\parallel$ to $(1,1,1)$, $L = \{ P + sA \}$.  \\
Consider $M=\{ P + sA + tB \}$, containing $Q$.  Then $Q = (2,3,5) = P + sA + tB$.  If we let $s=t=1$, then
\[
B = (2,3,5) - (1,2,3) - (1,1,1)  = (0,0,1)
\]
$(0,0,1)$ is linearly independent of $(1,1,1)$ and so a possible $M$ is $M = (1,2,3) + s(1,1,1) + t(0,0,1)$ or $-x + y =1$.  
\exercisehead{14} $L = \{ Q + sA \}$  We have a plane that we want: $M = \{ Q + sA + t(P-Q) \}$.  \\
Consider $M'$ s.t. $L \subseteq M'$ and $P \in M'$, $M' = \{ R + sB + tC \}$.   \\
Note that $P \neq Q + sA$ \quad $\forall \, s \in \mathbb{R}$  \\

In general, $X = Q + sA = (R + b_0 B + c_0 C) + s(b_1 B + c_1 C)$ so that \medskip \\
$\begin{aligned}
  Q & = R + b_0 B + c_0 C, \quad b_0, c_0 \neq 0  \\
  A & = (b_1 B + c_1 C), \quad b_1,c_1 \neq 0
\end{aligned}$ \quad \quad $P = R + b_2 B + c_2 C$, $b_2,c_2 \neq 0$, since $P \in M'$.   \\
So then $Q - P = (b_0 - b_2)B + (c_0 - c_2)C = b_3 B + c_3 C$ and \\
\[
\begin{gathered}
  b_3 B + c_3 C = b_3 ( A - c_1 C)/b_1 + c_3 C = \frac{b_3}{b_1} A + (c_3 - \frac{c_1}{b_1} )C = Q- P \\
  C = \frac{ Q - P  - \frac{b_3}{b_1} A }{ (c_3 - \frac{c_1}{b_1} ) }
\end{gathered}
\]

\[
\begin{gathered}
  \begin{aligned}
    X & = R + sB + tC \\
    & = Q - b_0 B - c_0 C + sB + tC = \\
    & = Q + (s-b_0)B + (t-c_0)C = \\
    & = Q + (s-b_0)((Q-P) - c_3 C)/b_3 + (t-c_0)C = Q + \frac{ (s-b_0)}{b_3} (Q-P) + \frac{ -(s-b_0)c_3 + b_3(t-c_0) }{b_3} C = \\
    & = Q + \frac{ (s-b_0)}{b_3} (Q- P) + \frac{ -(s-b_0)c_3 + b_3 (t-c_0) }{b_3} \frac{ Q - P  - \frac{b_3}{b_1} A }{ (c_3 - \frac{c_1}{b_1} ) }
  \end{aligned} \\
  \Longrightarrow M' \subseteq M
\end{gathered}
\]

\emph{ Suppose } if $c_1 = 0$ (without loss of generality) \smallskip \\
$A = b_1 B (B \Longrightarrow A)$(without loss of generality) 
\[
\begin{gathered} 
  P - Q = (b_2 - b_0) A + (c_2 - c_0) C  \text{ since } ( P \neq Q + sA, \text{ so } (c_2 - c_0) \neq 0 ) \\ 
  \quad \\ 
  \begin{aligned}
    R + sB + tC & = Q - b_0 A - c_0 C + sA + tC = \\
    & = Q + (s-b_0)A + (t-c_0) \left( \frac{ P-Q - (b_2 - b_0) A }{ c_2 -c_0 } \right) = \\
    & = Q + \left( (s-b_0) + \frac{ (b_0 - b_2 )(t-c_0 ) }{ c_2 - c_0 } \right) A + \left( \frac{ t-c_0}{c_2 -c_0} \right)(P - Q ) \in M 
  \end{aligned} \\
  \Longrightarrow M' \subseteq M 
\end{gathered}
\]

$Q + sA + t(P-Q)$.  \\
If $P = R + b_2 B+ c_2 C$ and $\begin{aligned}
  Q & = R + b_0 B + c_0 C \\
  A & = (b_1 B + c_1 C)
\end{aligned}$ \medskip \\
\[
\begin{gathered}
  \begin{aligned}
    Q + sA + t(P-Q) & = R + b_0 B + c_0 C + s(b_1 B + c_1 C) + t (b_2 B + c_2 C - b_0 B - c_0 C) = \\
    & = R + (b_0 + s b_1 + t(b_2 - b_0))B + (c_0 + sc_1 + t(c_2 - c_0))C
  \end{aligned} \\
  \Longrightarrow M \subseteq M'
\end{gathered}
\]
So then $M = M'$.  

%-----------------------------------%-----------------------------------%-----------------------------------
\subsection*{ 13.11 Exercises - The cross product, The cross product expressed as a determinant }
%-----------------------------------%-----------------------------------%-----------------------------------
\quad \\
\exercisehead{1} Given $\begin{aligned}
  & A = (-1,0,2) \\
  & B = (2,1,-1) \\
  & C = (1,2,2) 
\end{aligned}$ 
\begin{enumerate}
  \item $A \times B = \left| \begin{matrix} e_1 & e_2 & e_3 \\
    -1 & 0 & 2 \\
    2 & 1 & -1 
\end{matrix} \right| = (-2,3,-1)$
  \item $B\times C = (4,-5,3)$
  \item $C \times A = (4,-4,2)$
  \item $A \times (C \times A) = (8,10,4)$ 
  \item $(A\times B)\times C = (8,3,-7)$
  \item $A \times (B\times C) = (10,11,5)$
  \item $(A\times C)\times B = (-2,-8,-12)$
  \item $(A+B)\times (A-C) = (A+B)\times A - (A+B)\times C = B\times A - A\times C - B \times C = (2,-3,1) + (4,-4,2) - (4,-5,3) = (2,-2,0)$
  \item $(A \times B)\times (A\times C) = (-2,0,4)$
\end{enumerate}

\exercisehead{2}  \begin{enumerate}
\item $\begin{aligned}
  & A = (1,1,1) \\
  & B = (2,3,-1)
\end{aligned}$  \quad $A \times B = (-4,3,1)$ $\Longrightarrow \boxed{ \frac{1}{ \sqrt{26} } (-4,3,1) }$
\item $\begin{aligned}
  & A= (2,-3,4) \\
  & B = ( -1,5,7)
\end{aligned}$ \quad $A \times B = (-41,-18,7)$ $\Longrightarrow \boxed{ \frac{ \pm 1 }{ \sqrt{ 2054 } } (-41,-18,7) }$
\item $\begin{aligned}
  & A = (1,-2,3) \\
  & B = (-3,2,-1)
\end{aligned}$ \quad $A \times B = (-4,-8,-4)$ $\Longrightarrow \boxed{ \frac{ \pm (1,2,1) }{ \sqrt{6} } }$
\end{enumerate}

\exercisehead{3} 
\begin{enumerate}
  \item $\begin{aligned}
    & A = (0,2,2) \\
    & B = (2,0,-1) \\
    & C = (3,4,0)
\end{aligned}$ \quad \, $C-A \times B-A = (3,2,-2) \times (2,-2,-3) = (-10,5,-10)$ \\
    $\| C - A \times B-A \| = \boxed{ 15/2 } $ 
  \item $\begin{aligned}
    A & = (-2,3,1) \\
    B & = (1,-3,4) \\
    C & = (1,2,1) 
\end{aligned}$ \quad \, $C-A \times B-a = (3,-1,0) \times (3,-6,3) = (-3,-9, -15)$ \\
    $ C-A \times B-A = \sqrt{315} = \boxed{ 3\sqrt{35}/2 } $
  \item $\begin{aligned}
    A & = (0,0,0) \\ 
    B & = (0,1,1) \\
    C & = (1,0,1) 
\end{aligned}$ \quad $\| C- A \times B-A \| = \boxed{ \frac{ \sqrt{3} }{2 } }$ 
\end{enumerate}

\exercisehead{4}
$ \begin{aligned}
    & A = (2,5,3) \\
    & B = (2,7,4) \\
    & C = (3,3,6) 
\end{aligned}$ \quad $(A-C) \times (B-A) = (-1,2,-3) \times (0,2,1) = (8,1,-2)$  

\exercisehead{5} $\| A \times B \|^2 = \| A \|^2 \| B\|^2 - (A\cdot B)^2 $ \medskip \\
If $\| A \times B \| = \| A \| \| B \|$, $A\cdot B = 0$ so $\cos{ \theta } = 0$ or $\theta = \pi/2$.  \\

If $A\cdot B = 0$, then $\| A \times B \| = \| A \| \|B \|$  

\exercisehead{6} Given $A,B \in V_3$ \medskip \\
$C = (B\times A )  - B$ 
\begin{enumerate}
\item $A \cdot ( B+C ) = A\cdot B + A\cdot ( B\times A) - A\cdot B = 0$
\item 
\[
\begin{gathered}
  \begin{aligned}
    & B\cdot C = B\cdot (B \times A) - B\cdot B = -B^2 \\
    & C\cdot C = ((B\times A) - B) \cdot ((B\times A) - B) = (B\times A)^2 + B^2
  \end{aligned} \\
\begin{aligned}
  \frac{ B\cdot C }{ \| B \| \|C \| } & = \cos{ \theta_{BC} } = \frac{ -B^2 }{ B ((B\times A)^2 + B^2 )^{1/2} } = \frac{ - B }{ \sqrt{ (BA \sin{\theta_{AB}})^2 + B^2 } } = \\
  & = \frac{-1}{ \sqrt{ A^2 \sin^2{\theta_{AB}} + 1 } }
\end{aligned}
\end{gathered}
\]
So $\cos{\theta_{BC}}$, since $A^2 \sin^2{\theta_{AB}} \geq 0$, ranges from $-1$ (if $\sin{\theta_{AB}} =0 $), to $0$ (if $\| A \| \to \infty$).  So then $\theta_{BC}$ ranges from $\pi/2$ to $\pi$.    
\item $\| B \| = 1$, $\| B \times A \| =2$.  $C^2 = 4 + 1 =5$  $\Longrightarrow C = \sqrt{5}$
\end{enumerate}

\exercisehead{7} 
\begin{enumerate}
\item $(A \times B)^2 = A^2 B^2 - A\cdot B =1$.  $A,B, A\times B$ normal.  \\
  $A \cdot (A\times B) = B\cdot (A\times B) = 0$ orthonormal.  \\
$c_1 A + c_2 B + c_3 (A\times B) =0$ \\
  $
\begin{aligned}
  & \xrightarrow{A} c_1 A \cdot A = 0 \quad & c_1 = 0 \\
  & \xrightarrow{B} c_2 B^2 = 0 \quad & c_2 = 0 \\
  & \xrightarrow{(A\times B)} c_3 (A\times B)^2 = 0 \quad & c_3  = 0 
\end{aligned}
$ \quad \, $A,B, A\times B$ linearly independent.  By Thm., $A,B, A\times B$ form a basis for $V_3$.  
\item $C = (A\times B)\times A$ \\
  \[
\| ((A\times B) \times A) \| = \sqrt{ \| A \times B\|^2 \| A \|^2 - ((A \times B) \cdot A)^2 } = 1 
\]
\item Remember that $
  \begin{aligned}
    & A\cdot B = A_j B_J = 0 \text{ and } \\
    & A\cdot A = A_j A_J = B\cdot B = B_j B_J = 1
\end{aligned}$ 

\[
\begin{gathered}
\begin{aligned}
  (A \times (A \times B) )_j & = \epsilon_{jkl}A_k (A\times B)_l  = \epsilon_{jkl} A_k \epsilon_{lmn} A_m B_n = -\epsilon_{jkl} \epsilon_{nml} A_k A_m B_n =  \\
  & = \text{ (notice how the $l$ index is repeated) } = -(-A_k A_j B_k) + - (A_k A_k B_j) = 0 - B_j
\end{aligned} \\
\Longrightarrow (A \times (A \times B)) = -B
\end{gathered}
\]

Now $A,B$ are interchangeable labels (the choice of name or label is not special or unique at all), so that we could reuse what we had just proven 
\[
\begin{gathered}
  (A\times B) \times A = B \Longrightarrow (B\times A) \times B = A \\
  \Longrightarrow -(A \times B) \times B = A \text{ or } (A\times B) \times B = -A
\end{gathered}
\]
\end{enumerate}

\exercisehead{8} 
\begin{enumerate}
\item We know that $
  \begin{aligned}
    & A\cdot B = \| A \| \| B \| \cos{ \theta_{AB} } \\
    & \| A \times B \| = \| A \| \| B \| \sin{ \theta_{AB} }
    \end{aligned}$ Certainly, if $A\times B$ is $0$, then its magnitude is zero.  Suppose $\|A\|, \|B\| \neq 0$ and so $\sin{ \theta_{AB}} =0$.  But then $A\cdot B = 0$ and so $\cos{ \theta_{AB}} = 0$.  But $\cos$ and $\sin$ must obey $1= C^2 + S^2$.  So then at least one of $A$ or $B$ must be zero.  
\item Given that $A \neq 0 $ \\
 and  $\begin{aligned}
   & A \times B = A \times C \\
   & A \cdot B = A \cdot C
\end{aligned}$, then $\begin{aligned}
   & A \times (B-C ) = 0 \\
   & A \cdot (B-C) = 0 
\end{aligned}$ so then $B -C = 0$ or $B=C$
\end{enumerate}

\exercisehead{9} Given that $\begin{aligned}
  A & = (2,-1,2) \\
  C & = (3,4,-1)
\end{aligned}$, 
\begin{enumerate}
\item what I did was this: $A \times C = (-7,8,11) = B_p$  \\
$B$ must have some of this $A \times C$ part to twist $A$ into $C$, because $C$ is orthogonal to both $A,B$: ($B$ must not have any part of $C$).  \\
If $B$ has any part of $A$, $A \times c A = 0$ anyways. 
\[
A \times \frac{B_p}{-9} = C \quad \quad \frac{ B_p}{-9} = \left( \frac{7}{9}, \frac{-8}{9}, \frac{-11}{9} \right)
\]
Yes, there's more than one solution, since we could attach scalar multiples of $A$ to $B_p/-9$.  
\item $A \cdot \frac{B_p}{-9} = \frac{ 14 + 8 - 22 }{ 9 } = 0 $ \\
  $A \cdot cA = 1$  $\Longrightarrow c = \frac{1}{9}$  So then the desired vector is $\frac{B_p}{-9} + \frac{1}{9} A$.  Yes, there's only one vector.  
\end{enumerate}

\exercisehead{10} Given that $A\neq =0$ and $C\cdot A = 0$.  Suppose that for $B$, there's $A\times B = C$ and $A\cdot B = 1$.  \smallskip \\
Suppose $\exists \, B_1$ s.t. $A\times B_1 = C$ and $A \cdot B_1 = 1$.  Then \\
$\begin{gathered}
  A \times B_1 = C = A \times B \text{ and } A \cdot B_1 = A \cdot B \\
  \Longrightarrow A \times (B_1 - B) = 0 \text{ and } A \cdot (B_1 - B) = 0 
\end{gathered}$.  Then $B_1 -B = 0$ or $B_1 = B$, from Exercise 8.

\exercisehead{11} Given \\
$\begin{aligned}
  & A = (1,0,1) \\
  & B = (-1,1,1) \\
  & C = (2,-1,2) 
\end{aligned}$ 
\begin{enumerate}
\item \[
\begin{gathered}
  \begin{gathered}
    A = (1,0,1) \\
    \begin{aligned}
      B-A & = (-2,1,0) \\
      C-B & = (3,-2,1) \\
      D-C & = D - (2,-1,2) \\ 
      A-D & = (1,0,1) - D 
\end{aligned} \\
    D- C = -(B-A) \\
    A-D = -(C-B)  \\
    \boxed{ D = (4,-2,2) }
\end{gathered} 
 \quad \quad 
 \begin{gathered}
   A = (1,0,1) \\
   \begin{aligned}
     C-A & = (1,-1,1) \\
     B-C & = -(3,-2,1) \\
     D-B & = D-(-1,1,1) \\
     A- D & = (1,0,1) - D  
   \end{aligned} \\
   D-B = -(C-A) \\
   A-D = -(B-C) \\
   \boxed{ D = (-2,2,0) } 
\end{gathered} 
\quad \quad 
\begin{gathered}
  B = (-1,1,1) \\
  \begin{aligned}
    A - B & = -(-2,1,0) \\
    C-A & = (1,-1,1) \\
    D-C & = D - (2,-1,2) \\
    B-D & = (-1,1,1) - D  
\end{aligned} \\
  D-C = -(A-B) = (-2,1,0) \\
  B-D = -(C-A) = -(1,-1,1) \\
  \boxed{ D = (0,0,2) }
\end{gathered}
\end{gathered}
\]

$3$ is the total number of possible $D$'s, fourth vertex (imagine permutating a 4 vertex ring).    
\item $B-A \times C-A = (-2,1,0) \times (1,-1,1) = (1,2,1)$
\[
\frac{ \| B-A \times C-A \| }{ 2 } = \boxed{ \frac{ \sqrt{6}}{2} }
\]
\end{enumerate}

\exercisehead{12} Given that $A\cdot B = 2$; $\| A \| = 1$, $\| B \| = 4$, and $C = 2 (A\times B) - 3B$
\[
\begin{gathered}
  A \cdot (B + C) = A \cdot B + A \cdot C = A\cdot B - 3A\cdot B = \boxed{ -4 } \\
  \begin{aligned}
  \| C \|^2 & = 4 (A\times B)^2 + 9B^2 = 4 ((1)^2 4^2 \frac{ 8}{4} ) + 9(16) = 16(3+9) = 16\cdot 12 \\
  \| C \| = \boxed{ 8 \sqrt{3} } 
  \end{aligned} \\
  \frac{ B \cdot C }{ \| B \| \| C \| } = \frac{ -3 B^2 }{ BC } = \boxed{ \frac{ -\sqrt{3}}{2} }
\end{gathered}
\]

\exercisehead{13} 
\begin{enumerate}
\item $c_1 ( A+B) +c_2 ( A-B) = (c_1 + c_2 ) A + (c_1 -c_2) B = 0$ \smallskip \\
$\begin{aligned}
  & c_1 + c_2  = 0 
  & c_1 - c_2 = 0 
\end{aligned}$, so then $c_1 = c_2 = 0$.  \medskip \\
$\Longrightarrow A + B , A-B$ linearly independent.  \\

So then $A+B, A-B, A\times B$ must be linearly independent (since $A\times B \perp A+B, A-B$).  
\item $
  \begin{gathered}
    c_1 ( A + B) + c_2 (A + (A \times B) ) + c_3 (B + (A\times B)) =  \\
    = (c_1 + c_2 ) A + (c_1 + c_3 )B + (c_2 + c_3 )(A\times B) = 0 
\end{gathered}$ \smallskip \\
  $\begin{aligned}
    & c_1 + c_2 = 0 \\ 
    & c_1 + c_3 = 0 \\
    & c_2 + c_3 = 0 
\end{aligned}$ \quad $\Longrightarrow c_3 = 0 = c_2 = c_1$.  
\item $(A + B) \times (A-B) = B\times A - A \times B = 2(B\times A)$.  Linearly independent since we already know that $A\times B$ is orthogonal to $A$ and $B$, and $A,B$ are linearly independent.  
\end{enumerate}

\exercisehead{14}
\begin{enumerate}
  \item If $A,B,C$ lie on a line, then $C = A + t(B-A)$.  
\[
(B-A) \times (C-A) = (B-A) \times (t(B-A)) = 0
\]

If $(B-A) \times (C-A) = 0$, then $t(B-A) = C-A$ (otherwise, at least one is zero, and two points have a line through them, and so we'd be done).  Then $C = A + t(B-A)$ and so $C$ would be part of a line joining $A$ and $B$.    
  \item $A \neq B$, \\
    $L = \{ A + t (B-A) \} $
\[
\begin{gathered}
  X = A + t(B-A) = B + s(A- B) \\
  \begin{aligned}
    & X - A  = t(B-A) \\
    & X - B = s(A-B)
  \end{aligned} \quad \quad \Longrightarrow (X-A) \times (X-B) = 0 
\end{gathered}
\]
\end{enumerate}

\exercisehead{15} Given that $A \perp B$, $\| A \| = \| B \| = 1$ and $P \times B = A - P$,
\begin{enumerate}
\item $P \cdot B = (A - P\times B) \cdot B = A\cdot B =0$ \\

\[
\begin{gathered}
  \begin{gathered}
    P^2 = A^2 - 2A \cdot (P\times B) + (P \times B)^2  = A^2  - 2A \cdot ( P \times B ) + P^2 B^2 = 1 - 2A \cdot (P\times B) + P^2  \\
    \Longrightarrow A \cdot ( P \times B) = 1/2 \\
    A \cdot (P\times B) = A \cdot ( A - P) = A^2 - P\cdot A = 1/2 \quad \Longrightarrow P \cdot A = 1/2 = P \cos{ \theta_{PA} }
  \end{gathered} \\
\quad \\
A \times (P \times B) = (A\cdot B)  P - (P\cdot A)B = -(P\cdot A) B = -A \times P \quad \Longrightarrow (P\cdot A) B = (A \times P) \\
(P\cdot A)^2 B^2 = A^2 P^2 \sin^2{\theta_{PA} } = \frac{1}{4} = P^2 \sin^2{ \theta_{PA}} \\
P^2 S^2{ \theta_{PA}} + P^2 C^2{ \theta_{PA}} = \frac{1}{4} + \frac{1}{4} = \frac{1}{2} = P^2 \\ 
\quad \Longrightarrow \boxed{ P = \frac{1}{\sqrt{2}} }
\end{gathered}
\]

Note that we had used the cab-bac rule above, which we'll prove with tensors in the next set of exercises.  
\item $P,B,P\times B$.  \\
  $P\cdot B = 0$, so $P,B$ orthogonal.  $P,B$ orthogonal to $P\times B$.  By thm., $P,B,P\times B$ form a basis (3 orthogonal vectors in $V_3$ form a basis).  
\item 
\[
\begin{aligned}
  ( B \times (B \times P))_j & = \epsilon_{jkl} B_k (B\times P)_l = \epsilon_{jkl} B_k \epsilon_{lmn} B_m P_n = \epsilon_{jkl}\epsilon_{mnl} B_k B_m P_n = \\
  & = B_k B_j P_k - B_k B_k P_j = (B\cdot P) B_j - (B\cdot B) P_j = -P_j \\
  & \Longrightarrow \boxed{ (B\times (B\times P)) = ((P\times B) \times B) = P }
\end{aligned}
\]
\item \[
\begin{gathered}
  \begin{aligned}
    & A \times B = (P \times B) \times B + P \times B = -P + P \times B \\
    & (P \times B) + P = A 
  \end{aligned} \\
P = \frac{ A - (A \times B)}{ 2 } 
\end{gathered}
\]
\end{enumerate}

%-----------------------------------%-----------------------------------%-----------------------------------
\subsection*{ 13.14 Exercises - The scalar triple product, Cramer's rule for solving a system of three linear equations }
%-----------------------------------%-----------------------------------%-----------------------------------
\quad \\
\exercisehead{1}
\begin{enumerate}
\item $\begin{aligned}
  A & = (3,0,0) \\ 
  B & = (0,4,0) \\
  C & = (0,0,8)
\end{aligned}$ \quad $\Longrightarrow \boxed{ 96 }$
\item $\begin{aligned}
  A & = (2,3,-1) \\ 
  B & = (3,-7,5) \\
  C & = (1,-5,2)
\end{aligned}$ \quad $22 - 3 + 8 = \boxed{ 27 }$
\item $\begin{aligned}
  A & = (2,1,3) \\
  B & = (-3,0,6) \\
  C & = (4,5,-1)
\end{aligned}$ \quad $-60 + 21 + 45 = \boxed{ 6 }$
\end{enumerate}

\exercisehead{2} Given \\
$\begin{aligned}
  (0,1,t) \\
  (1,t,1) \\
  (t,1,0) 
\end{aligned}$ \quad $\Longrightarrow t + t(1-t^2) = 0$ or $\boxed{ t= 0, \pm \sqrt{2} }$.  Then, by theorem, since the scalar product is zero, then the vectors are linearly dependent.  

\exercisehead{3} Given \\
$\begin{aligned}
  (1,1,0) \\
  (0,1,1) \\
  (1,0,1)
\end{aligned}$ \quad $\Longrightarrow 1 +1 =\boxed{2}$

\exercisehead{4} Want $A\times B = A\cdot (B\times e_1)e_1 + A \cdot ( B\times e_2)e_2 + A\cdot (B \times e_3)e_3$
\[
\begin{gathered}
\text{ Now } (A \times B)_j = \epsilon_{jkl} A_k B_l \\
\begin{aligned}
  A\cdot (B\times e_j) & = A_l (B\times e_j)_l = A_l \epsilon_{lmn} B_m (e_j)_n = A_l \epsilon_{lmj} B_m = \\
    & = \epsilon_{jlm} A_l B_m = \epsilon_{jkl} A_k B_l
\end{aligned}
\end{gathered}
\]

\exercisehead{5} Prove that $i\times (A\times i) + j \times (A \times j) + k \times (A \times k ) = 2A$.  

We'll show this in two ways to be instructive.  

\[
\begin{gathered}
  i \times (A \times i ) = ? = -i \times (i \times A) \\
  \left| \begin{matrix}
    e_1 & e_2 & e_3 \\
    1 & 0 & 0 \\
    a_1 & a_2 & a_3 
\end{matrix} \right| = (0,-a_3, a_2)  \quad \quad \left| \begin{matrix} 
      e_1 & e_2 & e_3 \\
      1 & 0 & 0 \\
      0 & -a_3 & a_2 
      \end{matrix} \right|
    = (0,-a_2, -a_3) \\
    -i \times (i \times A ) = -i \times (0, -a_3, a_2) = -( -a_3 e_3 - a_2 e_2) = a_3 e_3 + a_2 e_2
\end{gathered}
\]
We could interchange the labels to get the results for $j\times (A \times j)$ and $k \times (A \times k)$.  

We could also use the tensor notation:
\[
\begin{aligned}
(e_j \times (A \times e_j))_l & = \epsilon_{lmn} ((e_j)_m (A \times e_j)_n ) = \epsilon_{lmn}(e_j)_m \epsilon_{noj} A_o e_j =   \\
  & = \epsilon_{ljn} \epsilon_{noj} e_j e_j A_o = \epsilon_{jkl} \epsilon_{jko} e_j e_j A_o = \epsilon_{jkl}\epsilon_{jkl} e_j e_j A_l = \\
  & = A_l \text{ for $l \neq j$ }
\end{aligned}
\]
So for each $j$, we run through $l\neq j$ so that we have the $A_k e_k$ and $A_l e_l$ components.  Then we get $2A$.  

\exercisehead{6} 
\begin{enumerate}
\item $(a,b,c) \cdot ((0,0,1) \times (6,3,4)) = 3 = (a,b,c) \cdot (3,6,0) = -3a + 6b = 3$  \quad $\Longrightarrow -a + 2b = 1 \text{ or } a = 2b -1$.  \smallskip \\
  $\boxed{ (2b-1,b,c) } = b(2,1,0) + (-1,0,c) $
\item We need to minimize $a^2 + b^2 + c^2$.  Since each term in this sum is positive and $c$ is arbitrary, let $c=0$.  
\[
\begin{gathered}
  a^2 + b^2 + c^2 = a^2 + b^2 = (2b-1)^2 + b^2 = 5b^2 -4b + 1 \\
  \Longrightarrow b= 2/5 \, \quad \Longrightarrow a = 2b - 1 = \frac{4}{5} - 1 = -1/5 \\ 
  \Longrightarrow \boxed{ (-1/5,2/5,0) }
\end{gathered}
\]
\end{enumerate}

\exercisehead{7}
\begin{enumerate}
\item $(A+B) \cdot (A+B)\times C = D \cdot (D\times C) = 0$
\item Want: $A\cdot (B\times C) = -B\cdot (A\times C) $ \\
  $(A+B) \cdot (A\times C + B\times C) = 0 \Longrightarrow B\cdot (A\times C) = -A \cdot (B\times C)$
\item Want: $A\cdot (B\times C) = -A \cdot (C \times B)$ \\
  $A\cdot (B\times C) = A \cdot ( -C \times B) = -A \cdot (C\times B)$
\item $A \cdot B\times C = -B \cdot A \times C = B \cdot C \times A = -C \cdot B \times A$
\end{enumerate}

\exercisehead{9} Tensor notation way:
\[
\begin{gathered}
  \begin{aligned}
  (A \times (B\times C))_j & = \epsilon_{jkl} A_k (B\times C)_l = \epsilon_{jkl} A_k \epsilon_{lmn} B_m C_n = \epsilon_{jkl} \epsilon_{lmn} A_k B_m C_ n = \\
    & = \epsilon_{jkl} \epsilon_{mnl} A_k B_m C_n = A_k B_j C_k - A_k B_k C_j = (C\cdot A) B_j - (B\cdot A) C_j
  \end{aligned} \\
\text{ or }  \\
\begin{aligned}
  (i \times (B\times C))_j & = \epsilon_{jkl} (e_1)_k (B\times C)_l = \epsilon_{j1l} e_1 \epsilon_{lmn} B_m C_n = \\
  & = \epsilon_{j1l} \epsilon_{mnl} B_m C_n = B_j C_1 - B_1 C_j \\
  i \times (B\times C) & = c_1 B - b_1 C 
\end{aligned} \\
\text{ more generally } \\
\begin{aligned}
  (e_t \times (B\times C))_j & = \epsilon_{jkl} (e_t)_k (B\times C)_l = \epsilon_{jtl} \epsilon_{lmn} B_m C_n = \\
  & = \epsilon_{jtl} \epsilon_{mnl} B_m C_n = B_j C_t - c_j B_t = c_t B_j - b_t C_j \\
  \Longrightarrow e_t \times (B\times C) = c_t B - b_t C 
\end{aligned} \\
a_1 i \times (B\times C) + a_2 j \times (B\times C) + a_3 k \times (B\times C) = (C\cdot A) B - (B\cdot A) C
\end{gathered}
\]

\exercisehead{10} 
\begin{enumerate}
  \item $(A \times B) \times (C \times D) = (D \cdot A \times B)C - (A\times B\cdot C)D$
  \item \[
\begin{gathered}
  A \times (B \times C) + B \times ( C \times A ) + C \times ( A \times B) = \\
  = (C \cdot A) B - (B\cdot A) C + (B \cdot A) C - (B\cdot C) A + (C \cdot B)A - (C \cdot A) B = 0 
\end{gathered}
\]
  \item Want: $A \times (B \times C) = (A \times B)\times C$ iff $B \times (C \times A) = 0$ \\
    $A \times (B \times C) + B \times (C \times A) + C \times (A \times B) = 0$ (in general).  \\
If $B\times (C \times A) = 0$, $A \times (B \times C) = -C \times (A \times B) = (A \times B) \times C$. \\
If $A \times (B \times C) = (A \times B)\times C$, \\
$
\begin{gathered}
  A \times (B \times C) + B \times (C \times A) + - (A \times B) \times C = (A \times B) \times C - (A \times B)\times C + B \times (C \times A) = \\
  = 0 + B \times (C \times A) = 0 \Longrightarrow B \times (C \times A) = 0
\end{gathered}$
  \item Remember, we can cyclically permute the scalar triple:
\[
\begin{aligned}
  (A \times B) \cdot (C \times D) & = C \cdot ( D \times (A \times B)) = C \cdot ((D\cdot B) A - (D\cdot A) B ) = \\
  & = (B\cdot D)(A\cdot C) - (B\cdot C)(A \cdot D)
\end{aligned}
\]
\end{enumerate}

\exercisehead{11} Given that $A,B,C,D \in V_3$ \quad \quad $\begin{aligned}
  & A \times C \cdot B = 5 \\
  & A \times D \cdot B = 3 \\
  & C +D = (1,2,1) \\
  & C- D = (1,0,-1)
\end{aligned}$
\[
\begin{aligned}
  (A \times B) \times (C \times D) & = (D \cdot (A \times B))C - ((A \times B)\cdot C)D = (B\cdot D \times A)C + (A \times C \cdot B)D = \\
  & = (-B \cdot A \times D)C + (A \times C \cdot B)D = (-3)(1,1,0) + 5 (0,1,1) = (-3,-3,0) + (0,5,5) = \boxed{ (-3,2,5) }
\end{aligned}
\]

\exercisehead{12} 
\[
\begin{gathered}
  \text{ Consider } (A \times B) \cdot (B\times C) \times (C \times A) \\
  (B \times C) \times (C \times A) = (A \cdot (B\times C))C - (C \cdot (B\times C))A = (A \cdot (B \times C))C \\
  \Longrightarrow (A \times B) \cdot (B\times C) \times (C \times A) = (A \cdot (B\times C)) C\cdot (A\times B) = (A \cdot (B \times C))^2
\end{gathered}
\]
In the last step, we cyclically permutated the scalar triple.  

\exercisehead{13} Prove or disprove $A \times (A \times (A \times B)) \cdot C = - A^2 A \cdot B \times C$ 
\[
A \times ( (A \cdot B) A - A^2 B) \cdot C = -A^2 (A \times B)\cdot C = -A^2 C \cdot (A \times B) = -A^2 A \cdot (B\times C)
\]

\exercisehead{14} $A,B,C,D \in V_3$.  
\begin{enumerate}
\item If we consider $A$ to be the origin, then consider the parallelogram with sides $B-A, C-A, D-A \to B,C,D$.  The total volume of the parallelogram is given by  \\
$| (B-A) \cdot (C-A)\times (D-A) = (D-A) \cdot (B-A) \times (C-A) |$.  \\

Note that we have a parallelogram from $B,C,D$ vectors.  $B,C,D$ themselves are $(3+1)$ (1 vertex from $A$, the ``origin'') vertices of this parallelogram, and a parallelogram has 8 vertices.  \medskip \\
We can translate $D$ vector from the $A$ origin to 3 other unique positions - on $B$, on $C$, and on $B+C$ (diagonal across from $A$, containing $A$).  This will generate 3 new tetrahedrons that are congruent to the the $BCD$ tetrahedron, because of either the parallelogram has congruent opposite parallel sides or uses or shares the same vertices and vectors as the $BCD$ tetrahedron.  

Be careful because the parallelogram is \emph{not} ``filled up'', its total volume is \emph{not} accounted for, by these 4 congruent tetrahedrons, even though each tetrahedron shares 3 of its sides each with each of the other 3 tetrahedron.  

Indeed, consider a cube of sides $a$,$a$,$a$.  Consider a tetrahedron made up of its sides.  The diagonal connecting $B$ and $C$, $B-C$, is of length $a\sqrt{2}$.  
\[
\begin{gathered}
\text{ Volume of parallelogram } = 6 \times ( \text{ Volume of a tetrahedron } ) \\
\Longrightarrow \boxed{ \text{ Volume of a tetrahedron } = \frac{1}{6} | (B-A )\cdot (C-A) \times (D-A) | }
\end{gathered}
\]
\item  
\[
\text{ For } \begin{aligned}
  & A = (1,1,1) \\
  & B = (0,0,2) \\
  & C = (0,3,0) \\
  & D = (4,0,0)
\end{aligned} \quad \quad 
\begin{gathered}
  \frac{1}{4} | (B-A) \cdot (C-A) \times (D-A) | = \frac{1}{4} | (-1,-1,1) \cdot (-1,2,-1) \times (3,-1,-1) | = \\
  = \frac{1}{4} | (-1,-1,1) \cdot (-3,-4,-5)| = \boxed{ \frac{1}{2} }
\end{gathered}
\]
\end{enumerate}

\exercisehead{15}
\begin{enumerate}
\item 
\[
\| \frac{ C - B }{ \| C- B \| } \times (A-B) \| = \| A - B \| \sin{\theta_1}
\]
$\theta_1$ is the angle between $C-B/\| C-B \|$ and $A-B$.  From the geometry, it's clear that $\| A- B\| \sin{ \theta_1}$ is the perpendicular distance of $A$ from the line $L = \{ B + t (C-B) \}$.  
\item Given that $A = (1,-2,-5), B = (-1,1,1), C = (4,5,1)$
\[
\| (2,-3,-6) \times (5,4,0) \|/ \| (-5,-4,0)  \|  = \| (24,-30,23) \|/ \sqrt{41} = \boxed{ \sqrt{ 2005/41 } }
\]
\end{enumerate}

\exercisehead{16}
\begin{enumerate}
\item Recall that $(A \times B)_j = \epsilon_{jkl} A_k B_l$ so that 
\[
(A \times B)^2 = \epsilon_{jkl} A_k B_l \epsilon_{jmn} A_m B_n = \epsilon_{jkl} \epsilon_{jmn} A_k A_m B_l B_n = A^2 B^2 - (A\cdot B)^2 
\]
and $-2A\cdot B = \| A - B\|^2 - \|A \|^2 - \| B\|^2$ is immediately apparent by doing the algebra.  Then

\[
\begin{aligned}
  4 S^2 & = 4 (\text{ area of the triangle })^2 = 4 ( \frac{1}{2} \| A \times B \| )^2 = \| A \times B \|^2 = a^2 b^2 - (A\cdot B)^2 = \\
  & = a^2 b^2 - \left( \frac{-1}{2} (c^2 - a^2 - b^2 ) \right)^2 = \boxed{ a^2 b^2 - \frac{1}{4} \left( c^2 -a^2 -b^2 \right)^2 } = \\
  & = \boxed{ \frac{1}{4} (2ab - c^2 + a^2 + b^2)(2ab + c^2 - a^2 - b^2 ) } \text{ (just complete the square) }
\end{aligned}
\]
\item 
\[
\begin{gathered}
  16 S^2 = ((a +b)^2 - c^2)(-(a-b)^2 + c^2) = ((a+b)+c)((a+b) - c)(c - (a-b))(c +(a-b))  \\
  \Longrightarrow \boxed{ S^2 = \frac{1}{16} (a+b+c)(a+b-c)(c-a+b)(c+a-b) }
\end{gathered}
\]
\end{enumerate}

\exercisehead{17}
\[
\begin{gathered}
  \begin{aligned}
    & x + 2y + 3z = 5 \\
    & 2x - y + 4z = 11 \\
    & 0 + -y +z = 3 
\end{aligned} \quad \, \Longrightarrow \begin{aligned}
    & \left( \begin{matrix} 
    1 \\
    2 \\
    0 
\end{matrix} \right) x + \left( \begin{matrix} 
    2 \\
    -1 \\
    -1 
\end{matrix} \right)y  +  \left( \begin{matrix} 
    3 \\
    4 \\
    1 
\end{matrix} \right) z = \left( \begin{matrix}
    5 \\
    11 \\
    3 
\end{matrix} \right) \\
 & Ax + By + Cz = D 
\end{aligned} 
\end{gathered}
\]

Recall that 
\[
x = \frac{DBC}{ABC} \quad \quad y = \frac{DCA}{ABC} \quad \quad z = \frac{DAB}{ABC}
\]

\[
\begin{gathered}
\begin{aligned}
  & A \cdot (B\times C) = (1,2,0) \cdot (3,-5,11) = -7 \\
  & D \cdot (B\times C) = (5,11,3) \cdot (3,-5,11) = -7 \\
  & D \cdot (C \times A) = -D \cdot (2,-1,-2) = -(5,11,3) \cdot (2,-1,-2) = 7 \\
  & D \cdot (A \times B) = D \cdot (-2,1,-5) = (5,11,3) \cdot (-2,1,-5) = -14
\end{aligned} \\
\Longrightarrow \boxed{ x = 1, \, y = -2 , \, z = 2 }
\end{gathered}
\]

\exercisehead{18}
\[
\begin{gathered}
  \begin{aligned}
    & x + y+ 2z = 4 \\
    & 3x - y -z = 2 \\
    & 2x + 5y + 3z  3 
  \end{aligned} \quad \quad \left( \begin{matrix}
1  \\
3 \\
2 
\end{matrix} \right) x + \left( \begin{matrix} 1 \\ -1 \\ 5 \end{matrix} \right) y + \left( \begin{matrix} 2 \\ -1 \\ 3 \end{matrix} \right) z = \left( \begin{matrix} 4 \\ 2 \\ 3 \end{matrix} \right) \\
  \begin{aligned}
   & A \cdot (B\times C) = (1,3,2) \cdot (2,7,1) = 25 \\
   &  D\cdot (B \times C) = (4,2,3) \cdot (2,7,1) =25 \\
    &  D \cdot (C\times A) = -D \cdot (11,1,-7) = -(4,2,3) \cdot (11,1,-7) = -25 
  \end{aligned} \\
  \Longrightarrow \boxed{ x= 1, \, y = -1, \, z =2 }
\end{gathered}
\]

\exercisehead{19} 
\[
\begin{gathered}
  \begin{aligned}
    & x +y = 5 \\ 
    & x+ z = 2 \\
    & y + z =5 
\end{aligned} \quad \quad \left( \begin{matrix} 1 \\ 1 \\ 0 \end{matrix} \right) x + \left( \begin{matrix} 1 \\ 0 \\ 1 \end{matrix} \right) y + \left( \begin{matrix} 0 \\ 1 \\ 1 \end{matrix} \right) z = \left( \begin{matrix} 5 \\ 2 \\ 5 \end{matrix} \right) \\
  \Longrightarrow \boxed{ x = 1= z, \, y = 4 }
\end{gathered}
\]

\exercisehead{20} $\begin{aligned}
  & P = (1,1,1) \\
  & A = (2,1,-1)
\end{aligned}$ \quad $X = P + tA $ \quad $\Longrightarrow \left( \begin{matrix} x \\ y \\ z \end{matrix} \right) = \left( \begin{matrix} 1 \\ 1 \\ 1 \end{matrix} \right) + t \left( \begin{matrix} 2 \\ 1 \\ -1 \end{matrix} \right)$.  \medskip \\
Note that $X \cdot N = P\cdot N + tA \cdot N = P\cdot N$.  

\[
  \begin{aligned}
    x - y + z & = 1 \\
    x + y + 3z & = 5 \\
    3x +y + 7z & = 11 
  \end{aligned} \quad \quad \Longrightarrow \begin{aligned}
    (1,-1,1) \cdot X & = 1 \\
    (1,1,3) \cdot X & = 5 \\
    (3,1,7) \cdot X & = 11 
\end{aligned}
\]
We could immediately then see that for all $X = P +tA$, $X$ satisfies this system of linear equations because $X\cdot N = P\cdot N$ and $P\cdot N$ is equal to the left hand side (LHS) of each of the equations in this system.  Also, $A \cdot N = 0$ for each of the normal vectors, as expected.  

%-----------------------------------%-----------------------------------%-----------------------------------
\subsection*{ 13.17 Exercises - Normal vectors to planes, Linear Cartesian equations for planes }
%-----------------------------------%-----------------------------------%-----------------------------------
\quad \\
\exercisehead{1} Given $\begin{aligned}
  A & = (2,3,-4) \\
  B & = (0,1,1) 
\end{aligned}$
\begin{enumerate}
  \item $A \times B = (7,-2,2)$ 
  \item $\Longrightarrow 7x -2y + 2z = 0$
  \item $7x -2y + 2z = 9$
\end{enumerate}

\exercisehead{2} $x + 2y - 2z + 7 = 0$ or $x + 2y - 2z = -7$.  
\begin{enumerate}
\item $\frac{1}{3} (1,2,-2) $
\item $ \frac{ x}{-7} - \frac{2y}{7 } + \frac{ 2z }{7} = 1 $ so the intercepts are $\begin{aligned}
  & (-7,0,0) \\
  & (0,-7/2,0) \\
  & (0,0,7/2)
\end{aligned}$
\item $\frac{N}{|N|} \cdot P = d$ \quad $\Longrightarrow \frac{1}{3} (1,2,-2)\cdot ( 0,0,7/2) = \boxed{ 7/3 }$
\item $\boxed{ -7/9 (1,2,-2)}$
\end{enumerate}

\exercisehead{3} $(1,2,-3)$ is in $3x - y + 2z =-5$ and $(2,0,-1)$ is in $3x -y + 2z =4$.  \\
$\frac{N}{ |N|} = \frac{(3,-1,2)}{ \sqrt{14}}$ \\
$\frac{N}{ |N|} \cdot P_1 = \frac{ (3,-1,2)}{ \sqrt{14}} \cdot (1,2,-3) = \frac{-5}{\sqrt{14}}$ \\
$\frac{N}{ |N| } \cdot P_2 = \frac{ (3,-1,2) }{ \sqrt{14} } \cdot (2,0,-1) = \frac{4}{\sqrt{14}}$ \medskip \\
So the distance between the two planes is $\boxed{ \frac{9}{ \sqrt{14}} }$

\exercisehead{4} 
\begin{enumerate}
  \item 
\[
\begin{aligned}
  x + 2y - 2z & = 5 \\
  3x - 6y + 3z & = 2 \\
  2x + y + 2z & = -1 \\
  x - 2y + z & = 7 
\end{aligned}
\quad \quad 
\begin{aligned}
  & (1,2,-2)  \\
  & 3(1,-2,1) \\ 
  & (2,1,2) \\
  & (1,-2,1)
\end{aligned}
\]
So by definition, the second and fourth are parallel to each other and the first and the third are perpendicular to each other, since $(1,2,-2)\cdot (2,1,2) = 0$
  \item $N = \frac{ (1,-2,1) }{\sqrt{6}}$.  $(0,-1/3,0)$ is on the second plane, $3x-6y+3z = 2$, and $(3,0,4)$ is on the fourth plane, $x-2y + z = 7$.  \\
\[
\begin{aligned}
  & N \cdot (0,-1/3,0) = \frac{ 2/3}{\sqrt{6}} \\
  & N \cdot (3,0,4) = \frac{ 7 }{ \sqrt{6}}
\end{aligned} \quad \quad \Longrightarrow 7 /\sqrt{6} - \frac{ 2/3}{\sqrt{6}} = \frac{19/3}{\sqrt{6}}
\]
\end{enumerate}

\exercisehead{5} Given $(1,1,-1),(3,3,2),(3,-1,-2)$, the spanning vectors for a plane through all three points is \\
$ (2,2,3), (2,-2,-1)$ \\
\begin{enumerate}
\item The normal vector to this plane is $(4,8,-8)$
\item $x + 2y - 2z =5$
\item $\frac{ (1,2,-2)}{3} \cdot (1,1,-1) = 5/3$
\end{enumerate}

\exercisehead{6} Given $(1,2,3), (2,3,4), (-1,7,-2)$, the spanning vectors for a plane through all three points are \\
$(1,1,1), (-2,5,-5)$ \\
The normal vector to this plane is $(-10,3,7)$.  \\
Then the Cartesian equation could easily be obtained: $-10x + 3y + 7z = 17$

\exercisehead{7} 
\[
\begin{gathered}
\begin{aligned}
  x+ y & = 1 \\
  y + z & = 2 
\end{aligned} \quad \quad \begin{aligned}
  & (1,1,0) \\
  & (0,1,1)
\end{aligned} \\
\frac{1}{ \sqrt{2} \sqrt{2}} = \frac{1}{2} = \cos{ \theta} \Longrightarrow \boxed{ \theta = \pi/3 \text{ or } 2\pi /3 }
\end{gathered}
\] 

\exercisehead{8} $(1,2,9)$ \quad $\Longrightarrow x + 2y + 9z = -55$

\exercisehead{9} $4x - 3y + z =5 \to (4,-3,1)$.  $(4,-3,1)$ is perpendicular to this plane.  \\
\[
\left( \begin{matrix} x \\ y \\z \end{matrix} \right) = \left( \begin{matrix} 2 \\ 1 \\ -3 \end{matrix} \right) + s \left( \begin{matrix} 4 \\ -3 \\ 1 \end{matrix} \right)
\]

\exercisehead{10} $X(t) = (1-t,2-3t,2t-1)$
\begin{enumerate}
\item $X(t) = \left( \begin{matrix} 1 \\ 2 \\ -1 \end{matrix} \right) + t \left( \begin{matrix} -1 \\ -3 \\ 2 \end{matrix} \right)$
\item $(-1,1,1)$
\item Plug in $(1,2,1)$ and $t(-1,-3,2)$ into $2x + 3y + 2z$ to see which $t$ will give $-1$ on the right hand side.  $t=1$
\item $X(3) = (-2,-7,5)$.  For a plane to be parallel to another plane, then they each have the same normal vector.  \\
  $\Longrightarrow (2,3,2)$.  So the Cartesian equation is $\boxed{ 2x +3y + 2z = -15 }$  
\item $X(2) = (-1,-4,3)$.  So then a plane perpendicular to $L$ and through $X(2)$ would be \\
$-x -3y + 2z = 19$
\end{enumerate}

\exercisehead{11} Try $N = (n_1,n_2,n_3)$.  
\[
\begin{gathered}
  \begin{aligned}
    \frac{ N \cdot e_1 }{ |N| } = \frac{ n_1 }{ \sqrt{ n_1^2 + n_2^2 + n_3^2 } } & = \frac{1}{2} \\
    \frac{ n_3 }{ \sqrt{ n_1^2 + n_2^2 + n_3^2 } } & = \frac{1}{2} 
   \end{aligned} \quad \, \Longrightarrow n_1  = n_3 \\
  \quad \\
  \frac{n_2}{ \sqrt{ 2 n_1^2 + n_2^2 } } = \frac{1}{ \sqrt{2}} \Longrightarrow \frac{n_1 }{n_2 } = \frac{1}{\sqrt{2}} \\
  N = \frac{ (1,\sqrt{2}, 1 )}{ 2 } \\
  \Longrightarrow \boxed{ x + \sqrt{2} y + z = 2 + \sqrt{2} }
\end{gathered}
\]

\exercisehead{12} 
\[
\begin{gathered}
  \begin{aligned}
    & x + 2y + 3 z = 6 \\
    & \frac{x}{6} + \frac{y}{3} + \frac{z}{2} = 1 
  \end{aligned} \\
  \quad \\
  \begin{aligned}
    & (6,0,0) \\
    & (0,3,0) \\
    & (0,0,2) 
  \end{aligned} \quad \Longrightarrow \frac{36}{6} = \boxed{ 6 }
\end{gathered}
\]

\exercisehead{13} $x-y+5z = 1$ Then two linearly independent spanning vectors for this could be \\
$\begin{aligned}
  & (1,1,0) \\
  & (0,5,1) 
\end{aligned}$.  

We want this vector to be perpendicular to $(1,2,-3)$:
\[
\begin{gathered}
(a(1,1,0) + b(0,5,1)) \cdot (1,2,-3) = (a + 2(a+5b) - 3b) = 3a + 7b = 0 \quad \Longrightarrow b = -3a/7 \\
  \text{ so }
  a((1,1,0) + \frac{-3}{7} (0,5,1) ) = a ( 1 , \frac{-8}{7}, \frac{-3}{7} ) \Longrightarrow \boxed{ \frac{1}{\sqrt{122}} (7,-8,-3 ) }
\end{gathered}
\]

\exercisehead{14}  Given a plane that needs to span with $\begin{aligned}
  & (1,1,0) \\
  & (0,1,1)
\end{aligned}$ since it needs to be parallel to these two vectors, and intercept the intercept $(2,0,0)$ \medskip \\
$(1,1,0) \times (0,1,1) = N = (1,-1,1)$ \quad $\Longrightarrow \boxed{ x - y +z = 2 }$

\exercisehead{15}
\[
\begin{gathered}
  \begin{aligned}
    3x + y+ z & = 5 \\
    3x + y + 5z & = 7 \\
    x - y + 3z & = 3 
  \end{aligned} \quad \, \Longrightarrow \left( \begin{matrix} 3 \\ 3 \\ 1 \end{matrix} \right) x + \left( \begin{matrix} 1 \\ 1 \\ -1 \end{matrix} \right) y + \left( \begin{matrix} 1 \\ 5 \\ 3 \end{matrix} \right) z = \left( \begin{matrix} 5 \\ 7 \\ 3 \end{matrix} \right) \\
  (1,1,-1) \times (1,5,3) = (8,-4,4) \Longrightarrow A \cdot B \times C  = 16 \\
  x = \frac{ D \cdot B \times C }{ A \cdot B \times C } = \frac{24}{18} = \frac{3}{2} \\
  \quad \\
  (1,5,3) \times (3,3,1) = (-4,8,-12) \Longrightarrow y = 0 \quad \, \text{ then } z = 1/2 \\
  \boxed{ x = 3/2, y = 0, z = 1/2 }
\end{gathered}
\]

\exercisehead{16} We could try to solve for a system of 3 linear equations that represents the 3 planes:
\[
\begin{gathered}
  \begin{aligned}
    & a_1 x + b_1 y + c_1 z = d_1 \\
    & a_2 x + b_2 y + c_2 z = d_2 \\
    & a_3 x + b_3 y + c_3 z = d_3
  \end{aligned} \\
  \text{ If } a_1, a_2, a_3 \neq 0 \\
  \Longrightarrow   \begin{aligned}
    &  x + b_1 y + c_1 z = d_1 \\
    &  x + b_2 y + c_2 z = d_2 \\
    &  x + b_3 y + c_3 z = d_3
  \end{aligned} \Longrightarrow \left[ \begin{matrix}
      1 & b_1 & c_1 \\
      1 & b_2 & c_2 \\
      1 & b_3 & c_3 
\end{matrix} \right. \left| \begin{matrix} d_1 \\ d_2 \\ d_3 \end{matrix} \right]
\end{gathered}
\]
Since the normals of the planes are linearly independent, this means that 
\[
\left[ \begin{matrix}
    1 & 1 & 1 \\
    b_1 & b_2 & b_3 \\
    c_1 & c_2 & c_3 
\end{matrix} \right. \left| \begin{matrix} 0 \\ 0 \\ 0 \end{matrix} \right]
\]
has a unique solution and we'll be left with the identity matrix.  Then for our problem above, we can do the same operations, since the equations are linear, and leave the LHS to be an identity matrix.  Then we've solved for a unique point in $\mathbb{R}^3$

\exercisehead{17} Given a line through $(1,2,3)$ and parallel to the two planes given by $x + 2y + 3z=4$ and $2x + 3y + 4z =5$, to find a direction vector for the line that is parallel to both planes, determine a vector that would make each of the two previous equations equal to $0$ on the LHS.  $(1,-2,1)$, plugged into the LHS of each of the 2 equations, would make it equal $0$.  The line is through $(1,2,3)$ so then $\boxed{ X = (1,2,3) + t (1,-2,1) }$

\exercisehead{18} So given a line $L$, s.t. $X = P + u A$ and plane $M$, s.t. $X = Q + sB + tC$, and that \\
$L \nparallel M$, so that $sB + tC \nparallel u A \quad \, \forall \, s, t, u \in \mathbb{R}$,  \\

Consider $L\bigcap M$.  Consider $X_1 \in L \bigcap M$
\[
\begin{gathered}
  X_1 = P + t_1 A \\
  X_1 - P = t_1 A \Longrightarrow X = P + tA = P + (X_1 - P) + (t-t_1) A = X_1 + t A \quad \text{ (since $t\in \mathbb{R}$ ) }
\end{gathered}
\]
Since $X_1 \in M$ as well, 
\[
\begin{gathered}
  X_1 = Q + a_1 B + b_1 C \\
  X = Q + sB + tC = Q + a_1 B + b_1 C + (s-a_1)B + (t-b_1)C = X_1 + sB + t C \\
  \text{ so we have } \\
  \begin{aligned}
    X_L & = X_1 + tA \\
    X_M & = X_1 + sB + tC 
  \end{aligned}
\end{gathered}
\]
Suppose $X_2 \in L \bigcup M$
\[
\begin{gathered}
  \begin{aligned}
    X_2 & = X_1 + u_2 A \\
    X_2 & = X_1 + s_2 B + t_2 C 
  \end{aligned} \quad \quad \Longrightarrow u_2 A = s_2 B + t_2 C \\
  \text{ but } sB + tC \neq u A \quad \, \forall \, s,t,u \in \mathbb{R} \\
  \text{ Contradiction, so $X_2 = X_1$ }
\end{gathered}
\]

\exercisehead{19}
\begin{enumerate}
\item Given $ax + by + cz = -d$, then the normal vector for this plane, that's normalized, is $\frac{ N}{|N|} = \frac{ (a,b,c)}{ \sqrt{ a^2 + b^2 + c^2 } }$.  \smallskip \\
Note that the Cartesian equation for the plane is simply $X \cdot N = P \cdot N$. \smallskip \\
Then the perpendicular distance from the point $X_0$ to a point on the plane, $P$, is given by $(X_0 - P) \cdot N = X_0 \cdot N - P \cdot N$.  
\[
\text{ dist. } = \frac{ |a x_0 + b y_0 + cz_0 + d | }{ \sqrt{ a^2 + b^2 + c^2 } }
\]
\item $5x - 14y + 2z = -9$.  The normal to this plane is $\frac{N}{|N|} = e_n = \frac{ (5,-14,2)}{15} $.  \smallskip \\
  The perpendicular distance from $Q$ to plane is $\text{ dist. } = \frac{ (Q-P)\cdot N}{ |N|} = \frac{ -234 + 9}{15 } = -15$ 

So to get this point on the plane, we simply have to go from $Q$ and go $+15$ distance along the normalized normal vector $e_n$:
\[
(-2,15,-7) + 15 e_n = \boxed{ (3,1,-5) }
\]
\end{enumerate}

\exercisehead{20} Given $2x-y +2z +4 =0$, a plane parallel to this plane will have the same normal vector, $\frac{N}{|N|} = e_n = \frac{ (2,-1,2)}{3}$.  \smallskip \\
The distance from $Q$ to this plane is 
\[
\frac{ (Q-P)\cdot N }{ |N| } = \frac{ (3,2,-1)\cdot (2,-1,2) - (-4) }{ 3 } = (2+4)/3 = 2 
\]
To find the other plane, simply go a distance $2$ along $e_n$ from $Q$, to get a point on that other plane.  
\[
\begin{gathered}
  (3,2,-1) + 2 \left( \frac{ (2,-1,2) }{ 3 } \right) = \frac{ (9,6,-3) + (4,-2,4) }{ 3 } = \frac{ (13,4,1) }{3} = P_2 \\
  2x -y + 2z = \frac{ 26}{3} - \frac{4}{3} + \frac{2}{3} = P_2 \cdot N = 8 \quad \Longrightarrow \boxed{ 2x - y + 2z = 8 }
\end{gathered}
\]

\exercisehead{21} 
\begin{enumerate}
\item Given 3 points $A,B,C$, they determine a plane $\Longrightarrow M = \{ A + s(B-A) + t (C-A) \}$ \medskip \\
The normal vector, normalized, to this plane is $\frac{N}{|N|} = e_n = \frac{ (B-A) \times (C-A) }{ |(B-A) \times (C-A) | }$
\[
\frac{ (Q-P)\cdot N}{ |N| } \xrightarrow{ P = A } \frac{ (Q-A) \cdot (B-A) \times (C-A) }{ | (B-A) \times (C-A) | }
\]
\item Given \\
  $
  \begin{aligned}
    & Q = (1,0,0) \\
    & A = (0,1,1) \\
    & B = (1,-1,1) \\
    & C = (2,3,4)
\end{aligned}
$ \quad \, $\begin{aligned}
    & Q - A = (1,-1,-1) \\
    & B - A = (1,-2,0) \\
    & C - A = (2,2,3)
\end{aligned}$ 
\[
\frac{ (Q-A) \cdot (B-A) \times (C-A) }{ | (B-A)\times (C-A) | } = \frac{ (1,-1,-1) \cdot (-6,-3,6) }{ 9 } = \boxed{ - 1 }
\]
\end{enumerate}

\exercisehead{22} $M, \, M'$ not parallel.  $e_{N_m} \neq e_{N_{m'}}$. 
\[
\begin{gathered}
\begin{aligned}
  & x \in M \\
  &  x \in M'
\end{aligned} \quad \, \begin{aligned}
  & x + b_1 y + c_1 z = d_1 
  & x + b_2 y + c_2 z = d_2
\end{aligned} \quad \, \Longrightarrow (1,b_1,c_1) \neq (1,b_2,c_2) \text{ so that either } \\
b_1 \neq b_2 \text{ or } c_1 \neq c_2
\end{gathered}
\]
Let's solve for $\begin{aligned}
  & x + b_1 y + c_1 z = 0 \\
  & x + b_2 y + c_2 z = 0
\end{aligned}$ \quad \medskip \\
If $b_2 - b_1 \neq 0$,
\[
\begin{gathered}
  \left[ \begin{matrix} 1 & b_1 & c_1 \\
      1 & b_2 & c_2 
      \end{matrix} \right] \Longrightarrow \left[ \begin{matrix} 1 & 0 & \frac{ c_1 b_2 -b_1 c_2 }{ b_2 - b_1 } \\
      & b_2 -b_1 & c_2 - c_1 \end{matrix} \right] \\
  \Longrightarrow (x,y,z) = z \left( \frac{ b_2 c_1 - c_2 b_1}{ b_1 - b_2 }, \frac{ - (c_2 -c_1) }{ b_2 - b_1 }, 1 \right)
\end{gathered}
\]
Otherwise, if $c_2 - c_1 \neq 0$
\[
\begin{gathered}
  \left[ \begin{matrix} 1 & b_1 & c_1 \\
      0 & b_2 - b_1 & c_2 - c_1 
      \end{matrix} \right] \Longrightarrow \left[ 
    \begin{matrix} 
      1 & \frac{ b_1 ( c_2 - c_1 ) }{ c_2 -c_1 } - \left( \frac{ b_2 - b_1 }{ c_2 -c_1 } \right) c_1 & 0 \\
      0 & \frac{ b_2 -b_1 }{c_2 -c_1 } & 1 
\end{matrix} \right] \\
  \Longrightarrow (x,y,z) = y \left( \frac{ b_2 c_1 - b_1 c_2 }{ c_2 -c_1 }, 1 , \frac{ -(b_2 -b_1) }{c_2 - c_1} \right)
\end{gathered}
\]

So these two equations with RHS being $0$ result in a solution that's completely determined, give a multiplicative factor.  We can now find at least one point in the intersection by solving the system of equations \\

$\begin{aligned}
  & x + b_1 y + c_1 z = d_1 \\
  & x + b_2 y + c_2 z = d_2
\end{aligned}$
  If $b_2-b_1 \neq 0$,
\[
\begin{gathered}
  \left[  \begin{matrix} 1 & b_1 & c_1 \\
      1 & b_2 & c_2 
      \end{matrix} \right. \left| \begin{matrix} d_1 \\ d_2 \end{matrix} \right] \Longrightarrow \left[ \begin{matrix} 1 & 0 & \frac{ c_1 b_2 -b_1 c_2 }{ b_2 - b_1 } \\
      & 1 & \frac{ c_2 - c_1 }{b_2 - b_1 } \end{matrix}\right. \left| \begin{matrix} \frac{ b_2 d_1 - b_1 d_2 }{ b_2 -b_1 } \\ \frac{ d_2 - d_1 }{ b_2 - b_1 } \end{matrix} \right] \\
  \Longrightarrow \text{ If $z=1$, } (x,y,z) =  \left( \frac{ b_2 d_1-b_1 d_2 -(b_2 c_1 - c_2 b_1)}{ b_2 - b_1 }, \frac{ d_2 -d_1 - (c_2 -c_1) }{ b_2 - b_1 }, 1 \right)
\end{gathered}
\]
Otherwise, if $c_2 - c_1 \neq 0$
\[
\begin{gathered}
  \left[ \begin{matrix} 1 & b_1 & c_1 \\
      0 & b_2 - b_1 & c_2 - c_1 
      \end{matrix} \right. \left| \begin{matrix} d_1 \\ d_2-d_1 \end{matrix} \right] \Longrightarrow \left[ 
    \begin{matrix} 
      1 & \frac{ b_1 c_2 - c_1 b_2 }{ c_2 -c_1 }  & 0 \\
      0 & \frac{ b_2 -b_1 }{c_2 -c_1 } & 1 
\end{matrix} \right. \left| \begin{matrix} \frac{ d_1 c_2 - c_1 d_2 }{c_2 - c_1 } \\
    \frac{d_2 - d_1 }{c_2 - c_1 } \end{matrix} \right] \\
  \Longrightarrow \text{ If $y=1$, } (x,y,z) = \left( \frac{ d_1 c_2 - c_1 d_2 - (b_1 c_2 -c_1 b_2 ) }{c_2 -c_1}, 1, \frac{ d_2 - d_1 - (b_2 -b_1) }{ c_2 -c_1 } \right)
\end{gathered}
\]

Thus, it's always possible to find a point in $M \bigcap M'$.  This intersection is spanned by only one vector, give a multiplicative factor, which we have found above.  Then the intersection $M \bigcap M'$ of 2 planes is a line.  

\exercisehead{23} Given $\begin{aligned}
  x + 2 y + 3z  & = 4 \\
  2x + y + z  & = 2 
\end{aligned}$ 
\[
\begin{gathered}
  \left[ \begin{matrix} 1 & 2 & 3 \\
      2 & 1 & 1 \end{matrix} \right] \Longrightarrow \left[ \begin{matrix} 1 & 0 & -1/3 \\
      0 & 1 & 5/3 \end{matrix} \right] \\
  \Longrightarrow \left( \frac{1}{3}, \frac{-5}{3}, 1 \right)
\end{gathered}
\]
Let's find a point on the intersection, which is a line, for the two planes: 
\[
\begin{gathered}
  \left[ \begin{matrix} 1 & 2 & 3 \\
    2 & 1 & 1 \end{matrix} \right. \left| \begin{matrix} 4 \\ 2 \end{matrix} \right] \Longrightarrow \left[ \begin{matrix} 1 & 0 & -1/3 \\
      0 & 1 & 5/3 \end{matrix} \right. \left| \begin{matrix} 0 \\ 2 \end{matrix} \right] \\
  \Longrightarrow k(1,-3,3) 
\end{gathered}
\]
The normal to this plane containing the intersection line and is parallel to $e_2=j$ is \smallskip \\
$ (0,1,0) \times (1/3,-5/3,1) = (1,0,1/3)$  \\

Then $\boxed{ x+ \frac{1}{3} z = 2 }$

\exercisehead{24} Given two equations for the two planes, \\
$\begin{aligned}
  x+y & = 3 \\
  2y + 3z & = 4 
\end{aligned}$, we can immediately get the spanning vector for the line that is the intersection of these planes: $(-1,1,-2/3)$.  \bigskip \\
Also, by inspection, we can find a point on the intersection of these 2 planes: $(1,2,0)$.  

We want our plane to be parallel to $(3,-1,2)$.  We can now determine the normal to this plane: 
\[
\begin{gathered}
  (3,-1,2) \times (-1,1,-2/3) = \left( \frac{-4}{3},0,2 \right) \\
  \Longrightarrow \frac{-4}{3} x + 2 z = \frac{-4}{3}
\end{gathered}
\]


%-----------------------------------%-----------------------------------%-----------------------------------
\subsection*{ 13.21 Exercises - The conic sections, Eccentricity of conic sections, Polar equations for conic sections } 
%-----------------------------------%-----------------------------------%-----------------------------------
\quad \\ 

\exercisehead{1} $F$ is in the positive half-plane determined by $N$.  
\[
  \begin{aligned}
    \left\| X - F \right\| &= e d(X,L)  \\
    \left\| X- F \right\| & = e | (X-F) \cdot N + d | 
  \end{aligned} \\
\]

\exercisehead{2} 
\begin{enumerate}
\item \[
\begin{gathered}
  \begin{aligned}
    \left\| X - F \right\| &= e d(X,L)  \\
    \left\| X- F \right\| & = e | (X-F) \cdot N + d | 
  \end{aligned} \\
  F = 0 \Longrightarrow \left\| X \right\| = e |(X\cdot N) + d |; \quad r = e|r\cos{\theta} +d | \\
  \Longrightarrow r=e(r\cos{\theta} + d ) \Longrightarrow r = \frac{ed}{ 1 - e\cos{\theta}}
\end{gathered}
\]
\item The right branch for the hyperbola is given by $r = \frac{ ed}{ 1 - e \cos{\theta}}$ because $X\cdot N > 0 $.  
The left branch for $e>1$,
\[
\begin{gathered}
  \begin{aligned}
    \left\| X - F \right\| & = ed(X,L) = e |(X-F) \cdot N + d | = \\
    & = e | X\cdot N + d | = -e(d + r\cos{\theta}) = r 
  \end{aligned} \\
  r = \frac{ -ed}{ (1+ e\cos{\theta})}
\end{gathered}
\]
\end{enumerate}

\exercisehead{3} For points below the horizontal directrix, 
\[
\begin{gathered}
  \left\| X - F \right\| = e d(X,L) \\
  F = 0 \Longrightarrow \left\| X \right\| = e d(X,L) = e ( |(X-F)\cdot N - d |) = e|X\cdot N - d| = e |r\sin{\theta} - d | \\
  \text{ Now Thm. 13.18 says } r = \frac{ed}{ e\cos{\theta} + 1 } \quad \text{ if } 0 < e \leq 1 \\
  \Longrightarrow r = e(d-r\sin{\theta}) \Longrightarrow r = \frac{ed}{1+e\sin{\theta}}
\end{gathered}
\]

For the ``right'' or upper-half branch of a hyperbola.  
\[
\begin{gathered}
  \left\| X - F \right\| = e ( |(X-F) \cdot N - d | ) = e(r\sin{\theta} - d) = r \\
  r = \frac{ -ed}{ 1-e\sin{\theta}} = \frac{ed}{ e\sin{\theta} - 1 }
\end{gathered}
\]

\exercisehead{4} $\left\| X -F \right\| = e d(X,L)$; $\left\| X \right\| = e | (X-F)\cdot N - d | = e | r\cos{\theta} - d | = e(d-r\cos{\theta})$ $\quad \Longrightarrow r = \frac{ed}{ 1 + e\cos{\theta}}$

$\boxed{ e = 1, d=2}$.  


\exercisehead{5} $r = \frac{3}{ 1 + \frac{1}{2} \cos{\theta}} = \frac{ 6 \left( \frac{1}{2} \right)}{ 1 + \frac{1}{2} \cos{\theta}}$.  $e = \frac{1}{2}; \, d=6$  

\exercisehead{6} $ r = \frac{6}{ 3 + \cos{\theta}} = \frac{2}{ 1 + \frac{1}{3} \cos{\theta}}$.  $e=\frac{1}{3}; \, d =6$.  

\exercisehead{7} $ r = \frac{1}{ \frac{-1}{2} + \cos{\theta}}$.  
\[
\begin{gathered}
  ed(X,L) = \left\| X -F \right\| = e |(X-F)\cdot N -d | = e|r \cos{\theta} - d | = er\cos{\theta} -ed = r \\
  \frac{-ed}{ 1 - e \cos{\theta}} = r = \frac{ed}{ e\cos{\theta} - 1 }
\end{gathered}
\]
So for $r = \frac{2}{ 2\cos{\theta} - 1 }, \quad e =2, d = 1 $.  

\exercisehead{8} $r= \frac{4}{1+2\cos{\theta}} \quad e=2, \, d=2 $.  

\exercisehead{9} $r = \frac{4}{ 1+\cos{\theta}} \quad e = 1 \, d=4$.  

\exercisehead{10} $3x + 4y = 25 \Longrightarrow \frac{3}{5} x + \frac{4}{5} y = 5$.  $N = \left( \frac{3}{5}, \frac{4}{5} \right)$.  \medskip \\
$L  = \{ x = P +tA \}, \quad N \cdot X = N \cdot P$.  

To find the distance from the focus, at the origin, to the directrix, 
\[
dN = P + tA; \quad d N \cdot N = d = N \cdot P
\]
So for this problem, $d = 5$.  
\[
\begin{gathered}
  r = \left\| X -F \right\| = ed(X,L) = e| (X-F) \cdot N - d | = e | X \cdot N - d | = \left| \frac{3}{5} r \cos{\theta} + \frac{4}{5} r\sin{\theta} -5 \right| \\
  r = \frac{1}{2} \left( 5 - \frac{3}{5} r \cos{\theta} - \frac{4}{5} r \sin{\theta} \right) \\
  r = \frac{5/2}{ 1 + \frac{3}{10} \cos{\theta} + \frac{4}{10} \sin{\theta}}
\end{gathered}
\]

\exercisehead{11} $e=1, \quad 4x + 3y = 25 \quad \frac{4}{5} x + \frac{3}{5} y = 5; \quad N = \left( \frac{4}{5}, \frac{3}{5} \right)$.  

$d=5$.  
\[
\begin{gathered}
  \left\| X - F \right\| = ed(X,L) = e| (X-F)\cdot N -d | = e \left| \frac{4}{5} r \cos{\theta} + \frac{3}{5} r \sin{\theta} -5 \right| \\
  r = 5 - r \left( \frac{4}{5} \cos{\theta} + \frac{3}{5} \sin{\theta} \right) \\
  r = \frac{5}{ 1+ \frac{4}{5} \cos{\theta} + \frac{3}{5} \sin{\theta}}
\end{gathered}
\]

\exercisehead{12} $e=2$, hyperbola, so there's 2 branches.  
\[
\begin{gathered}
  \frac{1}{ \sqrt{2}} x  + \frac{1}{ \sqrt{2}} y = \frac{1}{ \sqrt{2}} \quad L = \{ x = P + tA \} \quad X\cdot N = N \cdot P \\
  dN = P + tA; \quad dN\cdot N = d = N \cdot P = \frac{1}{ \sqrt{2}} 
\end{gathered}
\]
Note that the sign of $d$ here tells you what side the focus, at the origin, lies on.  
\[
\begin{gathered}
  \left\| X -F \right\| = ed(X,L) = \left\| X \right\| = e | (X-F)\cdot N  - d | = e (d - \frac{1}{\sqrt{2}} r \cos{\theta} - \frac{1}{ \sqrt{2}} r \sin{\theta} ) \\
  r = \frac{ 2/\sqrt{2} }{ 1 + \frac{2}{\sqrt{2}} \cos{\theta} + \frac{2}{\sqrt{2}} \sin{\theta}}
\end{gathered}
\]

But for the right side branch,
\[
\begin{gathered}
  \left\| X -F \right\| = ed(X,L) = \left\| X \right\| = e | (X-F)\cdot N  - d | = -e (d - \frac{1}{\sqrt{2}} r \cos{\theta} - \frac{1}{ \sqrt{2}} r \sin{\theta} ) \\
  r = \frac{ -2/\sqrt{2} }{ 1 - \frac{2}{\sqrt{2}} \cos{\theta} - \frac{2}{\sqrt{2}} \sin{\theta}}
\end{gathered}
\]

\exercisehead{13} $e=1$ parabola.  
\begin{enumerate}
\item 
\[
\begin{gathered}
  \left\| X - F \right\| = \left\| X \right\| = ed(X,L) = 1 | (X-F) \cdot N - d | = d - X\cdot N = d -r\cos{\frac{\pi}{3}} \\
  d = r \left( \frac{3}{2} \right) = \frac{3}{2} \times 10^8 mi \\
  \boxed{ r = \frac{ \frac{3}{2} \times 10^8 mi }{ 1 + \cos{\theta}} } \quad \quad \theta = 0, \quad r = \frac{3}{4} \times 10^6 mi
\end{gathered}
\]
\item Focus is in the positive half-plane determined by $N$.  
\[
\begin{gathered}
  \left\| X - F \right\| = \left\| X \right\| = e d(X,L) = | (X-F)\cdot N + d | = r\cos{\theta} + d \\
  d = r(1-\cos{\theta}) = 10^8 mi (1-\cos{\frac{\theta}{3} } ) = \frac{1}{2} \times 10^8 mi \\
  \boxed{ r = \frac{d}{1-\cos{\theta}} = \frac{ \frac{1}{2} \times 10^8 mi }{ 1 - \cos{\theta}} } \quad \boxed{ r(\theta= \pi ) = \frac{1}{4} \times 10^8 mi }
\end{gathered}
\]
\end{enumerate}

%-----------------------------------%-----------------------------------%-----------------------------------
\subsection*{ 13.24 Exercises - Conic sections symmetric about the origin, Cartesian equations for the conic sections }
%-----------------------------------%-----------------------------------%-----------------------------------
\quad 

Quick Review.\bigskip \\
Consider symmetry about the origin.  
\[
\begin{gathered}
  \left\| X - F \right\| = e d(X,L) = e | (X-F) \cdot N - d | = e|X\cdot N - F \cdot N - d| = |eX \cdot N - e (F \cdot N +d ) | \\
  \left\| X -F \right\|^2 = \left\| X \right\|^2 - 2 X \cdot F + \left\| F \right\|^2 = e^2 (X \cdot N)^2 - 2ae X \cdot N + a^2 \\
  X \to -X ; \quad \quad X \cdot F = aeX \cdot N \\
  X = (F-aeN) = 0 \Longrightarrow F =aeN; \quad F \cdot N = ae; \, a = \frac{ed}{1-e^2} \quad F = \frac{e^2 d}{ 1-e^2} N \\
\Longrightarrow \left\| X \right\|^2 + (ae)^2 = e^2(X \cdot N)^2 + a^2 \\
\begin{aligned}
  & \text{ if } X = \pm a N; \quad \left\| X \right\|^2 + (ae)^2 = e^2(X \cdot N)^2 + a^2 \text{ is satisfied } \\
  & \text{ if } X = \pm b N'; \quad b^2 + (ae)^2 = e^2(0) + a^2 \quad b^2 = a^2 (1-e^2)
\end{aligned}
\end{gathered}
\]

\exercisehead{1} $b^2 = a^2(1-e^2)$  $\quad \frac{x^2}{100} + \frac{y^2}{36} = 1 $ \medskip \\
$ \sqrt{1- \frac{b^2}{a^2}} = e \Longrightarrow \boxed{ e=  \frac{4}{5} }$.  

$|F| = |aeN| = 10 \left( \frac{4}{5} \right) = 8$.  $f = (\pm8,0)$.  $(0,0)$ center.  Vertices $(\pm 0 ,6)$.  

\exercisehead{2} $\frac{y^2}{100} + \frac{x^2}{36} =1$.  $\frac{4}{5} = e$; $\quad f = (0, \pm 8)$.  \medskip \\
$(0,0)$ center; vertices $(\pm 6, 0 ), (0, \pm 10)$.  

\exercisehead{3} $ \frac{(x-2)^2}{16} + \frac{(y-3)^2}{9} = 1$.  Center $(2,-3)$.    $|F| = ae = 4 \frac{ \sqrt{7}}{4} = \sqrt{7}$.  \bigskip \\
$\sqrt{ 1 - \frac{b^2}{a^2}} = \sqrt{ 1 - \frac{9}{16} } = \frac{ \sqrt{7}}{ 4}  = e$; $\quad (2+\sqrt{7}, -3), \, (2-\sqrt{7}, -3)$ foci.  

Vertices $(6,-3), (-2,-3), (2,6), (2,-12)$.  

\exercisehead{4} $\frac{x^2}{ \left( \frac{25}{9} \right)} + y^2 =1$.  Center $x=(0,0)$.  $|F| = ae = \left( \frac{5}{3} \right)\left( \frac{4}{5} \right)  = \frac{4}{3}$.  

$e=\sqrt{1-\frac{b^2}{a^2}} = \sqrt{ 1 - \frac{9}{25} } = \frac{4}{5}$.  Foci: $(\pm \frac{4}{3}, 0)$.  Vertices $(\pm \frac{5}{3}, 0 ), (0, \pm 1)$.  

\exercisehead{5} $\frac{y^2}{ (1/4)} + \frac{x^2}{(1/3)} = 1 $  \medskip \\
$|F| = ae = \frac{1}{\sqrt{3}} \left( \frac{1}{2} \right) = \frac{1}{ 2\sqrt{3}}$.   \medskip \\
$\sqrt{ 1 - \frac{1/4}{1/3}} = \frac{1}{2} = e$.  Foci: $\left( \frac{ \pm 1}{ 2\sqrt{3}}, 0 \right)$.  Center $(0,0)$.   \medskip \\
Vertices $(\pm 1/\sqrt{3}, 0), (0, \pm 1/2)$.  

\exercisehead{6} Center $(-1,-2)$.  \medskip \\
$\sqrt{ 1 - \frac{b^2}{a^2} } = \sqrt{ 1 - \frac{16}{25}} = \frac{3}{5} = e$; $\quad |F| = ae = 5 \frac{3}{5} = 3$.  \medskip \\
Foci: $(-1,-1), \, (-1,-5)$.   \medskip \\
Vertices: $(-1,3), (-1,-7), (3,-2), (-5,-2)$.  

\exercisehead{7}  $F=ae = \frac{3}{4}$.  $a=1, \, e = \frac{3}{4}$.  $b^2 = a^2 (1-e^2); \quad b^2 = 1 \left( \frac{1}{4} \right)$.  \medskip \\
$\boxed{ x^2 + 4 y^2 = 1}$.  

\exercisehead{8} $2a = 4$.  $a^2 = 4$.  $2b =3$.  $b^2 = 9/4$.  $\quad \Longrightarrow \frac{ (x+3)^2}{4} + \frac{(y-4)^2}{4} = 1$ 

\exercisehead{9} $\frac{ (x+3)^2}{ 9/4} + \frac{(y-4)^2}{ 4} = 1$.  

\exercisehead{10} $ 2a = 6, \, a=3$.  $\frac{(x+4)^2}{9} + \frac{(y-2)^2 }{1} =1$.  

\exercisehead{11} $2a = 10, \, a=5$.    $ |F|  = ae = 5e = 4 \, e = 4/5$.  $ b^2 = a^2 (1-e^2) = 25 \left( 1 - \frac{16}{25} \right) = 9$.  

\exercisehead{12} $\frac{(x-2)^2}{a^2} + \frac{ (y-1)^2}{b^2} = 1$;  \medskip \\
$a=4$ from $(6,1)$.  $b=2$ from $(2,3)$.  $\Longrightarrow \frac{ (x-2)^2}{4^2} + \frac{(y-1)^2}{ 4} =1$  

\exercisehead{13} $b^2 = a^2 (1-e^2)$.  \medskip \\
$ \frac{x^2}{100} - \frac{y^2}{64} = 1; \quad b^2 = 100(1-e^2) = -64$.  $\quad 1 + \frac{64}{100} = e^2$.  

Center $(0,0)$.  $\quad e = \frac{2\sqrt{41}}{10} = \frac{\sqrt{41}}{5}$.  \\
Vertices; $(\pm 10, 0 )$.  $F = ae = 2\sqrt{41}$.  Foci: $(\pm 2 \sqrt{41}, 0)$.  \bigskip \\
 $\frac{x^2}{100} = \frac{y^2}{64} + 1  \xrightarrow{x,y\to \infty} ; \quad y = \frac{ \pm 4}{5} x $

\exercisehead{14} $\frac{y^2}{100} - \frac{x^2}{64} = 1$;  Center $(0,0)$,  $a^2 = 100$; $\quad b^2  = -64$.  \medskip \\
$b^2 = a^2(1-e^2)$.  $\quad e = \frac{\sqrt{41}}{5} $.  Vertices $(0,\pm 10)$.    $F=ae = (0, \pm 2 \sqrt{41})$.  

$\frac{x^2}{64} + 1 = \frac{y^2}{100} \xrightarrow{x,y \to \infty} \frac{\pm 5}{4} x = y$.  

\exercisehead{15} $\frac{ (x+3)^2}{4} - (y-3)^2 = 1$.  \medskip \\
Center $(-3,3)$.   $e=  \sqrt{ 1 - \frac{b^2}{a^2} } = \sqrt{ 1  - \frac{-1}{4} } = \frac{ \sqrt{5}}{2}$.   \medskip \\
Foci: $ae = 2 \frac{\sqrt{5}}{2} = \sqrt{5}$.  $(-3+\sqrt{5}, 3 ), (-3-\sqrt{5}, 3)$.  \medskip \\
Vertices: $(-3,4), (-3,2); \quad (1,3), (-7,3)$.  

$\frac{ (x+3)^2 }{4} = 1 + (y-3)^2 \xrightarrow{ x,y \to \infty} \boxed{ \frac{ \pm (x+3)}{2} = y-3}$

\exercisehead{16} $ \frac{x^2}{ 144/9} - \frac{y^2}{ 144/16} = 1 = \frac{x^2}{16} - \frac{y^2}{9} $.  \medskip \\
$e = \sqrt{1 - \frac{ -9}{16}} = \frac{5}{4} $.  Center $(0,0)$. $|F| = ae = 5$.  Foci: $(5,0), (-5,0)$.  Vertices $(\pm 4,0)$.  

\exercisehead{17} $20 = 5y^2 -4x^2$.  Center $(0,0)$.  $|F| = ae = 2\left( \frac{3}{2} \right) =3$.  Foci: $(0,\pm 3)$.  $ 1 = \frac{y^2}{4} - \frac{x^2}{5} $.  $e = \sqrt{1- \frac{-5}{4} } = \frac{3}{2} $.   Vertices: $(0,\pm 2)$

\exercisehead{18} $\frac{ (x-1)^2}{4} - \frac{(y+2)^2}{9} = 1$.  \medskip \\
Center $(1,-2)$.  $e = \sqrt{ 1 - \frac{-9}{4} } = \frac{\sqrt{13}}{2}$;  $\quad |F| = 2 \frac{ \sqrt{13}}{2} = \sqrt{13}$.    Foci: $(1+ \sqrt{13}, -2), (1-\sqrt{13}, -2)$.    \medskip  \\
Vertices: $(5,-2), (-3,-2)$.  

\exercisehead{19} $F=ae = 2(2)=4$.   \medskip \\
$\frac{x^2}{4} + \frac{y^2}{-12} = 1$.  $\quad \frac{y^2}{12} + 1 = \frac{x^2}{4} \xrightarrow{ x,y \to \infty} y = \pm \sqrt{3} x $.  \medskip \\
$b^2 = a^2(1-e^2) = 4(1-4) = -12$.  

\exercisehead{20} $F =ae = \sqrt{2} = (1)e$.  $\quad b^2 = a^2 (1-e^2) = 1(1-2)=-1$.  $\Longrightarrow y^2 -x^2 = 1$.  

\exercisehead{21} $\frac{x^2}{4} - \frac{y^2}{16} = 1$

\exercisehead{22} $ (y-4)^2 - \frac{(x+1)^2}{-3} = 1$  \text{ where }  \medskip \\
$F = ae = |-2| = ae$.    $b^2 = a^2(1-e^2) = 1(1-4)= -3$  

\exercisehead{23} $\pm \frac{(x-2)^2}{ a^2 } \mp \frac{ (y+3)^2}{ b^2 } = 1$
\[
\begin{gathered}
\begin{aligned}
  & (3,-1) \Longrightarrow \pm \frac{1}{a^2} \mp \frac{4}{b^2} = 1 \\
  & (-1,0) \Longrightarrow \pm \frac{9}{a^2} \mp \frac{9}{b^2} = 1 
\end{aligned}
\Longrightarrow \frac{(y+3)^2}{ 27/8} - \frac{(x-2)^2}{ (27/5)} = 1 
\end{gathered}
\]

\exercisehead{24} $\frac{x^2-1}{3} = y^2$.    $ \frac{2x}{3} =2yy'$.  $\quad yy' = \frac{x}{3}$.   

$3x- 2y = C$.  $m = \frac{3}{2}$  $\quad \Longrightarrow y_0 \frac{9}{2} = x_0$.  $\frac{81}{4} y_0^2 -1 = 3y_0^2 \Longrightarrow y_0 = \frac{ \pm 2}{ \sqrt{69}}$.  \bigskip \\
The asymptotes of $y^2 = \frac{x^2 - 1}{3}$ are $y=\pm \frac{x}{\sqrt{3}}$.  
\[
\begin{gathered}
  3 \left( \frac{ \pm 9}{ \sqrt{ 69} } \right) - 2 \left( \pm \frac{2}{ \sqrt{ 69 }} \right) = \frac{ \pm 23}{ \sqrt{ 69 }} = C \\ 
  \boxed{ 3x \pm \sqrt{ \frac{23}{3} } = 2y }
\end{gathered}
\]

\exercisehead{25} $ \frac{ \pm x^2}{ a^2 } + \mp \frac{y^2}{b^2}  = 1 \quad \frac{ \pm x^2 }{a^2} \mp \frac{y^2}{ 4a^2} = 1$.  

$(3,-5) \to \pm 9 \mp \frac{25}{4} = a^2; \quad a^2 = \frac{11}{4}$.  \medskip  \\
$\Longrightarrow \boxed{ \frac{x^2}{ 11/4} - \frac{ y^2}{ 11 } = 1}$.  

Quick Review of Parabolas.  \bigskip \\
$F$ on positive half plane to $N$.  \medskip \\
$\left\| X - F \right\| = e | (X-F) \cdot N + d | $

Let $N=\vec{e}_x; \, d = 2c; \, F = (c,0); \, e =1$.  
\[
\begin{gathered}
  (x-c)^2 + y^2 = e^2 ((x-c) + 2c)^2 = (x-c)^2 + 4c(x-c) + 4c^2 \\
  y^2 = 4cx 
\end{gathered}
\]
Thus, for ellipses, the vertex is equidistant to the focus and directrix (confirming the other definition).

Let $N=\vec{e}_y$, $d=2c; \quad F = (0,c), \, e=1$.  
\[
\begin{gathered}
  x^2 +(y-c)^2 = ((y-c) + 2c)^2 = (y-c)^2 + 4c(y-c) + 4c^2 \\
  x^2 = 4cy
\end{gathered}
\]

\exercisehead{26}
$4c = -8$  $\quad (0,0)$ vertex.  $y=0$ symmetry axis.  $x=5$ directrix.  

\exercisehead{27} $4c = 3$.  Vertex: $(0,0)$.  Symmetry axis: $y=0$.  Directrix: $x = -3/4$.  

\exercisehead{28} $(y-1)^2 = 12 (x-\frac{1}{2})$.  $4c = 12, \, c=3$.  Symmetry axis: $y=1$.  Directrix: $\left( \frac{-5}{2}, 1\right)$.  

\exercisehead{29} $x^2/6 =y$.    $4c = \frac{1}{6}  \quad c= \frac{1}{24}$.  Vertex: $(0, 0)$.  Directrix: $y = -\frac{1}{24}$.  Symmetry axis: $x=0$.  

\exercisehead{30} $x^2 + 8y = 0$.  $\quad 4c = \frac{-1}{8}; \quad c = \frac{-1}{32}$.  $y=\frac{1}{32}$ directrix; $x=0$ axis.  

\exercisehead{31} $(x+2)^2 = 4(y+\frac{9}{4})$.  $4c=4; \quad c=1$.   Center $(-2,-9/4)$.  Directrix: $y=-13/4$.  Axis: $x=-2$.  

\exercisehead{32} $y=-x^2$.  

\exercisehead{33} $x^2 =8y$.  

\exercisehead{34} $(y-3) = -8(x+4)^2$.  

\exercisehead{35} $c =\frac{5}{4} \quad 5(x-\frac{7}{4} )= (y +1)^2$

\exercisehead{36} $y= ax^2 + bx + c$
\[
\begin{aligned}
  &  (0,1) \to c = 1
  & (1,0) \to 0 = a+b +1
  &  (2,0) \to 0 = 4a + 2b + 1 
\end{aligned}
\quad a=\frac{1}{2} \quad \Longrightarrow y = \frac{1}{2} x^2 - \frac{3}{2} x + 1
\]

\exercisehead{37} $4c (x-1) = (y-3)^2$.  $4c (-2) = (-4)^2 = 16$.  $c=-2$.    $\quad -8(x-1) = (y-3)^2$.  

\exercisehead{38} $\left\| X-F \right\| = ed(X,L) = |(X-F) \cdot N -d | $\medskip \\
$L = \{ (x,y) | 2x+y = 10; \frac{2}{\sqrt{5}} x + \frac{y}{\sqrt{5}} = \frac{10}{ \sqrt{5}} \}$.  \medskip  \\
 $d = N = x_L \quad dN \cdot N = d = x_L \cdot N = \frac{10}{ \sqrt{5}}$.  

\[
\begin{gathered}
  F = 0 \Longrightarrow \| X \|^2 = | X \cdot N - d |^2 = \left( \frac{-2}{\sqrt{5}} x + - \frac{y}{ \sqrt{5}} + \frac{10}{ \sqrt{5}} \right)^2 = x^2 + y^2 \\ 
  5x^2 + 5y^2 = (-2x -y+10)^2 = 4x^2 + y^2 + 100 + 4xy -40x -20 y \\
  \Longrightarrow x^2 + 4y^2 -4xy + 40x + 20y -100 = 0
\end{gathered}
\]

%-----------------------------------%-----------------------------------%-----------------------------------
\subsection*{ 13.25 Miscellaneous exercises on conic sections }
%-----------------------------------%-----------------------------------%-----------------------------------
\quad 

\exercisehead{1} 
\[
\begin{gathered}
  \frac{y^2}{b^2} = 1 - \frac{x^2}{a^2} \quad y^2 = b^2 - \left( \frac{bx}{a} \right)^2 = b^2 \left( 1 - \left( \frac{x}{a} \right)^2 \right) \\
  y = 2 \int_{-a}^a b \sqrt{ 1 - \left( \frac{x}{a} \right)^2 } dx = 2 \int_{-1}^1 ab \sqrt{ 1-x^2 } dx = (ab) \text{ area of a circle of radius $1$ }
\end{gathered}
\]

\exercisehead{2} \begin{enumerate}
\item Without loss of generality, let the major axis be $2a$ in the $x$-axis.  $y= b \sqrt{1- \left( \frac{x}{a} \right)^2 }$
\[
V = \int_{-a}^a \pi b^2 \left( 1 - \frac{x^2}{a^2} \right) dx = \pi b^2 a \int_{-1}^1 (1-x^2) dx = \frac{4}{3} \pi (1)^3 b^2 a
\]
\item If rotated about the minor axis, suppose, without loss of generality, $2a$ is the minor axis (just note that $\frac{x^2}{a^2} + \frac{b^2}{a^2} = 1 $ have $x,y,a,b$ as dummy labels).  \medskip \\
  $\Longrightarrow V = \frac{4}{3} \pi (1)^3 b^2 a $, where $2a$ is the minor axis, $2b$ is the major axis.  
\end{enumerate} 

\exercisehead{3}  $\frac{ x^2}{ (3/A) } + \frac{ y^2}{ (3/B) } = 1$  $\quad By^2 = 3 - Ax^2$  $\quad \Longrightarrow y^2 = \frac{3}{B} - \frac{A x^2 }{B}$; $\quad y = \sqrt{ \frac{3}{B} - \frac{A x^2 }{B} }$.  So the area inside this ellipse is
\[
2 \sqrt{ \frac{1}{B} } \int_{ -\sqrt{3/A} }^{ \sqrt{3/A}} \sqrt{3-A x^2 } dx = 2 \sqrt{ \frac{3}{B} } \int_{-\sqrt{3/A}}^{\sqrt{3/A}} \sqrt{ 1 - \frac{x^2}{ \left( \frac{3}{A} \right) } }
\]
For the other ellipse equation, $\frac{ x^2}{ 3/(A+B)} + \frac{y^2}{ 3/(A-B)} =  1 $.  $\quad y^2 = \left( \frac{3}{A-B} \right) \left( 1 - \frac{x^2}{ (3/(A+B)) } \right)$; $\quad y = \sqrt{ \frac{3}{A-B} } \sqrt{ 1 - \frac{x^2}{ (3/(A+B) )} } $.  Thus, the area inside this ellipse is 
\[
2 \sqrt{ \frac{3}{A-B} } \int_{ -\sqrt{ \frac{3}{A+B} } }^{ \sqrt{ \frac{3}{A+B}} } \sqrt{ 1 - \left( \frac{x}{ \sqrt{ \frac{3}{ A+B } } } \right)^2 }
\]
Equating the two areas after making an appropriate scale change,
\[
2 \sqrt{ \frac{3}{B}} \sqrt{ \frac{3}{A} } \int_{-1}^1 \sqrt{ 1 - x^2 } dx = 2 \sqrt{ \frac{3}{A-B} } \sqrt{ \frac{3}{A+B} } \int_{-1}^1 \sqrt{ 1 - x^2 } dx 
\]
Thus $A^2 - B^2 = AB$  $\quad \Longrightarrow A^2 - BA - B^2$.  Simply try treating $B$ as a number and solve the quadratic equation in terms of $A$.  
\[
A = \frac{ B \pm \sqrt{ B^2 - 4(1)(-B^2) }}{ 2(1) } = \frac{ B \pm B\sqrt{5}}{ 2 } = \frac{ B (1+ \sqrt{5} )}{ 2 } 
\]


\exercisehead{4} $y = -\frac{4h}{b^2} x^2$.  
\[
\int_{-b/2}^{b/2} \left( \frac{4h}{-b^2} x^2 + h \right) = \left. \frac{4h}{-3b^2} x^3 \right|_{-b/2}^{b/2} + h \left( \frac{b}{2} + \frac{b}{2} \right) = \frac{2hb}{3}
\]

\exercisehead{5} $y^2 = 8x$.  $\int_0^2 \pi 8 t dt = 4\pi (2)^2 = 16 pi$

\exercisehead{6} $y^2 = 2(x-1)$.  $y^2 =4(x-2)$.  
\begin{enumerate}
\item 
\[
\begin{aligned}
  A & =  2 \int_1^2 \sqrt{ 2 (x-1)} + 2 \int_2^3 \sqrt{ 2 (x-1)} - 2\sqrt{x-2} = \\
  & = 2\sqrt{2} \left. \frac{2}{3} (x-1)^{3/2} \right|_1^2 + 2 \sqrt{2} \frac{2}{3} \left. (x-1)^{3/2} \right|_2^3 - 4 \frac{2}{3} \left. (x-2)^{3/2} \right|_2^3 \\
  & = 2\sqrt{2} \frac{2}{3} + \sqrt{2} \frac{4}{3} (2)^{3/2} -2\sqrt{2} \frac{2}{3} - 4 \frac{2}{3} = 8/3 
\end{aligned}
\]
\item \[
\begin{gathered}
  \int_1^2 2(x-1) = 2 \left. \left( \frac{1}{2} x^2 - x \right) \right|_1^2 = 2 \left( \frac{1}{2}(4-1)-(2-1) \right) = 1 \\
  \int_2^3 \left( 2(x-1) - 4(x-2) \right) dx = \int_2^3 (-2x +6) dx = \left. -x^2 \right|_2^3 + \left. 6x \right|_2^3 = (-9 +4 + 6(3-2)) = 1 \\
  \Longrightarrow V = \pi \int_1^2 2(x-1) + \pi \int_2^3 (2(x-1)-4(x-2)) = 2\pi
\end{gathered}
\]
\item $\frac{y^2}{2} + 1 = x, \quad \frac{y^2}{4} + 2 = x$
\[
\begin{aligned}
  2 \pi \int_0^2 \left( \left( \frac{y^2}{4} + 2 \right)^2 - \left( \frac{y^2}{2} + 1 \right)^2 \right) & = 2\pi \int_0^2 \left( \frac{-3y^4}{16} + 3 \right) dy = 2\pi \left. \left( \frac{-3}{80} y^5+ 3y \right) \right|_0^2 = \\
  & = 2\pi \left( \frac{-3(32)}{80} + 6 \right) = 2\pi \left( \frac{ -96 + 480 }{ 80 }\right) = 2\pi \left( \frac{384}{80} \right) = \pi \frac{48}{5}
\end{aligned}
\]
\end{enumerate}

\exercisehead{7} By Apostol's definition of conic sections, we are basically given the conic section definition with $e=\frac{1}{2}$.  So just plug in the pt. $(0,4)$.  
\[
\begin{gathered}
  \frac{x^2}{a^2} + \frac{y^2}{b^2} =  1 \xrightarrow{(0,4)} b=4 \quad \quad b^2 = a^2(1-e^2) = 16 = a^2 \left( 1 - \left( \frac{1}{2} \right)^2 \right) \\
  \frac{x^2}{64/3} + \frac{y^2}{16} = 1
\end{gathered}
\]

\exercisehead{8} $F=0$  $\quad \left\| X - F \right\| = \left\| X \right\| = ed(X,F) = |X \cdot N + d | = \frac{x}{\sqrt{2}} + \frac{y}{ \sqrt{2}} + \frac{1}{\sqrt{2}}$ because for the directrix
\[
\begin{gathered}
  y+x = -1 \\
  N = \left( \frac{1}{\sqrt{2}}, \frac{1}{ \sqrt{2}} \right) \quad \frac{1}{ \sqrt{3}} y + \frac{x}{ \sqrt{2}} = -\frac{1}{ \sqrt{2}} \\
  \begin{aligned}
    X_L & = P + tA \\
    X_L \cdot N & N \cdot P = -1/\sqrt{2}
  \end{aligned}
  dN = X_L \quad X_L \cdot N = d = -1/\sqrt{2} 
\end{gathered}
\]
So by squaring both sides of the vector equation,
\[
\begin{gathered}
  x^2 + y^2 = \frac{x^2}{2} + xy + \frac{y^2}{2} + \frac{1}{2} + x + y \\
  \frac{x^2}{2} + \frac{y^2}{2} - xy - x - y = \frac{1}{2} \\
  x^2 + y^2 -2xy -2x -2y = 1
\end{gathered}
\]

\exercisehead{9} Center $(1/2,2)$ because we equate the asymptotes to see where they intersect: $y = 2x + 1 = -2x+3$.  

\[
\begin{gathered}
  \frac{ (y-2)^2 }{a^2 } - \frac{ (x-1/2)^2}{ a^2/4 } = 1 \xrightarrow{(0,0)} \frac{4}{a^2} - \frac{1}{a^2} = \frac{3}{a^2} = 1 \\
  \frac{ (y-2)^2}{ 3 } - \frac{(x-1/2)^2}{ 3/4} =  1
\end{gathered}
\]

\exercisehead{10} $px^2 + (p+2) y^2 = p^2 + 2p$.  $\quad \frac{x^2}{ p+2} + \frac{y^2}{ p } =1$.  
\begin{enumerate}
  \item Since $p+2 > p$, the foci must lie on the $x$ axis.  $a^2 = p+2$; $\quad b^2 = a^2 (1-e^2) = p = (p+2)(1-e^2)$.  $e=\sqrt{ \frac{2}{ p+2}}$ \medskip \\
    $F=ae = \sqrt{2}$.  $\quad (\pm \sqrt{2}, 0)$.  
  \item $F= ae = \sqrt{2} = a(\sqrt{3}) \Longrightarrow a =\sqrt{\frac{2}{3}}$; $\quad b^2 = \frac{2}{3} (1-3) = \frac{-4}{3}$.  
\[
\frac{x^2}{2/3} - \frac{y^2}{4/3} = 1 
\]
\end{enumerate}

\exercisehead{11} $e=1$ for an ellipse.  
\[
\begin{aligned}
  & \left\| X - F \right\| = | X \cdot N - a | = a - X \cdot N \\
  & \left\| -X - F \right\| = \left\| X + F \right\| = | -X \cdot N - a | = a + X \cdot N  \\
  & \left\| X - F \right\| + \left\| X  +F \right\| = 2a
\end{aligned}
\]

\exercisehead{12} 
\[
\begin{gathered}
\begin{aligned}
  &  \left\| X - F \right\|  = e | (X-F) \cdot N - d | = e (d - (X-F) \cdot N ) \\
  & \left\| X + F \right\| = e d(X,L) = e | (X-F) \cdot N + d | = e (-d - (X-F) \cdot N ) 
\end{aligned} \\
\left\| X - F \right\| - \left\| X + F \right\| = 2ed \\
 X \to -X \quad \text{ so for the other branch, $\left\| X + F \right\| - \left\| X -F \right\| = 2ed $ }
\end{gathered}
\]

\exercisehead{13} 
\begin{enumerate}
\item $\frac{ (tx)^2}{ a^2 } + \frac{ (by)^2}{ b^2 } = 1$ $\quad \left( \frac{b}{t} \right)^2 = \frac{a^2(1-e^2)}{ t^2} = \left( \frac{a}{t} \right)^2 (1-e^2 ) $
\item $b_1^2 = a_1^2 (1-e^2 ) $ $\quad b_2^2 = a_2^2 (1-e^2 )$.  
\[
\begin{gathered}
  1 - \frac{b_1^2}{a_1^2} = 1 - \frac{b_2^2}{ a_2^2}; \quad \frac{b_1^2}{ a_1^2} = \frac{b_2^2}{ a_2^2} \\
  \frac{x_1^2}{ a_1^2} + \frac{y^2}{ b_1^2} = 1 = \frac{ \left( \left( \frac{b_2}{b_1} \right) x \right)^2}{ a_2^2} + \frac{ \left( \frac{b_2}{b_1} y \right)^2 }{ (b_2 y)^2 } 
\end{gathered}
\]
\item $ \pm \frac{(tx)^2}{ a^2} \mp \frac{(ty)^2}{ b^2 } = 1 $ $\quad \left( \frac{b}{t} \right)^2 = \frac{ -a^2 (e^2-1)}{ t^2 } = - \left( \frac{a}{t} \right)^2 (e^2 - 1 )$.  
\[
\begin{gathered}
  \begin{aligned}
    b_1^2 & = a_1^2 (e^2 - 1) \quad     b_2^2 & = a_2^2 (e^2 - 1)  \\
    \frac{b_1^2}{a_1^2} + 1 & = e^2   \quad     \frac{b_2^2}{a_2^2} + 1 & = e^2  \quad \quad \frac{b_1^2}{a_1^2} = \frac{b_2^2}{a_2^2}   
  \end{aligned} \\
\pm \left( \frac{x}{a_1} \right)^2 \mp \left( \frac{y}{b_1} \right)^2 = \pm \left( \frac{ \left( \frac{b_2}{b_1} x \right) }{ a_2 } \right)^2 \mp \left( \frac{ \frac{b_2}{b_1} y }{ b_2 } \right)^2 = 1 
\end{gathered}
\]
\end{enumerate}

\exercisehead{14} $\frac{x^2}{a^2} + \frac{y^2}{b^2 } = 1 $.  $\Longrightarrow \frac{x}{a^2} + \frac{y}{b^2}y' = 0 $ \medskip \\
$\Longrightarrow y' = \frac{ -b^2 x }{ya^2 } = \frac{ - a^2 (1-e^2) x }{ a^2 y } = \frac{(e^2 -1) x }{y}$

\exercisehead{15} \begin{enumerate}
  \item $y = ax^2 + bx + c$  $\quad ty = a(tx)^2 + btx + c \to y = atx^2 + bx + c/t = y = Ax^2 + b + C$
  \item $ y = tx^2, \, t \neq 0$
\end{enumerate}

\exercisehead{16} $x - y + 4 = 0$  $y= 4\sqrt{x} \quad (y^2 = 16 x)$; $y' = 2x^{-1/2}$.  \medskip \\
$y'(x=4) = 1$  $\quad (x,y) = (4,8)$.  

\exercisehead{17} 
\begin{enumerate}
\item If we treat the two given parabolas, $y^2 = 4p(x-a)$ and $x^2 = 4qy$, as two vector objects free from any specific coordinate system then we observe that we can disregard the sign of $q$ and $p$ and simply state that they are both positive.  What matters is that we observe that $p$ and $q$ are the distance of the foci to the vertex for each of the respective parabolas.  

Second, observe that $a$ is not given.  By diagram, if $p,q$ are given, $a$ must be moved along the $x$-axis to fit the tangency condition.  Thus, in terms of doing the algebra, just eliminate $p$ and $q$ from the relations.  

If $(h,k)$ is the point of contact, 
\[
\begin{gathered}
\begin{aligned}
  x^2 & = 4qy \\
  \frac{x}{2q} & = y' \\
  y'(h) = \frac{h}{2q} 
\end{aligned}
\quad 
\begin{aligned}
  y^2 & = 4p (x-a) \quad y = 2\sqrt{ p } \sqrt{x-a} \\
  y' & = \sqrt{p} \frac{1}{ \sqrt{x-a}} \\
  y'(h) & = \sqrt{p} \frac{1}{ \sqrt{h-a }}
\end{aligned} \\
\text{ (Tangent condition) } \left( \frac{h}{2q} \right)^2 = \frac{ p }{ h-a}  \quad \Longrightarrow (h^2)(h-a) = (2q)^2 p  \\
\text{ (one point of contact condition) } \, \text{ with } q = \frac{h^2}{ 4k }, \, p = \frac{k^2}{4(h-a)}   \\
\Longrightarrow h^2(h-a) = \left( \frac{h^2}{ 2k} \right)^2 \frac{k^2}{ 4(h-a) }  \Longrightarrow (h-a)^2 = \frac{h^2}{16} \\
\Longrightarrow h = \frac{ 2a \pm a/2}{ 15/8} = \boxed{ 4a/3}
\end{gathered}
\]
\item \[
\begin{gathered}
  \frac{h}{2q} = \frac{ \sqrt{p}}{ \sqrt{ h-a} } \\
  \frac{2a}{3q} = \frac{ \sqrt{p}}{ \sqrt{ a/3}} = \frac{\sqrt{3p}}{ \sqrt{a}}; \quad 2a\sqrt{a} = 3\sqrt{3p} q \\
  \Longrightarrow 4a^3 = 27 pq^2
\end{gathered}
\]
\end{enumerate}

\exercisehead{18} First hint: Vector methods triumph over algebraic manipulations of Cartesian coordinates.  Think of the locus in terms of vector objects that are coordinate-free and the conic section will emerge.  I mean, try evaluating $\left\| P -A \right\|^2 = (x-2)^2 + (x-3)^2 = (x+y)^2$

$A=(2,3), \quad N = \frac{1}{\sqrt{2}} (1,1), \quad X = (x,y)$.  \medskip \\
$ \| X - A \| = x + y = \sqrt{2} (X\cdot N ) = \sqrt{2} ( X \cdot N - (F \cdot N - d ))$ \medskip \\
\text{ where } $F \cdot N = d = A \cdot N = \frac{5}{\sqrt{2}}$

$d = \text{ distance from focus to the directrix }$.  \medskip \\
$y = x + 1 $ (axis of the hyperbola) \medskip \\
$d = \frac{5}{ \sqrt{2}} = \sqrt{ (2-x)^2 + (3-y)^2 } = \sqrt{2} (2-x) \quad \quad x = -\frac{1}{2}, \, y = \frac{1}{2}$ \medskip \\
$\left( \frac{-1}{2}, \frac{1}{2} \right)$ must also be the center.  $y-\frac{1}{2} = - \left( x + \frac{1}{2} \right)$ is the directrix.  \medskip \\
$ \left( y - \frac{1}{2} \right) = \alpha \left( x + \frac{1}{2} \right)$ is the general form of the asymptote.  

Consider asymptotes in general.  $\| X - F \| = e d(X,L)$.  
\[
\frac{ \| X - F \| }{ d(X,L) } = e = \frac{ \| X - F \| }{ | X\cdot N - (F\cdot N - d ) | } = \frac{ \| X - F \| }{ (X-F)\cdot N + d | }
\]
For $\| X - F \| \to \infty$, $\quad \| X - F \| > d$.  To keep ratio of $e$, $X-F$ must be ultimately directed by $N$ by a ratio of $e$.  
\[
\Longrightarrow e = \frac{ \| X - F \| }{ \| X - F \| \cos{\phi} } = \frac{1}{ \cos{\phi}}
\]
e.g. Consider $N = \vec{e}_x$.  $\quad \frac{x^2}{a^2} - \frac{y^2}{b^2} =1 \Longrightarrow y = \frac{b}{a}x = \sqrt{e^2-1} x $.  
\medskip \\
From the vector equation,
\[
\begin{gathered}
  (X-F) \cdot N = (x-c,y)\cdot N = \sqrt{ (x-c)^2 +y^2 }\cos{\phi} = x - c \\
  \frac{ \sqrt{ (x-c)^2 +y^2 }}{ x-c} = \frac{1}{ \cos{\phi}} = e; \quad \frac{ (x-c)^2 + y^2 }{ (x-c)^2 } = e^2 ; \\
  \frac{y^2}{ (x-c)^2 } = e^2 - 1 \Longrightarrow y = \sqrt{ e^2 - 1 } x 
\end{gathered}
\]

For our problem, consider the conic section approaching the asymptote.  Then the conic section will look more like those linear asymptotes.  
\[
\begin{gathered}
  \sqrt{ (x-2)^2 + (y-3)^2 } = x + y \\
  \xrightarrow{ y - \frac{1}{2} = \alpha \left( x + \frac{1}{2} \right) } \sqrt{ \left( \left( x + \frac{1}{2} \right) - \frac{5}{2} \right)^2 + \left( \alpha \left( x + \frac{1}{2} \right) - \frac{5}{2} \right)^2 } = x + \alpha \left( x + \frac{1}{2} \right) + \frac{1}{2}  \\
  \Longrightarrow \sqrt{ (1+ \alpha^2 ) \left( x + \frac{1}{2} \right)^2 - 5 (1 + \alpha) \left( x +  \frac{1}{2} \right) + \frac{25}{2} } \xrightarrow{x\to \infty} \frac{\sqrt{ 1 + \alpha^2}}{ (1+\alpha ) }  \\
  \Longrightarrow \alpha = 0
\end{gathered}
\]
The asymptotes are $y = \frac{1}{2}$ and $x = \frac{ -1}{2}$.  

\quad \\

In the second part, each quadrant must be checked.  So far, I only have that quadrant II is filled: points in quadrant III and quadrant IV cannot satisfy the given condition.  To see this, consider quadrant II.  
\[
  \| x - A \| = -x + y  = \sqrt{2} (x,y) \left( \frac{-1}{\sqrt{2}}, \frac{1}{ \sqrt{2}} \right) 
\]
For quadrant II, $N =  \left( \frac{-1}{\sqrt{2}}, \frac{1}{ \sqrt{2}} \right) $.  By diagram, $(X-F)\cdot N > 0$ and $X \cdot N >0$.  
\[
\begin{gathered}
  A \cdot N = \frac{1}{ \sqrt{2}} \quad - A \cdot N = \frac{-1}{\sqrt{2}} \quad d = \frac{1}{ \sqrt{2}} \\
  |(X-F)\cdot N + d | = (X-F) \cdot N + d 
\end{gathered}
\]

The equation for the axis of the conic section is $y=-(x-5)$.  

By taking the asymptotic limit like above, we can show that $\alpha = 0 $ again.  We only sketch the part of the hyperbola in quadrant II.  

By similar procedure, I found that quadrant III, IV cannot satisfy the condition.  

\exercisehead{19}
\[
\begin{gathered}
  \| X-F \| = d(X,L) = | (X-F)\cdot N + d | \\
  x^2 + y^2 = (X\cdot N +d_1)^2 = y^2 + 2y d_1 +d_1^2 \\
  \xrightarrow{ F = 0 } x^2 = 2yd_1 + d_1^2 \quad \quad y_1' = \frac{x}{d_1} 
\end{gathered}
\]

\[
\begin{gathered}
  \| X- F \| = |(X-F)\cdot N - d_2| = d_2 - (X-F)\cdot N \\
  \xrightarrow{ F = 0 } \quad \| X \| = d_2 - y \\
  x^2 + y^2 = d_2^2 - 2d_2 y + y^2 \\
  \Longrightarrow x^2 = d_2 - 2d_2 y \quad \quad y_2' = \frac{-x}{d_2}
\end{gathered}
\]

\[
\begin{gathered}
  \text{ Point of intersection } x_0^2 = 2y_0 d_1 + d_1^2 = d_2^2 -2d_2 y_0 \\
  2(d_1 + d_2) y_0 = d_2^2 - d_1^2  \Longrightarrow y_0 = \frac{d_2 -d_1}{2} \\
  x_0^2 = d_2(d_2-2y_0) = d_2 d_1 \\
  \Longrightarrow 
\begin{aligned}
  y_1' & = \frac{ \pm \sqrt{d_2 d_1}}{ d_1} = \pm \sqrt{ \frac{d_2}{ d_1 } } \\
  y_2' & = \frac{ \mp \sqrt{ d_2 d_1}}{ d_2} = \mp \frac{1}{ \sqrt{ \frac{d_2}{d_1} } }
\end{aligned}
\end{gathered}
\]


\exercisehead{20} 
\begin{enumerate}
  \item Use $X \to -X$ symmetry.  
\[
\begin{gathered}
  \| X - F \| = ed(X,L) = e| (X-F)\cdot N + d | = e |X \cdot N - F \cdot N + d | = |eX \cdot N - a | \\
  \left\| X \right\|^2 - 2X \cdot F + \left\| F \right\|^2 = e^2 (X\cdot N )^2 - 2ea (X\cdot N) + a^2 \\
  X \to -X \Longrightarrow   \left\| X \right\|^2 + 2X \cdot F + \left\| F \right\|^2 = e^2 (X\cdot N )^2 + 2ea (X\cdot N) + a^2 \\ 
  \Longrightarrow \| X^2 \| + \| F \|^2 = e^2 (X \cdot N)^2 + a^2 \\
  x^2 + y^2 + c^2 = e^2 x^2 + a^2 \quad \quad |F| = c = ae \\
  \left( \frac{a^2 - c^2}{ a^2 } \right) x^2 + y^2 = a^2 - c^2 \Longrightarrow \frac{x^2}{a^2} + \frac{y^2}{a^2-c^2} = 1 
\end{gathered}
\] 
  \item $\frac{x}{a^2} + \frac{yy'}{a^2 - c^2} = 0$  $\Longrightarrow y' = \frac{ - (a^2 - c^2) x}{ ya^2 } $  
\[
\begin{gathered}
  \begin{aligned}
    &    xy (y')^2  = \left( \frac{a^2 -c^2 }{ a^2 } \right)^2 \frac{x^3}{y} 
  \end{aligned} \\
  \begin{aligned}
    (x^2 - y^2 -c^2)y' & = \left( x^2 + \frac{a^2 - c^2}{a^2 } x^2 - a^2 \right) \left( \frac{ -(a^2 -c^2) x }{ ya^2 } \right) = \\
    & = (a^4 + -(a^2 -c^2)x^2 - a^2 x^2 ) \frac{ (a^2-c^2)}{ a^4} \frac{x}{y} \\    
  \end{aligned} \\
  \begin{aligned}
    -xy & = \frac{ -xy^2}{y} = - \frac{x (a^2-c^2 - \left( \frac{a^2 -c^2}{a^2 } \right) x^2 )}{ y } = \\
    & = (a^2-c^2) (-a^4 + a^2 x^2 )x /(a^4 y) 
\end{aligned} \\
  \Longrightarrow xy y'^2 + (x^2-y^2 - c^2) y' -xy = 0 
\end{gathered}
\]
  \item For $y'$, consider $-\frac{1}{y'}$ at every $(x,y)$.  
\[
\begin{gathered}
  xy \left( \frac{-1}{y'} \right)^2 + (x^2 - y^2 -c^2 ) \frac{-1}{y'} - xy = 0 = \frac{-xy}{y'} + (x^2 -y^2 -c^2)\frac{1}{y'} + xy \\
  \xrightarrow{ \text{ if } y' \neq 0 } -xy + (x^2 -y^2 -c^2 )y' + xy (y')^2 = 0 
\end{gathered}
\]
Thus $S\to S$ since the defining differential equation is invariant under the transformation of the slope.  
\end{enumerate}

\exercisehead{21} For a circle centered at $C$, then $\left\| X - C \right\| = r_0^2$ for all points $X$ on that circle.  \medskip \\
For the condition of being tangent to a given line, $L = P +tA$, then $(X_C - C)\cdot A = 0 $ and the point lies on the circle so $\left\| X_C - C \right\| = r_0^2$.  \medskip \\
Call the point that all the circles pass through $F$.  Then $\left\| C - F \right\| = \left\| C - X_C \right\|$.  $\left\| C -X_C \right\|$ is by definition $d(X,L)$, the distance from the circle center to the line.  
$\left\| C - F \right\| = \left\| C - X_0 \right\|$ is by definition a parabola.  

\exercisehead{22}
Consider a circle that's part of the mentioned family that has its center directly below the given circle with radius $r_0$, and center $Q$.  \medskip \\
It's given that the center is equidistant from the point of tangency and the line.  This hints at a parabola because the parabola's vertex is equidistant from the focus and the directrix.  Thus, we need to show that $d(X,L)$ is equal to the distance from the circle center $C$ to the bottom point of $Q$.  

Let $N$ be a unit normal vector pointing from the line towards the focus, placing the focus in the positive half-plane.  \\
Let $C$ be the center of an arbitrary circle in the family and $r_1$ its radius.  \\
Let $X_1$ be the point of tangency between circle $Q$ and circle $C$.  \\
We want $\left\| (Q + r_0 N) - C \right\| = \left\| X_2 - C \right\|$.  

The tangency condition between circle $Q$ and $C$ means that
\[
(X_1 -C ) = - \alpha (X_1 - Q) ; \, \alpha > 0 \quad \quad \alpha = \frac{r_1}{r_0}
\]

\[
\begin{gathered}
  Q  - r_0 N -C = Q - X_1 - r_0 N - C + X_1 \\
  \xrightarrow{ \text{ take the magnitude} } \left\| Q - X_1 \right\|^2 + \left\| X_1 - C \right\|^2 + r_0^2 + 2(Q- X_1)(X_1 - C) + 2 (X_1 - Q) r_0 N + 2 (C-X)r_0 N \\
  r_0^2 + r_1^2 +r_0^2 + 2 \alpha r_0^2 + 2r_0 (1+\alpha)(x_1 - Q) \cdot N \\
  2r_0^2 + r_1^2 + 2r_1 r_0 + 2(r_1 + r_0)(X_1 - Q) \cdot N
\end{gathered}
\]

I had thought the key is to use \emph{ the law of cosines } to evaluate $(X_1 - Q)\cdot N = \frac{1}{\alpha} (C-X_1) \cdot N$.  \medskip \\
Length $l= d(X,L) = d(C,L)$. \medskip \\
But that just gets us back to the same place.  

I had found the solution by a clever construction.  But to come to that conclusion it required me to be ``unstuck'' - if something doesn't work, move onto the next - don't try to make something work and go in circles.  And persistence is key because there can be many \textbf{false eurekas}.  

Again, consider a particular circle with its center $C_2$ right below the given $Q$ circle that just makes $C_2$ tangent with the given line $L_2$.  The directrix is not going to be $L_2$ but $L_1$, a line translated below $L_2$, line of tangency, by $r_0$, so that $\| Q - C_2 \| = r_2 + r_0 = d(C_2,L_1)$.  It is a clever artificial construction.

Let's show this for any circle $C$ of radius $r_1$ in the family.  \medskip \\
Tangent to the circle $Q$ condition: $X_1 - C = \alpha(Q-X_1)$.  \medskip \\
So then $\| Q-C \| = r_1 + r_0$  \bigskip \\
Tangent to the line $L_2 = B_2 + tA_2$: $(X_2 - C) \cdot A_2 = 0$  \medskip \\
$\| X_2 - C \| = r_1$  

Consider $L_1$, a line translated by $r_0$ from $L_2$ away from $Q$.   \medskip \\
If $L_2 = B_2+tA_2$, \quad $L_1 = B_2 - r_0 N  + tA_2$.  \medskip \\
Since $X_2 - C = r_1 (-N)$ then $X_2 -r_0 N -C = (r_1 + r_0)(-N)$ will point from $C$ to $L_1$, because $(X_2 - r_0 N) = ((B_2 + t A_2 ) -r_0 N ) \in L_1$.  

$\Longrightarrow \| Q - C \| = r_1 + r_0 = \| X_2 - r_0 N - C \| = d(C,L_1)$.    


\exercisehead{23} Without loss of generality, use $y^2 = 4cx$.  \medskip \\
The latus rectum intersect the parabola at $(c,+2c), \, (c,-2c)$.  \bigskip \\
Thus $4c = \text{ length of latus rectum } = 2d = 2 ( \text{ distance from focus to directrix } )$.  
\[
y = 2 \sqrt{cx} \quad y' = \sqrt{c}/\sqrt{x} \quad y'(c) = \pm 1 
\]
Tangent lines: $y = \pm (x+c)$.  \medskip \\
$\xrightarrow{ \text{ intersection } } +(x+c) = -(x+c) \quad x = -c \, \text{ (at the directrix) } $  

\exercisehead{24}  Center of circle is given to be $0$.  \medskip \\
Collinear with center and center not between them: $P=\alpha Q$; $\quad \alpha > 0$  
\[
\left\| P \right\| \left\| Q \right\| = r_0^2 = \alpha \left\| Q \right\|^2 
\]

For the line defined in Cartesian coordinates as $x +2y -5 =0 $, the vector form of this line is given by 
\[
\begin{gathered}
  X_L = B+tA \quad X_L \cdot N = (x,y)\cdot \left( \frac{1}{\sqrt{5}}, \frac{2}{\sqrt{5}} \right)  = N\cdot B + 0 = \sqrt{5} \\
  A = \left( \frac{-2}{\sqrt{5}}, \frac{1}{ \sqrt{5}} \right) \text{ is a vector that's perpendicular to $N$ }; \\
  \quad B = (1,2) \text{ since we can simply plug it in to satisfy the equation } \\
  X_L = (1,2) + t \left( \frac{-2}{\sqrt{5}}, \frac{1}{ \sqrt{5}} \right) \quad t \in \mathbb{R} \\
  Q = B+tA \Longrightarrow \left\| Q \right\|^2 = B^2 + 2tB\cdot A + t^2 A^2 = 5 + t(0) + t^2 = 5 + t^2 \\
  (5+t^2)(\alpha) = r_0^2 = 4 \quad \alpha = \frac{ 4}{ 5+t^2} \\
  P = \alpha Q = \frac{4}{ 5 +t^2 } \left( (1,2) + t \left( \frac{-2}{ \sqrt{5}}, \frac{1}{ \sqrt{5}} \right) \right)
\end{gathered}
\]


%-----------------------------------%-----------------------------------%-----------------------------------
\subsection*{ 14.4 Exercises - Vector-valued functions of a real variable, Algebraic operations.  Components; Limits, derivatives, and integrals } 
%-----------------------------------%-----------------------------------%-----------------------------------
\quad \\
\exercisehead{1} $F' = (1,2t, 3t^2,4t^3 )$.  

\exercisehead{2} $F' = (-\sin{t}, 2\sin{t} \cos{t}, 2\cos{2t}, \sec^2{t} )$  

\exercisehead{3} $F' = \left( \frac{1}{ \sqrt{1-t^2 } }, \frac{-1}{\sqrt{1-t^2 }} \right)$

\exercisehead{4} $F' = \left( 2e^t, 3e^t \right)$.  

\exercisehead{5} $F' = \left( \sinh{t}, 2\cosh{2t}, -3e^{-3t} \right)$

\exercisehead{6} $\left( \frac{t}{1+t^2}, \frac{1}{ 1+t^2 }, \frac{-2t}{(1+t^2)^2 } \right)$ 

\exercisehead{7} $F' = \left( \frac{2}{1+t^2} + \frac{-4t^2}{ (1+t^2)^2 }, \frac{ 4t}{(1+t^2)^2 }, 0 \right)$.  
\[
F' \cdot F = \frac{4t (1+t^2 )}{ (1+t^2)^3 } + \frac{-4t^2 (2t)}{ (1+t^2)^3 } + \frac{4t -4t^3 }{ (1+t^2)^3 } = 0
\]

\exercisehead{8} $\left( \frac{1}{2}, \frac{2}{3}, e^1 -1 \right)$ 

\exercisehead{9} 
\[
\begin{aligned}
  \left. \left( -\cos{t}, \sin{t}, -\ln{ |\cos{t}| } \right) \right|_0^{\pi/4} & = \left( \frac{-\sqrt{3}}{ 2 } + 1 , \frac{ \sqrt{2}}{2} , -\ln{\frac{ \sqrt{2} }{2} } \right)
\end{aligned}
\]

\exercisehead{10} $\left. \left( \ln{(1+e^t)}, t - \ln{ (1+e^t)} \right) \right|_0^1 = \left(  \ln{ \left( \frac{1+e}{2} \right) }, 1 - \ln{ \left( \frac{1+e}{2} \right) } \right)$ 

\exercisehead{11}$\left. \left( te^t - e^t, t^2 e^t - 2te^t +2e^t, -te^{-t} - e^{-t} \right) \right|_0^1 = (1,e-2,-2e^{-1} + 1 )$ 

\exercisehead{12} $(2,-4,1)=A$.  \medskip \\
$\int_0^1 (te^{2t}, t\cosh{2t}, 2te^{-2t}) dt = \left. \left( \frac{1}{2} te^{2t} + \frac{-1}{4} e^{2t} , \frac{ t \sinh{2t}}{2} - \frac{ \cosh{2t}}{4}, 2 \left( \frac{-1}{2} te^{-2t} - \frac{e^{-2t}}{4} \right) \right) \right|_0^1 = $ \medskip \\
$ = \left( \frac{1}{4} e^2 + \frac{1}{4} , \frac{ \sinh{2}}{2} - \frac{\cosh{2}}{4} + \frac{1}{4}, 2 \left( \frac{-3}{4} e^{-2} + \frac{1}{4} \right) \right)$.  

$A\cdot B = \frac{1}{2} e^2 + -2\sinh{2} + \cosh{2} + \frac{-3}{2} e^{-2}$.  

\exercisehead{13} $F'(t) = B = 1 = \left\| F'(t) \right\| |B| \cos{\theta(t)}$.   \medskip \\
Given $\theta(t) = \theta_0$ constant, $\left\| F'(t) \right\| $ must be a constant.   \bigskip \\
$\left\| F'(t) \right\|^2 = F'(t) \cdot F'(t) = g \quad \quad g'  = 2F''(t) \cdot F'(t) = 0 \text{ since } \left\| F' \right\|^2 \text{ constant }$.  \medskip \\
$ \Longrightarrow F''(t) \cdot F'(t) = 0$ 

\exercisehead{14} 
\[
\begin{aligned}
  F' & = 2e^{2t} A + -2e^{-2t} B \\
  F''& = 4e^{2t} A + 4e^{-2t} B = 4 (F) 
\end{aligned}
\]

\exercisehead{15}  $G' = F' \times F' + F \times F'' = F \times F''$

\exercisehead{16} \[
\begin{aligned}
  G & = F \cdot (F' \times F'' ) \\
  G' & = F' \cdot (F' \times F'' ) + F \cdot (F''\times F'' + F' \times F''') = F \cdot (F' \times F''')
\end{aligned}
\]

\exercisehead{17} If $\lim_{t \to p} F(t) = A$, $\quad \forall j$th component, 
\[
\begin{gathered}
  \forall \sqrt{ \frac{\epsilon}{n} } > 0, \, \exists \, \delta_j > 0 \quad \text{ such that } |F_j(t) -A_j | < \sqrt{ \frac{\epsilon}{n } } \quad \text{ if } \, |t - p | < \delta_j \\
  \text{ Consider } \min_{j = 1, \dots n } \delta_j = \delta_0 \\
  \sum_{j=1}^n |F_j(t) - A_j |^2 < \sum_{j=1}^n \left( \sqrt{ \frac{\epsilon}{n} } \right)^2 = \epsilon \quad \text{ whenever } |t- p| < \delta_0 \\
  \Longrightarrow \lim_{t\to p } \| F(t) - A \| = 0
\end{gathered}
\]

If $\lim_{t\to p } \| F(t) - A \| = 0 $, \quad $\forall \epsilon > 0, \, \exists \, \delta > 0 $ \text{ such that } $\sqrt{ \sum_{j=1}^n (F_j(t) -A_j)^2 } < \epsilon \quad $ \text{ if } $|t-p| < \delta$.  \medskip \\
$\Longrightarrow \sum_{j=1}^n (F_j(t) - A_j)^2 < \epsilon$

$\epsilon > \sum_{j=1}^n (F_j(t) - A_j)^2 > (F_k(t) -A_k)^2 > 0$ \medskip \\
$\Longrightarrow \epsilon > |F_k(t) - A_k| $ \text{ if } $|t-p| < \delta$.  


\exercisehead{18}
If $F$ is differentiable on $I$, then 
\[
\begin{gathered}
  F' = \sum_{j=1}^n f_j' \vec{e}_j \quad \quad f_j' = \lim_{h\to 0 } \frac{1}{h} (f_j(t+h) - f_j(t)) \\
  F' = \sum_{j=1}^n \lim_{h\to 0} \frac{1}{h} (f_j(t+h) - f_j(t) ) = \lim_{h\to 0 } \frac{1}{h} \sum_{j=1}^n (f_j (t+h) -f_j (t)) e_j = \lim{h \to 0 } \frac{1}{h} (F(t+h) - F(t))
\end{gathered}
\]
If $F'(t) = \lim_{h to 0} \frac{1}{h} (F(t+h) - F(t)) = \lim_{h \to 0 } \frac{1}{h} \sum_{j=1}^n (f_j (t+h) - f_j(t))e_j = $ \\
$\quad \quad = \sum_{j=1}^n \lim_{h\to 0 } \frac{1}{h} (f_j(t+h) -f_j(t)) e_j = \sum_{j=1}^n f_j'(t) e_j$ \medskip \\
So $F'$ is differentiable.  

\exercisehead{19} $F'(t) = 0, \, \forall j = 1 \dots n $, $f_j'(t) =0$.  By one-dimensional zero-derivative theorem, $f_j(t) = c_j$ constant.  Thus $F(t) = \sum_{j=1}^n c_j \vec{e}_j = C$ on an open interval $I$.  

\exercisehead{20} $\frac{1}{6} t^3 A + \frac{1}{2} t^2 B + Ct + D $

\exercisehead{21} $Y'(x) + p(x) Y(x) = Q(x)$.  Then $\forall j =1, \dots , n $ 
\[
y_j'(x) + p(x) y_j(x) = Q_j(x)
\]
Since $p,Q$ are continuous on $I$, and given this initial value condition $y_k(a) = b_k$, 
\[
\begin{gathered}
  y_j(x) = e^{-\int_a^x p(t)dt} \left( b_j + \int_a^x Q_j(t) e^{\int_a^t p(u) du } dt \right)   \\
  \Longrightarrow \sum_{j=1}^n j_j(x) = Y(x) = e^{-\int_a^x p } \left( B + \int_a^x Q e^{\int_a^t p } dt \right)
\end{gathered}
\]

\exercisehead{22} \[
\begin{gathered}
  tF' = F + tA \Longrightarrow F' + tF'' = F' + A \\
  \Longrightarrow tF'' = A \\
  F''(t) = A/t \\
  \Longrightarrow F'(t) = A \ln{t} + B \\
  \Longrightarrow F(t) = A (t\ln{t} - t) + Bt + C \\
  F(1) = A(-1) + B + C = 2A 
\end{gathered}
\quad \quad 
\begin{gathered}
  tF' = F + tA \Longrightarrow A t \ln{t} + Bt = A(t\ln{t} -t) +Bt+ C + tA \\
  C = 0, \quad B = 3A \\
  F(t) = A (t\ln{t} - t ) + 3A t \\
  F(3) = A(3\ln{3}-3) + 9A = \boxed{ 3A \ln{3} +6A }
\end{gathered}
\]

\exercisehead{23}\[
\begin{gathered}
  \begin{aligned}
    F'(x) & = e^x A + xe^x A + -\frac{1}{x^2} \int_1^x F(t) dt + \frac{1}{x} F(x) = \\
    & = e^x A + xe^x A + e^x A - \frac{F(x)}{x} + \frac{F(x)}{x} = 2e^x A + xe^x A = (2+x)e^x A 
\end{aligned} \\
  F'(x) = (2+x)e^x A; \quad \quad F(x) = 2e^x A + A(xe^x - e^x ) + C = Axe^x +e^x A + C \\
\begin{aligned}
  \int_1^x (Ate^t + e^t A + C ) dt & = \left. \left( A (te^t -e^t ) + e^t A + Ct \right) \right|_1^x = \\ 
  & = A(xe^x -e^x ) + e^x A + C (x-1) - eA = Ax e^x + C(x-1) -eA \end{aligned}\\
xe^x A + Ae^x + \frac{C(x-1)}{x} - \frac{eA}{x} \Longrightarrow C = eA
\end{gathered}
\]

\exercisehead{24} $F'(t) = \alpha(t) F(t) $ \medskip 
\[
\begin{gathered}
\Longrightarrow f_k'(t) = \alpha(t) f_k(t) ; \quad \ln{ \left( \frac{f_k(t)}{ f_k(a) } \right)} = \int_a^t \alpha(x) dx \\
f_k  = f_k(a) e^{\int_a^t \alpha } \\
F(t) = \sum_{j=1}^n f_j(a) e^{\int_a^t \alpha} e_j = e^{\int_a^t \alpha} \sum_{j=1}^n f_j(a) e_j = u(t) A
\end{gathered}
\]

%-----------------------------------%-----------------------------------%-----------------------------------
\subsection*{ 14.7 Exercises - Applications to curves.  Tangency.  Applications to curvilinear motion.  Velocity, speed, and acceleration. }
%-----------------------------------%-----------------------------------%-----------------------------------
\quad \\
\exercisehead{1}
\[
\begin{aligned}
   \vec{r}(t) & = ((3t-t^3),3t^2,(3t+t^3)) \\
   v & = (3-3t^2,6t,3+3t^2) = 3(1-t^2,2t,1+t^2) \\ 
   a & = 3(-2t,2,2t) = 6(-t,1,t) \\
   s & = 3 \sqrt{ (1-2t^2 +t^4 + 4t^2 + 1+ 2t^2 + t^4) } = 3\sqrt{2} (1+t^2) 
\end{aligned}
\]

\exercisehead{2}
\[
\begin{aligned}
  r & = (c,s,e^t) \\
  v & = (-s,c,e^t) \\
  a & = (-c,-s,e^t) \\
  s & = \sqrt{ 1 + e^{2t} }
\end{aligned}
\]

\exercisehead{3} 
\[
\begin{aligned}
  r & = (3tc, 3ts, 4t) \\
  v & = (3(c+-ts), 3(s+tc),4) = 3(c,s,0) + (-3ts,3tc,4) \\
  a & = 6(-s,c,0) + (-3tc,-3ts, 0) \\
  s & = \sqrt{ 9 (c^2 - 2tsc+t^2s^2) + 9(s^2 + 2tsc + t^2 c^2) +16} = \boxed{ \sqrt{ 9 + 9 t^2 + 16 } }
\end{aligned}
\]

\exercisehead{4} 
\[
\begin{aligned}
  r(t) & = ((t-\sin{t}), (1-\cos{t}), 4 \sin{ (t/2) } ) \\
  v & = (1-c, s, 2\cos{ (t/2) } ) \\
  a & = (s,c,-\sin{(t/2) } ) \\
  s & = \sqrt{ (1-2c +c^2 +s^2 + 4c^2(t/2) ) } = \sqrt{ 2 - 2c + 4 \left( \frac{1+c}{2} \right) } = \boxed{ 2 } 
\end{aligned}
\]

\exercisehead{5} 
\[
\begin{aligned}
  r & = (3t^2,2t^3,3t) \\
  v & = (6t,6t^2, 3) \\
  a & = (6,12t, 0) \\
  s & = \sqrt{ 36t^2 + 36 t^4 + 9 } = \boxed{ 3 \sqrt{ 4t^2 + 4t^4 + 1 } }
\end{aligned}
\]

\exercisehead{6} 
\[
\begin{aligned}
  r & = (t,\sin{t}, (1-\cos{t}) ) \\
  v & = (1,c,s) \\
  a & = (0,-s,c) \\
  s & = \sqrt{2} 
\end{aligned}
\]

\exercisehead{7} 
\[
\begin{gathered}
  \begin{aligned}
    r & = (ac(\omega t), a s(\omega t), b\omega t) \\
    r' & = (-\omega a s(\omega t), \omega a c(\omega t), b\omega ) = \omega(-as,ac,b) \\
    |r'| & = \omega \sqrt{ a^2 + b^2 } 
  \end{aligned} \\
  \frac{ r'}{|r'|} \cdot e_z = \frac{ b\omega}{ \omega \sqrt{ a^2 + b^2 } } = \boxed{ \frac{ b}{ \sqrt{ a^2 + b^2 } } }
\end{gathered}
\]

\exercisehead{8} $|r'| = \omega \sqrt{a^2 + b^2 } $
\[
\begin{gathered}
  a = \omega^2 (-ac,-as,0), \quad |a| = \omega^2 a \medskip \\
  \frac{ \| v \times a \| }{ \| v \|^3 } = \frac{ 1}{ (\omega \sqrt{a^2 + b^2 } )^3 } | (\omega^3(bas,-bac,a^2) ) | = \frac{ \omega^3 \sqrt{ b^2 a^2 + a^4 } }{ \omega^3 (\sqrt{ a^2 + b^2 })^3 } = \boxed{ \frac{ a}{ a^2 + b^2 }  }
\end{gathered}
\]

\exercisehead{9} $u= (\sin{(\omega t)}, -\cos{ (\omega t) }, 0 )$ 
\[
\begin{gathered}
  v \times a = \omega^3 (bas,-bac,a^2) = \omega^3 bau + \omega^3 a^2 e_z \\
  \begin{aligned}
    A & = \omega^3 ba \\
    B & = \omega^3 a^2 
  \end{aligned}
\end{gathered}
\]

\exercisehead{10} 
\[
\frac{d}{dt} (v\cdot v) = a\cdot v + v\cdot a = 2 v\cdot a
\]

\exercisehead{11}
\begin{enumerate}
  \item \quad \\
\[
\begin{gathered}
  \frac{v}{|v|} \cdot c = \cos{ \theta}; \quad \, \frac{ v\cdot c}{ \cos{\theta} } = |v| \\
  (r\cdot c)' = r'\cdot c = 2e^{2t}  \\
  \Longrightarrow \frac{ 2 e^{2t}}{ \cos{ \theta} } = |v|
\end{gathered}
\]
  \item 
\[
\begin{gathered}
  \frac{ 4 e^{4t}}{ \cos^2{\theta}} = |v|^2 \\
  \text{ from the previous exercise, } \frac{1}{2} \frac{d}{dt} v^2 = \frac{ 8e^{4t}}{ \cos^2{\theta}} = a\cdot v
\end{gathered}
\]
\end{enumerate}

\exercisehead{12} $r = (a\cosh{\theta},b\sinh{\theta})$.  $a=b=1$.  \\
Draw a diagram.  You'll immediately see that to get sector OAP, we need to take the area of a right triangle, $\frac{1}{2} \cosh{\theta}\sin{theta} = \frac{1}{2} (\text{ $x$-coordinate })(\text{ $y$-coordinate} )$ and subtract away from it the curved piece below the graph of the hyperbola.  Thus
\[
\frac{1}{2} \cosh{\theta} \sinh{\theta}  - \int_1^{\cosh{\theta}} \sqrt{ x^2 - 1 } dx = A(\theta) \\
\text{ with } \begin{aligned}
  x^2 - y^2 & = 1 \\
  \sqrt{x^2 - 1 } & = y 
\end{aligned}
\]
Then 
\[
\begin{aligned}
  A'(\theta) & = \frac{1}{2} (\cosh^2{\theta} + \sinh^2{\theta} )  - \sqrt{ \cosh^2{\theta} - 1 } \sinh{\theta} = \frac{1}{2} \\
  A(\theta) & = \boxed{ \frac{\theta}{2}  }
  \end{aligned}
\]

\exercisehead{13}
\[
\begin{aligned}
  r & = (a \cosh{\omega t}, b \sinh{\omega t } ) \\
  v & = \omega(a \sinh{\omega t}, b\cosh{\omega t} ) \\
  a & = \omega^2(a\cosh{\omega t}, b\sinh{\omega t}) = \boxed{ \omega^2 r }
\end{aligned}
\]
So $a$ is centrifugal, since $a = \omega^2 r$.  

\exercisehead{14} 
Parabola.  
\[
\| X - F \| = d(X,L) = | (X - (-2a)e_x) \cdot e_x | = X\cdot e_x + 2a 
\]
Let $F=0$, \quad $\| X - F \| = \| X \|$ \\
Let $X = d_1 u_1$
\[
\begin{gathered}
  \begin{aligned}
    & X' = d_1' u_1 + d_1 u_1' \\
    & X'\cdot u_1 = d_1' = X'\cdot e_x
  \end{aligned} \quad \quad 
  \begin{gathered}
    \| X \| = d_1 = X \cdot e_x + 2a \\
    d_1' = X' \cdot e_x 
\end{gathered} \medskip \\
  \xrightarrow{ \frac{ X'}{ \| X' \| } = T } \begin{aligned}
    T\cdot u_1 & = T\cdot e_x \\
    \cos{ \theta_1 } &  = \cos{ \theta_2 }
\end{aligned}
\end{gathered}
\]

\exercisehead{15}
\begin{enumerate}
\item $|a| = 4|r|$.  $\Longrightarrow a = -4r$. 
\[
\xrightarrow{ r(0) = 4e_x; \, v(0) = 6e_y } \boxed{ \begin{aligned}
  x & = 4 \cos{ (2t) } \\
  y & = 3 \sin{(2t) }
\end{aligned} }
\]
\item $\boxed{ \left( \frac{x}{4} \right)^2 + \left( \frac{y}{3} \right)^2  = 1 }$.  Motion is counterclockwise along an ellipse.  
\end{enumerate}

\exercisehead{16} Given that $x^2 + c(y-x) =0$, then $y = x - \frac{x^2}{c}$ 
\[
\begin{gathered}
  \begin{aligned}
    r & = (x,y) = (x,x-\frac{x^2}{c}) \\
    r' & = (x',x' - \frac{2xx'}{c} ) \\
    r'' & = (x'',x'' - \frac{2}{c} (x'^2 + xx'') )
  \end{aligned} \\
  \Longrightarrow \begin{gathered}
    x'' = x'' - \frac{2}{c} (x'^2 + xx'') \\
    xx'' + x'^2 = 0 \text{ or } \\
    (xx')' = 0 \text{ so then } \boxed{ xx' = \text{ constant } = k }
  \end{gathered} \bigskip \\
  x \frac{dx}{dt} = k \Longrightarrow x dx = k dt \\
  \frac{1}{2} (x_f^2 - x_i^2) = k(t_f - t_i)  \Longrightarrow \frac{1}{2}(-c^2) = k(T) \text{ so that } k = \frac{-c^2}{2T } \\
  \frac{1}{2} \left( \left( \frac{c}{2} \right)^2 - c^2 \right) = \frac{1}{2} \left( \frac{-3c^2}{4} \right) = \frac{-c^2}{2T } (t_f - 0) \\
  \boxed{ \frac{3T}{4} = t_f }
\end{gathered}
\]

\exercisehead{17} Given that $Y(t) = X(u(t))$, 
\[
Y'(t) = X'(u) u'(t)
\]
So $Y'$ is a scalar multiple of $X'$; $Y' \parallel X'$ 

\[
Y'' = X''(u)(u'(t))^2 + X'(u)u''(t)
\]
So $Y'' \parallel X''$ only if $X'(u) \parallel X''(u)$ as well.  This is not necessarily so.  

%-----------------------------------%-----------------------------------%-----------------------------------
\subsection*{ 14.9 Exercises - The unit tangent, the principal normal, and the osculating plane of a curve }
%-----------------------------------%-----------------------------------%-----------------------------------
\quad \\
\exercisehead{1} $t=2$
\begin{enumerate}
\item 
\[
\begin{aligned}
  T & = \frac{v}{s} = \frac{ (1-t^2,2t, 1+t^2) }{ \sqrt{2} (1+t^2) } \\
  T(t=2) & = \boxed{ \frac{ (-3,4,5)}{ \sqrt{2} (5) }  } \\ 
  T' & = \frac{v'}{s} - \frac{s'v}{s^2} = \frac{(-2t,2,2t) }{ \sqrt{2} (1+t^2) } - \frac{ \sqrt{2} (2t)(1-t^2,2t,1+t^2) }{ 2(1+t^2)^2 } \\
  T'(t=2) & = \frac{\sqrt{2} (-2,1,2) }{ 5 } - \frac{\sqrt{2} (2)(-3,4,5) }{ 25} = \frac{\sqrt{2}}{25} \left( (-10,5,10) - (-6,8,10)  \right) = \frac{\sqrt{2}}{25} (-4,-3,0) \\
  N(t=2) & = \boxed{ \frac{ (-4,-3,0) }{5} }
\end{aligned}
\]
\item 
\[
a(t=2)  = 6(-2,1,2) = \boxed{ 12\sqrt{2} T(t=2) + 6(N) }
\]
\end{enumerate}

\exercisehead{2} $t=\pi$
\begin{enumerate}
\item 
\[
\begin{aligned}
  T & = \frac{v}{s} = \frac{ (-s,c,e^t)}{ \sqrt{1 + e^{2t} } } \\
  T(t=\pi) & = \boxed{ \frac{ (0,-1,e^{\pi}) }{ \sqrt{ 1 + e^{2 \pi} } } } \\
  T' & = \frac{v'}{s} - \frac{s' v}{s^2} = \frac{ (-c,-s,e^t) }{ \sqrt{ 1 + e^{2t} } } - \frac{ e^{2t}}{ \sqrt{ 1 + e^{2t} } } \frac{ (-s,c,e^t) }{ (1+e^{2t} ) } \\
  T'(t=\pi) & = \frac{ (1,0,e^{\pi})}{ \sqrt{ 1 + e^{2\pi} }} - \frac{e^{2\pi }}{ \sqrt{ 1 + e^{2\pi}} } \frac{(0,-1,e^{\pi}}{ (1+e^{2\pi} )} = \\
  & = \frac{ (1+e^{2\pi},e^{2\pi} ,e^{\pi} )}{ (1+e^{2\pi})^{3/2} } \\
  N(t=\pi) & = \boxed{ \frac{ (1+ e^{2\pi},e^{2\pi},e^{\pi}) }{ \sqrt{ 1 + 3e^{2\pi} + 2e^{4\pi} } } }
\end{aligned}
\]
\item
\[
a(t=\pi) = \boxed{ \frac{ e^{2\pi} \sqrt{ 1 + e^{2\pi} } T(t=\pi) + \sqrt{ 1 + 3e^{2\pi} + 2e^{4\pi }} N(t=\pi)  }{ 1+e^{2\pi}} }
\]
\end{enumerate}

\exercisehead{3} 
\begin{enumerate}
\item \[
\begin{gathered}
  \begin{aligned}
    T & = \frac{v}{s} = \frac{3(1,0,0)+(0,0,4) }{ 5 } = \boxed{ \frac{(3,0,4)}{5} } \\
    T' & = \left( \frac{v}{s} \right)' = \frac{v'}{s} - \frac{s'v}{s^2} = \\
    & = \frac{ 6(-s,c,0) + (-3tc,3ts,0) }{ \sqrt{ 9 + 9t^2 + 16} } - \frac{ \frac{9t}{ \sqrt{9 + 9t^2 + 16 } } ( 3(c,s,0) + (-3ts,3tc,4) ) }{ 9 + 9t^2 + 16 } 
\end{aligned} \\
    T'(t=0) = \frac{6}{5} (0,1,0) - 0 =  \frac{6}{5} (0,1,0)  \\
    N'(t=0) = \boxed{ (0,1,0) }
\end{gathered}
\]
\item 
\[
a(t=0) = 6(0,1,0) = \boxed{ 0T + 6N'(t=0) }
\]
\end{enumerate}

\exercisehead{4} 
\begin{enumerate}
\item \[
\begin{aligned}
  T & = \frac{v}{s} = \frac{1}{2} (1-c,s,2\cos{(t/2)}) \\
  T' & = \frac{1}{2} (s,c,-\sin{(t/2)}) 
\end{aligned} \quad \quad \quad 
\begin{aligned}
  |T'| & = \frac{1}{2} \sqrt{ (1+s^2(t/2)) } \\
  N & = \frac{T'}{|T'|} = \frac{(s,c,-\sin{(t/2)})}{\sqrt{ (1+s^2(t/2)) }} \\
  T(t=\pi) & = \boxed{ (1,0,0) } \\
  N(t=\pi) & = \boxed{ (0,-1/\sqrt{2},-1/\sqrt{2}) } 
\end{aligned}
\]
\item \[
\begin{aligned}
  a(t=\pi) & = (0,-1,-1) \\
  a(t=\pi) & = 0T + 2N(t=\pi)
\end{aligned}
\]
\end{enumerate} 

\exercisehead{5} $t=1$
\begin{enumerate}
  \item \[
\begin{aligned}
  T & = \frac{ (2t,2t^2,1) }{ \sqrt{4t^2 + 4t^4 + 1 } } \\
  T(t=1) & = \boxed{ 1/3(2,2,1) } 
\end{aligned} \quad \quad \quad 
\begin{aligned}
  T' & = \frac{v'}{s} - \frac{s'}{s^2}v = \frac{ (2,4t,0)}{ \sqrt{4t^2 + 4t^4 + 1 } } - \frac{ 4t +8t^3}{ (4t^2 + 4t^4 + 1)^{3/2} } (2t,2t^2, 1) \\
  T'(t=1) & = \frac{ (2,4,0)}{3} - \frac{12}{27}(2,2,1) = \boxed{ 1/9(-2,4,-4) } \\
  N(t=1) & = \boxed{ 1/3(-1,2,-2)  }
\end{aligned}
\]
  \item
\[
\begin{aligned}
  a & = (6,12,0) = 4(3T) + 2(3N) = \boxed{ 12T+6N }
\end{aligned}
\]
\end{enumerate}

\exercisehead{6}
\begin{enumerate}
  \item 
\[
\begin{aligned}
  T & = \frac{v}{s} = \frac{(1,c,s)}{\sqrt{2}} \\
  T(t=\pi/4) & = \boxed{ \left( \frac{1}{\sqrt{2}}, \frac{1}{2}, \frac{1}{2} \right) }
\end{aligned} \quad \quad \quad 
\begin{aligned} 
  T' & = \frac{v'}{s} - \frac{s'v}{s^2} = \frac{ (0,-s,c)}{ \sqrt{2}} \\
  N(t=\pi/4) & = \boxed{ ( 0,-1/\sqrt{2},1/\sqrt{2}) }
\end{aligned}
\]
  \item
\[
\begin{aligned}
  a(t=\pi/4) & = (0,-1/\sqrt{2}, 1/\sqrt{2}) = \boxed{ N }
\end{aligned}
\]
\end{enumerate}

\exercisehead{7} 
\[
\begin{gathered}
  a = 0 \\
  v= v_0 = \text{ const. } \\
  \boxed{ r = v_0 t + r_0 }
\end{gathered}
\]

\exercisehead{8}
\[
\frac{ \| v \times a \|}{ \| v \| } = \| \frac{ v}{ \| v\| } \times a \| = \| T \times a \|
\]

$a = s' T + sT'$, so the normal component of $a$ is $s|T'|$ since $sT' = s|T'| N$.  
\[
\begin{gathered}
  T\times a = sT \times T' \\
  \begin{aligned}
    \| T \times a \| & = \| sT \times T' \| = \sqrt{ |sT|^2 |T'|^2 - (sT \cdot T')^2 }  = \\
    & \xrightarrow{ T\cdot T' = 0} |sT||T'| = s|T'|
\end{aligned}
\end{gathered}
\]

\exercisehead{9}
\begin{enumerate}
  \item 
\[
\begin{gathered}
  v=v_0 \\
  r= v_0 t + r_0 \quad \, r \in \text{ line }
\end{gathered}
\]
  \item Counterexample: imagine the particle moving at constant speed, but in an S-shape in one plane and then an S shape in another plane, out of the first plane.    
  \item \[
\begin{gathered}
  a = a_0 \\
  v=a_0 t + v_0 \\
  r = \frac{a_0}{2} t^2 + v_0 t + r_0
\end{gathered}
\]
$r = \frac{a_0}{2} t^2 + v_0 t + r_0$ defines a plane.  
\item $v\cdot a =0$
\[
\begin{gathered}
  a = s' T + sT' \\
  a\cdot T = s' = 0 \quad \Longrightarrow s \text{ constant } \\
  \Longrightarrow \begin{aligned}
    a & = sT' \\
    v & = kT
\end{aligned}
\end{gathered}
\]
So $s$ is constant and $a = sT'$.  We already showed in part (b) of this exercise that there's a counterexample.  
\end{enumerate}

\exercisehead{10}
\begin{enumerate}
  \item \[
    \begin{gathered}
      \frac{d}{dt} ( r\times v) = v\times v + r\times a = r\times a = 0 \\
      \Longrightarrow r \times v = \text{ const. }
\end{gathered}
\]
\item \textbf{ Use what you've learned in these sections: use what you've learned about $a$ being completely in the osculating plane }.  \emph{ The osculating plane idea is very useful}.  
\[
\begin{gathered}
  r \times v =c \Longrightarrow r \times a = 0  \\
  a = s'T + sT' \Longrightarrow r \times (s' T + sT') = 0 \medskip \\
  \text{ Possibilities:} \\
  \begin{aligned}
    & r = c(s'T + sT') \text{ ($r$ is a scalar multiple of $s'T + sT'$) } \Longrightarrow r \in \text{ osculating plane } \\
    & a = 0, \quad \, v = v_0, \quad \, r = v_0 t + r_0 , \quad \text{ $r$ is on a line } \\
    & r = 0, \quad \text{ $r$ stationary }
  \end{aligned}
\end{gathered}
\]
\item If $a = |a(t)| r(t)/|r(t)|$ then $r\times a =0$ since $a$ is a scalar multiple of $r$.  Then from above, $r\times a=0$ imples that $r$ is on a plane.  
\item $r\times v$ not necessarily constant if $r$ lies in a plane.  \medskip \\
Counterexample: 
\[
\begin{aligned}
  r & = (tc,t^2s,0) \\
  v & = (c -ts, 2ts + t^2 c,0 ) \\
  r\times v & = (0,0, 2t^2 cs + t^3 c^2 - t^2 sc - t^3s^2 ) = (0,0, t^2 sc+ t^3(c^2-s^2) )
\end{aligned}
\]
\end{enumerate}

\exercisehead{11} Given that $r \in $ plane, the $v$ is also in the plane, otherwise, for some time $t$, $r(t + \Delta t) = r + v\Delta t$ out of the plane.   
\begin{enumerate}
  \item We \textbf{ want } $T \times a = v\times a =0$.  \medskip \\
    Given that $T\cdot c = \cos{\theta_0} = k_0$, $(T\cdot c)' = 0 = T'\cdot c +T \cdot c' = T' \cdot c$.  \\
    $c\neq 0$, so if $T\cdot c \neq 1$, $T' \cdot c =0$ implies $T'=0$.  Then $T\times a =0$.  \bigskip \\
    If $T\cdot c =1$, $v=sc$ \quad \, $a = v' = s'c$ 
\[
v\times a = sc \times s'c = ss'(c\times c) = 0 
\]
\item Example:  \\
$\begin{aligned}
  r  & = (c(t),s(t), t^3) \\
  v  & = (-s,c,3t^2)
  a  & = (-c,-s,6t)
\end{aligned}$
\end{enumerate}

\exercisehead{12} 
\begin{enumerate}
  \item If $y= g(t) \gtrless 0$, $v_x \lessgtr 0$.  $\boxed{ \text{ counterclockwise } }$
  \item \[
\begin{gathered}
  3x^2 + y^2 = 1 \\
  r(t) = (f(t),g(t)) = (x,y) = (x,\pm \sqrt{ 1 - 3x^2} ) \\
  r' = \left( x' , \mp \frac{3x}{ \sqrt{ 1 - 3x^2 }} x' \right) \\
  x' = v_x = -g(t) = y \\
  \frac{ \mp 3x' x}{ \sqrt{ 1 - 3x^2 } } = \frac{ \mp 3y }{y } x = \mp 3x \\
  \text{ so } v_y = \boxed{ 3x } \, \text{(since we want counterclockwise motion)}
\end{gathered}
\]
  \item $x' = \sqrt{ 1 - 3x^2} \Longrightarrow $\[
\begin{aligned}
  & \int_{x_0}^{x_f} \frac{dx}{ \sqrt{ 1 - 3x^2 }} = \int_{t_i}^{t_f} dt = T = \left. \frac{ \arcsin{(\sqrt{3} x) }}{ \sqrt{3} } \right|_{x_0}^{x_f} = \frac{1}{\sqrt{3}} (\pi/2 - -\pi/2) = \pi/\sqrt{3} \\
  & \int_{x_0}^{x_f} = T = \left. \frac{ \arccos{ (\sqrt{3} x) } }{ \sqrt{3}} \right|_{-1/\sqrt{3}}^{1/\sqrt{3}} = \frac{ 0 - (-\pi)}{\sqrt{3}} = \pi/\sqrt{3} \quad \text{(we need $T$ to be positive)} \\
    & \Longrightarrow \boxed{ 2\pi /\sqrt{3}}
\end{aligned}
\]
\end{enumerate}

\exercisehead{13} 
\[
\begin{gathered}
  \begin{aligned}
    r &= (x,y) = (x(t),y(t)) \\
    r' & = (x',y')
\end{aligned} \quad \, \begin{aligned}
    & r\cdot e_x = x = \sqrt{ x^2 + y^2 } \cos{\theta} = r\cos{\theta} \\
    & r' \cdot e_x = x' = |r'|\cos{\phi}
\end{aligned} \bigskip \\
  \tan{ \phi} = \frac{4}{3} \cot{\theta} \Longrightarrow \pm \frac{y'}{x'} = \frac{4}{3} \frac{x}{y} \\
  \frac{y'}{x'} = \frac{dy}{dx} = \frac{-4x}{3y} \quad \text{(since $\frac{x}{r} > 0$ for $C$ in first quadrant, and $\frac{dy}{dx} <0$ in first quadrant)} \\
  \Longrightarrow y^2 = \frac{-4}{3}x^2 +C \\
  \xrightarrow{ (3/2,1) } \boxed{ \frac{x^2}{3} + \frac{y^2}{4} = 1 } 
\end{gathered}
\]

\exercisehead{14}
\[
\begin{gathered}
  \begin{aligned}
    T & = \frac{ (x',y') }{ \sqrt{ x'^2 + y'^2 } } \\
    N & = \frac{ (y',-x')}{ \sqrt{ x'^2 + y'^2 } }
  \end{aligned} \bigskip \\
  (x,y) + (\pm d) N = (x,y) + \frac{ (\pm d) (y',-x') }{ \sqrt{ x'^2 + y'^2 } } = (x_2,0) \bigskip \\
  \begin{aligned}
    x + \frac{ ((\pm) d) y' }{ \sqrt{ x'^2 + y'^2 } } & = x_2 \\
    y + \frac{ ((\pm) d) (-x') }{ \sqrt{ x'^2 + y'^2 } } & = 0
  \end{aligned} \\
  |x_2 - x| = 2 = \left| x + \frac{ \pm d y' }{ \sqrt{ x'^2  +y'^2 } } - x \right| = \frac{ |d| |y'| }{ \sqrt{ x'^2 + y'^2 } } 
\end{gathered}
\]
\[
\begin{gathered}  
\text{ Divide the above $2$ by the expression for $y$ } \Longrightarrow \frac{ |d| |y'| }{ \pm d x' } = \frac{2}{y} \\
   \pm \frac{dy}{dx} = \frac{2}{y} \Longrightarrow \pm y dy = 2 dx \\
   \Longrightarrow \pm \frac{1}{2} (y_f^2 - y_i^2 ) = 2 (x_f - x_i)  \\
   \xrightarrow{ (1,2) } \pm (y^2 - 4) = 4(x-1) \Longrightarrow \boxed{ \begin{gathered}
       y^2 = 4 x \\
       y^2 = -4x + 8 
\end{gathered} }
\end{gathered}
\]

\exercisehead{15} $r' = A \times r$
\begin{enumerate}
\item $r'' = a = A\times v$  \quad $\Longrightarrow a \cdot A = A \cdot A \times v = 0$
\item 
\[
\begin{gathered}
  \begin{aligned}
    v & = A \times r \\
    a & = A \times v 
  \end{aligned} \\
  a = s'T + sT', \quad \, a\cdot T = s' \\
  a\cdot T = (A \times v) \cdot T = 0 = s' \quad \Longrightarrow s \text{ constant } 
\end{gathered}
\]
Now note that 
\[
\begin{gathered}
  |r|' = (\sqrt{ r\cdot r})' = \frac{1}{2} \frac{1}{\sqrt{ r\cdot r }} (v\cdot r + r\cdot v) = \frac{v\cdot r}{\sqrt{ r\cdot r}} \\
 \text{ however } v\cdot r = (A\times r)\cdot r = 0 \\
 \Longrightarrow |r|' =0
\end{gathered}
\]
So the length of the position vector $r$ is held constant.  So if $r(0) = B$, then $|r|=\| B\|$ for all time $t>0$.  \\

We have 
\[
|r'(t)| = s = |A \times r| = \| A \| \|r \| \sin{ \theta_{Ar}}
\]
Now $s$ is constant, $\|A\|, \, \| B\|$ are constant.  Then $\sin{ \theta_{Ar}}$ must be constant as well.  $r(0) =B$ so $\theta_{Ar} = \theta_{AB}$.  
\[
\boxed{ |r'(t)| = s = |A \times r| = \| A \| \|B \| \sin{ \theta_{AB}} }
\]
\item See sketch.  $r$ starts off as $B$ and $r$ rotates around $A$ like a top.  
\end{enumerate}

\exercisehead{16}\begin{enumerate}
\item \[
\begin{gathered}
  \begin{aligned} 
    Y(t) & = X[u(t)] \\
    Y'(t) & = \frac{dY(t)}{dt} = \frac{d}{dt} (X(u(t))) = \frac{dX}{du}(u) \frac{du}{dt}
  \end{aligned} \\
  \frac{ Y'(t)}{ \| Y'(t) \| } = T_Y(t) = \frac{ \frac{d}{dt} (X(u(t))) }{ \| \frac{d}{dt} (X(u(t))) \| } = \frac{ \frac{dX(u)}{du} \frac{du}{dt} }{ \| \frac{dX(u)}{du} \| \left| \frac{du}{dt} \right| } = T_X(u(t)) \frac{ \frac{du}{dt} }{ \left| \frac{du}{dt} \right| } \medskip \\
  \text{ since for } X(u(t)), \quad \frac{dX}{du}(u) = X' \quad \Longrightarrow
  \frac{X'}{\| X' \| } = T_X(u) = \frac{ \frac{dX(u)}{du} }{ \| \frac{dX(u)}{du} \| } \bigskip \\
  \text{ Now } \left| \frac{ \frac{du}{dt} }{ \left| \frac{du}{dt} \right| } \right| = 1 \, \text{ always, so for } \medskip \\
  \begin{aligned}
    & \text{ strictly increasing $u$ }, \frac{du}{dt} > 0 \quad & \Longrightarrow \frac{du}{dt} / \left| \frac{du}{dt} \right| = 1 \\
    & \text{ strictly decreasing $u$ }, \frac{du}{dt} < 0 \quad & \Longrightarrow \frac{du}{dt} / \left| \frac{du}{dt} \right| =- 1 
  \end{aligned} \\
\boxed{  \begin{aligned}
    & \text{ strictly increasing $u$ }, & T_Y(t) = T_X(u) \\
    & \text{ strictly decreasing $u$ }, & T_Y(t) = - T_X(u)
  \end{aligned} }
\end{gathered}
\]
\item \[
\begin{gathered}
  (T_Y)' = \left( \frac{Y'}{ |Y'|} \right)' = \frac{Y''}{|Y'|} - \frac{Y'}{|Y'|^2} |Y'|' \\ 
  \begin{aligned}
    Y' & = X'u' \\
    Y'' & = \frac{d}{dt} \left( \frac{dX(u)}{du} u'(t) \right) = \frac{d^2 X(u)}{du^2} u'^2 + \frac{dX(u)}{du}u'' = X''u'^2 + X' u'' \\
    \frac{Y''}{|Y'|} & = \frac{ X'' u'^2 + X' u'' }{ \sqrt{ X' u' \cdot X'u' }} \\
    |Y'|' & = (\sqrt{ Y'\cdot Y'})' = \frac{Y' \cdot Y''}{ \sqrt{ Y'\cdot Y' } } = \frac{X'u'\cdot (X''u'^2 + X'u'') }{ \sqrt{ X'u' \cdot X'u' } }
  \end{aligned} \\
  \frac{-Y'}{|Y'|^2} |Y'|' = \frac{- (X')(X'\cdot (X''u'^2 + X'u'') )}{ (X')^2 \sqrt{ X'u'\cdot X'u' } } \\
  \begin{aligned}
    \frac{Y''}{ |Y'|} - \frac{Y'}{|Y'|^2} |Y'|' & = \frac{X'' u'^2 + X'u''}{ \sqrt{ X'u' \cdot X' u' } } - \frac{ (X')(X'\cdot X'' u'^2 + X'^2 u'' ) }{ X'^2 (\sqrt{ X'u' \cdot X'u' } ) } = \frac{ X'' u'^2 }{ \sqrt{ X'u' \cdot X'u' } } - \frac{X' (X'\cdot X'')u'^2 }{ X'^2 \sqrt{ X'u' \cdot X'u' } } =  \\
    \quad \\
    & \xrightarrow{ \begin{aligned}
	u'^2 & = |u'|^2 \\
	\sqrt{u'^2} & = |u'| 
      \end{aligned} } \left( \frac{X''}{|X'|} - \frac{X' |X'|'}{|X'|^2 } \right)|u'|
  \end{aligned}
\end{gathered}
\] 
\quad \\
\[
\begin{gathered}
  \begin{aligned}
    \| T_Y' \|^2 & = \left( \frac{Y''}{|Y'|} - \frac{Y'}{|Y'|^2} |Y'|' \right) \cdot \left( \frac{Y''}{|Y'|} - \frac{Y'}{|Y'|^2} |Y'|' \right) = \frac{Y''^2}{Y'^2} - \frac{2 Y''\cdot Y' }{ |Y'|^3} \frac{Y'\cdot Y'' }{ |Y'| } + \frac{Y'^2}{ |Y'|^4} \frac{ (Y'\cdot Y'' )^2}{ Y'^2 } = \\
    & = \frac{Y''^2}{Y'^2 } - \frac{(Y'\cdot Y'')^2}{ (Y')^4 } \\
    & = \frac{(X'' u'^2 + X'u'')^2 }{ (X'u')^2 } - \frac{ (X'u' \cdot (X''u'^2 + X'u'') )^2 }{ (X'u')^4} = \\
    & = \frac{ X''^2 u'^4 + 2X''\cdot X' u'^2 u'' + X'^2 u''^2 }{ (X'u')^2 } - \frac{ (X'\cdot X'')^2 u'^6 + 2X'\cdot X'' u'^4 X'^2 u'' + (X'^2)^2 u'^2 u''^2 }{ (X'u')^4 } = \\
    & = \left( \frac{X''^2}{X'^2} - \frac{ (X'\cdot X'')^2}{ X'^4} \right) u'^2 
  \end{aligned}  \\
  \frac{T_Y'}{ \| T_Y' \| } = \frac{T_X' |u'| }{ \| T_X' \| |u'|} = \frac{T_X'}{\| T_X' \| } \Longrightarrow \boxed{ N_Y = N_X }
\end{gathered}
\]

Thus, the osculating plane is invariant under a parameter change, since the osculating plane can still be made up of $N_Y=N_X$ and $T_Y = \pm T_X$.  
\end{enumerate}


%-----------------------------------%-----------------------------------%-----------------------------------
\subsection*{ 14.13 Exercises - The definition of arc length, Additivity of arc length, The arc-length function }
%-----------------------------------%-----------------------------------%-----------------------------------
\quad \\
\exercisehead{1}
\[
\begin{aligned}
  r(t) & = a (1-\cos{t}, t- \sin{t} ) \quad \, & 0 \leq t \leq 2 \pi, \, a >0   \\
  r'(t) & = a(\sin{t},1-\cos{t})
\end{aligned} \quad \, v^2 = a^2 (\sin^2{t} + 1 - 2 \cos{t} + \cos^2{t}) = 2a^2 ( 1 - \cos{t})
\]
\[
\begin{gathered}
  v = \sqrt{2} a \sqrt{ 1 - \cos{t}} = 2a \sin{t/2} \\
  \int v dt = 2a \int \sin{t/2} = -(4a) \left. \cos{t/2} \right|_0^{2\pi } = \boxed{ 8a}
\end{gathered}
\]

\exercisehead{2}
\[
\begin{aligned}
  r(t) & = (e^t \cos{t}, e^t \sin{t} ); \quad \, 0 \leq t \leq 2 \\
  r'(t) & = (e^t(c-s), e^t (s+c) ) \\
  v^2 & = e^{2t} ( c^2 + s^2 + s^2 + c^2 ) = 2e^{2t} \quad \Longrightarrow v = \sqrt{2} e^t \\
  \int_0^2 \sqrt{2} e^t dt & = \boxed{ \sqrt{2} (e^2 - 1) }
\end{aligned}
\]

\exercisehead{3} 
\[
\begin{aligned}
  r(t) & = (a(\cos{t} + t\sin{t}), a(\sin{t} - t \cos{t}) ) = a(\cos{t} + t\sin{t}, \sin{t} - t\cos{t}) \\
  r' & = a(-s + s + tc, c - c +ts) = a(tc,ts) \\
  v & = at \\
  \int_0^{2\pi} at & = a \frac{1}{2} 4\pi^2 = \boxed{ 2\pi^2 a}
\end{aligned}
\]

\exercisehead{4} 
\[
\begin{aligned}
  r(t) & = \left( \frac{c^2}{a} \cos^3{t}, \frac{c^2}{b} \sin^3{t} \right) = C^2 \left( \frac{c^3}{A}, \frac{s^3}{B} \right) \quad \, \begin{aligned} 
    & 0 \leq t \leq 2 \pi \\
    & c^2 = a^2 - b^2 , \, 0 < b < a 
  \end{aligned} \\
  r' & = 3C\left( \frac{-c^2 s }{A}, \frac{s^2 c}{B} \right) \\
  v^2 & = 9C^2 \left( \frac{c^4 s^2}{A^2} + \frac{s^4 c^2}{B^2} \right) = 9 C^4 c^2 s^2 \left( \frac{c^2}{A^2} + \frac{s^2}{B^2} \right) \\
  v & = 3C^2 |cs| \sqrt{ \frac{c^2}{A^2} + \frac{s^2}{B^2} } = 3C^2 |cs| \frac{ \sqrt{B^2 + C^2 s^2}}{AB}
\end{aligned}
\]
\[
\begin{gathered}
  \int cs \sqrt{ B^2 + C^2 s^2} = \frac{1}{3C^2} (B^2 + C^2 s^2)^{3/2} \\
  \begin{aligned} 
    s & = \frac{3C^2}{AB} \left( \int_0^{\pi/2} cs \sqrt{ B^2 + C^2 s^2 } - \int_{\pi/2}^{\pi} cs \sqrt{ B^2 + C^2 s^2 } + \int_{\pi}^{3\pi/2} cs \sqrt{ B^2 + C^2 s^2 } - \int_{3\pi/2}^{2\pi} cs \sqrt{ B^2 + C^2 s^2 } \right) = \\
    & = \frac{3C^2}{AB} \frac{4}{3C^2}( (B^2 + C^2)^{3/2} - B^3) = \boxed{ 4 (A^3 - B^3)/AB }
  \end{aligned}
\end{gathered}
\]

\exercisehead{5} 
\[
\begin{aligned}
  r(t) & = (a(\sinh{t}-t), a(\cosh{t} -1) ) \\
  r'(t) & = a(\cosh{t} - 1,\sinh{t}) \\
  v^2 & = a^2 (\cosh^2{t} - 2\cosh{t} + 1 + \sinh^2{t} ) = a^2 (2) (\cosh{t})(\cosh{t}-1)
\end{aligned}
\]
\[
\begin{gathered}
\text{ Knowing that } \cosh{(2t)} = \cosh^2{t} + \sinh^2{t} = 2 \cosh^2{t} - 1 = 1 +2 \sinh^2{t} \\
  \begin{aligned}
    s & = \int |v| dt = \int_0^T a\sqrt{2} \sqrt{ \cosh{t}(\cosh{t} -1) } dt = \sqrt{2} a \int_0^T \sqrt{ 2 \cosh^2{t/2}-1} \sqrt{2} \sinh{t/2} dt = \\
    & \xrightarrow{ \begin{aligned}
	u & = \cosh{t/2} \\
	du & = \frac{1}{2} \sinh{t/2} dt 
\end{aligned} } 2a \int \sqrt{ 2u^2 - 1} 2 du \xrightarrow{ \begin{aligned} \sqrt{2} u & = y \\
	du & = \frac{dy}{\sqrt{2}} \end{aligned} } \frac{4 a}{\sqrt{2}} \int \sqrt{ y^2 - 1 } dy 
  \end{aligned} \\
\begin{gathered}
  \int \sqrt{ y^2 - 1 } dy = y \sqrt{ y^2 - 1 } - \int \frac{y^2}{ \sqrt{y^2 -1 } } = y \sqrt{ y^2 -1} -\int \frac{ y^2 - 1 + 1 }{ \sqrt{ y^2 - 1 }} = y \sqrt{ y^2 - 1 } - \int \sqrt{ y^2 - 1 } dy + \int \frac{-1}{ \sqrt{ y^2 -1 } } \\
  \text{ since } ( \ln{ (y \pm \sqrt{ y^2 - 1 } ) } )' = \frac{ 1}{ y \pm \sqrt{ y^2 - 1 } } \left( 1 \pm \frac{y}{\sqrt{ y^2 - 1 } } \right) = \pm \frac{1}{ \sqrt{ y^2 - 1 }} \\
  \Longrightarrow \int \sqrt{ y^2 - 1 } dy = \frac{1}{2} (y \sqrt{ y^2 - 1 } - \ln{ (y + \sqrt{ y^2 - 1 } ) } ) 
\end{gathered}
\end{gathered}
\]
\[
\begin{aligned}
  s & = \frac{4a}{\sqrt{2}} \frac{1}{2} \left. \left( \sqrt{2} \cosh{t/2} \sqrt{ 2\cosh^2{t/2} - 1 } - \ln{ (\sqrt{2} \cosh{t/2} + \sqrt{ 2 \cosh^2{t/2} - 1 } ) } \right) \right|_0^T = \\
  & = \boxed{ 2a (\cosh{T/2} \sqrt{ \cosh{T}} - 1 ) + -\sqrt{2} a \ln{ \left( \frac{ \sqrt{2} \cosh{T/2} + \sqrt{\cosh{T}} }{ \sqrt{2} + 1 } \right) } }
\end{aligned}
\]

\exercisehead{6}
\[
\begin{aligned}
  r(t) & = (\sin{t}, t, (1-\cos{t}) ) ;\quad \, (0 \leq t \leq 2 \pi) \\
  r' & = (c,1,s) \\
  v^2 & = c^2 + 1 + s^2 = 2 \quad \Longrightarrow \int_0^{2\pi} \sqrt{2} dt = \boxed{ \sqrt{2} (2\pi)  }
\end{aligned}
\]

\exercisehead{7} 
\[
\begin{aligned}
  r(t) & = (t,3t^2,6t^3), \quad \, (0\leq t \leq 2) \\
  r' & = (1,6t, 18t^2) \\
  v^2 & = 1 + 36 t^2 + 18^2 t^4 \\ 
  s & = \int_0^2 \sqrt{ (18t^2)^2 + 2 (18)t^2 + 1 } dy = \int_0^2 (18t^2 + 1) dt = \left. ( 6t^3 +t)\right|_0^2 = \boxed{ 50 }
\end{aligned}
\]

\exercisehead{8} 
\[
\begin{aligned}
  r & = (t,\log{(\sec{t})}, \log{ (\sec{t} + \tan{t}) } ), \quad \, (0 \leq t \leq \frac{ \pi}{4} ) \\
  r' & = (1,\tan{t}, \sec{t}) \\
  v^2 & = 1 + \tan^2{t} + \sec^2{t} = 2 \sec^2{t} \\
  s & = \int_0^{\pi/4} \sqrt{2} \sec{t} dt = \sqrt{2} \left. (\ln{ |\tan{t} + \sec{t}| })\right|_0^{\pi/4} = \boxed{ \sqrt{2} \ln{ (1+\sqrt{2}) } }
\end{aligned}
\]

\exercisehead{9}
\[
\begin{aligned}
  r(t) & = (a\cos{\omega t}, a \sin{\omega t}, b\omega t), \quad (t_0 \leq t \leq t_1) \\
  r' & = (\omega)(-as(\omega t), a c(\omega t), b) \\
  v^2 = \omega^2(a^2 + b^2 ) \\
  s & = \int_{t_0}^{t_1} |\omega | \sqrt{ a^2 + b^2} = \boxed{ |\omega | \sqrt{ a^2 + b^2} (t_1 -t_0) }
\end{aligned}
\]

\exercisehead{10}
\[
\begin{aligned}
  r & = (x,y) = (g(y),y) \\
  r' & = \left( \frac{dg(y(t))}{ dy } \frac{dy}{dt}, \frac{dy}{dt} \right) \\
  r'^2 & = \left( \left( \frac{dg}{dy} \right)^2 + 1 \right) \left( \frac{dy}{dt} \right)^2 \\
  s & = \int_{t_0}^{t_f} \sqrt{ 1 + \left( \frac{dg(y(t))}{dy} \right)^2 } \frac{dy}{dt} dt = \boxed{ \int_{y(t_0)}^{y(t_f)} \sqrt{ 1 + \left( \frac{dg}{dy} \right)^2 } dy  }
\end{aligned}
\]

\exercisehead{11} Given $y^2 = x^3$, 
\[
\begin{aligned}
  r & = (x,\pm x^{3/2} ) \\
  \frac{dy}{dx} & = \pm \frac{3}{2} x^{1/2} \\
  \int_0^1 \sqrt{ 1 + \frac{9}{4}x } & = \left. \frac{8}{27} (1 + \frac{9}{4} x)^{3/2} \right|_0^1 = \boxed{ \frac{8}{27} \left( \frac{26\sqrt{13}}{ 8} - 2 \right) } 
\end{aligned}
\]
\[
\begin{aligned}
  r & = (y^{2/3},y) \\
  \frac{dg}{dy} & = \frac{2}{3} y^{-1/3} \\
  \int_{-1}^1 \sqrt{ 1 + \left( \frac{2}{3} \right)^2 y^{-2/3} } dy & = \int_{-1}^1 \frac{ \sqrt{ y^{2/3} + \left( \frac{2}{3} \right)^2 } }{ |y|^{1/3}} dy = \left. (y^{2/3} + (2/3)^2 )^{3/2} \right|_0^1 - \left. (y^{2/3} + (2/3)^2)^{3/2} \right|_{-1}^0 = \\
  & = \left( \frac{13}{9} \right)^{3/2} - \left( \frac{2}{3} \right)^3 - \left( \left( \frac{2}{3} \right)^3 - \left( \frac{13}{9} \right)^{3/2} \right) = \boxed{ 2 \left( \frac{13\sqrt{13}}{ 27} - \frac{8}{27} \right) }
\end{aligned}
\]

\exercisehead{12} 
\[
\begin{aligned}
  r & = (\cos{\theta}, \sin{\theta}) \\
  r' & = (-\sin{\theta}, \cos{\theta} ) \\
  v^2 & = 1 \\
  s & = \int_{\theta_0}^{\theta_1} 1 d\theta = (\theta_1 - \theta_0 ) = 2 \left( \frac{\theta_1 - \theta_0 }{2} \right)
\end{aligned}
\]
So the arc $AB$ has a length equal to twice the area of the sector.  

\exercisehead{13}  
\begin{enumerate}
\item \[
\begin{gathered}
  y = e^x, \quad 0 \leq x \leq 1 \\
  \int_0^1 \sqrt{ 1 + e^{2x}} dx 
\end{gathered}
\]
\item \[
\begin{gathered}
  \begin{aligned} 
    (x,y) & = (t + \log{t}, t - \log{t}); \quad 1 \leq t \leq e \\
     r' & = (1 + \frac{1}{t}, 1-\frac{1}{t} ) \\
     r'^2 & = 2 (1+\left( \frac{1}{t} \right)^2 ); \Longrightarrow \sqrt{2} \sqrt{ 1 + \left( \frac{1}{t} \right)^2 }
  \end{aligned} \\
  \int_1^e \sqrt{2} \sqrt{ 1 + (1/t)^2 } dt \xrightarrow{ \begin{aligned}
      t & = e^x \\
      dt &= e^x dx 
\end{aligned} } \int_0^1 \sqrt{2} \sqrt{ 1 + e^{-2x} } e^x dx = \sqrt{2} \int_0^1 \sqrt{ 1 + e^{2x}} dx
\end{gathered}
\]
\end{enumerate}

\exercisehead{14} 
\[
\begin{aligned}
  y & = c\cosh{ (x/c)} \\
  \frac{dy}{dx} & = \sinh{ (x/c) } \\
  \int_0^a \sqrt{ 1 + \sinh^2{ (x/c)} }dx & = \int_0^a \cosh{x/a} dx = \left. c \sinh{ \frac{x}{c} } \right|_0^a = c \sinh{ \frac{a}{c} }
\end{aligned}
\]
So indeed, $ \int_0^a c \cosh{(x/a)} \,  dx$ gives the area underneath a curve $\cosh{x/a}$ from $x=0$ to $x=a$.  

\exercisehead{15} Given $y= \cosh{x}$, \quad $(0,1)$ and $(x,\cosh{x})$ is $\sinh{x}$ if $x>0$ 
\[
\int_0^x \sqrt{ 1 + \sinh^2{t}} dt = \left. \sinh{t} \right|_0^x = \boxed{ \sinh{x} }
\]

\exercisehead{16} 
\[
\begin{gathered}
  A = \int_a^b f(x) dx = k \int_a^b \sqrt{ 1 + \left( \frac{dy}{dx} \right)^2 } dx \Longrightarrow \int_a^b \left( k \sqrt{ 1 + \left( \frac{dy}{dx} \right)^2 } - f(x) \right) dx = 0 \\
    \pm \sqrt{ \left( \frac{y}{k} \right)^2 - 1 } = \frac{dy}{dx}; \quad \, \pm (x-x_0) = \int_{y_0}^{y} \frac{dy}{ \sqrt{ \left( \frac{y}{k} \right)^2 - 1 } }    
\end{gathered}
\]
$\int_{y_0}^y \frac{dy}{ \sqrt{ \left( \frac{y}{k} \right)^2 - 1 } }$ could be evaluated by $\cosh{u} = y/k$ substitution, and then solve for $u$, (inverse of $\cosh{u}$), by $\cosh{u} = \frac{ e^u + e^{-u}}{2} = y/k$ and then use the quadratic equation trick to get $e^u$.  

\[
\begin{gathered}
  \pm (x-x_0) = k \ln{ \frac{ \left( \frac{y}{k} + \sqrt{ \frac{y^2}{k^2} - 1 } \right) }{ \left( \frac{y}{k} + \sqrt{ \frac{y^2}{k^2} - 1 } \right) } } \quad \text{ Let } C_0 = \frac{y_0}{k} \pm \sqrt{ \frac{y_0^2}{k^2} - 1 } \\
  \left( C_0 e\left( \frac{ \pm(x-x_0) }{k} \right) - \frac{y}{k} \right)^2 = \frac{y^2}{k^2} - 1 \\
  C_0^2 e\left( \frac{ \pm 2 (x-x_0) }{k} \right) + 1 = 2 C_0 e\left( \frac{ \pm (x-x_0)}{k} \right) \frac{y}{k} \\
  \boxed{ y = \frac{ k \left( C_0 e\left( \frac{\pm (x-x_0)}{k} \right) + e\left( \frac{ \mp (x-x_0) }{k} \right) \right) }{2} }
\end{gathered}
\]
If we choose $y_0 =k$, so that $C_0 = 1$, then $\boxed{ y = k \cosh{ \left( \frac{x-x_0}{k} \right) } }$.  Also note that $y=k$ works as well.  

\exercisehead{17}
\[
\begin{aligned}
  r & = (a\sin{t},b\cos{t}) \quad \, 0 < b < a \\
  r' & = (ac,-bs) \\
  r'^2 & = a^2 c^2 + b^2 s^2 = a^2 (1-s^2 ) + b^2 s^2 = a^2 \left( 1 - \left( \frac{a^2 -b^2 }{a^2} \right) s^2 \right) \\
  L & = \int_0^{2\pi} |a| \sqrt{ 1 -e^2 \sin^2{t} } dt = \boxed{ 4a \int_0^{\pi/2} \sqrt{ 1 - e^2 \sin^2{t} } dt }
\end{aligned}
\]

\exercisehead{18}
\[
\begin{aligned}
  r(t) & = (a(t-\sin{t}),a(1- \cos{t}), b\sin{t/2} ) = a(t-\sin{t}, (1-\cos{t}), \frac{b}{a} \sin{t/2} ) \\
  r' & = a(1-\cos{t}, \sin{t}, \frac{b}{2a} \cos{t/2} ) \\
  r'^2 & = a^2 ( 1 - 2\cos{t} + 1 + \left( \frac{b}{2a} \right)^2 \cos^2{t/2} ) = 2a^2 ( 1 - \cos{t} + \left( \frac{b}{4a} \right)^2 (1 + \cos{t}) ) = \\
  & = 2a^2 ( 2 \sin^2{t/2} + \left( \frac{b}{4a} \right)^2 ( 2 + -2 \sin^2{t/2} )) = 4a^2 \left( \left( \frac{b}{4a} \right)^2 + k^2 \sin^2{t/2} \right) = 4a^2 ( 1 - k^2 + k^2 \sin^2{t/2} ) = \\
  & = 4a^2 ( 1 -k^2 \cos^2{t/2} ) \\
  |r'| & = 2a \sqrt{ 1 - k^2 \cos^2{t/2} }
\end{aligned}
\]
\[
  \begin{aligned}
    \int_0^{2\pi} 2a \sqrt{ 1-  k^2 \cos^2{t/2} } \, dt & = 4a \int_0^{\pi} \sqrt{ 1-  k^2 \cos^2{t}} \, dt \\
    \xrightarrow{ t = \pi/2 - \theta } & = (-4a) \int_{\pi/2}^{-\pi/2} \sqrt{ 1-  k^2 \sin^2{t}} \, dt = \boxed{ 8a \int_0^{\pi/2} \sqrt{ 1 - k^2 \sin^2{t}} \, dt }
  \end{aligned}
\]


\exercisehead{19} 
\[
  \begin{aligned}
    A\cdot B & = \cos{ \pi/3} = 1/2 \\
    (A\times B)^2 & = A^2 B^2 - (A\cdot B)^2 = 1 - 1/4 = 3/4 
  \end{aligned} 
\]
\[
\begin{gathered}
  \begin{aligned}
    r & = tA + t^2 B + 2 \left( \frac{2}{3} t \right)^{3/2} A \times B \\
    r' & = A + 2tB + 2 \left( \frac{2}{3} t \right)^{1/2} A \times B \\
    r'^2 & = 1 + 2t \left( \frac{1}{2} \right) + \frac{2t}{2} + (2t)^2 + 4 \left( \frac{2 t}{3} \right) \frac{3}{4} = 1 + 4t + 4t^2 = (2t+1)^2 
\end{aligned} \\
L = \int_0^T (2t+1) dt = T^2 + T = 12 \Longrightarrow (T+4)(T-3) = 0 \quad \boxed{ T = 3 }
\end{gathered}
\]

\exercisehead{20}
\begin{enumerate}
\item \quad \\
  The circle rotates about its center and the circles \\
  \phantom{ The circle } rolls (moves along $x$-axis) without slipping, translating its origin by how far its circumference moved through.  \\

$\theta_0 = -\pi/2$ \smallskip \\
  $a (\cos{\theta},-\sin{\theta})$ \quad (clockwise rotation of pt. on circumference) \\
  $a(\cos{(\theta - \theta_0)}, - \sin{ (\theta- \theta_0) }) = a (-\sin{\theta},-\cos{\theta})$ \\
  $(\theta a,a)$ is the position of the circle center.  \bigskip \\
  $(\theta a,a) + a (-\sin{\theta}, -\cos{\theta}) = \boxed{ a (\theta - \sin{\theta}, 1 - \cos{\theta} ) }$
\item $\frac{dy}{dx} = \frac{\sin{\theta} }{ 1 - \cos{\theta} } = \frac{ 2 \sin{ \theta/2} \cos{\theta/2}}{ 2 \sin^2{ \theta/2} } = \boxed{ \cot{\theta/2} }$ 
\[
\begin{gathered}
  \begin{aligned}
    T & = \frac{ (1,\cot{\theta/2})}{ \sqrt{ 1 + \cot^2{\theta/2}}} = (\sin{\theta/2}, \cos{\theta/2}) \\
    T\cdot e_x & = \sin{\theta/2} = \cos{ \phi} = \cos{ \left( \frac{\pi}{2} - \frac{ \theta}{2} \right)} 
  \end{aligned} \\
  \boxed{ \phi = \frac{ \pi - \theta}{2} }
\end{gathered}
\]

We'll show that the tangent line passes through the highest point at each place the marked point rotates through.  \smallskip \\
Marked point on the circle rotated through by $\theta$: $P = a (\theta- \sin{\theta}, 1 - \cos{\theta} )$ \\
Tangent line: $P + sT = a(\theta-\sin{\theta},1 - \cos{\theta}) + s (\sin{\theta/2},\cos{\theta/2})$ \\
Highest pt. on the circle: $(a\theta, 2a)$ \\

\[
  \begin{aligned}
 & \text{ equate the $y$-coordinates: } & 2a = a(1-\cos{\theta}) + s \cos{\theta/2} \Longrightarrow s = 2a |\cos{\theta/2}| \\
    & \quad & \quad \\
    & \text{ equate the $x$-coordinates: } & 
    \begin{aligned}
      a(\theta - \sin{\theta}) + s \sin{\theta/2} & = a \theta - a \sin{\theta} +s \sin{\theta/2} = \\
      & = a \theta - a\sin{\theta} + a 2\cos{\theta/2}\sin{\theta/2} = \\
      & = \boxed{ a \theta }
    \end{aligned}
  \end{aligned}
\]
So indeed, the tangent line will always intersect the highest point of the circle.  
\end{enumerate}

\exercisehead{21} 
\[
\begin{gathered}
  \frac{dY}{dt} = \frac{dX}{du} \frac{du}{dt} \\
  \int_c^d \| Y'(t) \| dt = \int_c^d \| X'(u(t)) \| \left| \frac{du(t)}{dt} \right| dt 
\end{gathered} \quad \quad \, 
\begin{aligned}
  g(u) & = t \\
  u & = u(t) \\
  \frac{du}{dt} & = 1 / \frac{dg}{du}(u(t)) 
\end{aligned}
\]
$g(u) =t $ is strictly increasing as it traces the arc, since $c<t<d$.  \smallskip \\
Then $\frac{du}{dt} > 0$
\[
\Longrightarrow \int_c^d \| X'(u) \| \left( \frac{du}{dt} \right) dt = \int_{u(c)}^{u(d)} \| X'(u) \| du
\]

\exercisehead{22} For $t= \frac{1}{k}$, \quad $k = 2,\dots, 2n$
\[
\| \pi (P) \| = \sum_{k=2}^{2n} \| r (t_k) - r(t_{k-1}) \| + \| r(t=0) - r(t_{2n}) \| + \text{(term to close up this polygon)}
\]
\[
\| r (t = 0) - r(t_{2n}) \| = \left( \frac{1}{2n} \right) \sqrt{2}
\]
\[
\begin{aligned}
  r & = (t,t\cos{ \left( \frac{ \pi }{2t } \right) } ) \\
  r(t_k) - r(t_{k-1}) & = \left( \frac{-1}{ k(k-1)}, \frac{1}{k} \cos{ \left( \frac{k\pi}{2} \right) } - \frac{1}{k-1} \sin{ \left( \frac{k \pi}{2} \right) } \right) \\
  \| r(t_k) - r(t_{k-1}) \|^2 & = \left( \frac{1}{k (k-1)} \right)^2 + \frac{1}{k^2} \cos^2{ \left( \frac{k \pi}{2} \right) } + \frac{ -2 \cos{ \left( \frac{k \pi }{2} \right) } \sin{ \left( \frac{k \pi}{2} \right) } }{ k (k-1) } + \frac{1}{(k-1)^2} \sin^2{ \left( \frac{k \pi }{2} \right) } = \\
  & = \left( \frac{1}{k (k-1)} \right)^2 + \frac{1}{k^2} \cos^2{ \left( \frac{k \pi}{2} \right) }  + \frac{1}{(k-1)^2} \sin^2{ \left( \frac{k \pi }{2} \right) } \text{ since $k \in \mathbb{Z}$ }
\end{aligned}
\]

$k=2, \dots, 2n$
\[
\begin{aligned}
  & k = 2j \quad j = 1, \dots, n \\
  & \begin{aligned}
    \| r(t_k) - r(t_{k-1}) \|^2 & = \left( \frac{1}{ k (k-1)} \right)^2 + \frac{1}{k^2} \\
    \| r(t_k) - r(t_{k-1}) \| & = \frac{1}{k} \sqrt{ 1 + \frac{1}{ (k-1)^2} } > \frac{1}{k}
  \end{aligned} \\
  & k = 2j+1, \quad j = 1, \dots , n-1 \\
  & \begin{aligned}
    \| r(t_k) - r(t_{k-1}) \|^2 & = \left( \frac{1}{ k (k-1)} \right)^2 + \left( \frac{1}{k-1} \right)^2 = \left( \frac{1}{k-1} \right)^2 \left( 1 + \frac{1}{k^2} \right) \\
    \| r(t_k) - r(t_{k-1}) \| & = \frac{1}{k-1} \sqrt{ 1 + \frac{1}{k^2} } > \frac{1}{k-1} > \frac{1}{k}
\end{aligned} 
\end{aligned}
\]
These don't seem like very ``tight'' inequalities, but it'll work for the purposes of this problem.  
\[
\begin{aligned}
  \| \pi (P) \| & = \sum_{k=2}^{2n} \| r (t_k) - r(t_{k-1}) \| + \| r(t=0) - r(t_{2n}) \| + \text{(term to close up this polygon)} > \\
  & > \sum_{k=2}^{2n} \frac{1}{k} + \frac{\sqrt{2}}{2n} + \| r(t=0) - r(t=1) \| =  \sum_{k=2}^{2n} \frac{1}{k} + \frac{\sqrt{2}}{2n} + 1 > \boxed{ > \sum_{k=1}^{2n} \frac{1}{k} }
\end{aligned}
\]
Now, by definition, $\Lambda$, the arclength, is the least upper bound on all possible polgonal approximations to the length of a curve.  $\sum_{k=2}^{2n} \frac{1}{k}$ is a series that is unbounded as $n\to \infty$.  So $\Lambda$ does not exist.  

%-----------------------------------%-----------------------------------%-----------------------------------
\subsection*{ 14.15 Exercises - Curvature of a curve }
%-----------------------------------%-----------------------------------%-----------------------------------
\quad \\
\exercisehead{1}
\begin{enumerate}
\item \[
\begin{gathered}
  \begin{aligned}
    r & = ((3t-t^3), 3t^2,(3t+t^3)) \\
    v & = (3-3t^2, 6t, 3+3t^2) = 3(1-t^2, 2t, 1+t^2) \\
    a & = 3(-2t,2,2t) = 6(-t,1,t) 
  \end{aligned} \quad \quad \,
  \begin{aligned}
    a (t=2) &  = 6(-2,1,2); \quad & |a| = 18 \\
    v(t=2) & = 3(-3,4,5); \quad & |v| = 15\sqrt{2} \\
    a\cdot v & = 18(6+4+10) = 360 
  \end{aligned}\\ 
  \kappa = \frac{ \sqrt{ 18^2 15^2 2 - (18\cdot 20)^2 } }{ 15^3 \sqrt{2} \cdot 2 } = \boxed{ \frac{1}{75 } }
\end{gathered}
\]
\item \[
\begin{gathered}
  \begin{aligned}
    r & = (c,s,e^t) \\
    v & = (-s,c,e^t) \\
    a & = (-c,-s,e^t)
  \end{aligned} \quad \quad \, 
\begin{aligned}
  a(t=\pi) & = (1,0,e^{\pi}) \quad & |a| = \sqrt{ 1 + e^{2\pi} } \\
  v(t=\pi) & = (0,-1,e^{\pi}) \quad & |v| = \sqrt{ 1 + e^{2\pi} } \\
  a\cdot v & = e^{2\pi }
\end{aligned} \\
\kappa = \frac{ \sqrt{ 1 + 2 e^{2\pi } + e^{4\pi} - e^{4\pi} } }{ (1+ e^{2\pi} )^{3/2} } = \boxed{ \frac{ \sqrt{ 1 + 2e^{2\pi }} }{ (1+e^{2\pi})^{3/2} } }
\end{gathered}
\]
\item \[
\begin{gathered}
  \begin{aligned}
    r & = (3tc,3ts,4t) \\
    v & = (3(c-ts), 3(s+tc), 4) \\
    a & = 6(-s,c,0) + (-3tc,3ts, 0)
  \end{aligned} \quad \quad 
  \begin{aligned}
    a(t=0) & = 6(0,1,0) \quad \, & |a| = 6 \\
    v(t=0) & = (3,0,4) \quad \, & |v| = 5 \\
    a\cdot v & = 0 
  \end{aligned} \\
  \boxed{ \kappa = \frac{6}{25} }
\end{gathered}
\]
\item \[
\begin{gathered}
  \begin{aligned}
    r' & = (1-c,s,2\cos{t/2} ) \\
    r'' & = (s,c,-\sin{t/2} ) 
  \end{aligned}  \quad \quad \, 
  \begin{aligned}
    & r'' \cdot r' = s - cs + sc - 2\cos{t/2}\sin{t/2} = 0 \\
    & v = \| (2,0,0) \| = 2 \\
    & a = \| (0,-1,-1) \| =\sqrt{2} 
  \end{aligned} \\
  \boxed{ \kappa = \frac{ \sqrt{2}}{4 } }
\end{gathered}
\]
\item \[
  \begin{gathered}
    \begin{aligned}
      r & = (3t^2,2t^3,3t) \\
      r' & = (6t,6t^2, 3) = 3(2t,2t^2,1 ) \\
      r'' & = 3(2,4t,0) \\
    \end{aligned} \quad \quad \, 
    \begin{gathered} 
      r'' \cdot r' = 9 (4t+ 8t^3) \xrightarrow{ t= 1 } 9 (12) \\
    \begin{aligned}
      a & = 3 \sqrt{ (4 + 16t^2 ) } \\
      a(t=1) & = 6 \sqrt{5} \\
      v & = 3 \sqrt{ 4t^2 + 4t^4 + 1 }; \quad \, v(t=1) = 9 
    \end{aligned} 
    \end{gathered} \\
    \frac{ \| a \times v \| }{ v^3 } = \frac{ \sqrt{ (6 \sqrt{5})^2 9^2 - (9(12))^2 } }{ 27^2 } = \boxed{ \frac{2}{27} }
\end{gathered}
\]
\item \[
\begin{gathered}
  \begin{aligned}
    r(t) & = (t,\sin{t},(1-\cos{t}) ) \\
    r'(t) & = (1,\cos{t},\sin{t}) \\
    r'' & = (0,-s,c) 
  \end{aligned} \quad \quad \,
  \begin{aligned}
    r'' \cdot r' & = 0 \\
    a & = 1 \\
    \kappa & = \frac{ \| a \times v \| }{ v^3 }  = \frac{ \sqrt{2}}{ 2^{3/2} } = \boxed{ \frac{1}{2} }
  \end{aligned}
\end{gathered}
\]
\end{enumerate}

\exercisehead{2}
\[
\begin{gathered}
  \begin{aligned}
    r & = (ac(\omega t), a s(\omega t), b\omega t) \\
    r' & = \omega ( -as(\omega t), a c(\omega t), b ) \\
    r'' & = \omega^2 ( -ac(\omega t), -as(\omega t), 0)
  \end{aligned} \quad \quad \, 
\begin{aligned}
  & r'\cdot r'' = 0 \\
  & |r'| = \omega \sqrt{ a^2 + b^2 } \\
  & |r''| = \omega^2 a 
\end{aligned}  \\
\frac{ \| r'' \times r' \| }{ |r'|^3 } = \frac{ a \sqrt{ a^2 + b^2 } }{ (a^2 + b^2 )^{3/2} } = \boxed{ \frac{ a}{ a^2 + b^2 } }
\end{gathered}
\]

\exercisehead{3} Given $A,B$, \, $A\cdot B = \cos{\theta}$.  $A,B$ fixed.  $|A| = |B| = 1$.  
\[
\begin{gathered}
  \begin{aligned}
    r' & = A \times r \\
    r'' &  = A \times r'
  \end{aligned} \quad \Longrightarrow r(0) = B, \quad \| r(t) \| = 1 \\
  \begin{aligned}
    r'' \times r' & = (A\times r') \times (A \times r) = (A \times (A \times R)) \times (A \times r) = \\
    & = (A \times r) \times ((A \times r) \times A) = (A \cdot (A\times r) )( A \times r) - (A \times r)^2 A = -(A\times r)^2 A 
  \end{aligned} \\
  \frac{ \| r'' \times r' \| }{ \| A \times r \|^3 } = \frac{1}{ \| A \times r \| } = \boxed{ \frac{ 1 }{ \| B \| \sin{\theta} }  }
\end{gathered}
\]

\exercisehead{4} 
\begin{enumerate}
\item $r(t)$ lies on a plane since a plane $M = \{ P + sA + tB \}$ \\
$r(t) = 4(c,s,c) = 4c(1,0,1) + 4s(0,1,0) = 4cA + 4sB$ \\
  $r(t) \in M_1$ where $M_1 = \{ 0 + sA + tB \} = \{ sA + tB \}$ \\

Normalize $A,B \to A = \frac{ (1,0,1)}{\sqrt{2} }; \quad B = (0,1,0) $ \\
$X = 4 \sqrt{2} cA + 4sB$ \\

Recall an ellipse: $\| X - F \| = ed(X,L) = e | (X - (F + dN) )\cdot N | = e| (X-F)\cdot N - d |$\\
Suppose $ \| F \| = f$ and suppose $F = fA$ 
\[
\begin{gathered}
  \begin{aligned}
    X-F & = (4\sqrt{2} c -f)A + 4sB \\
    \| X- F \|^2 & = 32 c^2 -8\sqrt{2} cf + f^2 + 16s^2 = 16 + 16 c^2 - 8\sqrt{2} cf + f^2 \\
    e^2 ((X-F)\cdot N -d)^2 & = e^2 (((X-F)\cdot N)^2 - 2((X-F)\cdot N )d + d^2) 
  \end{aligned}\\
  \begin{aligned}
    ((X-F)\cdot N)^2 & = 32 c^2 - 8\sqrt{2} cf + f^2 \xrightarrow{ e^2 = 1/2} 16c^2 - 4\sqrt{2} cf + \frac{1}{2} f^2 \\
    -2e^2 ((X-F)\cdot N)d & = -d (4\sqrt{2} c - f) \Longrightarrow d=f \\
    e^2 d^2 & = \frac{1}{2} f^2 \Longrightarrow 16 + f^2 = 2f^2 \text{ or } f=4 
  \end{aligned} 
\end{gathered}
\]
\[
\begin{gathered}
\text{ Indeed for }  \quad \, \begin{aligned}
  \| X - F_1 \| & = \| (4\sqrt{2} c - 4) A + 4sB \| = \sqrt{ 32c^2 - 32\sqrt{2} c + 16 + 16s^2 } = \\
  & = \sqrt{ 16c^2 - 32\sqrt{2} c + 32 } = 4(\sqrt{2} - c) \\
  \| X - F_2 \| & = \| (4\sqrt{2} c + 4 ) A + 4sB \| = 4 (\sqrt{2} +c ) 
  \end{aligned} \\
\Longrightarrow \| X - F_1 \| + \| X - F_2 \| = 8 \sqrt{2} 
\end{gathered}
\]
A Cartesian equation for the plane containing this ellipse would be $\boxed{ -x + z = 0}$.
\item 
\[
\begin{gathered}
  \begin{aligned}
    r & = 4(c,s,c) \\
    r' & = 4(-s,c,-s) \\
    r'' & = 4(-c,-s,-c) = -4 (c,s,c)
  \end{aligned} \quad \quad 
\begin{aligned}
  r'^2 & = 16 ( 1 + s^2) \\
  r''^2 & = 16 ( 1 +c^2 ) \\
  r' \cdot r'' & = -16 (-sc) = 16sc
\end{aligned} \\
\frac{ \| r'' \times r' \| }{ |r'|^3} = \frac{ \sqrt{ 16^2 ( 1 + c^2)(1+s^2) - 16^2 (sc)^2 } }{ |r'|^3 } = \frac{ \sqrt{2}}{ 4 (1+s^2)^{3/2} } \\
\boxed{ \frac{1}{\kappa} = 2\sqrt{2} ( 1 + s^2)^{3/2} }
\end{gathered}
\]
\end{enumerate}

\exercisehead{5} 
\[
\begin{gathered}
  \begin{aligned}
    r & = (e^t, e^{-t}, \sqrt{2} t ) \\
    r' & = (e^t, -e^{-t}, \sqrt{2} ) \\
    r'' & = (e^t, e^{-t}, 0)
\end{aligned} \quad \quad \, 
\begin{aligned}
  r'' \times r' & = (\sqrt{2} e^{-t}, \sqrt{2} e^{t}, -1 -1) = (\sqrt{2} e^{-t}, \sqrt{2} e^{t}, -2 ) \\
  \| r'' \times r' \| & = \sqrt{ 2 e^{-2t} + 2 e^{2t} + 4 } = \sqrt{2} (e^t + e^{-t} ) \\
  r'^2 & = e^{2t} + e^{-2t} + 2 = (e^t + e^{-t})^2 
\end{aligned} \\
\boxed{ \frac{ \| r'' \times r' \| }{ | r'|^3} = \frac{ \sqrt{2}}{ e^t + e^{-t} } }
\end{gathered}
\]

\exercisehead{6}
\begin{enumerate}
\item \[
  \begin{aligned}
    r(t) & = (x(t),y(t) ) \\
    r' & = (x',y') \\
    r'' & = (x'',y'') 
\end{aligned} \quad \, 
  \begin{aligned}
    \frac{ \| r'' \times r' \| }{ |r'|^3 } & = \frac{ \sqrt{ (x''^2 + y''^2 )(x'^2 + y'^2) - (x''x'+y''y')^2 } }{ (x'^2 + y'^2 )^{3/2} } = \\
    & = \frac{ \sqrt{ x''^2 x'^2 + x''^2 y'^2 + y''^2 x'^2 + y''^2 y'^2 - (x''^2 x'^2 + 2x'' x' y'' y' + y''^2 y'^2)  } }{ (x'^2 + y'^2)^{3/2} } = \\
    &= \frac{ \sqrt{ (x''y' - y'' x')^2 } }{ (x'^2 + y'^2)^{3/2} } =\boxed{ \frac{ | x'' y' - y'' x' | }{ (x'^2 + y'^2)^{3/2} } }
  \end{aligned}
\]
\item For $(x,y(x))$, then we could treat the $x$ as the parameter.  Then $x'=1$ and $x''=0$.  We can reuse our derived expression above: 
\[
\kappa = \frac{  | x''y' - y'' x' |}{ (x'^2 + y'^2)^{3/2} } = \boxed{ \frac{ |y''| }{ (1 + y'^2)^{3/2} }   }
\]
\end{enumerate}

\exercisehead{7} A point moves so its velocity and acceleration vectors always have constant lengths.  
\[
\begin{gathered}
  \begin{aligned}
    X & \quad & \\
    X' & \quad & |X'| =v_0 \\ 
    X'' & \quad & |X''| = a_0 
  \end{aligned} \quad \quad \quad \begin{aligned}
    X' & = v_0 T \\
    X'' & = v_0 T' = v_0^2 \kappa N \\
    \| X'' \| & = v_0 \kappa = a_0 
  \end{aligned} \\
\text{ since } \kappa = \frac{\|T'\|}{v_0}, \quad \text{ so then } \boxed{ \kappa = \frac{a_0}{v_0^2} }
\end{gathered}
\]

\exercisehead{8} If 2 plane curves with $y=f(x)$, $y=g(x)$ have the same tangent at a point $(a,b)$ and the same curvature, prove $|f''(a)| = |g''(a)|$.  
\[
\begin{gathered}
  \begin{aligned}
    X_1 & = (x,f(x)) \\
    X_1' & = (1,f'(x)) \\
    X_1'' & = (0,f''(x))
  \end{aligned} \quad \quad 
  \begin{aligned}
    X_2 & = (x,g(x)) \\
    X_2' & = (1,g'(x)) \\
    X_2'' & = (0,g''(x))
  \end{aligned} \\
  \begin{aligned}
    \frac{ \| X_1'' \times X_1' \| }{ (1+f'^2)^{3/2} } & = \frac{ \sqrt{ |f''|^2 (1+f'^2) - (f'' f')^2} }{ (1+f'^2)^{3/2} } = \\
    & = \frac{ |f''|}{ (1+f'^2)^{3/2} } 
  \end{aligned}, \quad \text{ likewise } \frac{ \| X_2'' \times X_2' \| }{ (1+g'^2)^{3/2} } = \frac{ |g''| }{ (1+g'^2)^{3/2} } 
\end{gathered}
\]
For $(a,b)$, the 2 plane curves have the same tangent, so then $f'(a) = g'(a)$.  So if the curvature is the same at $(a,b)$, then
\[
\frac{ |f''(a)| }{ (1+f'^2(a))^{3/2}} = \frac{ |g''(a)|}{ (1+g'^2(a))^{3/2} } \Longrightarrow \boxed{ |g''(a)| = |f''(a)| }
\]

\exercisehead{9} 
\begin{enumerate}
\item This problem could be done without reference to the same curvature or the same tangent line.  We only need to use the fact that they intersect at the same point.  
\[
\begin{gathered}
  \text{ Given } f = ax(b-x), \quad (x+2)g = x, \\
  \text{ If } x = b, 
  f = 0 \quad \quad \, \begin{gathered}
    (b+2)g = b \\
    \text{ If } b \neq 2 \\
    g= \frac{b}{b+2} 
\end{gathered} \\
  \quad \\
  \text{ Same as } x=0, \text{ so that } f = 0, \, g= 0
\end{gathered}
\]
So one point of intersection for any $a,b$ is $(0,0)$.  Otherwise, 

\[
\begin{gathered}
  ax(b-x) = \frac{x}{x+2} \quad \, x \neq = -2 (\text{ this is okay, plug $x=-2$ into $(x+2)g=x$ and you'll get a contradiction }) \\
  \Longrightarrow x = \frac{ -(2-b)}{2} \pm \frac{ \sqrt{ 4 + 4b + b^2 - 4/a}}{2} 
\end{gathered}
\]
So we must have, for only one pt. of intersection, $(0,0)$, $\boxed{ b=2 }$ and 
\[
\sqrt{ 4 + 4b+ b^2 - 4/a } = \sqrt{ 4 + 8 + 4 - 4/a} = 0 \quad \, \Longrightarrow \boxed{ a =1/4 }
\] 
\item See sketch.  The parabola and $(x+2)g=x$ coincide at $(0,0)$, have the same tangent line, and have the same direction at this intersection point.  
\end{enumerate}

\exercisehead{10} Remember that the radius of curvature is the reciprocal of the curvature.  
\begin{enumerate}
  \item Without loss of generality, $y= 4cx^2$.  Then \\
    $\begin{aligned}
    y' & = 8cx \\
    y'' & = 8c
\end{aligned}$ \quad $\Longrightarrow \kappa = \frac{ |f''| }{  (1+f'^2)^{3/2} }$.  Since $1>0$, \quad $64 + c^2 x^2 > 0 $, and $|8c|$ constant, for $\kappa$ maximized, then $\boxed{ x=0}$.  
  \item Given 2 fixed unit vectors $A,B$, \, $A\cdot B = \cos{\theta}$, 
\[
\begin{gathered}
  \begin{aligned}
    r & = t A + t^2 B \\
    r' & = A + 2tB \\
    r'' & = 2B 
  \end{aligned} \quad \quad \quad 
  \begin{aligned}
    & r'' \times r' = 2B \times (A + 2tB) = 2B\times A \\
    & |r'' \times r'| = 2\sin{\theta} \\
    & r'^2 = 1 + 4t \cos{\theta} + 4t^2 
  \end{aligned} \\ 
  \begin{aligned}
  (r'^2)' = 8t + 4 \cos{\theta} & = 0 
    t & = -\cos{\theta}/2 
  \end{aligned} \quad \quad \quad 
  (r'^2)'' = 8 > 0 
\end{gathered}
\]
So $t = -\cos{\theta}/2$ minimizes $r'^2$ and so maximizes $\kappa$.  Then $\boxed{ r = -\frac{\cos{\theta}}{2} A + \frac{\cos^2{\theta}}{4 } B }$
\end{enumerate}

\exercisehead{11}
\begin{enumerate}
\item We have for this plane curve, \\
  $\begin{aligned}
  r & = (x,y) \\
  r' & = (x',y') \\
  r'' & = (x'',y'')
\end{aligned}$ \quad \, Given $\sqrt{ x'^2 + y'^2 } = 5$, and given $\begin{aligned}
  (x(0),y(0)) & = (0,0) \\
  (x'(0),y'(0)) & = (0,5) 
\end{aligned}$ \medskip \\
  \[
\begin{gathered}
  a\times v = e_z (x'' y' -x'y'') \\
  \kappa = \frac{ |a\times v| }{v^3} = \frac{ |x'' y' - x'y''|}{ 125}  =2t 
\end{gathered}
\]
We want $(x',y')\cdot e_x = 5\cos{\theta} = x'$.  Then $y' = 5 \sin{\theta}$.  Also, note that $x'(0) = 0 \Longrightarrow \theta_0 \pi/2$ 

\[
\begin{gathered}
  \Longrightarrow 250 t = | - 5 \sin{\theta} \theta'(5 \sin{\theta} ) - 5 \cos{\theta} 5 \cos{\theta} \theta' | \\
  10t = |\theta'| \quad \, \boxed{ \theta = 5 t^2 + \pi/2} 
\end{gathered}
\]
\item $v= 5 (\cos{ (5t^2 + \pi/2) }, \sin{ (5t^2 + \pi/2) } )$
\end{enumerate}

\exercisehead{12} 
\[
\begin{gathered}
\text{ Given that }  \sqrt{ x'^2 + y'^2 } = 2, \quad \text{ then } 
  \begin{aligned}
    x' & = 2\cos{\theta} \\
    y' & = 2 \sin{\theta} 
  \end{aligned} \quad \quad \theta_0 = 0 \text{ since } v_0 = 2i \\
  \kappa(t) = \frac{ | x'' y' - x' y'' | }{ v^3} = \frac{ | -2\sin{\theta} \theta' 2 \sin{\theta} - 2 \cos{\theta} 2\cos{\theta} \theta'| }{ 2^3 } = \frac{ |\theta'|}{2} = 4t \quad \quad 
  \begin{aligned}
    \theta' & = 8t \\
    \theta & = 4t^2 
  \end{aligned} \\
  \boxed{ v = \sqrt{2} (1,1) }
\end{gathered}
\]

%-----------------------------------%-----------------------------------%-----------------------------------
\subsection*{ 14.19 Exercises - Velocity and acceleration in polar coordinates, Plane motion with radial acceleration, Cylindrical coordinates }
%-----------------------------------%-----------------------------------%-----------------------------------
\quad \\

\exercisehead{1}

\[
\begin{aligned}
  \mathbf{v} & = \frac{ d\mathbf{r}}{dt} = \frac{dr}{dt} \vec{e}_r + r \frac{d\theta}{dt} \vec{e}_{\theta} = \vec{e}_r + r \vec{e}_{\theta} \quad ( \theta = t )  \\
  \mathbf{a} & = \left( \frac{d^2 r}{dt^2} - r \left( \frac{ d\theta}{dt} \right)^2 \right) \vec{e}_r + \frac{1}{r} \frac{d}{dt} \left( r^2 \frac{d\theta}{dt} \right) \vec{e}_{\theta} = -r \vec{e}_r + 2 \vec{e}_{\theta} \\ 
  \mathbf{v} &=  \vec{e}_r + t \vec{e}_{\theta} = \cos{t} \vec{e}_x + \sin{t} \vec{e}_y + -t \sin{t} \vec{e}_x + t \cos{t} \vec{e}_y = \\
  &= (\cos{t} - t\sin{t}) \vec{e}_x + (\sin{t} + t\cos{t} )\vec{e}_y \\
  \mathbf{a} &= (-t \cos{t} - 2 \sin{t} )\vec{e}_x + (-t \sin{t} + 2 \cos{t}) \vec{e}_y 
\end{aligned}
\]

\exercisehead{2}
\[
\begin{aligned}
  \mathbf{v} & = \vec{e}_r + r \vec{e}_{\theta} + \vec{e}_z = (\cos{t} - t \sin{t} ) \vec{e}_x + (\sin{t} + t\cos{t} ) \vec{e}_y + \vec{e}_z \\
  \mathbf{a} & = (-t \cos{t} - 2 \sin{t} )\vec{e}_x + (t\sin{t} + 2 \cos{t} ) \vec{e}_y 
\end{aligned}
\]

\exercisehead{3}
(a). \\
\[
\begin{gathered}
  r = \sin{t}; \theta =t ; z = \log{\sec{t}}; \theta \leq t < \frac{\pi}{2} \\
  (r \cos{\theta})^2 + (r \sin{\theta} - \frac{1}{2} )^2 = r^2 - r \sin{\theta} + \frac{1}{4}  = \frac{1}{4} 
\end{gathered}
\]

(b).\\
\[
\begin{gathered}
\mathbf{v} = \frac{dt}{dt} \vec{e}_r + r \frac{d\theta}{dt} \vec{e}_{\theta} + \log{\sec{t}} \vec{e}_z = \cos{t} \vec{e}_r + r\vec{e}_{\theta} + \frac{\tan{\theta} \sec{\theta}}{\sec{\theta} }\vec{e}_z \\
v_z = \tan{\theta}; \, v^2 = \cos^2{t} + r^2 + \tan^2{\theta} = \sec^2{\theta} \\
\cos{\phi} = \frac{ \tan{\theta}}{\sec{\theta}} = \sin{\theta} = r = \sin{t} \\
\boxed{ \phi = \arccos{(\sin{\theta}) } }
\end{gathered}
\]

\exercisehead{4} Given $r = f(\theta)$; \quad $a\leq \theta \leq b \leq a + 2\pi$
\[
\begin{gathered}
  \begin{aligned}
    X & = r u_r \\
    X' & = r' u_r + r u_{\theta} \theta' 
\end{aligned} \quad \quad \quad \begin{aligned}
    X'^2 & = r'^2 + r^2 \theta'^2 \\
    |X'| & = \sqrt{ r^2 + r'^2 }
  \end{aligned} \\
  s= \int \sqrt{ r^2 + r'^2 \theta'^2} dt \xrightarrow{ t= \theta } s = \int_a^b \sqrt{ r^2 + \left( \frac{ dr}{d\theta} \right)^2 } d\theta
\end{gathered}
\]

\exercisehead{5} Given \\
$\begin{aligned}
  r & = a(1+\cos{\theta} ) \\
  r' & = -a\sin{\theta }
\end{aligned}$ \quad \, \text{ so then } $r^2 + r'^2 = a^2 ( 1 + 2 \cos{\theta} + \cos^2{\theta} + \sin^2{\theta} ) = 4 a^2 \cos^2{\theta/2}$ 
\[
\begin{aligned}
  s & = \int \sqrt{ r^2 + r'^2 } d\theta = 2a \int_0^{2\pi }\cos{\theta/2} d\theta = 2a \left( \int_0^{\pi} \cos{\theta/2} d\theta + - \int_{\pi}^{2\pi} \cos{\theta/2} d\theta \right) = \\
  & = 4a \left( \left. \sin{\theta/2} \right|_0^{\pi} - \left. \sin{\theta/2} \right|_{\pi}^{2\pi } \right) = 4a (1 + 1 ) = \boxed{ 8 a}
\end{aligned}
\]

\exercisehead{6}
\[
A = \int R^2(\theta)d\theta = \int_0^{2 \pi} \frac{1}{2} e^{2c \theta} d\theta = \left. \frac{ e^{2c \theta}}{ 4 c} \right|_0^{2 \pi} = \frac{ e^{4 \pi c} - 1 }{4c}
\]

\exercisehead{7}
\[
\begin{aligned}
  &  \int_0^{\pi} \frac{1}{2} \sin^4{\theta} d\theta = \frac{1}{2} \int_0^{\pi} \sin^2{\theta}(1-\cos^2{\theta}) d\theta = \frac{1}{2} \int_0^{\pi} \left( \left( \frac{1 - \cos{2 \theta}}{2} \right) - \left( \frac{ 1 - \cos{ 4 \theta} }{2 (4) } \right) \right) d \theta 
  & = 3 \pi / 16
\end{aligned}
\]
\quad \\

See sketches for exercises 8-11.  
\exercisehead{8} Given $r=\theta$; \quad $0 \leq \theta \leq \pi$ \\
\[
\begin{aligned}
  s & = \int_0^{\pi} \sqrt{ \theta^2 + 1 } d\theta = \frac{1}{2} \left. \left( \theta \sqrt{ \theta^2 + 1 } + \ln{ ( \theta + \sqrt{ \theta^2 + 1 } ) } \right) \right|_0^{\pi} = \\
  & = \boxed{ \frac{1}{2} (\pi \sqrt{ \pi^2 + 1 } + \ln{ ( \pi + \sqrt{ \pi^2 + 1 } )} ) }
\end{aligned}
\]

\exercisehead{9} Given $r = e^{\theta}$; \quad $0\leq \theta \leq \pi$
\[
\begin{aligned}
  \frac{dr}{d\theta} & = e^{\theta} \\
  s & = \int_0^{\pi} \sqrt{ e^{2\theta} + e^{2\theta} } d\theta = \sqrt{2} \int_0^{\pi} e^{\theta} d\theta = \boxed{ \sqrt{2} (e^{\pi} - 1 ) }
\end{aligned}
\]

\exercisehead{10} $r= 1 + \cos{\theta}$ ; \quad \, $\frac{dr}{d\theta} = - \sin{\theta}$; \quad $0\leq \theta \leq \pi$
\[
\int_0^{\pi} \sqrt{ r^2 + \left( \frac{dr}{d\theta} \right)^2 } = \int_0^{\pi} \sqrt{2} \sqrt{ 1 + \cos{\theta} } = \sqrt{2} \int_0^{\pi} \sqrt{2} \cos{\theta/2} = 4 \left. (\sin{\theta/2})\right|_0^{\pi} = \boxed{ 4}
\]
 
\exercisehead{11} Given $r= 1-\cos{\theta}$; \quad \, $0 \leq \theta \leq 2\pi$, then $\frac{dr}{d\theta} = \sin{\theta}$
\[
\int_0^{2\pi} \sqrt{ r^2 + \left( \frac{dr}{d\theta} \right)^2 } = \int_0^{2\pi} \sqrt{2} \sqrt{ 1 - \cos{\theta}} = \sqrt{2} \int_0^{2\pi} \sqrt{2} |\sin{\theta/2} | = 4 \left. -\cos{\theta/2} \right|_0^{2\pi} = \boxed{ 8 }
\]

\exercisehead{12} Given $r = f(\theta)$, $\rho = \text{ radius of curvature } = \frac{1}{\kappa}$, \, $r' = f'(\theta)$, \, $r''= f''(\theta)$
\[
\begin{gathered}
  \begin{aligned}
    X & = ru_r \\
    X' & = r' u_r + ru_{\theta} \theta' \\
    X'' & = r''u_r + 2r' u_{\theta} \theta' + -r\theta' u_r + r\theta'' u_{\theta} = \\
    & = (r'' -r\theta')u_r + \frac{1}{r} (r^2 \theta')' u_{theta} 
  \end{aligned} \quad \quad \quad 
  \begin{aligned}
    u_r & = (\cos{\theta},\sin{\theta}) \\
    u_{\theta} & = (-\sin{\theta}, \cos{\theta} ) \\
    u_r \times u_{\theta} & = e_z 
  \end{aligned} \\
  X'' \times X' = ((r'' - r\theta')u_r + \frac{1}{r} (r^2 \theta')' u_{\theta} ) \times (r' u_r + ru_{\theta} \theta' ) = \frac{1}{r} (r^2 \theta')' r'(-e_z) + (r'' - r\theta')r \theta' e_z \\
  \| X'' \times X' \| = r'' r\theta' - r^2 \theta'^2 - 2r'^2 \theta' - rr' \theta'' \\
  |X'|^2 = r'^2 + r^2 \theta'^2 
\end{gathered}
\]
If we let $\theta =t$, since the curve is described completely by $r = f(\theta)$, then 
\[
\kappa = \frac{ |r^2 - r'' r+ 2r'^2 | }{ (r'^2 + r^2 )^{3/2} } \Longrightarrow \rho = \frac{ (r'^2 + r^2 )^{3/2} }{ |r^2 - r'' r + 2r'^2 | }
\]

\exercisehead{13} Using $\rho = \frac{ (r^2 +r'^2)^{3/2}}{ |r^2 - rr'' + 2r'^2| } $
\begin{enumerate}
  \item \[
\begin{aligned}
  r & = \theta \\
  r' & = 1 \\
  r'' & = 0 
\end{aligned} \quad \quad \quad 
\boxed{ \rho = \frac{ (\theta^2 + 1)^{3/2} }{ \theta^2 + 2 }  }
\]
  \item 
\[
\begin{aligned}
  r & = e^{\theta} \\
  r' & = e^{\theta} \\
  r'' & = e^{\theta} 
\end{aligned} \quad \quad \quad 
\rho = \frac{ (e^{2\theta} + e^{2\theta} )^{3/2} }{ 2e^{2\theta} } = \boxed{ \sqrt{2} e^{\theta} }
\]
  \item $\theta = \pi/4$
\[
\begin{gathered}
\begin{aligned}
  r & = 1 + \cos{\theta} \\
  r' & = -\sin{\theta} \\
  r'' & = -\cos{\theta}
\end{aligned}  \quad \quad \quad 
 \rho = \frac{ (1+2c +c^2 + s^2)^{3/2} }{ |1 + 2c +c^2 +(c +c^2) + 2s^2 | } = \frac{ (2(1+c))^{3/2} }{3 (1+c) } = \boxed{ \frac{ 2 \sqrt{ 2 + \sqrt{2} }}{3} }
\end{gathered}
\]
  \item $\theta = \pi/2$ 
\[
\begin{gathered}
\begin{aligned}
  r & = 1 - \cos{\theta} \\
  r' & = \sin{\theta} \\
  r'' & = \cos{\theta} 
\end{aligned}  \quad \quad \quad 
\rho = \frac{ (2(1-c))^{3/2} }{ (3+2c)(1-c) } = \frac{ 2\sqrt{2} (1-c)^{1/2} }{ 3+2c } \xrightarrow{ \pi/2} \boxed{ \frac{ 2\sqrt{2}}{3} }
\end{gathered}
\]
\end{enumerate}

\exercisehead{14}
\[
\begin{gathered}
  \begin{aligned}
    X & = ru_r \\
    X' & = r'u_r + ru_{\theta} \theta' \\
    X'\cdot X' & = r'^2 + r^2 \theta'^2 = v^2 
  \end{aligned} \quad \quad \, \Longrightarrow X\cdot X' = rr' = rv\cos{\phi} \text{ or } \boxed{ r' = v\cos{\phi} } \bigskip \\
  v^2 = v^2 \cos^2{\phi} + r^2 \theta'^2
\end{gathered}
\]
Use $\theta$ as the parameter for $t$, since the curve is invariant.  $\Longrightarrow \boxed{ r = v\sin{\phi} }$

\exercisehead{15}
Place target at the center (without loss of generality).  The strategy is to break up $\mathbf{v}$ into the polar coordinate unit vectors.  
\[
\begin{aligned}
  \mathbf{r} & = r \vec{e}_r \\
  \mathbf{v} & = \frac{ dr}{ dt} \vec{e}_r + r \frac{ d\theta}{dt} \vec{e}_{\theta} \\
  & \begin{aligned}
    \frac{dr}{dt} & = v_r = v \cos{ (\pi - \alpha)} = -v \cos{ \alpha} \\
    r \frac{d\theta}{dt} & = v \sin{\alpha} 
    \end{aligned}
\end{aligned}
\]
\[
\begin{gathered}
  \frac{ v\sin{\alpha}}{ - v \cos{\alpha}} = - \tan{\alpha} = \frac{ r \frac{d\theta}{dt} }{ \frac{dr}{dt} } = r \frac{d\theta}{dr} ; \, \frac{1}{r} \frac{ dr}{d\theta} = -\tan{\alpha} \\
  r = e^{ -\tan{\alpha} \theta }
\end{gathered}
\]

\exercisehead{16} Admittedly, I had {\large \emph{ looked }} in the {\large back of the book}.  \\

The {\LARGE big trick} is that {\Large \textbf{ the choice of origin is important!}}.  The clever choice of origin will cut out the algebra in dealing with Cartesian coordinates.  The clever choice of origin will illuminate the problem entirely.  So try flipping around, reflecting, and turning around the problem until the right choice of origin is selected.  \\

Let the positive axis be from where the missle, the target missile, $m$, is sighted to the ground crew.  \textbf{So the origin is at the target missile, not the ground crew} from which the anti-missile missile, $a$, is launched!  It'll make the algebra much easier.  

Let $a$ go $3$ miles in the \textbf{negative} $x$ direction, to allow for the possibility that the missile is returning to the ground crew, that is $4$ mi. in the positive $x$ direction from the origin, at the time of sighting.  Let $t=0$ be the time that missile $a$ finishes going these $3$ miles and is $1$ mile in the positive $x$ direction from the origin.  

Let
\[
  \begin{aligned}
    & (x_m,y_m) = 1(t+1)(\cos{\theta_{m0}},\sin{\theta_{m0}}) = (t+1)u_{m_0} = X_m \\
    & (x_m',y_m')=u_{m_0} 
  \end{aligned}
  \quad \quad \quad 
\begin{aligned}
  & (x_a,y_a) = X_a = r_a u_a \\
  & (x_a',y_a') = X_a' = r_a'u_a + r_a \theta' u_{\theta} \\
  & |X_a'| = \sqrt{ r_a'^2 + (r_a \theta')^2 } = 3 \\
  & X_a(t=0) = X_{a0} = (1,0) = r_a(0) (\cos{\theta(0)},\sin{\theta(0)} ) \\
  & \theta(0) =0, \quad r_a(0) = 1
\end{aligned} 
\]
At some point $t_2$, $(x_m,y_m) = (x_a,y_a)$, i.e. $X_m(t_2) = X_a(t_2)$.  Note that \textbf{ we only need one } $t_2$; $\theta_{m_0}$ constant, but unknown.  

\[
\begin{gathered}
  \begin{aligned}
    (t_2 + 1) \cos{\theta_{m_0}} & = r_a \cos{\theta} \\
    (t_2 + 1) \sin{\theta_{m_0}} & = r_a \sin{\theta} =
  \end{aligned} \\
  (t_2 + 1)^2 = r_a^2; \quad (t_2+1) = r_a(t_2); \quad r_a'(t_2) = ? \quad \, \text{ well } \\
  (r_a'(t_2))^2 + ((t_2+1)\frac{d\theta}{dt}(t_2) )^2  = 9 
\end{gathered}
\]

Try $1 = r_a'(t_2)$.  
\[
\begin{gathered}
  1 + (t_2 + 1)^2 \theta'^2 = 9 \\
  (t_2+1)\left( \frac{d\theta}{dt} \right) = 2\sqrt{2} \Longrightarrow d\theta = \frac{2\sqrt{2}}{ t_2 + 1 } dt \\
  \Longrightarrow \boxed{ \theta = 2\sqrt{2} \ln{ (t_2 + 1 ) } } \quad \quad t_2 = e\left( \frac{ \theta}{2 \sqrt{2} } \right) -1 \\
  \boxed{ r = e\left( \frac{\theta}{2\sqrt{2}} \right) }
\end{gathered}
\]

\exercisehead{17}

A first order differential equation of the form $y' = f(x,y)$ is homogeneous if $f(tx,ty)=f(x,y)$.  Then 
\[
f(r \cos{\theta}, r\sin{\theta}) = f(\cos{\theta}, \sin{\theta}) = f(\theta) 
\]
We find that
\[
\begin{aligned}
  \frac{dy}{d\theta} & = \frac{dr}{d\theta} \sin{\theta} + r \cos{\theta} \\
  \frac{dx}{d\theta} & = \frac{dr}{d\theta} \cos{\theta} - r \sin{\theta}
\end{aligned}
\]
Thus
\[
\frac{dy}{dx} = \frac{ \frac{dr}{d\theta} \sin{\theta} + r \cos{\theta} }{ \frac{dr}{d\theta} \cos{\theta} - r \sin{\theta} } = f(\theta)
\]

\exercisehead{18}
\[
\begin{gathered}
  \mathbf{v} = \omega \mathbf{k} \times \mathbf{r} \\
  \mathbf{v} = \frac{dr}{dt} \vec{e}_r + r \frac{ d\theta}{dt} \vec{e}_{\theta} \\
  \mathbf{v} \cdot \vec{e}_r = 0, \text{ so } \frac{ dt}{dt} = 0 ; \, \omega \mathbf{k} \times \mathbf{r} = r \frac{ d\theta}{dt} \vec{e}_{\theta} = \omega r \vec{e}_{\theta} = r \frac{d\theta}{dt} \\
  \left| \omega \mathbf{k} \times \mathbf{r}  \right|^2 = \omega^2 r^2 = r^2 \left( \frac{ d\theta}{dt} \right)^2 \\
  \boxed{ \omega = \left| \frac{ d \theta}{dt} \right|, \, \omega > 0 }
\end{gathered}
\]

\exercisehead{19} (a)
\[
\begin{aligned}
  \mathbf{v} = \frac{dr}{dt} \vec{e}_r + r \frac{d\theta}{dt} \vec{e}_{\theta} = r\frac{d\theta}{dt} \vec{e}_{\theta} ; \, \frac{dr}{dt} = 0; \, \vec{e}_{\theta} = \vec{e}_z \times \vec{e}_r \\
  \mathbf{v} = r \frac{d\theta}{dt} \vec{e}_z \times \vec{e}_r = \frac{d\theta}{dt} \vec{e}_z \times r \vec{e}_r = \mathbf{\omega} \times \mathbf{r} 
\end{aligned}
\]
(b). 
\[
\begin{gathered}
  a = v' = \omega' \times r + \omega \times r' = \alpha \times r + \omega \times r' \\
  \omega \times r' = \omega \times ( \omega \times r ) = (\omega \cdot r ) \omega - \omega^2 r 
\end{gathered}
\]
(c). 
\[
\text{ Now } \omega \cdot r = 0  \Longrightarrow a = -\omega^2 r
\]

\exercisehead{20} 

The distance $| \mathbf{r}_p(t) - \mathbf{r}_q(t) |$ is independent of $t$, so $\frac{d}{dt} | \mathbf{r}_p(t) - \mathbf{r}_q(t) |  = 0 $, which implies
\[
\begin{gathered}
  \frac{d}{dt} (r_p^2 - 2 r_p \cdot r_q + r_q^2 ) = 2 r_p \cdot \frac{ dr_p}{dt} - 2 \frac{dr_p}{dt} \cdot r_q  - 2 r_p \cdot \frac{ dr_q}{dt} + 2 \frac{dr_q}{dt} \cdot r_q = 0 \\
  \frac{dr_p}{dt} \cdot ( r_q - r_p) = - \frac{dr_q}{dt} \cdot (r_p - r_q)
\end{gathered}
\]
Suppose
\[
\begin{aligned}
  v_p & = \omega_p \times r_p \\
  v_q & = \omega_q \times r_q
  \end{aligned}
\]
Then
\[
\begin{gathered}
\begin{aligned}
  v_p \cdot (r_q - r_p ) & = - v_q\cdot (r_p-r_q) \\
  \omega_p \times r_p \cdot r_q & = - \omega_q \times r_q \cdot r_p = 
  & = \omega_q \times r_p \cdot r_q 
\end{aligned}\\
( (\omega_p - \omega_q) \times r_p ) \cdot r_q = 0 
\end{gathered}
\]
Thus, $\mathbf{\omega}_p = \mathbf{\omega}_q$.  

\subsection*{ 14.21 Miscellaneous review Exercises - Applications to planetary motion }
\quad \\
\exercisehead{1} \emph{ Use polar coordinates if Cartesian coordinates doesn't help! } 
\[
\begin{gathered}
  \begin{aligned}
    y^2 & = x \\
    X & = (r\cos{\theta},r\sin{\theta}) = ru_r \\
    X' & = r'u_r + r\theta' u_{\theta} \\
    |X'| & = \sqrt{ r'^2 + (r\theta')^2 } \\
    T & =\frac{X'}{|X'|} = \frac{ r'u_r + r\theta' u_{\theta}}{ \sqrt{ r'^2 + (r\theta')^2 } }
  \end{aligned} \quad \quad  \quad 
\begin{aligned}
  x & = r\cos{\theta} \\
  y & = r\sin{\theta} 
\end{aligned} \quad \Longrightarrow \begin{gathered}
  r^2 \sin^2{\theta} = r \cos{\theta} \\
  r = \frac{\cos{\theta}}{ \sin^2{\theta}} \\
  \quad \\
  r' s^2 + 2r^2 sc \theta' = -s \theta' \\
  r' = - \left( \frac{ 1 + c^2 }{ s^3 } \right) \theta'
\end{gathered} 
\end{gathered}
\]
\[
\begin{gathered}
  r\cdot T = \frac{ rr' }{ \sqrt{ r'^2 + (r\theta')^2 } } = r\cos{\alpha } \quad \, \Longrightarrow \frac{ r' }{ \sqrt{ r'^2 + (r\theta')^2 } } = \cos{\alpha} \bigskip \\
  X \cdot e_r = r\cos{\theta}  \\
  \begin{gathered}
    r'^2 + (r\theta')^2 = \left( \frac{ 1 + 2c^2 + c^4}{ s^6 } \right) \theta'^2 + \frac{ c^2}{s^4} \theta'^2 = \frac{1 + 3c^2}{s^4} \\
    \cos{\alpha} = \frac{ (1+c^2)/s^3}{ ((1+3c^2)/s^6)^{1/2} } = \frac{ 1 +c^2 }{ \sqrt{ 1 + 3c^2 } } \medskip \\
    \sin^2{\alpha} = 1 - \cos^2{\alpha} = \frac{ 1 + 3c^2 - (1+2c^2 + c^4) }{ (1+3c^2) } = \frac{ c^2 s^2 }{ (1+3c^2) } \medskip 
  \end{gathered} \\
  \frac{ \sin{\alpha}}{\cos{\alpha}} = \tan{\alpha} = \frac{ cs/\sqrt{ 1 + 3c^2} }{ (1+c^2)/ \sqrt{ 1 + 3c^2 } } = \boxed{ \frac{ \tan{\theta}}{ 1+\sec^2{\theta} } }
\end{gathered}
\]

\exercisehead{2} 
\[
\begin{gathered}
  \begin{aligned}
    X & = \left( \frac{y^2}{4c} , y \right) \\
    X' & = y' \left( \frac{y}{2c} , 1 \right) 
  \end{aligned} \quad \quad \quad  
  \begin{gathered}
  \frac{X'}{|X'|} = \frac{ \left( \frac{y}{2c} , 1 \right) }{ \sqrt{ 1 + (y/2c)^2 } } = \frac{1}{ \sqrt{ 1 + (y/2c)^2 } } T \\
  N = (2c,-y)  \Longrightarrow T\cdot N = 0 
\end{gathered}
\end{gathered}
\]

\exercisehead{3}
\[
\begin{gathered}
  \begin{aligned}
    (x,y) & = \left( \frac{y^2}{4c} , y \right) \\
    X' & = \left( \frac{y}{2c}, 1 \right)y' \\
    |X'| & = y' \sqrt{ (1 + \left( \frac{y}{2c} \right)^2 ) }
\end{aligned}
\quad \quad \, \frac{ X'}{ |X'|} = \frac{ \left( \frac{y}{2c} , 1 \right) }{ \sqrt{ 1 + (y/2c)^2 } }; \quad \, m = 1/(y/2c) = \frac{2c}{y} \bigskip \\
m = m(x_0) = \frac{2c}{y_0} 
\end{gathered}
\]
\[
\begin{gathered}
  y = mx + b  \quad  \,
  \Longrightarrow y_0 = \frac{2c}{y_0} \frac{y_0^2}{ 4 c} + b = \frac{y_0}{2} + b  \text{ so that } b= \frac{y_0}{2} = c/m \\
  \Longrightarrow \boxed{ y = mx + c/m } \quad \quad \quad 
\boxed{  \begin{aligned}
    y_0 & = \frac{2c}{m } \\
    x_0 & = \frac{4c^2}{m^2} \frac{1}{4c} = \frac{c}{m^2}
\end{aligned} }
\end{gathered}
\]

\exercisehead{4}
\begin{enumerate}
\item 
\[
\begin{gathered}
  \begin{gathered}
    (y-y_0)^2 = 4c(x-x_0) \\
    \frac{ (y-y_0)^2}{4c} + x_0 = x 
  \end{gathered} \quad \quad \quad \Longrightarrow \begin{gathered}
    X = \left( \frac{ (y-y_0)^2}{4c} + x_0, y \right)  \\
    T = \frac{X'}{|X'|} = \frac{ y' \left( \frac{(y-y_0) }{2c} ,1 \right) }{ |y'| \sqrt{ 1 + \left( \frac{(y-y_0)}{2c} \right)^2 } } \\
    m = \frac{2c}{y-y_0}
\end{gathered} 
\end{gathered}
\]
\[
\begin{gathered}
  \begin{gathered}
    m(y_1) = \frac{2c}{ y_1 - y_0 } \\
    y = m(y_1) (x-x_0) + b+ y_0 
  \end{gathered} \quad \quad \, \Longrightarrow 
  \begin{gathered}
    y_1 = \left( \frac{2c}{ y_1 - y_0 } \right)\left( \frac{(y_1 -y_0)^2}{4c} \right) + b + y_0 \\
    y_1 - y_0 = \frac{y_1 -y_0 }{2} + b 
  \end{gathered}  \medskip \\
  (y-y_0) = m(x-x_0) + \frac{c}{m}
\end{gathered}
\] 
\item 
\[
\begin{gathered}
  (x-x_0)^2 = 4c(y-y_0) \\
  \begin{aligned}
    X & = (x,\frac{ (x-x_0)^2}{4c} + y_0) \\
    T & = \frac{ \left( 1 , \frac{x-x_0}{2c} \right) }{ \sqrt{ 1 + \frac{ (x-x_0)^2}{4c^2} } } \\
    m & = \frac{x-x_0}{2c} = m(x) 
  \end{aligned} 
  \quad \quad \quad 
  \begin{aligned}
    y - y_0 & = m(x-x_0) + b \\
    y_1 - y_0 & = \frac{ (x_1- x_0)^2}{4c} = \left( \frac{(x_1-x_0)^2}{2c} \right) + b 
  \end{aligned}  
\end{gathered}
\]
\[
\begin{gathered}
\begin{gathered}
  b = \frac{ -(x_1-x_0)^2}{4c} \\
  \Longrightarrow y - y_0 = m(x-x_0) + -m^2 c
\end{gathered}\\
\quad \medskip \\ 
\begin{gathered}
  x^2 = 4cy \\
  \begin{aligned}
    X & = (x,x^2/4c) \\
    T & = \frac{ \left( 1 , \frac{x}{2c} \right) x' }{ |x'| \sqrt{ 1 + (x/2c)^2 } } \\
    m & = x/2c 
  \end{aligned} \quad \quad \quad \, 
  \begin{aligned}
    y & = mx + b \\
    y_1 & = \frac{x_1^2 }{2c} + b = \frac{x_1^2}{4c} = \frac{x_1^2}{2c} + b 
  \end{aligned} \\
  b= \frac{-x_1^2}{4c} \\
  \Longrightarrow y = mx + c m^2
\end{gathered}
\end{gathered}
\]
\end{enumerate}

\exercisehead{5} Given $\begin{aligned}
  X & = \left( \frac{y^2}{4c}, y \right) \\
  X' & = y' \left( \frac{y}{2c}, 1 \right) 
\end{aligned}$ \quad \, $\Longrightarrow m = \frac{2c}{y}$ 
\[
\begin{gathered}
y_1 = \frac{2c}{y_1}x_1 + b = \frac{2c}{y_1} \frac{y_1^2}{4c} = \frac{y_1}{2} + b \quad \, \Longrightarrow b = \frac{y_1}{2} \\
y = \frac{2c}{y_1}x + \frac{y_1}{2}; \quad \, \Longrightarrow y_1 y = 2cx + 2x_1 c = 2c(x+x_1) 
\end{gathered}
\]

\exercisehead{6} $(y-y_0)^2 = 4c(x-x_0)$ \quad \, $\Longrightarrow m(y) = \frac{2c}{y-y_0}$.  
\[
\begin{gathered}
  y- y_0 = \left( \frac{2c}{y_1-y_0 }\right)(x-x_0) + b \\
  b = (y_1 - y_0) - \frac{2c (y_1 -y_0) }{4c} = \frac{c}{ m} \\
  y-y_0 = \left( \frac{2c}{y_1 - y_0} \right)(x-x_0) + \frac{ (y_1 - y_0) }{2}  \\
  (y_1-y_0)(y-y_0) = 2c(x-x_0) + \frac{ (y_1-y_0)^2}{2} = \boxed{ 2c ((x-x_0) + (x_1-x_0)) }
\end{gathered}
\]

For $X = (x, \frac{x^2}{4c} )$ \quad \, $\Longrightarrow m(x) = \frac{x}{2c}$
\[
\begin{gathered}
  y = \frac{x_1}{2c} x + b \quad \, \Longrightarrow b = \frac{x_1^2}{-4c} = -cm^2 \\
  y = \frac{x_1 x}{2c} - y_1 \quad \, \Longrightarrow \boxed{ 2c (y+y_1) = x_1 x }
\end{gathered}
\]

For $X = \left( x, \frac{ (x-x_0)^2}{4c} + y_0 \right)$ \quad \, $\Longrightarrow m = \frac{x_1 - x_0}{2c}$ \\
\[
\begin{gathered}
  \begin{gathered}
    y- y_0 = \left( \frac{x_1 - x_0}{2c} \right)(x-x_0) + b \\
    \frac{ (x_1 - x_0)^2 }{4c} - \frac{ (x_1 - x_0)^2}{2c} = \frac{ - (x_1-x_0)^2}{4c} = b 
  \end{gathered} \quad \quad \begin{gathered}
    y-y_0 = \left( \frac{x_1 - x_0 }{2c} \right)(x-x_0) + - \frac{(x_1 -x_0)^2}{4c} \\
    y - y_0 = \frac{ (x_1 - x_0)(x-x_0)}{2c} -(y_1 - y_0) 
\end{gathered} \\
  \boxed{ (y-y_0) + (y_1 -y_0) = \frac{ (x_1 - x_0)(x-x_0)}{2c} }
\end{gathered}
\]

\exercisehead{7}
\begin{enumerate}
\item Given $y=x^2$, then 
\[
  \begin{aligned}
  X & = (x,x^2) \\
  \frac{X'}{\|X' \| } & = \frac{ (1,2x) x' }{ |x'| \sqrt{ 1 + 4x^2 } } \\
  N & = \frac{ (-2x,1)x' }{ |x'| \sqrt{ 1 + 4x^2 } }
\end{aligned} \quad \quad \, 
  \begin{gathered}
    tN + (x,y) \\
    (x,x^2) + t \frac{ (-2x,1)x' }{ |x'| \sqrt{ 1 + 4x^2 } } = (0,y_q) = Q \\
    \begin{aligned}
    \Longrightarrow x &= \frac{ 2xt}{\sqrt{ 1 + 4x^2 } } \\
    t &= \sqrt{ \frac{1}{4} + x^2 } 
\end{aligned} \quad \quad \quad \begin{aligned}
      y & = \frac{t}{\sqrt{ 1 + 4x^2} } + x^2 = \frac{\sqrt{ 1 + 4x^2}}{2 \sqrt{ 1 + 4x^2 }} +x^2 \\
      & = \frac{1}{2} + x^2 
\end{aligned}
\end{gathered}
\]
\item
\end{enumerate}

\exercisehead{8} Given $y=x^2$ \\
$\pm \sqrt{c} = x$ when $y=c$ \\
Circle condition: $\| C - R \| = \| (0,y_0) - (\pm \sqrt{c}, c) \| = \sqrt{ c + (y_0 -c)^2 } = y_0 $ \\
$\Longrightarrow \boxed{ \frac{1+c}{2} = y_0 }$  \medskip \\
$c\to 0$, then $\boxed{ y_0 = 1/2 }$, radius of circle.  

\exercisehead{9} Consider the condition, $\frac{x^2}{a^2} + \frac{y^2}{b^2} = 1$, pts. on an ellipse.  
\[
\begin{aligned}
  \frac{x^2}{a^2} + \frac{y^2}{b^2} \gtreqqless 1 & \Longleftrightarrow 
  \begin{gathered} 
     b^2 x^2 + a^2 y^2 \gtreqqless a^2 b^2 \\
     \begin{aligned}
       r^2 (b^2 \cos^2{\theta} + a^2(1-\cos^2{\theta}) ) & = r^2 (a^2 + (b^2-a^2)\cos^2{\theta} ) = \\
       & = r^2 a^2 ( 1 -e^2 \cos^2{\theta} ) 
       \end{aligned} 
\end{gathered} \\
  & \Longleftrightarrow r^2 ( 1 - e^2 \cos^2{\theta}) \gtreqqless b^2 
\end{aligned}
\]
Suppose for each fixed $\theta_1$, we consider all pts. along a line ray from the origin (which is inside the ellipse) to infinity, radially out.  Then $r^2 ( 1 -e^2 \cos^2{\theta_1} ) \gtreqqless b^2$\quad $\forall \, r \gtreqqless r_1$

\exercisehead{10} For $X = (x,y,)$ and given that $\frac{x^2}{a^2} + \frac{y^2}{b^2} = 1$, or $b^2 x^2 + a^2 y^2 = a^2 b^2$, so then \\
\[
\begin{gathered}
  2b^2 xx' + 2a^2 yy' =0 \Longrightarrow y' = \frac{ -b^2 xx'}{ a^2 y } \\
  X= (x,y) \Longrightarrow X' = (x',y') = x'(1, \frac{-b^2 x}{ a^2 y } ) = \frac{ - x'b^2}{y } \left( \frac{-y}{b^2}, \frac{x}{a^2} \right) \medskip \\
  T = \left( \frac{-y}{b^2}, \frac{x}{a^2} \right) \parallel X' \quad \quad \quad N = \left( \frac{x}{a^2}, \frac{y}{b^2} \right) \perp X' \text{ since } N \cdot X' = 0 
\end{gathered}
\]
\[
\begin{gathered}
  m = \frac{ x_0 /a^2}{ -y_0/b^2 } = \frac{ b^2 x_0 }{ -a^2 y_0 } \\
  (y-y_0) = m(x-x_0)  \Longrightarrow y = \frac{ b^2 x_0 }{-a^2 y_0 } x + \frac{ b^2 x_0^2 }{ a^2 y_0 } + \frac{ a^2 y_0}{ a^2 y_0} y_0 = \frac{ b^2 x_0 }{-a^2 y_0} x + \frac{b^2}{y_0 } \\
  \frac{ x_0 x}{a^2 }  + \frac{ y y_0}{b^2 } = 1 \Longrightarrow \boxed{ \frac{x\cos{\theta_0}}{a} + \frac{y \sin{\theta_0} }{b} = 1 }
\end{gathered}
\]

\exercisehead{11} Using the work of the previous exercise, Exercise 10, then \[
\frac{x\cos{\theta_0}}{a} + \frac{y \sin{\theta_0} }{b} = 1 = \frac{xa\cos{\theta_0}}{a^2} + \frac{y b\sin{\theta_0} }{b^2} = xx_0/a^2 + yy_0/b^2 
\]

\exercisehead{12} Use $T = \left( \frac{-y}{b^2}, \frac{x}{a^2} \right)$; \quad $N = \left( \frac{x}{a^2}, \frac{y}{b^2} \right)$ from the previous problems.  

Without loss of generality, center the ellipse on the origin and the major axis on the $x$ axis.  Then $F= ae e_x$ and $P = (x,y)$.  

\[
\begin{gathered}
  \left| \frac{ (F-P)\cdot N }{ |N| } \right| = \frac{ 1 - \frac{ex}{a} }{ \sqrt{ \frac{x^2}{a^4} + \frac{y^2}{b^4} } } \quad \quad \quad \, \left| \frac{ (-F - P)\cdot }{ | N | } \right| = \frac{ 1 + \frac{xe}{a} }{ \sqrt{ \frac{x^2}{a^4} + \frac{y^2}{b^4} } } \\
  \text{ Multiply the above distances together to get } \\
  \left( \frac{x^2}{a^4} + \frac{y^2}{b^4} \right)^{-1/2} \left( 1 + \frac{xe}{a} \right)\left( 1 - \frac{ex}{a} \right) =  \left( \frac{x^2}{a^4} + \frac{y^2}{b^4} \right)^{-1/2} \left( \frac{x^2}{a^2} + \frac{y^2}{b^2} - \frac{e^2}{a^2} x^2 \right)  = \boxed{ b^2 }
\end{gathered}
\]

\exercisehead{13} $x^2 + 4y^2 = 8$ \text{ or } $ \frac{x^2}{8} + \frac{y^2}{2} = 1$. We want to find the tangent lines parallel to $x+2y = 7$ or $y = \frac{7}{2} - \frac{x}{2}$.  So the slope of this line is $-\frac{1}{2}$.  Using $T =\left( \frac{-y}{b^2}, \frac{x}{a^2}\right) = \left( \frac{-y}{2}, \frac{x}{8} \right)$, then 
\[
\begin{gathered}
\frac{x/8}{-y/2} = \frac{x}{-4y} = -1/2 \text{ or } x = 2y \\
\text{ plugging back into the ellipse equation }, (2,1), (-2,-1)
\end{gathered}
\]

\exercisehead{14} By equidistant property of the circle, the focus length from origin, $ae$, must equal the minor axis length, $b$.  $ae = b=a\sqrt{ 1 -e^2}$ \quad $\Longrightarrow \boxed{ e = 1/\sqrt{2} }$

\exercisehead{15}
\begin{enumerate}
\item $\boxed{ \frac{x^2}{a^2} - \frac{y^2}{b^2} = 1 }$.  Recall that $b^2 = a^2 ( 1 -e^2) = -3a^2$, so that $\boxed{ \frac{1}{a^2} ( x^2 - \frac{y^2}{3} ) = 1}$.  
\item 
\[
\boxed{ A = \int_a^x b \sqrt{ \left( \frac{t}{a} \right)^2 - 1 } dt  - \frac{1}{2} ( x-a) b \sqrt{ \left( \frac{x}{a} \right)^2 - 1 } }
\]
$r$ is the length of $VP$: 
\end{enumerate}

\exercisehead{16} Given $\frac{x^2}{a^2} - \frac{y^2}{b^2} = 1$, then $ y = b \sqrt{ \left( \frac{x}{a} \right)^2 - 1 }$
\[
\begin{gathered} 
  X' = \left( 1 , \frac{ b \left( \frac{x}{a^2} \right) }{ \sqrt{ \left( \frac{x}{a} \right)^2 - 1 } } \right) = \left( 1 , \frac{ b^2 \left( \frac{x}{a^2} \right) }{ y} \right) = \frac{b^2}{y} \left( \frac{y}{b^2} , \frac{x}{a^2} \right)
\end{gathered}
\]
So $X' \parallel T$; and $X' \cdot N = 0$

\exercisehead{17} $m(x_0) = \frac{b^2 x_0}{ a^2 y_0}$ from the tangent vector above and so, using the ellipse equation, 
\[
(y-y_0) = \frac{b^2}{a^2} \frac{x_0}{y_0} (x-x_0) \Longrightarrow \frac{x_0 x}{a^2} - \frac{y_0 y}{b^2} = 1 
\]

\exercisehead{18} Given $X= (x,y)$, 
\[
\begin{gathered}
  X' = (x',y'); \quad \, N \parallel (y',-x') \\
  (x,y) + s(y',-x') = (x_1,0)  \quad \Longrightarrow \begin{aligned}
    y & = sx' \\
    x + sy' & = x_1
\end{aligned} \quad \Longrightarrow x_1 = x + yy'/x' 
\end{gathered}
\]
Isosceles condition: $\sqrt{ x^2 +y^2 } = \| (x_1,0)- (x,y) \| = \sqrt{ \left( \frac{yy'}{x'} \right)^2 + y^2 } $ 
\[
\frac{x}{y} = \frac{dy/dt}{dx/dt} = \frac{dy}{dx} \Longrightarrow \frac{y^2}{2C} - \frac{x^2}{2C} = 1 
\]
Note there's some ambiguity with the signs.  

\exercisehead{19}
$N \parallel (y',-x')$
\[
\begin{gathered}
  \begin{gathered}
    (x,y) + s(y',-x') = (x_1,0) \\
    y = sx' \\
    x + sy' = x_1 = x + \frac{yy'}{x' }
  \end{gathered} \quad \quad \quad 
\begin{gathered}
  (x,y) + t (y',-x') = (0,y_1) \\
  x = -ty' \\
  y- tx' = y_1 \\
  y +\frac{xx'}{y'} = y_1 
\end{gathered} \medskip \\
\| (x,y) - (x_1,0) \| = \| (x,y) - (0,y_1) \| \\
\Longrightarrow (x-x_1)^2 + y^2 = x^2 + (y-y_1)^2 \Longrightarrow \left( \frac{yy'}{x'} \right)^2 + y^2 =x^2 + \left( \frac{xx'}{y'} \right)^2 \Longrightarrow \left( \frac{dy}{dx} \right)^2 = \frac{x^2}{y^2} \\
\xrightarrow{ (4,5) } y^2 - x^2 = 9 
\end{gathered}
\]

\exercisehead{20}
Without loss of generality, consider a hyperbola given by $\frac{x^2}{a^2} - \frac{y^2}{b^2} = 1$ with asymptotes of $\frac{y}{x} = \pm \frac{b}{a}$.  The distances from a point on the hyperbola to the asymptotes are the following:
\[
\begin{gathered}
  \frac{ (X-P)\cdot N_1}{ |N_1 |} = \frac{ (x,y) \cdot ( -b,a) }{ \sqrt{ a^2 + b^2 } } = \frac{ -bx + ay}{ \sqrt{ a^2 + b^2 }} \\
  \frac{ (X-P)\cdot N_2}{ |N_2 |} = \frac{ (x,y) \cdot ( b,a) }{ \sqrt{ a^2 + b^2 } } = \frac{ bx + ay}{ \sqrt{ a^2 + b^2 }} \\
  \left( \frac{ |-bx + ay |}{ \sqrt{ a^2 +b^2 } } \right)\left( \frac{ bx + ay}{ \sqrt{ a^2 + b^2 } } \right) = \frac{ b^2 x^2 - a^2 y^2 }{ a^2 + b^2 } = \frac{a^2 b^2 }{ a^2 + b^2 } 
\end{gathered}
\]

\exercisehead{21} Recall that for $X =ru_r$ (useful trick to change into polar coordinates), 
\[
\begin{aligned}
  X' & = r' u_r + ru_{\theta} \theta' \\
  |X'| & = \sqrt{ r'^2 + (r\theta')^2 } 
\end{aligned} \quad \quad \Longrightarrow \int \sqrt{ r'^2 + (r\theta')^2 } dt 
\]
\begin{enumerate}
\item $\int_{\theta_0}^{\theta} \sqrt{ r'^2 + r^2 } d\theta = k (\theta -\theta_0)$
\[
\begin{gathered}
  \begin{gathered}
  \sqrt{ r'^2 + r^2 } - C = k  \\\
  r'^2 = k-r^2 
  \end{gathered} \quad \quad \, \Longrightarrow 
  \begin{gathered}
    \left( \frac{r'}{\sqrt{k}} \right)^2 + \left( \frac{r}{ \sqrt{k} } \right)^2 = 1 \\
  \boxed{  r =\sqrt{k} \sin{\theta} \text{ or } r = \sqrt{k } }
  \end{gathered}
\end{gathered}
\]
\item $\int_{\theta_0}^{\theta} \sqrt{ \left( \frac{dr}{d\theta} \right)^2 + r^2 } d\theta = k (r(\theta) - r(\theta_0) )$
\[
\Longrightarrow \begin{gathered}
  r'^2 + r^2 = (kr')^2 \\
  r'^2 = \frac{r^2 }{k^2 -1 } 
\end{gathered} \quad \quad 
\begin{gathered}
  \ln{ \frac{r}{r_0} } = \frac{ \theta - \theta_0}{ \sqrt{ k^2 - 1 } } \\
  \boxed{ r = C e\left( \frac{\theta}{\sqrt{ k^2 - 1 } } \right) }
\end{gathered}
\]
\item $\int_{\theta_0}^{\theta} \sqrt{ r'^2 + r^2 } d\theta = k \int_{\theta_0}^{\theta} \frac{1}{2} r^2 d\theta$ 
\[
\begin{gathered}
  \begin{gathered}
    r'^2 + r^2 = \frac{ k^2 }{4} r^4 \\
    \Longrightarrow \frac{ r' }{ |r| \sqrt{ \left( \frac{kr}{2} \right)^2 - 1 } } = 1 
\end{gathered} \quad \, \xrightarrow{ 
    \begin{aligned} 
      \sec{t} & = \frac{kr}{2} \\
      \tan{t} \sec{t} & = \frac{k}{2} \frac{dr}{dt } 
\end{aligned} } \quad \begin{gathered}
    \int \frac{ \frac{2}{k} \tan{t}\sec{t} }{ \frac{2}{k} |\sec{t} | \sqrt{ \sec^2{t} - 1 }} = \theta - \theta_0  \\
    arcsec{ \left( \frac{kr}{2} \right) } = \theta - C \\
    \boxed{ r = \frac{2}{k} \sec{(\theta -\theta_0) } } \text{ or } r = \frac{2}{k}
\end{gathered}    
\end{gathered}
\]
\end{enumerate}


\exercisehead{22} Given $r'(t) = (r(a) \times r(b))$, 
$r'(t) = \frac{r(b) - r(a) }{b-a}$ for some $t \in [a,b]$ (by mean-value thm. on each of the components).  
\[
r'(t) \cdot (r(a) \times r(b)) = 0 
\]
Geometrically it means that there must be some component of $r'(t)$ that belongs in the plane containing $r(b)-r(a)$ because the velocity vector $r'(t)$ must ``take'' $r(a)$ to $r(b)$.  





\end{document}
