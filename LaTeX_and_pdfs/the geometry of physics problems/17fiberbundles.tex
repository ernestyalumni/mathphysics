% 17fiberbundles.tex

\begin{quote}
  A vector bundle is a family of vector spaces parameterized by points in the base space. How do we parameterize a family of manifolds, say Lie groups?
\end{quote}

\section{ 17.1. Fiber Bundles and Principal Bundles }

\subsection{ 17.1a. Fiber Bundles }

\subsection{ 17.1b. Principal Bundles and Frame Bundles }

frame $\mathbf{e}$ at $p$ chosen

\begin{equation}
  f(p) = e_{\alpha}(p) g_{\alpha}(p) \quad \quad \quad \, (17.4)
\end{equation}

\begin{equation}
\begin{aligned}
  & \Phi_{\alpha} : U_{\alpha} \times G \to \pi^{-1}(U_{\alpha}) \\ 
  & \Phi_{\alpha}(p,g) = e_{\alpha}(p) g = (e_{\alpha})_i g^i_{ \, \, j } = f_j
\end{aligned}
\end{equation}

in an overlap, the same frame (17.4) will have another representation

\begin{equation}
  \mathbf{f}(p) = \mathbf{e}_{\beta}{(p)} g_{\beta}(p) \quad \quad \quad \, (17.5)
\end{equation}

\[
\begin{gathered}
  e_{\beta}(p) = e_{\alpha}(p) \tau_{\alpha \beta}(p) \\ 
  \tau_{\alpha \beta}(p) \equiv \tau_{\alpha \beta} \\ 
  g_{\alpha}(p) = \tau_{\alpha \beta}(p) g_{\beta}(p)
\end{gathered}
\]

diffeomorphism 
\[
\tau_{\alpha \beta}(p)  : G \to G
\]
left translation of $G$ by (transition) matrix $\tau_{\alpha \beta}(p)$



\subsection{ 17.1c. Action of the Structure Group on a Principal Bundle}

Let $\mathbf{f} = (\mathbf{f}_1 \dots \mathbf{f}_n)$ frame at $p$, $\mathbf{f} \in P$

\begin{theorem}[17.8]
  \[
(f \in P , g\in G) \to (fg) \in P 
\]
freely when $g\neq e$ and 
\[
\pi(fg) = \pi(f)
\]
i.e. preserves fibers
\end{theorem}

Proof:
  $\pi(\mathbf{f}) = p$ \\

$\Phi_{\alpha} : U_{\alpha} \times G \to \pi^{-1}(U_{\alpha})$ \quad \quad \, local trivialization \\
$\Phi_{\alpha}(p,g_{\alpha}) = \mathbf{f} \Longrightarrow \Phi_{\alpha}^{-1}(\mathbf{f}) = (p,g_{\alpha})$ 
\quad $\exists \, ! \, g_{\alpha}$ for $\mathbf{f}$

Let $g \in G$, 

right action of $g$ on $\pi^{-1}(U_{\alpha})$ is (locally action)

\[
\Phi_{\alpha}(p,g_{\alpha}g) = fg
\]

if $p \in U_{\alpha} \bigcap U_{\beta}$ 

\[
\begin{gathered}
  fg = \Phi_{\beta}(p,g_{\beta}g) = \Phi_{\beta}(p, \tau_{\beta \alpha}(p)g_{\alpha} g ) = \Phi_{\alpha}(p,g_{\alpha} g) 
\end{gathered}
\]
$\tau_{\beta \alpha} = \Phi_{\beta}^{-1} \Phi_{\alpha}$

\hrulefill

We see in this proof that the essential point is that \emph{left translations} in $G$ (say by $\tau_{\beta \alpha}$) \emph{ commute with right translations} (say by $g$).  



\subsection{ 17.2. }


\subsection{ 17.3. Chern's Proof of the Gauss-Bonnet-Poincar\'{e} Theorem }


\subsubsection{ 17.3a. A Connection in the Frame Bundle of a Surface }


\begin{equation}
  \omega \left( \frac{d\mathbf{x}}{dt} \right) \in \mathfrak{g} = \mathfrak{u}{ (1) } \quad \quad \quad \, (17.14)
\end{equation}






