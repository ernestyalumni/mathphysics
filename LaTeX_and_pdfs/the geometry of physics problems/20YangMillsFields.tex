% 20YangMillsFields.tex

\subsection{20.1. Noether's Theorem for Internal Symmetries }

\begin{quote}
\emph{How do symmetries yield conservation laws?}
\end{quote}

$\phi$ $N$-tuple $\phi^a(t,\mathbf{x}) = \phi^a(x)$, local representation of a section of some vector bundle $E$, 

\begin{tikzpicture}
  \matrix (m) [matrix of math nodes, row sep=2em, column sep=3em, minimum width=1em]
  {    
    E  &   \\     
    M &    \\ };
  \path[->]  (m-1-1) edge node [auto] {$\pi$} (m-2-1);
\end{tikzpicture}

%\begin{tikzpicture}
%  \matrix (m) [matrix of math nodes, row sep=2em, column sep=3em, minimum width=1em]
%  {
%    U_i \subset \mathbb{R}^{n+1} - 0 &  \\
%    V_i \subset \mathbb{R}P^n  & \mathbb{R}^n  \\ };
%  \path[-stealth]
%  (m-1-1) edge node [right] {$\varphi_i \pi$} (m-2-2)
%  edge node [left] { $\pi$} (m-2-1)
%  (m-2-1) edge node [below] {$\varphi$} (m-2-2);
%\end{tikzpicture}

In the case of a Dirac electorn, we have seen that $E$ is the bundle of complex 4-component Dirac spinors over a perhaps curved spacetime. If $E$ is not a trivial bundle (or if we insist on using curvilinear coordinates) we shall have to deal with the fact that $\partial_j \phi^a$ do not form a tensor.  

\subsubsection{ 20.1a. The Tensorial Nature of Lagrange's Equations }

Let $M^{n+1}$ (pseudo-) Riemannian manifold, let $E$ vector bundle over $M$; for definiteness, let fiber be $\mathbb{R}^N$. \\
section of this bundle over $U \subset M$ is described by $N$ real-valued functions $\lbrace \phi^a_U \rbrace$, \\
\quad where $\phi_V = c_{VU}\phi_U$ and \\
\quad \quad $c_{VU}(x)$ is $N\times N$ transition matrix function, $c^a_{VUb}$.   \\

notation $\begin{gathered} \quad \\
  \lbrace \Phi^a \rbrace \\ 
  \lbrace \Phi^a_{\alpha} \rbrace \\
\Phi^a_{\alpha} = \tau_{\alpha \beta} \Phi_{\beta} \end{gathered}$ \\

Lagrangian $L_0(x, \phi, \phi_x) \equiv L_0(x,\Phi, \partial_j \Phi^a)$



\subsection{20.2. Weyl's Gauge Invariance Revisited}

\subsubsection{ 20.2a. The Dirac Lagrangian }

\subsubsection{ 20.2b. Weyl's Gauge Invariance Revisited}

\subsubsection{ 20.2c. The Electromagnetic Lagrangian }

Instead of considering a change of (spacetime) coordinates $x$, we shall look at a change of the \emph{field} (fiber) coordinate $\psi$, i.e. \emph{a gauge transformation}.  

Since the phase of $\psi$ is not measurable, we \emph{should} be able to have invariance under a \emph{local} gauge transformation, where $\alpha = \alpha(x)$ varies with the spacetime point $x$!  \\
\quad Clearly the Dirac equation and Lagrangian are \emph{not} invariant under such a substitution because of the appearance of terms involving $d\alpha$.  \\
\quad It must be that \emph{there is some background field that is interacting with the electron}.  This background field will manifest itself through the appearance of the connection.  






\subsection{ The Yang-Mills Nucleon }

\begin{quote}
  How did the groups $SU(2)$ and $SU(3)$ appear in particle physics?
\end{quote}



\subsubsection{ 20.3a. The Heisenberg Nucleon }

\subsubsection{ 20.3b. The Yang-Mills Nucleon }

\subsubsection{ 20.3c. A Remark on Terminology }

We have related the connection matrices $\omega$ to the gauge potentials $A$ by 
\[
\omega = -i q A
\]
$q$ is called a generalized \textbf{charge}.  




