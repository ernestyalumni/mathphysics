% file: AGAT_dump.tex
% Algebraic Geometry, Algebraic Topology, in unconventional ``grande'' format; fitting a widescreen format
% 
% github        : ernestyalumni
% linkedin      : ernestyalumni 
% wordpress.com : ernestyalumni
%
% This code is open-source, governed by the Creative Common license.  Use of this code is governed by the Caltech Honor Code: ``No member of the Caltech community shall take unfair advantage of any other member of the Caltech community.'' 
% 

\documentclass[10pt]{amsart}
\pdfoutput=1
%\usepackage{mathtools,amssymb,lipsum,caption}
\usepackage{mathtools,amssymb,caption}

\usepackage{graphicx}
\usepackage{hyperref}
\usepackage[utf8]{inputenc}
\usepackage{listings}
\usepackage[table]{xcolor}
\usepackage{pdfpages}
%\usepackage[version=3]{mhchem}
%\usepackage{mhchem}

\usepackage{tikz}
\usetikzlibrary{matrix,arrows}

\usepackage{multicol}

\hypersetup{colorlinks=true,citecolor=[rgb]{0,0.4,0}}

\oddsidemargin=15pt
\evensidemargin=5pt
\hoffset-45pt
\voffset-55pt
\topmargin=-4pt
\headsep=5pt
\textwidth=1120pt
\textheight=595pt
\paperwidth=1200pt
\paperheight=700pt
\footskip=40pt








\newtheorem{theorem}{Theorem}
\newtheorem{corollary}{Corollary}
%\newtheorem*{main}{Main Theorem}
\newtheorem{lemma}{Lemma}
\newtheorem{proposition}{Proposition}

\newtheorem{definition}{Definition}
\newtheorem{remark}{Remark}

\newenvironment{claim}[1]{\par\noindent\underline{Claim:}\space#1}{}
\newenvironment{claimproof}[1]{\par\noindent\underline{Proof:}\space#1}{\hfill $\blacksquare$}

%This defines a new command \questionhead which takes one argument and
%prints out Question #. with some space.
\newcommand{\questionhead}[1]
  {\bigskip\bigskip
   \noindent{\small\bf Question #1.}
   \bigskip}

\newcommand{\problemhead}[1]
  {
   \noindent{\small\bf Problem #1.}
   }

\newcommand{\exercisehead}[1]
  { \smallskip
   \noindent{\small\bf Exercise #1.}
  }

\newcommand{\solutionhead}[1]
  {
   \noindent{\small\bf Solution #1.}
   }


\title{The Algebraic Geometry Algebraic Topology Dump}
\author{Ernest Yeung \href{mailto:ernestyalumni@gmail.com}{ernestyalumni@gmail.com}}
\date{5 mars 2017}
\keywords{Algebraic Geometry, Algebraic Topology}
\begin{document}

\definecolor{darkgreen}{rgb}{0,0.4,0}
\lstset{language=Python,
 frame=bottomline,
 basicstyle=\scriptsize,
 identifierstyle=\color{blue},
 keywordstyle=\bfseries,
 commentstyle=\color{darkgreen},
 stringstyle=\color{red},
 }
%\lstlistoflistings

\maketitle

From the beginning of 2016, I decided to cease all explicit crowdfunding for any of my materials on physics, math.  I failed to raise \emph{any} funds from previous crowdfunding efforts.  I decided that if I was going to live in \emph{abundance}, I must lose a scarcity attitude.  I am committed to keeping all of my material \textbf{open-sourced}.  I give all my stuff \emph{for free}.   

In the beginning of 2017, I received a very generous donation from a reader from Norway who found these notes useful, through \emph{PayPal}.  If you find these notes useful, feel free to donate directly and easily through \href{https://www.paypal.com/cgi-bin/webscr?cmd=_donations&business=ernestsaveschristmas%2bpaypal%40gmail%2ecom&lc=US&item_name=ernestyalumni&currency_code=USD&bn=PP%2dDonationsBF%3abtn_donateCC_LG%2egif%3aNonHosted}{PayPal}, which won't go through a 3rd. party such as indiegogo, kickstarter, patreon.  Otherwise, under the \emph{open-source MIT license}, feel free to copy, edit, paste, make your own versions, share, use as you wish.    

\noindent gmail        : ernestyalumni \\
linkedin     : ernestyalumni \\
twitter      : ernestyalumni \\

\begin{multicols*}{2}

  
\setcounter{tocdepth}{1}
\tableofcontents



\begin{abstract}
Everything about Algebraic Geometry, Algebraic Topology

\end{abstract}

\part{Reading notes on Cox, Little, O'Shea's \emph{Ideals, Varieties, and Algorithms: An Introduction to Computational Algebraic Geometry and Commutative Algebra}}

\section{Geometry, Algebra, and Algorithms}

\subsection{Polynomials and Affine Space}

fields are important is that linear algebra works over \emph{any} field

\begin{definition}[2] set of all polynomials in $x_1 , \dots , x_n$ with coefficients in $k$, denoted $k[x_1, \dots , x_n]$

\end{definition}

polynomial $f$ \emph{divides} polynomial $g$ provided $g= fh$ for some $h \in k[x_1, \dots , x_n ]$

$k[x_1, \dots, x_n]$ satisfies all field axioms except for existence of multiplicative inverses; commutative ring, $k[x_1, \dots , x_n]$ \emph{polynomial ring}


\subsubsection*{Exercises for 1 }

\exercisehead{1}
$\mathbb{F}_2$ commutative ring since it's an abelian group under addition, commutative in multiplication, and multiplicative identity exists, namely $1$.  It is a field since for $1\neq 0$, the multiplicative identity is $1$.  

\exercisehead{2}
\begin{enumerate}
\item[(a)]
\item[(b)]
\item[(c)]
\end{enumerate}




\subsection{Affine Varieties }


\subsection{Parametrizations of Affine Varieties}


\subsection{Ideals}



\subsection{Polynomials of One Variable}




\section{Groebner Bases}

\subsection{Introduction}




\subsection{Orderings on the Monomials in $k[x_1, \dots , x_n]$ }




\subsection{A Division Algorithm in $k[x_1, \dots , x_n ]$ }



\subsection{Monomial Ideals and Dickson's Lemma }


\subsection{The Hilbert Basis Theorem and Groebner Bases}


\subsection{Properties of Groebner Bases}



\subsection{Buchberger's Algorithm}






\section{Elimination Theory}



\subsection{The Elimination and Extension Theorems}


\subsection{The Geometry of Elimination}



\section{The Algebra-Geometry Dictionary}


\subsection{Hilbert's Nullstellensatz}


\subsection{Radical Ideals and the Ideal-Variety Correspondence}



\section{Polynomial and Rational Functions on a Variety}


\subsection{Polynomial Mappings }


\section{Robotics and Automatic Geometric Theorem Proving}



\subsection{Geometric Description of Robots}






\part{Reading notes on Cox, Little, O'Shea's \emph{Using Algebraic Geometry}}

\textbf{Using Algebraic Geometry}.  David A. Cox.  John Little. Donal O'Shea. Second Edition.  Springer.  2005.  ISBN 0-387-20706-6 QA564.C6883 2004

\section{ Introduction }

\subsection{ Polynomials and Ideals }

\emph{monomial } 

\begin{equation}
  (1.1) \quad \quad \, x_1^{\alpha_1} \dots x_n^{\alpha_n}
\end{equation}

total degree of $x^{\alpha}$ is $\alpha_1 + \dots + \alpha_n \equiv |\alpha|$ \\



field $k$, $k[x_1 \dots x_n]$ collection of all polynomials in $x_1 \dots x_n$ with coefficients $k$.   \\

polynomials in $k[x_1 \dots x_n]$ can be added and multiplied as usual, so $k[x_1 \dots x_n]$ has structure of commutative ring (with identity) \\
however, only nonzero constant polynomials have multiplicative inverses in $k[x_1 \dots x_n]$, so $k[x_1 \dots x_n]$ not a field \\
\quad however set of rational functions $\lbrace f/g | f,g \in k[x_1 \dots x_n], \, g\neq 0\rbrace$ is a field, denoted $k(x_1 \dots x_n)$ \\

so
\[
f = \sum_{\alpha} c_{\alpha}x^{\alpha}
\]
where $c_{\alpha} \in k$

so

\[
f \in k [x_1 \dots x_n ] = \lbrace f | f = \sum_{\alpha} c_{\alpha} x^{\alpha} , x^{\alpha} = x_1^{\alpha_1} \dots x_n^{\alpha_n}, c_{\alpha} \in k \rbrace
\]

$f$ homogeneous if all monomials have same total degrees

polynomial $f$ is homogeneous if all monomials have the \emph{same total degree} \\

Given a collection of polynomials $f_1 \dots f_s \in k[x_1 \dots x_n]$, we can consider all polynomials which can be built up from these by multiplication by arbitrary polynomials and by taking sums

\begin{definition}[1.3] Let $f_1 \dots f_s \in k[x_1 \dots x_n]$ \\
Let $\langle f_1 \dots f_s \rangle = \lbrace p_1 f_1  + \dots + p_s f_s | p_i \in k[x_1 \dots x_n] \text{ for } i = 1 \dots s \rbrace$
\end{definition}


\exercisehead{1} 
\begin{enumerate}
\item[(a)] $x^2 = x \cdot ( x  -y^2 ) + y \cdot ( xy )$  
\item[(b)] 
\[
 p \cdot ( x - y^2 ) = p x - p y^2
\]
and for $p xy = (py)x$
\item[(c)] 
\[
p(y) ( x - y^2) = p(y)x - p(y) y^2 \notin \langle x^2, xy \rangle
\]
\end{enumerate}

\exercisehead{2} 
\[
\begin{gathered}
  \sum_{i=1}^s p_i f_i  + \sum_{j=1}^s q_j f_j = \sum_{i=1}^s (p_i + q_i )f_i, \quad \, p_i + q_i \in k[x_1 \dots x_n] \end{gathered}
\]
$\langle f_1 \dots f_s \rangle$ closed under sums in $k[x_1 \dots x_n]$ \\

If $f\in \langle f_1 \dots f_s \rangle$, \\
\phantom{If }$ p \in k [x_1 \dots x_n]$

\[
\begin{gathered}
p\cdot f = p \sum_{i=1}^s q_j f_j = \sum_{i=1}^s pq_j f_j, \quad \, pq_j \in k[x_1 \dots x_n] \text{ so }  \\
p\cdot f \in \langle f_1 \dots f_s \rangle
\end{gathered}
\]

Done.  \\

The 2 properties in Ex. 2 are defining properties of ideals in the ring $k[x_1 \dots x_n]$

\begin{definition}[1.5]
Let $I \subset k[x_1 \dots x_n]$, \, $I \neq \emptyset$ \\
$I$ ideal if 
\begin{enumerate}
\item[(a)] $f+ g \in I$, \, $\forall \, f,g \in I$ 
\item[(b)] $pf \in I$, \, $\forall \, f \in I$, arbitrary $p \in k[x_1 \dots x_n]$
\end{enumerate}
\end{definition}

Thus $\langle f_1 \dots f_s\rangle$ is an ideal by Ex. 2.  \\

we call it the ideal generated by $f_1 \dots f_s$.  

\exercisehead{3} Suppose $\exists \, $ ideal $J$, $f_1 \dots f_s \in J$ s.t. $J \subset \langle f_1 \dots f_s \rangle$ \\
if $f\in \langle f_1 \dots f_s \rangle$, $f = \sum_{i=1}^s p_i f_i$, \, $p_i \in k[x_1 \dots x_n]$ \\

$\forall \, i = 1 \dots s$, $p_i f_i \in J$ and so $\sum_{i=1}^s p_i f_i \in J$, by def. of $J$ as an ideal.

\[
\langle f_1 \dots f_s \rangle \subseteq J \quad \quad \, \Longrightarrow J = \langle f_1 \dots f_s \rangle
\]

$\Longrightarrow \langle f_1 \dots f_s \rangle$ is smallest ideal in $k[x_1 \dots x_n]$ containing $f_1 \dots f_s$


\exercisehead{4} For $\begin{aligned} & \quad \\
  & I = \langle f_1 \dots f_s \rangle \\
  & J = \langle g_1 \dots g_t \rangle \end{aligned}$ \\

 $I = J$ iff $s=t$ and $\forall \, f \in I$, $f = \sum_{i=1}^t q_i g_i$ and if $ 0  = \sum_{i=1}^t q_i g_i$, $q_i =0$, \, $\forall \, i = 1 \dots t$, and if $0 = \sum_{i=1}^s p_i f_i$, \, $p_i = 0$, \, $\forall \,  i = 1 \dots s$


\begin{definition}[1.6]
\[
\sqrt{I} = \lbrace g \in k[x_1\dots x_n] | g^m \in I \text{ for some } m \geq 1 \rbrace
\]
\end{definition}

e.g. $x+y \in \sqrt{ \langle x^2 + 3 xy , 3xy + y^2 \rangle }$

in $\mathbb{Q}[x,y]$ since 

\[
(x+y)^3 =x(x^2 + 3xy) + y(3xy + y^2) \in \langle x^2 + 3xy, 3xy + y^2\rangle
\]


%\begin{definition}[1.6]
%\[
%\sqrt{I } = \lbrace g \in k[x_1 \dots x_n ] | g^m \in I \text{ for some m } \geq 1 \rbrace
%\]
%\end{definition}

%e.g. $x+y \in \sqrt{ \langle x^2 + 3xy, 3xy + y^2 \rangle }$

%in $\mathbb{Q}[x,y]$ since

%\[
%(x+y)^3 = x(x^2 + 3xy) + y(3xy + y^2) \in \langle x^2 + 3xy, 3xy + y^2 \rangle
%\]

\begin{itemize}
\item (Radical Ideal Property) $\forall \, $ ideal $I\subset k[x_1 \dots x_n]$, $\sqrt{I}$ ideal, $\sqrt{I} \supset I$
\item \textbf{(Hilbert basis Thm.)} $\forall \, $ ideal $I\subset k[x_1\dots x_n]$ \\
$\exists \, $ finite generating set, \\
i.e. $\exists \, \lbrace f_1 \dots f_2 \rbrace \subset k [x_1 \dots x_n]$ s.t. $I=\langle f_1 \dots f_s \rangle$
\item (Division Algorithm in $k[x]$) $\forall \, f,g \in  k[x]$ (EY : in 1 variable) \\
$\forall \, f, g \in k[x]$ (in 1 variable )\\
$f= qg + r$, $\exists \, !$ quotient $q$, $\exists \, $ remainder $r$
\end{itemize}

\subsection{}

\subsection{Gr\"obner Bases}

\begin{definition}[3.1]
  Gr\"obner basis for $I$ $\equiv G = \lbrace g_1 \dots g_k \rbrace \subset I$ s.t. $\forall \, f \in I$, $\text{LT}(f)$ divisible by $\text{LT}(g_i)$ for some $i$
\end{definition}

\begin{itemize}
\item (Uniqueness of Remainders) let ideal $I\subset k[x_1 \dots x_n]$ \\
division of $f\in k[x_1 \dots x_n]$ by Gr\"o bner basis for $I$, produces $f=g+r$, $g\in I$, and no term in $r$ divisible by any element of $\text{LT}(I)$
\end{itemize}





\subsection{Affine Varieties}

affine $n$-dim. space over $k$ \quad \, $k^n = \lbrace (a_1 \dots a_n ) | a_1 \dots a_n \in k \rbrace$

$\forall \, $ polynomial $f\in k[x_1 \dots x_n ]$, $(a_1 \dots a_n) \in k^n$ \\
\phantom{ \quad } $f: k^n \to k$ \\
\phantom{ \quad } $f(a_1 \dots a_n)$ s.t. $x_i = a_i$ i.e. \\

if $f= \sum_{\alpha} c_{\alpha} x^{\alpha}$ for $c_{\alpha} \in k$, then  \\
\phantom{ \quad } $f(a_1 \dots a_n) =\sum_{\alpha} c_{\alpha}a^{\alpha} \in k$, where $a^{\alpha} = a_1^{\alpha_1} \dots a_n^{\alpha_n}$

\begin{definition}[4.1]
affine variety $\mathbf{V}(f_1 \dots f_s) = \lbrace ( a_1 \dots a_n) | (a_1 \dots a_n) \in k^n, \, f_1(x_1 \dots x_n) = \dots = f_s(x_1 \dots x_n) = 0 \rbrace$ \\
subset $V\subset k^n$ is affine variety if $V = V(f_1 \dots f_s)$ for some $\lbrace f_i \rbrace$, polynomial $f_i \in k[x_1 \dots x_n]$
\end{definition}

\begin{itemize}
  \item (Equal Ideals Have Equal Varieties) If $\langle f_1 \dots f_s \rangle = \langle g_1 \dots g_t \rangle$ in $k[x_1 \dots x_n]$, then $\mathbf{V}(f_1 \dots f_s) = \mathbf{V}(g_1 \dots g_t)$
\end{itemize}

so, recap \\
if $\langle f_1 \dots f_s \rangle = \langle g_1 \dots g_t \rangle $ in $k[x_1 \dots x_n]$, \\
then $V(f_1 \dots f_s) = V(g_1 \dots g_t)$   \\

Recall Hilbert basis Thm. $\forall \, $ ideal $I \subset k[x_1 \dots x_n]$ 
\[
I= \langle f_1 \dots f_s \rangle
\]
$\Longrightarrow $ if $I=J$, then $V(I) = V(J)$

think of $V$ defined by $I$, rather than $f_1 = \dots = f_s =0$

\exercisehead{3}

Recall Def. 1.5 Let $I\subset k[x_1 \dots x_n]$ \\
$I$ ideal if $\begin{aligned} & \quad \\
  & f + g \in I  \quad \, \forall \, f,g \in I \\
  & pf \in I, \quad \, \forall \, f \in I \text{ arbitrary } p \in k[x_1 \dots x_n]
\end{aligned}$

Let $f,g \in I(V)$ 
\[
\begin{gathered}
  (f+g)(a_1 \dots a_n) = f(a_1 \dots a_n) + g(a_1 \dots a_n) = 0 + 0 = 0 \quad \quad \, f+g \in I(V) \\ 
  pf(a_1 \dots a_n) = p(a_1 \dots a_n) f(a_1 \dots a_n) = 0 \quad \quad \, pf \in I(V)
\end{gathered}
\]
Then $I(V)$ an ideal.

%\exercisehead{3} Recall Def. 1.5. Let $I \subset k[x_1 \dots x_n]$, $I$ ideal if $\begin{aligned} & \quad \\
%  & f+g \in I \quad \, \forall \, f,g\in I \\
%  & pf \in I , \quad \, \forall \, f \in I, \text{ arbitrary } p \in k[x_1 \dots x_n ] \end{aligned}$

%Let $f,g \in I(V)$ 
%\[

$V = V(x^2)$ in $\mathbb{R}^2$ \\
$I=\langle x^2 \rangle$ in $\mathbb{R}[x,y]$, \, $I= \lbrace px^2 | p \in k[x,y]\rbrace$ \\
\phantom{ \quad } $I \subset I(V)$, since $px^2 = 0$ for $x^2=0$, $(0,b)$, \, $b\in \mathbb{R}$ \\
But $p(x,y) = x\in I(V)$, as 
\[
I(V) = \lbrace f \in k[x_1 \dots x_n] | f(a_1 \dots a_n)=0, \, \forall \, (a_1\dots a_n) \in V\rbrace
\]
\phantom{ \quad \quad } $p(0,b) = x = 0$

But $x\notin I$

\exercisehead{4} $I\subset \sqrt{I}$

Recall Def. 1.6 $\sqrt{I} = \lbrace g \in k[x_1 \dots x_n] |g^m \in I \text{ for some } m\geq 1\rbrace$ \\
$\forall \, f \in I$, $f=f^1$, $m=1$, so $f\in \sqrt{I}$, \quad \, $I\subset \sqrt{I}$ \\
\phantom{\quad \quad } Hilbert basis thm., $\forall \, $ ideal $I\subset k[x_1 \dots x_n]$ s.t. $I=\langle f_1 \dots f_s \rangle$ \\
\phantom{\quad } $V(I) = \lbrace (a_1 \dots a_n) |(a_1 \dots a_n) \in k^n, \, f_1(a_1\dots a_n) = \dots = f_s(a_1\dots a_n)=0\brace$ \\
$\mathbf{I}(\mathbf{V}(I)) = \lbrace f \in k[x_1 \dots x_n] | f(a_1 \dots a_n) =0 \quad \, \forall \, (a_1 \dots a_n) \in V(I) \rbrace$ \\
Let $g\in \sqrt{I}$, \, $g^m \in I$, \, $g^m=g^{m-1}g$  \\
\phantom{\quad \quad \,} $g^m(a_1 \dots a_n) =0 = g^{m-1}(a_1 \dots a_n)g(a_1 \dots a_n) =0$.  Then $g(a_1 \dots a_n)=0$ or $g^{m-1}(a_1\dots a_m)=0$ \\
\phantom{\quad }as $g^m\in I$, and $V(I)$ is s.t. $f_1(a_1 \dots a_n) = \dots = f_s(a_1 \dots a_n)=0$ for $I=\langle f_1 \dots f_s \rangle$

\begin{itemize}
  \item (Strong Nullstellensatz) if $k$ algebraically closed (e.g. $\mathbb{C}$), $I$ ideal in $k[x_1 \dots x_n]$, then 
\[
\mathbf{I}(\mathbf{V}(I) = \sqrt{I}
\]
\item (Ideal-variety correspondence) Let $k$ arbitrary field
\[
\begin{aligned}
  & I \subset I(V(I)) \\ 
  & V(I(V)) = V \quad \, \forall \, V
\end{aligned}
\]
\end{itemize}

\subsection*{Additional Exercises for Sec.4}

\exercisehead{6}



\section{ Solving Polynomial Equations}

\subsection{}

\subsection{Finite-Dimensional Algebras}

Gr\"obner basis $G = \lbrace g_1 \dots g_t \rbrace$ of ideal $I\subset k[x_1\dots x_n]$, \\
recall def.: Gr\"obner basis $G = \lbrace g_1 \dots g_t\rbrace \subset I$ of ideal $I$, \, $\forall \, f \in I$, $\text{LT}(f)$ divisible by $\text{LT}(g_i)$ for some $i$ \\
\phantom{\quad \, } $f \in k[x_1\dots x_n]$ divide by $G$ produces $f=g+r$, $g\in I$, $r$ not divisible by any $\text{LT}(I)$ uniqueness of $r$ \\
$f\in k[x_1 \dots x_n]$ divide by $G$, 

Recall from Ch. 1, divide $f\in k[x_1 \dots x_n]$ by $G$, the division algorithm yields

\begin{equation}
  (2.1)  \quad \quad \quad \, f = h_1 g_1 + \dots + h_t g_t + \overline{f}^G
\end{equation}
where remainder $\overline{f}^G$ is a linear combination of monomials $x^{\alpha} \notin \langle \text{LT}(I) \rangle $ \\
\phantom{\quad } since Gr\"obner basis, $f\in I$ iff $\overline{f}^G=0$

$\forall \, f \in k[x_1\dots x_n]$, we have coset $[f] = f+I = \lbrace f +h|h\in I\rbrace$ s.t. $[f]=[g]$ iff $f- g \in I$

We have a 1-to-1 correspondence 
\[
\begin{gathered}
\text{remainders } \leftrightarrow \text{ cosets } \\
\overline{f}^G \leftrightarrow [f]
\end{gathered}
\]
algebraic
\[
\begin{aligned}
  & \overline{f}^G + \overline{g}^G \leftrightarrow [f] + [g] \\ 
  & \overline{ \overline{f}^G \cdot \overline{g}^G } \leftrightarrow [f]\cdot [g]
\end{aligned}
\]
$B = \lbrace x^{\alpha} | x^{\alpha} \notin \langle \text{LT}(I) \rangle \rbrace$ is a basis of $A$, basis monomials, standard monomials

20141023 EY's take

$\forall \, [f] \in A = k[x_1 \dots x_n]/I$, \, $[f] = p_ib_i$; \, $b_i \in B = \lbrace x^{\alpha} | x^{\alpha} \notin \langle \text{LT}(I) \rangle \rbrace$ \\
For $I = \langle G \rangle$ \\
\phantom{\quad } e.g. $G=\lbrace x^2 + \frac{3}{2} xy + \frac{1}{2} y^2 - \frac{3}{2} x - \frac{3}{2} y, xy^2-x, y^3-y \rbrace$ \\
$\langle \text{LT}(I) \rangle = \langle x^2, xy^2,y^3 \rangle$ \\
e.g. $B=\lbrace 1,x,y,xy,y^2\rbrace$ \\
\phantom{\quad } $[f]\cdot[g] = [fg]$ \\
e.g. $f=x, \, g=xy, \, [fg] = [x^2y]$ \\
now $f=h_1g_1 + \dots +h_tg_t+ \overline{f}^G$

\subsection{}

\subsection{Solving Equations via Eigenvalues and Eigenvectors}


\section{ Resultants }

\section{Computation in Local Rings}

\subsection{Local Rings}


\begin{definition}[1.1]
  \[
k[x_1 \dots x_n]_{\langle x_1 \dots x_n \rangle} \equiv \lbrace \frac{f}{g} | \text{ rational functions } \frac{f}{g} \text{ of } x_1 \dots x_n \text{ with } g(p) \neq 0 \text{ at } p \rbrace
\]
\end{definition}

main properties of $k[x_1 \dots x_n]_{\langle x_1 \dots x_n \rangle }$

\begin{proposition}[1.2]
  Let $R= k[x_1 \dots x_n]_{\langle x_1 \dots x_n \rangle }$.  Then
\begin{enumerate}
\item[(a)] $R$ subring of field of rational functions $k(x_1 \dots x_n) \supset k[x_1 \dots x_n]$
\item[(b)] Let $M=\langle x_1 \dots x_n \rangle \subset R$ (ideal generated by $x_1 \dots X_n$ in $R$) \\
Then $\forall \, \frac{f}{g} \in R \backslash M$, $\frac{f}{g}$ unit in $R$ (i.e. multiplicative inverse in $R$)
\item[(c)] $M$ maximal ideal in $R$
\end{enumerate}
\end{proposition}


\exercisehead{1} if $p=(a_1 \dots a_n) \in k^n$, $R = \lbrace \frac{f}{g} | f,g\in k[x_1 \dots x_n] , \, g(p) \neq 0 \rbrace$ 
\begin{enumerate}
\item[(a)] $R$ subring of field of rational functions $k(x_1 \dots x_n)$ 
\item[(b)] Let $M$ ideal generated by $x_1 - a_1 \dots x_n -a_n$ in $R$  \\
Then $\forall \, \frac{f}{g} \in R\backslash M$, $\frac{f}{g}$ unit in $R$ (i.e. multiplicative inverse in $R$)
\item[(c)]  $M$ maximal ideal in $R$
\end{enumerate}


\begin{proof}
let $p = (a_1 \dots a_n) \in k^n$ \\
let $g_1(p) \neq 0$, $g_2(p) \neq 0$ 
\[
\begin{gathered}
  \frac{f_1}{g_1 } + \frac{f_2}{g_2} = \frac{f_1 g_2 + f_2 g_1}{ g_1 g_2 } \quad \quad \,  g_1(p)g_2(p) \neq 0 \text{ so } \frac{f_1}{g_1} + \frac{f_2}{g_2} \in R \\
 \frac{f_1}{g_1} \cdot \frac{f_2}{g_2} = \frac{f_1 f_2}{g_1 g_2} \quad \quad \, g_1(p) g_2(p) \neq 0 \text{ so } \frac{f_1}{g_1}\frac{f_2}{g_2} \in R
\end{gathered}
\]
$f= \frac{f}{I} \in R$, \quad \, $\forall \, f\in k[x_1 \dots x_n]$, so $k[x_1 \dots x_n]\subset R$

\end{proof}

EY : 20141027, to recap, 

Let $V = k^n$ \\
Let $p = (a_1 \dots a_n)$ \\
single pt. $\lbrace p \rbrace$ is (an example of) a variety \\
$I(\lbrace p \rbrace) = \lbrace x_1 -a_1 \dots x_n -a_n \rangle \subset k[x_1 \dots x_n]$ \\

$R \equiv k[x_1 \dots x_n]_{\langle x_1 - a_1 \dots x_n-a_n \rangle }$ 
\[
R = \lbrace \frac{f}{g} | \text{ rational function $\frac{f}{g}$ of $x_1 \dots x_n$, $g(p) \neq 0$, $p=(a_1 \dots a_n) $ } \rbrace
\]

Prop. 1.2. properties 

\begin{enumerate}
\item[(a)] $R$ subring of field of rational functions $k(x_1 \dots x_n)$ \quad \, $k(x_1 \dots x_n) \subset R$ 
\item[(b)] $M = \langle x_1 \dots a_1 \dots x_n -a_n \rangle \subset R$.  ideal generated by $x_1 - a_1 \dots x_n-a_n$ \\
Then $\forall \, \frac{f}{g} \in R\backslash M$, $\frac{f}{g}$ unit in $R$ ( $\exists \, $ multiplicative inverse in $R$ )
\item[(c)] $M$ maximal ideal in $R$. \\
in $R$ we allow denominators that are not elements of this ideal $I(\lbrace p \rbrace)$ 
\end{enumerate}

\begin{definition}[1.3] local ring is a ring that has exactly 1 maximal ideal \end{definition}

\begin{proposition}[1.4] ring $R$ with proper ideal $M\subset R$ is local ring if $\forall \, \frac{f}{g} \in R\backslash M$ is unit in $R$
\end{proposition}

localization Ex. 8, Ex. 9 \\
parametrization

\exercisehead{2} \[
\begin{aligned}
  & x = x(t) = \frac{-2t^2 }{1+t^2} \\ 
 &  y = y(t) = \frac{2t}{1+t^2}
\end{aligned}
\]
$k[t]_{\langle t \rangle}$ \quad \, $\frac{-2t^2}{1+t^2}$ rational function of $t$.  $1+t^2 \neq 0$

if $k = \mathbb{C}$ or $\mathbb{R}$ \\

Consider set of convergent power series in $n$ variables \\

\begin{equation}
(1.5) \quad \quad \,   k\lbrace x_1 \dots x_n \rbrace = \lbrace \sum_{\alpha \in \mathbb{Z}^n_{\geq 0}} c_{\alpha} x^{\alpha} | c_{\alpha} \in k, \text{ series converges in some open $U\ni 0 \in k^n $ } \rbrace
\end{equation}

Consider set $k[[x_1 \dots x_n]]$ of formal power series

\begin{equation}
  (1.6) \quad \quad \, k[[x_1 \dots x_n]] = \lbrace \sum_{\alpha \in \mathbb{Z}^n_{\geq 0}} c_{\alpha} x^{\alpha} | c_{\alpha} \in k \rbrace \text{ series need not converge }
\end{equation} 


variety $V$ \\

$k[x_1\dots x_n]/\mathbf{I}(V)$ \phantom{ \quad \quad \quad } variety $V$


\subsection{Multiplicities and Milnor Numbers}


if $I$ ideal in $k[x_1\dots x_n]$, then denote $Ik[x_1\dots x_n]_{\langle x_1 \dots x_n \rangle}$ ideal generated by $I$ in larger ring $k[x_1\dots x_n]_{\langle x_1 \dots x_n \rangle}$

\begin{definition}[2.1] Let $I$ $0$-dim. ideal in $k[x_1 \dots x_n]$, so $V(I)$ consists of finitely many pts. in $k^n$.  \\
Assume $(0 \dots 0) \in V(I)$ \\
multiplicity of $(0\dots 0)\in V(I)$ is 
\[
\text{dim}_k{ k[x_1\dots x_n]_{\langle x_1\dots x_n \rangle}} / Ik[x_!\dots x_n]_{\langle x_1 \dots x_n \rangle}
\]
\end{definition}


generally, if $p=(a_1 \dots a_n) \in V(I)$ \\
multiplicity of $p$, $m(p) = \text{dim}{ k[x_1 \dots x_n]_M } / Ik[x_1 \dots x_n]_M$

\[
\text{dim}{ k[x_1 \dots x_n]_M } / Ik[x_1 \dots x_n]_M
\]

localizing $k[x_1 \dots x_n]$ at maximal ideal $M = I(\lbrace p \rbrace) = \langle x_1 - a_1 \dots x_n-a_n \rangle$


\section{}

\section{}

\section{ Polytopes, Resultants, and Equations }

\section{ Polyhedral Regions and Polynomials }

\subsection{ Integer Programming }

Prop. 1.12. \\

Suppose 2 customers $A, B$ ship to same location \\
\quad A: ship 400 kg pallet taking up $2 \, m^3$ volume \\
\quad B: ship 500 kg pallet taking up $3 \, m^3$ volume \\

shipping firm trucks carry up to 3700 \, kg, up to $20 \, m^3$ \\

B's product more perishable, paying \$ 15 per pallet \\

A pays \$ 11 per pallet

How many pallets from A, B each in truck to maximize revenues?

\begin{equation}
(1.1) \quad \quad \, \begin{gathered}
    4A + 5B \leq 37 \\
    2A  + 3B \leq 20 \\
    A, B \in \mathbb{Z}^*_{ \geq 0 } \end{gathered}
\end{equation}

maximize $11 A + 15 B$ \\

integer programming. \\
max. or min. value of some linear function 

\[
l(A_1 \dots A_n) = \sum_{i=1}^n c_i A_i 
\]

on set $(A_1 \dots A_n) \in \mathbb{Z}^n_{ \geq 0}$ s.t. 


3. Finally, by introducing additional variables; rewrite linear constraint inequalities as equalities. The new variables are called ``slack variables''

\begin{equation}
(1.4) \quad \quad \, a_{ij} A_j = b_i, \quad \, A_j \in \mathbb{Z}_{\geq 0}
\end{equation}

introduce indeterminate $z_i$, \, $\forall \, $ equation in (1.4)

\[
z_i^{a_{ij} A_j} = z_i^{b_i}
\]

$m$ constraints

\[
\prod_{i=1}^m z_i^{a_{ij}A_j} = \prod_{i=1}^m z_i^{b_i} = \left( \prod_{i=1}^m z_i^{a_{ij}} \right)^{ A_j}
\]

\begin{proposition}[1.6]
  Let $k$ field, define $\varphi: k[w_1 \dots w_n] \to k[z_1 \dots z_m]$ by 
\[
\varphi(w_j) = \prod_{i=1}^m z_i^{a_{ij}} \quad \quad \quad \, \forall \, j = 1 \dots n 
\]

and 

\[
\varphi(g(w_1 \dots w_n) ) = g(\varphi(w_1) \dots \varphi(w_n))
\]
$\forall \, $ general polynomial $g\in k[w_1 \dots w_n]$

Then $(A_1 \dots A_n)$ integer pt. in feasible region iff $\varphi: w_1^{A_1} \dots w_n^{A_n} \mapsto z_1^{b_1} \dots z_m^{b_m}$



\end{proposition}

\exercisehead{3}

Now 

\[
\begin{gathered}
\varphi(w_j) = \prod_{i=1}^m z_i^{a_{ij}} \\
z_i^{a_{ij} A_j} = z_i^{b_i}
\end{gathered}
\]

If $(A_1 \dots A_n)$ an integer pt. in feasible region, $a_{ij} A_j = b_i$

\[
\begin{gathered}
z_i^{a_{ij}A_j } = z_i^{b_i} = \prod_{j=1}^n z_i^{a_{ij} A_j} \Longrightarrow \prod_{j=1}^n \prod_{i=1}^m (z_i^{a_{ij} })^{A_j} = \prod_{i=1}^m z_i^{b_i} = \prod_{j=1}^n \varphi(w_j)^{ A_j} = \prod_{j=1}^n \varphi(w_j)^{A_j} = \varphi\left( \prod_{j=1}^n w_j^{ A_j } \right) = \prod_{i=1}^m z_i^{b_i}
\end{gathered}
\]
since $\varphi(g(w_1 \dots w_n)) = g(\varphi(w_1) \dots \varphi(w_n))$ \\

If $\varphi: \prod_{j=1}^n w_j^{A_j} \mapsto \prod_{i=1}^m z_i^{b_i}$

\[
\varphi\left( \prod_{j=1}^n w_j^{A_j} \right) = \prod_{j=1}^n (\varphi(w_j))^{A_j} = \prod_{i=1}^m z_i^{b_i} = \prod_{j=1}^n \left( \prod_{i=1}^m z_i^{a_{ij}} \right)^{ A_j} \Longrightarrow \prod_{j=1}^n z_i^{a_{ij} A_j} = z_i^{b_i}
\]
or $a_{ij}A_j = b_i$.  So $(A_1\dots A_n)$ integer pt.  




\exercisehead{4} 
\[
\prod_{i=1}^m z_i^{b_i} = \prod_{i=1}^m \prod_{j=1}^n z_i^{ a_{ij} A_j } = \prod_{j=1}^n \left( \prod_{i=1}^m z_i^{a_{ij}} \right)^{A_j} = \prod_{j=1}^n \varphi(w_j)^{A_j} = \varphi\left( \prod_{j=1}^n w_j^{A_j} \right)
\]
So if given $(b_1 \dots b_m) \in \mathbb{Z}^m$, and for a given $a_{ij}$, $a_{ij}A_j = b_i$ \\

For $m\leq n$, then $a_{ij}$ is surjective, so $\exists \, A_j$ s.t. $\prod_{i=1}^m z_i^{b_i} = \varphi\left( \prod_{j=1}^n w_j^{A_j} \right)$



\begin{proposition}[1.8]
Suppose $f_1 \dots f_n \in k[z_1 \dots z_m]$ given \\
Fix monomial order in $k[z_1 \dots z_n, w_1 \dots w_n ]$ with elimination property: \\
$\forall \, $ monomial containing 1 of $z_i$ greater than any monomial containing only $w_j$ \\

Let $\mathcal{G}$ Gr\"{o}bner basis for ideal
\[
I = \langle f_1 - w_1 \dots f_n - w_n \rangle \subset k[z_1 \dots z_m, w_1 \dots w_n]
\]
$\forall \, f \in k[z_1 \dots z_m]$, let $\overline{f}^{ \mathcal{G}}$ be remainder on division of $f$ by $\mathcal{G}$ \\
Then
\begin{enumerate}
\item[(a)] polynomial $f$ s.t. $f\in k[f_1 \dots f_n]$ iff $g= \overline{f}^{ \mathcal{G}} \in k[w_1 \dots w_n]$
\item[(b)] if $\begin{aligned} & \quad \\
  & f \in k [f_1 \dots f_n ] \\
  & g = \overline{f}^{\mathcal{G}}\in k[ w_1 \dots w_n] \end{aligned}$ \quad as in part (a), \\

then $f = g(f_1 \dots f_n)$ , giving an expression for $f$ as polynomial in $f_j$
\item[(c)] if $\forall \, f_i, f$ monomials, $f\in k[f_1 \dots f_n]$, \\
then $g$ also a monomial.  
\end{enumerate}
\end{proposition}



\subsection{Integer Programming and Combinatorics}



\section{Algebraic Coding Theory}


\section{The Berlekamp-Massey-Sakata Decoding Algorithm}





\href{https://martinralbrecht.files.wordpress.com/2010/07/20131022_buchberger_dtu.pdf}{Gr\"{o}bner Bases, Martin R. Albrecht of the DTU Crypto Group}

\part{Algebraic Topology}  

cf. Bredon (1997) \cite{Bred1997}


\section{Simplicial Complexes}  

cf. pp. 245, from Sec. 21 Simplicial Complexes of Ch. 4 Homology Theory in Bredon (1997) \cite{Bred1997}

$\mathbf{v}_0, \dots \mathbf{v}_n \in \mathbb{R}^{\infty}$, "affinely independent" if they span an affine $n$-plane, i.e. 
\[
\text{ if } \left( \sum_{i=0}^n \lambda_i \mathbf{v}_i =0 , \, \sum_{i=0}^n \lambda_i = 0 \right), \text{ then } \Longrightarrow \forall \, \lambda_i = 0
\]
If not, then, e.g. $\lambda_0 \neq 0$, assume $\lambda_0 =-1$, and solve the equations to get 

\[
\begin{gathered}
\mathbf{v}_0 = \sum_{i=1}^n \lambda_i \mathbf{v}_i \\
\sum_{i=1}^n \lambda_i = 1
\end{gathered}
\]
i.e. $\mathbf{v}_0$ is in affine space spanned by $\mathbf{v}_1\dots \mathbf{v}_n$.  

If $\mathbf{v}_0, \dots \mathbf{v}_n$ affinely independent, then 
\begin{equation}
\sigma = ( \mathbf{v}_0, \dots \mathbf{v}_n) = \lbrace \sum_{i=0}^n \lambda_i \mathbf{v}_i | \sum_{i=0}^n \lambda_i = 1, \, \lambda_i \geq 0 \rbrace
\end{equation}
is "affine simplex" spanned by $\mathbf{v}_i$; also convex hull of $\mathbf{v}_i$.  

$\forall \, k \leq n$, $k$-face of $\sigma$ is any affine simplex of form $(\mathbf{v}_{i_1}, \dots \mathbf{v}_{i_k})$, where vertices all distinct, so are affinely independent.  

\begin{definition}
	(geometric) simplicial complex $K:= $ collection of affine simplices s.t. \begin{enumerate}
		\item $\sigma \in K \Longrightarrow $ any face of $\sigma \in K$; and 
		\item $\sigma, \tau \in K \Longrightarrow \sigma \bigcap \tau $ is a face of both $\sigma$ and $\tau$, or $\sigma \bigcap \tau =\emptyset$
	\end{enumerate}

If $K$ simplicial complex, $|K| = \bigcup \lbrace \sigma | \sigma \in K \rbrace \equiv $ "polyhedron" of $K$
\end{definition}

\begin{definition}[Def. 21.2 of Bredon (1997) \cite{Bred1997}]
	polyhedron $:= $ space $X$ if $\exists \, $ homeomorphism $h: |K| \xrightarrow{ \approx } X$ for some simplicial complex $K$.  
	$h,K$ is triangulation of $X$; (map $h$, complex $K$)
\end{definition}

Let $K$ finite simplicial complex.  \\
Choose ordering of vertices $\mathbf{v}_0,\mathbf{v}_1\dots $ of $K$.  \\
If $\sigma = (\mathbf{v}_{\sigma_0}, \dots \mathbf{v}_{\sigma_n})$ is simplex of $K$, where $\sigma_0 < \dots < \sigma_n$, then  \\
\phantom{If } let $f_{\sigma} : \Delta_n \to |K|$ be 
\[
f_{\sigma} = [\mathbf{v}_{\sigma_b}, \dots \mathbf{v}_{\sigma_n}]
\]
in notation of Def. 1.2.  Bredon (1997) \cite{Bred1997}.  

Then this gives CW-complex structure on $|K|$ with $f_{\sigma}$ as characteristic maps.  




\part{Graphs, Finite Graphs}

\section{Graphs, Finite Graphs, Trees }

Serre (1980) \cite{Serr1980}  

cf. Chapter I. Trees and Amalgams, Section 1 Amalgams, Subsection 1.1 Direct limits of Serre (1980) \cite{Serr1980}  


Let $(G_i)_{i\in I}$, family of groups.    

$\forall \, $ pair $(i,j)$, let $F_{ij} = $ set of homomorphisms of $G_i$ into $G_j$

Want: group $G= \varinjlim G_i$ and 
\[
\lbrace f_i | f_i : G_i \to G \rbrace \text{s.t. } f_j \circ f = f_i \quad \, \forall \, f \in F_{ij}
\]
group $G$ and family $\lbrace f_i\rbrace$ universal in that  

(*) if $H$ group, if $\lbrace h_i | h_i :G_i \to H ; h_j \circ f = h_i \qquad \, \forall \, f \in F_{ij} \rbrace$, \\
then $\exists \, ! h: G\to H$ s.t. $h_i = h\circ f_i$  \\
i.e. $\text{Hom}(G,H) \simeq \varprojlim \text{Hom}(G_i,H)$, the inverse limit being taken relative to $F_{ij}$.  \\
i.e. $G$ direct limit of $G_i$ relative to the $F_{ij}$.  

\begin{proposition}
	$\exists \, !$ pair $G$, family $(f_i)_{i\in I}$, i.e. (pair consisting of $G, (f_i)_{i\in I}$, unique up to unique isomorphism.  
\end{proposition}
\begin{proof}
Define $G$ by generators and relations.  \\
Take generating family to be disjoint union of those for $G_i$.  \\
relations - $xyz^{-1}$ where $x,y,z \in G_i$, $z=xy \in G_i$ \\
\phantom{relations - } $xy^{-1}$ where $x\in G_i$, $y \in G_j$, $y=f(x)$ for at least $f\in F_{ij}$.  

Thus, existence of $G,\lbrace f_i\rbrace$.  

$G$ represents functor $H\mapsto \varprojlim \text{Hom}{(G_i,H)}$.  

Thus, uniqueness (also from universal property).  
\end{proof}

e.g. groups $A,G_1,G_2$, homomorphisms $\begin{aligned} & \quad \\ 
& f_1 : A \to G_1 \\
& f_2 : A \to G_2 
\end{aligned}$.  

$G$ obtained by amalgamating $A$ in $G_1,G_2$ by $f_1,f_2 \equiv G_1 *_A G_2$.  \\
1 can have $G=\lbrace 1 \rbrace$, even though $f_1,f_2$ non-trivial.  

\emph{Application}: (Van Kampen Thm.)

Let topological space $X$ be covered by open $U_1,U_2$.   \\
Suppose $U_1,U_2, U_{12}=U_1\bigcap U_2$ arcwise connected.  

Let basept. $x\in U_{12}$.  

Then $\pi_1(X;x)$ obtained by taking 3 groups 
\[
\pi_1(U_1;x), \pi_1(U_2;x), \pi_1(U_{12};x)
\]
and amalagamating them according to homomorphism
\[
\begin{aligned}
& \pi_1(U_{12};x) \to \pi_1(U_1;x)  \\
& \pi_1(U_{12};x) \to \pi_1(U_2;x)
\end{aligned}
\]

\exercisehead{1}  
Let homomorphisms $\begin{aligned} & \quad \\
	& f_1 : A \to G_1 \\ 
	& f_2: A \to G_2 \end{aligned}
$
amalgam $G=G_1 *_A G_2$.  

Define subgroups $A^n,G_1^n, G_2^n$, of $A,G_1,G_2$ recursively by 
\[
\begin{aligned}
& A^1 = \lbrace 1 \rbrace \\
& G_1^1 = \lbrace 1 \rbrace \\ 
& G_2^1 = \lbrace 1 \rbrace \\ 
\end{aligned}
\]

$A^n = $ subgroup of $A$ generated by $f_1^{-1}(G_1^{n-1})$ and $f_2^{-1}(G_2^{n-1})$  
\[
G_1^n = \text{subgroup of $G_i$ generated by $f_i(A^n)$ }
\]

Let $A^{\infty}, G_i^{\infty}$ be unions of $A^n, G_i^n$ resp.  

Show that $f_i$ defines injection $A/A^{\infty} \to G_i/G_i^{\infty}$.  

So the amalgamation is $G \simeq G_1/G_1^{\infty} *_{A/A^{\infty}} G_2/G_2^{\infty}$.  

Take the first induction case (for intuition about the solution).  

\[
\begin{aligned}
	& A^2 = \langle f_1^{-1}( G_1^1), f_2^{-1}(G_2^1) \rangle = \langle f_1^{-1}(\lbrace 1 \rbrace), f_2^{-1}(\lbrace 1 \rbrace ) \rangle \\
	& G_i^2 = f_i(A^2)
\end{aligned}
\]
Let $f_i(a) = f_i(b) \in G_i/G_i^{\infty}$; $a,b\in A/A^{\infty}$.  

Then since $f_i(a),f_i(b) \in G_i/G_i^{\infty}$, $f_i(a),f_i(b) \in \lbrace gG_i^{\infty} | g\in G_i \rbrace$ (quotient is defined to be the set of all left cosets of $G_i^{\infty}$, which has to be a normal subgroup for $G_i/G_i^{\infty}$ to be a quotient group).  

Since $a,b \in A/A^{\infty}$, suppose we take $a,b\in A$.  

And suppose we take 
\[
\begin{aligned}
& 	f_i(a) = f_i(a)G_i^{\infty} = f_i(a) f_i(A^{n_a} ) = f_i(aA^{n_a}) \\ 
& 	f_i(b) = f_i(b)G_i^{\infty} = f_i(b) f_i(A^{n_b} ) = f_i(bA^{n_b})  
\end{aligned}
\]
Taking $f_i^{-1}$ (recall for group homomorphisms, they map inverse of element of 1st. group to inverse of image of this element).  

$aA^{n_a}=bA^{n_b}\in A/A^{\infty}$ (This is okay as we've "quotiented out $A^{\infty}$; so indeed, they're equal)


cf. Subsection 1.2 Structure of amalgams of Serre (1980) \cite{Serr1980} 

Suppose given group $A$, family of groups $(G_i)_{i\in I}$, and, $\forall \, i\in I$, injective homomorphism $A\to G_i$.  

$*_A G_i \equiv $ direct limit (cf. no. 1.1) of family $(A,G_i)$ with respect to these homomorphisms, call it \emph{sum} (in category theory sense, i.e. product) of $G_i$ with $A$ amalgamated.  

e.g. $A=\lbrace 1 \rbrace$, \\
$*G_i \equiv $ free product of $G_i$.  

\subsubsection{reduced word}  

$\forall \, i \in I$, choose set $S_i$ of right coset representations of $G_i$ modulo $A$, 

assume $1 \in S_i$, 

$(a,s)\mapsto as$ is bijection of $A\times S_i$ onto $G_i$,  \\
$A\times (S_i-\lbrace 1 \rbrace) \to G_i-A$ (onto)

Let $\mathbf{i} = (i_1\dots i_n)$, $n\geq 0$, $i_j \in I$, s.t. 
\begin{equation}
i_m \neq i_{m+1} \text{ for } 1 \leq m \leq n-1
\end{equation}
cf. (T) of Serre (1980) \cite{Serr1980}.  

So \emph{reduced word} $m$ is defined as 
\[
m = (a;s_1\dots s_n)
\]
where $a\in A, s_1\in S_{i_1} \dots s_n \in S_{i_n}$, and $s-j \neq 1\, \forall \, j$.  

$f\equiv $ canonical homomorphism of $A$ into group $G= *_A G_i$ \\ 
$f_i \equiv $ canonical homomorphism of $G_i$ into group $G= *_A G_i$

EY : 20170611 (Further explanations, basic examples, from me):  

Given $A, \lbrace G_i\rbrace_{i\in I}$, injective (group) homomorphisms $\lbrace f_i: A \to G_i\rbrace_i$.  

$G_i \backslash f_i(A) = \lbrace f_i(A)g | g\in G_i\rbrace$.  

Right coset representation of $f_i(A)g\mapsto g$.  

e.g. $A,G_1,G_2$, $\begin{aligned} & \quad \\
	& f_1:A \to G_1 \\
		& f_2 : A\to G_2 \end{aligned}$.  
		
	\[
	\begin{aligned}
	& G_1\backslash f_1(A) = \lbrace f_1(A)g| g\in G_1\rbrace \\
	& G_2\backslash f_2(A) = \lbrace f_2(A)g | g\in G_2 \rbrace
	\end{aligned}
	\]

$\mathbf{i} = (i_1\dots i_n)$, $i_j\in I$, $i_m\neq i_{m+1}$ for $1\leq m \leq n-1$.  

Consider $(1212\dots 12)$  

$m=(a;f_1 g_2 f_3 g_4 \dots f_{2n-1}, g_{2n})$ where $f$'s $\in S_1 \subset G_1$, $g$'s $\in S_2 \subset G_2$.  

and so 
\begin{definition}[reduced word]
	\textbf{reduced word} of type $\mathbf{i}$, $m$, \begin{equation}
	m=(a;s_1\dots s_n)
	\end{equation}
	where $a\in A, s_1 \in S_{i_1}, \dots s_n \in S_{i_n}$, $s_j\neq 1$ \, $\forall \, j$, \\
	\phantom{where } $\mathbf{i} = (i_1\dots i_n)$, $i_j \in I$, s.t. $i_m \neq i_{m+1}$ for $1\leq m \leq n-1$, \\
	with $S_i = \lbrace g | g\in f_i(A)g \in f_i(A) G_i\rbrace$  
\end{definition}




\begin{theorem}[1 of Serre (1980) \cite{Serr1980} ]
	$\forall \, g \in G$, $\exists \, $ sequence $\mathbf{i}$ s.t. $i_m \neq i_{m+1}$ for $1\leq m \leq n-1$ and 
	
	reduced word 
	\[
	m = (a;s_1\dots s_n) 
	\]
	of type $\mathbf{i}$ s.t. 
	\[
	g = f(a)f_{i_1}(s_1) \dots f_{i_n}(s_n)
	\]
\end{theorem}

Furthermore, $\mathbf{i}$ and $m$ unique.  

\emph{Remark}.  Thm. 1 implies $f;f_i$ injective.  

Then identify $A$ and $G_i$ with images $f(A), f_i(G_i)$ in $G$, and reduced decomposition (*) of $g\in G$ 
\[
g = as_1\dots s_n, \quad \, a\in A, \, s_1 \in S_{i_1} - \lbrace 1 \rbrace \dots s_n \in S_{i_n} - \lbrace 1 \rbrace
\]
Likewise, $G_i \bigcap G_j = A$ if $i\neq j$.  

In particular, $S_i - \lbrace 1 \rbrace$ pairwise disjoint in $G$.  

\begin{proof}
Let $X_i \equiv $set of reduced words of type $\mathbf{i}$, $X = \coprod X_i$.  

Make $G$ act on $X$.  

In view of universal property of $G$, sufficient to make $\forall \, i, G_i$ act, 

check action induced on $A$ doesn't depend on $i$  

Suppose then that $i\in I$, and let $Y_i = $ set of reduced words of form $(1;s_1 \dots s_n)$, with $i_1\neq i$.  

EY : 20170611

Recall that 
\[
S_i = \lbrace g| g\in f_i(A) g \in f_i(A) G_i \rbrace
\]
\[
\begin{aligned}
	& A \times S_i \to G_i \text{ onto } \\ 
	& A\times (S_i - \lbrace 1 \rbrace) \to G_i - A \text{ onto } \\ 
	& (a,s) \mapsto as \text{ bijection }
\end{aligned}
\]

Let $Y_i = $ set of reduced words of form $(1; s_1 \dots s_n) = \lbrace (1;s_1 \dots s_n) | 1\in A; s_1 \in S_{i_1}\dots s_n \in S_{i_n} ; \mathbf{i} = (i_1\dots i_n), \, i_j \in I \text{ s.t. } i_m \neq i_{m+1} \text{ for } 1\leq m \leq n -1 \rbrace$.  

\[
\begin{gathered}
\begin{gathered}
A \times Y_i \to X = \coprod_i X_i \\
(a,(1; s_1 \dots s_n)) \mapsto (a; s_1 \dots s_n)
\end{gathered} \\
\begin{gathered}
A\times \lbrace S_i - \lbrace 1 \rbrace ) \times Y_i \to X \\
((a,s), (1;s_1 \dots s_n)) \mapsto (a;s,s_1\dots s_n)
\end{gathered}
\end{gathered}
\]
and remember that $X_i = $ set of reduced words of type $\mathbf{i}$.  

It's clear that this yields a bijection $A\times Y_i \bigcup A\times (S_i - \lbrace 1 \rbrace) \times Y_i \to X$.  

Let $x\in X$.  Then $x\in X_{\mathbf{i}}$ for some $\mathbf{i}$.  So $x$ is a reduced word of type $\mathbf{i}$: $x = (a;s_1\dots s_n)$.  Then clearly $x = (a;s_1\dots s_n) \mapsto (a,(1;s_1\dots s_n)) \in A\times Y_i$.  







\end{proof}


cf. pp. 13, Sec. 2. Trees, 2.1 Graphs of Serre (1980) \cite{Serr1980}

\begin{definition}[1. of Serre (1980) \cite{Serr1980}]
	\textbf{graph} $\Gamma = (X,Y, Y\to X\times X, Y\to Y)$, where $\begin{aligned} & \quad \\
		& \text{ set } X = \text{ vert } \Gamma \\ 
		& \text{ set } Y = \text{ edge } \Gamma \end{aligned}$  

\[
\begin{aligned}
& Y \to X\times X \\ 
& y\mapsto (o(y), t(y)) \\
& Y\to Y \\
& y\mapsto \overline{y}
\end{aligned}
\]
s.t. $\forall \, y \in Y$, $\overline{ \overline{y}} = y$, $\overline{y} \neq y$, $o(y) = t(\overline{y})$.  

vertex $P \in X$ of $\Gamma$. 

(oriented) edge $y\in Y$, $\overline{y} \equiv $ inverse edge.  

origin of $y := $ vertex $o(y) = t(\overline{y})$.  

terminus of $y:= $ vertex $t(y) = o(\overline{y})$   

extremities of $y:= \lbrace o(y),t(y)\rbrace$  

If 2 vertices \textbf{adjacent}, they're extremities of some edge.  

orientation of graph $\Gamma = Y_+ \subset Y = \text{ edge } \Gamma$ s.t. $Y = Y_+ \coprod \overline{Y}_+$.  It always exists.  

oriented graph defined, up to isomorphism, by giving 2 sets $X,Y_+$ and $Y+ \to X\times X$.  
	
	corresponding set of edges is $Y = Y_+\coprod \overline{Y}_+$ where $\overline{Y}_+ \equiv $ copy of $Y_+$  
\end{definition}

\subsubsection{Realization of a Graph}

cf. Realization of a Graph in Serre (1980) \cite{Serr1980}.  

Let graph $\Gamma$, $X = \text{vert}\Gamma$, $Y = \text{edge}\Gamma$.

topological space $T = X \coprod Y \times [0,1]$, where $X,Y$ provided with discrete topology.  

Let $R$ be finest equivalence relation on $T$ for which 
\begin{equation}
\begin{gathered} \begin{aligned}
	& (y,t) \equiv (\overline{y}, 1-t) \\ 
	& (y,0) \equiv o(y) \\
	& (y,1) \equiv t(y)
\end{aligned} \qquad \, \forall \, y \in Y, \, \forall \, t \in [0,1]
\end{gathered}
\end{equation}

quotient space $\text{real}(\Gamma) = T/R $ is \emph{realization} of graph $\Gamma$.  (realization is a functor which commutes with direct limits).  

Let $n\in \mathbb{Z}^+$.  Consider oriented graph of $n+1$ vertices $0,1,\dots n$,  \\

\begin{definition}
	path (of length $n$) in graph $\Gamma$ is morphism $c$ of $\mathbf{\text{Path}}_n$ into $\Gamma$  
\end{definition}

orientation given by $n$ edges $[i,i+1]$, $0\leq i <n$, $\begin{aligned} & \quad \\
& o([i,i+1]) = i \\
& t([i,i+1]) = i+1 \end{aligned}$  

For $n\geq 1$, \\
$(y_1\dots y_n)$ sequence of edges $y_i = c([i-1,i])$ s.t. 
\[
t(y_i) = o(y_{i+1}), \qquad \, 1 \leq i < n \text{ determine } c
\]
If $P_i = c(i)$,  \\
$c$ is a path from $P_0$ to $P_n$, and $P_0$ and $P_n$ are \emph{extremities of the path $c$}.  

pair of form $(y_i,y_{i+1})=(y_i, \overline{y}_i)$ in path is \textbf{backtracking}.  

path (of length $n-2$), from $P_0$ to $P_n$ given (for $n>2$) by $(y_1\dots y_{i-1}, y_{i+2}\dots y_n)$  

If $\exists \, $ path from $P$ to $Q$ in $\Gamma$, $\exists \, $ one without backtracking (by induction)  

direct limit $\mathbf{\text{Path}}_{\infty} = \varinjlim \mathbf{\text{Path}}_n$ provides notion of infinite path.  \\
$\mathbf{\text{Path}}_{\infty} \ni $ infinite sequence $(y_1,y_2 , \dots)$ of edges s.t. $t(y_i) = o(y_{i+1})$ \, $\forall \, i \geq 1$.  


\begin{definition}[connected graph; Def. 3 of Serre (1980) \cite{Serr1980}]
	graph connected if $\forall \, $ 2 vertices, 2 vertices are extremities of at least 1 path.  
	
	maximal connected subgraphs (under relation of inclusion) are \emph{connected components} of graph.  
\end{definition}

\subsubsection{Circuits}  

Let $n\in \mathbb{Z}^+$, $n\geq 1$.  

Consider \\
set of vertices $\mathbb{Z}/n\mathbb{Z}$, orientation given by $n$ edges $[i,i+1]$, $(i\in \mathbb{Z}/n\mathbb{Z})$ with $\begin{aligned} & \quad \\
 & o([i,i+1]) = i \\
 & t([i,i+1]) = i+1 \end{aligned}$

\begin{definition}[circuit; Def. 4 of Serre (1980) \cite{Serr1980}]
	circuit (length $n$) in graph is subgraph isormorphic to $\mathbf{\text{Circ}}_n$.  
\end{definition}
i.e. subgraph = path $(y_1\dots y_n)$, without backtracking, s.t. $P_i = t(y_i)$, \, $(1\leq i \leq n)$ distinct, s.t. $P_n = o(y_1)$

$n=1$ case: $\mathbf{\text{Circ}}_1$, $\mathbb{Z}/\mathbb{Z} = \lbrace 0 \rbrace$, $1$ edge, $[0,1]$, $0 \in \mathbb{Z}/1\mathbb{Z}$, $\begin{aligned} & \quad \\
	& o([0,1]) = 0 \\
	& t([0,1]) = 1 \end{aligned}$  
	
	Note $\mathbf{\text{Circ}}_1$ has automorphism of order 2, which changes its orientation, i.e. \\
	$\exists \, $ automorphism $\sigma \in \text{Aut}( \mathbf{\text{Circ}}_1) $ s.t. $|\sigma | = 2$, i.e. $\sigma^2=1$.  \\
	loop $:= $ circuit of length $1$; so loop $\in \overline{ \mathbf{\text{Circ}} }_1$.  
	
	path $(y_1)$, $P_1 = t(y_1) = o(y_1)$.  
	
	$n=2$ case: $\mathbf{\text{Circ}}_2$, $\mathbb{Z}/2\mathbb{Z} = \lbrace 0 ,1\rbrace$, 2 edges $[0,1], [1,2]$,  \\
	path $(y_1,y_2)$, $(1\leq i \leq 2)$, $\begin{aligned} & \quad \\
		& P_1 = t(y_1) \\
		& P_2 = t(y_2) = o(y_1) \end{aligned}$  
		
		
	

\subsection{Combinatorial graphs}

Let $(X,S)\equiv $ simplicial complex of dim. $\leq 1$, with \\
$X \equiv $ set \\
$S \equiv $ set of subsets of $X$ with $1$ or $2$ elements, containing all the 1-element subsets.  

associates with it a graph $\Gamma = (X, \lbrace (P,Q) \rbrace)$.  

$X$ is its set of vertices.  

edges $=\lbrace (P,Q)\in X\times X\rbrace$ s.t. $P\neq Q$, $\lbrace P ,Q \rbrace \in S$, with $\overline{(P,Q)} = (Q,P)$
\[
\begin{aligned}
& o(P,Q) = P \\
& t(P,Q) = Q
\end{aligned}
\]

In this graph, 2 edges with same origin and same terminus are equal.  This is equivalent to (see following Def.)

\begin{definition}[combinatorial; Def. 5 of Serre (1980) \cite{Serr1980}]
	graph is combinatorial if it has no circuit of length $\leq 2$
\end{definition}
Conversely, it's easy to see that 

every combinatorial graph $\Gamma$ derived (up to isomorphism) by construction above from simplicial complex $(X,S)$, where \\
$X = \text{vert} \Gamma$ \\
$S=$ set of subset $\lbrace P,Q \rbrace$ of $X$ s.t. $P$ and $Q$ either adjacent or equal.  



\end{multicols*}

\begin{thebibliography}{9}

\bibitem{CLS2005}
David A. Cox.  John Little. Donal O'Shea. \textbf{Using Algebraic Geometry}.  Second Edition.  Springer.  2005.  ISBN 0-387-20706-6 QA564.C6883 2004

\bibitem{CLS2015}
David Cox, John Little, Donal O'Shea. \textbf{Ideals, Varieties, and Algorithms: An Introduction to Computational Algebraic Geometry and Commutative Algebra}, Fourth Edition, Springer

\bibitem{Bred1997}
Glen E. Bredon.  \textbf{Topology and Geometry}. Graduate Texts in Mathematics (Book 139).  Springer; Corrected edition (October 17, 1997).  ISBN-13: 978-0387979267


\bibitem{Serr1980}  
Jean-Pierre Serre (Author), J. Stilwell (Translator).  \textbf{Trees} (Springer Monographs in Mathematics) 1st ed. 1980. Corr. 2nd printing 2002 Edition.  ISBN-13: 978-3540442370





\end{thebibliography}


\end{document}
