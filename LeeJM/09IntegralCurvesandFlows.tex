

\subsection*{Integral Curves }

If smooth curve $ \begin{aligned} & \quad \\ 
 c:  & I \to M \\  
 c(t) & = p \end{aligned}$ \\
$\forall \, t \in I ,  \, c' \equiv c'(t) \in T_{c(t)}M \equiv T_p M$ \\

If $X$ vector field on $M$, \\
integral curve of $X$ is differentiable $c: I  \to M$ s.t. \\
\quad $\forall \, t \in I$, $c'(t) \equiv \dot{c} = X_{c(t)} = X_p$  \\

EY  \\
\[
\begin{aligned}
  & c:I \to M \\ 
  & c(t) = p  \\
  & \varphi c(t) = \varphi(p) \Longrightarrow \begin{aligned} & c^i(t) = x^i(t) \\
    & \dot{c}^i(t) = \dot{x}^i(t) \end{aligned} \\
  X_p = X^i(p) \frac{ \partial }{ \partial x^i } \equiv X^i \frac{ \partial }{ \partial x^i } = \dot{c}^i \frac{ \partial }{ \partial x^i } 
\end{aligned}
\]

\[
\begin{gathered}
  f: M \to \mathbb{R} \\ 
  X_p f = X_p f(p) = X_pf \varphi^{-1} \varphi(p) = X_(f\varphi^{-1})(x^j) = X^i \frac{ \partial f}{ \partial x^i }(x^j) \equiv X^i(p)  \frac{ \partial f}{ \partial x^i}(x^j) = \dot{c}^i \left. \frac{ \partial f}{ \partial x^i} \right|_p = \frac{d}{dt}( f \circ c)(t)
\end{gathered}
\]




Example 9.1.  (Integral Curves)

\begin{enumerate}
\item[(a)] Let $X = \frac{ \partial }{ \partial x}$, $(x,y) \in \mathbb{R}^2$ 

\[
\begin{gathered}
  c(t) = (x(t), y(t)) = x \partial_x + y\partial_y \\ 
  c' = \dot{x} \partial_x + \dot{y} \partial_y = \partial_y   \Longrightarrow \dot{y} = 0 \quad \, y = b \\
  \dot{x} = 1 \quad \quad \, x = a+t \\ 
  c = (a+t ,b) 
\end{gathered}
\]


\item[(b)] $X = x\partial_x - y \partial_x$.  Comparing the components of these vectors, we see that this is equivalent to 

\[
\begin{gathered}
  \begin{aligned}
    \dot{x} & = -y \\
    \dot{y}  & = x \end{aligned} \Longrightarrow \begin{aligned}
    y & = a \sin{t} + b\cos{t} \\ 
    x & = a\cos{t} - b\sin{t} \end{aligned}
\end{gathered}
\]



\end{enumerate}


\begin{proposition}[9.2] Let smooth vector field $X$ on smooth $M$, \\
$\forall \, p \in M$, $\exists \, \epsilon > 0 $, $\exists \, $ smooth $c:(-\epsilon, \epsilon) \to M$ i.e. integral curve of $X$ starting at $p$ \end{proposition}

\begin{proof}
  existence from Thm. D.1
\end{proof}


\begin{lemma}[9.3] (Rescaling Lemma) 
$\widetilde{ c}(t) = c(at)$ integral curve of $aX$, where $\widetilde{I} = \lbrace t | at \in I \rbrace$
\end{lemma}
\begin{proof} Let smooth $f$ defined in neighborhood of $\widetilde{c}(t_0)$ \\
e.g. of rescaling  - $2t = 2\cdot 1 = 2 \quad \, a=2$ 

\[
\begin{gathered}
  \widetilde{c}(t) = c(at) = c(\tau) = p \in M \\ 
  \dot{ \widetilde{c}}(t) f = \frac{d}{dt} (f\circ \widetilde{c})(t) = \frac{d}{dt} (f\varphi^{-1})( \varphi\widetilde{c}(t)) = \frac{d}{dt} (f\varphi^{-1})(\varphi c(at) ) = \frac{d}{dt} (f\varphi^{-1})(c^i(at)) = \left. \frac{ \partial f}{ \partial x^i} \right|_p \frac{dc^i}{ d\tau}(\tau)a = a \frac{d}{d \tau}(f\circ c)(\tau) = aX_p f
\end{gathered}
\]


\end{proof}

\begin{lemma}[9.4] (Translation Lemma)
\[
\begin{aligned}
  & \widetilde{I} = \lbrace t | t + a \in I \rbrace \\ 
  & \widetilde{c}(t) = c(t+a)
\end{aligned}
\]



\end{lemma}

\exercisehead{9.5}
\begin{proof}
\[  
\begin{aligned}
&  \widetilde{c}(t) = c(t+a) = c(\tau) = p \in M \\ 
&  \dot{ \widetilde{c}}(t) f = \frac{d}{dt} (f\circ \widetilde{c})(t) = \frac{d}{dt} f(\widetilde{c}(t)) = \frac{d}{dt} f(c(t+a) ) = \left. \frac{ \partial f}{ \partial x^i } \right|_p \frac{d c^i(\tau) }{ d\tau } = \dot{c}^i(\tau)  \left. \frac{ \partial f}{ \partial x^i } \right|_p = X^i_p \left. \frac{ \partial f}{ \partial x^i} \right|_p = X_p f
\end{aligned}
\]
\end{proof}




\begin{proposition}[9.6] (Naturality of Integral curves)
  Suppose smooth $F: M \to N$ \\
  \quad Then $\begin{aligned} & \quad \\ 
    & X \in \mathfrak{X}(M) \\
    & Y \in \mathfrak{X}(N) \end{aligned}$ \quad \, $F$-related iff $F$ takes integral curves of $X$ to integral curves of $Y$
\end{proposition}

\begin{proof}
Recall 
\[
X,Y \, F-\text{related means } \, dF(X) =Y
\]

Let $\gamma = Fc$ 
\[
\dot{\gamma} = \frac{d}{dt} (F\circ c)(t) = (dF)(\dot{c}) = dF(X) = Y
\]

$\Longrightarrow \gamma $ integral curve of $Y$ \\

if $\gamma = Fc$ integral curve of $Y$, $\dot{\gamma} = Y$.  \quad $q = F(p)$.  $p=c(t)$

\[
\begin{gathered}
  Yg= Y_qg = Y^j \left. \frac{ \partial g}{ \partial y^j } \right|_q = \dot{\gamma}^j(t) \left. \frac{ \partial g}{ \partial y^j } \right|_{F(p)} = \frac{d}{dt} (F\circ c ) \left. \frac{ \partial g}{ \partial y^j } \right|_{F(p)} = \frac{ \partial y^j}{ \partial x^k } \dot{c}^k(t) \left. \frac{ \partial g}{ \partial y^j } \right|_q = \frac{ \partial y^j}{ \partial x^k} X^k_p \left. \frac{ \partial g}{ \partial y^j } \right|_q = (F_* X)g = dF(X)g \\
\Longrightarrow dF(X) = Y
\end{gathered}
\]
\end{proof}


\subsection*{Flows }


Let $X \in \mathfrak{X}(M)$ \\
Suppose $\forall \, p \in M$, $\exists \, !$ \, integral curve starting at $p$, $\phi^{(p)}: \mathbb{R} \to M$ \\
$\forall \, t \in \mathbb{R}$, define $\begin{aligned} & \quad \\
  & \phi_t: M \to M \\
  & \phi_t(p) = \phi^{(p)}(t) \end{aligned}$

\[
\theta_0(p) = \theta^{(p)}(0) = p
\]

EY: $\phi_t$ pushes $p$ to $\phi^{(p)}(t)$ over time interval $t$ \\

translation lemma implies $t\mapsto \phi^{(p)}(t+s)$ is integral curve of $X$ starting at $q = \phi^{(p)}(s)$  \\

assuming uniqueness of integral curves, $\phi^{(p)}(t) = \phi^{(p)}(t+s)$, so 

\[
\begin{gathered}
  \phi_t \circ \phi_s(p) = \phi_{t+s}(p) \\ 
  \phi_0(p) = \phi^{(p)}(0) = p 
\end{gathered}
\]

$\Longrightarrow \phi : \mathbb{R}\times M \to M$ is an action of additive group $\mathbb{R}$ on $M$.  \\

define \emph{ global flow } on $M$ (1-parameter group action) - cont. left $\mathbb{R}$-action on $M$, i.e. \\
\quad cont. $\phi : \mathbb{R} \times M \to M$ s.t. $\forall \, s,t \in \mathbb{R}$, $\forall \, p M$ 

\begin{equation}
\begin{aligned}
& \phi(t, \phi(s,p) ) = \phi(t+s, p )
& \phi(0,p) = p 
\end{aligned} \quad \quad \quad \, (9.2)
\end{equation}

given global flow $\phi$ \\

$\forall \, t \in \mathbb{R}$, define cont. $\begin{aligned} & \quad \\
  & \phi_t: M \to M \\
  & \phi_t(p) \to \phi(t,p)
\end{aligned}$  
\[
\xrightarrow{ (9.2) } \begin{gathered}
  \phi_t \cdot \phi_s = \phi_{t+s} \\ 
  \phi_0 = 1_M
\end{gathered}
\]

$\phi_t : M \to M$ homeomorphism; if flow smooth, $\phi_t$ diffeomorphism.   \\

$\forall \, p \in M$, define $ \begin{aligned} & \quad \\ & \phi^{(p)}:\mathbb{R} \to M \\ & \phi^{(p)}(t) = \phi(t,p) \end{aligned}$ \\

$\phi^{(p)}$ is orbit of $p$ under group action.  

smooth global flow $\theta:\mathbb{R} \times M \to M$ \\
\quad $\forall \, p \in M$, define $V_p \in T_pM$
\[
V_p = (\theta^{(p)})'(0)
\]


\begin{proposition}[9.7] Let smooth global flow $\phi: \mathbb{R}\times M \to M$ on smooth $M$ \\
infinitesimal generator $X$ of $\phi$ \quad $\begin{aligned} & \quad \\
  & p \mapsto X_p \\
  & X_p  = \dot{\phi}^{(p)}(0) \end{aligned}$ \quad is smooth vector field on $M$, and $\forall \, \phi^{(p)}$, $\phi^{(p)}$ integral curve of $X$ 
\end{proposition}

\begin{proof}
  Show $X$ smooth.  Use Prop. 8.14,  \\
$f$ smooth on open $U\subseteq M$, $f:U\to \mathbb{R}$ 

\[
Xf(p) = X_pf = \dot{\phi}^{(p)}(0) f \equiv ( \dot{\phi}^{(p)}(0) )[f] = \left. \frac{d}{dt} ( f\circ \phi^{(p)} ) \right|_{t=0} = \frac{ \partial }{ \partial t} \left. (f\circ \phi(t,p) ) \right|_{t=0} 
\]

$f\circ \phi(t,p) = f(\phi(t,p))$ smooth function of $(t,p)$ by composition, so $\partial_t(f\circ \phi)$ smooth. So $Xf$ smooth, so $X$ smooth.   \\

Let $q = \phi^{(p)}(a) = \phi_a(p)$ 

\begin{equation}
  \phi^{(q)}(t) = \phi_t(q) = \phi_t( \phi_a(p)) = \phi_{t+a}(p) = \phi^{(p)}(t+a) \quad \quad \quad \, (9.4)
\end{equation}

\begin{equation}
  X_qf = \dot{\phi}^{(q)}(0)f = \dot{\phi}^{(q)}(0)[f] = \frac{d}{dt} \left. (f \circ \phi^{(q)}(t)) \right|_{t=0} = \frac{d}{dt} \left. (f\circ \phi^{(p)}(t+a) ) \right|_{t=0} = \dot{\phi}^{(p)}(a) f = X_{ \phi^{(p)}(a) } f \quad \quad \quad \, (9.5)
\end{equation}

\end{proof}

So by def., $\phi^{(p)}(t)$ integral curve of $X$


\subsubsection*{ The Fundamental Theorem on Flows }

flow domain for $M$ is open $\mathcal{D} \subseteq \mathbb{R} \times M$ s.t. $\forall \, p \in M$, $\mathcal{D}^{(p)} = \lbrace t \in \mathbb{R} | (t,p) \in \mathcal{D} \rbrace$ is an open interval containing $0$.  \\

\textbf{flow} on $M$ is cont. $\phi : \mathcal{D} \to M$ s.t. group laws : 
\begin{equation}
  \forall \, p \in M, \, \phi(0, p ) = p \quad \quad \quad \, (9.6)
\end{equation}

\begin{equation}
  \begin{aligned} & \quad \\ 
    & \forall \, s \in \mathcal{D}^{(p)} \\ 
    & \forall \, t \in \mathcal{D}^{ (\phi(s,p))} \end{aligned} \quad \text{ s.t. } s + t \in \mathcal{D}^{(p)}, \quad \quad \, \phi(t, \phi(s,p) ) = \phi(t+s, p ) \quad \quad \quad \, (9.7)
\end{equation}

\begin{proposition}[9.11] If $\phi: \mathcal{D} \to M$ smooth flow, \\
then infinitesimal generator $X$ of $\phi$ smooth vector field and $\forall \, \phi^{(p)}$ integral curve of $X$
\end{proposition}

\begin{proof}
Recall that  \\
\emph{infinitesimal generator} $X$ of $\phi$, $\begin{aligned} & \quad \\ 
  & p \mapsto X_p \\
  & X_p = \dot{\phi}(0) \end{aligned}$ \quad now on open $\mathcal{D}$, $\mathcal{D} \subseteq \mathbb{R} \times M$ s.t. $\forall \, p \in M$, $\mathcal{D}^{(p)} = \lbrace t \in \mathbb{R} | (t,p) \in \mathcal{D} \rbrace$ open, \\

given $\phi$ smooth flow, \\
\quad $\phi(t,q)$ defined and smooth $\forall \, (t,q)$ sufficiently close to $(0,p)$ since $\mathcal{D}$ open.  With $f$ smooth on open $U$ in this open neighborhood of $(0,p)$, 

\[
\Longrightarrow Xf(p) = \left. \frac{ \partial }{ \partial t } (f \circ \phi(t,p)) \right|_{t=0}
\]

$f\phi$ smooth (by composition), so $\partial_t f\phi$ smooth, so $X$ smooth itself around $\forall \, p \in M$.   \\

Suppose $t\in \mathcal{D}^{(p)}$ \\
$\mathcal{D}^{(p)}$, $\mathcal{D}^{ (\phi_t(p))} = \mathcal{D}^{(q)}$ open (by def.)

$\phi_{\Delta t}\phi_t(p) = \phi_{ \Delta t + t }(p)$ by def. of flow. 









\end{proof}

\begin{tikzpicture}
  \matrix (m) [matrix of math nodes, row sep=2em, column sep=4.8em, minimum width=2em]
  {
\mathfrak{X}(M) &  \\
M &  \\
\mathcal{D} &       M \\
\mathcal{D}^{(p)} & \\
};
  \path[->]
  (m-2-1) edge node [above] {$$} (m-1-1)
  edge node [above] {$\phi_t$} (m-3-2)
  (m-3-1) edge node [auto] {$$} (m-2-1)
  edge node [below left] {$\phi$} (m-3-2)
  edge node [auto] {$$} (m-4-1)
  (m-4-1) edge node [below] {$\phi^{(p)}$} (m-3-2)
  (m-3-2) edge node [right] {$\left. \frac{ \partial }{ \partial t} \right|_{t=0} = \left. \frac{d}{dt} \right|_{t=0}$} (m-1-1)
;
\end{tikzpicture} \quad \quad \quad \, \begin{tikzpicture}
  \matrix (m) [matrix of math nodes, row sep=2em, column sep=4.8em, minimum width=2em]
  {
X_p &  \\
p &  \\
(t,p)       &       \phi(t,p) = \phi^{(p)}(t) = \phi_t(p) \\
t & \\
};
  \path[|->]
  (m-2-1) edge node [above] {$$} (m-1-1)
  edge node [above] {$\phi_t$} (m-3-2)
  (m-3-1) edge node [auto] {$$} (m-2-1)
  edge node [below left] {$\phi$} (m-3-2)
  edge node [auto] {$$} (m-4-1)
  (m-4-1) edge node [below] {$\phi^{(p)}$} (m-3-2)
  (m-3-2) edge node [right] {$\left. \frac{ \partial }{ \partial t} \right|_{t=0} = \left. \frac{d}{dt} \right|_{t=0}$} (m-1-1)
;
\end{tikzpicture}





\begin{theorem}[9.12] (Fundamental Theorem on Flows)
  Let smooth vector field $X$ on smooth manifold $M$.  \\
$\exists \, ! \, \, $ smooth maximal flow $\phi : \mathcal{D} \to M$ whose infinitesimal generator is $X$ (recall $\begin{aligned} & \quad \\ 
    & p \mapsto X_p \\
    & X_p = \dot{\phi}{(0)} \end{aligned}$) s.t. 
\begin{enumerate}
\item[(a)] $\forall \, p \in M$, curve $\phi^{(p)} : \mathcal{D}^{(p)} \to M$ is unique maximal integral curve of $X$ starting at $p$. 
\item[(b)] If $s\in \mathcal{D}^{(p)}$, then $\mathcal{D}^{(\phi(s,p))}$ is interval $\mathcal{D}^{(p)}-s = \lbrace t - s | t\in \mathcal{D}^{(p)} \rbrace$
\item[(c)] $\forall \, t \in \mathbb{R}$, $M_t$ open in $M$, and $\phi_t:M_t \to M_{-t}$ diffeomorphism with inverse $\phi_{-t}$
\end{enumerate}
\end{theorem}

\begin{proof}
From Proposition 9.2 ($\forall \, p \in M$, $\exists \, \epsilon > 0$, \, $\exists \, $ smooth $c:(-\epsilon, \epsilon) \to M$, i.e. integral curve $X$ starting at $p$) \\

Suppose $c,\widetilde{c} : I \to M$ \, 2 integral curves of $X$, open $I$ s.t. $c(t_0) = \widetilde{c}(t_0)$ for some $t_0 \in I$ \\
Let $S = \lbrace t | t \in I, \text{ s.t. } c(t) = \widetilde{c}(t) \rbrace$ \\
Clearly $S \neq \emptyset$ since $c(t_0) = \widetilde{c}(t_0)$ \quad (hypothesis) \\
\quad $S$ closed in $I$ by continuity (of $c, \widetilde{c}$) \\
Suppose $t_1 \in S$ \\
\quad $c(t_1) = \widetilde{c}(t_1)=p$
Then in smooth coordinate neighborhood around $p=c(t_1)$, \, $c, \widetilde{c}$ both solutions to same ODE with same initial conditions $c(t_1) = \widetilde{c}(t_1) = p$ \\

By uniqueness part of Thm. D.1, $c\equiv \widetilde{c}$ on interval containing $t_1$ \\
\quad $\Longrightarrow S $ \, open in $I$. \\
Since $I$ connected, $S=I$ \quad ($S$ clopen) \\
$c= \widetilde{c} \quad \, \forall \, t \in I$ \\
Thus, $\forall \, c , \widetilde{c}$ \, that agrees at 1 pt. agree on common domain.   \\ 

$\forall \, p \in M$, let $\mathcal{D}^{(p)} = \bigcup_{\alpha} I_{\alpha}$, open $I_{\alpha} \subseteq \mathbb{R}$ 
\quad s.t. $0 \in I_{\alpha}$, and integral curve $\begin{aligned} & \quad \\ 
  & c_{\alpha} : I_{\alpha} \to M \\
  & c_{\alpha}(0) = p \end{aligned}$ \, starting at $p$ is defined.  \\

define $\begin{aligned} & \quad \\ 
  & \phi^{(p)}: \mathcal{D}^{(p)} \to M  \\
  & \phi^{(p)}(t) = c(t) \end{aligned}$ \, where $c$ is any integral curve s.t. $c(0) = p$ and $c$ defined on $I_{\alpha}$ s.t. $0,t \in I_{\alpha}$. \\

since all integral curves agree at $t$ by argument above, $\phi^{(p)}$ well-defined \\
\quad \quad and is obviously unique maximal integral curve starting at $p$.   \\


Let $\mathcal{D} = \lbrace (t,p) \in \mathbb{R} \times M | t \in \mathcal{D}^{(p)} \rbrace$ \\
define $\begin{aligned} & \quad \\ 
  & \phi :\mathcal{D} \to M \\
  & \phi{(t,p)} = \phi^{(p)}{(t)} \equiv \phi_t{(p)} \end{aligned}$ (notation for last statement)  \\

By def. $\phi$ satisfies (a): $\forall \, p \in M$, $\exists \, ! \, $ maximal integral curve of $X$, $\phi^{(p)}$, starting at $p$.  \\

Fix $p\in M$, $s \in \mathcal{D}^{(p)}$ \\
write $q = \phi{ (s,p) } = \phi^{(p)}(s)$ \\

define $\begin{aligned} & \quad \\ 
  & \widetilde{c}: \mathcal{D}^{(p)} -s \to M \\
  & \widetilde{c}(t) = \phi^{(p)}(t+s) \, \text{ s.t. } \, \widetilde{c}{(0)} = \phi^{(p)}(s) = q \end{aligned}$ \\

By translation lemma (9.4), $\begin{aligned} & \quad \\
  & \widetilde{c}{(t)} = c{(t+s)} \\
  & \widetilde{I} = \lbrace t | t+s \in I \rbrace \end{aligned}$ \quad \, e.g. $\begin{aligned} & \quad \\ 
  & I = (-2,6) , \, \, s = 1 \\
  & \widetilde{I} = (-3,5) \end{aligned}$ \quad \, $\widetilde{c}(t)$ also integral curve of $X$.  

By uniqueness of ODE solutions, \\
\quad $\widetilde{c}$ agrees with $\phi^{(q)}$ on their common domain, \\
\quad \quad equivalent to second group law (9.7) 
\[
\widetilde{c}{(t)} = \phi^{(p)}{ (t+s)} = \phi{(t+s,p)} = \phi^{(q)}{(t)} = \phi{(t,q)} = \phi{(t,\phi{(s,p) }) }
\]



\end{proof}


\begin{lemma}[9.19] \textbf{(Escape Lemma)}
Suppose smooth $M$, $V\in \mathfrak{X}(M)$.  \\
If $\gamma: J \to M$ maximal integral curve of $V$ s.t. domain $J$ has finite least upper bound $b$, \\
\phantom{ \quad \, } then $\forall \, t_0 \in J$, $\gamma([t_0,b))$ not contained in any compact subset of $M$
\end{lemma}

\begin{proof}
See Problem 9-6.  Solution there.  
\end{proof}

\subsection*{Flowouts}

Suppose smooth $M$, $S\subseteq M$ embedded $k$-dim. submanifold. \\
smooth $V \in \mathfrak{X}(M)$ s.t. $V$ nowhere tangent to $S$.  \\
Let $\theta:\mathcal{D} \to M$ be flow of $V$  \\
Let $\mathcal{O} = (\mathbb{R}\times S) \bigcap \mathcal{D}$ \\
\phantom{Let } $\Phi= \left. \theta \right|_{\mathcal{O}}$
\begin{enumerate}
\item[(a)] $\Phi : \mathcal{O} \to M$ immersion 
\item[(b)] $\frac{ \partial }{ \partial t} \in \mathfrak{X}(\mathcal{O})$ is $\Phi$-related to $V$
\item[(c)] $\exists \, $ smooth $\delta >0$, $\delta : S \to \mathbb{R}$ s.t. \\
$\left. \Phi \right|_{\mathcal{O}_{\delta}}$ injective, where $\mathcal{O}_{\delta} \subseteq \mathcal{O}$ flow domain.  
\begin{equation}
  \mathcal{O}_{\delta} = \lbrace (t,p) \in \mathcal{O} | |t| < \delta(p) \rbrace \quad \quad \quad \, (9.9)
\end{equation}
Thus $\Phi(\mathcal{O}_{\delta})$ immersed submanifold of $M$ containing $S$. $V$ tangent to $\Phi(\mathcal{O}_{\delta})$
\item[(d)] If $S$ codim. $1$, $\left. \Phi \right|_{\mathcal{O}_{\delta}}$ diffeomorphism onto open submanifold of $M$
\end{enumerate}

\subsection*{Flows and Flowouts on Manifolds with Boundary }

\subsection*{Lie Derivatives }

\begin{equation}
  D_vW(p) = \left. \frac{d}{dt} \right|_{t=0} W_{p+tv} = \lim_{t\to 0} \frac{ W_{p+tv} - W_p }{t} \quad \quad \quad \, (9.15)
\end{equation}

\[
D_vW(p) = D_vW^i(p) \left. \frac{ \partial }{ \partial x^i } \right|_p
\]

\textbf{Lie derivative of $W$ with respect to $V$ }

\begin{equation}
  (\mathcal{L}_VW)_p = \left. \frac{d}{dt} \right|_{t=0} d(\theta_{-t})_{\theta_t(p)}(W_{\theta_t(p)}) = \lim_{t \to 0} \frac{ d(\theta_{-t})_{\theta_t(p)}(W_{\theta_t(p)}) - W_p }{ t } \quad \quad \quad \, (9.16)
\end{equation}

\begin{tikzpicture}
  \matrix (m) [matrix of math nodes, row sep=2em, column sep=6.8em, minimum width=2em]
  {
\mathfrak{X}(M) & \mathfrak{X}(M)  \\
\mathfrak{X}(M) &   \\
M    &       M \\
};
  \path[->]
  (m-3-1) edge node [above] {$$} (m-2-1)
  edge node [above] {$\phi_t$} (m-3-2)
  (m-3-2) edge node [auto] {$$} (m-1-2)
  edge [bend left] node [auto] {$\phi_{-t}$} (m-3-1)
  (m-1-2) edge node [above] {$d(\phi_{-t})=(\phi_{-t})_*$} (m-1-1)
;
\end{tikzpicture} \quad \quad \quad \, \begin{tikzpicture}
  \matrix (m) [matrix of math nodes, row sep=2em, column sep=6.8em, minimum width=2em]
  {
d(\phi_{-t})_{\phi_t(p)}(W_{\phi_t(p)}) & W_{\phi_t(p)}  \\
W_p   &    \\
p    &       \phi_t(p) \\
};
  \path[|->]
  (m-3-1) edge node [above] {$$} (m-2-1)
  edge node [above] {$\phi_t$} (m-3-2)
  (m-3-2) edge node [auto] {$$} (m-1-2)
  edge [bend left] node [auto] {$\phi_{-t}$} (m-3-1)
  (m-1-2) edge node [above] {$d(\phi_{-t})=(\phi_{-t})_*$} (m-1-1)
;
\end{tikzpicture}




\begin{lemma}[9.36]
  Suppose smooth $M$ with or without $\partial$, \, $V, W \in \mathfrak{X}(M)$ \\
If $\partial M \neq \emptyset$, assume $v$ trangent to $\partial M$ \\
Then $\exists \, $ smooth $(\mathcal{L}_VW)_p$ \, $\forall \, p \in M$
\end{lemma}

\begin{proof}
  $\forall \, (t,x ) \in J_0 \times U_0$, \\
  matrix $d(\theta_{-t})_{\theta_t(x)}: T_{\theta_t(x)}M \to T_xM$
\[
\left( \frac{ \partial \theta^i}{ \partial x^j}(-t, \theta(t,x)) \right)
\]

Therefore,
\[
d(\theta_{-t})_{\theta_t(x)}(W_{\theta_t(x)}) = \frac{ \partial \theta^i}{ \partial x^j}(-t,\theta(t,x)) W^j(\theta(t,x)) \left. \frac{ \partial }{ \partial x^i} \right|_x
\]

\end{proof}

\exercisehead{9.37}

Given $V = v^i \frac{ \partial }{ \partial x^i}$ with constant coefficients (i.e. $v^i$ constant)

\[
\begin{aligned}
  & \dot{\theta}^{(x)}(t) = V \\ 
  & \dot{\theta}^i(t) = v^i  \\
  & \theta^i(t)= v^it + x^i
\end{aligned} \quad \quad \Longrightarrow \quad 
\frac{ \partial \theta^i}{ \partial x^j} = \delta^i_{ \, \, j}
\]

\[
\begin{gathered}
\frac{d}{dt} W^i(\theta(t,x))  = \frac{ \partial W^i}{ \partial y^j}(v^j) 
\end{gathered}
\]

From the proof of Lemma 9.36, 
\[
\begin{gathered}
  d(\theta_{-t})_{\theta_t(x)}(W_{\theta_t(x)}) = \frac{ \partial \theta^i}{ \partial x^j}(-t,\theta(t,x)) W^j(\theta(t,x)) \left. \frac{ \partial }{ \partial x^i} \right|_x = \\
 = \delta^i_{ \, \, j} W^j(\theta(t,x)) \left. \frac{ \partial }{ \partial x^i} \right|_x  = W^i(\theta(t,x)) \left. \frac{ \partial }{ \partial x^i} \right|_x
\end{gathered}
\]
From (9.16)
\[
\begin{gathered}
  (\mathcal{L}_VW)_p = \left. \frac{d}{dt} \right|_{t=0} d(\theta_{-t})_{\theta_t(p)}(W_{ \theta_t(p) } ) = v^j \frac{ \partial W^i}{ \partial x^j} \left. \frac{ \partial }{ \partial x^i} \right|_p = D_VW^i(p) \left. \frac{ \partial }{ \partial x^i} \right|_p = D_VW(p)
\end{gathered}
\]

\hrulefill

\begin{theorem}[9.38] If smooth $M$, and $V,W \in \mathfrak{X}(M)$, 
\[
\mathcal{L}_VW = [V,W]
\]
\end{theorem}



\begin{corollary}[9.39]
  \begin{enumerate}
    \item[(a)]
    \item[(b)]
    \item[(c)]
    \item[(d)]
    \item[(e)]
\end{enumerate}
\end{corollary}

\exercisehead{9.40}

  \begin{enumerate}
    \item[(a)] \[
\mathcal{L}_VW = [V,W] = -[W,V] = \mathcal{L}_WV
\]
    \item[(b)]
    \item[(c)]
    \item[(d)]
    \item[(e)]
\end{enumerate}


\hrulefill


Prop. 9.41 is about derivative of $d(\theta_{-t})_{\theta_t(p)}(W_{\theta_t(p)})$ at other times

\begin{proposition}[9.41] Suppose smooth $M$ with or without $\partial$ and $V, W \in \mathfrak{X}(M)$ \\
If $\partial M \neq \emptyset$, assume $V$ tangent to $\partial M$ \\
Let $\theta$ flow of $V$.  \\

$\forall \, (t_0,p)$ in domain of $\theta$

\[
\left. \frac{d}{dt} \right|_{t=t_0} d(\theta_{-t})_{\theta_t(p)}(W_{\theta_t(p)})  = d(\theta_{-t_0}) ( (\mathcal{L}_VW)_{\theta_{t_0}(p)} )
\]
\end{proposition}


\subsection*{ Commuting Vector Fields }


\subsection*{ Time-Dependent Vector Fields }

Let smooth manifold $M$ 

\textbf{time-dependent vector field on $M$}, $V$ \\ 
\quad cont. $V: J \times M \to TM$, interval $J \subseteq \mathbb{R}$ \\
\quad \quad s.t. 
\[
V(t,p) \in T_p M \quad \, \forall \, (t,p) \in J \times M
\]
i.e. $\forall \, t \in J$, \\
$\begin{aligned}
  & V_t : M \to TM \\
  & V_t(p) = V(t,p) \end{aligned}$ is a vector field on $M$

EY : 20150226

\textbf{time-dependent vector field on $M$}, $V$ is \\
\phantom{ \quad \quad \, }  $\begin{aligned}
& \quad \\
 \text{ cont. } & V : J \times M \to TM \quad \, \text{ interval } J \subseteq \mathbb{R} \\
  & V(t,p) \in T_p M \quad \, \forall \, (t,p) \in J \times M \end{aligned}$

i.e.  $\forall \, t \in J$ \\
$\begin{aligned}
& \quad \\
  & V_t : M \to TM \\
  & V_t(p) = V(t,p ) \in \mathfrak{X}(M) \end{aligned}$


\textbf{integral curve of $V$ } is diff. $\begin{aligned} 
& \quad \\
  & \gamma : J_0 \to M \\
  & \dot{\gamma}(t) = V(t,\gamma(t)) \quad \, \forall \, t \in J_0 \end{aligned}$ \quad \, where $J_0 \subset J $ s.t. 

$\forall \, X \in \mathfrak{X}(M)$, determines time dependent vector field $V: \mathbb{R} \times M \to TM$ by 
\[
V(t,p) = X_p
\]

\begin{tikzpicture}
  \matrix (m) [matrix of math nodes, row sep=3.2em, column sep=4.8em, minimum width=3.2em]
  {
& \mathfrak{X}(\mathbb{R}\times M) = T\mathbb{R}\oplus TM     & TM   \\
J \times M & \mathbb{R} \times M &  M\\
& \mathcal{E} \subseteq J \times J \times M         &       M \\
& \mathcal{E}^{(t_0,p)}                            & \\
};
  \path[->]
  (m-2-2) edge node [left] {$\widetilde{V}$} (m-1-2)
         edge node [auto] {$V$} (m-1-3)
          edge node [above] {$$} (m-2-3)
  (m-3-2) edge node [auto] {$$} (m-2-2)
          edge node [auto] {$$} (m-2-3)
          edge node [below left] {$\psi$} (m-3-3)
          edge node [auto] {$$} (m-4-2)
          edge node [auto] {$\widetilde{\theta}$} (m-2-1)
  (m-4-2) edge node [below] {$\psi^{(t_0,p)}$} (m-3-3)
  (m-2-3) edge node [right] {$\psi_{tt_0}$} (m-3-3)
          edge node [right] {$X$} (m-1-3)
  (m-1-3) edge node [above] {$\oplus T\mathbb{R}$} (m-1-2)
  (m-2-1) edge node [above] {$t=t_0$} (m-2-2)
          edge node [auto] {$\left. \frac{ \partial }{ \partial t} \right|_{t=t_0}$} (m-1-2)
;
\end{tikzpicture} \quad \quad \quad \, 

\begin{tikzpicture}
  \matrix (m) [matrix of math nodes, row sep=3.2em, column sep=4.8em, minimum width=3.2em]
  {
& \left. \frac{ \partial }{ \partial t} \right|_{t=t_0} \widetilde{\theta} = \widetilde{V}_{(t_0,p)} = \left( \left. \frac{ \partial }{ \partial s} \right|_{t_0}, V(t_0,p) \right) & V(t_0,p) = X_p  \\
\widetilde{\theta}(t,(t_0,p)) = (\alpha(t,(t_0,p)), \beta(t,(t_0,p)))  & (t_0,p)        & p  \\
& (t,t_0,p)       &       \psi(t,t_0,p) = \psi^{(t_0,p)}(t)  \\
& t & \\
};
  \path[|->]
 (m-2-2) edge node [left] {$\widetilde{V}$} (m-1-2)
         edge node [auto] {$V$} (m-1-3)
          edge node [above] {$$} (m-2-3)
  (m-3-2) edge node [auto] {$$} (m-2-2)
          edge node [auto] {$$} (m-2-3)
          edge node [below left] {$\psi$} (m-3-3)
          edge node [auto] {$$} (m-4-2)
          edge node [auto] {$\widetilde{\theta}$} (m-2-1)
  (m-4-2) edge node [below] {$\psi^{(t_0,p)}$} (m-3-3)
  (m-2-3) edge node [right] {$\psi_{tt_0}$} (m-3-3)
          edge node [right] {$X$} (m-1-3)
  (m-1-3) edge node [above] {$+ \left. \frac{\partial}{\partial s} \right|_{t_0}$} (m-1-2)
  (m-2-1) edge node [above] {$t=t_0$} (m-2-2)
          edge node [auto] {$\left. \frac{ \partial }{ \partial t} \right|_{t=t_0}$} (m-1-2)
; 
\end{tikzpicture}

with \[
\begin{gathered}
\left. \frac{ \partial }{ \partial t}\right|_{t=t_0} \widetilde{\theta}( t,(t_0,p)) = \left. \left( \frac{ \partial \alpha }{\partial t}(t,(t_0,p)), \frac{\partial \beta}{ \partial t}(t,(t_0,p))  \right)\right|_{t=t_0} = (1, V(t_0,p))
\end{gathered}
\]

EY : 20150725 I don't like Lee's choice of notation.  Let me rewrite the above diagrams:

\begin{tikzpicture}
  \matrix (m) [matrix of math nodes, row sep=3.2em, column sep=4.8em, minimum width=3.2em]
  {
& \mathfrak{X}(\mathbb{R}\times M) = T\mathbb{R}\oplus TM     & TM   \\
J \times M & \mathbb{R} \times M &  M\\
& \mathcal{E} \subseteq J \times J \times M         &       M \\
& \mathcal{E}^{(t,p)}                            & \\
};
  \path[->]
  (m-2-2) edge node [left] {$\widetilde{V}$} (m-1-2)
         edge node [auto] {$V$} (m-1-3)
          edge node [above] {$$} (m-2-3)
  (m-3-2) edge node [auto] {$$} (m-2-2)
          edge node [auto] {$$} (m-2-3)
          edge node [below left] {$\psi$} (m-3-3)
          edge node [auto] {$$} (m-4-2)
          edge node [auto] {$\widetilde{\theta}$} (m-2-1)
  (m-4-2) edge node [below] {$\psi^{(t,p)}$} (m-3-3)
  (m-2-3) edge node [right] {$\psi_{st}$} (m-3-3)
          edge node [right] {$V_t$} (m-1-3)
  (m-1-3) edge node [above] {$\oplus T\mathbb{R}$} (m-1-2)
  (m-2-1) edge node [above] {$s=0$} (m-2-2)
          edge node [auto] {$\left. \frac{ \partial }{ \partial s} \right|_{s=0}$} (m-1-2)
;
\end{tikzpicture} \quad \quad \quad \, 

\begin{tikzpicture}
  \matrix (m) [matrix of math nodes, row sep=3.2em, column sep=4.8em, minimum width=3.2em]
  {
& \left. \frac{ \partial }{ \partial s} \right|_{s=0} \widetilde{\theta} = \widetilde{V}_{(t,p)} = \left( \left. \frac{ \partial }{ \partial t} \right|_{s=0}, V(t,p) \right) & V(t,p) = V_t(p)  \\
\widetilde{\theta}(s,(t,p)) = (\alpha(s,(t,p)), \beta(s,(t,p)))  & (t,p)        & p  \\
& (s,t,p)       &       \psi(s,t,p) = \psi^{(t,p)}(s)  \\
& s & \\
};
  \path[|->]
 (m-2-2) edge node [left] {$\widetilde{V}$} (m-1-2)
         edge node [auto] {$V$} (m-1-3)
          edge node [above] {$$} (m-2-3)
  (m-3-2) edge node [auto] {$$} (m-2-2)
          edge node [auto] {$$} (m-2-3)
          edge node [below left] {$\psi$} (m-3-3)
          edge node [auto] {$$} (m-4-2)
          edge node [auto] {$\widetilde{\theta}$} (m-2-1)
  (m-4-2) edge node [below] {$\psi^{(t,p)}$} (m-3-3)
  (m-2-3) edge node [right] {$\psi_{st}$} (m-3-3)
          edge node [right] {$V_t$} (m-1-3)
  (m-1-3) edge node [above] {$+ \left. \frac{\partial}{\partial t} \right|_{s=0}$} (m-1-2)
  (m-2-1) edge node [above] {$s=0$} (m-2-2)
          edge node [auto] {$\left. \frac{ \partial }{ \partial s} \right|_{s=0}$} (m-1-2)
; 
\end{tikzpicture}

with \[
\begin{gathered}
\left. \frac{ \partial }{ \partial s}\right|_{s=0} \widetilde{\theta}( s,(t,p)) = \left. \left( \frac{ \partial \alpha }{\partial s}(s,(t,p)), \frac{\partial \beta}{ \partial s}(s,(t,p))  \right)\right|_{s=0} = (1, V(t,p))
\end{gathered}
\]


\begin{theorem}[9.48] \textbf{(Fundamental Theorem on Time-Dependent Flows)} \\
Let $M$ smooth manifold \\
open $J \subseteq \mathbb{R}$ \\
$V: J \times M \to TM$ smooth time-dependent vector field on $M$ \\
$\exists \, $ open $\mathcal{E} \subseteq J\times J \times M$, smooth $\psi : \mathcal{E} \to M$ called time-dependent flow of $V$ s.t. 

\begin{enumerate}
\item[(a)] $\forall \, t_0 \in J$, \, $\forall \, p \in M$, \\
open $\mathcal{E}^{(t_0,p)} = \lbrace t \in J | (t,t_0, p) \in \mathcal{E}_0 \rbrace$ s.t. $t_0 \in \mathcal{E}^{(t_0, p)}$ \\
smooth curve $\begin{aligned}
  & \quad \\ 
  & \psi^{(t_0,p)}: \mathcal{E}^{(t_0,p)} \to M \\
  & \psi^{(t_0,p)}(t) = \psi(t,t_0,p) \end{aligned}$ \\
is unique maximal integral curve of $V$ with $\psi^{(t_0,p)}(t_0) = p$
\item[(b)] If $ \begin{aligned}
  & \quad \\ 
  & t_1 \in \mathcal{E}^{(t_0,p)} \\
  & q = \psi^{(t_0,p)}(t_1) \end{aligned}$

then $\begin{aligned}
  & \quad \\
  & \mathcal{E}^{(t_1,q)} = \mathcal{E}^{(t_0,p)} \text{ and } \\
  & \psi^{(t_1,q)} = \psi^{(t_0,p)} \end{aligned}$
\item[(c)] $\forall \, (t_1,t_0) \in J \times J $ \\
$M_{t_1,t_0} = \lbrace p \in M | (t_1,t_0, p) \in \mathcal{E} \rbrace$ open in $M$ and \\
  $\begin{aligned}
  & \quad \\
  & \psi_{t_1t_0} : M_{t_1t_0} \to M \\
  & \psi_{t_1t_0}(p) = \psi(t_1,t_0,p) \end{aligned}$
is a diffeomorphism from $M_{t_1t_0}$ onto $M_{t_0t_1}$ with inverse $\psi_{t_0t_1}$
\item[(d)] If $p \in M_{t_1t_0} $, $\psi_{t_1t_0}(p) \in M_{t_0t_1}$, \\
then $p\in M_{t_2 t_0}$ and 
\begin{equation}
\psi_{t_2t_1} \psi_{t_1t_0}(p) = \psi_{t_2t_0}(p) \quad \quad \quad \, (9.18)
\end{equation}
\end{enumerate}

\end{theorem}

\begin{proof}
Consider smooth vector field $\begin{aligned} 
& \quad  \\
  & \widetilde{V} \in \mathfrak{X}(J \times M ) \text{ defined by } \\
  & \widetilde{V}_{(s,p)} = \left( \left. \frac{ \partial }{ \partial s } \right|_s, V(s,p) \right) \end{aligned}$ \\

identify $T_{(s,p)}(J \times M)$ with $T_sJ \oplus T_pM$ (Prop. 3.14)

Let $\begin{aligned}
& \quad \\ 
  & \widetilde{\theta} : \widetilde{\mathcal{D}} \to J \times M \\
  & \widetilde{\theta}(t,(s,p)) = (\alpha(t,(s,p)) , \beta(t,(s,p)) ) \end{aligned}$ \quad \, flow of $\widetilde{V}$

then $\begin{aligned} 
 & \quad \\
  & \alpha: \widetilde{D} \to J \\
  & \beta: \widetilde{D} \to M \end{aligned}$ s.t. 

\[
\begin{aligned}
  & \frac{ \partial \alpha}{ \partial t}(t,(s,p) ) = 1 \quad \quad \, & \alpha(0,(s,p)) = s \\ 
  & \frac{ \partial \beta}{ \partial t}(t,(s,p)) = V(\alpha(t,(s,p)), \beta(t,(s,p))) \quad \quad \, & \beta(0,(s,p)) = p 
\end{aligned}
\]
$\Longrightarrow \alpha(t,(s,p)) = t+s$ so 
\begin{equation}
  \frac{ \partial \beta}{ \partial t}(t,(s,p)) = V(t+s,\beta(t,(s,p)) \quad \quad \quad \, (9.19)
\end{equation}
\end{proof}


Let $\begin{aligned}
  & \quad \\
  & \mathcal{E} \subseteq \mathbb{R} \times J \times M \text{ defined } \\
  & \mathcal{E} = \lbrace (t,t_0, p) | (t-t_0, (t_0,p) ) \in \widetilde{\mathcal{D}} \rbrace \end{aligned}$

$\mathcal{E}$ open because $\widetilde{\mathcal{D}}$ is. \\
since $\alpha: \widetilde{D} \to J$, if $(t,t_0, p) \in \mathcal{E}$, then $t= \alpha(t-t_0,(t_0,p)) \in J$, implies $\mathcal{E} \subseteq J \times J \times M$ \\
$\mathcal{E}$ open so $M_{t_1t_0} = \lbrace p \in M | (t_1,t_0, p) \in \mathcal{E} \rbrace$ open. 

define $\begin{aligned} & \quad \\ 
  & \psi : \mathcal{E} \to M \\
  & \psi(t,t_0,p) = \beta(t-t_0,(t_0,p))
\end{aligned}$

EY : 20150725 Remark: Out of the proof immediately above, there are a number of takeaways that really \emph{should} be mentioned.  

Let's collect the facts:
\[
\begin{aligned}
  & \widetilde{\theta}(s,(t,p)) = (\alpha(s,(t,p)), \beta(s,(t,p)) ) := \widetilde{\theta}^{(t,p)}(s) \\ 
  & \widetilde{\theta}(0,(t,p)) = (\alpha(0,(t,p)), \beta(0,(t,p)) ) = (t,p) \\ 
  & \frac{ \partial \alpha}{ \partial s}(s,(t,p)) = 1 \\
  & \frac{ \partial \beta }{ \partial s}(s,(t,p)) = V(\alpha(s,(t,p)), \beta(s,(t,p))) \text{ so } \\ 
  & \alpha(s,(t,p)) = s+t \\ 
  & \frac{ \partial \beta}{ \partial s}(s,(t,p)) = V(s+t,\beta(s,(t,p)) ) \\
  & \left. \frac{ \partial }{ \partial s} \right|_{s=0} \widetilde{\theta}(s,(t,p)) = \left. \left( \frac{ \partial \alpha }{ \partial s}(s,(t,p)) , \frac{ \partial \beta}{ \partial s}(s,(t,p)) \right) \right|_{s=0} = (1,V(t,p)) = \frac{d\widetilde{\theta}^{(t,p)}}{ds}(s=0) := \widetilde{V}_{(t,p)}
\end{aligned}
\]

Also, we can write the flow $\widetilde{\theta}_s$ as
\[
\widetilde{\theta}^{(t,p)}(s) = \widetilde{\theta}(s,(t,p)) = \widetilde{\theta}_s(t,p) = (\alpha(s,(t,p)),\beta(s,(t,p))) = (s+t, \beta(s,(t,p)))
\]

Now consider the Lie derivative:

\begin{tikzpicture}
  \matrix (m) [matrix of math nodes, row sep=2em, column sep=6.8em, minimum width=2em]
  {
\mathfrak{X}(\mathbb{R} \times M) & \mathfrak{X}(\mathbb{R} \times M)  \\
\mathfrak{X}(\mathbb{R} \times M) &   \\
J\times M    &       J\times M \\
};
  \path[->]
  (m-3-1) edge node [above] {$$} (m-2-1)
  edge node [above] {$\widetilde{\theta}_s$} (m-3-2)
  (m-3-2) edge node [auto] {$$} (m-1-2)
  edge [bend left] node [auto] {$\widetilde{\theta}_{-s}$} (m-3-1)
  (m-1-2) edge node [above] {$d(\widetilde{\theta}_{-s})=(\widetilde{\theta}_{-s})_*$} (m-1-1)
;
\end{tikzpicture} \quad \quad \quad \, \begin{tikzpicture}
  \matrix (m) [matrix of math nodes, row sep=2em, column sep=6.8em, minimum width=2em]
  {
d(\widetilde{\theta}_{-s})_{\widetilde{\theta}_s(t,p)}(W_{\widetilde{\theta}_s(t,p)}) & W_{\widetilde{\theta}_s(t,p)}  \\
W_{(t,p)}   &    \\
(t,p)    &       \widetilde{\theta}_s(t,p) \\
};
  \path[|->]
  (m-3-1) edge node [above] {$$} (m-2-1)
  edge node [above] {$\widetilde{\theta}_s$} (m-3-2)
  (m-3-2) edge node [auto] {$$} (m-1-2)
  edge [bend left] node [auto] {$\widetilde{\theta}_{-s}$} (m-3-1)
  (m-1-2) edge node [above] {$d(\widetilde{\theta}_{-s})=(\widetilde{\theta}_{-s})_*$} (m-1-1)
;
\end{tikzpicture}

with $\widetilde{\theta}$ being the flow of $\widetilde{V}$.  Let's define the Lie derivative:

\begin{equation}
\begin{gathered}
  \mathcal{L}_{\widetilde{V}}W = (\mathcal{L}_{\widetilde{V}}W)_{(t,p)} = \left. \frac{d}{ds} \right|_{s=0}(d\widetilde{\theta}_{-s})_{\widetilde{\theta}_s(t,p)}(W_{\widetilde{\theta}_s(t,p)})   = \lim_{s\to 0} \frac{ (d\widetilde{\theta}_{-s})_{\widetilde{\theta}_s(t,p)}( W_{\widetilde{\theta}_s(t,p)}) - W_{\widetilde{\theta}_s(t,p)} }{s}
\end{gathered}
\end{equation}


Use Case 1 of the proof of Lee's Theorem 9.38, for showing $\mathcal{L}_VW = [V,W]$.  \\
Let open neighborhood $U \subseteq J \times M$, with $(t,p) \in U$.  On open $U$, choose smooth coordinates $(t,u^i)$ on $U$.  By Theorem 9.22, that at a regular point $p\in M$, $\exists \, (u^i)$ coordinates s.t. $V_p = \frac{ \partial }{ \partial u^1}$, then consider 

\[
\widetilde{V} = \frac{ \partial }{ \partial t} + \frac{ \partial }{ \partial u^1} \in \mathfrak{X}(\mathbb{R} \times M)
\]
with $V(t)(p) = \frac{ \partial }{ \partial u^1} \in \mathfrak{X}(M)$.  (Remember, $V(t)$ is a vector-field that is time-dependent, but is on $M$.  I will use this as a justification for using Thm. 9.22).  

Now the flow $\widetilde{\theta}_s$ takes on these forms:
\[
\begin{gathered}
  \widetilde{\theta}^{(t,p)}(s) = \widetilde{\theta}(s,(t,p)) = \widetilde{\theta}_s(t,p) = \\
  = (\alpha(s,(t,p)) , \beta(s,(t,p))) = (s+t, \beta(s,(t,p)) )
\end{gathered}
\]
Given these conditions, that \\
$\beta(0,(t,p)) = p = (u^1,u^2, \dots u^n)$ and 
\[
\left. \frac{ \partial \beta}{ \partial s}(s, (t,p)) \right|_{s=0} = V(t,p) = \frac{ \partial }{ \partial u^1} = \left. \frac{d}{ds} \beta^{(t,p)}(s) \right|_{s=0}
\]
then a $\beta$ that satisfies these conditions above is 
\[
\beta(s,(t,p)) = \beta_s(t,p) = (u^1 + s, u^2 \dots u^n)
\]
so that we can conclude that 
\[
\widetilde{\theta}_s(t,p) = (t+s, u^1 + s, u^2 , \dots , u^n)
\]

For fixed $s$, then
\[
d(\widetilde{\theta}_{-s})_{\widetilde{\theta}_s(t,p)} =1_{T_{\widetilde{\theta}_s(t,p)}(\mathbb{R}\times M)}
\]
so that 
\[
\begin{gathered}
  d(\widetilde{\theta}_{-s})_{\widetilde{\theta}_s(t,p)}(W_{\widetilde{\theta}_s(t,p)}) = d(\widetilde{\theta}_{-s})_{\widetilde{\theta}_s(t,p)} \cdot W^j(t+s,u^1 +s, u^2 \dots u^n) \left. \frac{ \partial }{ \partial u^j} \right|_{\widetilde{\theta}_s(t,p)} = W^j(t+s,u^1+s, u^2 \dots u^n) \left. \frac{ \partial }{ \partial u^j} \right|_{(t,p)} \\
\Longrightarrow \left. \frac{d}{ds} \right|_{s=0} W^j(t+s,u^1+s, u^2 \dots u^n) \left. \frac{ \partial }{ \partial u^j} \right|_{(t,p)} = \left( \frac{\partial }{ \partial t} W^j(t,u^1 \dots u^n) + \frac{ \partial }{ \partial u^1 } W^j(t,u^1 \dots u^n) \right) \left. \frac{ \partial}{ \partial u^j} \right|_{(t,p)}
\end{gathered}
\]


Thus, we can conclude that 
\begin{equation}
  \boxed{ \mathcal{L}_{\widetilde{V}}W = \mathcal{L}_{ \frac{ \partial }{ \partial t} +V}W = \left( \mathcal{L}_{ \frac{ \partial}{ \partial t} V } W \right)_{(t,p)} = \left( \left( \frac{ \partial }{ \partial t} + V \right) W^j \right) \left. \frac{ \partial }{ \partial x^j} \right|_{(t,p)} }
\end{equation}

\subsection*{First-Order Partial Differential Equations }



\subsection*{ Problems }


\problemhead{9-21} Note that from wikipedia, 

ambient isotopy \\
Let $N, M$ manifolds, \\
\quad $g,h$ embeddings of $N$ in $M$ 

cont. map $F: M \times [0,1] \to M$ s.t.  \\
$F: g\mapsto h$ \\
if $F_0 = 1$ \\
\phantom{ if } $F_t$ homeomorphism, $F_t: M \to M$ \\
\phantom{ if } $F_1 : g \mapsto h$ \\

\textbf{smooth isotopy} of $M$ is smooth $H: M \times J \to M$, \, $J \subseteq R$ interval s.t. \\
\quad \, $\forall \, t \in J$, $\begin{aligned} & \quad \\
  & H_t : M \to M \\
  & H_t(p) = H(p,t) \end{aligned}$ is a diffeomorphism.  

Suppose open interval $J \subseteq \mathbb{R}$ \\
\phantom{Suppose } smooth isotopy $H:M \times J \to M$ \\

EY : 20140206 

By definition \\
$\forall \, t, \, \begin{aligned} & \quad \\
  & H_t : M \to M \\
  & H_t(p) = H(p,t) \end{aligned}$

Then $DH_t = (H_t)_*$ \\
\[
DH_t : TM \to TM
\]

I tried, 

for some time $t$, consider $H(x^i(p),t) = y^j(x^i,t)$ 
\[
\frac{ \partial H(p,t)}{ \partial t } = \frac{ \partial y^j(x^i,t) }{ \partial t }
\]

Consider integral curve $x = x(t)$ s.t. $\dot{x} = X(t)$ 

\[
\begin{gathered}
  \dot{y} = \frac{dy}{dt} = \frac{d}{dt} y(x(t),t) = \frac{ \partial y^j}{ \partial x^i }\dot{x}^i = (DH_t)X \\
\begin{aligned}
  & (H_t)_* : TM \to TM \\ 
  & DH_t : X \mapsto \dot{y} = Y
\end{aligned}
\end{gathered}
\]



$\psi(t,t_0,p) = H_t \circ H_{t_0}^{-1}(p)$ domain $J \times J \times M$ ??

